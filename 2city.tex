% The Project Gutenberg Etext of A Tale of Two Cities, by Dickens
% 
% Please take a look at the important information in this header.
% We encourage you to keep this file on your own disk, keeping an
% electronic path open for the next readers.  Do not remove this.
% 
% 
% **Welcome To The World of Free Plain Vanilla Electronic Texts**
% 
% **Etexts Readable By Both Humans and By Computers, Since 1971**
% 
% *These Etexts Prepared By Hundreds of Volunteers and Donations*
% 
% Information on contacting Project Gutenberg to get Etexts, and
% further information is included below.  We need your donations.
% 
% 
% A Tale of Two Cities, by Charles Dickens
% [A story of the French Revolution]
% 
% January, 1994  [Etext #98]
% 
% 
% The Project Gutenberg Etext of A Tale of Two Cities, by Dickens
% *****This file should be named 2city11.txt or 2city11.zip******
% 
% Corrected EDITIONS of our etexts get a new NUMBER, 2city12.txt.
% VERSIONS based on separate sources get new LETTER, 2city10a.txt.
% 
% This etext was created by Judith Boss, Omaha, Nebraska.  The
% equipment: an IBM-compatible 486/50, a Hewlett-Packard ScanJet
% IIc flatbed scanner, and a copy of Calera Recognition Systems'
% M/600 Series Professional OCR software and RISC accelerator
% board donated by Calera.
% 
% For more information about this,
% ask about:
% 
% M/Series Profesional Software
% M/Series Accelerator Card
% Calera Recognition Systems
% 475 Potrero
% Sunnyvale, CA 94086
% 1-408-720-8300
% 
% mikel@calera.com Mike Lynch
% 
% 
% The official release date of all Project Gutenberg Etexts is at
% Midnight, Central Time, of the last day of the stated month.  A
% preliminary version may often be posted for suggestion, comment
% and editing by those who wish to do so.  To be sure you have an
% up to date first edition [xxxxx10x.xxx] please check file sizes
% in the first week of the next month.  Since our ftp program has
% a bug in it that scrambles the date [tried to fix and failed] a
% look at the file size will have to do, but we will try to see a
% new copy has at least one byte more or less.
% 
% 
% Information about Project Gutenberg (one page)
% 
% We produce about two million dollars for each hour we work.  The
% fifty hours is one conservative estimate for how long it we take
% to get any etext selected, entered, proofread, edited, copyright
% searched and analyzed, the copyright letters written, etc.  This
% projected audience is one hundred million readers.  If our value
% per text is nominally estimated at one dollar, then we produce 2
% million dollars per hour this year we, will have to do four text
% files per month:  thus upping our productivity from one million.
% The Goal of Project Gutenberg is to Give Away One Trillion Etext
% Files by the December 31, 2001.  [10,000 x 100,000,000=Trillion]
% This is ten thousand titles each to one hundred million readers,
% which is 10% of the expected number of computer users by the end
% of the year 2001.
% 
% We need your donations more than ever!
% 
% All donations should be made to "Project Gutenberg/IBC", and are
% tax deductible to the extent allowable by law ("IBC" is Illinois
% Benedictine College).  (Subscriptions to our paper newsletter go
% to IBC, too)
% 
% For these and other matters, please mail to:
% 
% Project Gutenberg
% P. O. Box  2782
% Champaign, IL 61825
% 
% When all other email fails try our Michael S. Hart, Executive Director:
% hart@vmd.cso.uiuc.edu (internet)   hart@uiucvmd   (bitnet)
% 
% We would prefer to send you this information by email
% (Internet, Bitnet, Compuserve, ATTMAIL or MCImail).
% 
% ******
% If you have an FTP program (or emulator), please
% FTP directly to the Project Gutenberg archives:
% [Mac users, do NOT point and click. . .type]
% 
% ftp mrcnext.cso.uiuc.edu
% login:  anonymous
% password:  your@login
% cd etext/etext91
% or cd etext92
% or cd etext93 [for new books]  [now also in cd etext/etext93]
% or cd etext/articles [get suggest gut for more information]
% dir [to see files]
% get or mget [to get files. . .set bin for zip files]
% GET 0INDEX.GUT
% for a list of books
% and
% GET NEW GUT for general information
% and
% MGET GUT* for newsletters.
% 
% **Information prepared by the Project Gutenberg legal advisor**
% (Three Pages)
% 
% 
% ***START**THE SMALL PRINT!**FOR PUBLIC DOMAIN ETEXTS**START***
% Why is this "Small Print!" statement here?  You know: lawyers.
% They tell us you might sue us if there is something wrong with
% your copy of this etext, even if you got it for free from
% someone other than us, and even if what's wrong is not our
% fault.  So, among other things, this "Small Print!" statement
% disclaims most of our liability to you.  It also tells you how
% you can distribute copies of this etext if you want to.
% 
% *BEFORE!* YOU USE OR READ THIS ETEXT
% By using or reading any part of this PROJECT GUTENBERG-tm
% etext, you indicate that you understand, agree to and accept
% this "Small Print!" statement.  If you do not, you can receive
% a refund of the money (if any) you paid for this etext by
% sending a request within 30 days of receiving it to the person
% you got it from.  If you received this etext on a physical
% medium (such as a disk), you must return it with your request.
% 
% ABOUT PROJECT GUTENBERG-TM ETEXTS
% This PROJECT GUTENBERG-tm etext, like most PROJECT GUTENBERG-
% tm etexts, is a "public domain" work distributed by Professor
% Michael S. Hart through the Project Gutenberg Association at
% Illinois Benedictine College (the "Project").  Among other
% things, this means that no one owns a United States copyright
% on or for this work, so the Project (and you!) can copy and
% distribute it in the United States without permission and
% without paying copyright royalties.  Special rules, set forth
% below, apply if you wish to copy and distribute this etext
% under the Project's "PROJECT GUTENBERG" trademark.
% 
% To create these etexts, the Project expends considerable
% efforts to identify, transcribe and proofread public domain
% works.  Despite these efforts, the Project's etexts and any
% medium they may be on may contain "Defects".  Among other
% things, Defects may take the form of incomplete, inaccurate or
% corrupt data, transcription errors, a copyright or other
% intellectual property infringement, a defective or damaged
% disk or other etext medium, a computer virus, or computer
% codes that damage or cannot be read by your equipment.
% 
% LIMITED WARRANTY; DISCLAIMER OF DAMAGES
% But for the "Right of Replacement or Refund" described below,
% [1] the Project (and any other party you may receive this
% etext from as a PROJECT GUTENBERG-tm etext) disclaims all
% liability to you for damages, costs and expenses, including
% legal fees, and [2] YOU HAVE NO REMEDIES FOR NEGLIGENCE OR
% UNDER STRICT LIABILITY, OR FOR BREACH OF WARRANTY OR CONTRACT,
% INCLUDING BUT NOT LIMITED TO INDIRECT, CONSEQUENTIAL, PUNITIVE
% OR INCIDENTAL DAMAGES, EVEN IF YOU GIVE NOTICE OF THE
% POSSIBILITY OF SUCH DAMAGES.
% 
% If you discover a Defect in this etext within 90 days of
% receiving it, you can receive a refund of the money (if any)
% you paid for it by sending an explanatory note within that
% time to the person you received it from.  If you received it
% on a physical medium, you must return it with your note, and
% such person may choose to alternatively give you a replacement
% copy.  If you received it electronically, such person may
% choose to alternatively give you a second opportunity to
% receive it electronically.
% 
% THIS ETEXT IS OTHERWISE PROVIDED TO YOU "AS-IS".  NO OTHER
% WARRANTIES OF ANY KIND, EXPRESS OR IMPLIED, ARE MADE TO YOU AS
% TO THE ETEXT OR ANY MEDIUM IT MAY BE ON, INCLUDING BUT NOT
% LIMITED TO WARRANTIES OF MERCHANTABILITY OR FITNESS FOR A
% PARTICULAR PURPOSE.
% 
% Some states do not allow disclaimers of implied warranties or
% the exclusion or limitation of consequential damages, so the
% above disclaimers and exclusions may not apply to you, and you
% may have other legal rights.
% 
% INDEMNITY
% You will indemnify and hold the Project, its directors,
% officers, members and agents harmless from all liability, cost
% and expense, including legal fees, that arise directly or
% indirectly from any of the following that you do or cause:
% [1] distribution of this etext, [2] alteration, modification,
% or addition to the etext, or [3] any Defect.
% 
% DISTRIBUTION UNDER "PROJECT GUTENBERG-tm"
% You may distribute copies of this etext electronically, or by
% disk, book or any other medium if you either delete this
% "Small Print!" and all other references to Project Gutenberg,
% or:
% 
% [1]  Only give exact copies of it.  Among other things, this
%      requires that you do not remove, alter or modify the
%      etext or this "small print!" statement.  You may however,
%      if you wish, distribute this etext in machine readable
%      binary, compressed, mark-up, or proprietary form,
%      including any form resulting from conversion by word pro-
%      cessing or hypertext software, but only so long as
%      *EITHER*:
% 
%      [*]  The etext, when displayed, is clearly readable, and
%           does *not* contain characters other than those
%           intended by the author of the work, although tilde
%           (~), asterisk (*) and underline (_) characters may
%           be used to convey punctuation intended by the
%           author, and additional characters may be used to
%           indicate hypertext links; OR
% 
%      [*]  The etext may be readily converted by the reader at
%           no expense into plain ASCII, EBCDIC or equivalent
%           form by the program that displays the etext (as is
%           the case, for instance, with most word processors);
%           OR
% 
%      [*]  You provide, or agree to also provide on request at
%           no additional cost, fee or expense, a copy of the
%           etext in its original plain ASCII form (or in EBCDIC
%           or other equivalent proprietary form).
% 
% [2]  Honor the etext refund and replacement provisions of this
%      "Small Print!" statement.
% 
% [3]  Pay a trademark license fee to the Project of 20% of the
%      net profits you derive calculated using the method you
%      already use to calculate your applicable taxes.  If you
%      don't derive profits, no royalty is due.  Royalties are
%      payable to "Project Gutenberg Association / Illinois
%      Benedictine College" within the 60 days following each
%      date you prepare (or were legally required to prepare)
%      your annual (or equivalent periodic) tax return.
% 
% WHAT IF YOU *WANT* TO SEND MONEY EVEN IF YOU DON'T HAVE TO?
% The Project gratefully accepts contributions in money, time,
% scanning machines, OCR software, public domain etexts, royalty
% free copyright licenses, and every other sort of contribution
% you can think of.  Money should be paid to "Project Gutenberg
% Association / Illinois Benedictine College".
% 
% This "Small Print!" by Charles B. Kramer, Attorney
% Internet (72600.2026@compuserve.com); TEL: (212-254-5093)
% *END*THE SMALL PRINT! FOR PUBLIC DOMAIN ETEXTS*Ver.04.29.93*END*

%
% converted to LaTeX by Peter Monta <pmonta@pmonta.com>
% July 2002
%

\input gutenberg-toc.tex

\begin{document}

% The Project Gutenberg Etext of A Tale of Two Cities
\gtitle{A Tale of Two Cities}

% by Charles Dickens  [The rest of Dickens is forthcoming]
\gauthor{Charles Dickens}

% CONTENTS
% 
% 
% 
% Book the First--Recalled to Life
% 
% Chapter I      The Period
% Chapter II     The Mail
% Chapter III    The Night Shadows
% Chapter IV     The Preparation
% Chapter V      The Wine-shop
% Chapter VI     The Shoemaker
% 
% 
% Book the Second--the Golden Thread
% 
% Chapter I      Five Years Later
% Chapter II     A Sight
% Chapter III    A Disappointment
% Chapter IV     Congratulatory
% Chapter V      The Jackal
% Chapter VI     Hundreds of People
% Chapter VII    Monseigneur in Town
% Chapter VIII   Monseigneur in the Country
% Chapter IX     The Gorgon's Head
% Chapter X      Two Promises
% Chapter XI     A Companion Picture
% Chapter XII    The Fellow of Delicacy
% Chapter XIII   The Fellow of no Delicacy
% Chapter XIV    The Honest Tradesman
% Chapter XV     Knitting
% Chapter XVI    Still Knitting
% Chapter XVII   One Night
% Chapter XVIII  Nine Days
% Chapter XIX    An Opinion
% Chapter XX     A Plea
% Chapter XXI    Echoing Footsteps
% Chapter XXII   The Sea still Rises
% Chapter XXIII  Fire Rises
% Chapter XXIV   Drawn to the Loadstone Rock
% 
% 
% Book the Third--the Track of a Storm
% 
% Chapter I      In Secret
% Chapter II     The Grindstone
% Chapter III    The Shadow
% Chapter IV     Calm in Storm
% Chapter V      The Wood-sawyer
% Chapter VI     Triumph
% Chapter VII    A Knock at the Door
% Chapter VIII   A Hand at Cards
% Chapter IX     The Game Made
% Chapter X      The Substance of the Shadow
% Chapter XI     Dusk
% Chapter XII    Darkness
% Chapter XIII   Fifty-two
% Chapter XIV    The Knitting Done
% Chapter XV     The Footsteps die out For ever





\part{Book the First\\Recalled to Life}



\chapter{The Period}


It was the best of times, it was the worst of times,
it was the age of wisdom, it was the age of foolishness,
it was the epoch of belief, it was the epoch of incredulity,
it was the season of Light, it was the season of Darkness,
it was the spring of hope, it was the winter of despair,
we had everything before us, we had nothing before us,
we were all going direct to Heaven, we were all going direct
the other way---in short, the period was so far like the present
period, that some of its noisiest authorities insisted on its
being received, for good or for evil, in the superlative degree
of comparison only.

There were a king with a large jaw and a queen with a plain face,
on the throne of England; there were a king with a large jaw and
a queen with a fair face, on the throne of France.  In both
countries it was clearer than crystal to the lords of the State
preserves of loaves and fishes, that things in general were
settled for ever.

It was the year of Our Lord one thousand seven hundred and
seventy-five.  Spiritual revelations were conceded to England at
that favoured period, as at this.  Mrs.\ Southcott had recently
attained her five-and-twentieth blessed birthday, of whom a
prophetic private in the Life Guards had heralded the sublime
appearance by announcing that arrangements were made for the
swallowing up of London and Westminster.  Even the Cock-lane
ghost had been laid only a round dozen of years, after rapping
out its messages, as the spirits of this very year last past
(supernaturally deficient in originality) rapped out theirs.
Mere messages in the earthly order of events had lately come to
the English Crown and People, from a congress of British subjects
in America:  which, strange to relate, have proved more important
to the human race than any communications yet received through
any of the chickens of the Cock-lane brood.

France, less favoured on the whole as to matters spiritual than
her sister of the shield and trident, rolled with exceeding
smoothness down hill, making paper money and spending it.
Under the guidance of her Christian pastors, she entertained
herself, besides, with such humane achievements as sentencing
a youth to have his hands cut off, his tongue torn out with
pincers, and his body burned alive, because he had not kneeled
down in the rain to do honour to a dirty procession of monks
which passed within his view, at a distance of some fifty or
sixty yards.  It is likely enough that, rooted in the woods of
France and Norway, there were growing trees, when that sufferer
was put to death, already marked by the Woodman, Fate, to come
down and be sawn into boards, to make a certain movable framework
with a sack and a knife in it, terrible in history.  It is likely
enough that in the rough outhouses of some tillers of the heavy
lands adjacent to Paris, there were sheltered from the weather
that very day, rude carts, bespattered with rustic mire, snuffed
about by pigs, and roosted in by poultry, which the Farmer, Death,
had already set apart to be his tumbrils of the Revolution.
But that Woodman and that Farmer, though they work unceasingly,
work silently, and no one heard them as they went about with
muffled tread:  the rather, forasmuch as to entertain any suspicion
that they were awake, was to be atheistical and traitorous.

In England, there was scarcely an amount of order and protection
to justify much national boasting.  Daring burglaries by armed
men, and highway robberies, took place in the capital itself
every night; families were publicly cautioned not to go out of
town without removing their furniture to upholsterers' warehouses
for security; the highwayman in the dark was a City tradesman in
the light, and, being recognised and challenged by his fellow-%
tradesman whom he stopped in his character of ``the Captain,''
gallantly shot him through the head and rode away; the mall was
waylaid by seven robbers, and the guard shot three dead, and then
got shot dead himself by the other four, ``in consequence of the
failure of his ammunition:'' after which the mall was robbed in
peace; that magnificent potentate, the Lord Mayor of London, was
made to stand and deliver on Turnham Green, by one highwayman,
who despoiled the illustrious creature in sight of all his
retinue; prisoners in London gaols fought battles with their
turnkeys, and the majesty of the law fired blunderbusses in among
them, loaded with rounds of shot and ball; thieves snipped off
diamond crosses from the necks of noble lords at Court
drawing-rooms; musketeers went into St. Giles's, to search for
contraband goods, and the mob fired on the musketeers, and the
musketeers fired on the mob, and nobody thought any of these
occurrences much out of the common way.  In the midst of them,
the hangman, ever busy and ever worse than useless, was in
constant requisition; now, stringing up long rows of miscellaneous
criminals; now, hanging a housebreaker on Saturday who had been
taken on Tuesday; now, burning people in the hand at Newgate by
the dozen, and now burning pamphlets at the door of Westminster Hall;
to-day, taking the life of an atrocious murderer, and to-morrow of a
wretched pilferer who had robbed a farmer's boy of sixpence.

All these things, and a thousand like them, came to pass in
and close upon the dear old year one thousand seven hundred
and seventy-five.  Environed by them, while the Woodman and the
Farmer worked unheeded, those two of the large jaws, and those
other two of the plain and the fair faces, trod with stir enough,
and carried their divine rights with a high hand.  Thus did the
year one thousand seven hundred and seventy-five conduct their
Greatnesses, and myriads of small creatures---the creatures of this
chronicle among the rest---along the roads that lay before them.


\chapter{The Mail}


It was the Dover road that lay, on a Friday night late in November,
before the first of the persons with whom this history has business.
The Dover road lay, as to him, beyond the Dover mail, as it
lumbered up Shooter's Hill.  He walked up hill in the mire
by the side of the mail, as the rest of the passengers did;
not because they had the least relish for walking exercise, under the
circumstances, but because the hill, and the harness, and the mud,
and the mail, were all so heavy, that the horses had three times
already come to a stop, besides once drawing the coach across the road,
with the mutinous intent of taking it back to Blackheath.  Reins and whip
and coachman and guard, however, in combination, had read that article
of war which forbade a purpose otherwise strongly in favour of the argument,
that some brute animals are endued with Reason; and the team had capitulated
and returned to their duty.

With drooping heads and tremulous tails, they mashed their way
through the thick mud, floundering and stumbling between whiles,
as if they were falling to pieces at the larger joints.  As often
as the driver rested them and brought them to a stand, with a
wary ``Wo-ho! so-ho- then!'' the near leader violently shook his
head and everything upon it---like an unusually emphatic horse,
denying that the coach could be got up the hill.  Whenever the
leader made this rattle, the passenger started, as a nervous
passenger might, and was disturbed in mind.

There was a steaming mist in all the hollows, and it had roamed
in its forlornness up the hill, like an evil spirit, seeking rest
and finding none.  A clammy and intensely cold mist, it made its
slow way through the air in ripples that visibly followed and
overspread one another, as the waves of an unwholesome sea might
do.  It was dense enough to shut out everything from the light of
the coach-lamps but these its own workings, and a few yards of
road; and the reek of the labouring horses steamed into it, as if
they had made it all.

Two other passengers, besides the one, were plodding up the hill
by the side of the mail.  All three were wrapped to the cheekbones
and over the ears, and wore jack-boots.  Not one of the three
could have said, from anything he saw, what either of the other
two was like; and each was hidden under almost as many wrappers
from the eyes of the mind, as from the eyes of the body, of his
two companions.  In those days, travellers were very shy of being
confidential on a short notice, for anybody on the road might be
a robber or in league with robbers.  As to the latter, when every
posting-house and ale-house could produce somebody in ``the Captain's''
pay, ranging from the landlord to the lowest stable non-descript,
it was the likeliest thing upon the cards.  So the guard of the
Dover mail thought to himself, that Friday night in November, one
thousand seven hundred and seventy-five, lumbering up Shooter's
Hill, as he stood on his own particular perch behind the mail,
beating his feet, and keeping an eye and a hand on the arm-chest
before him, where a loaded blunderbuss lay at the top of six or
eight loaded horse-pistols, deposited on a substratum of cutlass.

The Dover mail was in its usual genial position that the guard
suspected the passengers, the passengers suspected one another
and the guard, they all suspected everybody else, and the coachman
was sure of nothing but the horses; as to which cattle he could
with a clear conscience have taken his oath on the two Testaments
that they were not fit for the journey.

``Wo-ho!'' said the coachman.  ``So, then!  One more pull and you're
at the top and be damned to you, for I have had trouble enough to
get you to it!---Joe!''

``Halloa!'' the guard replied.

``What o'clock do you make it, Joe?''

``Ten minutes, good, past eleven.''

``My blood!'' ejaculated the vexed coachman, ``and not atop of
Shooter's yet!  Tst!  Yah!  Get on with you!''

The emphatic horse, cut short by the whip in a most decided
negative, made a decided scramble for it, and the three other
horses followed suit.  Once more, the Dover mail struggled on,
with the jack-boots of its passengers squashing along by its
side.  They had stopped when the coach stopped, and they kept
close company with it.  If any one of the three had had the
hardihood to propose to another to walk on a little ahead into
the mist and darkness, he would have put himself in a fair way
of getting shot instantly as a highwayman.

The last burst carried the mail to the summit of the hill.
The horses stopped to breathe again, and the guard got down to
skid the wheel for the descent, and open the coach-door to let
the passengers in.

``Tst!  Joe!'' cried the coachman in a warning voice, looking down
from his box.

``What do you say, Tom?''

They both listened.

``I say a horse at a canter coming up, Joe.''

``\emph{I} say a horse at a gallop, Tom,'' returned the guard, leaving
his hold of the door, and mounting nimbly to his place.
``Gentlemen!  In the kings name, all of you!''

With this hurried adjuration, he cocked his blunderbuss, and
stood on the offensive.

The passenger booked by this history, was on the coach-step,
getting in; the two other passengers were close behind him, and
about to follow.  He remained on the step, half in the coach and
half out of; they re-mained in the road below him.  They all
looked from the coachman to the guard, and from the guard to the
coachman, and listened.  The coachman looked back and the guard
looked back, and even the emphatic leader pricked up his ears and
looked back, without contradicting.

The stillness consequent on the cessation of the rumbling and
labouring of the coach, added to the stillness of the night, made
it very quiet indeed.  The panting of the horses communicated a
tremulous motion to the coach, as if it were in a state of
agitation.  The hearts of the passengers beat loud enough perhaps
to be heard; but at any rate, the quiet pause was audibly
expressive of people out of breath, and holding the breath, and
having the pulses quickened by expectation.

The sound of a horse at a gallop came fast and furiously up the hill.

``So-ho!'' the guard sang out, as loud as he could roar.  ``Yo there!
Stand!  I shall fire!''

The pace was suddenly checked, and, with much splashing and floundering,
a man's voice called from the mist, ``Is that the Dover mail?''

``Never you mind what it is!'' the guard retorted.  ``What are you?''

``\emph{Is} that the Dover mail?''

``Why do you want to know?''

``I want a passenger, if it is.''

``What passenger?''

``Mr.\ Jarvis Lorry.''

Our booked passenger showed in a moment that it was his name.
The guard, the coachman, and the two other passengers eyed him
distrustfully.

``Keep where you are,'' the guard called to the voice in the mist,
``because, if I should make a mistake, it could never be set right
in your lifetime.  Gentleman of the name of Lorry answer straight.''

``What is the matter?'' asked the passenger, then, with mildly
quavering speech.  ``Who wants me?  Is it Jerry?''

(``I don't like Jerry's voice, if it is Jerry,'' growled the guard
to himself.  ``He's hoarser than suits me, is Jerry.'')

``Yes, Mr.\ Lorry.''

``What is the matter?''

``A despatch sent after you from over yonder.  T. and Co.''

``I know this messenger, guard,'' said Mr.\ Lorry, getting down into
the road---assisted from behind more swiftly than politely by the
other two passengers, who immediately scrambled into the coach,
shut the door, and pulled up the window.  ``He may come close;
there's nothing wrong.''

``I hope there ain't, but I can't make so 'Nation sure of that,''
said the guard, in gruff soliloquy.  ``Hallo you!''

``Well!  And hallo you!'' said Jerry, more hoarsely than before.

``Come on at a footpace! d'ye mind me?  And if you've got holsters
to that saddle o' yourn, don't let me see your hand go nigh 'em.
For I'm a devil at a quick mistake, and when I make one it takes
the form of Lead.  So now let's look at you.''

The figures of a horse and rider came slowly through the eddying
mist, and came to the side of the mail, where the passenger stood.
The rider stooped, and, casting up his eyes at the guard, handed
the passenger a small folded paper.  The rider's horse was blown,
and both horse and rider were covered with mud, from the hoofs of
the horse to the hat of the man.

``Guard!'' said the passenger, in a tone of quiet business confidence.

The watchful guard, with his right hand at the stock of his raised
blunderbuss, his left at the barrel, and his eye on the horseman,
answered curtly, ``Sir.''

``There is nothing to apprehend.  I belong to Tellson's Bank.
You must know Tellson's Bank in London.  I am going to Paris
on business.  A crown to drink.  I may read this?''

``If so be as you're quick, sir.''

He opened it in the light of the coach-lamp on that side,
and read---first to himself and then aloud:  ``\,`Wait at Dover for
Mam'selle.' It's not long, you see, guard.  Jerry, say that my
answer was, \emph{recalled to life}.''

Jerry started in his saddle.  ``That's a Blazing strange answer, too,''
said he, at his hoarsest.

``Take that message back, and they will know that I received this,
as well as if I wrote.  Make the best of your way.  Good night.''

With those words the passenger opened the coach-door and got in;
not at all assisted by his fellow-passengers, who had
expeditiously secreted their watches and purses in their boots,
and were now making a general pretence of being asleep.  With no
more definite purpose than to escape the hazard of originating
any other kind of action.

The coach lumbered on again, with heavier wreaths of mist closing
round it as it began the descent. The guard soon replaced his
blunderbuss in his arm-chest, and, having looked to the rest of its
contents, and having looked to the supplementary pistols that he wore
in his belt, looked to a smaller chest beneath his seat, in which
there were a few smith's tools, a couple of torches, and a tinder-box.
For he was furnished with that completeness that if the coach-lamps
had been blown and stormed out, which did occasionally happen, he had
only to shut himself up inside, keep the flint and steel sparks well
off the straw, and get a light with tolerable safety and ease (if he
were lucky) in five minutes.

``Tom!'' softly over the coach roof.

``Hallo, Joe.''

``Did you hear the message?''

``I did, Joe.''

``What did you make of it, Tom?''

``Nothing at all, Joe.''

``That's a coincidence, too,'' the guard mused, ``for I made the
same of it myself.''

Jerry, left alone in the mist and darkness, dismounted meanwhile,
not only to ease his spent horse, but to wipe the mud from his
face, and shake the wet out of his hat-brim, which might be
capable of holding about half a gallon.  After standing with the
bridle over his heavily-splashed arm, until the wheels of the
mail were no longer within hearing and the night was quite still
again, he turned to walk down the hill.

``After that there gallop from Temple Bar, old lady, I won't trust
your fore-legs till I get you on the level,'' said this hoarse
messenger, glancing at his mare.  ``\,`Recalled to life.' That's a
Blazing strange message.  Much of that wouldn't do for you, Jerry!
I say, Jerry!  You'd be in a Blazing bad way, if recalling to life was
to come into fashion, Jerry!''



\chapter{The Night Shadows}


A wonderful fact to reflect upon, that every human creature is
constituted to be that profound secret and mystery to every other.
A solemn consideration, when I enter a great city by night, that
every one of those darkly clustered houses encloses its own secret;
that every room in every one of them encloses its own secret; that
every beating heart in the hundreds of thousands of breasts there,
is, in some of its imaginings, a secret to the heart nearest it!
Something of the awfulness, even of Death itself, is referable to
this.  No more can I turn the leaves of this dear book that I loved,
and vainly hope in time to read it all.  No more can I look into the
depths of this unfathomable water, wherein, as momentary lights
glanced into it, I have had glimpses of buried treasure and other
things submerged.  It was appointed that the book should shut with
a spring, for ever and for ever, when I had read but a page.  It was
appointed that the water should be locked in an eternal frost, when
the light was playing on its surface, and I stood in ignorance on the
shore.  My friend is dead, my neighbour is dead, my love, the darling
of my soul, is dead; it is the inexorable consolidation and
perpetuation of the secret that was always in that individuality,
and which I shall carry in mine to my life's end.  In any of the
burial-places of this city through which I pass, is there a sleeper
more inscrutable than its busy inhabitants are, in their innermost
personality, to me, or than I am to them?

As to this, his natural and not to be alienated inheritance,
the messenger on horseback had exactly the same possessions as
the King, the first Minister of State, or the richest merchant
in London.  So with the three passengers shut up in the narrow
compass of one lumbering old mail coach; they were mysteries to
one another, as complete as if each had been in his own coach and
six, or his own coach and sixty, with the breadth of a county
between him and the next.

The messenger rode back at an easy trot, stopping pretty often at
ale-houses by the way to drink, but evincing a tendency to keep his
own counsel, and to keep his hat cocked over his eyes.  He had eyes
that assorted very well with that decoration, being of a surface
black, with no depth in the colour or form, and much too near
together---as if they were afraid of being found out in something,
singly, if they kept too far apart.  They had a sinister expression,
under an old cocked-hat like a three-cornered spittoon, and over a
great muffler for the chin and throat, which descended nearly to the
wearer's knees.  When he stopped for drink, he moved this muffler
with his left hand, only while he poured his liquor in with his
right; as soon as that was done, he muffled again.

``No, Jerry, no!'' said the messenger, harping on one theme as he rode.
``It wouldn't do for you, Jerry.  Jerry, you honest tradesman, it
wouldn't suit \emph{your} line of business!  Recalled---!  Bust me if I
don't think he'd been a drinking!''

His message perplexed his mind to that degree that he was fain,
several times, to take off his hat to scratch his head.  Except on
the crown, which was raggedly bald, he had stiff, black hair,
standing jaggedly all over it, and growing down hill almost to his
broad, blunt nose.  It was so like Smith's work, so much more like
the top of a strongly spiked wall than a head of hair, that the best
of players at leap-frog might have declined him, as the most
dangerous man in the world to go over.

While he trotted back with the message he was to deliver to the night
watchman in his box at the door of Tellson's Bank, by Temple Bar, who
was to deliver it to greater authorities within, the shadows of the
night took such shapes to him as arose out of the message, and took
such shapes to the mare as arose out of \emph{her} private topics of
uneasiness.  They seemed to be numerous, for she shied at every
shadow on the road.

What time, the mail-coach lumbered, jolted, rattled, and bumped upon
its tedious way, with its three fellow-inscrutables inside.  To whom,
likewise, the shadows of the night revealed themselves, in the forms
their dozing eyes and wandering thoughts suggested.

Tellson's Bank had a run upon it in the mail.  As the bank passenger---%
with an arm drawn through the leathern strap, which did what lay in
it to keep him from pounding against the next passenger, and driving
him into his corner, whenever the coach got a special jolt---nodded in
his place, with half-shut eyes, the little coach-windows, and the
coach-lamp dimly gleaming through them, and the bulky bundle of
opposite passenger, became the bank, and did a great stroke of business.
The rattle of the harness was the chink of money, and more drafts
were honoured in five minutes than even Tellson's, with all its
foreign and home connection, ever paid in thrice the time.  Then the
strong-rooms underground, at Tellson's, with such of their valuable
stores and secrets as were known to the passenger (and it was not a
little that he knew about them), opened before him, and he went in
among them with the great keys and the feebly-burning candle, and
found them safe, and strong, and sound, and still, just as he had
last seen them.

But, though the bank was almost always with him, and though the coach
(in a confused way, like the presence of pain under an opiate) was
always with him, there was another current of impression that never
ceased to run, all through the night.  He was on his way to dig some
one out of a grave.

Now, which of the multitude of faces that showed themselves before
him was the true face of the buried person, the shadows of the night
did not indicate; but they were all the faces of a man of five-and-%
forty by years, and they differed principally in the passions they
expressed, and in the ghastliness of their worn and wasted state.
Pride, contempt, defiance, stubbornness, submission, lamentation,
succeeded one another; so did varieties of sunken cheek, cadaverous
colour, emaciated hands and figures.  But the face was in the main
one face, and every head was prematurely white.  A hundred times the
dozing passenger inquired of this spectre:

``Buried how long?''

The answer was always the same:  ``Almost eighteen years.''

``You had abandoned all hope of being dug out?''

``Long ago.''

``You know that you are recalled to life?''

``They tell me so.''

``I hope you care to live?''

``I can't say.''

``Shall I show her to you?  Will you come and see her?''

The answers to this question were various and contradictory.
Sometimes the broken reply was, ``Wait!  It would kill me if I saw
her too soon.''  Sometimes, it was given in a tender rain of tears,
and then it was, ``Take me to her.''  Sometimes it was staring and
bewildered, and then it was, ``I don't know her.  I don't understand.''

After such imaginary discourse, the passenger in his fancy would dig,
and dig, dig---now with a spade, now with a great key, now with his
hands---to dig this wretched creature out.  Got out at last, with
earth hanging about his face and hair, he would suddenly fan away to
dust.  The passenger would then start to himself, and lower the
window, to get the reality of mist and rain on his cheek.

Yet even when his eyes were opened on the mist and rain, on the
moving patch of light from the lamps, and the hedge at the roadside
retreating by jerks, the night shadows outside the coach would fall
into the train of the night shadows within.  The real Banking-house
by Temple Bar, the real business of the past day, the real strong
rooms, the real express sent after him, and the real message returned,
would all be there.  Out of the midst of them, the ghostly face would
rise, and he would accost it again.

``Buried how long?''

``Almost eighteen years.''

``I hope you care to live?''

``I can't say.''

Dig---dig---dig---until an impatient movement from one of the two
passengers would admonish him to pull up the window, draw his arm
securely through the leathern strap, and speculate upon the two
slumbering forms, until his mind lost its hold of them, and they
again slid away into the bank and the grave.

``Buried how long?''

``Almost eighteen years.''

``You had abandoned all hope of being dug out?''

``Long ago.''

The words were still in his hearing as just spoken---distinctly in his
hearing as ever spoken words had been in his life---when the weary
passenger started to the consciousness of daylight, and found that
the shadows of the night were gone.

He lowered the window, and looked out at the rising sun.  There was a
ridge of ploughed land, with a plough upon it where it had been left
last night when the horses were unyoked; beyond, a quiet coppice-wood,
in which many leaves of burning red and golden yellow still remained
upon the trees.  Though the earth was cold and wet, the sky was
clear, and the sun rose bright, placid, and beautiful.

``Eighteen years!'' said the passenger, looking at the sun.
``Gracious Creator of day!  To be buried alive for eighteen years!''



\chapter{The Preparation}


When the mail got successfully to Dover, in the course of the
forenoon, the head drawer at the Royal George Hotel opened the
coach-door as his custom was.  He did it with some flourish of
ceremony, for a mail journey from London in winter was an achievement
to congratulate an adventurous traveller upon.

By that time, there was only one adventurous traveller left be
congratulated:  for the two others had been set down at their
respective roadside destinations.  The mildewy inside of the coach,
with its damp and dirty straw, its disageeable smell, and its
obscurity, was rather like a larger dog-kennel.  Mr.\ Lorry, the
passenger, shaking himself out of it in chains of straw, a tangle of
shaggy wrapper, flapping hat, and muddy legs, was rather like a
larger sort of dog.

``There will be a packet to Calais, tomorrow, drawer?''

``Yes, sir, if the weather holds and the wind sets tolerable fair.
The tide will serve pretty nicely at about two in the afternoon,
sir.  Bed, sir?''

``I shall not go to bed till night; but I want a bedroom, and a barber.''

``And then breakfast, sir?  Yes, sir.  That way, sir, if you please.
Show Concord!  Gentleman's valise and hot water to Concord.  Pull off
gentleman's boots in Concord. (You will find a fine sea-coal fire,
sir.) Fetch barber to Concord.  Stir about there, now, for Concord!''

The Concord bed-chamber being always assigned to a passenger by the
mail, and passengers by the mail being always heavily wrapped up from
head to foot, the room had the odd interest for the establishment of
the Royal George, that although but one kind of man was seen to go
into it, all kinds and varieties of men came out of it. Consequently,
another drawer, and two porters, and several maids and the landlady,
were all loitering by accident at various points of the road between
the Concord and the coffee-room, when a gentleman of sixty, formally
dressed in a brown suit of clothes, pretty well worn, but very well
kept, with large square cuffs and large flaps to the pockets, passed
along on his way to his breakfast.

The coffee-room had no other occupant, that forenoon, than the
gentleman in brown.  His breakfast-table was drawn before the fire,
and as he sat, with its light shining on him, waiting for the meal,
he sat so still, that he might have been sitting for his portrait.

Very orderly and methodical he looked, with a hand on each knee, and
a loud watch ticking a sonorous sermon under his flapped waist-coat,
as though it pitted its gravity and longevity against the levity and
evanescence of the brisk fire.  He had a good leg, and was a little
vain of it, for his brown stockings fitted sleek and close, and were
of a fine texture; his shoes and buckles, too, though plain, were
trim.  He wore an odd little sleek crisp flaxen wig, setting very
close to his head:  which wig, it is to be presumed, was made of hair,
but which looked far more as though it were spun from filaments of
silk or glass.  His linen, though not of a fineness in accordance
with his stockings, was as white as the tops of the waves that broke
upon the neighbouring beach, or the specks of sail that glinted in
the sunlight far at sea.  A face habitually suppressed and quieted,
was still lighted up under the quaint wig by a pair of moist bright
eyes that it must have cost their owner, in years gone by, some pains
to drill to the composed and reserved expression of Tellson's Bank.
He had a healthy colour in his cheeks, and his face, though lined,
bore few traces of anxiety.  But, perhaps the confidential bachelor
clerks in Tellson's Bank were principally occupied with the cares of
other people; and perhaps second-hand cares, like second-hand
clothes, come easily off and on.

Completing his resemblance to a man who was sitting for his portrait,
Mr.\ Lorry dropped off to sleep.  The arrival of his breakfast roused
him, and he said to the drawer, as he moved his chair to it:

``I wish accommodation prepared for a young lady who may come here at
any time to-day.  She may ask for Mr.\ Jarvis Lorry, or she may only
ask for a gentleman from Tellson's Bank.  Please to let me know.''

``Yes, sir.  Tellson's Bank in London, sir?''

``Yes.''

``Yes, sir.  We have oftentimes the honour to entertain your gentlemen
in their travelling backwards and forwards betwixt London and Paris,
sir.  A vast deal of travelling, sir, in Tellson and Company's House.''

``Yes.  We are quite a French House, as well as an English one.''

``Yes, sir.  Not much in the habit of such travelling yourself,
I think, sir?''

``Not of late years.  It is fifteen years since we---since I---%
came last from France.''

``Indeed, sir?  That was before my time here, sir.  Before our people's
time here, sir.  The George was in other hands at that time, sir.''

``I believe so.''

``But I would hold a pretty wager, sir, that a House like Tellson and
Company was flourishing, a matter of fifty, not to speak of fifteen
years ago?''

``You might treble that, and say a hundred and fifty, yet not be far
from the truth.''

``Indeed, sir!''

Rounding his mouth and both his eyes, as he stepped backward from the
table, the waiter shifted his napkin from his right arm to his left,
dropped into a comfortable attitude, and stood surveying the guest
while he ate and drank, as from an observatory or watchtower.
According to the immemorial usage of waiters in all ages.

When Mr.\ Lorry had finished his breakfast, he went out for a stroll
on the beach.  The little narrow, crooked town of Dover hid itself
away from the beach, and ran its head into the chalk cliffs, like a
marine ostrich.  The beach was a desert of heaps of sea and stones
tumbling wildly about, and the sea did what it liked, and what it
liked was destruction.  It thundered at the town, and thundered at
the cliffs, and brought the coast down, madly.  The air among the
houses was of so strong a piscatory flavour that one might have
supposed sick fish went up to be dipped in it, as sick people went
down to be dipped in the sea.  A little fishing was done in the port,
and a quantity of strolling about by night, and looking seaward:
particularly at those times when the tide made, and was near flood.
Small tradesmen, who did no business whatever, sometimes unaccountably
realised large fortunes, and it was remarkable that nobody in the
neighbourhood could endure a lamplighter.

As the day declined into the afternoon, and the air, which had been
at intervals clear enough to allow the French coast to be seen,
became again charged with mist and vapour, Mr.\ Lorry's thoughts
seemed to cloud too.  When it was dark, and he sat before the
coffee-room fire, awaiting his dinner as he had awaited his breakfast,
his mind was busily digging, digging, digging, in the live red coals.

A bottle of good claret after dinner does a digger in the red coals
no harm, otherwise than as it has a tendency to throw him out of
work.  Mr.\ Lorry had been idle a long time, and had just poured out
his last glassful of wine with as complete an appearance of
satisfaction as is ever to be found in an elderly gentleman of a
fresh complexion who has got to the end of a bottle, when a rattling
of wheels came up the narrow street, and rumbled into the inn-yard.

He set down his glass untouched.  ``This is Mam'selle!'' said he.

In a very few minutes the waiter came in to announce that Miss
Manette had arrived from London, and would be happy to see the
gentleman from Tellson's.

``So soon?''

Miss Manette had taken some refreshment on the road, and required
none then, and was extremely anxious to see the gentleman from
Tellson's immediately, if it suited his pleasure and convenience.

The gentleman from Tellson's had nothing left for it but to empty his
glass with an air of stolid desperation, settle his odd little flaxen
wig at the ears, and follow the waiter to Miss Manette's apartment.
It was a large, dark room, furnished in a funereal manner with black
horsehair, and loaded with heavy dark tables.  These had been oiled
and oiled, until the two tall candles on the table in the middle of
the room were gloomily reflected on every leaf; as if \emph{they} were
buried, in deep graves of black mahogany, and no light to speak of
could be expected from them until they were dug out.

The obscurity was so difficult to penetrate that Mr.\ Lorry,
picking his way over the well-worn Turkey carpet, supposed

Miss Manette to be, for the moment, in some adjacent room, until,
having got past the two tall candles, he saw standing to receive him
by the table between them and the fire, a young lady of not more than
seventeen, in a riding-cloak, and still holding her straw travelling-%
hat by its ribbon in her hand.  As his eyes rested on a short, slight,
pretty figure, a quantity of golden hair, a pair of blue eyes that
met his own with an inquiring look, and a forehead with a singular
capacity (remembering how young and smooth it was), of rifting and
knitting itself into an expression that was not quite one of perplexity,
or wonder, or alarm, or merely of a bright fixed attention, though it
included all the four expressions-as his eyes rested on these things,
a sudden vivid likeness passed before him, of a child whom he had
held in his arms on the passage across that very Channel, one cold
time, when the hail drifted heavily and the sea ran high.  The
likeness passed away, like a breath along the surface of the gaunt
pier-glass behind her, on the frame of which, a hospital procession
of negro cupids, several headless and all cripples, were offering
black baskets of Dead Sea fruit to black divinities of the feminine
gender-and he made his formal bow to Miss Manette.

``Pray take a seat, sir.''  In a very clear and pleasant young voice;
a little foreign in its accent, but a very little indeed.

``I kiss your hand, miss,'' said Mr.\ Lorry, with the manners of an
earlier date, as he made his formal bow again, and took his seat.

``I received a letter from the Bank, sir, yesterday, informing me that
some intelligence---or discovery---''

``The word is not material, miss; either word will do.''

``---respecting the small property of my poor father, whom I never
saw---so long dead---''

Mr.\ Lorry moved in his chair, and cast a troubled look towards the
hospital procession of negro cupids.  As if \emph{they} had any help for
anybody in their absurd baskets!

``---rendered it necessary that I should go to Paris, there to
communicate with a gentleman of the Bank, so good as to be despatched
to Paris for the purpose.''

``Myself.''

``As I was prepared to hear, sir.''

She curtseyed to him (young ladies made curtseys in those days), with
a pretty desire to convey to him that she felt how much older and
wiser he was than she.  He made her another bow.

``I replied to the Bank, sir, that as it was considered necessary, by
those who know, and who are so kind as to advise me, that I should go
to France, and that as I am an orphan and have no friend who could go
with me, I should esteem it highly if I might be permitted to place
myself, during the journey, under that worthy gentleman's protection.
The gentleman had left London, but I think a messenger was sent after
him to beg the favour of his waiting for me here.''

``I was happy,'' said Mr.\ Lorry, ``to be entrusted with the charge.
I shall be more happy to execute it.''

``Sir, I thank you indeed.  I thank you very gratefully.  It was told
me by the Bank that the gentleman would explain to me the details of
the business, and that I must prepare myself to find them of a
surprising nature.  I have done my best to prepare myself, and I
naturally have a strong and eager interest to know what they are.''

``Naturally,'' said Mr.\ Lorry.  ``Yes---I---''

After a pause, he added, again settling the crisp flaxen wig at the ears,
``It is very difficult to begin.''

He did not begin, but, in his indecision, met her glance.  The young
forehead lifted itself into that singular expression---but it was
pretty and characteristic, besides being singular---and she raised
her hand, as if with an involuntary action she caught at, or stayed
some passing shadow.

``Are you quite a stranger to me, sir?''

``Am I not?''  Mr.\ Lorry opened his hands, and extended them outwards
with an argumentative smile.

Between the eyebrows and just over the little feminine nose, the line
of which was as delicate and fine as it was possible to be, the
expression deepened itself as she took her seat thoughtfully in the
chair by which she had hitherto remained standing.  He watched her as
she mused, and the moment she raised her eyes again, went on:

``In your adopted country, I presume, I cannot do better than address
you as a young English lady, Miss Manette?''

``If you please, sir.''

``Miss Manette, I am a man of business.  I have a business charge to
acquit myself of.  In your reception of it, don't heed me any more
than if I was a speaking machine-truly, I am not much else.  I will,
with your leave, relate to you, miss, the story of one of our
customers.''

``Story!''

He seemed wilfully to mistake the word she had repeated, when he
added, in a hurry, ``Yes, customers; in the banking business we
usually call our connection our customers.  He was a French
gentleman; a scientific gentleman; a man of great acquirements---%
a Doctor.''

``Not of Beauvais?''

``Why, yes, of Beauvais.  Like Monsieur Manette, your father,
the gentleman was of Beauvais.  Like Monsieur Manette, your father,
the gentleman was of repute in Paris.  I had the honour of knowing
him there.  Our relations were business relations, but confidential.
I was at that time in our French House, and had been---oh! twenty years.''

``At that time---I may ask, at what time, sir?''

``I speak, miss, of twenty years ago.  He married---an English
lady---and I was one of the trustees.  His affairs, like the affairs
of many other French gentlemen and French families, were entirely in
Tellson's hands.  In a similar way I am, or I have been, trustee of
one kind or other for scores of our customers.  These are mere business
relations, miss; there is no friendship in them, no particular
interest, nothing like sentiment.  I have passed from one to another,
in the course of my business life, just as I pass from one of our
customers to another in the course of my business day; in short, I
have no feelings; I am a mere machine.  To go on---''

``But this is my father's story, sir; and I begin to think''%
---the curiously roughened forehead was very intent upon him---``that
when I was left an orphan through my mother's surviving my father
only two years, it was you who brought me to England.  I am almost
sure it was you.''

Mr.\ Lorry took the hesitating little hand that confidingly advanced
to take his, and he put it with some ceremony to his lips.  He then
conducted the young lady straightway to her chair again, and, holding
the chair-back with his left hand, and using his right by turns to
rub his chin, pull his wig at the ears, or point what he said, stood
looking down into her face while she sat looking up into his.

``Miss Manette, it \emph{was} I. And you will see how truly I spoke of
myself just now, in saying I had no feelings, and that all the
relations I hold with my fellow-creatures are mere business
relations, when you reflect that I have never seen you since.
No; you have been the ward of Tellson's House since, and I have been
busy with the other business of Tellson's House since.  Feelings!
I have no time for them, no chance of them.  I pass my whole life,
miss, in turning an immense pecuniary Mangle.''

After this odd description of his daily routine of employment, Mr.\ %
Lorry flattened his flaxen wig upon his head with both hands (which
was most unnecessary, for nothing could be flatter than its shining
surface was before), and resumed his former attitude.

``So far, miss (as you have remarked), this is the story of your
regretted father.  Now comes the difference.  If your father had not
died when he did---Don't be frightened!  How you start!''

She did, indeed, start.  And she caught his wrist with both her hands.

``Pray,'' said Mr.\ Lorry, in a soothing tone, bringing his left hand
from the back of the chair to lay it on the supplicatory fingers that
clasped him in so violent a tremble:  ``pray control your agitation---%
a matter of business.  As I was saying---''

Her look so discomposed him that he stopped, wandered, and began anew:

``As I was saying; if Monsieur Manette had not died; if he had
suddenly and silently disappeared; if he had been spirited away;
if it had not been difficult to guess to what dreadful place, though
no art could trace him; if he had an enemy in some compatriot who
could exercise a privilege that I in my own time have known the boldest
people afraid to speak of in a whisper, across the water there; for
instance, the privilege of filling up blank forms for the consignment
of any one to the oblivion of a prison for any length of time; if his
wife had implored the king, the queen, the court, the clergy, for any
tidings of him, and all quite in vain;---then the history of your father
would have been the history of this unfortunate gentleman, the Doctor
of Beauvais.''

``I entreat you to tell me more, sir.''

``I will.  I am going to.  You can bear it?''

``I can bear anything but the uncertainty you leave me in at this moment.''

``You speak collectedly, and you---\emph{are} collected.  That's good!''
(Though his manner was less satisfied than his words.) ``A matter of
business.  Regard it as a matter of business-business that must be
done.  Now if this doctor's wife, though a lady of great courage and
spirit, had suffered so intensely from this cause before her little
child was born---''

``The little child was a daughter, sir.''

``A daughter.  A-a-matter of business---don't be distressed.  Miss,
if the poor lady had suffered so intensely before her little child
was born, that she came to the determination of sparing the poor
child the inheritance of any part of the agony she had known the
pains of, by rearing her in the belief that her father was dead---%
No, don't kneel!  In Heaven's name why should you kneel to me!''

``For the truth. O dear, good, compassionate sir, for the truth!''

``A-a matter of business.  You confuse me, and how can I transact
business if I am confused?  Let us be clear-headed.  If you could
kindly mention now, for instance, what nine times ninepence are,
or how many shillings in twenty guineas, it would be so encouraging.
I should be so much more at my ease about your state of mind.''

Without directly answering to this appeal, she sat so still when
he had very gently raised her, and the hands that had not ceased
to clasp his wrists were so much more steady than they had been,
that she communicated some reassurance to Mr.\ Jarvis Lorry.

``That's right, that's right.  Courage!  Business!  You have business
before you; useful business.  Miss Manette, your mother took this
course with you.  And when she died---I believe broken-hearted---%
having never slackened her unavailing search for your father,
she left you, at two years old, to grow to be blooming, beautiful,
and happy, without the dark cloud upon you of living in uncertainty
whether your father soon wore his heart out in prison, or wasted
there through many lingering years.''

As he said the words he looked down, with an admiring pity, on the
flowing golden hair; as if he pictured to himself that it might have
been already tinged with grey.

``You know that your parents had no great possession, and that what
they had was secured to your mother and to you.  There has been no
new discovery, of money, or of any other property; but---''

He felt his wrist held closer, and he stopped.  The expression in the
forehead, which had so particularly attracted his notice, and which
was now immovable, had deepened into one of pain and horror.

``But he has been---been found.  He is alive.  Greatly changed, it is
too probable; almost a wreck, it is possible; though we will hope the
best.  Still, alive.  Your father has been taken to the house of an
old servant in Paris, and we are going there:  I, to identify him if
I can:  you, to restore him to life, love, duty, rest, comfort.''

A shiver ran through her frame, and from it through his.  She said,
in a low, distinct, awe-stricken voice, as if she were saying it in a
dream,

``I am going to see his Ghost!  It will be his Ghost---not him!''

Mr.\ Lorry quietly chafed the hands that held his arm.  ``There, there,
there!  See now, see now! The best and the worst are known to you, now.
You are well on your way to the poor wronged gentleman, and, with a fair
sea voyage, and a fair land journey, you will be soon at his dear side.''

She repeated in the same tone, sunk to a whisper, ``I have been free,
I have been happy, yet his Ghost has never haunted me!''

``Only one thing more,'' said Mr.\ Lorry, laying stress upon it as a
wholesome means of enforcing her attention:  ``he has been found under
another name; his own, long forgotten or long concealed.  It would be
worse than useless now to inquire which; worse than useless to seek
to know whether he has been for years overlooked, or always designedly
held prisoner.  It would be worse than useless now to make any inquiries,
because it would be dangerous.  Better not to mention the subject,
anywhere or in any way, and to remove him---for a while at all events---%
out of France.  Even I, safe as an Englishman, and even Tellson's,
important as they are to French credit, avoid all naming of the
matter.  I carry about me, not a scrap of writing openly referring to
it.  This is a secret service altogether.  My credentials, entries,
and memoranda, are all comprehended in the one line, `Recalled to
Life;' which may mean anything.  But what is the matter!  She doesn't
notice a word!  Miss Manette!''

Perfectly still and silent, and not even fallen back in her chair,
she sat under his hand, utterly insensible; with her eyes open and
fixed upon him, and with that last expression looking as if it were
carved or branded into her forehead.  So close was her hold upon his
arm, that he feared to detach himself lest he should hurt her;
therefore he called out loudly for assistance without moving.

A wild-looking woman, whom even in his agitation, Mr.\ Lorry observed
to be all of a red colour, and to have red hair, and to be dressed in
some extraordinary tight-fitting fashion, and to have on her head a
most wonderful bonnet like a Grenadier wooden measure, and good
measure too, or a great Stilton cheese, came running into the room in
advance of the inn servants, and soon settled the question of his
detachment from the poor young lady, by laying a brawny hand upon his
chest, and sending him flying back against the nearest wall.

(``I really think this must be a man!'' was Mr.\ Lorry's breathless
reflection, simultaneously with his coming against the wall.)

``Why, look at you all!'' bawled this figure, addressing the inn
servants.  ``Why don't you go and fetch things, instead of standing
there staring at me?  I am not so much to look at, am I?  Why don't
you go and fetch things?  I'll let you know, if you don't bring
smelling-salts, cold water, and vinegar, quick, I will.''

There was an immediate dispersal for these restoratives, and she
softly laid the patient on a sofa, and tended her with great skill
and gentleness:  calling her ``my precious!'' and ``my bird!'' and spreading
her golden hair aside over her shoulders with great pride and care.

``And you in brown!'' she said, indignantly turning to Mr.\ Lorry;
``couldn't you tell her what you had to tell her, without frightening
her to death?  Look at her, with her pretty pale face and her cold
hands.  Do you call \emph{that} being a Banker?''

Mr.\ Lorry was so exceedingly disconcerted by a question so hard to
answer, that he could only look on, at a distance, with much feebler
sympathy and humility, while the strong woman, having banished the
inn servants under the mysterious penalty of ``letting them know''
something not mentioned if they stayed there, staring, recovered her
charge by a regular series of gradations, and coaxed her to lay her
drooping head upon her shoulder.

``I hope she will do well now,'' said Mr.\ Lorry.

``No thanks to you in brown, if she does.  My darling pretty!''

``I hope,'' said Mr.\ Lorry, after another pause of feeble sympathy and
humility, ``that you accompany Miss Manette to France?''

``A likely thing, too!'' replied the strong woman.  ``If it was ever
intended that I should go across salt water, do you suppose
Providence would have cast my lot in an island?''

This being another question hard to answer, Mr.\ Jarvis Lorry withdrew
to consider it.


\chapter{The Wine-shop}


A large cask of wine had been dropped and broken, in the street.
The accident had happened in getting it out of a cart; the cask had
tumbled out with a run, the hoops had burst, and it lay on the stones
just outside the door of the wine-shop, shattered like a
walnut-shell.

All the people within reach had suspended their business, or their
idleness, to run to the spot and drink the wine.  The rough,
irregular stones of the street, pointing every way, and designed,
one might have thought, expressly to lame all living creatures that
approached them, had dammed it into little pools; these were surrounded,
each by its own jostling group or crowd, according to its size.
Some men kneeled down, made scoops of their two hands joined, and
sipped, or tried to help women, who bent over their shoulders, to
sip, before the wine had all run out between their fingers.  Others,
men and women, dipped in the puddles with little mugs of mutilated
earthenware, or even with handkerchiefs from women's heads, which
were squeezed dry into infants' mouths; others made small mud-%
embankments, to stem the wine as it ran; others, directed by
lookers-on up at high windows, darted here and there, to cut off
little streams of wine that started away in new directions; others
devoted themselves to the sodden and lee-dyed pieces of the cask,
licking, and even champing the moister wine-rotted fragments with
eager relish.  There was no drainage to carry off the wine, and not
only did it all get taken up, but so much mud got taken up along with
it, that there might have been a scavenger in the street, if anybody
acquainted with it could have believed in such a miraculous presence.

A shrill sound of laughter and of amused voices---voices of men,
women, and children---resounded in the street while this wine game
lasted.  There was little roughness in the sport, and much playfulness.
There was a special companionship in it, an observable inclination on
the part of every one to join some other one, which led, especially
among the luckier or lighter-hearted, to frolicsome embraces,
drinking of healths, shaking of hands, and even joining of hands and
dancing, a dozen together.  When the wine was gone, and the places
where it had been most abundant were raked into a gridiron-pattern by
fingers, these demonstrations ceased, as suddenly as they had broken
out.  The man who had left his saw sticking in the firewood he was
cutting, set it in motion again; the women who had left on a door-step
the little pot of hot ashes, at which she had been trying to soften
the pain in her own starved fingers and toes, or in those of her
child, returned to it; men with bare arms, matted locks, and cadaverous
faces, who had emerged into the winter light from cellars, moved
away, to descend again; and a gloom gathered on the scene that
appeared more natural to it than sunshine.

The wine was red wine, and had stained the ground of the narrow
street in the suburb of Saint Antoine, in Paris, where it was
spilled.  It had stained many hands, too, and many faces, and many
naked feet, and many wooden shoes.  The hands of the man who sawed
the wood, left red marks on the billets; and the forehead of the
woman who nursed her baby, was stained with the stain of the old rag
she wound about her head again.  Those who had been greedy with the
staves of the cask, had acquired a tigerish smear about the mouth;
and one tall joker so besmirched, his head more out of a long squalid
bag of a nightcap than in it, scrawled upon a wall with his finger
dipped in muddy wine-lees---\emph{blood}.

The time was to come, when that wine too would be spilled on the
street-stones, and when the stain of it would be red upon many there.

And now that the cloud settled on Saint Antoine, which a momentary
gleam had driven from his sacred countenance, the darkness of it was
heavy-cold, dirt, sickness, ignorance, and want, were the lords in
waiting on the saintly presence-nobles of great power all of them;
but, most especially the last.  Samples of a people that had
undergone a terrible grinding and regrinding in the mill, and
certainly not in the fabulous mill which ground old people young,
shivered at every corner, passed in and out at every doorway, looked
from every window, fluttered in every vestige of a garment that the
wind shook.  The mill which had worked them down, was the mill that
grinds young people old; the children had ancient faces and grave
voices; and upon them, and upon the grown faces, and ploughed into
every furrow of age and coming up afresh, was the sigh, Hunger.  It
was prevalent everywhere.  Hunger was pushed out of the tall houses,
in the wretched clothing that hung upon poles and lines; Hunger was
patched into them with straw and rag and wood and paper; Hunger was
repeated in every fragment of the small modicum of firewood that the
man sawed off; Hunger stared down from the smokeless chimneys, and
started up from the filthy street that had no offal, among its refuse,
of anything to eat.  Hunger was the inscription on the baker's
shelves, written in every small loaf of his scanty stock of bad
bread; at the sausage-shop, in every dead-dog preparation that was
offered for sale.  Hunger rattled its dry bones among the roasting
chestnuts in the turned cylinder; Hunger was shred into atomics in
every farthing porringer of husky chips of potato, fried with some
reluctant drops of oil.

Its abiding place was in all things fitted to it.  A narrow winding
street, full of offence and stench, with other narrow winding streets
diverging, all peopled by rags and nightcaps, and all smelling of
rags and nightcaps, and all visible things with a brooding look upon
them that looked ill.  In the hunted air of the people there was yet
some wild-beast thought of the possibility of turning at bay. Depressed
and slinking though they were, eyes of fire were not wanting among
them; nor compressed lips, white with what they suppressed; nor
foreheads knitted into the likeness of the gallows-rope they mused
about enduring, or inflicting.  The trade signs (and they were almost
as many as the shops) were, all, grim illustrations of Want.  The
butcher and the porkman painted up, only the leanest scrags of meat;
the baker, the coarsest of meagre loaves.  The people rudely pictured
as drinking in the wine-shops, croaked over their scanty measures of
thin wine and beer, and were gloweringly confidential together.
Nothing was represented in a flourishing condition, save tools and
weapons; but, the cutler's knives and axes were sharp and bright, the
smith's hammers were heavy, and the gunmaker's stock was murderous.
The crippling stones of the pavement, with their many little
reservoirs of mud and water, had no footways, but broke off abruptly
at the doors.  The kennel, to make amends, ran down the middle of the
street---when it ran at all:  which was only after heavy rains, and
then it ran, by many eccentric fits, into the houses.  Across the
streets, at wide intervals, one clumsy lamp was slung by a rope and
pulley; at night, when the lamplighter had let these down, and lighted,
and hoisted them again, a feeble grove of dim wicks swung in a sickly
manner overhead, as if they were at sea.  Indeed they were at sea,
and the ship and crew were in peril of tempest.

For, the time was to come, when the gaunt scarecrows of that region
should have watched the lamplighter, in their idleness and hunger,
so long, as to conceive the idea of improving on his method, and
hauling up men by those ropes and pulleys, to flare upon the
darkness of their condition.  But, the time was not come yet; and
every wind that blew over France shook the rags of the scarecrows
in vain, for the birds, fine of song and feather, took no warning.

The wine-shop was a corner shop, better than most others in its
appearance and degree, and the master of the wine-shop had stood
outside it, in a yellow waistcoat and green breeches, looking on at
the struggle for the lost wine.  ``It's not my affair,'' said he,
with a final shrug of the shoulders.  ``The people from the market
did it.  Let them bring another.''

There, his eyes happening to catch the tall joker writing up his
joke, he called to him across the way:

``Say, then, my Gaspard, what do you do there?''

The fellow pointed to his joke with immense significance, as is often
the way with his tribe.  It missed its mark, and completely failed,
as is often the way with his tribe too.

``What now?  Are you a subject for the mad hospital?'' said the
wine-shop keeper, crossing the road, and obliterating the jest with
a handful of mud, picked up for the purpose, and smeared over it.
``Why do you write in the public streets?  Is there---tell me thou---is
there no other place to write such words in?''

In his expostulation he dropped his cleaner hand (perhaps accidentally,
perhaps not) upon the joker's heart.  The joker rapped it with his
own, took a nimble spring upward, and came down in a fantastic
dancing attitude, with one of his stained shoes jerked off his foot
into his hand, and held out.  A joker of an extremely, not to say
wolfishly practical character, he looked, under those circumstances.

``Put it on, put it on,'' said the other.  ``Call wine, wine; and finish
there.''  With that advice, he wiped his soiled hand upon the joker's
dress, such as it was---quite deliberately, as having dirtied the hand
on his account; and then recrossed the road and entered the wine-shop.

This wine-shop keeper was a bull-necked, martial-looking man of
thirty, and he should have been of a hot temperament, for, although
it was a bitter day, he wore no coat, but carried one slung over his
shoulder.  His shirt-sleeves were rolled up, too, and his brown arms
were bare to the elbows.  Neither did he wear anything more on his
head than his own crisply-curling short dark hair.  He was a dark man
altogether, with good eyes and a good bold breadth between them.
Good-humoured looking on the whole, but implacable-looking, too;
evidently a man of a strong resolution and a set purpose; a man not
desirable to be met, rushing down a narrow pass with a gulf on either
side, for nothing would turn the man.

Madame Defarge, his wife, sat in the shop behind the counter as he
came in.  Madame Defarge was a stout woman of about his own age, with
a watchful eye that seldom seemed to look at anything, a large hand
heavily ringed, a steady face, strong features, and great composure
of manner.  There was a character about Madame Defarge, from which
one might have predicated that she did not often make mistakes against
herself in any of the reckonings over which she presided.  Madame
Defarge being sensitive to cold, was wrapped in fur, and had a
quantity of bright shawl twined about her head, though not to the
concealment of her large earrings.  Her knitting was before her, but
she had laid it down to pick her teeth with a toothpick.  Thus
engaged, with her right elbow supported by her left hand, Madame
Defarge said nothing when her lord came in, but coughed just one
grain of cough.  This, in combination with the lifting of her darkly
defined eyebrows over her toothpick by the breadth of a line, suggested
to her husband that he would do well to look round the shop among the
customers, for any new customer who had dropped in while he stepped
over the way.

The wine-shop keeper accordingly rolled his eyes about, until they
rested upon an elderly gentleman and a young lady, who were seated in
a corner.  Other company were there:  two playing cards, two playing
dominoes, three standing by the counter lengthening out a short
supply of wine.  As he passed behind the counter, he took notice that
the elderly gentleman said in a look to the young lady, ``This is our
man.''

``What the devil do \emph{you} do in that galley there?'' said Monsieur
Defarge to himself; ``I don't know you.''

But, he feigned not to notice the two strangers, and fell into
discourse with the triumvirate of customers who were drinking at the
counter.

``How goes it, Jacques?'' said one of these three to Monsieur Defarge.
``Is all the spilt wine swallowed?''

``Every drop, Jacques,'' answered Monsieur Defarge.

When this interchange of Christian name was effected, Madame Defarge,
picking her teeth with her toothpick, coughed another grain of cough,
and raised her eyebrows by the breadth of another line.

``It is not often,'' said the second of the three, addressing Monsieur
Defarge, ``that many of these miserable beasts know the taste of wine,
or of anything but black bread and death.  Is it not so, Jacques?''

``It is so, Jacques,'' Monsieur Defarge returned.

At this second interchange of the Christian name, Madame Defarge,
still using her toothpick with profound composure, coughed another
grain of cough, and raised her eyebrows by the breadth of another line.

The last of the three now said his say, as he put down his empty
drinking vessel and smacked his lips.

``Ah!  So much the worse!  A bitter taste it is that such poor cattle
always have in their mouths, and hard lives they live, Jacques.
Am I right, Jacques?''

``You are right, Jacques,'' was the response of Monsieur Defarge.

This third interchange of the Christian name was completed at the
moment when Madame Defarge put her toothpick by, kept her eyebrows
up, and slightly rustled in her seat.

``Hold then!  True!'' muttered her husband.  ``Gentlemen---my wife!''

The three customers pulled off their hats to Madame Defarge, with
three flourishes.  She acknowledged their homage by bending her head,
and giving them a quick look.  Then she glanced in a casual manner
round the wine-shop, took up her knitting with great apparent
calmness and repose of spirit, and became absorbed in it.

``Gentlemen,'' said her husband, who had kept his bright eye
observantly upon her, ``good day.  The chamber, furnished bachelor-%
fashion, that you wished to see, and were inquiring for when I
stepped out, is on the fifth floor.  The doorway of the staircase
gives on the little courtyard close to the left here,'' pointing with
his hand, ``near to the window of my establishment.  But, now that I
remember, one of you has already been there, and can show the way.
Gentlemen, adieu!''

They paid for their wine, and left the place.  The eyes of Monsieur
Defarge were studying his wife at her knitting when the elderly
gentleman advanced from his corner, and begged the favour of a word.

``Willingly, sir,'' said Monsieur Defarge, and quietly stepped with him
to the door.

Their conference was very short, but very decided.  Almost at the
first word, Monsieur Defarge started and became deeply attentive.
It had not lasted a minute, when he nodded and went out.  The
gentleman then beckoned to the young lady, and they, too, went out.
Madame Defarge knitted with nimble fingers and steady eyebrows, and
saw nothing.

Mr.\ Jarvis Lorry and Miss Manette, emerging from the wine-shop thus,
joined Monsieur Defarge in the doorway to which he had directed his
own company just before.  It opened from a stinking little black
courtyard, and was the general public entrance to a great pile of
houses, inhabited by a great number of people.  In the gloomy tile-%
paved entry to the gloomy tile-paved staircase, Monsieur Defarge bent
down on one knee to the child of his old master, and put her hand to
his lips.  It was a gentle action, but not at all gently done; a very
remarkable transformation had come over him in a few seconds.  He had
no good-humour in his face, nor any openness of aspect left, but had
become a secret, angry, dangerous man.

``It is very high; it is a little difficult.  Better to begin slowly.''
Thus, Monsieur Defarge, in a stern voice, to Mr.\ Lorry, as they began
ascending the stairs.

``Is he alone?'' the latter whispered.

``Alone!  God help him, who should be with him!'' said the other, in the
same low voice.

``Is he always alone, then?''

``Yes.''

``Of his own desire?''

``Of his own necessity.  As he was, when I first saw him after they
found me and demanded to know if I would take him, and, at my peril
be discreet---as he was then, so he is now.''

``He is greatly changed?''

``Changed!''

The keeper of the wine-shop stopped to strike the wall with his hand,
and mutter a tremendous curse.  No direct answer could have been half
so forcible.  Mr.\ Lorry's spirits grew heavier and heavier, as he and
his two companions ascended higher and higher.

Such a staircase, with its accessories, in the older and more crowded
parts of Paris, would be bad enough now; but, at that time, it was
vile indeed to unaccustomed and unhardened senses.  Every little
habitation within the great foul nest of one high building---that is
to say, the room or rooms within every door that opened on the
general staircase---left its own heap of refuse on its own landing,
besides flinging other refuse from its own windows.  The uncontrollable
and hopeless mass of decomposition so engendered, would have polluted
the air, even if poverty and deprivation had not loaded it with their
intangible impurities; the two bad sources combined made it almost
insupportable.  Through such an atmosphere, by a steep dark shaft of
dirt and poison, the way lay.  Yielding to his own disturbance of
mind, and to his young companion's agitation, which became greater
every instant, Mr.\ Jarvis Lorry twice stopped to rest.  Each of these
stoppages was made at a doleful grating, by which any languishing
good airs that were left uncorrupted, seemed to escape, and all
spoilt and sickly vapours seemed to crawl in.  Through the rusted
bars, tastes, rather than glimpses, were caught of the jumbled
neighbourhood; and nothing within range, nearer or lower than the
summits of the two great towers of Notre-Dame, had any promise on it
of healthy life or wholesome aspirations.

At last, the top of the staircase was gained, and they stopped for
the third time.  There was yet an upper staircase, of a steeper
inclination and of contracted dimensions, to be ascended, before the
garret story was reached.  The keeper of the wine-shop, always going
a little in advance, and always going on the side which Mr.\ Lorry
took, as though he dreaded to be asked any question by the young
lady, turned himself about here, and, carefully feeling in the
pockets of the coat he carried over his shoulder, took out a key.

``The door is locked then, my friend?'' said Mr.\ Lorry, surprised.

``Ay.  Yes,'' was the grim reply of Monsieur Defarge.

``You think it necessary to keep the unfortunate gentleman so retired?''

``I think it necessary to turn the key.''  Monsieur Defarge whispered it
closer in his ear, and frowned heavily.

``Why?''

``Why!  Because he has lived so long, locked up, that he would be
frightened-rave-tear himself to pieces-die-come to I know not what
harm---if his door was left open.''

``Is it possible!'' exclaimed Mr.\ Lorry.

``Is it possible!'' repeated Defarge, bitterly.  ``Yes.  And a beautiful
world we live in, when it \emph{is} possible, and when many other such
things are possible, and not only possible, but done---done, see
you!---under that sky there, every day.  Long live the Devil.  Let us
go on.''

This dialogue had been held in so very low a whisper, that not a word
of it had reached the young lady's ears.  But, by this time she
trembled under such strong emotion, and her face expressed such deep
anxiety, and, above all, such dread and terror, that Mr.\ Lorry felt
it incumbent on him to speak a word or two of reassurance.

``Courage, dear miss!  Courage!  Business!  The worst will be over
in a moment; it is but passing the room-door, and the worst is over.
Then, all the good you bring to him, all the relief, all the
happiness you bring to him, begin.  Let our good friend here,
assist you on that side.  That's well, friend Defarge.  Come, now.
Business, business!''

They went up slowly and softly.  The staircase was short, and they
were soon at the top.  There, as it had an abrupt turn in it, they
came all at once in sight of three men, whose heads were bent down
close together at the side of a door, and who were intently looking
into the room to which the door belonged, through some chinks or
holes in the wall.  On hearing footsteps close at hand, these three
turned, and rose, and showed themselves to be the three of one name
who had been drinking in the wine-shop.

``I forgot them in the surprise of your visit,'' explained Monsieur
Defarge.  ``Leave us, good boys; we have business here.''

The three glided by, and went silently down.

There appearing to be no other door on that floor, and the keeper of
the wine-shop going straight to this one when they were left alone,
Mr.\ Lorry asked him in a whisper, with a little anger:

``Do you make a show of Monsieur Manette?''

``I show him, in the way you have seen, to a chosen few.''

``Is that well?''

``\emph{I} think it is well.''

``Who are the few?  How do you choose them?''

``I choose them as real men, of my name---Jacques is my name---to whom
the sight is likely to do good.  Enough; you are English; that is
another thing.  Stay there, if you please, a little moment.''

With an admonitory gesture to keep them back, he stooped, and looked
in through the crevice in the wall.  Soon raising his head again, he
struck twice or thrice upon the door---evidently with no other object
than to make a noise there.  With the same intention, he drew the key
across it, three or four times, before he put it clumsily into the
lock, and turned it as heavily as he could.

The door slowly opened inward under his hand, and he looked into the
room and said something.  A faint voice answered something.  Little
more than a single syllable could have been spoken on either side.

He looked back over his shoulder, and beckoned them to enter.
Mr.\ Lorry got his arm securely round the daughter's waist, and held
her; for he felt that she was sinking.

``A-a-a-business, business!'' he urged, with a moisture that was not of
business shining on his cheek.  ``Come in, come in!''

``I am afraid of it,'' she answered, shuddering.

``Of it?  What?''

``I mean of him.  Of my father.''

Rendered in a manner desperate, by her state and by the beckoning of
their conductor, he drew over his neck the arm that shook upon his
shoulder, lifted her a little, and hurried her into the room.  He sat
her down just within the door, and held her, clinging to him.

Defarge drew out the key, closed the door, locked it on the inside,
took out the key again, and held it in his hand.  All this he did,
methodically, and with as loud and harsh an accompaniment of noise as
he could make.  Finally, he walked across the room with a measured
tread to where the window was.  He stopped there, and faced round.

The garret, built to be a depository for firewood and the like, was
dim and dark:  for, the window of dormer shape, was in truth a door in
the roof, with a little crane over it for the hoisting up of stores
from the street:  unglazed, and closing up the middle in two pieces,
like any other door of French construction.  To exclude the cold, one
half of this door was fast closed, and the other was opened but a
very little way.  Such a scanty portion of light was admitted through
these means, that it was difficult, on first coming in, to see
anything; and long habit alone could have slowly formed in any one,
the ability to do any work requiring nicety in such obscurity.  Yet,
work of that kind was being done in the garret; for, with his back
towards the door, and his face towards the window where the keeper of
the wine-shop stood looking at him, a white-haired man sat on a low
bench, stooping forward and very busy, making shoes.



\chapter{The Shoemaker}


``Good day!'' said Monsieur Defarge, looking down at the white head
that bent low over the shoemaking.

It was raised for a moment, and a very faint voice responded to the
salutation, as if it were at a distance:

``Good day!''

``You are still hard at work, I see?''

After a long silence, the head was lifted for another moment, and the
voice replied, ``Yes---I am working.''  This time, a pair of haggard eyes
had looked at the questioner, before the face had dropped again.

The faintness of the voice was pitiable and dreadful.  It was not the
faintness of physical weakness, though confinement and hard fare no
doubt had their part in it.  Its deplorable peculiarity was, that it
was the faintness of solitude and disuse.  It was like the last
feeble echo of a sound made long and long ago.  So entirely had it
lost the life and resonance of the human voice, that it affected the
senses like a once beautiful colour faded away into a poor weak
stain.  So sunken and suppressed it was, that it was like a voice
underground.  So expressive it was, of a hopeless and lost creature,
that a famished traveller, wearied out by lonely wandering in a
wilderness, would have remembered home and friends in such a tone
before lying down to die.

Some minutes of silent work had passed:  and the haggard eyes had
looked up again:  not with any interest or curiosity, but with a dull
mechanical perception, beforehand, that the spot where the only
visitor they were aware of had stood, was not yet empty.

``I want,'' said Defarge, who had not removed his gaze from the
shoemaker, ``to let in a little more light here.  You can bear a
little more?''

The shoemaker stopped his work; looked with a vacant air of listening,
at the floor on one side of him; then similarly, at the floor on the
other side of him; then, upward at the speaker.

``What did you say?''

``You can bear a little more light?''

``I must bear it, if you let it in.''  (Laying the palest shadow of a
stress upon the second word.)

The opened half-door was opened a little further, and secured at that
angle for the time.  A broad ray of light fell into the garret, and
showed the workman with an unfinished shoe upon his lap, pausing in
his labour.  His few common tools and various scraps of leather were
at his feet and on his bench.  He had a white beard, raggedly cut,
but not very long, a hollow face, and exceedingly bright eyes.  The
hollowness and thinness of his face would have caused them to look
large, under his yet dark eyebrows and his confused white hair,
though they had been really otherwise; but, they were naturally
large, and looked unnaturally so.  His yellow rags of shirt lay open
at the throat, and showed his body to be withered and worn.  He, and
his old canvas frock, and his loose stockings, and all his poor
tatters of clothes, had, in a long seclusion from direct light and
air, faded down to such a dull uniformity of parchment-yellow, that
it would have been hard to say which was which.

He had put up a hand between his eyes and the light, and the very
bones of it seemed transparent.  So he sat, with a steadfastly vacant
gaze, pausing in his work.  He never looked at the figure before him,
without first looking down on this side of himself, then on that, as
if he had lost the habit of associating place with sound; he never
spoke, without first wandering in this manner, and forgetting to speak.

``Are you going to finish that pair of shoes to-day?'' asked Defarge,
motioning to Mr.\ Lorry to come forward.

``What did you say?''

``Do you mean to finish that pair of shoes to-day?''

``I can't say that I mean to.  I suppose so.  I don't know.''

But, the question reminded him of his work, and he bent over it again.

Mr.\ Lorry came silently forward, leaving the daughter by the door.
When he had stood, for a minute or two, by the side of Defarge, the
shoemaker looked up.  He showed no surprise at seeing another figure,
but the unsteady fingers of one of his hands strayed to his lips as
he looked at it (his lips and his nails were of the same pale lead-%
colour), and then the hand dropped to his work, and he once more bent
over the shoe.  The look and the action had occupied but an instant.

``You have a visitor, you see,'' said Monsieur Defarge.

``What did you say?''

``Here is a visitor.''

The shoemaker looked up as before, but without removing a hand from his work.

``Come!'' said Defarge.  ``Here is monsieur, who knows a well-made shoe
when he sees one.  Show him that shoe you are working at.  Take it, monsieur.''

Mr.\ Lorry took it in his hand.

``Tell monsieur what kind of shoe it is, and the maker's name.''

There was a longer pause than usual, before the shoemaker replied:

``I forget what it was you asked me.  What did you say?''

``I said, couldn't you describe the kind of shoe, for monsieur's information?''

``It is a lady's shoe.  It is a young lady's walking-shoe.  It is in the
present mode.  I never saw the mode.  I have had a pattern in my hand.''
He glanced at the shoe with some little passing touch of pride.

``And the maker's name?'' said Defarge.

Now that he had no work to hold, he laid the knuckles of the right hand
in the hollow of the left, and then the knuckles of the left hand in the
hollow of the right, and then passed a hand across his bearded chin,
and so on in regular changes, without a moment's intermission.
The task of recalling him from the vagrancy into which he always
sank when he had spoken, was like recalling some very weak person
from a swoon, or endeavouring, in the hope of some disclosure,
to stay the spirit of a fast-dying man.

``Did you ask me for my name?''

``Assuredly I did.''

``One Hundred and Five, North Tower.''

``Is that all?''

``One Hundred and Five, North Tower.''

With a weary sound that was not a sigh, nor a groan, he bent to work
again, until the silence was again broken.

``You are not a shoemaker by trade?'' said Mr.\ Lorry, looking steadfastly
at him.

His haggard eyes turned to Defarge as if he would have transferred
the question to him:  but as no help came from that quarter, they
turned back on the questioner when they had sought the ground.

``I am not a shoemaker by trade?  No, I was not a shoemaker by trade.
I-I learnt it here.  I taught myself.  I asked leave to---''

He lapsed away, even for minutes, ringing those measured changes on
his hands the whole time.  His eyes came slowly back, at last, to the
face from which they had wandered; when they rested on it, he started,
and resumed, in the manner of a sleeper that moment awake,
reverting to a subject of last night.

``I asked leave to teach myself, and I got it with much difficulty
after a long while, and I have made shoes ever since.''

As he held out his hand for the shoe that had been taken from him,
Mr.\ Lorry said, still looking steadfastly in his face:

``Monsieur Manette, do you remember nothing of me?''

The shoe dropped to the ground, and he sat looking fixedly at the
questioner.

``Monsieur Manette''; Mr.\ Lorry laid his hand upon Defarge's arm;
``do you remember nothing of this man?  Look at him.  Look at me.
Is there no old banker, no old business, no old servant, no old time,
rising in your mind, Monsieur Manette?''

As the captive of many years sat looking fixedly, by turns, at
Mr.\ Lorry and at Defarge, some long obliterated marks of an actively
intent intelligence in the middle of the forehead, gradually forced
themselves through the black mist that had fallen on him.  They were
overclouded again, they were fainter, they were gone; but they had
been there.  And so exactly was the expression repeated on the fair
young face of her who had crept along the wall to a point where she
could see him, and where she now stood looking at him, with hands
which at first had been only raised in frightened compassion, if not
even to keep him off and shut out the sight of him, but which were
now extending towards him, trembling with eagerness to lay the
spectral face upon her warm young breast, and love it back to life
and hope---so exactly was the expression repeated (though in stronger
characters) on her fair young face, that it looked as though it had
passed like a moving light, from him to her.

Darkness had fallen on him in its place.  He looked at the two, less
and less attentively, and his eyes in gloomy abstraction sought the
ground and looked about him in the old way.  Finally, with a deep
long sigh, he took the shoe up, and resumed his work.

``Have you recognised him, monsieur?'' asked Defarge in a whisper.

``Yes; for a moment.  At first I thought it quite hopeless, but I have
unquestionably seen, for a single moment, the face that I once knew
so well.  Hush!  Let us draw further back.  Hush!''

She had moved from the wall of the garret, very near to the bench on
which he sat.  There was something awful in his unconsciousness of
the figure that could have put out its hand and touched him as he
stooped over his labour.

Not a word was spoken, not a sound was made.  She stood, like a
spirit, beside him, and he bent over his work.

It happened, at length, that he had occasion to change the instrument
in his hand, for his shoemaker's knife.  It lay on that side of him
which was not the side on which she stood.  He had taken it up, and
was stooping to work again, when his eyes caught the skirt of her
dress.  He raised them, and saw her face.  The two spectators started
forward, but she stayed them with a motion of her hand.  She had no
fear of his striking at her with the knife, though they had.

He stared at her with a fearful look, and after a while his lips
began to form some words, though no sound proceeded from them.  By
degrees, in the pauses of his quick and laboured breathing, he was
heard to say:

``What is this?''

With the tears streaming down her face, she put her two hands to her
lips, and kissed them to him; then clasped them on her breast, as if
she laid his ruined head there.

``You are not the gaoler's daughter?''

She sighed ``No.''

``Who are you?''

Not yet trusting the tones of her voice, she sat down on the bench
beside him.  He recoiled, but she laid her hand upon his arm.  A
strange thrill struck him when she did so, and visibly passed over
his frame; he laid the knife down' softly, as he sat staring at her.

Her golden hair, which she wore in long curls, had been hurriedly
pushed aside, and fell down over her neck.  Advancing his hand by
little and little, he took it up and looked at it.  In the midst of
the action he went astray, and, with another deep sigh, fell to work
at his shoemaking.

But not for long.  Releasing his arm, she laid her hand upon his
shoulder.  After looking doubtfully at it, two or three times, as if
to be sure that it was really there, he laid down his work, put his
hand to his neck, and took off a blackened string with a scrap of
folded rag attached to it.  He opened this, carefully, on his knee,
and it contained a very little quantity of hair:  not more than one or
two long golden hairs, which he had, in some old day, wound off upon
his finger.

He took her hair into his hand again, and looked closely at it.  ``It
is the same.  How can it be!  When was it!  How was it!''

As the concentrated expression returned to his forehead, he seemed to
become conscious that it was in hers too.  He turned her full to the
light, and looked at her.

``She had laid her head upon my shoulder, that night when I was
summoned out---she had a fear of my going, though I had none---and when
I was brought to the North Tower they found these upon my sleeve.
'You will leave me them?  They can never help me to escape in the
body, though they may in the spirit.' Those were the words I said.
I remember them very well.''

He formed this speech with his lips many times before he could utter
it. But when he did find spoken words for it, they came to him
coherently, though slowly.

``How was this?---\emph{was it you}?''

Once more, the two spectators started, as he turned upon her with a
frightful suddenness.  But she sat perfectly still in his grasp, and
only said, in a low voice, ``I entreat you, good gentlemen, do not
come near us, do not speak, do not move!''

``Hark!'' he exclaimed.  ``Whose voice was that?''

His hands released her as he uttered this cry, and went up to his
white hair, which they tore in a frenzy.  It died out, as everything
but his shoemaking did die out of him, and he refolded his little
packet and tried to secure it in his breast; but he still looked at
her, and gloomily shook his head.

``No, no, no; you are too young, too blooming.  It can't be.  See what
the prisoner is.  These are not the hands she knew, this is not the
face she knew, this is not a voice she ever heard.  No, no.  She
was---and He was---before the slow years of the North Tower---ages ago.
What is your name, my gentle angel?''

Hailing his softened tone and manner, his daughter fell upon her
knees before him, with her appealing hands upon his breast.

``O, sir, at another time you shall know my name, and who my mother
was, and who my father, and how I never knew their hard, hard
history.  But I cannot tell you at this time, and I cannot tell you
here.  All that I may tell you, here and now, is, that I pray to you
to touch me and to bless me. Kiss me, kiss me!  O my dear, my dear!''

His cold white head mingled with her radiant hair, which warmed and
lighted it as though it were the light of Freedom shining on him.

``If you hear in my voice---I don't know that it is so, but I hope it
is---if you hear in my voice any resemblance to a voice that once was
sweet music in your ears, weep for it, weep for it!  If you touch,
in touching my hair, anything that recalls a beloved head that lay on
your breast when you were young and free, weep for it, weep for it!
If, when I hint to you of a Home that is before us, where I will be
true to you with all my duty and with all my faithful service, I
bring back the remembrance of a Home long desolate, while your poor
heart pined away, weep for it, weep for it!''

She held him closer round the neck, and rocked him on her breast
like a child.

``If, when I tell you, dearest dear, that your agony is over, and that
I have come here to take you from it, and that we go to England to be
at peace and at rest, I cause you to think of your useful life laid
waste, and of our native France so wicked to you, weep for it, weep
for it!  And if, when I shall tell you of my name, and of my father
who is living, and of my mother who is dead, you learn that I have to
kneel to my honoured father, and implore his pardon for having never
for his sake striven all day and lain awake and wept all night,
because the love of my poor mother hid his torture from me, weep for
it, weep for it!  Weep for her, then, and for me!  Good gentlemen,
thank God!  I feel his sacred tears upon my face, and his sobs strike
against my heart. O, see!  Thank God for us, thank God!''

He had sunk in her arms, and his face dropped on her breast:  a sight
so touching, yet so terrible in the tremendous wrong and suffering
which had gone before it, that the two beholders covered their faces.

When the quiet of the garret had been long undisturbed, and his
heaving breast and shaken form had long yielded to the calm that must
follow all storms---emblem to humanity, of the rest and silence into
which the storm called Life must hush at last---they came forward to
raise the father and daughter from the ground.  He had gradually
dropped to the floor, and lay there in a lethargy, worn out.  She had
nestled down with him, that his head might lie upon her arm; and her
hair drooping over him curtained him from the light.

``If, without disturbing him,'' she said, raising her hand to Mr.\ Lorry
as he stooped over them, after repeated blowings of his nose, ``all
could be arranged for our leaving Paris at once, so that, from the,
very door, he could be taken away---''

``But, consider.  Is he fit for the journey?'' asked Mr.\ Lorry.

``More fit for that, I think, than to remain in this city, so dreadful to him.''

``It is true,'' said Defarge, who was kneeling to look on and hear.
``More than that; Monsieur Manette is, for all reasons, best out of
France.  Say, shall I hire a carriage and post-horses?''

``That's business,'' said Mr.\ Lorry, resuming on the shortest notice
his methodical manners; ``and if business is to be done, I had better do it.''

``Then be so kind,'' urged Miss Manette, ``as to leave us here.  You see
how composed he has become, and you cannot be afraid to leave him
with me now.  Why should you be?  If you will lock the door to secure
us from interruption, I do not doubt that you will find him, when you
come back, as quiet as you leave him.  In any case, I will take care
of him until you return, and then we will remove him straight.''

Both Mr.\ Lorry and Defarge were rather disinclined to this course,
and in favour of one of them remaining.  But, as there were not only
carriage and horses to be seen to, but travelling papers; and as time
pressed, for the day was drawing to an end, it came at last to their
hastily dividing the business that was necessary to be done, and
hurrying away to do it.

Then, as the darkness closed in, the daughter laid her head down on
the hard ground close at the father's side, and watched him.  The
darkness deepened and deepened, and they both lay quiet, until a
light gleamed through the chinks in the wall.

Mr.\ Lorry and Monsieur Defarge had made all ready for the journey,
and had brought with them, besides travelling cloaks and wrappers,
bread and meat, wine, and hot coffee.  Monsieur Defarge put this
provender, and the lamp he carried, on the shoemaker's bench (there
was nothing else in the garret but a pallet bed), and he and
Mr.\ Lorry roused the captive, and assisted him to his feet.

No human intelligence could have read the mysteries of his mind, in
the scared blank wonder of his face.  Whether he knew what had
happened, whether he recollected what they had said to him, whether
he knew that he was free, were questions which no sagacity could have
solved.  They tried speaking to him; but, he was so confused, and so
very slow to answer, that they took fright at his bewilderment, and
agreed for the time to tamper with him no more.  He had a wild, lost
manner of occasionally clasping his head in his hands, that had not
been seen in him before; yet, he had some pleasure in the mere sound
of his daughter's voice, and invariably turned to it when she spoke.

In the submissive way of one long accustomed to obey under coercion,
he ate and drank what they gave him to eat and drink, and put on the
cloak and other wrappings, that they gave him to wear.  He readily
responded to his daughter's drawing her arm through his, and
took---and kept---her hand in both his own.

They began to descend; Monsieur Defarge going first with the lamp,
Mr.\ Lorry closing the little procession.  They had not traversed many
steps of the long main staircase when he stopped, and stared at the
roof and round at the wails.

``You remember the place, my father?  You remember coming up here?''

``What did you say?''

But, before she could repeat the question, he murmured an answer as
if she had repeated it.

``Remember?  No, I don't remember.  It was so very long ago.''

That he had no recollection whatever of his having been brought from
his prison to that house, was apparent to them.  They heard him mutter,
``One Hundred and Five, North Tower;'' and when he looked about him, it
evidently was for the strong fortress-walls which had long encompassed him.
On their reaching the courtyard he instinctively altered his tread,
as being in expectation of a drawbridge; and when there was no
drawbridge, and he saw the carriage waiting in the open street, he
dropped his daughter's hand and clasped his head again.

No crowd was about the door; no people were discernible at any of the
many windows; not even a chance passerby was in the street.  An unnatural
silence and desertion reigned there.  Only one soul was to be seen,
and that was Madame Defarge---who leaned against the door-post,
knitting, and saw nothing.

The prisoner had got into a coach, and his daughter had followed him,
when Mr.\ Lorry's feet were arrested on the step by his asking,
miserably, for his shoemaking tools and the unfinished shoes.  Madame
Defarge immediately called to her husband that she would get them,
and went, knitting, out of the lamplight, through the courtyard.  She
quickly brought them down and handed them in;---and immediately
afterwards leaned against the door-post, knitting, and saw nothing.

Defarge got upon the box, and gave the word ``To the Barrier!''
The postilion cracked his whip, and they clattered away under
the feeble over-swinging lamps.

Under the over-swinging lamps---swinging ever brighter in the better
streets, and ever dimmer in the worse---and by lighted shops, gay
crowds, illuminated coffee-houses, and theatre-doors, to one of the
city gates.  Soldiers with lanterns, at the guard-house there.
``Your papers, travellers!''  ``See here then, Monsieur the Officer,''
said Defarge, getting down, and taking him gravely apart, ``these are
the papers of monsieur inside, with the white head.  They were
consigned to me, with him, at the---'' He dropped his voice, there was
a flutter among the military lanterns, and one of them being handed
into the coach by an arm in uniform, the eyes connected with the arm
looked, not an every day or an every night look, at monsieur with the
white head.  ``It is well.  Forward!'' from the uniform.  ``Adieu!'' from
Defarge.  And so, under a short grove of feebler and feebler
over-swinging lamps, out under the great grove of stars.

Beneath that arch of unmoved and eternal lights; some, so remote from
this little earth that the learned tell us it is doubtful whether
their rays have even yet discovered it, as a point in space where
anything is suffered or done:  the shadows of the night were broad and
black.  All through the cold and restless interval, until dawn, they
once more whispered in the ears of Mr.\ Jarvis Lorry---sitting opposite
the buried man who had been dug out, and wondering what subtle powers
were for ever lost to him, and what were capable of restoration---the
old inquiry:

``I hope you care to be recalled to life?''

And the old answer:

``I can't say.''



% The end of the first book.




\cleartorecto
\part{Book the Second\\The Golden Thread}



\chapter{Five Years Later}


Tellson's Bank by Temple Bar was an old-fashioned place, even in the
year one thousand seven hundred and eighty.  It was very small, very
dark, very ugly, very incommodious.  It was an old-fashioned place,
moreover, in the moral attribute that the partners in the House were
proud of its smallness, proud of its darkness, proud of its ugliness,
proud of its incommodiousness.  They were even boastful of its
eminence in those particulars, and were fired by an express conviction
that, if it were less objectionable, it would be less respectable.
This was no passive belief, but an active weapon which they flashed
at more convenient places of business.  Tellson's (they said) wanted
no elbow-room, Tellson's wanted no light, Tellson's wanted no
embellishment.  Noakes and Co.'s might, or Snooks Brothers' might;
but Tellson's, thank Heaven!---%

Any one of these partners would have disinherited his son on the
question of rebuilding Tellson's.  In this respect the House was much
on a par with the Country; which did very often disinherit its sons
for suggesting improvements in laws and customs that had long been
highly objectionable, but were only the more respectable.

Thus it had come to pass, that Tellson's was the triumphant
perfection of inconvenience.  After bursting open a door of idiotic
obstinacy with a weak rattle in its throat, you fell into Tellson's
down two steps, and came to your senses in a miserable little shop,
with two little counters, where the oldest of men made your cheque
shake as if the wind rustled it, while they examined the signature by
the dingiest of windows, which were always under a shower-bath of mud
from Fleet-street, and which were made the dingier by their own iron
bars proper, and the heavy shadow of Temple Bar.  If your business
necessitated your seeing ``the House,'' you were put into a species of
Condemned Hold at the back, where you meditated on a misspent life,
until the House came with its hands in its pockets, and you could
hardly blink at it in the dismal twilight.  Your money came out of,
or went into, wormy old wooden drawers, particles of which flew up
your nose and down your throat when they were opened and shut.  Your
bank-notes had a musty odour, as if they were fast decomposing into
rags again.  Your plate was stowed away among the neighbouring
cesspools, and evil communications corrupted its good polish in a day
or two.  Your deeds got into extemporised strong-rooms made of
kitchens and sculleries, and fretted all the fat out of their
parchments into the banking-house air.  Your lighter boxes of family
papers went up-stairs into a Barmecide room, that always had a great
dining-table in it and never had a dinner, and where, even in the
year one thousand seven hundred and eighty, the first letters written
to you by your old love, or by your little children, were but newly
released from the horror of being ogled through the windows, by the
heads exposed on Temple Bar with an insensate brutality and ferocity
worthy of Abyssinia or Ashantee.

But indeed, at that time, putting to death was a recipe much in vogue
with all trades and professions, and not least of all with Tellson's.
Death is Nature's remedy for all things, and why not Legislation's?
Accordingly, the forger was put to Death; the utterer of a bad note
was put to Death; the unlawful opener of a letter was put to Death;
the purloiner of forty shillings and sixpence was put to Death; the
holder of a horse at Tellson's door, who made off with it, was put to
Death; the coiner of a bad shilling was put to Death; the sounders of
three-fourths of the notes in the whole gamut of Crime, were put to
Death.  Not that it did the least good in the way of prevention---it
might almost have been worth remarking that the fact was exactly the
reverse---but, it cleared off (as to this world) the trouble of each
particular case, and left nothing else connected with it to be looked
after.  Thus, Tellson's, in its day, like greater places of business,
its contemporaries, had taken so many lives, that, if the heads laid
low before it had been ranged on Temple Bar instead of being
privately disposed of, they would probably have excluded what little
light the ground floor had, in a rather significant manner.

Cramped in all kinds of dun cupboards and hutches at Tellson's, the
oldest of men carried on the business gravely.  When they took a
young man into Tellson's London house, they hid him somewhere till he
was old.  They kept him in a dark place, like a cheese, until he had
the full Tellson flavour and blue-mould upon him.  Then only was he
permitted to be seen, spectacularly poring over large books, and
casting his breeches and gaiters into the general weight of the
establishment.

Outside Tellson's---never by any means in it, unless called in---was an
odd-job-man, an occasional porter and messenger, who served as the
live sign of the house.  He was never absent during business hours,
unless upon an errand, and then he was represented by his son:  a
grisly urchin of twelve, who was his express image.  People
understood that Tellson's, in a stately way, tolerated the
odd-job-man.  The house had always tolerated some person in that
capacity, and time and tide had drifted this person to the post.  His
surname was Cruncher, and on the youthful occasion of his renouncing
by proxy the works of darkness, in the easterly parish church of
Hounsditch, he had received the added appellation of Jerry.

The scene was Mr.\ Cruncher's private lodging in Hanging-sword-alley,
Whitefriars:  the time, half-past seven of the clock on a windy March
morning, Anno Domini seventeen hundred and eighty. (Mr.\ Cruncher
himself always spoke of the year of our Lord as Anna Dominoes:
apparently under the impression that the Christian era dated from the
invention of a popular game, by a lady who had bestowed her name upon it.)

Mr.\ Cruncher's apartments were not in a savoury neighbourhood, and
were but two in number, even if a closet with a single pane of glass
in it might be counted as one.  But they were very decently kept.
Early as it was, on the windy March morning, the room in which he lay
abed was already scrubbed throughout; and between the cups and
saucers arranged for breakfast, and the lumbering deal table, a very
clean white cloth was spread.

Mr.\ Cruncher reposed under a patchwork counterpane, like a Harlequin
at home.  At fast, he slept heavily, but, by degrees, began to roll
and surge in bed, until he rose above the surface, with his spiky
hair looking as if it must tear the sheets to ribbons.  At which
juncture, he exclaimed, in a voice of dire exasperation:

``Bust me, if she ain't at it agin!''

A woman of orderly and industrious appearance rose from her knees in
a corner, with sufficient haste and trepidation to show that she was
the person referred to.

``What!'' said Mr.\ Cruncher, looking out of bed for a boot.  ``You're at
it agin, are you?''

After hailing the mom with this second salutation, he threw a boot at
the woman as a third.  It was a very muddy boot, and may introduce
the odd circumstance connected with Mr.\ Cruncher's domestic economy,
that, whereas he often came home after banking hours with clean
boots, he often got up next morning to find the same boots
covered with clay.

``What,'' said Mr.\ Cruncher, varying his apostrophe after missing
his mark---``what are you up to, Aggerawayter?''

``I was only saying my prayers.''

``Saying your prayers!  You're a nice woman!  What do you mean by
flopping yourself down and praying agin me?''

``I was not praying against you; I was praying for you.''

``You weren't.  And if you were, I won't be took the liberty with.
Here! your mother's a nice woman, young Jerry, going a praying agin
your father's prosperity.  You've got a dutiful mother, you have, my
son.  You've got a religious mother, you have, my boy:  going and
flopping herself down, and praying that the bread-and-butter may be
snatched out of the mouth of her only child.''

Master Cruncher (who was in his shirt) took this very ill, and,
turning to his mother, strongly deprecated any praying away of his
personal board.

``And what do you suppose, you conceited female,'' said Mr.\ Cruncher,
with unconscious inconsistency, ``that the worth of \emph{your} prayers may be?
Name the price that you put \emph{your} prayers at!''

``They only come from the heart, Jerry.  They are worth no more than that.''

``Worth no more than that,'' repeated Mr.\ Cruncher.
``They ain't worth much, then.  Whether or no,
I won't be prayed agin, I tell you.  I can't afford it.
I'm not a going to be made unlucky by \emph{your} sneaking.
If you must go flopping yourself down, flop in favour
of your husband and child, and not in opposition to 'em. If I
had had any but a unnat'ral wife, and this poor boy had had any but
a unnat'ral mother, I might have made some money last week instead
of being counter-prayed and countermined and religiously circumwented
into the worst of luck.  B-u-u-ust me!'' said Mr.\ Cruncher, who all
this time had been putting on his clothes, ``if I ain't, what with
piety and one blowed thing and another, been choused this last week
into as bad luck as ever a poor devil of a honest tradesman met with!
Young Jerry, dress yourself, my boy, and while I clean my boots keep
a eye upon your mother now and then, and if you see any signs of more
flopping, give me a call.  For, I tell you,'' here he addressed his
wife once more, ``I won't be gone agin, in this manner.  I am as
rickety as a hackney-coach, I'm as sleepy as laudanum, my lines is
strained to that degree that I shouldn't know, if it wasn't for the
pain in 'em, which was me and which somebody else, yet I'm none the
better for it in pocket; and it's my suspicion that you've been at it
from morning to night to prevent me from being the better for it in pocket,
and I won't put up with it, Aggerawayter, and what do you say now!''

Growling, in addition, such phrases as ``Ah! yes!  You're religious, too.
You wouldn't put yourself in opposition to the interests of your husband
and child, would you?  Not you!'' and throwing off other sarcastic sparks
from the whirling grindstone of his indignation, Mr.\ Cruncher betook
himself to his boot-cleaning and his general preparation for business.
In the meantime, his son, whose head was garnished with tenderer spikes,
and whose young eyes stood close by one another, as his father's did,
kept the required watch upon his mother.  He greatly disturbed that
poor woman at intervals, by darting out of his sleeping closet,
where he made his toilet, with a suppressed cry of ``You are going to flop,
mother.  ---Halloa, father!'' and, after raising this fictitious alarm,
darting in again with an undutiful grin.

Mr.\ Cruncher's temper was not at all improved when he came to his
breakfast.  He resented Mrs.\ Cruncher's saying grace with particular
animosity.

``Now, Aggerawayter!  What are you up to?  At it again?''

His wife explained that she had merely ``asked a blessing.''

``Don't do it!'' said Mr.\ Crunches looking about, as if he rather
expected to see the loaf disappear under the efficacy of his wife's
petitions.  ``I ain't a going to be blest out of house and home.
I won't have my wittles blest off my table.  Keep still!''

Exceedingly red-eyed and grim, as if he had been up all night at a
party which had taken anything but a convivial turn, Jerry Cruncher
worried his breakfast rather than ate it, growling over it like any
four-footed inmate of a menagerie.  Towards nine o'clock he smoothed
his ruffled aspect, and, presenting as respectable and business-like
an exterior as he could overlay his natural self with, issued forth
to the occupation of the day.

It could scarcely be called a trade, in spite of his favourite
description of himself as ``a honest tradesman.''  His stock consisted
of a wooden stool, made out of a broken-backed chair cut down, which
stool, young Jerry, walking at his father's side, carried every
morning to beneath the banking-house window that was nearest Temple
Bar:  where, with the addition of the first handful of straw that
could be gleaned from any passing vehicle to keep the cold and wet
from the odd-job-man's feet, it formed the encampment for the day.
On this post of his, Mr.\ Cruncher was as well known to Fleet-street
and the Temple, as the Bar itself,---and was almost as in-looking.

Encamped at a quarter before nine, in good time to touch his three-%
cornered hat to the oldest of men as they passed in to Tellson's,
Jerry took up his station on this windy March morning, with young
Jerry standing by him, when not engaged in making forays through the
Bar, to inflict bodily and mental injuries of an acute description on
passing boys who were small enough for his amiable purpose.  Father
and son, extremely like each other, looking silently on at the
morning traffic in Fleet-street, with their two heads as near to one
another as the two eyes of each were, bore a considerable resemblance
to a pair of monkeys.  The resemblance was not lessened by the
accidental circumstance, that the mature Jerry bit and spat out
straw, while the twinkling eyes of the youthful Jerry were as
restlessly watchful of him as of everything else in Fleet-street.

The head of one of the regular indoor messengers attached to
Tellson's establishment was put through the door, and the word was
given:

``Porter wanted!''

``Hooray, father!  Here's an early job to begin with!''

Having thus given his parent God speed, young Jerry seated himself on
the stool, entered on his reversionary interest in the straw his
father had been chewing, and cogitated.

``Al-ways rusty!  His fingers is al-ways rusty!'' muttered young Jerry.
``Where does my father get all that iron rust from?  He don't get no
iron rust here!''



\chapter{A Sight}


``You know the Old Bailey, well, no doubt?'' said one of the oldest of
clerks to Jerry the messenger.

``Ye-es, sir,'' returned Jerry, in something of a dogged manner.  ``I
\emph{do} know the Bailey.''

``Just so.  And you know Mr.\ Lorry.''

``I know Mr.\ Lorry, sir, much better than I know the Bailey.  Much
better,'' said Jerry, not unlike a reluctant witness at the
establishment in question, ``than I, as a honest tradesman, wish to
know the Bailey.''

``Very well.  Find the door where the witnesses go in, and show the
door-keeper this note for Mr.\ Lorry.  He will then let you in.''

``Into the court, sir?''

``Into the court.''

Mr.\ Cruncher's eyes seemed to get a little closer to one another, and
to interchange the inquiry, ``What do you think of this?''

``Am I to wait in the court, sir?'' he asked, as the result of that
conference.

``I am going to tell you.  The door-keeper will pass the note to Mr.\ %
Lorry, and do you make any gesture that will attract Mr.\ Lorry's
attention, and show him where you stand.  Then what you have to do,
is, to remain there until he wants you.''

``Is that all, sir?''

``That's all.  He wishes to have a messenger at hand.  This is to tell
him you are there.''

As the ancient clerk deliberately folded and superscribed the note,
Mr.\ Cruncher, after surveying him in silence until he came to the
blotting-paper stage, remarked:

``I suppose they'll be trying Forgeries this morning?''

``Treason!''

``That's quartering,'' said Jerry.  ``Barbarous!''

``It is the law,'' remarked the ancient clerk, turning his surprised
spectacles upon him.  ``It is the law.''

``It's hard in the law to spile a man, I think.  Ifs hard enough to
kill him, but it's wery hard to spile him, sir.''

``Not at all,'' retained the ancient clerk.  ``Speak well of the law.
Take care of your chest and voice, my good friend, and leave the law
to take care of itself.  I give you that advice.''

``It's the damp, sir, what settles on my chest and voice,'' said Jerry.
``I leave you to judge what a damp way of earning a living mine is.''

``Well, well,'' said the old clerk; ``we all have our various ways of
gaining a livelihood.  Some of us have damp ways, and some of us have
dry ways.  Here is the letter.  Go along.''

Jerry took the letter, and, remarking to himself with less internal
deference than he made an outward show of, ``You are a lean old one,
too,'' made his bow, informed his son, in passing, of his destination,
and went his way.

They hanged at Tyburn, in those days, so the street outside Newgate
had not obtained one infamous notoriety that has since attached to
it.  But, the gaol was a vile place, in which most kinds of
debauchery and villainy were practised, and where dire diseases were
bred, that came into court with the prisoners, and sometimes rushed
straight from the dock at my Lord Chief Justice himself, and pulled
him off the bench.  It had more than once happened, that the Judge in
the black cap pronounced his own doom as certainly as the prisoner's,
and even died before him.  For the rest, the Old Bailey was famous as
a kind of deadly inn-yard, from which pale travellers set out
continually, in carts and coaches, on a violent passage into the
other world:  traversing some two miles and a half of public street
and road, and shaming few good citizens, if any.  So powerful is use,
and so desirable to be good use in the beginning.  It was famous,
too, for the pillory, a wise old institution, that inflicted a
punishment of which no one could foresee the extent; also, for the
whipping-post, another dear old institution, very humanising and
softening to behold in action; also, for extensive transactions in
blood-money, another fragment of ancestral wisdom, systematically
leading to the most frightful mercenary crimes that could be
committed under Heaven.  Altogether, the Old Bailey, at that date,
was a choice illustration of the precept, that ``Whatever is is right;''
an aphorism that would be as final as it is lazy, did it not include
the troublesome consequence, that nothing that ever was, was wrong.

Making his way through the tainted crowd, dispersed up and down this
hideous scene of action, with the skill of a man accustomed to make
his way quietly, the messenger found out the door he sought, and
handed in his letter through a trap in it.  For, people then paid to
see the play at the Old Bailey, just as they paid to see the play in
Bedlam---only the former entertainment was much the dearer. Therefore,
all the Old Bailey doors were well guarded---except, indeed, the
social doors by which the criminals got there, and those were always
left wide open.

After some delay and demur, the door grudgingly turned on its hinges
a very little way, and allowed Mr.\ Jerry Cruncher to squeeze himself
into court.

``What's on?'' he asked, in a whisper, of the man he found himself next to.

``Nothing yet.''

``What's coming on?''

``The Treason case.''

``The quartering one, eh?''

``Ah!'' returned the man, with a relish; ``he'll be drawn on a hurdle
to be half hanged, and then he'll be taken down and sliced before
his own face, and then his inside will be taken out and burnt while
he looks on, and then his head will be chopped off, and he'll be
cut into quarters.  That's the sentence.''

``If he's found Guilty, you mean to say?'' Jerry added, by way of proviso.

``Oh! they'll find him guilty,'' said the other.  ``Don't you be afraid of that.''

Mr.\ Cruncher's attention was here diverted to the door-keeper, whom
he saw making his way to Mr.\ Lorry, with the note in his hand.  Mr.\ %
Lorry sat at a table, among the gentlemen in wigs:  not far from a
wigged gentleman, the prisoner's counsel, who had a great bundle of
papers before him:  and nearly opposite another wigged gentleman with
his hands in his pockets, whose whole attention, when Mr.\ Cruncher
looked at him then or afterwards, seemed to be concentrated on the
ceiling of the court.  After some gruff coughing and rubbing of his
chin and signing with his hand, Jerry attracted the notice of
Mr.\ Lorry, who had stood up to look for him, and who quietly nodded
and sat down again.

``What's \emph{he} got to do with the case?'' asked the man he had spoken with.

``Blest if I know,'' said Jerry.

``What have \emph{you} got to do with it, then, if a person may inquire?''

``Blest if I know that either,'' said Jerry.

The entrance of the Judge, and a consequent great stir and settling
down in the court, stopped the dialogue.  Presently, the dock became
the central point of interest.  Two gaolers, who had been standing
there, wont out, and the prisoner was brought in, and put to the bar.

Everybody present, except the one wigged gentleman who looked at the
ceiling, stared at him.  All the human breath in the place, rolled at
him, like a sea, or a wind, or a fire.  Eager faces strained round
pillars and corners, to get a sight of him; spectators in back rows
stood up, not to miss a hair of him; people on the floor of the
court, laid their hands on the shoulders of the people before them,
to help themselves, at anybody's cost, to a view of him---stood
a-tiptoe, got upon ledges, stood upon next to nothing, to see every
inch of him.  Conspicuous among these latter, like an animated bit of
the spiked wall of Newgate, Jerry stood:  aiming at the prisoner the
beery breath of a whet he had taken as he came along, and discharging
it to mingle with the waves of other beer, and gin, and tea, and
coffee, and what not, that flowed at him, and already broke upon the
great windows behind him in an impure mist and rain.

The object of all this staring and blaring, was a young man of about
five-and-twenty, well-grown and well-looking, with a sunburnt cheek
and a dark eye.  His condition was that of a young gentleman.  He was
plainly dressed in black, or very dark grey, and his hair, which was
long and dark, was gathered in a ribbon at the back of his neck; more
to be out of his way than for ornament.  As an emotion of the mind
will express itself through any covering of the body, so the paleness
which his situation engendered came through the brown upon his cheek,
showing the soul to be stronger than the sun.  He was otherwise quite
self-possessed, bowed to the Judge, and stood quiet.

The sort of interest with which this man was stared and breathed at,
was not a sort that elevated humanity.  Had he stood in peril of a
less horrible sentence---had there been a chance of any one of its
savage details being spared---by just so much would he have lost in
his fascination.  The form that was to be doomed to be so shamefully
mangled, was the sight; the immortal creature that was to be so
butchered and torn asunder, yielded the sensation.  Whatever gloss
the various spectators put upon the interest, according to their
several arts and powers of self-deceit, the interest was, at the
root of it, Ogreish.

Silence in the court!  Charles Darnay had yesterday pleaded Not Guilty
to an indictment denouncing him (with infinite jingle and jangle) for
that he was a false traitor to our serene, illustrious, excellent,
and so forth, prince, our Lord the King, by reason of his having, on
divers occasions, and by divers means and ways, assisted Lewis, the
French King, in his wars against our said serene, illustrious,
excellent, and so forth; that was to say, by coming and going,
between the dominions of our said serene, illustrious, excellent, and
so forth, and those of the said French Lewis, and wickedly, falsely,
traitorously, and otherwise evil-adverbiously, revealing to the said
French Lewis what forces our said serene, illustrious, excellent, and
so forth, had in preparation to send to Canada and North America.
This much, Jerry, with his head becoming more and more spiky as the
law terms bristled it, made out with huge satisfaction, and so
arrived circuitously at the understanding that the aforesaid, and
over and over again aforesaid, Charles Darnay, stood there before him
upon his trial; that the jury were swearing in; and that
Mr.\ Attorney-General was making ready to speak.

The accused, who was (and who knew he was) being mentally hanged,
beheaded, and quartered, by everybody there, neither flinched from
the situation, nor assumed any theatrical air in it.  He was quiet
and attentive; watched the opening proceedings with a grave interest;
and stood with his hands resting on the slab of wood before him, so
composedly, that they had not displaced a leaf of the herbs with
which it was strewn.  The court was all bestrewn with herbs and
sprinkled with vinegar, as a precaution against gaol air and gaol
fever.

Over the prisoner's head there was a mirror, to throw the light down
upon him.  Crowds of the wicked and the wretched had been reflected
in it, and had passed from its surface and this earth's together.
Haunted in a most ghastly manner that abominable place would have
been, if the glass could ever have rendered back its reflections, as
the ocean is one day to give up its dead.  Some passing thought of
the infamy and disgrace for which it had been reserved, may have
struck the prisoner's mind.  Be that as it may, a change in his
position making him conscious of a bar of light across his face, he
looked up; and when he saw the glass his face flushed, and his right
hand pushed the herbs away.

It happened, that the action turned his face to that side of the
court which was on his left.  About on a level with his eyes, there
sat, in that corner of the Judge's bench, two persons upon whom his look
immediately rested; so immediately, and so much to the changing of his aspect,
that all the eyes that were tamed upon him, turned to them.

The spectators saw in the two figures, a young lady of little more
than twenty, and a gentleman who was evidently her father; a man of
a very remarkable appearance in respect of the absolute whiteness
of his hair, and a certain indescribable intensity of face:  not of
an active kind, but pondering and self-communing.  When this expression
was upon him, he looked as if he were old; but when it was stirred
and broken up---as it was now, in a moment, on his speaking to his
daughter---he became a handsome man, not past the prime of life.

His daughter had one of her hands drawn through his arm, as she sat
by him, and the other pressed upon it.  She had drawn close to him,
in her dread of the scene, and in her pity for the prisoner.  Her
forehead had been strikingly expressive of an engrossing terror and
compassion that saw nothing but the peril of the accused.  This had
been so very noticeable, so very powerfully and naturally shown, that
starers who had had no pity for him were touched by her; and the
whisper went about, ``Who are they?''

Jerry, the messenger, who had made his own observations, in his own
manner, and who had been sucking the rust off his fingers in his
absorption, stretched his neck to hear who they were.  The crowd
about him had pressed and passed the inquiry on to the nearest
attendant, and from him it had been more slowly pressed and passed
back; at last it got to Jerry:

``Witnesses.''

``For which side?''

``Against.''

``Against what side?''

``The prisoner's.''

The Judge, whose eyes had gone in the general direction, recalled
them, leaned back in his seat, and looked steadily at the man whose
life was in his hand, as Mr.\ Attorney-General rose to spin the rope,
grind the axe, and hammer the nails into the scaffold.



\chapter{A Disappointment}


Mr.\ Attorney-General had to inform the jury, that the prisoner before
them, though young in years, was old in the treasonable practices
which claimed the forfeit of his life.  That this correspondence with
the public enemy was not a correspondence of to-day, or of yesterday,
or even of last year, or of the year before.  That, it was certain
the prisoner had, for longer than that, been in the habit of passing
and repassing between France and England, on secret business of which
he could give no honest account.  That, if it were in the nature of
traitorous ways to thrive (which happily it never was), the real
wickedness and guilt of his business might have remained undiscovered.
That Providence, however, had put it into the heart of a person who
was beyond fear and beyond reproach, to ferret out the nature of the
prisoner's schemes, and, struck with horror, to disclose them to his
Majesty's Chief Secretary of State and most honourable Privy Council.
That, this patriot would be produced before them.  That, his position
and attitude were, on the whole, sublime.  That, he had been the
prisoner's friend, but, at once in an auspicious and an evil hour
detecting his infamy, had resolved to immolate the traitor he could
no longer cherish in his bosom, on the sacred altar of his country.
That, if statues were decreed in Britain, as in ancient Greece and
Rome, to public benefactors, this shining citizen would assuredly
have had one.  That, as they were not so decreed, he probably would
not have one.  That, Virtue, as had been observed by the poets (in
many passages which he well knew the jury would have, word for word,
at the tips of their tongues; whereat the jury's countenances
displayed a guilty consciousness that they knew nothing about the
passages), was in a manner contagious; more especially the bright
virtue known as patriotism, or love of country.  That, the lofty
example of this immaculate and unimpeachable witness for the Crown,
to refer to whom however unworthily was an honour, had communicated
itself to the prisoner's servant, and had engendered in him a holy
determination to examine his master's table-drawers and pockets, and
secrete his papers.  That, he (Mr.\ Attorney-General) was prepared to
hear some disparagement attempted of this admirable servant; but that,
in a general way, he preferred him to his (Mr.\ Attorney-General's)
brothers and sisters, and honoured him more than his
(Mr.\ Attorney-General's) father and mother.  That, he called with
confidence on the jury to come and do likewise.  That, the evidence
of these two witnesses, coupled with the documents of their
discovering that would be produced, would show the prisoner to have
been furnished with lists of his Majesty's forces, and of their
disposition and preparation, both by sea and land, and would leave no
doubt that he had habitually conveyed such information to a hostile
power.  That, these lists could not be proved to be in the prisoner's
handwriting; but that it was all the same; that, indeed, it was
rather the better for the prosecution, as showing the prisoner to be
artful in his precautions.  That, the proof would go back five years,
and would show the prisoner already engaged in these pernicious
missions, within a few weeks before the date of the very first action
fought between the British troops and the Americans.  That, for these
reasons, the jury, being a loyal jury (as he knew they were), and
being a responsible jury (as \emph{they} knew they were), must positively
find the prisoner Guilty, and make an end of him, whether they liked
it or not.  That, they never could lay their heads upon their pillows;
that, they never could tolerate the idea of their wives laying their
heads upon their pillows; that, they never could endure the notion of
their children laying their heads upon their pillows; in short, that
there never more could be, for them or theirs, any laying of heads
upon pillows at all, unless the prisoner's head was taken off.  That
head Mr.\ Attorney-General concluded by demanding of them, in the name
of everything he could think of with a round turn in it, and on the
faith of his solemn asseveration that he already considered the
prisoner as good as dead and gone.

When the Attorney-General ceased, a buzz arose in the court as if
a cloud of great blue-flies were swarming about the prisoner, in
anticipation of what he was soon to become.  When toned down again,
the unimpeachable patriot appeared in the witness-box.

Mr.\ Solicitor-General then, following his leader's lead, examined
the patriot:  John Barsad, gentleman, by name.  The story of his pure
soul was exactly what Mr.\ Attorney-General had described it to be---%
perhaps, if it had a fault, a little too exactly.  Having released
his noble bosom of its burden, he would have modestly withdrawn
himself, but that the wigged gentleman with the papers before him,
sitting not far from Mr.\ Lorry, begged to ask him a few questions.
The wigged gentleman sitting opposite, still looking at the ceiling
of the court.

Had he ever been a spy himself?  No, he scorned the base insinuation.
What did he live upon?  His property.  Where was his property?
He didn't precisely remember where it was.  What was it?  No business
of anybody's.  Had he inherited it?  Yes, he had.  From whom?  Distant
relation.  Very distant?  Rather.  Ever been in prison?  Certainly not.
Never in a debtors' prison?  Didn't see what that had to do with it.
Never in a debtors' prison?---Come, once again.  Never?  Yes.  How many
times?  Two or three times.  Not five or six?  Perhaps.  Of what profession?
Gentleman.  Ever been kicked?  Might have been. Frequently?  No.
Ever kicked downstairs?  Decidedly not; once received a kick on the
top of a staircase, and fell downstairs of his own accord.  Kicked on
that occasion for cheating at dice?  Something to that effect was said
by the intoxicated liar who committed the assault, but it was not
true.  Swear it was not true?  Positively. Ever live by cheating at
play?  Never.  Ever live by play?  Not more than other gentlemen do.
Ever borrow money of the prisoner?  Yes.  Ever pay him?  No. Was not
this intimacy with the prisoner, in reality a very slight one, forced
upon the prisoner in coaches, inns, and packets?  No.  Sure he saw
the prisoner with these lists?  Certain.  Knew no more about the
lists?  No. Had not procured them himself, for instance?  No. Expect
to get anything by this evidence?  No. Not in regular government pay
and employment, to lay traps?  Oh dear no.  Or to do anything?  Oh dear no.
Swear that?  Over and over again.  No motives but motives of sheer patriotism?
None whatever.

The virtuous servant, Roger Cly, swore his way through the case at a
great rate.  He had taken service with the prisoner, in good faith
and simplicity, four years ago.  He had asked the prisoner, aboard
the Calais packet, if he wanted a handy fellow, and the prisoner had
engaged him.  He had not asked the prisoner to take the handy fellow
as an act of charity---never thought of such a thing.  He began to
have suspicions of the prisoner, and to keep an eye upon him, soon
afterwards.  In arranging his clothes, while travelling, he had seen
similar lists to these in the prisoner's pockets, over and over again.
He had taken these lists from the drawer of the prisoner's desk.
He had not put them there first.  He had seen the prisoner show these
identical lists to French gentlemen at Calais, and similar lists to
French gentlemen, both at Calais and Boulogne.  He loved his country,
and couldn't bear it, and had given information.  He had never been
suspected of stealing a silver tea-pot; he had been maligned respecting
a mustard-pot, but it turned out to be only a plated one.  He had
known the last witness seven or eight years; that was merely a
coincidence.  He didn't call it a particularly curious coincidence;
most coincidences were curious.  Neither did he call it a curious
coincidence that true patriotism was \emph{his} only motive too.  He was a
true Briton, and hoped there were many like him.

The blue-flies buzzed again, and Mr.\ Attorney-General called Mr.\ Jarvis Lorry.

``Mr.\ Jarvis Lorry, are you a clerk in Tellson's bank?''

``I am.''

``On a certain Friday night in November one thousand seven hundred and
seventy-five, did business occasion you to travel between London and
Dover by the mail?''

``It did.''

``Were there any other passengers in the mail?''

``Two.''

``Did they alight on the road in the course of the night?''

``They did.''

``Mr.\ Lorry, look upon the prisoner.  Was he one of those two passengers?''

``I cannot undertake to say that he was.''

``Does he resemble either of these two passengers?''

``Both were so wrapped up, and the night was so dark, and we were all
so reserved, that I cannot undertake to say even that.''

``Mr.\ Lorry, look again upon the prisoner.  Supposing him wrapped up
as those two passengers were, is there anything in his bulk and
stature to render it unlikely that he was one of them?''

``No.''

``You will not swear, Mr.\ Lorry, that he was not one of them?''

``No.''

``So at least you say he may have been one of them?''

``Yes.  Except that I remember them both to have been---like myself---%
timorous of highwaymen, and the prisoner has not a timorous air.''

``Did you ever see a counterfeit of timidity, Mr.\ Lorry?''

``I certainly have seen that.''

``Mr.\ Lorry, look once more upon the prisoner.  Have you seen him,
to your certain knowledge, before?''

``I have.''

``When?''

``I was returning from France a few days afterwards, and, at Calais,
the prisoner came on board the packet-ship in which I returned, and
made the voyage with me.''

``At what hour did he come on board?''

``At a little after midnight.''

``In the dead of the night.  Was he the only passenger who came on
board at that untimely hour?''

``He happened to be the only one.''

``Never mind about `happening,' Mr.\ Lorry.  He was the only passenger
who came on board in the dead of the night?''

``He was.''

``Were you travelling alone, Mr.\ Lorry, or with any companion?''

``With two companions.  A gentleman and lady.  They are here.''

``They are here.  Had you any conversation with the prisoner?''

``Hardly any.  The weather was stormy, and the passage long and rough,
and I lay on a sofa, almost from shore to shore.''

``Miss Manette!''

The young lady, to whom all eyes had been turned before, and were now
turned again, stood up where she had sat.  Her father rose with her,
and kept her hand drawn through his arm.

``Miss Manette, look upon the prisoner.''

To be confronted with such pity, and such earnest youth and beauty,
was far more trying to the accused than to be confronted with all the
crowd.  Standing, as it were, apart with her on the edge of his grave,
not all the staring curiosity that looked on, could, for the moment,
nerve him to remain quite still.  His hurried right hand parcelled
out the herbs before him into imaginary beds of flowers in a garden;
and his efforts to control and steady his breathing shook the lips
from which the colour rushed to his heart.  The buzz of the great
flies was loud again.

``Miss Manette, have you seen the prisoner before?''

``Yes, sir.''

``Where?''

``On board of the packet-ship just now referred to, sir, and on the
same occasion.''

``You are the young lady just now referred to?''

``O! most unhappily, I am!''

The plaintive tone of her compassion merged into the less musical
voice of the Judge, as he said something fiercely:
``Answer the questions put to you, and make no remark upon them.''

``Miss Manette, had you any conversation with the prisoner on that
passage across the Channel?''

``Yes, sir.''

``Recall it.''

In the midst of a profound stillness, she faintly began:  ``When the
gentleman came on board---''

``Do you mean the prisoner?'' inquired the Judge, knitting his brows.

``Yes, my Lord.''

``Then say the prisoner.''

``When the prisoner came on board, he noticed that my father,'' turning
her eyes lovingly to him as he stood beside her, ``was much fatigued
and in a very weak state of health.  My father was so reduced that I
was afraid to take him out of the air, and I had made a bed for him
on the deck near the cabin steps, and I sat on the deck at his side
to take care of him.  There were no other passengers that night, but
we four.  The prisoner was so good as to beg permission to advise me
how I could shelter my father from the wind and weather, better than
I had done.  I had not known how to do it well, not understanding how
the wind would set when we were out of the harbour.  He did it for me.
He expressed great gentleness and kindness for my father's state, and
I am sure he felt it.  That was the manner of our beginning to speak
together.''

``Let me interrupt you for a moment.  Had he come on board alone?''

``No.''

``How many were with him?''

``Two French gentlemen.''

``Had they conferred together?''

``They had conferred together until the last moment, when it was
necessary for the French gentlemen to be landed in their boat.''

``Had any papers been handed about among them, similar to these lists?''

``Some papers had been handed about among them, but I don't know what
papers.''

``Like these in shape and size?''

``Possibly, but indeed I don't know, although they stood whispering
very near to me:  because they stood at the top of the cabin steps to
have the light of the lamp that was hanging there; it was a dull lamp,
and they spoke very low, and I did not hear what they said, and saw
only that they looked at papers.''

``Now, to the prisoner's conversation, Miss Manette.''

``The prisoner was as open in his confidence with me---which arose out
of my helpless situation---as he was kind, and good, and useful to my
father.  I hope,'' bursting into tears, ``I may not repay him by doing
him harm to-day.''

Buzzing from the blue-flies.

``Miss Manette, if the prisoner does not perfectly understand that you
give the evidence which it is your duty to give---which you must give---%
and which you cannot escape from giving---with great unwillingness,
he is the only person present in that condition.  Please to go on.''

``He told me that he was travelling on business of a delicate and
difficult nature, which might get people into trouble, and that he
was therefore travelling under an assumed name.  He said that this
business had, within a few days, taken him to France, and might,
at intervals, take him backwards and forwards between France and
England for a long time to come.''

``Did he say anything about America, Miss Manette?  Be particular.''

``He tried to explain to me how that quarrel had arisen, and he said that,
so far as he could judge, it was a wrong and foolish one on England's
part.  He added, in a jesting way, that perhaps George Washington
might gain almost as great a name in history as George the Third.
But there was no harm in his way of saying this:  it was said laughingly,
and to beguile the time.''

Any strongly marked expression of face on the part of a chief actor
in a scene of great interest to whom many eyes are directed, will be
unconsciously imitated by the spectators.  Her forehead was painfully
anxious and intent as she gave this evidence, and, in the pauses when
she stopped for the Judge to write it down, watched its effect upon
the counsel for and against.  Among the lookers-on there was the same
expression in all quarters of the court; insomuch, that a great
majority of the foreheads there, might have been mirrors reflecting
the witness, when the Judge looked up from his notes to glare at that
tremendous heresy about George Washington.

Mr.\ Attorney-General now signified to my Lord, that he deemed it
necessary, as a matter of precaution and form, to call the young
lady's father, Doctor Manette.  Who was called accordingly.

``Doctor Manette, look upon the prisoner.  Have you ever seen him before?''

``Once.  When he caged at my lodgings in London.  Some three years, or
three years and a half ago.''

``Can you identify him as your fellow-passenger on board the packet,
or speak to his conversation with your daughter?''

``Sir, I can do neither.''

``Is there any particular and special reason for your being unable to
do either?''

He answered, in a low voice, ``There is.''

``Has it been your misfortune to undergo a long imprisonment, without
trial, or even accusation, in your native country, Doctor Manette?''

He answered, in a tone that went to every heart, ``A long imprisonment.''

``Were you newly released on the occasion in question?''

``They tell me so.''

``Have you no remembrance of the occasion?''

``None.  My mind is a blank, from some time---I cannot even say what time---%
when I employed myself, in my captivity, in making shoes,
to the time when I found myself living in London with my dear
daughter here.  She had become familiar to me, when a gracious God
restored my faculties; but, I am quite unable even to say how she
had become familiar.  I have no remembrance of the process.''

Mr.\ Attorney-General sat down, and the father and daughter sat down together.

A singular circumstance then arose in the case.  The object in hand
being to show that the prisoner went down, with some fellow-plotter
untracked, in the Dover mail on that Friday night in November five
years ago, and got out of the mail in the night, as a blind, at a
place where he did not remain, but from which he travelled back some
dozen miles or more, to a garrison and dockyard, and there collected
information; a witness was called to identify him as having been at
the precise time required, in the coffee-room of an hotel in that
garrison-and-dockyard town, waiting for another person.  The prisoner's
counsel was cross-examining this witness with no result, except that
he had never seen the prisoner on any other occasion, when the wigged
gentleman who had all this time been looking at the ceiling of the
court, wrote a word or two on a little piece of paper, screwed it up,
and tossed it to him.  Opening this piece of paper in the next pause,
the counsel looked with great attention and curiosity at the prisoner.

``You say again you are quite sure that it was the prisoner?''

The witness was quite sure.

``Did you ever see anybody very like the prisoner?''

Not so like (the witness said) as that he could be mistaken.

``Look well upon that gentleman, my learned friend there,'' pointing to
him who had tossed the paper over, ``and then look well upon the prisoner.
How say you?  Are they very like each other?''

Allowing for my learned friend's appearance being careless and
slovenly if not debauched, they were sufficiently like each other to
surprise, not only the witness, but everybody present, when they were
thus brought into comparison.  My Lord being prayed to bid my learned
friend lay aside his wig, and giving no very gracious consent, the
likeness became much more remarkable.  My Lord inquired of Mr.\ Stryver
(the prisoner's counsel), whether they were next to try Mr.\ Carton
(name of my learned friend) for treason?  But, Mr.\ Stryver replied to
my Lord, no; but he would ask the witness to tell him whether what
happened once, might happen twice; whether he would have been so
confident if he had seen this illustration of his rashness sooner,
whether he would be so confident, having seen it; and more.
The upshot of which, was, to smash this witness like a crockery vessel,
and shiver his part of the case to useless lumber.

Mr.\ Cruncher had by this time taken quite a lunch of rust off his
fingers in his following of the evidence.  He had now to attend while
Mr.\ Stryver fitted the prisoner's case on the jury, like a compact
suit of clothes; showing them how the patriot, Barsad, was a hired spy
and traitor, an unblushing trafficker in blood, and one of the greatest
scoundrels upon earth since accursed Judas---which he certainly did
look rather like.  How the virtuous servant, Cly, was his friend and
partner, and was worthy to be; how the watchful eyes of those forgers
and false swearers had rested on the prisoner as a victim, because
some family affairs in France, he being of French extraction, did
require his making those passages across the Channel---though what
those affairs were, a consideration for others who were near and dear
to him, forbade him, even for his life, to disclose.  How the evidence
that had been warped and wrested from the young lady, whose anguish in
giving it they had witnessed, came to nothing, involving the mere
little innocent gallantries and politenesses likely to pass between
any young gentleman and young lady so thrown together;---with the
exception of that reference to George Washington, which was altogether
too extravagant and impossible to be regarded in any other light than
as a monstrous joke.  How it would be a weakness in the government to
break down in this attempt to practise for popularity on the lowest
national antipathies and fears, and therefore Mr.\ Attorney-General had
made the most of it; how, nevertheless, it rested upon nothing, save
that vile and infamous character of evidence too often disfiguring
such cases, and of which the State Trials of this country were full.
But, there my Lord interposed (with as grave a face as if it had not
been true), saying that he could not sit upon that Bench and suffer
those allusions.

Mr.\ Stryver then called his few witnesses, and Mr.\ Cruncher had next
to attend while Mr.\ Attorney-General turned the whole suit of clothes
Mr.\ Stryver had fitted on the jury, inside out; showing how Barsad and
Cly were even a hundred times better than he had thought them, and the
prisoner a hundred times worse.  Lastly, came my Lord himself, turning
the suit of clothes, now inside out, now outside in, but on the whole
decidedly trimming and shaping them into grave-clothes for the
prisoner.

And now, the jury turned to consider, and the great flies swarmed again.

Mr.\ Carton, who had so long sat looking at the ceiling of the court,
changed neither his place nor his attitude, even in this excitement.
While his teamed friend, Mr.\ Stryver, massing his papers before him,
whispered with those who sat near, and from time to time glanced
anxiously at the jury; while all the spectators moved more or less,
and grouped themselves anew; while even my Lord himself arose from his
seat, and slowly paced up and down his platform, not unattended by a
suspicion in the minds of the audience that his state was feverish;
this one man sat leaning back, with his torn gown half off him, his
untidy wig put on just as it had happened to fight on his head after
its removal, his hands in his pockets, and his eyes on the ceiling as
they had been all day.  Something especially reckless in his demeanour,
not only gave him a disreputable look, but so diminished the strong
resemblance he undoubtedly bore to the prisoner (which his momentary
earnestness, when they were compared together, had strengthened),
that many of the lookers-on, taking note of him now, said to one
another they would hardly have thought the two were so alike.
Mr.\ Cruncher made the observation to his next neighbour, and added,
``I'd hold half a guinea that \emph{he} don't get no law-work to do.
Don't look like the sort of one to get any, do he?''

Yet, this Mr.\ Carton took in more of the details of the scene than he
appeared to take in; for now, when Miss Manette's head dropped upon
her father's breast, he was the first to see it, and to say audibly:
``Officer! look to that young lady.  Help the gentleman to take her out.
Don't you see she will fall!''

There was much commiseration for her as she was removed, and much
sympathy with her father.  It had evidently been a great distress to
him, to have the days of his imprisonment recalled.  He had shown
strong internal agitation when he was questioned, and that pondering
or brooding look which made him old, had been upon him, like a heavy
cloud, ever since.  As he passed out, the jury, who had turned back
and paused a moment, spoke, through their foreman.

They were not agreed, and wished to retire.  My Lord (perhaps with
George Washington on his mind) showed some surprise that they were not
agreed, but signified his pleasure that they should retire under watch
and ward, and retired himself.  The trial had lasted all day, and the
lamps in the court were now being lighted.  It began to be rumoured
that the jury would be out a long while.  The spectators dropped off
to get refreshment, and the prisoner withdrew to the back of the dock,
and sat down.

Mr.\ Lorry, who had gone out when the young lady and her father went out,
now reappeared, and beckoned to Jerry:  who, in the slackened interest,
could easily get near him.

``Jerry, if you wish to take something to eat, you can.  But, keep in
the way.  You will be sure to hear when the jury come in.  Don't be a
moment behind them, for I want you to take the verdict back to the bank.
You are the quickest messenger I know, and will get to Temple Bar long
before I can.''

Jerry had just enough forehead to knuckle, and he knuckled it in
acknowedgment of this communication and a shilling.  Mr.\ Carton came
up at the moment, and touched Mr.\ Lorry on the arm.

``How is the young lady?''

``She is greatly distressed; but her father is comforting her, and she
feels the better for being out of court.''

``I'll tell the prisoner so.  It won't do for a respectable bank
gentleman like you, to be seen speaking to him publicly, you know.''

Mr.\ Lorry reddened as if he were conscious of having debated the point
in his mind, and Mr.\ Carton made his way to the outside of the bar.
The way out of court lay in that direction, and Jerry followed him,
all eyes, ears, and spikes.

``Mr.\ Darnay!''

The prisoner came forward directly.

``You will naturally be anxious to hear of the witness, Miss Manette.
She will do very well.  You have seen the worst of her agitation.''

``I am deeply sorry to have been the cause of it.  Could you tell her
so for me, with my fervent acknowledgments?''

``Yes, I could.  I will, if you ask it.''

Mr.\ Carton's manner was so careless as to be almost insolent.  He stood,
half turned from the prisoner, lounging with his elbow against the bar.

``I do ask it.  Accept my cordial thanks.''

``What,'' said Carton, still only half turned towards him, ``do you
expect, Mr.\ Darnay?''

``The worst.''

``It's the wisest thing to expect, and the likeliest.  But I think
their withdrawing is in your favour.''

Loitering on the way out of court not being allowed, Jerry heard no
more:  but left them---so like each other in feature, so unlike each
other in manner---standing side by side, both reflected in the glass
above them.

An hour and a half limped heavily away in the thief-and-rascal crowded
passages below, even though assisted off with mutton pies and ale.
The hoarse messenger, uncomfortably seated on a form after taking that
refection, had dropped into a doze, when a loud murmur and a rapid
tide of people setting up the stairs that led to the court, carried
him along with them.

``Jerry!  Jerry!''  Mr.\ Lorry was already calling at the door when
he got there.

``Here, sir!  It's a fight to get back again.  Here I am, sir!''

Mr.\ Lorry handed him a paper through the throng.
``Quick!  Have you got it?''

``Yes, sir.''

Hastily written on the paper was the word ``\emph{aquitted}.''

``If you had sent the message, `Recalled to Life,' again,'' muttered
Jerry, as he turned, ``I should have known what you meant, this time.''

He had no opportunity of saying, or so much as thinking, anything
else, until he was clear of the Old Bailey; for, the crowd came
pouring out with a vehemence that nearly took him off his legs, and a
loud buzz swept into the street as if the baffled blue-flies were
dispersing in search of other carrion.



\chapter{Congratulatory}


From the dimly-lighted passages of the court, the last sediment of the
human stew that had been boiling there all day, was straining off,
when Doctor Manette, Lucie Manette, his daughter, Mr.\ Lorry, the
solicitor for the defence, and its counsel, Mr.\ Stryver, stood
gathered round Mr.\ Charles Darnay---just released---congratulating him
on his escape from death.

It would have been difficult by a far brighter light, to recognise in
Doctor Manette, intellectual of face and upright of bearing, the
shoemaker of the garret in Paris.  Yet, no one could have looked at
him twice, without looking again:  even though the opportunity of
observation had not extended to the mournful cadence of his low grave
voice, and to the abstraction that overclouded him fitfully, without
any apparent reason.  While one external cause, and that a reference
to his long lingering agony, would always---as on the trial---evoke this
condition from the depths of his soul, it was also in its nature to
arise of itself, and to draw a gloom over him, as incomprehensible to
those unacquainted with his story as if they had seen the shadow of
the actual Bastille thrown upon him by a summer sun, when the
substance was three hundred miles away.

Only his daughter had the power of charming this black brooding from
his mind.  She was the golden thread that united him to a Past beyond
his misery, and to a Present beyond his misery:  and the sound of her
voice, the light of her face, the touch of her hand, had a strong
beneficial influence with him almost always.  Not absolutely always,
for she could recall some occasions on which her power had failed;
but they were few and slight, and she believed them over.

Mr.\ Darnay had kissed her hand fervently and gratefully, and had
turned to Mr.\ Stryver, whom he warmly thanked.  Mr.\ Stryver, a man of
little more than thirty, but looking twenty years older than he was,
stout, loud, red, bluff, and free from any drawback of delicacy,
had a pushing way of shouldering himself (morally and physically)
into companies and conversations, that argued well for his shouldering
his way up in life.

He still had his wig and gown on, and he said, squaring himself at his
late client to that degree that he squeezed the innocent Mr.\ Lorry
clean out of the group:  ``I am glad to have brought you off with honour,
Mr.\ Darnay.  It was an infamous prosecution, grossly infamous;
but not the less likely to succeed on that account.''

``You have laid me under an obligation to you for life---in two senses,''
said his late client, taking his hand.

``I have done my best for you, Mr.\ Darnay; and my best is as good as
another man's, I believe.''

It clearly being incumbent on some one to say, ``Much better,'' Mr.\ Lorry
said it; perhaps not quite disinterestedly, but with the interested
object of squeezing himself back again.

``You think so?'' said Mr.\ Stryver.  ``Well! you have been present all day,
and you ought to know.  You are a man of business, too.''

``And as such,'' quoth Mr.\ Lorry, whom the counsel learned in the law
had now shouldered back into the group, just as he had previously
shouldered him out of it---``as such I will appeal to Doctor Manette,
to break up this conference and order us all to our homes.
Miss Lucie looks ill, Mr.\ Darnay has had a terrible day, we are worn out.''

``Speak for yourself, Mr.\ Lorry,'' said Stryver; ``I have a night's work
to do yet.  Speak for yourself.''

``I speak for myself,'' answered Mr.\ Lorry, ``and for Mr.\ Darnay, and for
Miss Lucie, and---Miss Lucie, do you not think I may speak for us all?''
He asked her the question pointedly, and with a glance at her father.

His face had become frozen, as it were, in a very curious look at
Darnay:  an intent look, deepening into a frown of dislike and distrust,
not even unmixed with fear.  With this strange expression on him his
thoughts had wandered away.

``My father,'' said Lucie, softly laying her hand on his.

He slowly shook the shadow off, and turned to her.

``Shall we go home, my father?''

With a long breath, he answered ``Yes.''

The friends of the acquitted prisoner had dispersed, under the
impression---which he himself had originated---that he would not be
released that night.  The lights were nearly all extinguished in the
passages, the iron gates were being closed with a jar and a rattle,
and the dismal place was deserted until to-morrow morning's interest
of gallows, pillory, whipping-post, and branding-iron, should repeople
it.  Walking between her father and Mr.\ Darnay, Lucie Manette passed
into the open air.  A hackney-coach was called, and the father and
daughter departed in it.

Mr.\ Stryver had left them in the passages, to shoulder his way back
to the robing-room.  Another person, who had not joined the group,
or interchanged a word with any one of them, but who had been leaning
against the wall where its shadow was darkest, had silently strolled
out after the rest, and had looked on until the coach drove away.
He now stepped up to where Mr.\ Lorry and Mr.\ Darnay stood upon the
pavement.

``So, Mr.\ Lorry!  Men of business may speak to Mr.\ Darnay now?''

Nobody had made any acknowledgment of Mr.\ Carton's part in the day's
proceedings; nobody had known of it.  He was unrobed, and was none
the better for it in appearance.

``If you knew what a conflict goes on in the business mind, when the
business mind is divided between good-natured impulse and business
appearances, you would be amused, Mr.\ Darnay.''

Mr.\ Lorry reddened, and said, warmly, ``You have mentioned that before,
sir.  We men of business, who serve a House, are not our own masters.
We have to think of the House more than ourselves.''

``\emph{I} know, \emph{I} know,'' rejoined Mr.\ Carton, carelessly.  ``Don't be
nettled, Mr.\ Lorry.  You are as good as another, I have no doubt:
better, I dare say.''

``And indeed, sir,'' pursued Mr.\ Lorry, not minding him, ``I really
don't know what you have to do with the matter.  If you'll excuse me,
as very much your elder, for saying so, I really don't know that it is
your business.''

``Business!  Bless you, \emph{I} have no business,'' said Mr.\ Carton.

``It is a pity you have not, sir.''

``I think so, too.''

``If you had,'' pursued Mr.\ Lorry, ``perhaps you would attend to it.''

``Lord love you, no!---I shouldn't,'' said Mr.\ Carton.

``Well, sir!'' cried Mr.\ Lorry, thoroughly heated by his indifference,
``business is a very good thing, and a very respectable thing.  And, sir,
if business imposes its restraints and its silences and impediments,
Mr.\ Darnay as a young gentleman of generosity knows how to make allowance
for that circumstance.  Mr.\ Darnay, good night, God bless you, sir!
I hope you have been this day preserved for a prosperous and happy
life.---Chair there!''

Perhaps a little angry with himself, as well as with the barrister,
Mr.\ Lorry bustled into the chair, and was carried off to Tellson's.
Carton, who smelt of port wine, and did not appear to be quite sober,
laughed then, and turned to Darnay:

``This is a strange chance that throws you and me together.  This must
be a strange night to you, standing alone here with your counterpart
on these street stones?''

``I hardly seem yet,'' returned Charles Darnay, ``to belong to this world
again.''

``I don't wonder at it; it's not so long since you were pretty far
advanced on your way to another.  You speak faintly.''

``I begin to think I \emph{am} faint.''

``Then why the devil don't you dine?  I dined, myself, while those
numskulls were deliberating which world you should belong to---this,
or some other.  Let me show you the nearest tavern to dine well at.''

Drawing his arm through his own, he took him down Ludgate-hill to
Fleet-street, and so, up a covered way, into a tavern.  Here, they
were shown into a little room, where Charles Darnay was soon recruiting
his strength with a good plain dinner and good wine:  while Carton sat
opposite to him at the same table, with his separate bottle of port
before him, and his fully half-insolent manner upon him.

``Do you feel, yet, that you belong to this terrestrial scheme again,
Mr.\ Darnay?''

``I am frightfully confused regarding time and place; but I am so far
mended as to feel that.''

``It must be an immense satisfaction!''

He said it bitterly, and filled up his glass again:  which was a large one.

``As to me, the greatest desire I have, is to forget that I belong to
it.  It has no good in it for me---except wine like this---nor I for it.
So we are not much alike in that particular.  Indeed, I begin to think
we are not much alike in any particular, you and I.''

Confused by the emotion of the day, and feeling his being there with
this Double of coarse deportment, to be like a dream, Charles Darnay
was at a loss how to answer; finally, answered not at all.

``Now your dinner is done,'' Carton presently said, ``why don't you call
a health, Mr.\ Darnay; why don't you give your toast?''

``What health?  What toast?''

``Why, it's on the tip of your tongue.  It ought to be, it must be,
I'll swear it's there.''

``Miss Manette, then!''

``Miss Manette, then!''

Looking his companion full in the face while he drank the toast,
Carton flung his glass over his shoulder against the wall, where it
shivered to pieces; then, rang the bell, and ordered in another.

``That's a fair young lady to hand to a coach in the dark, Mr.\ Darnay!''
he said, ruing his new goblet.

A slight frown and a laconic ``Yes,'' were the answer.

``That's a fair young lady to be pitied by and wept for by!  How does it
feel?  Is it worth being tried for one's life, to be the object of such
sympathy and compassion, Mr.\ Darnay?''

Again Darnay answered not a word.

``She was mightily pleased to have your message, when I gave it her.
Not that she showed she was pleased, but I suppose she was.''

The allusion served as a timely reminder to Darnay that this
disagreeable companion had, of his own free will, assisted him in the
strait of the day.  He turned the dialogue to that point, and thanked
him for it.

``I neither want any thanks, nor merit any,'' was the careless rejoinder.
``It was nothing to do, in the first place; and I don't know why I did it,
in the second.  Mr.\ Darnay, let me ask you a question.''

``Willingly, and a small return for your good offices.''

``Do you think I particularly like you?''

``Really, Mr.\ Carton,'' returned the other, oddly disconcerted, ``I have
not asked myself the question.''

``But ask yourself the question now.''

``You have acted as if you do; but I don't think you do.''

``\emph{I} don't think I do,'' said Carton.  ``I begin to have a very good
opinion of your understanding.''

``Nevertheless,'' pursued Darnay, rising to ring the bell, ``there is
nothing in that, I hope, to prevent my calling the reckoning, and our
parting without ill-blood on either side.''

Carton rejoining, ``Nothing in life!'' Darnay rang.  ``Do you call the
whole reckoning?'' said Carton.  On his answering in the affirmative,
``Then bring me another pint of this same wine, drawer, and come and
wake me at ten.''

The bill being paid, Charles Darnay rose and wished him good night.
Without returning the wish, Carton rose too, with something of a
threat of defiance in his manner, and said, ``A last word, Mr.\ Darnay:
you think I am drunk?''

``I think you have been drinking, Mr.\ Carton.''

``Think?  You know I have been drinking.''

``Since I must say so, I know it.''

``Then you shall likewise know why.  I am a disappointed drudge, sir.
I care for no man on earth, and no man on earth cares for me.''

``Much to be regretted.  You might have used your talents better.''

``May be so, Mr.\ Darnay; may be not.  Don't let your sober face elate you,
however; you don't know what it may come to.  Good night!''

When he was left alone, this strange being took up a candle, went to a
glass that hung against the wall, and surveyed himself minutely in it.

``Do you particularly like the man?'' he muttered, at his own image;
``why should you particularly like a man who resembles you?  There is
nothing in you to like; you know that.  Ah, confound you!  What a
change you have made in yourself!  A good reason for taking to a man,
that he shows you what you have fallen away from, and what you might
have been!  Change places with him, and would you have been looked at
by those blue eyes as he was, and commiserated by that agitated face
as he was?  Come on, and have it out in plain words!  You hate the fellow.''

He resorted to his pint of wine for consolation, drank it all in a
few minutes, and fell asleep on his arms, with his hair straggling
over the table, and a long winding-sheet in the candle dripping down
upon him.



\chapter{The Jackal}

Those were drinking days, and most men drank hard.  So very great is
the improvement Time has brought about in such habits, that a moderate
statement of the quantity of wine and punch which one man would swallow
in the course of a night, without any detriment to his reputation as a
perfect gentleman, would seem, in these days, a ridiculous exaggeration.
The learned profession of the law was certainly not behind any other
learned profession in its Bacchanalian propensities; neither was
Mr.\ Stryver, already fast shouldering his way to a large and lucrative
practice, behind his compeers in this particular, any more than in the
drier parts of the legal race.

A favourite at the Old Bailey, and eke at the Sessions, Mr.\ Stryver
had begun cautiously to hew away the lower staves of the ladder on
which he mounted.  Sessions and Old Bailey had now to summon their
favourite, specially, to their longing arms; and shouldering itself
towards the visage of the Lord Chief Justice in the Court of King's
Bench, the florid countenance of Mr.\ Stryver might be daily seen,
bursting out of the bed of wigs, like a great sunflower pushing its
way at the sun from among a rank garden-full of flaring companions.

It had once been noted at the Bar, that while Mr.\ Stryver was a glib
man, and an unscrupulous, and a ready, and a bold, he had not that
faculty of extracting the essence from a heap of statements, which is
among the most striking and necessary of the advocate's accomplishments.
But, a remarkable improvement came upon him as to this.  The more
business he got, the greater his power seemed to grow of getting at
its pith and marrow; and however late at night he sat carousing with
Sydney Carton, he always had his points at his fingers' ends in the morning.

Sydney Carton, idlest and most unpromising of men, was Stryver's great
ally.  What the two drank together, between Hilary Term and Michaelmas,
might have floated a king's ship.  Stryver never had a case in hand,
anywhere, but Carton was there, with his hands in his pockets, staring
at the ceiling of the court; they went the same Circuit, and even there
they prolonged their usual orgies late into the night, and Carton was
rumoured to be seen at broad day, going home stealthily and unsteadily
to his lodgings, like a dissipated cat.  At last, it began to get about,
among such as were interested in the matter, that although Sydney Carton
would never be a lion, he was an amazingly good jackal, and that he
rendered suit and service to Stryver in that humble capacity.

``Ten o'clock, sir,'' said the man at the tavern, whom he had charged to
wake him---``ten o'clock, sir.''

``\emph{What's} the matter?''

``Ten o'clock, sir.''

``What do you mean?  Ten o'clock at night?''

``Yes, sir.  Your honour told me to call you.''

``Oh!  I remember.  Very well, very well.''

After a few dull efforts to get to sleep again, which the man dexterously
combated by stirring the fire continuously for five minutes, he got up,
tossed his hat on, and walked out.  He turned into the Temple, and,
having revived himself by twice pacing the pavements of King's Bench-walk
and Paper-buildings, turned into the Stryver chambers.

The Stryver clerk, who never assisted at these conferences, had gone home,
and the Stryver principal opened the door.  He had his slippers on,
and a loose bed-gown, and his throat was bare for his greater ease.
He had that rather wild, strained, seared marking about the eyes,
which may be observed in all free livers of his class, from the portrait
of Jeffries downward, and which can be traced, under various disguises
of Art, through the portraits of every Drinking Age.

``You are a little late, Memory,'' said Stryver.

``About the usual time; it may be a quarter of an hour later.''

They went into a dingy room lined with books and littered with papers,
where there was a blazing fire.  A kettle steamed upon the hob, and in
the midst of the wreck of papers a table shone, with plenty of wine
upon it, and brandy, and rum, and sugar, and lemons.

``You have had your bottle, I perceive, Sydney.''

``Two to-night, I think.  I have been dining with the day's client;
or seeing him dine---it's all one!''

``That was a rare point, Sydney, that you brought to bear upon the
identification.  How did you come by it?  When did it strike you?''

``I thought he was rather a handsome fellow, and I thought I should
have been much the same sort of fellow, if I had had any luck.''

Mr.\ Stryver laughed till he shook his precocious paunch.

``You and your luck, Sydney!  Get to work, get to work.''

Sullenly enough, the jackal loosened his dress, went into an adjoining
room, and came back with a large jug of cold water, a basin, and a towel
or two.  Steeping the towels in the water, and partially wringing them
out, he folded them on his head in a manner hideous to behold, sat down
at the table, and said, ``Now I am ready!''

``Not much boiling down to be done to-night, Memory,'' said Mr.\ Stryver,
gaily, as he looked among his papers.

``How much?''

``Only two sets of them.''

``Give me the worst first.''

``There they are, Sydney.  Fire away!''

The lion then composed himself on his back on a sofa on one side of
the drinking-table, while the jackal sat at his own paper-bestrewn
table proper, on the other side of it, with the bottles and glasses
ready to his hand.  Both resorted to the drinking-table without
stint, but each in a different way; the lion for the most part
reclining with his hands in his waistband, looking at the fire, or
occasionally flirting with some lighter document; the jackal, with
knitted brows and intent face, so deep in his task, that his eyes did
not even follow the hand he stretched out for his glass---which often
groped about, for a minute or more, before it found the glass for his
lips.  Two or three times, the matter in hand became so knotty, that
the jackal found it imperative on him to get up, and steep his towels
anew.  From these pilgrimages to the jug and basin, he returned with
such eccentricities of damp headgear as no words can describe; which
were made the more ludicrous by his anxious gravity.

At length the jackal had got together a compact repast for the lion,
and proceeded to offer it to him.  The lion took it with care and
caution, made his selections from it, and his remarks upon it,
and the jackal assisted both.  When the repast was fully discussed,
the lion put his hands in his waistband again, and lay down to mediate.
The jackal then invigorated himself with a bum for his throttle,
and a fresh application to his head, and applied himself to the
collection of a second meal; this was administered to the lion in the
same manner, and was not disposed of until the clocks struck three in
the morning.

``And now we have done, Sydney, fill a bumper of punch,'' said Mr.\ Stryver.

The jackal removed the towels from his head, which had been steaming
again, shook himself, yawned, shivered, and complied.

``You were very sound, Sydney, in the matter of those crown witnesses
to-day.  Every question told.''

``I always am sound; am I not?''

``I don't gainsay it.  What has roughened your temper?
Put some punch to it and smooth it again.''

With a deprecatory grunt, the jackal again complied.

``The old Sydney Carton of old Shrewsbury School,'' said Stryver,
nodding his head over him as he reviewed him in the present and the
past, ``the old seesaw Sydney.  Up one minute and down the next; now
in spirits and now in despondency!''

``Ah!'' returned the other, sighing:  ``yes!  The same Sydney, with the
same luck.  Even then, I did exercises for other boys, and seldom did
my own.

``And why not?''

``God knows.  It was my way, I suppose.''

He sat, with his hands in his pockets and his legs stretched out
before him, looking at the fire.

``Carton,'' said his friend, squaring himself at him with a bullying
air, as if the fire-grate had been the furnace in which sustained
endeavour was forged, and the one delicate thing to be done for the
old Sydney Carton of old Shrewsbury School was to shoulder him into it,
``your way is, and always was, a lame way.  You summon no energy and
purpose.  Look at me.''

``Oh, botheration!'' returned Sydney, with a lighter and more good-%
humoured laugh, ``don't \emph{you} be moral!''

``How have I done what I have done?'' said Stryver; ``how do I do what I do?''

``Partly through paying me to help you, I suppose.  But it's not worth
your while to apostrophise me, or the air, about it; what you want to
do, you do.  You were always in the front rank, and I was always behind.''

``I had to get into the front rank; I was not born there, was I?''

``I was not present at the ceremony; but my opinion is you were,'' said
Carton.  At this, he laughed again, and they both laughed.

``Before Shrewsbury, and at Shrewsbury, and ever since Shrewsbury,''
pursued Carton, ``you have fallen into your rank, and I have fallen
into mine.  Even when we were fellow-students in the Student-Quarter
of Paris, picking up French, and French law, and other French crumbs
that we didn't get much good of, you were always somewhere, and I was
always nowhere.''

``And whose fault was that?''

``Upon my soul, I am not sure that it was not yours.  You were always
driving and riving and shouldering and passing, to that restless
degree that I had no chance for my life but in rust and repose.  It's
a gloomy thing, however, to talk about one's own past, with the day
breaking.  Turn me in some other direction before I go.''

``Well then!  Pledge me to the pretty witness,'' said Stryver, holding
up his glass.  ``Are you turned in a pleasant direction?''

Apparently not, for he became gloomy again.

``Pretty witness,'' he muttered, looking down into his glass.  ``I have
had enough of witnesses to-day and to-night; who's your pretty
witness?''

``The picturesque doctor's daughter, Miss Manette.''

``\emph{She} pretty?''

``Is she not?''

``No.''

``Why, man alive, she was the admiration of the whole Court!''

``Rot the admiration of the whole Court!  Who made the Old Bailey a
judge of beauty?  She was a golden-haired doll!''

``Do you know, Sydney,'' said Mr.\ Stryver, looking at him with sharp
eyes, and slowly drawing a hand across his florid face:  ``do you know,
I rather thought, at the time, that you sympathised with the
golden-haired doll, and were quick to see what happened to the
golden-haired doll?''

``Quick to see what happened!  If a girl, doll or no doll, swoons
within a yard or two of a man's nose, he can see it without a
perspective-glass.  I pledge you, but I deny the beauty.
And now I'll have no more drink; I'll get to bed.''

When his host followed him out on the staircase with a candle,
to light him down the stairs, the day was coldly looking in through
its grimy windows.  When he got out of the house, the air was cold
and sad, the dull sky overcast, the river dark and dim, the whole
scene like a lifeless desert.  And wreaths of dust were spinning
round and round before the morning blast, as if the desert-sand had
risen far away, and the first spray of it in its advance had begun to
overwhelm the city.

Waste forces within him, and a desert all around, this man stood
still on his way across a silent terrace, and saw for a moment,
lying in the wilderness before him, a mirage of honourable ambition,
self-denial, and perseverance.  In the fair city of this vision,
there were airy galleries from which the loves and graces looked upon
him, gardens in which the fruits of life hung ripening, waters of Hope
that sparkled in his sight.  A moment, and it was gone.  Climbing to
a high chamber in a well of houses, he threw himself down in his
clothes on a neglected bed, and its pillow was wet with wasted tears.

Sadly, sadly, the sun rose; it rose upon no sadder sight than the man
of good abilities and good emotions, incapable of their directed
exercise, incapable of his own help and his own happiness, sensible
of the blight on him, and resigning himself to let it eat him away.



\chapter{Hundreds of People}


The quiet lodgings of Doctor Manette were in a quiet street-corner
not far from Soho-square.  On the afternoon of a certain fine Sunday
when the waves of four months had roiled over the trial for treason,
and carried it, as to the public interest and memory, far out to sea,
Mr.\ Jarvis Lorry walked along the sunny streets from Clerkenwell
where he lived, on his way to dine with the Doctor.  After several
relapses into business-absorption, Mr.\ Lorry had become the Doctor's
friend, and the quiet street-corner was the sunny part of his life.

On this certain fine Sunday, Mr.\ Lorry walked towards Soho, early in
the afternoon, for three reasons of habit.  Firstly, because, on fine
Sundays, he often walked out, before dinner, with the Doctor and Lucie;
secondly, because, on unfavourable Sundays, he was accustomed to be
with them as the family friend, talking, reading, looking out of window,
and generally getting through the day; thirdly, because he happened
to have his own little shrewd doubts to solve, and knew how the ways
of the Doctor's household pointed to that time as a likely time for
solving them.

A quainter corner than the corner where the Doctor lived, was not to
be found in London.  There was no way through it, and the front windows
of the Doctor's lodgings commanded a pleasant little vista of street
that had a congenial air of retirement on it.  There were few buildings
then, north of the Oxford-road, and forest-trees flourished, and wild
flowers grew, and the hawthorn blossomed, in the now vanished fields.
As a consequence, country airs circulated in Soho with vigorous freedom,
instead of languishing into the parish like stray paupers without a
settlement; and there was many a good south wall, not far off, on which
the peaches ripened in their season.

The summer light struck into the corner brilliantly in the earlier
part of the day; but, when the streets grew hot, the corner was in
shadow, though not in shadow so remote but that you could see beyond
it into a glare of brightness.  It was a cool spot, staid but cheerful,
a wonderful place for echoes, and a very harbour from the raging streets.

There ought to have been a tranquil bark in such an anchorage, and
there was.  The Doctor occupied two floors of a large stiff house,
where several callings purported to be pursued by day, but whereof
little was audible any day, and which was shunned by all of them at
night.  In a building at the back, attainable by a courtyard where a
plane-tree rustled its green leaves, church-organs claimed to be
made, and silver to be chased, and likewise gold to be beaten by some
mysterious giant who had a golden arm starting out of the wall of the
front hall---as if he had beaten himself precious, and menaced a similar
conversion of all visitors.  Very little of these trades, or of a
lonely lodger rumoured to live up-stairs, or of a dim coach-trimming
maker asserted to have a counting-house below, was ever heard or seen.
Occasionally, a stray workman putting his coat on, traversed the
hall, or a stranger peered about there, or a distant clink was heard
across the courtyard, or a thump from the golden giant.  These,
however, were only the exceptions required to prove the rule that the
sparrows in the plane-tree behind the house, and the echoes in the
corner before it, had their own way from Sunday morning unto Saturday
night.

Doctor Manette received such patients here as his old reputation,
and its revival in the floating whispers of his story, brought him.
His scientific knowledge, and his vigilance and skill in conducting
ingenious experiments, brought him otherwise into moderate request,
and he earned as much as he wanted.

These things were within Mr.\ Jarvis Lorry's knowledge, thoughts, and
notice, when he rang the door-bell of the tranquil house in the corner,
on the fine Sunday afternoon.

``Doctor Manette at home?''

Expected home.

``Miss Lucie at home?''

Expected home.

``Miss Pross at home?''

Possibly at home, but of a certainty impossible for handmaid to anticipate
intentions of Miss Pross, as to admission or denial of the fact.

``As I am at home myself,'' said Mr.\ Lorry, ``I'll go upstairs.''

Although the Doctor's daughter had known nothing of the country of
her birth, she appeared to have innately derived from it that ability
to make much of little means, which is one of its most useful and
most agreeable characteristics.  Simple as the furniture was, it was
set off by so many little adornments, of no value but for their taste
and fancy, that its effect was delightful.  The disposition of
everything in the rooms, from the largest object to the least; the
arrangement of colours, the elegant variety and contrast obtained by
thrift in trifles, by delicate hands, clear eyes, and good sense;
were at once so pleasant in themselves, and so expressive of their
originator, that, as Mr.\ Lorry stood looking about him, the very
chairs and tables seemed to ask him, with something of that peculiar
expression which he knew so well by this time, whether he approved?

There were three rooms on a floor, and, the doors by which they
communicated being put open that the air might pass freely through
them all, Mr.\ Lorry, smilingly observant of that fanciful resemblance
which he detected all around him, walked from one to another.
The first was the best room, and in it were Lucie's birds, and flowers,
and books, and desk, and work-table, and box of water-colours;
the second was the Doctor's consulting-room, used also as the
dining-room; the third, changingly speckled by the rustle of the
plane-tree in the yard, was the Doctor's bedroom, and there, in a
corner, stood the disused shoemaker's bench and tray of tools,
much as it had stood on the fifth floor of the dismal house by the
wine-shop, in the suburb of Saint Antoine in Paris.

``I wonder,'' said Mr.\ Lorry, pausing in his looking about, ``that he
keeps that reminder of his sufferings about him!''

``And why wonder at that?'' was the abrupt inquiry that made him start.

It proceeded from Miss Pross, the wild red woman, strong of hand,
whose acquaintance he had first made at the Royal George Hotel at Dover,
and had since improved.

``I should have thought---'' Mr.\ Lorry began.

``Pooh!  You'd have thought!'' said Miss Pross; and Mr.\ Lorry left off.

``How do you do?'' inquired that lady then---sharply, and yet as if to
express that she bore him no malice.

``I am pretty well, I thank you,'' answered Mr.\ Lorry, with meekness;
``how are you?''

``Nothing to boast of,'' said Miss Pross.

``Indeed?''

``Ah! indeed!'' said Miss Pross.  ``I am very much put out about my Ladybird.''

``Indeed?''

``For gracious sake say something else besides `indeed,' or you'll
fidget me to death,'' said Miss Pross:  whose character (dissociated
from stature) was shortness.

``Really, then?'' said Mr.\ Lorry, as an amendment.

``Really, is bad enough,'' returned Miss Pross, ``but better.  Yes, I am
very much put out.''

``May I ask the cause?''

``I don't want dozens of people who are not at all worthy of Ladybird,
to come here looking after her,'' said Miss Pross.

``\emph{Do} dozens come for that purpose?''

``Hundreds,'' said Miss Pross.

It was characteristic of this lady (as of some other people before her
time and since) that whenever her original proposition was questioned,
she exaggerated it.

``Dear me!'' said Mr.\ Lorry, as the safest remark he could think of.

``I have lived with the darling---or the darling has lived with me,
and paid me for it; which she certainly should never have done,
you may take your affidavit, if I could have afforded to keep either
myself or her for nothing---since she was ten years old.  And it's
really very hard,'' said Miss Pross.

Not seeing with precision what was very hard, Mr.\ Lorry shook his head;
using that important part of himself as a sort of fairy cloak that
would fit anything.

``All sorts of people who are not in the least degree worthy of the pet,
are always turning up,'' said Miss Pross.  ``When you began it---''

``\emph{I} began it, Miss Pross?''

``Didn't you?  Who brought her father to life?''

``Oh!  If \emph{that} was beginning it---'' said Mr.\ Lorry.

``It wasn't ending it, I suppose?  I say, when you began it, it was hard
enough; not that I have any fault to find with Doctor Manette, except
that he is not worthy of such a daughter, which is no imputation on
him, for it was not to be expected that anybody should be, under any
circumstances.  But it really is doubly and trebly hard to have crowds
and multitudes of people turning up after him (I could have forgiven him),
to take Ladybird's affections away from me.''

Mr.\ Lorry knew Miss Pross to be very jealous, but he also knew her by
this time to be, beneath the service of her eccentricity, one of those
unselfish creatures---found only among women---who will, for pure love
and admiration, bind themselves willing slaves, to youth when they
have lost it, to beauty that they never had, to accomplishments that
they were never fortunate enough to gain, to bright hopes that never
shone upon their own sombre lives.  He knew enough of the world to
know that there is nothing in it better than the faithful service of
the heart; so rendered and so free from any mercenary taint, he had
such an exalted respect for it, that in the retributive arrangements
made by his own mind---we all make such arrangements, more or less---%
he stationed Miss Pross much nearer to the lower Angels than many
ladies immeasurably better got up both by Nature and Art, who had
balances at Tellson's.

``There never was, nor will be, but one man worthy of Ladybird,'' said
Miss Pross; ``and that was my brother Solomon, if he hadn't made a
mistake in life.''

Here again:  Mr.\ Lorry's inquiries into Miss Pross's personal history
had established the fact that her brother Solomon was a heartless
scoundrel who had stripped her of everything she possessed, as a
stake to speculate with, and had abandoned her in her poverty for
evermore, with no touch of compunction.  Miss Pross's fidelity of
belief in Solomon (deducting a mere trifle for this slight mistake)
was quite a serious matter with Mr.\ Lorry, and had its weight in his
good opinion of her.

``As we happen to be alone for the moment, and are both people of
business,'' he said, when they had got back to the drawing-room and
had sat down there in friendly relations, ``let me ask you---does the
Doctor, in talking with Lucie, never refer to the shoemaking time, yet?''

``Never.''

``And yet keeps that bench and those tools beside him?''

``Ah!'' returned Miss Pross, shaking her head.  ``But I don't say he
don't refer to it within himself.''

``Do you believe that he thinks of it much?''

``I do,'' said Miss Pross.

``Do you imagine---'' Mr.\ Lorry had begun, when Miss Pross took him up
short with:

``Never imagine anything.  Have no imagination at all.''

``I stand corrected; do you suppose---you go so far as to suppose, sometimes?''

``Now and then,'' said Miss Pross.

``Do you suppose,'' Mr.\ Lorry went on, with a laughing twinkle in his
bright eye, as it looked kindly at her, ``that Doctor Manette has any
theory of his own, preserved through all those years, relative to the
cause of his being so oppressed; perhaps, even to the name of his
oppressor?''

``I don't suppose anything about it but what Ladybird tells me.''

``And that is---?''

``That she thinks he has.''

``Now don't be angry at my asking all these questions; because I am a
mere dull man of business, and you are a woman of business.''

``Dull?'' Miss Pross inquired, with placidity.

Rather wishing his modest adjective away, Mr.\ Lorry replied, ``No, no,
no. Surely not.  To return to business:---Is it not remarkable that
Doctor Manette, unquestionably innocent of any crime as we are all
well assured he is, should never touch upon that question?  I will not
say with me, though he had business relations with me many years ago,
and we are now intimate; I will say with the fair daughter to whom he
is so devotedly attached, and who is so devotedly attached to him?
Believe me, Miss Pross, I don't approach the topic with you, out of
curiosity, but out of zealous interest.''

``Well!  To the best of my understanding, and bad's the best,
you'll tell me,'' said Miss Pross, softened by the tone of the apology,
``he is afraid of the whole subject.''

``Afraid?''

``It's plain enough, I should think, why he may be.  It's a dreadful
remembrance.  Besides that, his loss of himself grew out of it.
Not knowing how he lost himself, or how he recovered himself, he may
never feel certain of not losing himself again.  That alone wouldn't
make the subject pleasant, I should think.''

It was a profounder remark than Mr.\ Lorry had looked for.  ``True,''
said he, ``and fearful to reflect upon.  Yet, a doubt lurks in my mind,
Miss Pross, whether it is good for Doctor Manette to have that
suppression always shut up within him.  Indeed, it is this doubt and
the uneasiness it sometimes causes me that has led me to our present
confidence.''

``Can't be helped,'' said Miss Pross, shaking her head.  ``Touch that
string, and he instantly changes for the worse.  Better leave it
alone.  In short, must leave it alone, like or no like. Sometimes,
he gets up in the dead of the night, and will be heard, by us
overhead there, walking up and down, walking up and down, in his room.
Ladybird has learnt to know then that his mind is walking up and
down, walking up and down, in his old prison.  She hurries to him,
and they go on together, walking up and down, walking up and down,
until he is composed.  But he never says a word of the true reason of
his restlessness, to her, and she finds it best not to hint at it to him.
In silence they go walking up and down together, walking up and down
together, till her love and company have brought him to himself.''

Notwithstanding Miss Pross's denial of her own imagination, there was
a perception of the pain of being monotonously haunted by one sad idea,
in her repetition of the phrase, walking up and down, which testified
to her possessing such a thing.

The corner has been mentioned as a wonderful corner for echoes;
it had begun to echo so resoundingly to the tread of coming feet,
that it seemed as though the very mention of that weary pacing to and
fro had set it going.

``Here they are!'' said Miss Pross, rising to break up the conference;
``and now we shall have hundreds of people pretty soon!''

It was such a curious corner in its acoustical properties, such a
peculiar Ear of a place, that as Mr.\ Lorry stood at the open window,
looking for the father and daughter whose steps he heard, he fancied
they would never approach.  Not only would the echoes die away,
as though the steps had gone; but, echoes of other steps that never
came would be heard in their stead, and would die away for good when
they seemed close at hand.  However, father and daughter did at last
appear, and Miss Pross was ready at the street door to receive them.

Miss Pross was a pleasant sight, albeit wild, and red, and grim, taking
off her darling's bonnet when she came up-stairs, and touching it up
with the ends of her handkerchief, and blowing the dust off it, and
folding her mantle ready for laying by, and smoothing her rich hair
with as much pride as she could possibly have taken in her own hair
if she had been the vainest and handsomest of women.  Her darling was
a pleasant sight too, embracing her and thanking her, and protesting
against her taking so much trouble for her---which last she only dared
to do playfully, or Miss Pross, sorely hurt, would have retired to
her own chamber and cried.  The Doctor was a pleasant sight too,
looking on at them, and telling Miss Pross how she spoilt Lucie, in
accents and with eyes that had as much spoiling in them as Miss Pross
had, and would have had more if it were possible.  Mr.\ Lorry was a
pleasant sight too, beaming at all this in his little wig, and thanking
his bachelor stars for having lighted him in his declining years to a
Home.  But, no Hundreds of people came to see the sights, and Mr.\ Lorry
looked in vain for the fulfilment of Miss Pross's prediction.

Dinner-time, and still no Hundreds of people.  In the arrangements of
the little household, Miss Pross took charge of the lower regions,
and always acquitted herself marvellously.  Her dinners, of a very
modest quality, were so well cooked and so well served, and so neat
in their contrivances, half English and half French, that nothing
could be better.  Miss Pross's friendship being of the thoroughly
practical kind, she had ravaged Soho and the adjacent provinces, in
search of impoverished French, who, tempted by shillings and half-%
crowns, would impart culinary mysteries to her.  From these decayed
sons and daughters of Gaul, she had acquired such wonderful arts,
that the woman and girl who formed the staff of domestics regarded
her as quite a Sorceress, or Cinderella's Godmother:  who would send
out for a fowl, a rabbit, a vegetable or two from the garden, and
change them into anything she pleased.

On Sundays, Miss Pross dined at the Doctor's table, but on other days
persisted in taking her meals at unknown periods, either in the lower
regions, or in her own room on the second floor---a blue chamber,
to which no one but her Ladybird ever gained admittance.  On this
occasion, Miss Pross, responding to Ladybird's pleasant face and
pleasant efforts to please her, unbent exceedingly; so the dinner was
very pleasant, too.

It was an oppressive day, and, after dinner, Lucie proposed that the
wine should be carried out under the plane-tree, and they should sit
there in the air.  As everything turned upon her, and revolved about
her, they went out under the plane-tree, and she carried the wine
down for the special benefit of Mr.\ Lorry.  She had installed herself,
some time before, as Mr.\ Lorry's cup-bearer; and while they sat under
the plane-tree, talking, she kept his glass replenished.  Mysterious
backs and ends of houses peeped at them as they talked, and the
plane-tree whispered to them in its own way above their heads.

Still, the Hundreds of people did not present themselves.  Mr.\ Darnay
presented himself while they were sitting under the plane-tree,
but he was only One.

Doctor Manette received him kindly, and so did Lucie.  But, Miss
Pross suddenly became afflicted with a twitching in the head and
body, and retired into the house.  She was not unfrequently the
victim of this disorder, and she called it, in familiar conversation,
``a fit of the jerks.''

The Doctor was in his best condition, and looked specially young.
The resemblance between him and Lucie was very strong at such times,
and as they sat side by side, she leaning on his shoulder, and he
resting his arm on the back of her chair, it was very agreeable to
trace the likeness.

He had been talking all day, on many subjects, and with unusual vivacity.
``Pray, Doctor Manette,'' said Mr.\ Darnay, as they sat under the
plane-tree---and he said it in the natural pursuit of the topic in
hand, which happened to be the old buildings of London---``have you
seen much of the Tower?''

``Lucie and I have been there; but only casually.  We have seen enough
of it, to know that it teems with interest; little more.''

``\emph{I} have been there, as you remember,'' said Darnay, with a smile,
though reddening a little angrily, ``in another character, and not in
a character that gives facilities for seeing much of it.  They told
me a curious thing when I was there.''

``What was that?'' Lucie asked.

``In making some alterations, the workmen came upon an old dungeon,
which had been, for many years, built up and forgotten.  Every stone
of its inner wall was covered by inscriptions which had been carved
by prisoners---dates, names, complaints, and prayers.  Upon a corner
stone in an angle of the wall, one prisoner, who seemed to have gone
to execution, had cut as his last work, three letters.  They were
done with some very poor instrument, and hurriedly, with an unsteady
hand.  At first, they were read as D. I. C.; but, on being more
carefully examined, the last letter was found to be G. There was no
record or legend of any prisoner with those initials, and many
fruitless guesses were made what the name could have been.
At length, it was suggested that the letters were not initials, but
the complete word, DiG.  The floor was examined very carefully under
the inscription, and, in the earth beneath a stone, or tile, or some
fragment of paving, were found the ashes of a paper, mingled with the
ashes of a small leathern case or bag.  What the unknown prisoner had
written will never be read, but he had written something, and hidden
it away to keep it from the gaoler.''

``My father,'' exclaimed Lucie, ``you are ill!''

He had suddenly started up, with his hand to his head.  His manner
and his look quite terrified them all.

``No, my dear, not ill.  There are large drops of rain falling,
and they made me start.  We had better go in.''

He recovered himself almost instantly.  Rain was really falling in
large drops, and he showed the back of his hand with rain-drops on it.
But, he said not a single word in reference to the discovery that had
been told of, and, as they went into the house, the business eye of
Mr.\ Lorry either detected, or fancied it detected, on his face, as it
turned towards Charles Darnay, the same singular look that had been
upon it when it turned towards him in the passages of the Court House.

He recovered himself so quickly, however, that Mr.\ Lorry had doubts
of his business eye.  The arm of the golden giant in the hall was not
more steady than he was, when he stopped under it to remark to them
that he was not yet proof against slight surprises (if he ever would
be), and that the rain had startled him.

Tea-time, and Miss Pross making tea, with another fit of the jerks
upon her, and yet no Hundreds of people.  Mr.\ Carton had lounged in,
but he made only Two.

The night was so very sultry, that although they sat with doors and
windows open, they were overpowered by heat.  When the tea-table was
done with, they all moved to one of the windows, and looked out into
the heavy twilight.  Lucie sat by her father; Darnay sat beside her;
Carton leaned against a window.  The curtains were long and white,
and some of the thunder-gusts that whirled into the corner, caught
them up to the ceiling, and waved them like spectral wings.

``The rain-drops are still falling, large, heavy, and few,'' said
Doctor Manette.  ``It comes slowly.''

``It comes surely,'' said Carton.

They spoke low, as people watching and waiting mostly do; as people
in a dark room, watching and waiting for Lightning, always do.

There was a great hurry in the streets of people speeding away to get
shelter before the storm broke; the wonderful corner for echoes
resounded with the echoes of footsteps coming and going, yet not a
footstep was there.

``A multitude of people, and yet a solitude!'' said Darnay, when they
had listened for a while.

``Is it not impressive, Mr.\ Darnay?'' asked Lucie.  ``Sometimes, I have
sat here of an evening, until I have fancied---but even the shade of a
foolish fancy makes me shudder to-night, when all is so black and
solemn---''

``Let us shudder too.  We may know what it is.''

``It will seem nothing to you.  Such whims are only impressive as we
originate them, I think; they are not to be communicated.  I have
sometimes sat alone here of an evening, listening, until I have made
the echoes out to be the echoes of all the footsteps that are coming
by-and-bye into our lives.''

``There is a great crowd coming one day into our lives, if that be so,''
Sydney Carton struck in, in his moody way.

The footsteps were incessant, and the hurry of them became more and
more rapid.  The corner echoed and re-echoed with the tread of feet;
some, as it seemed, under the windows; some, as it seemed, in the room;
some coming, some going, some breaking off, some stopping altogether;
all in the distant streets, and not one within sight.

``Are all these footsteps destined to come to all of us, Miss Manette,
or are we to divide them among us?''

``I don't know, Mr.\ Darnay; I told you it was a foolish fancy, but you
asked for it.  When I have yielded myself to it, I have been alone,
and then I have imagined them the footsteps of the people who are to
come into my life, and my father's.''

``I take them into mine!'' said Carton.  ``\emph{I} ask no questions and make
no stipulations.  There is a great crowd bearing down upon us, Miss
Manette, and I see them---by the Lightning.''  He added the last words,
after there had been a vivid flash which had shown him lounging in
the window.

``And I hear them!'' he added again, after a peal of thunder.
``Here they come, fast, fierce, and furious!''

It was the rush and roar of rain that he typified, and it stopped him,
for no voice could be heard in it.  A memorable storm of thunder and
lightning broke with that sweep of water, and there was not a moment's
interval in crash, and fire, and rain, until after the moon rose at
midnight.

The great bell of Saint Paul's was striking one in the cleared air,
when Mr.\ Lorry, escorted by Jerry, high-booted and bearing a lantern,
set forth on his return-passage to Clerkenwell.  There were solitary
patches of road on the way between Soho and Clerkenwell, and Mr.\ Lorry,
mindful of foot-pads, always retained Jerry for this service:  though
it was usually performed a good two hours earlier.

``What a night it has been!  Almost a night, Jerry,'' said Mr.\ Lorry,
``to bring the dead out of their graves.''

``I never see the night myself, master---nor yet I don't expect to---%
what would do that,'' answered Jerry.

``Good night, Mr.\ Carton,'' said the man of business.  ``Good night,
Mr.\ Darnay.  Shall we ever see such a night again, together!''

Perhaps.  Perhaps, see the great crowd of people with its rush and
roar, bearing down upon them, too.



\chapter{Monseigneur in Town}


Monseigneur, one of the great lords in power at the Court, held his
fortnightly reception in his grand hotel in Paris.  Monseigneur was
in his inner room, his sanctuary of sanctuaries, the Holiest of
Holiests to the crowd of worshippers in the suite of rooms without.
Monseigneur was about to take his chocolate.  Monseigneur could
swallow a great many things with ease, and was by some few sullen
minds supposed to be rather rapidly swallowing France; but, his
morning's chocolate could not so much as get into the throat of
Monseigneur, without the aid of four strong men besides the Cook.

Yes.  It took four men, all four ablaze with gorgeous decoration,
and the Chief of them unable to exist with fewer than two gold
watches in his pocket, emulative of the noble and chaste fashion set
by Monseigneur, to conduct the happy chocolate to Monseigneur's lips.
One lacquey carried the chocolate-pot into the sacred presence;
a second, milled and frothed the chocolate with the little instrument
he bore for that function; a third, presented the favoured napkin;
a fourth (he of the two gold watches), poured the chocolate out.
It was impossible for Monseigneur to dispense with one of these
attendants on the chocolate and hold his high place under the
admiring Heavens.  Deep would have been the blot upon his escutcheon
if his chocolate had been ignobly waited on by only three men; he
must have died of two.

Monseigneur had been out at a little supper last night, where the
Comedy and the Grand Opera were charmingly represented.  Monseigneur
was out at a little supper most nights, with fascinating company.
So polite and so impressible was Monseigneur, that the Comedy and
the Grand Opera had far more influence with him in the tiresome
articles of state affairs and state secrets, than the needs of all
France.  A happy circumstance for France, as the like always is for
all countries similarly favoured!---always was for England (by way of
example), in the regretted days of the merry Stuart who sold it.

Monseigneur had one truly noble idea of general public business,
which was, to let everything go on in its own way; of particular
public business, Monseigneur had the other truly noble idea that it
must all go his way---tend to his own power and pocket.  Of his
pleasures, general and particular, Monseigneur had the other truly
noble idea, that the world was made for them.  The text of his order
(altered from the original by only a pronoun, which is not much) ran:
``The earth and the fulness thereof are mine, saith Monseigneur.''

Yet, Monseigneur had slowly found that vulgar embarrassments crept
into his affairs, both private and public; and he had, as to both
classes of affairs, allied himself perforce with a Farmer-General.
As to finances public, because Monseigneur could not make anything
at all of them, and must consequently let them out to somebody who
could; as to finances private, because Farmer-Generals were rich, and
Monseigneur, after generations of great luxury and expense, was
growing poor.  Hence Monseigneur had taken his sister from a convent,
while there was yet time to ward off the impending veil, the cheapest
garment she could wear, and had bestowed her as a prize upon a very
rich Farmer-General, poor in family.  Which Farmer-General, carrying
an appropriate cane with a golden apple on the top of it, was now
among the company in the outer rooms, much prostrated before by
mankind---always excepting superior mankind of the blood of Monseigneur,
who, his own wife included, looked down upon him with the loftiest
contempt.

A sumptuous man was the Farmer-General.  Thirty horses stood in his
stables, twenty-four male domestics sat in his halls, six body-women
waited on his wife.  As one who pretended to do nothing but plunder
and forage where he could, the Farmer-General---howsoever his
matrimonial relations conduced to social morality---was at least the
greatest reality among the personages who attended at the hotel of
Monseigneur that day.

For, the rooms, though a beautiful scene to look at, and adorned with
every device of decoration that the taste and skill of the time could
achieve, were, in truth, not a sound business; considered with any
reference to the scarecrows in the rags and nightcaps elsewhere
(and not so far off, either, but that the watching towers of Notre
Dame, almost equidistant from the two extremes, could see them both),
they would have been an exceedingly uncomfortable business---if that
could have been anybody's business, at the house of Monseigneur.
Military officers destitute of military knowledge; naval officers
with no idea of a ship; civil officers without a notion of affairs;
brazen ecclesiastics, of the worst world worldly, with sensual eyes,
loose tongues, and looser lives; all totally unfit for their several
callings, all lying horribly in pretending to belong to them, but all
nearly or remotely of the order of Monseigneur, and therefore foisted
on all public employments from which anything was to be got; these were
to be told off by the score and the score.  People not immediately
connected with Monseigneur or the State, yet equally unconnected with
anything that was real, or with lives passed in travelling by any
straight road to any true earthly end, were no less abundant.
Doctors who made great fortunes out of dainty remedies for imaginary
disorders that never existed, smiled upon their courtly patients in
the ante-chambers of Monseigneur.  Projectors who had discovered
every kind of remedy for the little evils with which the State was
touched, except the remedy of setting to work in earnest to root out
a single sin, poured their distracting babble into any ears they
could lay hold of, at the reception of Monseigneur.  Unbelieving
Philosophers who were remodelling the world with words, and making
card-towers of Babel to scale the skies with, talked with Unbelieving
Chemists who had an eye on the transmutation of metals, at this
wonderful gathering accumulated by Monseigneur.  Exquisite gentlemen
of the finest breeding, which was at that remarkable time---and has
been since---to be known by its fruits of indifference to every
natural subject of human interest, were in the most exemplary state
of exhaustion, at the hotel of Monseigneur.  Such homes had these
various notabilities left behind them in the fine world of Paris,
that the spies among the assembled devotees of Monseigneur---forming a
goodly half of the polite company---would have found it hard to
discover among the angels of that sphere one solitary wife, who, in
her manners and appearance, owned to being a Mother.  Indeed, except
for the mere act of bringing a troublesome creature into this world---%
which does not go far towards the realisation of the name of mother---%
there was no such thing known to the fashion.  Peasant women kept the
unfashionable babies close, and brought them up, and charming grandmammas
of sixty dressed and supped as at twenty.

The leprosy of unreality disfigured every human creature in attendance
upon Monseigneur.  In the outermost room were half a dozen exceptional
people who had had, for a few years, some vague misgiving in them
that things in general were going rather wrong.  As a promising way
of setting them right, half of the half-dozen had become members of a
fantastic sect of Convulsionists, and were even then considering within
themselves whether they should foam, rage, roar, and turn cataleptic
on the spot---thereby setting up a highly intelligible finger-post to
the Future, for Monseigneur's guidance.  Besides these Dervishes,
were other three who had rushed into another sect, which mended
matters with a jargon about ``the Centre of Truth:'' holding that Man
had got out of the Centre of Truth---which did not need much
demonstration---but had not got out of the Circumference, and that he
was to be kept from flying out of the Circumference, and was even to
be shoved back into the Centre, by fasting and seeing of spirits.
Among these, accordingly, much discoursing with spirits went on---and
it did a world of good which never became manifest.

But, the comfort was, that all the company at the grand hotel of
Monseigneur were perfectly dressed.  If the Day of Judgment had only
been ascertained to be a dress day, everybody there would have been
eternally correct.  Such frizzling and powdering and sticking up of
hair, such delicate complexions artificially preserved and mended,
such gallant swords to look at, and such delicate honour to the sense
of smell, would surely keep anything going, for ever and ever.
The exquisite gentlemen of the finest breeding wore little pendent
trinkets that chinked as they languidly moved; these golden fetters
rang like precious little bells; and what with that ringing, and with
the rustle of silk and brocade and fine linen, there was a flutter in
the air that fanned Saint Antoine and his devouring hunger far away.

Dress was the one unfailing talisman and charm used for keeping all
things in their places.  Everybody was dressed for a Fancy Ball that
was never to leave off.  From the Palace of the Tuileries, through
Monseigneur and the whole Court, through the Chambers, the Tribunals
of Justice, and all society (except the scarecrows), the Fancy Ball
descended to the Common Executioner:  who, in pursuance of the charm,
was required to officiate ``frizzled, powdered, in a gold-laced coat,
pumps, and white silk stockings.''  At the gallows and the wheel---the
axe was a rarity---Monsieur Paris, as it was the episcopal mode among
his brother Professors of the provinces, Monsieur Orleans, and the
rest, to call him, presided in this dainty dress.  And who among the
company at Monseigneur's reception in that seventeen hundred and
eightieth year of our Lord, could possibly doubt, that a system
rooted in a frizzled hangman, powdered, gold-laced, pumped, and
white-silk stockinged, would see the very stars out!

Monseigneur having eased his four men of their burdens and taken his
chocolate, caused the doors of the Holiest of Holiests to be thrown
open, and issued forth.  Then, what submission, what cringing and
fawning, what servility, what abject humiliation!  As to bowing down
in body and spirit, nothing in that way was left for Heaven---which
may have been one among other reasons why the worshippers of
Monseigneur never troubled it.

Bestowing a word of promise here and a smile there, a whisper on one
happy slave and a wave of the hand on another, Monseigneur affably
passed through his rooms to the remote region of the Circumference of
Truth.  There, Monseigneur turned, and came back again, and so in due
course of time got himself shut up in his sanctuary by the chocolate
sprites, and was seen no more.

The show being over, the flutter in the air became quite a little
storm, and the precious little bells went ringing downstairs.
There was soon but one person left of all the crowd, and he, with his
hat under his arm and his snuff-box in his hand, slowly passed among
the mirrors on his way out.

``I devote you,'' said this person, stopping at the last door on his
way, and turning in the direction of the sanctuary, ``to the Devil!''

With that, he shook the snuff from his fingers as if he had shaken
the dust from his feet, and quietly walked downstairs.

He was a man of about sixty, handsomely dressed, haughty in manner,
and with a face like a fine mask.  A face of a transparent paleness;
every feature in it clearly defined; one set expression on it.
The nose, beautifully formed otherwise, was very slightly pinched at
the top of each nostril.  In those two compressions, or dints, the
only little change that the face ever showed, resided.  They persisted
in changing colour sometimes, and they would be occasionally dilated
and contracted by something like a faint pulsation; then, they gave a
look of treachery, and cruelty, to the whole countenance.  Examined
with attention, its capacity of helping such a look was to be found
in the line of the mouth, and the lines of the orbits of the eyes,
being much too horizontal and thin; still, in the effect of the face
made, it was a handsome face, and a remarkable one.

Its owner went downstairs into the courtyard, got into his carriage,
and drove away.  Not many people had talked with him at the reception;
he had stood in a little space apart, and Monseigneur might have been
warmer in his manner.  It appeared, under the circumstances, rather
agreeable to him to see the common people dispersed before his horses,
and often barely escaping from being run down.  His man drove as if
he were charging an enemy, and the furious recklessness of the man
brought no check into the face, or to the lips, of the master.  The
complaint had sometimes made itself audible, even in that deaf city
and dumb age, that, in the narrow streets without footways, the fierce
patrician custom of hard driving endangered and maimed the mere vulgar
in a barbarous manner.  But, few cared enough for that to think of it
a second time, and, in this matter, as in all others, the common
wretches were left to get out of their difficulties as they could.

With a wild rattle and clatter, and an inhuman abandonment of
consideration not easy to be understood in these days, the carriage
dashed through streets and swept round corners, with women screaming
before it, and men clutching each other and clutching children out of
its way.  At last, swooping at a street corner by a fountain, one of
its wheels came to a sickening little jolt, and there was a loud cry
from a number of voices, and the horses reared and plunged.

But for the latter inconvenience, the carriage probably would not
have stopped; carriages were often known to drive on, and leave their
wounded behind, and why not?  But the frightened valet had got down in
a hurry, and there were twenty hands at the horses' bridles.

``What has gone wrong?'' said Monsieur, calmly looking out.

A tall man in a nightcap had caught up a bundle from among the feet
of the horses, and had laid it on the basement of the fountain,
and was down in the mud and wet, howling over it like a wild animal.

``Pardon, Monsieur the Marquis!'' said a ragged and submissive man,
``it is a child.''

``Why does he make that abominable noise?  Is it his child?''

``Excuse me, Monsieur the Marquis---it is a pity---yes.''

The fountain was a little removed; for the street opened, where it
was, into a space some ten or twelve yards square.  As the tall man
suddenly got up from the ground, and came running at the carriage,
Monsieur the Marquis clapped his hand for an instant on his sword-hilt.

``Killed!'' shrieked the man, in wild desperation, extending both arms
at their length above his head, and staring at him.  ``Dead!''

The people closed round, and looked at Monsieur the Marquis.
There was nothing revealed by the many eyes that looked at him but
watchfulness and eagerness; there was no visible menacing or anger.
Neither did the people say anything; after the first cry, they had
been silent, and they remained so.  The voice of the submissive man
who had spoken, was flat and tame in its extreme submission.
Monsieur the Marquis ran his eyes over them all, as if they had been
mere rats come out of their holes.

He took out his purse.

``It is extraordinary to me,'' said he, ``that you people cannot take
care of yourselves and your children.  One or the other of you is for
ever in the, way.  How do I know what injury you have done my horses.
See!  Give him that.''

He threw out a gold coin for the valet to pick up, and all the heads
craned forward that all the eyes might look down at it as it fell.
The tall man called out again with a most unearthly cry, ``Dead!''

He was arrested by the quick arrival of another man, for whom the
rest made way.  On seeing him, the miserable creature fell upon his
shoulder, sobbing and crying, and pointing to the fountain, where
some women were stooping over the motionless bundle, and moving
gently about it.  They were as silent, however, as the men.

``I know all, I know all,'' said the last comer.  ``Be a brave man, my
Gaspard!  It is better for the poor little plaything to die so, than
to live.  It has died in a moment without pain.  Could it have lived
an hour as happily?''

``You are a philosopher, you there,'' said the, Marquis, smiling.
``How do they call you?''

``They call me Defarge.''

``Of what trade?''

``Monsieur the Marquis, vendor of wine.''

``Pick up that, philosopher and vendor of wine,'' said the Marquis,
throwing him another gold coin, ``and spend it as you will.
The horses there; are they right?''

Without deigning to look at the assemblage a second time, Monsieur
the Marquis leaned back in his seat, and was just being driven away
with the air of a gentleman who had accidentally broke some common
thing, and had paid for it, and could afford to pay for it; when his
ease was suddenly disturbed by a coin flying into his carriage,
and ringing on its floor.

``Hold!'' said Monsieur the Marquis.  ``Hold the horses!  Who threw that?''

He looked to the spot where Defarge the vendor of wine had stood,
a moment before; but the wretched father was grovelling on his face
on the pavement in that spot, and the figure that stood beside him
was the figure of a dark stout woman, knitting.

``You dogs!'' said the Marquis, but smoothly, and with an unchanged front,
except as to the spots on his nose:  ``I would ride over any of you
very willingly, and exterminate you from the earth. If I knew which
rascal threw at the carriage, and if that brigand were sufficiently
near it, he should be crushed under the wheels.''

So cowed was their condition, and so long and hard their experience
of what such a man could do to them, within the law and beyond it,
that not a voice, or a hand, or even an eye was raised.  Among the
men, not one.  But the woman who stood knitting looked up steadily,
and looked the Marquis in the face.  It was not for his dignity to
notice it; his contemptuous eyes passed over her, and over all the
other rats; and he leaned back in his seat again, and gave the word
``Go on!''

He was driven on, and other carriages came whirling by in quick
succession; the Minister, the State-Projector, the Farmer-General,
the Doctor, the Lawyer, the Ecclesiastic, the Grand Opera, the
Comedy, the whole Fancy Ball in a bright continuous flow, came
whirling by.  The rats had crept out of their holes to look on,
and they remained looking on for hours; soldiers and police often
passing between them and the spectacle, and making a barrier behind
which they slunk, and through which they peeped.  The father had long
ago taken up his bundle and bidden himself away with it, when the
women who had tended the bundle while it lay on the base of the
fountain, sat there watching the running of the water and the rolling
of the Fancy Ball---when the one woman who had stood conspicuous,
knitting, still knitted on with the steadfastness of Fate.  The water
of the fountain ran, the swift river ran, the day ran into evening,
so much life in the city ran into death according to rule, time and
tide waited for no man, the rats were sleeping close together in
their dark holes again, the Fancy Ball was lighted up at supper,
all things ran their course.



\chapter{Monseigneur in the Country}


A beautiful landscape, with the corn bright in it, but not abundant.
Patches of poor rye where corn should have been, patches of poor peas
and beans, patches of most coarse vegetable substitutes for wheat.
On inanimate nature, as on the men and women who cultivated it,
a prevalent tendency towards an appearance of vegetating
unwillingly---a dejected disposition to give up, and wither away.

Monsieur the Marquis in his travelling carriage (which might have
been lighter), conducted by four post-horses and two postilions,
fagged up a steep hill.  A blush on the countenance of Monsieur the
Marquis was no impeachment of his high breeding; it was not from
within; it was occasioned by an external circumstance beyond his
control---the setting sun.

The sunset struck so brilliantly into the travelling carriage when it
gained the hill-top, that its occupant was steeped in crimson.
``It will die out,'' said Monsieur the Marquis, glancing at his hands,
``directly.''

In effect, the sun was so low that it dipped at the moment.  When the
heavy drag had been adjusted to the wheel, and the carriage slid down
hill, with a cinderous smell, in a cloud of dust, the red glow departed
quickly; the sun and the Marquis going down together, there was no
glow left when the drag was taken off.

But, there remained a broken country, bold and open, a little village
at the bottom of the hill, a broad sweep and rise beyond it, a church-%
tower, a windmill, a forest for the chase, and a crag with a fortress
on it used as a prison.  Round upon all these darkening objects as
the night drew on, the Marquis looked, with the air of one who was
coming near home.

The village had its one poor street, with its poor brewery, poor
tannery, poor tavern, poor stable-yard for relays of post-horses,
poor fountain, all usual poor appointments.  It had its poor people
too.  All its people were poor, and many of them were sitting at
their doors, shredding spare onions and the like for supper, while
many were at the fountain, washing leaves, and grasses, and any such
small yieldings of the earth that could be eaten.  Expressive sips of
what made them poor, were not wanting; the tax for the state, the tax
for the church, the tax for the lord, tax local and tax general, were
to be paid here and to be paid there, according to solemn inscription
in the little village, until the wonder was, that there was any
village left unswallowed.

Few children were to be seen, and no dogs.  As to the men and women,
their choice on earth was stated in the prospect---Life on the lowest
terms that could sustain it, down in the little village under the
mill; or captivity and Death in the dominant prison on the crag.

Heralded by a courier in advance, and by the cracking of his
postilions' whips, which twined snake-like about their heads in the
evening air, as if he came attended by the Furies, Monsieur the
Marquis drew up in his travelling carriage at the posting-house gate.
It was hard by the fountain, and the peasants suspended their
operations to look at him.  He looked at them, and saw in them,
without knowing it, the slow sure filing down of misery-worn face and
figure, that was to make the meagreness of Frenchmen an English
superstition which should survive the truth through the best part of
a hundred years.

Monsieur the Marquis cast his eyes over the submissive faces that
drooped before him, as the like of himself had drooped before
Monseigneur of the Court---only the difference was, that these faces
drooped merely to suffer and not to propitiate---when a grizzled
mender of the roads joined the group.

``Bring me hither that fellow!'' said the Marquis to the courier.

The fellow was brought, cap in hand, and the other fellows closed
round to look and listen, in the manner of the people at the Paris
fountain.

``I passed you on the road?''

``Monseigneur, it is true.  I had the honour of being passed on the road.''

``Coming up the hill, and at the top of the hill, both?''

``Monseigneur, it is true.''

``What did you look at, so fixedly?''

``Monseigneur, I looked at the man.''

He stooped a little, and with his tattered blue cap pointed under the
carriage.  All his fellows stooped to look under the carriage.

``What man, pig?  And why look there?''

``Pardon, Monseigneur; he swung by the chain of the shoe---the drag.''

``Who?'' demanded the traveller.

``Monseigneur, the man.''

``May the Devil carry away these idiots!  How do you call the man?
You know all the men of this part of the country.  Who was he?''

``Your clemency, Monseigneur!  He was not of this part of the country.
Of all the days of my life, I never saw him.''

``Swinging by the chain?  To be suffocated?''

``With your gracious permission, that was the wonder of it,
Monseigneur.  His head hanging over---like this!''

He turned himself sideways to the carriage, and leaned back, with his
face thrown up to the sky, and his head hanging down; then recovered
himself, fumbled with his cap, and made a bow.

``What was he like?''

``Monseigneur, he was whiter than the miller.  All covered with dust,
white as a spectre, tall as a spectre!''

The picture produced an immense sensation in the little crowd;
but all eyes, without comparing notes with other eyes, looked at
Monsieur the Marquis.  Perhaps, to observe whether he had any spectre
on his conscience.

``Truly, you did well,'' said the Marquis, felicitously sensible that
such vermin were not to ruffle him, ``to see a thief accompanying my
carriage, and not open that great mouth of yours.  Bah!  Put him aside,
Monsieur Gabelle!''

Monsieur Gabelle was the Postmaster, and some other taxing functionary
united; he had come out with great obsequiousness to assist at this
examination, and had held the examined by the drapery of his arm in
an official manner.

``Bah!  Go aside!'' said Monsieur Gabelle.

``Lay hands on this stranger if he seeks to lodge in your village
to-night, and be sure that his business is honest, Gabelle.''

``Monseigneur, I am flattered to devote myself to your orders.''

``Did he run away, fellow?---where is that Accursed?''

The accursed was already under the carriage with some half-dozen
particular friends, pointing out the chain with his blue cap.
Some half-dozen other particular friends promptly hauled him out,
and presented him breathless to Monsieur the Marquis.

``Did the man run away, Dolt, when we stopped for the drag?''

``Monseigneur, he precipitated himself over the hill-side, head first,
as a person plunges into the river.''

``See to it, Gabelle.  Go on!''

The half-dozen who were peering at the chain were still among the
wheels, like sheep; the wheels turned so suddenly that they were
lucky to save their skins and bones; they had very little else to
save, or they might not have been so fortunate.

The burst with which the carriage started out of the village and up
the rise beyond, was soon checked by the steepness of the hill.
Gradually, it subsided to a foot pace, swinging and lumbering upward
among the many sweet scents of a summer night.  The postilions, with
a thousand gossamer gnats circling about them in lieu of the Furies,
quietly mended the points to the lashes of their whips; the valet
walked by the horses; the courier was audible, trotting on ahead into
the dun distance.

At the steepest point of the hill there was a little burial-ground,
with a Cross and a new large figure of Our Saviour on it; it was a
poor figure in wood, done by some inexperienced rustic carver, but he
had studied the figure from the life---his own life, maybe---for it was
dreadfully spare and thin.

To this distressful emblem of a great distress that had long been
growing worse, and was not at its worst, a woman was kneeling.
She turned her head as the carriage came up to her, rose quickly,
and presented herself at the carriage-door.

``It is you, Monseigneur!  Monseigneur, a petition.''

With an exclamation of impatience, but with his unchangeable face,
Monseigneur looked out.

``How, then!  What is it?  Always petitions!''

``Monseigneur.  For the love of the great God!  My husband, the forester.''

``What of your husband, the forester?  Always the same with you people.
He cannot pay something?''

``He has paid all, Monseigneur.  He is dead.''

``Well!  He is quiet.  Can I restore him to you?''

``Alas, no, Monseigneur!  But he lies yonder, under a little heap of
poor grass.''

``Well?''

``Monseigneur, there are so many little heaps of poor grass?''

``Again, well?''

She looked an old woman, but was young.  Her manner was one of
passionate grief; by turns she clasped her veinous and knotted hands
together with wild energy, and laid one of them on the carriage-door%
---tenderly, caressingly, as if it had been a human breast, and could
be expected to feel the appealing touch.

``Monseigneur, hear me!  Monseigneur, hear my petition!  My husband
died of want; so many die of want; so many more will die of want.''

``Again, well?  Can I feed them?''

``Monseigneur, the good God knows; but I don't ask it.  My petition is,
that a morsel of stone or wood, with my husband's name, may be placed
over him to show where he lies.  Otherwise, the place will be quickly
forgotten, it will never be found when I am dead of the same malady,
I shall be laid under some other heap of poor grass.  Monseigneur,
they are so many, they increase so fast, there is so much want.
Monseigneur!  Monseigneur!''

The valet had put her away from the door, the carriage had broken
into a brisk trot, the postilions had quickened the pace, she was
left far behind, and Monseigneur, again escorted by the Furies, was
rapidly diminishing the league or two of distance that remained
between him and his chateau.

The sweet scents of the summer night rose all around him, and rose,
as the rain falls, impartially, on the dusty, ragged, and toil-worn
group at the fountain not far away; to whom the mender of roads, with
the aid of the blue cap without which he was nothing, still enlarged
upon his man like a spectre, as long as they could bear it.
By degrees, as they could bear no more, they dropped off one by one,
and lights twinkled in little casements; which lights, as the
casements darkened, and more stars came out, seemed to have shot up
into the sky instead of having been extinguished.

The shadow of a large high-roofed house, and of many over-hanging
trees, was upon Monsieur the Marquis by that time; and the shadow was
exchanged for the light of a flambeau, as his carriage stopped,
and the great door of his chateau was opened to him.

``Monsieur Charles, whom I expect; is he arrived from England?''

``Monseigneur, not yet.''



\chapter{The Gorgon's Head}


It was a heavy mass of building, that chateau of Monsieur the Marquis,
with a large stone courtyard before it, and two stone sweeps of
staircase meeting in a stone terrace before the principal door.
A stony business altogether, with heavy stone balustrades, and stone
urns, and stone flowers, and stone faces of men, and stone heads of
lions, in all directions.  As if the Gorgon's head had surveyed it,
when it was finished, two centuries ago.

Up the broad flight of shallow steps, Monsieur the Marquis, flambeau
preceded, went from his carriage, sufficiently disturbing the darkness
to elicit loud remonstrance from an owl in the roof of the great pile
of stable building away among the trees.  All else was so quiet, that
the flambeau carried up the steps, and the other flambeau held at the
great door, burnt as if they were in a close room of state, instead
of being in the open night-air.  Other sound than the owl's voice
there was none, save the failing of a fountain into its stone basin;
for, it was one of those dark nights that hold their breath by the hour
together, and then heave a long low sigh, and hold their breath again.

The great door clanged behind him, and Monsieur the Marquis crossed
a hall grim with certain old boar-spears, swords, and knives of the
chase; grimmer with certain heavy riding-rods and riding-whips, of
which many a peasant, gone to his benefactor Death, had felt the
weight when his lord was angry.

Avoiding the larger rooms, which were dark and made fast for the
night, Monsieur the Marquis, with his flambeau-bearer going on before,
went up the staircase to a door in a corridor.  This thrown open,
admitted him to his own private apartment of three rooms:
his bed-chamber and two others.  High vaulted rooms with cool
uncarpeted floors, great dogs upon the hearths for the burning
of wood in winter time, and all luxuries befitting the state
of a marquis in a luxurious age and country.  The fashion
of the last Louis but one, of the line that was never to break%
---the fourteenth Louis---was conspicuous in their rich furniture;
but, it was diversified by many objects that were illustrations
of old pages in the history of France.

A supper-table was laid for two, in the third of the rooms; a round
room, in one of the chateau's four extinguisher-topped towers.
A small lofty room, with its window wide open, and the wooden
jalousie-blinds closed, so that the dark night only showed in slight
horizontal lines of black, alternating with their broad lines of
stone colour.

``My nephew,'' said the Marquis, glancing at the supper preparation;
``they said he was not arrived.''

Nor was he; but, he had been expected with Monseigneur.

``Ah!  It is not probable he will arrive to-night; nevertheless, leave
the table as it is.  I shall be ready in a quarter of an hour.''

In a quarter of an hour Monseigneur was ready, and sat down alone
to his sumptuous and choice supper. His chair was opposite to the
window, and he had taken his soup, and was raising his glass of
Bordeaux to his lips, when he put it down.

``What is that?'' he calmly asked, looking with attention at the
horizontal lines of black and stone colour.

``Monseigneur?  That?''

``Outside the blinds.  Open the blinds.''

It was done.

``Well?''

``Monseigneur, it is nothing.  The trees and the night are all that
are here.''

The servant who spoke, had thrown the blinds wide, had looked out
into the vacant darkness, and stood with that blank behind him,
looking round for instructions.

``Good,'' said the imperturbable master.  ``Close them again.''

That was done too, and the Marquis went on with his supper.  He was
half way through it, when he again stopped with his glass in his
hand, hearing the sound of wheels.  It came on briskly, and came up
to the front of the chateau.

``Ask who is arrived.''

It was the nephew of Monseigneur.  He had been some few leagues
behind Monseigneur, early in the afternoon.  He had diminished the
distance rapidly, but not so rapidly as to come up with Monseigneur
on the road.  He had heard of Monseigneur, at the posting-houses,
as being before him.

He was to be told (said Monseigneur) that supper awaited him then and
there, and that he was prayed to come to it.  In a little while he came.
He had been known in England as Charles Darnay.

Monseigneur received him in a courtly manner, but they did not shake hands.

``You left Paris yesterday, sir?'' he said to Monseigneur, as he took
his seat at table.

``Yesterday.  And you?''

``I come direct.''

``From London?''

``Yes.''

``You have been a long time coming,'' said the Marquis, with a smile.

``On the contrary; I come direct.''

``Pardon me!  I mean, not a long time on the journey; a long time
intending the journey.''

``I have been detained by''---the nephew stopped a moment in his
answer---``various business.''

``Without doubt,'' said the polished uncle.

So long as a servant was present, no other words passed between them.
When coffee had been served and they were alone together, the nephew,
looking at the uncle and meeting the eyes of the face that was like a
fine mask, opened a conversation.

``I have come back, sir, as you anticipate, pursuing the object that
took me away.  It carried me into great and unexpected peril; but it
is a sacred object, and if it had carried me to death I hope it would
have sustained me.''

``Not to death,'' said the uncle; ``it is not necessary to say, to death.''

``I doubt, sir,'' returned the nephew, ``whether, if it had carried me
to the utmost brink of death, you would have cared to stop me there.''

The deepened marks in the nose, and the lengthening of the fine
straight lines in the cruel face, looked ominous as to that; the
uncle made a graceful gesture of protest, which was so clearly a
slight form of good breeding that it was not reassuring.

``Indeed, sir,'' pursued the nephew, ``for anything I know, you may
have expressly worked to give a more suspicious appearance to the
suspicious circumstances that surrounded me.''

``No, no, no,'' said the uncle, pleasantly.

``But, however that may be,'' resumed the nephew, glancing at him with
deep distrust, ``I know that your diplomacy would stop me by any
means, and would know no scruple as to means.''

``My friend, I told you so,'' said the uncle, with a fine pulsation in
the two marks.  ``Do me the favour to recall that I told you so, long ago.''

``I recall it.''

``Thank you,'' said the Marquise---very sweetly indeed.

His tone lingered in the air, almost like the tone of a musical
instrument.

``In effect, sir,'' pursued the nephew, ``I believe it to be at once
your bad fortune, and my good fortune, that has kept me out of a
prison in France here.''

``I do not quite understand,'' returned the uncle, sipping his coffee.
``Dare I ask you to explain?''

``I believe that if you were not in disgrace with the Court,
and had not been overshadowed by that cloud for years past, a letter
de cachet would have sent me to some fortress indefinitely.''

``It is possible,'' said the uncle, with great calmness.  ``For the
honour of the family, I could even resolve to incommode you to that
extent.  Pray excuse me!''

``I perceive that, happily for me, the Reception of the day before
yesterday was, as usual, a cold one,'' observed the nephew.

``I would not say happily, my friend,'' returned the uncle, with
refined politeness; ``I would not be sure of that.  A good opportunity
for consideration, surrounded by the advantages of solitude, might
influence your destiny to far greater advantage than you influence it
for yourself.  But it is useless to discuss the question.  I am, as
you say, at a disadvantage.  These little instruments of correction,
these gentle aids to the power and honour of families, these slight
favours that might so incommode you, are only to be obtained now by
interest and importunity.  They are sought by so many, and they are
granted (comparatively) to so few!  It used not to be so, but France
in all such things is changed for the worse.  Our not remote
ancestors held the right of life and death over the surrounding
vulgar.  From this room, many such dogs have been taken out to be
hanged; in the next room (my bedroom), one fellow, to our knowledge,
was poniarded on the spot for professing some insolent delicacy
respecting his daughter---\emph{his} daughter?  We have lost many privileges;
a new philosophy has become the mode; and the assertion of our
station, in these days, might (I do not go so far as to say would,
but might) cause us real inconvenience.  All very bad, very bad!''

The Marquis took a gentle little pinch of snuff, and shook his head;
as elegantly despondent as he could becomingly be of a country still
containing himself, that great means of regeneration.

``We have so asserted our station, both in the old time and in the
modern time also,'' said the nephew, gloomily, ``that I believe our
name to be more detested than any name in France.''

``Let us hope so,'' said the uncle.  ``Detestation of the high is the
involuntary homage of the low.''

``There is not,'' pursued the nephew, in his former tone, ``a face I can
look at, in all this country round about us, which looks at me with
any deference on it but the dark deference of fear and slavery.''

``A compliment,'' said the Marquis, ``to the grandeur of the family,
merited by the manner in which the family has sustained its grandeur.
Hah!''  And he took another gentle little pinch of snuff, and lightly
crossed his legs.

But, when his nephew, leaning an elbow on the table, covered his eyes
thoughtfully and dejectedly with his hand, the fine mask looked at him
sideways with a stronger concentration of keenness, closeness, and dislike,
than was comportable with its wearer's assumption of indifference.

``Repression is the only lasting philosophy.  The dark deference of
fear and slavery, my friend,'' observed the Marquis, ``will keep the
dogs obedient to the whip, as long as this roof,'' looking up to it,
``shuts out the sky.''

That might not be so long as the Marquis supposed.  If a picture of
the chateau as it was to be a very few years hence, and of fifty like
it as they too were to be a very few years hence, could have been
shown to him that night, he might have been at a loss to claim his
own from the ghastly, fire-charred, plunder-wrecked rains.  As for
the roof he vaunted, he might have found \emph{that} shutting out the sky
in a new way---to wit, for ever, from the eyes of the bodies into which
its lead was fired, out of the barrels of a hundred thousand muskets.

``Meanwhile,'' said the Marquis, ``I will preserve the honour and repose
of the family, if you will not.  But you must be fatigued.  Shall we
terminate our conference for the night?''

``A moment more.''

``An hour, if you please.''

``Sir,'' said the nephew, ``we have done wrong, and are reaping the
fruits of wrong.''

``\emph{We} have done wrong?'' repeated the Marquis, with an inquiring
smile, and delicately pointing, first to his nephew, then to himself.

``Our family; our honourable family, whose honour is of so much
account to both of us, in such different ways.  Even in my father's
time, we did a world of wrong, injuring every human creature who came
between us and our pleasure, whatever it was.  Why need I speak of my
father's time, when it is equally yours?  Can I separate my father's
twin-brother, joint inheritor, and next successor, from himself?''

``Death has done that!'' said the Marquis.

``And has left me,'' answered the nephew, ``bound to a system that is
frightful to me, responsible for it, but powerless in it; seeking to
execute the last request of my dear mother's lips, and obey the last
look of my dear mother's eyes, which implored me to have mercy and to
redress; and tortured by seeking assistance and power in vain.''

``Seeking them from me, my nephew,'' said the Marquis, touching him on
the breast with his forefinger---they were now standing by the
hearth---``you will for ever seek them in vain, be assured.''

Every fine straight line in the clear whiteness of his face, was
cruelly, craftily, and closely compressed, while he stood looking
quietly at his nephew, with his snuff-box in his hand.  Once again he
touched him on the breast, as though his finger were the fine point
of a small sword, with which, in delicate finesse, he ran him through
the body, and said,

``My friend, I will die, perpetuating the system under which I have lived.''

When he had said it, he took a culminating pinch of snuff, and put
his box in his pocket.

``Better to be a rational creature,'' he added then, after ringing a
small bell on the table, ``and accept your natural destiny.  But you
are lost, Monsieur Charles, I see.''

``This property and France are lost to me,'' said the nephew, sadly;
``I renounce them.''

``Are they both yours to renounce?  France may be, but is the property?
It is scarcely worth mentioning; but, is it yet?''

``I had no intention, in the words I used, to claim it yet.  If it
passed to me from you, to-morrow---''

``Which I have the vanity to hope is not probable.''

``---or twenty years hence---''

``You do me too much honour,'' said the Marquis; ``still, I prefer that
supposition.''

``---I would abandon it, and live otherwise and elsewhere.  It is
little to relinquish.  What is it but a wilderness of misery and ruin!''

``Hah!'' said the Marquis, glancing round the luxurious room.

``To the eye it is fair enough, here; but seen in its integrity, under
the sky, and by the daylight, it is a crumbling tower of waste,
mismanagement, extortion, debt, mortgage, oppression, hunger,
nakedness, and suffering.''

``Hah!'' said the Marquis again, in a well-satisfied manner.

``If it ever becomes mine, it shall be put into some hands better
qualified to free it slowly (if such a thing is possible) from the
weight that drags it down, so that the miserable people who cannot
leave it and who have been long wrung to the last point of endurance,
may, in another generation, suffer less; but it is not for me.
There is a curse on it, and on all this land.''

``And you?'' said the uncle.  ``Forgive my curiosity; do you, under your
new philosophy, graciously intend to live?''

``I must do, to live, what others of my countrymen, even with nobility
at their backs, may have to do some day-work.''

``In England, for example?''

``Yes.  The family honour, sir, is safe from me in this country.  The
family name can suffer from me in no other, for I bear it in no other.''

The ringing of the bell had caused the adjoining bed-chamber to be
lighted.  It now shone brightly, through the door of communication.
The Marquis looked that way, and listened for the retreating step of
his valet.

``England is very attractive to you, seeing how indifferently you have
prospered there,'' he observed then, turning his calm face to his
nephew with a smile.

``I have already said, that for my prospering there, I am sensible I
may be indebted to you, sir.  For the rest, it is my Refuge.''

``They say, those boastful English, that it is the Refuge of many.
You know a compatriot who has found a Refuge there?  A Doctor?''

``Yes.''

``With a daughter?''

``Yes.''

``Yes,'' said the Marquis.  ``You are fatigued.  Good night!''

As he bent his head in his most courtly manner, there was a secrecy
in his smiling face, and he conveyed an air of mystery to those
words, which struck the eyes and ears of his nephew forcibly.  At the
same time, the thin straight lines of the setting of the eyes, and
the thin straight lips, and the markings in the nose, curved with a
sarcasm that looked handsomely diabolic.

``Yes,'' repeated the Marquis.  ``A Doctor with a daughter.  Yes.
So commences the new philosophy!  You are fatigued.  Good night!''

It would have been of as much avail to interrogate any stone face
outside the chateau as to interrogate that face of his.  The nephew
looked at him, in vain, in passing on to the door.

``Good night!'' said the uncle.  ``I look to the pleasure of seeing you
again in the morning.  Good repose!  Light Monsieur my nephew to his
chamber there!---And burn Monsieur my nephew in his bed, if you will,''
he added to himself, before he rang his little bell again, and summoned
his valet to his own bedroom.

The valet come and gone, Monsieur the Marquis walked to and fro in
his loose chamber-robe, to prepare himself gently for sleep, that hot
still night.  Rustling about the room, his softly-slippered feet
making no noise on the floor, he moved like a refined tiger:---looked
like some enchanted marquis of the impenitently wicked sort, in story,
whose periodical change into tiger form was either just going off, or
just coming on.

He moved from end to end of his voluptuous bedroom, looking again at
the scraps of the day's journey that came unbidden into his mind; the
slow toil up the hill at sunset, the setting sun, the descent, the
mill, the prison on the crag, the little village in the hollow, the
peasants at the fountain, and the mender of roads with his blue cap
pointing out the chain under the carriage.  That fountain suggested
the Paris fountain, the little bundle lying on the step, the women
bending over it, and the tall man with his arms up, crying, ``Dead!''

``I am cool now,'' said Monsieur the Marquis, ``and may go to bed.''

So, leaving only one light burning on the large hearth, he let his
thin gauze curtains fall around him, and heard the night break its
silence with a long sigh as he composed himself to sleep.

The stone faces on the outer walls stared blindly at the black night
for three heavy hours; for three heavy hours, the horses in the
stables rattled at their racks, the dogs barked, and the owl made a
noise with very little resemblance in it to the noise conventionally
assigned to the owl by men-poets.  But it is the obstinate custom of
such creatures hardly ever to say what is set down for them.

For three heavy hours, the stone faces of the chateau, lion and
human, stared blindly at the night.  Dead darkness lay on all the
landscape, dead darkness added its own hush to the hushing dust on
all the roads.  The burial-place had got to the pass that its little
heaps of poor grass were undistinguishable from one another; the
figure on the Cross might have come down, for anything that could be
seen of it.  In the village, taxers and taxed were fast asleep.
Dreaming, perhaps, of banquets, as the starved usually do, and of
ease and rest, as the driven slave and the yoked ox may, its lean
inhabitants slept soundly, and were fed and freed.

The fountain in the village flowed unseen and unheard, and the
fountain at the chateau dropped unseen and unheard---both melting
away, like the minutes that were falling from the spring of Time---%
through three dark hours.  Then, the grey water of both began to be
ghostly in the light, and the eyes of the stone faces of the chateau
were opened.

Lighter and lighter, until at last the sun touched the tops of the
still trees, and poured its radiance over the hill.  In the glow,
the water of the chateau fountain seemed to turn to blood, and the
stone faces crimsoned.  The carol of the birds was loud and high,
and, on the weather-beaten sill of the great window of the bed-%
chamber of Monsieur the Marquis, one little bird sang its sweetest
song with all its might.  At this, the nearest stone face seemed
to stare amazed, and, with open mouth and dropped under-jaw, looked
awe-stricken.

Now, the sun was full up, and movement began in the village.
Casement windows opened, crazy doors were unbarred, and people came
forth shivering---chilled, as yet, by the new sweet air.  Then began
the rarely lightened toil of the day among the village population.
Some, to the fountain; some, to the fields; men and women here, to
dig and delve; men and women there, to see to the poor live stock,
and lead the bony cows out, to such pasture as could be found by the
roadside.  In the church and at the Cross, a kneeling figure or two;
attendant on the latter prayers, the led cow, trying for a breakfast
among the weeds at its foot.

The chateau awoke later, as became its quality, but awoke gradually
and surely.  First, the lonely boar-spears and knives of the chase
had been reddened as of old; then, had gleamed trenchant in the
morning sunshine; now, doors and windows were thrown open, horses
in their stables looked round over their shoulders at the light and
freshness pouring in at doorways, leaves sparkled and rustled at
iron-grated windows, dogs pulled hard at their chains, and reared
impatient to be loosed.

All these trivial incidents belonged to the routine of life, and the
return of morning.  Surely, not so the ringing of the great bell of
the chateau, nor the running up and down the stairs; nor the hurried
figures on the terrace; nor the booting and tramping here and there
and everywhere, nor the quick saddling of horses and riding away?

What winds conveyed this hurry to the grizzled mender of roads,
already at work on the hill-top beyond the village, with his day's
dinner (not much to carry) lying in a bundle that it was worth no
crow's while to peck at, on a heap of stones?  Had the birds, carrying
some grains of it to a distance, dropped one over him as they sow
chance seeds?  Whether or no, the mender of roads ran, on the sultry
morning, as if for his life, down the hill, knee-high in dust, and
never stopped till he got to the fountain.

All the people of the village were at the fountain, standing about in
their depressed manner, and whispering low, but showing no other
emotions than grim curiosity and surprise.  The led cows, hastily
brought in and tethered to anything that would hold them, were looking
stupidly on, or lying down chewing the cud of nothing particularly
repaying their trouble, which they had picked up in their interrupted
saunter.  Some of the people of the chateau, and some of those of the
posting-house, and all the taxing authorities, were armed more or less,
and were crowded on the other side of the little street in a
purposeless way, that was highly fraught with nothing.  Already,
the mender of roads had penetrated into the midst of a group of fifty
particular friends, and was smiting himself in the breast with his
blue cap.  What did all this portend, and what portended the swift
hoisting-up of Monsieur Gabelle behind a servant on horseback, and
the conveying away of the said Gabelle (double-laden though the horse
was), at a gallop, like a new version of the German ballad of Leonora?

It portended that there was one stone face too many, up at the chateau.

The Gorgon had surveyed the building again in the night, and had
added the one stone face wanting; the stone face for which it had
waited through about two hundred years.

It lay back on the pillow of Monsieur the Marquis.  It was like a
fine mask, suddenly startled, made angry, and petrified.  Driven home
into the heart of the stone figure attached to it, was a knife.
Round its hilt was a frill of paper, on which was scrawled:

``Drive him fast to his tomb.  This, from Jacques.''



\chapter{Two Promises}


More months, to the number of twelve, had come and gone, and Mr.\ %
Charles Darnay was established in England as a higher teacher of the
French language who was conversant with French literature.  In this
age, he would have been a Professor; in that age, he was a Tutor.
He read with young men who could find any leisure and interest for
the study of a living tongue spoken all over the world, and he
cultivated a taste for its stores of knowledge and fancy.  He could
write of them, besides, in sound English, and render them into sound
English.  Such masters were not at that time easily found; Princes
that had been, and Kings that were to be, were not yet of the Teacher
class, and no ruined nobility had dropped out of Tellson's ledgers,
to turn cooks and carpenters.  As a tutor, whose attainments made the
student's way unusually pleasant and profitable, and as an elegant
translator who brought something to his work besides mere dictionary
knowledge, young Mr.\ Darnay soon became known and encouraged.  He was
well acquainted, more-over, with the circumstances of his country,
and those were of ever-growing interest.  So, with great perseverance
and untiring industry, he prospered.

In London, he had expected neither to walk on pavements of gold, nor
to lie on beds of roses; if he had had any such exalted expectation,
he would not have prospered.  He had expected labour, and he found it,
and did it and made the best of it.  In this, his prosperity consisted.

A certain portion of his time was passed at Cambridge, where he read
with undergraduates as a sort of tolerated smuggler who drove a
contraband trade in European languages, instead of conveying Greek
and Latin through the Custom-house.  The rest of his time he passed
in London.

Now, from the days when it was always summer in Eden, to these days
when it is mostly winter in fallen latitudes, the world of a man has
invariably gone one way---Charles Darnay's way---the way of the love of
a woman.

He had loved Lucie Manette from the hour of his danger.  He had never
heard a sound so sweet and dear as the sound of her compassionate
voice; he had never seen a face so tenderly beautiful, as hers when
it was confronted with his own on the edge of the grave that had been
dug for him.  But, he had not yet spoken to her on the subject;
the assassination at the deserted chateau far away beyond the heaving
water and the long, tong, dusty roads---the solid stone chateau which
had itself become the mere mist of a dream---had been done a year,
and he had never yet, by so much as a single spoken word, disclosed
to her the state of his heart.

That he had his reasons for this, he knew full well.  It was again a
summer day when, lately arrived in London from his college occupation,
he turned into the quiet corner in Soho, bent on seeking an opportunity
of opening his mind to Doctor Manette.  It was the close of the
summer day, and he knew Lucie to be out with Miss Pross.

He found the Doctor reading in his arm-chair at a window.  The energy
which had at once supported him under his old sufferings and aggravated
their sharpness, had been gradually restored to him.  He was now a
very energetic man indeed, with great firmness of purpose, strength
of resolution, and vigour of action.  In his recovered energy he was
sometimes a little fitful and sudden, as he had at first been in the
exercise of his other recovered faculties; but, this had never been
frequently observable, and had grown more and more rare.

He studied much, slept little, sustained a great deal of fatigue with
ease, and was equably cheerful.  To him, now entered Charles Darnay,
at sight of whom he laid aside his book and held out his hand.

``Charles Darnay!  I rejoice to see you.  We have been counting on your
return these three or four days past.  Mr.\ Stryver and Sydney Carton
were both here yesterday, and both made you out to be more than due.''

``I am obliged to them for their interest in the matter,'' he answered,
a little coldly as to them, though very warmly as to the Doctor.
``Miss Manette---''

``Is well,'' said the Doctor, as he stopped short, ``and your return
will delight us all.  She has gone out on some household matters,
but will soon be home.''

``Doctor Manette, I knew she was from home.  I took the opportunity of
her being from home, to beg to speak to you.''

There was a blank silence.

``Yes?'' said the Doctor, with evident constraint.  ``Bring your chair here,
and speak on.''

He complied as to the chair, but appeared to find the speaking on
less easy.

``I have had the happiness, Doctor Manette, of being so intimate
here,'' so he at length began, ``for some year and a half, that I hope
the topic on which I am about to touch may not---''

He was stayed by the Doctor's putting out his hand to stop him.
When he had kept it so a little while, he said, drawing it back:

``Is Lucie the topic?''

``She is.''

``It is hard for me to speak of her at any time.  It is very hard for
me to hear her spoken of in that tone of yours, Charles Darnay.''

``It is a tone of fervent admiration, true homage, and deep love,
Doctor Manette!'' he said deferentially.

There was another blank silence before her father rejoined:

``I believe it.  I do you justice; I believe it.''

His constraint was so manifest, and it was so manifest, too, that it
originated in an unwillingness to approach the subject, that Charles
Darnay hesitated.

``Shall I go on, sir?''

Another blank.

``Yes, go on.''

``You anticipate what I would say, though you cannot know how earnestly
I say it, how earnestly I feel it, without knowing my secret heart,
and the hopes and fears and anxieties with which it has long been
laden.  Dear Doctor Manette, I love your daughter fondly, dearly,
disinterestedly, devotedly.  If ever there were love in the world,
I love her.  You have loved yourself; let your old love speak for me!''

The Doctor sat with his face turned away, and his eyes bent on the
ground.  At the last words, he stretched out his hand again, hurriedly,
and cried:

``Not that, sir!  Let that be!  I adjure you, do not recall that!''

His cry was so like a cry of actual pain, that it rang in Charles
Darnay's ears long after he had ceased.  He motioned with the hand he
had extended, and it seemed to be an appeal to Darnay to pause.
The latter so received it, and remained silent.

``I ask your pardon,'' said the Doctor, in a subdued tone, after some
moments.  ``I do not doubt your loving Lucie; you may be satisfied of it.''

He turned towards him in his chair, but did not look at him, or raise
his eyes.  His chin dropped upon his hand, and his white hair
overshadowed his face:

``Have you spoken to Lucie?''

``No.''

``Nor written?''

``Never.''

``It would be ungenerous to affect not to know that your self-denial
is to be referred to your consideration for her father.  Her father
thanks you.

He offered his hand; but his eyes did not go with it.

``I know,'' said Darnay, respectfully, ``how can I fail to know,
Doctor Manette, I who have seen you together from day to day,
that between you and Miss Manette there is an affection so unusual,
so touching, so belonging to the circumstances in which it has been
nurtured, that it can have few parallels, even in the tenderness
between a father and child.  I know, Doctor Manette---how can I fail
to know---that, mingled with the affection and duty of a daughter who
has become a woman, there is, in her heart, towards you, all the love
and reliance of infancy itself.  I know that, as in her childhood she
had no parent, so she is now devoted to you with all the constancy
and fervour of her present years and character, united to the
trustfulness and attachment of the early days in which you were lost
to her.  I know perfectly well that if you had been restored to her
from the world beyond this life, you could hardly be invested, in her
sight, with a more sacred character than that in which you are always
with her.  I know that when she is clinging to you, the hands of baby,
girl, and woman, all in one, are round your neck.  I know that in
loving you she sees and loves her mother at her own age, sees and
loves you at my age, loves her mother broken-hearted, loves you
through your dreadful trial and in your blessed restoration.  I have
known this, night and day, since I have known you in your home.''

Her father sat silent, with his face bent down.  His breathing was a
little quickened; but he repressed all other signs of agitation.

``Dear Doctor Manette, always knowing this, always seeing her and you
with this hallowed light about you, I have forborne, and forborne,
as long as it was in the nature of man to do it.  I have felt, and do
even now feel, that to bring my love---even mine---between you, is to
touch your history with something not quite so good as itself.
But I love her.  Heaven is my witness that I love her!''

``I believe it,'' answered her father, mournfully.  ``I have thought so
before now.  I believe it.''

``But, do not believe,'' said Darnay, upon whose ear the mournful voice
struck with a reproachful sound, ``that if my fortune were so cast as
that, being one day so happy as to make her my wife, I must at any
time put any separation between her and you, I could or would breathe
a word of what I now say.  Besides that I should know it to be
hopeless, I should know it to be a baseness.  If I had any such
possibility, even at a remote distance of years, harboured in my
thoughts, and hidden in my heart---if it ever had been there---if it
ever could be there---I could not now touch this honoured hand.''

He laid his own upon it as he spoke.

``No, dear Doctor Manette.  Like you, a voluntary exile from France;
like you, driven from it by its distractions, oppressions, and
miseries; like you, striving to live away from it by my own exertions,
and trusting in a happier future; I look only to sharing your fortunes,
sharing your life and home, and being faithful to you to the death.
Not to divide with Lucie her privilege as your child, companion, and
friend; but to come in aid of it, and bind her closer to you, if such
a thing can be.''

His touch still lingered on her father's hand.  Answering the touch
for a moment, but not coldly, her father rested his hands upon the
arms of his chair, and looked up for the first time since the
beginning of the conference.  A struggle was evidently in his face;
a struggle with that occasional look which had a tendency in it to
dark doubt and dread.

``You speak so feelingly and so manfully, Charles Darnay, that I thank
you with all my heart, and will open all my heart---or nearly so.
Have you any reason to believe that Lucie loves you?''

``None.  As yet, none.''

``Is it the immediate object of this confidence, that you may at once
ascertain that, with my knowledge?''

``Not even so.  I might not have the hopefulness to do it for weeks;
I might (mistaken or not mistaken) have that hopefulness to-morrow.''

``Do you seek any guidance from me?''

``I ask none, sir.  But I have thought it possible that you might have
it in your power, if you should deem it right, to give me some.''

``Do you seek any promise from me?''

``I do seek that.''

``What is it?''

``I well understand that, without you, I could have no hope.  I well
understand that, even if Miss Manette held me at this moment in her
innocent heart-do not think I have the presumption to assume so much---%
I could retain no place in it against her love for her father.''

``If that be so, do you see what, on the other hand, is involved in it?''

``I understand equally well, that a word from her father in any suitor's
favour, would outweigh herself and all the world.  For which reason,
Doctor Manette,'' said Darnay, modestly but firmly, ``I would not ask
that word, to save my life.''

``I am sure of it.  Charles Darnay, mysteries arise out of close love,
as well as out of wide division; in the former case, they are subtle
and delicate, and difficult to penetrate.  My daughter Lucie is, in
this one respect, such a mystery to me; I can make no guess at the
state of her heart.''

``May I ask, sir, if you think she is---'' As he hesitated, her father
supplied the rest.

``Is sought by any other suitor?''

``It is what I meant to say.''

Her father considered a little before he answered:

``You have seen Mr.\ Carton here, yourself.  Mr.\ Stryver is here too,
occasionally.  If it be at all, it can only be by one of these.''

``Or both,'' said Darnay.

``I had not thought of both; I should not think either, likely.
You want a promise from me.  Tell me what it is.''

``It is, that if Miss Manette should bring to you at any time, on her
own part, such a confidence as I have ventured to lay before you,
you will bear testimony to what I have said, and to your belief in it.
I hope you may be able to think so well of me, as to urge no influence
against me. I say nothing more of my stake in this; this is what I ask.
The condition on which I ask it, and which you have an undoubted right
to require, I will observe immediately.''

``I give the promise,'' said the Doctor, ``without any condition.
I believe your object to be, purely and truthfully, as you have
stated it.  I believe your intention is to perpetuate, and not to
weaken, the ties between me and my other and far dearer self.  If she
should ever tell me that you are essential to her perfect happiness,
I will give her to you.  If there were---Charles Darnay, if there were---''

The young man had taken his hand gratefully; their hands were joined
as the Doctor spoke:

``---any fancies, any reasons, any apprehensions, anything whatsoever,
new or old, against the man she really loved---the direct responsibility
thereof not lying on his head---they should all be obliterated for her
sake.  She is everything to me; more to me than suffering, more to me
than wrong, more to me---Well!  This is idle talk.''

So strange was the way in which he faded into silence, and so strange
his fixed look when he had ceased to speak, that Darnay felt his own
hand turn cold in the hand that slowly released and dropped it.

``You said something to me,'' said Doctor Manette, breaking into a smile.
``What was it you said to me?''

He was at a loss how to answer, until he remembered having spoken of
a condition.  Relieved as his mind reverted to that, he answered:

``Your confidence in me ought to be returned with full confidence on
my part.  My present name, though but slightly changed from my
mother's, is not, as you will remember, my own.  I wish to tell you
what that is, and why I am in England.''

``Stop!'' said the Doctor of Beauvais.

``I wish it, that I may the better deserve your confidence, and have
no secret from you.''

``Stop!''

For an instant, the Doctor even had his two hands at his ears; for
another instant, even had his two hands laid on Darnay's lips.

``Tell me when I ask you, not now.  If your suit should prosper, if
Lucie should love you, you shall tell me on your marriage morning.
Do you promise?''

``Willingly.

``Give me your hand.  She will be home directly, and it is better she
should not see us together to-night.  Go!  God bless you!''

It was dark when Charles Darnay left him, and it was an hour later
and darker when Lucie came home; she hurried into the room alone---%
for Miss Pross had gone straight up-stairs---and was surprised to find
his reading-chair empty.

``My father!'' she called to him.  ``Father dear!''

Nothing was said in answer, but she heard a low hammering sound in
his bedroom.  Passing lightly across the intermediate room, she
looked in at his door and came running back frightened, crying to
herself, with her blood all chilled, ``What shall I do!  What shall I do!''

Her uncertainty lasted but a moment; she hurried back, and tapped at
his door, and softly called to him.  The noise ceased at the sound of
her voice, and he presently came out to her, and they walked up and
down together for a long time.

She came down from her bed, to look at him in his sleep that night.
He slept heavily, and his tray of shoemaking tools, and his old
unfinished work, were all as usual.



\chapter{A Companion Picture}


``Sydney,'' said Mr.\ Stryver, on that self-same night, or morning, to his
jackal; ``mix another bowl of punch; I have something to say to you.''

Sydney had been working double tides that night, and the night before,
and the night before that, and a good many nights in succession, making
a grand clearance among Mr.\ Stryver's papers before the setting in of
the long vacation.  The clearance was effected at last; the Stryver
arrears were handsomely fetched up; everything was got rid of until
November should come with its fogs atmospheric, and fogs legal, and
bring grist to the mill again.

Sydney was none the livelier and none the soberer for so much application.
It had taken a deal of extra wet-towelling to pull him through the night;
a correspondingly extra quantity of wine had preceded the towelling;
and he was in a very damaged condition, as he now pulled his turban
off and threw it into the basin in which he had steeped it at intervals
for the last six hours.

``Are you mixing that other bowl of punch?'' said Stryver the portly,
with his hands in his waistband, glancing round from the sofa where
he lay on his back.

``I am.''

``Now, look here!  I am going to tell you something that will rather
surprise you, and that perhaps will make you think me not quite as
shrewd as you usually do think me.  I intend to marry.''

``\emph{Do} you?''

``Yes.  And not for money.  What do you say now?''

``I don't feel disposed to say much.  Who is she?''

``Guess.''

``Do I know her?''

``Guess.''

``I am not going to guess, at five o'clock in the morning, with my
brains frying and sputtering in my head. if you want me to guess, you
must ask me to dinner.''

``Well then, I'll tell you,'' said Stryver, coming slowly into a sitting
posture.  ``Sydney, I rather despair of making myself intelligible to you,
because you are such an insensible dog.''

``And you,'' returned Sydney, busy concocting the punch, ``are such a
sensitive and poetical spirit---''

``Come!'' rejoined Stryver, laughing boastfully, ``though I don't prefer
any claim to being the soul of Romance (for I hope I know better),
still I am a tenderer sort of fellow than \emph{you}.''

``You are a luckier, if you mean that.''

``I don't mean that.  I mean I am a man of more---more---''

``Say gallantry, while you are about it,'' suggested Carton.

``Well!  I'll say gallantry.  My meaning is that I am a man,'' said
Stryver, inflating himself at his friend as he made the punch,
``who cares more to be agreeable, who takes more pains to be agreeable,
who knows better how to be agreeable, in a woman's society, than you do.''

``Go on,'' said Sydney Carton.

``No; but before I go on,'' said Stryver, shaking his head in his bullying
way, I'll have this out with you.  You've been at Doctor Manette's
house as much as I have, or more than I have.  Why, I have been ashamed
of your moroseness there!  Your manners have been of that silent and
sullen and hangdog kind, that, upon my life and soul, I have been
ashamed of you, Sydney!``

``It should be very beneficial to a man in your practice at the bar,
to be ashamed of anything,'' returned Sydney; ``you ought to be much
obliged to me.''

``You shall not get off in that way,'' rejoined Stryver, shouldering the
rejoinder at him; ``no, Sydney, it's my duty to tell you---and I tell you
to your face to do you good---that you are a devilish ill-conditioned
fellow in that sort of society.  You are a disagreeable fellow.''

Sydney drank a bumper of the punch he had made, and laughed.

``Look at me!'' said Stryver, squaring himself; ``I have less need to
make myself agreeable than you have, being more independent in
circumstances.  Why do I do it?''

``I never saw you do it yet,'' muttered Carton.

``I do it because it's politic; I do it on principle.  And look at me!
I get on.''

``You don't get on with your account of your matrimonial intentions,''
answered Carton, with a careless air; ``I wish you would keep to that.
As to me---will you never understand that I am incorrigible?''

He asked the question with some appearance of scorn.

``You have no business to be incorrigible,'' was his friend's answer,
delivered in no very soothing tone.

``I have no business to be, at all, that I know of,'' said Sydney Carton.
``Who is the lady?''

``Now, don't let my announcement of the name make you uncomfortable,
Sydney,'' said Mr.\ Stryver, preparing him with ostentatious
friendliness for the disclosure he was about to make, ``because I know
you don't mean half you say; and if you meant it all, it would be of
no importance.  I make this little preface, because you once mentioned
the young lady to me in slighting terms.''

``I did?''

``Certainly; and in these chambers.''

Sydney Carton looked at his punch and looked at his complacent friend;
drank his punch and looked at his complacent friend.

``You made mention of the young lady as a golden-haired doll.  The young
lady is Miss Manette.  If you had been a fellow of any sensitiveness or
delicacy of feeling in that kind of way, Sydney, I might have been a
little resentful of your employing such a designation; but you are not.
You want that sense altogether; therefore I am no more annoyed when I
think of the expression, than I should be annoyed by a man's opinion of
a picture of mine, who had no eye for pictures:  or of a piece of music
of mine, who had no ear for music.''

Sydney Carton drank the punch at a great rate; drank it by bumpers,
looking at his friend.

``Now you know all about it, Syd,'' said Mr.\ Stryver.  ``I don't care
about fortune:  she is a charming creature, and I have made up my mind
to please myself:  on the whole, I think I can afford to please myself.
She will have in me a man already pretty well off, and a rapidly
rising man, and a man of some distinction:  it is a piece of good fortune
for her, but she is worthy of good fortune.  Are you astonished?''

Carton, still drinking the punch, rejoined, ``Why should I be astonished?''

``You approve?''

Carton, still drinking the punch, rejoined, ``Why should I not approve?''

``Well!'' said his friend Stryver, ``you take it more easily than I
fancied you would, and are less mercenary on my behalf than I thought
you would be; though, to be sure, you know well enough by this time
that your ancient chum is a man of a pretty strong will.  Yes, Sydney,
I have had enough of this style of life, with no other as a change
from it; I feel that it is a pleasant thing for a man to have a home
when he feels inclined to go to it (when he doesn't, he can stay away),
and I feel that Miss Manette will tell well in any station, and will
always do me credit.  So I have made up my mind.  And now, Sydney,
old boy, I want to say a word to \emph{you} about \emph{your} prospects.  You are
in a bad way, you know; you really are in a bad way.  You don't know
the value of money, you live hard, you'll knock up one of these days,
and be ill and poor; you really ought to think about a nurse.''

The prosperous patronage with which he said it, made him look twice
as big as he was, and four times as offensive.

``Now, let me recommend you,'' pursued Stryver, ``to look it in the face.
I have looked it in the face, in my different way; look it in the face,
you, in your different way.  Marry.  Provide somebody to take care of you.
Never mind your having no enjoyment of women's society, nor understanding
of it, nor tact for it.  Find out somebody.  Find out some respectable
woman with a little property---somebody in the landlady way, or
lodging-letting way---and marry her, against a rainy day.  That's the
kind of thing for \emph{you}.  Now think of it, Sydney.''

``I'll think of it,'' said Sydney.



\chapter{The Fellow of Delicacy}


Mr.\ Stryver having made up his mind to that magnanimous bestowal of
good fortune on the Doctor's daughter, resolved to make her happiness
known to her before he left town for the Long Vacation.  After some
mental debating of the point, he came to the conclusion that it would
be as well to get all the preliminaries done with, and they could
then arrange at their leisure whether he should give her his hand a
week or two before Michaelmas Term, or in the little Christmas vacation
between it and Hilary.

As to the strength of his case, he had not a doubt about it, but
clearly saw his way to the verdict.  Argued with the jury on substantial
worldly grounds---the only grounds ever worth taking into account---%
it was a plain case, and had not a weak spot in it.  He called himself
for the plaintiff, there was no getting over his evidence, the counsel
for the defendant threw up his brief, and the jury did not even turn
to consider.  After trying it, Stryver, C. J., was satisfied that no
plainer case could be.

Accordingly, Mr.\ Stryver inaugurated the Long Vacation with a
formal proposal to take Miss Manette to Vauxhall Gardens; that failing,
to Ranelagh; that unaccountably failing too, it behoved him to present
himself in Soho, and there declare his noble mind.

Towards Soho, therefore, Mr.\ Stryver shouldered his way from the
Temple, while the bloom of the Long Vacation's infancy was still upon
it.  Anybody who had seen him projecting himself into Soho while he
was yet on Saint Dunstan's side of Temple Bar, bursting in his
full-blown way along the pavement, to the jostlement of all weaker
people, might have seen how safe and strong he was.

His way taking him past Tellson's, and he both banking at Tellson's
and knowing Mr.\ Lorry as the intimate friend of the Manettes, it
entered Mr.\ Stryver's mind to enter the bank, and reveal to Mr.\ Lorry
the brightness of the Soho horizon.  So, he pushed open the door with
the weak rattle in its throat, stumbled down the two steps, got past
the two ancient cashiers, and shouldered himself into the musty back
closet where Mr.\ Lorry sat at great books ruled for figures, with
perpendicular iron bars to his window as if that were ruled for
figures too, and everything under the clouds were a sum.

``Halloa!'' said Mr.\ Stryver.  ``How do you do?  I hope you are well!''

It was Stryver's grand peculiarity that he always seemed too big for
any place, or space.  He was so much too big for Tellson's, that
old clerks in distant corners looked up with looks of remonstrance,
as though he squeezed them against the wall.  The House itself,
magnificently reading the paper quite in the far-off perspective,
lowered displeased, as if the Stryver head had been butted into its
responsible waistcoat.

The discreet Mr.\ Lorry said, in a sample tone of the voice he would
recommend under the circumstances, ``How do you do, Mr.\ Stryver?
How do you do, sir?'' and shook hands.  There was a peculiarity in his
manner of shaking hands, always to be seen in any clerk at Tellson's
who shook hands with a customer when the House pervaded the air.
He shook in a self-abnegating way, as one who shook for Tellson and Co.

``Can I do anything for you, Mr.\ Stryver?'' asked Mr.\ Lorry, in his
business character.

``Why, no, thank you; this is a private visit to yourself, Mr.\ Lorry;
I have come for a private word.''

``Oh indeed!'' said Mr.\ Lorry, bending down his ear, while his eye
strayed to the House afar off.

``I am going,'' said Mr.\ Stryver, leaning his arms confidentially on the
desk:  whereupon, although it was a large double one, there appeared to
be not half desk enough for him:  ``I am going to make an offer of myself
in marriage to your agreeable little friend, Miss Manette, Mr.\ Lorry.''

``Oh dear me!'' cried Mr.\ Lorry, rubbing his chin, and looking at his
visitor dubiously.

``Oh dear me, sir?'' repeated Stryver, drawing back.  ``Oh dear you, sir?
What may your meaning be, Mr.\ Lorry?''

``My meaning,'' answered the man of business, ``is, of course, friendly
and appreciative, and that it does you the greatest credit, and---%
in short, my meaning is everything you could desire.  But---really, you
know, Mr.\ Stryver---'' Mr.\ Lorry paused, and shook his head at him in
the oddest manner, as if he were compelled against his will to add,
internally, ``you know there really is so much too much of you!''

``Well!'' said Stryver, slapping the desk with his contentious hand,
opening his eyes wider, and taking a long breath, ``if I understand
you, Mr.\ Lorry, I'll be hanged!''

Mr.\ Lorry adjusted his little wig at both ears as a means towards
that end, and bit the feather of a pen.

``D---n it all, sir!'' said Stryver, staring at him, ``am I not eligible?''

``Oh dear yes!  Yes.  Oh yes, you're eligible!'' said Mr.\ Lorry.  ``If you
say eligible, you are eligible.''

``Am I not prosperous?'' asked Stryver.

``Oh! if you come to prosperous, you are prosperous,'' said  Mr.\ Lorry.

``And advancing?''

``If you come to advancing you know,'' said Mr.\ Lorry, delighted to be
able to make another admission, ``nobody can doubt that.''

``Then what on earth is your meaning, Mr.\ Lorry?'' demanded Stryver,
perceptibly crestfallen.

``Well!  I---Were you going there now?'' asked Mr.\ Lorry.

``Straight!'' said Stryver, with a plump of his fist on the desk.

``Then I think I wouldn't, if I was you.''

``Why?'' said Stryver.  ``Now, I'll put you in a corner,'' forensically
shaking a forefinger at him.  ``You are a man of business and bound
to have a reason.  State your reason.  Why wouldn't you go?''

``Because,'' said Mr.\ Lorry, ``I wouldn't go on such an object without
having some cause to believe that I should succeed.''

``D---n \emph{me}!'' cried Stryver, ``but this beats everything.''

Mr.\ Lorry glanced at the distant House, and glanced at the angry Stryver.

``Here's a man of business---a man of years---a man of experience---%
\emph{in} a Bank,'' said Stryver; ``and having summed up three leading reasons
for complete success, he says there's no reason at all!  Says it with
his head on!''  Mr.\ Stryver remarked upon the peculiarity as if it would
have been infinitely less remarkable if he had said it with his head off.

``When I speak of success, I speak of success with the young lady; and
when I speak of causes and reasons to make success probable, I speak
of causes and reasons that will tell as such with the young lady.
The young lady, my good sir,'' said Mr.\ Lorry, mildly tapping the
Stryver arm, ``the young lady.  The young lady goes before all.''

``Then you mean to tell me, Mr.\ Lorry,'' said Stryver, squaring his
elbows, ``that it is your deliberate opinion that the young lady at
present in question is a mincing Fool?''

``Not exactly so.  I mean to tell you, Mr.\ Stryver,'' said Mr.\ Lorry,
reddening, ``that I will hear no disrespectful word of that young lady
from any lips; and that if I knew any man---which I hope I do not---%
whose taste was so coarse, and whose temper was so overbearing,
that he could not restrain himself from speaking disrespectfully of
that young lady at this desk, not even Tellson's should prevent my
giving him a piece of my mind.''

The necessity of being angry in a suppressed tone had put Mr.\ Stryver's
blood-vessels into a dangerous state when it was his turn to be angry;
Mr.\ Lorry's veins, methodical as their courses could usually be,
were in no better state now it was his turn.

``That is what I mean to tell you, sir,'' said Mr.\ Lorry.
``Pray let there be no mistake about it.''

Mr.\ Stryver sucked the end of a ruler for a little while, and then
stood hitting a tune out of his teeth with it, which probably gave
him the toothache.  He broke the awkward silence by saying:

``This is something new to me, Mr.\ Lorry.  You deliberately advise
me not to go up to Soho and offer myself---\emph{my}self, Stryver of
the King's Bench bar?''

``Do you ask me for my advice, Mr.\ Stryver?''

``Yes, I do.''

``Very good.  Then I give it, and you have repeated it correctly.''

``And all I can say of it is,'' laughed Stryver with a vexed laugh,
``that this---ha, ha!---beats everything past, present, and to come.''

``Now understand me,'' pursued Mr.\ Lorry.  ``As a man of business, I
am not justified in saying anything about this matter, for, as a man
of business, I know nothing of it.  But, as an old fellow, who has
carried Miss Manette in his arms, who is the trusted friend of
Miss Manette and of her father too, and who has a great affection for
them both, I have spoken.  The confidence is not of my seeking,
recollect.  Now, you think I may not be right?''

``Not I!'' said Stryver, whistling.  ``I can't undertake to find third
parties in common sense; I can only find it for myself.  I suppose
sense in certain quarters; you suppose mincing bread-and-butter
nonsense.  It's new to me, but you are right, I dare say.''

``What I suppose, Mr.\ Stryver, I claim to characterise for myself---And
understand me, sir,'' said Mr.\ Lorry, quickly flushing again,
``I will not---not even at Tellson's---have it characterised for me by any
gentleman breathing.''

``There!  I beg your pardon!'' said Stryver.

``Granted.  Thank you.  Well, Mr.\ Stryver, I was about to say:---it
might be painful to you to find yourself mistaken, it might be painful
to Doctor Manette to have the task of being explicit with you, it
might be very painful to Miss Manette to have the task of being
explicit with you.  You know the terms upon which I have the honour
and happiness to stand with the family.  If you please, committing you
in no way, representing you in no way, I will undertake to correct my
advice by the exercise of a little new observation and judgment expressly
brought to bear upon it.  If you should then be dissatisfied with it,
you can but test its soundness for yourself; if, on the other hand,
you should be satisfied with it, and it should be what it now is,
it may spare all sides what is best spared.  What do you say?''

``How long would you keep me in town?''

``Oh!  It is only a question of a few hours.  I could go to Soho in the
evening, and come to your chambers afterwards.''

``Then I say yes,'' said Stryver:  ``I won't go up there now, I am not
so hot upon it as that comes to; I say yes, and I shall expect you
to look in to-night.  Good morning.''

Then Mr.\ Stryver turned and burst out of the Bank, causing such a
concussion of air on his passage through, that to stand up against it
bowing behind the two counters, required the utmost remaining strength
of the two ancient clerks.  Those venerable and feeble persons were
always seen by the public in the act of bowing, and were popularly
believed, when they had bowed a customer out, still to keep on bowing
in the empty office until they bowed another customer in.

The barrister was keen enough to divine that the banker would not
have gone so far in his expression of opinion on any less solid
ground than moral certainty.  Unprepared as he was for the large pill
he had to swallow, he got it down.  ``And now,'' said Mr.\ Stryver,
shaking his forensic forefinger at the Temple in general, when it
was down, ``my way out of this, is, to put you all in the wrong.''

It was a bit of the art of an Old Bailey tactician, in which he
found great relief.  ``You shall not put me in the wrong, young lady,''
said Mr.\ Stryver; ``I'll do that for you.''

Accordingly, when Mr.\ Lorry called that night as late as ten o'clock,
Mr.\ Stryver, among a quantity of books and papers littered out for
the purpose, seemed to have nothing less on his mind than the subject
of the morning.  He even showed surprise when he saw Mr.\ Lorry, and
was altogether in an absent and preoccupied state.

``Well!'' said that good-natured emissary, after a full half-hour of
bootless attempts to bring him round to the question.  ``I have
been to Soho.''

``To Soho?'' repeated Mr.\ Stryver, coldly.  ``Oh, to be sure!
What am I thinking of!''

``And I have no doubt,'' said Mr.\ Lorry, ``that I was right in the
conversation we had.  My opinion is confirmed, and I reiterate my advice.''

``I assure you,'' returned Mr.\ Stryver, in the friendliest way, ``that I
am sorry for it on your account, and sorry for it on the poor father's
account.  I know this must always be a sore subject with the family;
let us say no more about it.''

``I don't understand you,'' said Mr.\ Lorry.

``I dare say not,'' rejoined Stryver, nodding his head in a smoothing
and final way; ``no matter, no matter.''

``But it does matter,'' Mr.\ Lorry urged.

``No it doesn't; I assure you it doesn't.  Having supposed that there
was sense where there is no sense, and a laudable ambition where there
is not a laudable ambition, I am well out of my mistake, and no harm
is done.  Young women have committed similar follies often before,
and have repented them in poverty and obscurity often before.  In an
unselfish aspect, I am sorry that the thing is dropped, because it
would have been a bad thing for me in a worldly point of view;
in a selfish aspect, I am glad that the thing has dropped, because it
would have been a bad thing for me in a worldly point of view---%
it is hardly necessary to say I could have gained nothing by it.
There is no harm at all done.  I have not proposed to the young lady,
and, between ourselves, I am by no means certain, on reflection,
that I ever should have committed myself to that extent.  Mr.\ Lorry,
you cannot control the mincing vanities and giddinesses of
empty-headed girls; you must not expect to do it, or you will always
be disappointed.  Now, pray say no more about it.  I tell you,
I regret it on account of others, but I am satisfied on my own account.
And I am really very much obliged to you for allowing me to sound you,
and for giving me your advice; you know the young lady better
than I do; you were right, it never would have done.''

Mr.\ Lorry was so taken aback, that he looked quite stupidly at
Mr.\ Stryver shouldering him towards the door, with an appearance of
showering generosity, forbearance, and goodwill, on his erring head.
``Make the best of it, my dear sir,'' said Stryver; ``say no more
about it; thank you again for allowing me to sound you; good night!''

Mr.\ Lorry was out in the night, before he knew where he was.
Mr.\ Stryver was lying back on his sofa, winking at his ceiling.



\chapter{The Fellow of No Delicacy}


If Sydney Carton ever shone anywhere, he certainly never shone in the
house of Doctor Manette.  He had been there often, during a whole year,
and had always been the same moody and morose lounger there.  When he
cared to talk, he talked well; but, the cloud of caring for nothing,
which overshadowed him with such a fatal darkness, was very rarely
pierced by the light within him.

And yet he did care something for the streets that environed that house,
and for the senseless stones that made their pavements.  Many a night
he vaguely and unhappily wandered there, when wine had brought
no transitory gladness to him; many a dreary daybreak revealed his
solitary figure lingering there, and still lingering there when the first
beams of the sun brought into strong relief, removed beauties of
architecture in spires of churches and lofty buildings, as perhaps
the quiet time brought some sense of better things, else forgotten
and unattainable, into his mind.  Of late, the neglected bed in the
Temple Court had known him more scantily than ever; and often when he
had thrown himself upon it no longer than a few minutes, he had got up
again, and haunted that neighbourhood.

On a day in August, when Mr.\ Stryver (after notifying to his jackal
that ``he had thought better of that marrying matter'') had carried his
delicacy into Devonshire, and when the sight and scent of flowers in
the City streets had some waifs of goodness in them for the worst,
of health for the sickliest, and of youth for the oldest, Sydney's feet
still trod those stones.  From being irresolute and purposeless,
his feet became animated by an intention, and, in the working out of
that intention, they took him to the Doctor's door.

He was shown up-stairs, and found Lucie at her work, alone.  She had
never been quite at her ease with him, and received him with some
little  embarrassment as he seated himself near her table.  But,
looking up at his face in the interchange of the first few
common-places, she observed a change in it.

``I fear you are not well, Mr.\ Carton!''

``No.  But the life I lead, Miss Manette, is not conducive to health.
What is to be expected of, or by, such profligates?''

``Is it not---forgive me; I have begun the question on my lips---a pity
to live no better life?''

``God knows it is a shame!''

``Then why not change it?''

Looking gently at him again, she was surprised and saddened to see
that there were tears in his eyes.  There were tears in his voice too,
as he answered:

``It is too late for that.  I shall never be better than I am.
I shall sink lower, and be worse.''

He leaned an elbow on her table, and covered his eyes with his hand.
The table trembled in the silence that followed.

She had never seen him softened, and was much distressed.  He knew
her to be so, without looking at her, and said:

``Pray forgive me, Miss Manette.  I break down before the knowledge
of what I want to say to you.  Will you hear me?''

``If it will do you any good, Mr.\ Carton, if it would make you happier,
it would make me very glad!''

``God bless you for your sweet compassion!''

He unshaded his face after a little while, and spoke steadily.

``Don't be afraid to hear me.  Don't shrink from anything I say.
I am like one who died young.  All my life might have been.''

``No, Mr.\ Carton.  I am sure that the best part of it might still be;
I am sure that you might be much, much worthier of yourself.''

``Say of you, Miss Manette, and although I know better---although
in the mystery of my own wretched heart I know better---I shall
never forget it!''

She was pale and trembling.  He came to her relief with a fixed
despair of himself which made the interview unlike any other
that could have been holden.

``If it had been possible, Miss Manette, that you could have returned
the love of the man you see before yourself---flung away, wasted,
drunken, poor creature of misuse as you know him to be---he would have
been conscious this day and hour, in spite of his happiness, that he
would bring you to misery, bring you to sorrow and repentance, blight
you, disgrace you, pull you down with him.  I know very well that you
can have no tenderness for me; I ask for none; I am even thankful
that it cannot be.''

``Without it, can I not save you, Mr.\ Carton?  Can I not recall you---%
forgive me again!---to a better course?  Can I in no way repay your
confidence?  I know this is a confidence,'' she modestly said, after a
little hesitation, and in earnest tears, ``I know you would say this to
no one else.  Can I turn it to no good account for yourself, Mr.\ Carton?''

He shook his head.

``To none.  No, Miss Manette, to none.  If you will hear me through a
very little more, all you can ever do for me is done.  I wish you to
know that you have been the last dream of my soul.  In my degradation
I have not been so degraded but that the sight of you with your father,
and of this home made such a home by you, has stirred old shadows that
I thought had died out of me.  Since I knew you, I have been troubled
by a remorse that I thought would never reproach me again, and have
heard whispers from old voices impelling me upward, that I thought were
silent for ever.  I have had unformed ideas of striving afresh, beginning
anew, shaking off sloth and sensuality, and fighting out the abandoned
fight.  A dream, all a dream, that ends in nothing, and leaves the
sleeper where he lay down, but I wish you to know that you inspired it.''

``Will nothing of it remain?  O Mr.\ Carton, think again!  Try again!''

``No, Miss Manette; all through it, I have known myself to be quite
undeserving.  And yet I have had the weakness, and have still the
weakness, to wish you to know with what a sudden mastery you kindled me,
heap of ashes that I am, into fire---a fire, however, inseparable
in its nature from myself, quickening nothing, lighting nothing,
doing no service, idly burning away.''

``Since it is my misfortune, Mr.\ Carton, to have made you more unhappy
than you were before you knew me---''

``Don't say that, Miss Manette, for you would have reclaimed me,
if anything could.  You will not be the cause of my becoming worse.''

``Since the state of your mind that you describe, is, at all events,
attributable to some influence of mine---this is what I mean,
if I can make it plain---can I use no influence to serve you?
Have I no power for good, with you, at all?''

``The utmost good that I am capable of now, Miss Manette, I have come
here to realise.  Let me carry through the rest of my misdirected life,
the remembrance that I opened my heart to you, last of all the world;
and that there was something left in me at this time which you could
deplore and pity.''

``Which I entreated you to believe, again and again, most fervently,
with all my heart, was capable of better things, Mr.\ Carton!''

``Entreat me to believe it no more, Miss Manette.  I have proved myself,
and I know better.  I distress you; I draw fast to an end.  Will you let
me believe, when I recall this day, that the last confidence of my life
was reposed in your pure and innocent breast, and that it lies there
alone, and will be shared by no one?''

``If that will be a consolation to you, yes.''

``Not even by the dearest one ever to be known to you?''

``Mr.\ Carton,'' she answered, after an agitated pause, ``the secret is
yours, not mine; and I promise to respect it.''

``Thank you.  And again, God bless you.''

He put her hand to his lips, and moved towards the door.

``Be under no apprehension, Miss Manette, of my ever resuming this
conversation by so much as a passing word.  I will never refer to it
again.  If I were dead, that could not be surer than it is henceforth.
In the hour of my death, I shall hold sacred the one good remembrance---%
and shall thank and bless you for it---that my last avowal of myself was
made to you, and that my name, and faults, and miseries were gently
carried in your heart.  May it otherwise be light and happy!''

He was so unlike what he had ever shown himself to be, and it was
so sad to think how much he had thrown away, and how much he every
day kept down and perverted, that Lucie Manette wept mournfully for
him as he stood looking back at her.

``Be comforted!'' he said, ``I am not worth such feeling, Miss Manette.
An hour or two hence, and the low companions and low habits that I scorn
but yield to, will render me less worth such tears as those, than any
wretch who creeps along the streets.  Be comforted!  But, within myself,
I shall always be, towards you, what I am now, though outwardly I shall
be what you have heretofore seen me.  The last supplication but one
I make to you, is, that you will believe this of me.''

``I will, Mr.\ Carton.''

``My last supplication of all, is this; and with it, I will relieve
you of a visitor with whom I well know you have nothing in unison,
and between whom and you there is an impassable space.  It is useless
to say it, I know, but it rises out of my soul.  For you, and for any
dear to you, I would do anything.  If my career were of that better
kind that there was any opportunity or capacity of sacrifice in it,
I would embrace any sacrifice for you and for those dear to you.
Try to hold me in your mind, at some quiet times, as ardent and sincere
in this one thing.  The time will come, the time will not be long
in coming, when new ties will be formed about you---ties that will bind
you yet more tenderly and strongly to the home you so adorn---the dearest
ties that will ever grace and gladden you.  O Miss Manette, when the
little picture of a happy father's face looks up in yours, when you
see your own bright beauty springing up anew at your feet, think
now and then that there is a man who would give his life, to keep
a life you love beside you!''

He said, ``Farewell!'' said a last ``God bless you!'' and left her.



\chapter{The Honest Tradesman}


To the eyes of Mr.\ Jeremiah Cruncher, sitting on his stool in
Fleet-street with his grisly urchin beside him, a vast number and
variety of objects in movement were every day presented.  Who could
sit upon anything in Fleet-street during the busy hours of the day,
and not be dazed and deafened by two immense processions, one ever
tending westward with the sun, the other ever tending eastward
from the sun, both ever tending to the plains beyond the range of red
and purple where the sun goes down!

With his straw in his mouth, Mr.\ Cruncher sat watching the two streams,
like the heathen rustic who has for several centuries been on duty
watching one stream---saving that Jerry had no expectation of their
ever running dry.  Nor would it have been an expectation of a hopeful
kind, since a small part of his income was derived from the pilotage
of timid women (mostly of a full habit and past the middle term of life)
from Tellson's side of the tides to the opposite shore.  Brief as such
companionship was in every separate instance, Mr.\ Cruncher never
failed to become so interested in the lady as to express a strong desire
to have the honour of drinking her very good health.  And it was from
the gifts bestowed upon him towards the execution of this benevolent
purpose, that he recruited his finances, as just now observed.

Time was, when a poet sat upon a stool in a public place, and mused
in the sight of men.  Mr.\ Cruncher, sitting on a stool in a public place,
but not being a poet, mused as little as possible, and looked about him.

It fell out that he was thus engaged in a season when crowds were few,
and belated women few, and when his affairs in general were so
unprosperous as to awaken a strong suspicion in his breast that
Mrs.\ Cruncher must have been ``flopping'' in some pointed manner, when
an unusual concourse pouring down Fleet-street westward, attracted his
attention.  Looking that way, Mr.\ Cruncher made out that some kind of
funeral was coming along, and that there was popular objection to this
funeral, which engendered uproar.

``Young Jerry,'' said Mr.\ Cruncher, turning to his offspring,
``it's a buryin'.''

``Hooroar, father!'' cried Young Jerry.

The young gentleman uttered this exultant sound with mysterious
significance.  The elder gentleman took the cry so ill, that he
watched his opportunity, and smote the young gentleman on the ear.

``What d'ye mean?  What are you hooroaring at?  What do you want to
conwey to your own father, you young Rip?  This boy is a getting
too many for \emph{me}!'' said Mr.\ Cruncher, surveying him.  ``Him and
his hooroars!  Don't let me hear no more of you, or you shall feel
some more of me.  D'ye hear?''

``I warn't doing no harm,'' Young Jerry protested, rubbing his cheek.

``Drop it then,'' said Mr.\ Cruncher; ``I won't have none of \emph{your}
no harms.  Get a top of that there seat, and look at the crowd.''

His son obeyed, and the crowd approached; they were bawling and hissing
round a dingy hearse and dingy mourning coach, in which mourning coach
there was only one mourner, dressed in the dingy trappings that were
considered essential to the dignity of the position.  The position
appeared by no means to please him, however, with an increasing rabble
surrounding the coach, deriding him, making grimaces at him,
and incessantly groaning and calling out:  ``Yah!  Spies!  Tst!  Yaha!
Spies!'' with many compliments too numerous and forcible to repeat.

Funerals had at all times a remarkable attraction for Mr.\ Cruncher;
he always pricked up his senses, and became excited, when a funeral
passed Tellson's.  Naturally, therefore, a funeral with this uncommon
attendance excited him greatly, and he asked of the first man who ran
against him:

``What is it, brother?  What's it about?''

``\emph{I} don't know,'' said the man.  ``Spies!  Yaha!  Tst!  Spies!''

He asked another man.  ``Who is it?''

``\emph{I} don't know,'' returned the man, clapping his hands to his mouth
nevertheless, and vociferating in a surprising heat and with the
greatest ardour, ``Spies!  Yaha!  Tst, tst!  Spi---ies!''

At length, a person better informed on the merits of the case,
tumbled against him, and from this person he learned that the funeral
was the funeral of one Roger Cly.

``Was He a spy?'' asked Mr.\ Cruncher.

``Old Bailey spy,'' returned his informant.  ``Yaha!  Tst!  Yah!
Old Bailey Spi---i---ies!''

``Why, to be sure!'' exclaimed Jerry, recalling the Trial at which he
had assisted.  ``I've seen him.  Dead, is he?''

``Dead as mutton,'' returned the other, ``and can't be too dead.
Have 'em out, there!  Spies!  Pull 'em out, there!  Spies!''

The idea was so acceptable in the prevalent absence of any idea,
that the crowd caught it up with eagerness, and loudly repeating the
suggestion to have 'em out, and to pull 'em out, mobbed the two vehicles
so closely that they came to a stop.  On the crowd's opening the coach
doors, the one mourner scuffled out of himself and was in their hands
for a moment; but he was so alert, and made such good use of his time,
that in another moment he was scouring away up a bye-street, after
shedding his cloak, hat, long hatband, white pocket-handkerchief,
and other symbolical tears.

These, the people tore to pieces and scattered far and wide with
great enjoyment, while the tradesmen hurriedly shut up their shops;
for a crowd in those times stopped at nothing, and was a monster
much dreaded.  They had already got the length of opening the hearse
to take the coffin out, when some brighter genius proposed instead,
its being escorted to its destination amidst general rejoicing.
Practical suggestions being much needed, this suggestion, too, was
received with acclamation, and the coach was immediately filled with
eight inside and a dozen out, while as many people got on the roof of
the hearse as could by any exercise of ingenuity stick upon it.
Among the first of these volunteers was Jerry Cruncher himself, who
modestly concealed his spiky head from the observation of Tellson's,
in the further corner of the mourning coach.

The officiating undertakers made some protest against these changes
in the ceremonies; but, the river being alarmingly near, and several
voices remarking on the efficacy of cold immersion in bringing
refractory members of the profession to reason, the protest was faint
and brief.  The remodelled procession started, with a chimney-sweep
driving the hearse---advised by the regular driver, who was perched
beside him, under close inspection, for the purpose---and with a pieman,
also attended by his cabinet minister, driving the mourning coach.
A bear-leader, a popular street character of the time, was impressed
as an additional ornament, before the cavalcade had gone far down
the Strand; and his bear, who was black and very mangy, gave quite
an Undertaking air to that part of the procession in which he walked.

Thus, with beer-drinking, pipe-smoking, song-roaring, and infinite
caricaturing of woe, the disorderly procession went its way, recruiting
at every step, and all the shops shutting up before it.  Its destination
was the old church of Saint Pancras, far off in the fields.  It got
there in course of time; insisted on pouring into the burial-ground;
finally, accomplished the interment of the deceased Roger Cly in
its own way, and highly to its own satisfaction.

The dead man disposed of, and the crowd being under the necessity of
providing some other entertainment for itself, another brighter genius
(or perhaps the same) conceived the humour of impeaching casual
passers-by, as Old Bailey spies, and wreaking vengeance on them.
Chase was given to some scores of inoffensive persons who had never
been near the Old Bailey in their lives, in the realisation of this
fancy, and they were roughly hustled and maltreated.  The transition
to the sport of window-breaking, and thence to the plundering of
public-houses, was easy and natural.  At last, after several hours,
when sundry summer-houses had been pulled down, and some area-railings
had been torn up, to arm the more belligerent spirits, a rumour got
about that the Guards were coming.  Before this rumour, the crowd
gradually melted away, and perhaps the Guards came, and perhaps they
never came, and this was the usual progress of a mob.

Mr.\ Cruncher did not assist at the closing sports, but had remained
behind in the churchyard, to confer and condole with the undertakers.
The place had a soothing influence on him.  He procured a pipe from a
neighbouring public-house, and smoked it, looking in at the railings
and maturely considering the spot.

``Jerry,'' said Mr.\ Cruncher, apostrophising himself in his usual way,
``you see that there Cly that day, and you see with your own eyes that
he was a young 'un and a straight made 'un.''

Having smoked his pipe out, and ruminated a little longer, he turned
himself about, that he might appear, before the hour of closing, on his
station at Tellson's.  Whether his meditations on mortality had touched
his liver, or whether his general health had been previously at all
amiss, or whether he desired to show a little attention to an eminent
man, is not so much to the purpose, as that he made a short call upon
his medical adviser---a distinguished surgeon---on his way back.

Young Jerry relieved his father with dutiful interest, and reported No
job in his absence.  The bank closed, the ancient clerks came out, the
usual watch was set, and Mr.\ Cruncher and his son went home to tea.

``Now, I tell you where it is!'' said Mr.\ Cruncher to his wife, on
entering.  ``If, as a honest tradesman, my wenturs goes wrong to-night,
I shall make sure that you've been praying again me, and I shall work
you for it just the same as if I seen you do it.''

The dejected Mrs.\ Cruncher shook her head.

``Why, you're at it afore my face!'' said Mr.\ Cruncher, with signs of
angry apprehension.

``I am saying nothing.''

``Well, then; don't meditate nothing.  You might as well flop as
meditate.  You may as well go again me one way as another.
Drop it altogether.''

``Yes, Jerry.''

``Yes, Jerry,'' repeated Mr.\ Cruncher sitting down to tea.  ``Ah!
It \emph{is} yes, Jerry.  That's about it.  You may say yes, Jerry.''

Mr.\ Cruncher had no particular meaning in these sulky corroborations,
but made use of them, as people not unfrequently do, to express
general ironical dissatisfaction.

``You and your yes, Jerry,'' said Mr.\ Cruncher, taking a bite out of his
bread-and-butter, and seeming to help it down with a large invisible
oyster out of his saucer.  ``Ah! I think so.  I believe you.''

``You are going out to-night?'' asked his decent wife, when he took
another bite.

``Yes, I am.''

``May I go with you, father?'' asked his son, briskly.

``No, you mayn't.  I'm a going---as your mother knows---a fishing.
That's where I'm going to.  Going a fishing.''

``Your fishing-rod gets rayther rusty; don't it, father?''

``Never you mind.''

``Shall you bring any fish home, father?''

``If I don't, you'll have short commons, to-morrow,'' returned that
gentleman, shaking his head; ``that's questions enough for you; I
ain't a going out, till you've been long abed.''

He devoted himself during the remainder of the evening to keeping
a most vigilant watch on Mrs.\ Cruncher, and sullenly holding her in
conversation that she might be prevented from meditating any petitions
to his disadvantage.  With this view, he urged his son to hold her in
conversation also, and led the unfortunate woman a hard life by dwelling
on any causes of complaint he could bring against her, rather than he
would leave her for a moment to her own reflections.  The devoutest
person could have rendered no greater homage to the efficacy of an honest
prayer than he did in this distrust of his wife.  It was as if a
professed unbeliever in ghosts should be frightened by a ghost story.

``And mind you!'' said Mr.\ Cruncher.  ``No games to-morrow!  If I,
as a honest tradesman, succeed in providing a jinte of meat or two,
none of your not touching of it, and sticking to bread.  If I,
as a honest tradesman, am able to provide a little beer, none of your
declaring on water.  When you go to Rome, do as Rome does.  Rome will
be a ugly customer to you, if you don't.  \emph{I'm} your Rome, you know.''

Then he began grumbling again:

``With your flying into the face of your own wittles and drink!  I don't
know how scarce you mayn't make the wittles and drink here, by your
flopping tricks and your unfeeling conduct.  Look at your boy:  he \emph{is}
your'n, ain't he?  He's as thin as a lath.  Do you call yourself a
mother, and not know that a mother's first duty is to blow her boy out?''

This touched Young Jerry on a tender place; who adjured his mother to
perform her first duty, and, whatever else she did or neglected, above
all things to lay especial stress on the discharge of that maternal
function so affectingly and delicately indicated by his other parent.

Thus the evening wore away with the Cruncher family, until Young Jerry
was ordered to bed, and his mother, laid under similar injunctions,
obeyed them.  Mr.\ Cruncher beguiled the earlier watches of the night
with solitary pipes, and did not start upon his excursion until nearly
one o'clock.  Towards that small and ghostly hour, he rose up from his
chair, took a key out of his pocket, opened a locked cupboard, and
brought forth a sack, a crowbar of convenient size, a rope and chain,
and other fishing tackle of that nature.  Disposing these articles about
him in skilful manner, he bestowed a parting defiance on Mrs.\ Cruncher,
extinguished the light, and went out.

Young Jerry, who had only made a feint of undressing when he went to bed,
was not long after his father.  Under cover of the darkness he followed
out of the room, followed down the stairs, followed down the court,
followed out into the streets.  He was in no uneasiness concerning
his getting into the house again, for it was full of lodgers, and the
door stood ajar all night.

Impelled by a laudable ambition to study the art and mystery of his
father's honest calling, Young Jerry, keeping as close to house fronts,
walls, and doorways, as his eyes were close to one another, held his
honoured parent in view.  The honoured parent steering Northward,
had not gone far, when he was joined by another disciple of
Izaak Walton, and the two trudged on together.

Within half an hour from the first starting, they were beyond the
winking lamps, and the more than winking watchmen, and were out upon
a lonely road.  Another fisherman was picked up here---and that so
silently, that if Young Jerry had been superstitious, he might have
supposed the second follower of the gentle craft to have, all of a
sudden, split himself into two.

The three went on, and Young Jerry went on, until the three stopped
under a bank overhanging the road.  Upon the top of the bank was a
low brick wall, surmounted by an iron railing.  In the shadow of bank
and wall the three turned out of the road, and up a blind lane, of which
the wall---there, risen to some eight or ten feet high---formed one side.
Crouching down in a corner, peeping up the lane, the next object that
Young Jerry saw, was the form of his honoured parent, pretty well
defined against a watery and clouded moon, nimbly scaling an iron
gate.  He was soon over, and then the second fisherman got over, and
then the third.  They all dropped softly on the ground within the gate,
and lay there a little---listening perhaps.  Then, they moved away on
their hands and knees.

It was now Young Jerry's turn to approach the gate:  which he did,
holding his breath.  Crouching down again in a corner there, and looking
in, he made out the three fishermen creeping through some rank grass!
and all the gravestones in the churchyard---it was a large churchyard
that they were in---looking on like ghosts in white, while the church
tower itself looked on like the ghost of a monstrous giant.  They did
not creep far, before they stopped and stood upright.  And then they
began to fish.

They fished with a spade, at first.  Presently the honoured parent
appeared to be adjusting some instrument like a great corkscrew.
Whatever tools they worked with, they worked hard, until the awful
striking of the church clock so terrified Young Jerry, that he made off,
with his hair as stiff as his father's.

But, his long-cherished desire to know more about these matters, not
only stopped him in his running away, but lured him back again.  They
were still fishing perseveringly, when he peeped in at the gate for
the second time; but, now they seemed to have got a bite.  There was a
screwing and complaining sound down below, and their bent figures were
strained, as if by a weight.  By slow degrees the weight broke away the
earth upon it, and came to the surface.  Young Jerry very well knew what
it would be; but, when he saw it, and saw his honoured parent about to
wrench it open, he was so frightened, being new to the sight, that he
made off again, and never stopped until he had run a mile or more.

He would not have stopped then, for anything less necessary than
breath, it being a spectral sort of race that he ran, and one highly
desirable to get to the end of.  He had a strong idea that the coffin
he had seen was running after him; and, pictured as hopping on behind
him, bolt upright, upon its narrow end, always on the point of
overtaking him and hopping on at his side---perhaps taking his arm---%
it was a pursuer to shun.  It was an inconsistent and ubiquitous fiend
too, for, while it was making the whole night behind him dreadful,
he darted out into the roadway to avoid dark alleys, fearful of its
coming hopping out of them like a dropsical boy's-Kite without tail
and wings.  It hid in doorways too, rubbing its horrible shoulders
against doors, and drawing them up to its ears, as if it were laughing.
It got into shadows on the road, and lay cunningly on its back to
trip him up.  All this time it was incessantly hopping on behind and
gaining on him, so that when the boy got to his own door he had reason
for being half dead.  And even then it would not leave him, but followed
him upstairs with a bump on every stair, scrambled into bed with him,
and bumped down, dead and heavy, on his breast when he fell asleep.

From his oppressed slumber, Young Jerry in his closet was awakened
after daybreak and before sunrise, by the presence of his father in
the family room.  Something had gone wrong with him; at least, so
Young Jerry inferred, from the circumstance of his holding
Mrs.\ Cruncher by the ears, and knocking the back of her head against
the head-board of the bed.

``I told you I would,'' said Mr.\ Cruncher, ``and I did.''

``Jerry, Jerry, Jerry!'' his wife implored.

``You oppose yourself to the profit of the business,'' said Jerry,
``and me and my partners suffer.  You was to honour and obey;
why the devil don't you?''

``I try to be a good wife, Jerry,'' the poor woman protested, with tears.

``Is it being a good wife to oppose your husband's business?  Is it
honouring your husband to dishonour his business?  Is it obeying your
husband to disobey him on the wital subject of his business?''

``You hadn't taken to the dreadful business then, Jerry.''

``It's enough for you,'' retorted Mr.\ Cruncher, ``to be the wife of a
honest tradesman, and not to occupy your female mind with calculations
when he took to his trade or when he didn't.  A honouring and obeying
wife would let his trade alone altogether.  Call yourself a religious
woman?  If you're a religious woman, give me a irreligious one!
You have no more nat'ral sense of duty than the bed of this here Thames
river has of a pile, and similarly it must be knocked into you.''

The altercation was conducted in a low tone of voice, and terminated
in the honest tradesman's kicking off his clay-soiled boots, and lying
down at his length on the floor.  After taking a timid peep at him
lying on his back, with his rusty hands under his head for a pillow,
his son lay down too, and fell asleep again.

There was no fish for breakfast, and not much of anything else.
Mr.\ Cruncher was out of spirits, and out of temper, and kept an iron
pot-lid by him as a projectile for the correction of Mrs.\ Cruncher,
in case he should observe any symptoms of her saying Grace.  He was
brushed and washed at the usual hour, and set off with his son to
pursue his ostensible calling.

Young Jerry, walking with the stool under his arm at his father's
side along sunny and crowded Fleet-street, was a very different
Young Jerry from him of the previous night, running home through
darkness and solitude from his grim pursuer.  His cunning was fresh
with the day, and his qualms were gone with the night---in which
particulars it is not improbable that he had compeers in Fleet-street
and the City of London, that fine morning.

``Father,'' said Young Jerry, as they walked along:  taking care to
keep at arm's length and to have the stool well between them:
``what's a Resurrection-Man?''

Mr.\ Cruncher came to a stop on the pavement before he answered,
``How should I know?''

``I thought you knowed everything, father,'' said the artless boy.

``Hem!  Well,'' returned Mr.\ Cruncher, going on again, and lifting off
his hat to give his spikes free play, ``he's a tradesman.''

``What's his goods, father?'' asked the brisk Young Jerry.

``His goods,'' said Mr.\ Cruncher, after turning it over in his mind,
``is a branch of Scientific goods.''

``Persons' bodies, ain't it, father?'' asked the lively boy.

``I believe it is something of that sort,'' said Mr.\ Cruncher.

``Oh, father, I should so like to be a Resurrection-Man when I'm
quite growed up!''

Mr.\ Cruncher was soothed, but shook his head in a dubious and moral
way.  ``It depends upon how you dewelop your talents.  Be careful
to dewelop your talents, and never to say no more than you can help
to nobody, and there's no telling at the present time what you may
not come to be fit for.''  As Young Jerry, thus encouraged, went on
a few yards in advance, to plant the stool in the shadow of the Bar,
Mr.\ Cruncher added to himself:  ``Jerry, you honest tradesman, there's
hopes wot that boy will yet be a blessing to you, and a recompense
to you for his mother!''



\chapter{Knitting}


There had been earlier drinking than usual in the wine-shop of
Monsieur Defarge.  As early as six o'clock in the morning, sallow
faces peeping through its barred windows had descried other faces within,
bending over measures of wine.  Monsieur Defarge sold a very thin wine
at the best of times, but it would seem to have been an unusually thin
wine that he sold at this time.  A sour wine, moreover, or a souring,
for its influence on the mood of those who drank it was to make them
gloomy.  No vivacious Bacchanalian flame leaped out of the pressed grape
of Monsieur Defarge:  but, a smouldering fire that burnt in the dark,
lay hidden in the dregs of it.

This had been the third morning in succession, on which there had been
early drinking at the wine-shop of Monsieur Defarge.  It had begun
on Monday, and here was Wednesday come.  There had been more of early
brooding than drinking; for, many men had listened and whispered and
slunk about there from the time of the opening of the door, who could
not have laid a piece of money on the counter to save their souls.
These were to the full as interested in the place, however, as if
they could have commanded whole barrels of wine; and they glided from
seat to seat, and from corner to corner, swallowing talk in lieu
of drink, with greedy looks.

Notwithstanding an unusual flow of company, the master of the wine-shop
was not visible.  He was not missed; for, nobody who crossed the
threshold looked for him, nobody asked for him, nobody wondered to
see only Madame Defarge in her seat, presiding over the distribution
of wine, with a bowl of battered small coins before her, as much defaced
and beaten out of their original impress as the small coinage of humanity
from whose ragged pockets they had come.

A suspended interest and a prevalent absence of mind, were perhaps
observed by the spies who looked in at the wine-shop, as they looked in
at every place, high and low, from the kings palace to the criminal's
gaol.  Games at cards languished, players at dominoes musingly built
towers with them, drinkers drew figures on the tables with spilt drops
of wine, Madame Defarge herself picked out the pattern on her sleeve
with her toothpick, and saw and heard something inaudible and invisible
a long way off.

Thus, Saint Antoine in this vinous feature of his, until midday.  It
was high noontide, when two dusty men passed through his streets and
under his swinging lamps:  of whom, one was Monsieur Defarge:  the other
a mender of roads in a blue cap.  All adust and athirst, the two entered
the wine-shop.  Their arrival had lighted a kind of fire in the breast
of Saint Antoine, fast spreading as they came along, which stirred and
flickered in flames of faces at most doors and windows.  Yet, no one
had followed them, and no man spoke when they entered the wine-shop,
though the eyes of every man there were turned upon them.

``Good day, gentlemen!'' said Monsieur Defarge.

It may have been a signal for loosening the general tongue.
It elicited an answering chorus of ``Good day!''

``It is bad weather, gentlemen,'' said Defarge, shaking his head.

Upon which, every man looked at his neighbour, and then all cast down
their eyes and sat silent.  Except one man, who got up and went out.

``My wife,'' said Defarge aloud, addressing Madame Defarge:  ``I have
travelled certain leagues with this good mender of roads, called
Jacques.  I met him---by accident---a day and half's journey out of
Paris.  He is a good child, this mender of roads, called Jacques.
Give him to drink, my wife!''

A second man got up and went out.  Madame Defarge set wine before the
mender of roads called Jacques, who doffed his blue cap to the company,
and drank.  In the breast of his blouse he carried some coarse dark
bread; he ate of this between whiles, and sat munching and drinking
near Madame Defarge's counter.  A third man got up and went out.

Defarge refreshed himself with a draught of wine---but, he took less
than was given to the stranger, as being himself a man to whom it was
no rarity---and stood waiting until the countryman had made his breakfast.
He looked at no one present, and no one now looked at him; not even
Madame Defarge, who had taken up her knitting, and was at work.

``Have you finished your repast, friend?'' he asked, in due season.

``Yes, thank you.''

``Come, then!  You shall see the apartment that I told you you could
occupy.  It will suit you to a marvel.''

Out of the wine-shop into the street, out of the street into a
courtyard, out of the courtyard up a steep staircase, out of the
staircase into a garret,---formerly the garret where a white-haired
man sat on a low bench, stooping forward and very busy, making shoes.

No white-haired man was there now; but, the three men were there
who had gone out of the wine-shop singly.  And between them and the
white-haired man afar off, was the one small link, that they had once
looked in at him through the chinks in the wall.

Defarge closed the door carefully, and spoke in a subdued voice:

``Jacques One, Jacques Two, Jacques Three!  This is the witness
encountered by appointment, by me, Jacques Four.  He will tell you all.
Speak, Jacques Five!''

The mender of roads, blue cap in hand, wiped his swarthy forehead with
it, and said, ``Where shall I commence, monsieur?''

``Commence,'' was Monsieur Defarge's not unreasonable reply, ``at the
commencement.''

``I saw him then, messieurs,'' began the mender of roads, ``a year ago
this running summer, underneath the carriage of the Marquis, hanging by
the chain.  Behold the manner of it.  I leaving my work on the road,
the sun going to bed, the carriage of the Marquis slowly ascending
the hill, he hanging by the chain---like this.''

Again the mender of roads went through the whole performance; in which
he ought to have been perfect by that time, seeing that it had been
the infallible resource and indispensable entertainment of his village
during a whole year.

Jacques One struck in, and asked if he had ever seen the man before?

``Never,'' answered the mender of roads, recovering his perpendicular.

Jacques Three demanded how he afterwards recognised him then?

``By his tall figure,'' said the mender of roads, softly, and with his
finger at his nose.  ``When Monsieur the Marquis demands that evening,
'Say, what is he like?' I make response, `Tall as a spectre.'\,''

``You should have said, short as a dwarf,'' returned Jacques Two.

``But what did I know?  The deed was not then accomplished, neither did
he confide in me.  Observe!  Under those circumstances even, I do not
offer my testimony.  Monsieur the Marquis indicates me with his finger,
standing near our little fountain, and says, `To me!  Bring that rascal!'
My faith, messieurs, I offer nothing.''

``He is right there, Jacques,'' murmured Defarge, to him who had
interrupted.  ``Go on!''

``Good!'' said the mender of roads, with an air of mystery.  ``The tall
man is lost, and he is sought---how many months?  Nine, ten, eleven?''

``No matter, the number,'' said Defarge.  ``He is well hidden, but at last
he is unluckily found.  Go on!''

``I am again at work upon the hill-side, and the sun is again about to
go to bed.  I am collecting my tools to descend to my cottage down in
the village below, where it is already dark, when I raise my eyes,
and see coming over the hill six soldiers.  In the midst of them
is a tall man with his arms bound---tied to his sides---like this!''

With the aid of his indispensable cap, he represented a man with his
elbows bound fast at his hips, with cords that were knotted behind him.

``I stand aside, messieurs, by my heap of stones, to see the soldiers
and their prisoner pass (for it is a solitary road, that, where any
spectacle is well worth looking at), and at first, as they approach,
I see no more than that they are six soldiers with a tall man bound,
and that they are almost black to my sight---except on the side of the
sun going to bed, where they have a red edge, messieurs.  Also, I see
that their long shadows are on the hollow ridge on the opposite side
of the road, and are on the hill above it, and are like the shadows of
giants.  Also, I see that they are covered with dust, and that the dust
moves with them as they come, tramp, tramp!  But when they advance
quite near to me, I recognise the tall man, and he recognises me.
Ah, but he would be well content to precipitate himself over the
hill-side once again, as on the evening when he and I first encountered,
close to the same spot!''

He described it as if he were there, and it was evident that he saw
it vividly; perhaps he had not seen much in his life.

``I do not show the soldiers that I recognise the tall man; he does
not show the soldiers that he recognises me; we do it, and we know it,
with our eyes.  `Come on!' says the chief of that company, pointing to
the village, `bring him fast to his tomb!' and they bring him faster.
I follow.  His arms are swelled because of being bound so tight, his
wooden shoes are large and clumsy, and he is lame.  Because he is lame,
and consequently slow, they drive him with their guns---like this!''

He imitated the action of a man's being impelled forward by the
butt-ends of muskets.

``As they descend the hill like madmen running a race, he falls.
They laugh and pick him up again.  His face is bleeding and covered with
dust, but he cannot touch it; thereupon they laugh again.  They bring
him into the village; all the village runs to look; they take him past
the mill, and up to the prison; all the village sees the prison gate
open in the darkness of the night, and swallow him---like this!''

He opened his mouth as wide as he could, and shut it with a sounding
snap of his teeth.  Observant of his unwillingness to mar the effect
by opening it again, Defarge said, ``Go on, Jacques.''

``All the village,'' pursued the mender of roads, on tiptoe and in a
low voice, ``withdraws; all the village whispers by the fountain;
all the village sleeps; all the village dreams of that unhappy one,
within the locks and bars of the prison on the crag, and never to come
out of it, except to perish.  In the morning, with my tools upon my
shoulder, eating my morsel of black bread as I go, I make a circuit
by the prison, on my way to my work.  There I see him, high up,
behind the bars of a lofty iron cage, bloody and dusty as last night,
looking through.  He has no hand free, to wave to me; I dare not call
to him; he regards me like a dead man.''

Defarge and the three glanced darkly at one another.  The looks of
all of them were dark, repressed, and revengeful, as they listened to
the countryman's story; the manner of all of them, while it was secret,
was authoritative too.  They had the air of a rough tribunal; Jacques
One and Two sitting on the old pallet-bed, each with his chin resting
on his hand, and his eyes intent on the road-mender; Jacques Three,
equally intent, on one knee behind them, with his agitated hand always
gliding over the network of fine nerves about his mouth and nose;
Defarge standing between them and the narrator, whom he had stationed
in the light of the window, by turns looking from him to them, and
from them to him.

``Go on, Jacques,'' said Defarge.

``He remains up there in his iron cage some days.  The village looks
at him by stealth, for it is afraid.  But it always looks up, from
a distance, at the prison on the crag; and in the evening, when the
work of the day is achieved and it assembles to gossip at the fountain,
all faces are turned towards the prison.  Formerly, they were turned
towards the posting-house; now, they are turned towards the prison.
They whisper at the fountain, that although condemned to death he will
not be executed; they say that petitions have been presented in Paris,
showing that he was enraged and made mad by the death of his child;
they say that a petition has been presented to the King himself.
What do I know?  It is possible.  Perhaps yes, perhaps no.''

``Listen then, Jacques,'' Number One of that name sternly interposed.
``Know that a petition was presented to the King and Queen.  All here,
yourself excepted, saw the King take it, in his carriage in the street,
sitting beside the Queen.  It is Defarge whom you see here, who,
at the hazard of his life, darted out before the horses, with the
petition in his hand.''

``And once again listen, Jacques!'' said the kneeling Number Three:
his fingers ever wandering over and over those fine nerves, with a
strikingly greedy air, as if he hungered for something---that was
neither food nor drink; ``the guard, horse and foot, surrounded
the petitioner, and struck him blows.  You hear?''

``I hear, messieurs.''

``Go on then,'' said Defarge.

``Again; on the other hand, they whisper at the fountain,'' resumed the
countryman, ``that he is brought down into our country to be executed
on the spot, and that he will very certainly be executed.  They even
whisper that because he has slain Monseigneur, and because Monseigneur
was the father of his tenants---serfs---what you will---he will be
executed as a parricide.  One old man says at the fountain, that his
right hand, armed with the knife, will be burnt off before his face;
that, into wounds which will be made in his arms, his breast,
and his legs, there will be poured boiling oil, melted lead, hot resin,
wax, and sulphur; finally, that he will be torn limb from limb by four
strong horses.  That old man says, all this was actually done to a
prisoner who made an attempt on the life of the late King,
Louis Fifteen.  But how do I know if he lies?  I am not a scholar.''

``Listen once again then, Jacques!'' said the man with the restless hand
and the craving air.  ``The name of that prisoner was Damiens, and it
was all done in open day, in the open streets of this city of Paris;
and nothing was more noticed in the vast concourse that saw it done,
than the crowd of ladies of quality and fashion, who were full of eager
attention to the last---to the last, Jacques, prolonged until nightfall,
when he had lost two legs and an arm, and still breathed!  And it
was done---why, how old are you?''

``Thirty-five,'' said the mender of roads, who looked sixty.

``It was done when you were more than ten years old; you might
have seen it.''

``Enough!'' said Defarge, with grim impatience.  ``Long live the Devil!
Go on.''

``Well!  Some whisper this, some whisper that; they speak of nothing else;
even the fountain appears to fall to that tune.  At length, on Sunday
night when all the village is asleep, come soldiers, winding down from
the prison, and their guns ring on the stones of the little street.
Workmen dig, workmen hammer, soldiers laugh and sing; in the morning,
by the fountain, there is raised a gallows forty feet high, poisoning
the water.''

The mender of roads looked \emph{through} rather than \emph{at} the low ceiling,
and pointed as if he saw the gallows somewhere in the sky.

``All work is stopped, all assemble there, nobody leads the cows out,
the cows are there with the rest.  At midday, the roll of drums.
Soldiers have marched into the prison in the night, and he is in the
midst of many soldiers.  He is bound as before, and in his mouth there
is a gag---tied so, with a tight string, making him look almost as if he
laughed.''  He suggested it, by creasing his face with his two thumbs,
from the corners of his mouth to his ears.  ``On the top of the gallows
is fixed the knife, blade upwards, with its point in the air.  He is
hanged there forty feet high---and is left hanging, poisoning the water.''

They looked at one another, as he used his blue cap to wipe his face,
on which the perspiration had started afresh while he recalled the spectacle.

``It is frightful, messieurs.  How can the women and the children draw
water!  Who can gossip of an evening, under that shadow!  Under it,
have I said?  When I left the village, Monday evening as the sun was
going to bed, and looked back from the hill, the shadow struck across
the church, across the mill, across the prison---seemed to strike across
the earth, messieurs, to where the sky rests upon it!''

The hungry man gnawed one of his fingers as he looked at the other
three, and his finger quivered with the craving that was on him.

``That's all, messieurs.  I left at sunset (as I had been warned to do),
and I walked on, that night and half next day, until I met (as I was
warned I should) this comrade.  With him, I came on, now riding and
now walking, through the rest of yesterday and through last night.
And here you see me!''

After a gloomy silence, the first Jacques said, ``Good!  You have
acted and recounted faithfully.  Will you wait for us a little,
outside the door?''

``Very willingly,'' said the mender of roads.  Whom Defarge escorted
to the top of the stairs, and, leaving seated there, returned.

The three had risen, and their heads were together when he came
back to the garret.

``How say you, Jacques?'' demanded Number One.  ``To be registered?''

``To be registered, as doomed to destruction,'' returned Defarge.

``Magnificent!'' croaked the man with the craving.

``The chateau, and all the race?'' inquired the first.

``The chateau and all the race,'' returned Defarge.  ``Extermination.''

The hungry man repeated, in a rapturous croak, ``Magnificent!'' and began
gnawing another finger.

``Are you sure,'' asked Jacques Two, of Defarge, ``that no embarrassment
can arise from our manner of keeping the register?  Without doubt it
is safe, for no one beyond ourselves can decipher it; but shall we
always be able to decipher it---or, I ought to say, will she?''

``Jacques,'' returned Defarge, drawing himself up, ``if madame my wife
undertook to keep the register in her memory alone, she would not
lose a word of it---not a syllable of it.  Knitted, in her own stitches
and her own symbols, it will always be as plain to her as the sun.
Confide in Madame Defarge.  It would be easier for the weakest poltroon
that lives, to erase himself from existence, than to erase one letter
of his name or crimes from the knitted register of Madame Defarge.''

There was a murmur of confidence and approval, and then the man who
hungered, asked:  ``Is this rustic to be sent back soon?  I hope so.
He is very simple; is he not a little dangerous?''

``He knows nothing,'' said Defarge; ``at least nothing more than would
easily elevate himself to a gallows of the same height.  I charge myself
with him; let him remain with me; I will take care of him, and set him
on his road.  He wishes to see the fine world---the King, the Queen, and
Court; let him see them on Sunday.''

``What?'' exclaimed the hungry man, staring.  ``Is it a good sign, that
he wishes to see Royalty and Nobility?''

``Jacques,'' said Defarge; ``judiciously show a cat milk, if you wish
her to thirst for it.  Judiciously show a dog his natural prey,
if you wish him to bring it down one day.''

Nothing more was said, and the mender of roads, being found already
dozing on the topmost stair, was advised to lay himself down on the
pallet-bed and take some rest.  He needed no persuasion,
and was soon asleep.

Worse quarters than Defarge's wine-shop, could easily have been found
in Paris for a provincial slave of that degree.  Saving for a mysterious
dread of madame by which he was constantly haunted, his life was very
new and agreeable.  But, madame sat all day at her counter, so expressly
unconscious of him, and so particularly determined not to perceive that
his being there had any connection with anything below the surface,
that he shook in his wooden shoes whenever his eye lighted on her.
For, he contended with himself that it was impossible to foresee what
that lady might pretend next; and he felt assured that if she should
take it into her brightly ornamented head to pretend that she had seen
him do a murder and afterwards flay the victim, she would infallibly
go through with it until the play was played out.

Therefore, when Sunday came, the mender of roads was not enchanted
(though he said he was) to find that madame was to accompany monsieur
and himself to Versailles.  It was additionally disconcerting to have
madame knitting all the way there, in a public conveyance; it was
additionally disconcerting yet, to have madame in the crowd in the
afternoon, still with her knitting in her hands as the crowd waited
to see the carriage of the King and Queen.

``You work hard, madame,'' said a man near her.

``Yes,'' answered Madame Defarge; ``I have a good deal to do.''

``What do you make, madame?''

``Many things.''

``For instance---''

``For instance,'' returned Madame Defarge, composedly, ``shrouds.''

The man moved a little further away, as soon as he could, and the
mender of roads fanned himself with his blue cap:  feeling it mightily
close and oppressive.  If he needed a King and Queen to restore him,
he was fortunate in having his remedy at hand; for, soon the large-faced
King and the fair-faced Queen came in their golden coach, attended by
the shining Bull's Eye of their Court, a glittering multitude of
laughing ladies and fine lords; and in jewels and silks and powder and
splendour and elegantly spurning figures and handsomely disdainful faces
of both sexes, the mender of roads bathed himself, so much to his
temporary intoxication, that he cried Long live the King, Long live
the Queen, Long live everybody and everything! as if he had never
heard of ubiquitous Jacques in his time.  Then, there were gardens,
courtyards, terraces, fountains, green banks, more King and Queen,
more Bull's Eye,more lords and ladies, more Long live they all! until
he absolutely wept with sentiment.  During the whole of this scene,
which lasted some three hours, he had plenty of shouting and weeping
and sentimental company, and throughout Defarge held him by the collar,
as if to restrain him from flying at the objects of his brief devotion
and tearing them to pieces.

``Bravo!'' said Defarge, clapping him on the back when it was over,
like a patron; ``you are a good boy!''

The mender of roads was now coming to himself, and was mistrustful of
having made a mistake in his late demonstrations; but no.

``You are the fellow we want,'' said Defarge, in his ear; ``you make these
fools believe that it will last for ever.  Then, they are the more
insolent, and it is the nearer ended.''

``Hey!'' cried the mender of roads, reflectively; ``that's true.''

``These fools know nothing.  While they despise your breath, and would
stop it for ever and ever, in you or in a hundred like you rather than
in one of their own horses or dogs, they only know what your breath
tells them.  Let it deceive them, then, a little longer; it cannot
deceive them too much.''

Madame Defarge looked superciliously at the client, and nodded in
confirmation.

``As to you,'' said she, ``you would shout and shed tears for anything,
if it made a show and a noise.  Say!  Would you not?''

``Truly, madame, I think so.  For the moment.''

``If you were shown a great heap of dolls, and were set upon them to
pluck them to pieces and despoil them for your own advantage, you
would pick out the richest and gayest.  Say!  Would you not?''

``Truly yes, madame.''

``Yes.  And if you were shown a flock of birds, unable to fly, and were
set upon them to strip them of their feathers for your own advantage,
you would set upon the birds of the finest feathers; would you not?''

``It is true, madame.''

``You have seen both dolls and birds to-day,'' said Madame Defarge,
with a wave of her hand towards the place where they had last been
apparent; ``now, go home!''



\chapter{Still Knitting}


Madame Defarge and monsieur her husband returned amicably to the bosom
of Saint Antoine, while a speck in a blue cap toiled through the
darkness, and through the dust, and down the weary miles of avenue by
the wayside, slowly tending towards that point of the compass where the
chateau of Monsieur the Marquis, now in his grave, listened to the
whispering trees.  Such ample leisure had the stone faces, now, for
listening to the trees and to the fountain, that the few village
scarecrows who, in their quest for herbs to eat and fragments of dead
stick to burn, strayed within sight of the great stone courtyard and
terrace staircase, had it borne in upon their starved fancy that the
expression of the faces was altered.  A rumour just lived in the
village---had a faint and bare existence there, as its people had---that
when the knife struck home, the faces changed, from faces of pride to
faces of anger and pain; also, that when that dangling figure was
hauled up forty feet above the fountain, they changed again, and bore
a cruel look of being avenged, which they would henceforth bear
for ever.  In the stone face over the great window of the bed-chamber
where the murder was done, two fine dints were pointed out in the
sculptured nose, which everybody recognised, and which nobody had
seen of old; and on the scarce occasions when two or three ragged
peasants emerged from the crowd to take a hurried peep at Monsieur
the Marquis petrified, a skinny finger would not have pointed to it
for a minute, before they all started away among the moss and leaves,
like the more fortunate hares who could find a living there.

Chateau and hut, stone face and dangling figure, the red stain on the
stone floor, and the pure water in the village well---thousands of acres
of land---a whole province of France---all France itself---lay under the
night sky, concentrated into a faint hair-breadth line.  So does a
whole world, with all its greatnesses and littlenesses, lie in a
twinkling star.  And as mere human knowledge can split a ray of light
and analyse the manner of its composition, so, sublimer intelligences
may read in the feeble shining of this earth of ours, every thought
and act, every vice and virtue, of every responsible creature on it.

The Defarges, husband and wife, came lumbering under the starlight,
in their public vehicle, to that gate of Paris whereunto their journey
naturally tended.  There was the usual stoppage at the barrier
guardhouse, and the usual lanterns came glancing forth for the usual
examination and inquiry.  Monsieur Defarge alighted; knowing one or
two of the soldiery there, and one of the police.  The latter he was
intimate with, and affectionately embraced.

When Saint Antoine had again enfolded the Defarges in his dusky wings,
and they, having finally alighted near the Saint's boundaries, were
picking their way on foot through the black mud and offal of his streets,
Madame Defarge spoke to her husband:

``Say then, my friend; what did Jacques of the police tell thee?''

``Very little to-night, but all he knows.  There is another spy
commissioned for our quarter.  There may be many more, for all that
he can say, but he knows of one.''

``Eh well!'' said Madame Defarge, raising her eyebrows with a cool
business air.  ``It is necessary to register him.  How do they
call that man?''

``He is English.''

``So much the better.  His name?''

``Barsad,'' said Defarge, making it French by pronunciation.  But,
he had been so careful to get it accurately, that he then spelt
it with perfect correctness.

``Barsad,'' repeated madame.  ``Good.  Christian name?''

``John.''

``John Barsad,'' repeated madame, after murmuring it once to herself.
``Good.  His appearance; is it known?''

``Age, about forty years; height, about five feet nine; black hair;
complexion dark; generally, rather handsome visage; eyes dark, face thin,
long, and sallow; nose aquiline, but not straight, having a peculiar
inclination towards the left cheek; expression, therefore, sinister.''

``Eh my faith.  It is a portrait!'' said madame, laughing.  ``He shall
be registered to-morrow.''

They turned into the wine-shop, which was closed (for it was midnight),
and where Madame Defarge immediately took her post at her desk,
counted the small moneys that had been taken during her absence,
examined the stock, went through the entries in the book, made other
entries of her own, checked the serving man in every possible way,
and finally dismissed him to bed.  Then she turned out the contents
of the bowl of money for the second time, and began knotting them up
in her handkerchief, in a chain of separate knots, for safe keeping
through the night.  All this while, Defarge, with his pipe in his mouth,
walked up and down, complacently admiring, but never interfering;
in which condition, indeed, as to the business and his domestic affairs,
he walked up and down through life.

The night was hot, and the shop, close shut and surrounded by so foul
a neighbourhood, was ill-smelling.  Monsieur Defarge's olfactory
sense was by no means delicate, but the stock of wine smelt much
stronger than it ever tasted, and so did the stock of rum and brandy
and aniseed.  He whiffed the compound of scents away, as he put down
his smoked-out pipe.

``You are fatigued,'' said madame, raising her glance as she knotted
the money.  ``There are only the usual odours.''

``I am a little tired,'' her husband acknowledged.

``You are a little depressed, too,'' said madame, whose quick eyes had
never been so intent on the accounts, but they had had a ray or two
for him.  ``Oh, the men, the men!''

``But my dear!'' began Defarge.

``But my dear!'' repeated madame, nodding firmly; ``but my dear!
You are faint of heart to-night, my dear!''

``Well, then,'' said Defarge, as if a thought were wrung out of his breast,
``it \emph{is} a long time.''

``It is a long time,'' repeated his wife; ``and when is it not a long time?
Vengeance and retribution require a long time; it is the rule.''

``It does not take a long time to strike a man with Lightning,''
said Defarge.

``How long,'' demanded madame, composedly, ``does it take to make and
store the lightning?  Tell me.''

Defarge raised his head thoughtfully, as if there were something
in that too.

``It does not take a long time,'' said madame, ``for an earthquake to swallow
a town.  Eh well!  Tell me how long it takes to prepare the earthquake?''

``A long time, I suppose,'' said Defarge.

``But when it is ready, it takes place, and grinds to pieces everything
before it.  In the meantime, it is always preparing, though it is not
seen or heard.  That is your consolation.  Keep it.''

She tied a knot with flashing eyes, as if it throttled a foe.

``I tell thee,'' said madame, extending her right hand, for emphasis,
``that although it is a long time on the road, it is on the road and
coming.  I tell thee it never retreats, and never stops.  I tell thee
it is always advancing.  Look around and consider the lives of all the
world that we know, consider the faces of all the world that we know,
consider the rage and discontent to which the Jacquerie addresses itself
with more and more of certainty every hour.  Can such things last?
Bah!  I mock you.''

``My brave wife,'' returned Defarge, standing before her with his head
a little bent, and his hands clasped at his back, like a docile and
attentive pupil before his catechist, ``I do not question all this.
But it has lasted a long time, and it is possible---you know well,
my wife, it is possible---that it may not come, during our lives.''

``Eh well!  How then?'' demanded madame, tying another knot, as if
there were another enemy strangled.

``Well!'' said Defarge, with a half complaining and half apologetic shrug.
``We shall not see the triumph.''

``We shall have helped it,'' returned madame, with her extended hand in
strong action.  ``Nothing that we do, is done in vain.  I believe, with
all my soul, that we shall see the triumph.  But even if not, even if
I knew certainly not, show me the neck of an aristocrat and tyrant,
and still I would---''

Then madame, with her teeth set, tied a very terrible knot indeed.

``Hold!'' cried Defarge, reddening a little as if he felt charged with
cowardice; ``I too, my dear, will stop at nothing.''

``Yes!  But it is your weakness that you sometimes need to see your
victim and your opportunity, to sustain you.  Sustain yourself without
that.  When the time comes, let loose a tiger and a devil; but wait
for the time with the tiger and the devil chained---not shown---yet
always ready.''

Madame enforced the conclusion of this piece of advice by striking
her little counter with her chain of money as if she knocked its brains
out, and then gathering the heavy handkerchief under her arm in a
serene manner, and observing that it was time to go to bed.

Next noontide saw the admirable woman in her usual place in the
wine-shop, knitting away assiduously.  A rose lay beside her, and
if she now and then glanced at the flower, it was with no infraction
of her usual preoccupied air.  There were a few customers, drinking
or not drinking, standing or seated, sprinkled about.  The day was
very hot, and heaps of flies, who were extending their inquisitive
and adventurous perquisitions into all the glutinous little glasses
near madame, fell dead at the bottom.  Their decease made no impression
on the other flies out promenading, who looked at them in the coolest
manner (as if they themselves were elephants, or something as far
removed), until they met the same fate.  Curious to consider how heedless
flies are!---perhaps they thought as much at Court that sunny summer day.

A figure entering at the door threw a shadow on Madame Defarge which
she felt to be a new one.  She laid down her knitting, and began to
pin her rose in her head-dress, before she looked at the figure.

It was curious.  The moment Madame Defarge took up the rose, the
customers ceased talking, and began gradually to drop out of the
wine-shop.

``Good day, madame,'' said the new-comer.

``Good day, monsieur.''

She said it aloud, but added to herself, as she resumed her knitting:
``Hah!  Good day, age about forty, height about five feet nine, black
hair, generally rather handsome visage, complexion dark, eyes dark,
thin, long and sallow face, aquiline nose but not straight, having a
peculiar inclination towards the left cheek which imparts a sinister
expression!  Good day, one and all!''

``Have the goodness to give me a little glass of old cognac, and a
mouthful of cool fresh water, madame.''

Madame complied with a polite air.

``Marvellous cognac this, madame!''

It was the first time it had ever been so complemented, and Madame
Defarge knew enough of its antecedents to know better.  She said,
however, that the cognac was flattered, and took up her knitting.
The visitor watched her fingers for a few moments, and took the
opportunity of observing the place in general.

``You knit with great skill, madame.''

``I am accustomed to it.''

``A pretty pattern too!''

``\emph{You} think so?'' said madame, looking at him with a smile.

``Decidedly.  May one ask what it is for?''

``Pastime,'' said madame, still looking at him with a smile while her
fingers moved nimbly.

``Not for use?''

``That depends.  I may find a use for it one day.  If I do---Well,''
said madame, drawing a breath and nodding her head with a stern kind
of coquetry, ``I'll use it!''

It was remarkable; but, the taste of Saint Antoine seemed to be
decidedly opposed to a rose on the head-dress of Madame Defarge.
Two men had entered separately, and had been about to order drink, when,
catching sight of that novelty, they faltered, made a pretence of
looking about as if for some friend who was not there, and went away.
Nor, of those who had been there when this visitor entered, was there one
left.  They had all dropped off.  The spy had kept his eyes open, but had
been able to detect no sign.  They had lounged away in a poverty-stricken,
purposeless, accidental manner, quite natural and unimpeachable.

``\emph{John},'' thought madame, checking off her work as her fingers knitted,
and her eyes looked at the stranger.  ``Stay long enough, and I shall
knit `\emph{Barsad}' before you go.''

``You have a husband, madame?''

``I have.''

``Children?''

``No children.''

``Business seems bad?''

``Business is very bad; the people are so poor.''

``Ah, the unfortunate, miserable people!  So oppressed, too---as you say.''

``As \emph{you} say,'' madame retorted, correcting him, and deftly knitting
an extra something into his name that boded him no good.

``Pardon me; certainly it was I who said so, but you naturally think so.
Of course.''

``\emph{I} think?'' returned madame, in a high voice.  ``I and my husband
have enough to do to keep this wine-shop open, without thinking.  All
we think, here, is how to live.  That is the subject \emph{we} think of,
and it gives us, from morning to night, enough to think about, without
embarrassing our heads concerning others.  \emph{I} think for others?  No, no.''

The spy, who was there to pick up any crumbs he could find or make, did
not allow his baffled state to express itself in his sinister face; but,
stood with an air of gossiping gallantry, leaning his elbow on Madame
Defarge's little counter, and occasionally sipping his cognac.

``A bad business this, madame, of Gaspard's execution.  Ah! the poor
Gaspard!''  With a sigh of great compassion.

``My faith!'' returned madame, coolly and lightly, ``if people use knives
for such purposes, they have to pay for it.  He knew beforehand what
the price of his luxury was; he has paid the price.''

``I believe,'' said the spy, dropping his soft voice to a tone that
invited confidence, and expressing an injured revolutionary
susceptibility in every muscle of his wicked face:  ``I believe there
is much compassion and anger in this neighbourhood, touching the
poor fellow?  Between ourselves.''

``Is there?'' asked madame, vacantly.

``Is there not?''

``---Here is my husband!'' said Madame Defarge.

As the keeper of the wine-shop entered at the door, the spy saluted
him by touching his hat, and saying, with an engaging smile, ``Good
day, Jacques!''  Defarge stopped short, and stared at him.

``Good day, Jacques!'' the spy repeated; with not quite so much
confidence, or quite so easy a smile under the stare.

``You deceive yourself, monsieur,'' returned the keeper of the
wine-shop.  ``You mistake me for another.  That is not my name.
I am Ernest Defarge.''

``It is all the same,'' said the spy, airily, but discomfited too:
``good day!''

``Good day!'' answered Defarge, drily.

``I was saying to madame, with whom I had the pleasure of chatting when
you entered, that they tell me there is---and no wonder!---much sympathy
and anger in Saint Antoine, touching the unhappy fate of poor Gaspard.''

``No one has told me so,'' said Defarge, shaking his head.  ``I know
nothing of it.''

Having said it, he passed behind the little counter, and stood with
his hand on the back of his wife's chair, looking over that barrier
at the person to whom they were both opposed, and whom either of them
would have shot with the greatest satisfaction.

The spy, well used to his business, did not change his unconscious
attitude, but drained his little glass of cognac, took a sip of fresh
water, and asked for another glass of cognac.  Madame Defarge poured it
out for him, took to her knitting again, and hummed a little song over it.

``You seem to know this quarter well; that is to say, better than I do?''
observed Defarge.

``Not at all, but I hope to know it better.  I am so profoundly interested
in its miserable inhabitants.''

``Hah!'' muttered Defarge.

``The pleasure of conversing with you, Monsieur Defarge, recalls to me,''
pursued the spy, ``that I have the honour of cherishing some interesting
associations with your name.''

``Indeed!'' said Defarge, with much indifference.

``Yes, indeed.  When Doctor Manette was released, you, his old domestic,
had the charge of him, I know.  He was delivered to you.  You see I am
informed of the circumstances?''

``Such is the fact, certainly,'' said Defarge.  He had had it conveyed
to him, in an accidental touch of his wife's elbow as she knitted and
warbled, that he would do best to answer, but always with brevity.

``It was to you,'' said the spy, ``that his daughter came; and it was
from your care that his daughter took him, accompanied by a neat brown
monsieur; how is he called?---in a little wig---Lorry---of the bank of
Tellson and Company---over to England.''

``Such is the fact,'' repeated Defarge.

``Very interesting remembrances!'' said the spy.  ``I have known Doctor
Manette and his daughter, in England.''

``Yes?'' said Defarge.

``You don't hear much about them now?'' said the spy.

``No,'' said Defarge.

``In effect,'' madame struck in, looking up from her work and her little
song, ``we never hear about them.  We received the news of their safe
arrival, and perhaps another letter, or perhaps two; but, since then,
they have gradually taken their road in life---we, ours---and we have
held no correspondence.''

``Perfectly so, madame,'' replied the spy.  ``She is going to be married.''

``Going?'' echoed madame.  ``She was pretty enough to have been married
long ago.  You English are cold, it seems to me.''

``Oh!  You know I am English.''

``I perceive your tongue is,'' returned madame; ``and what the tongue is,
I suppose the man is.''

He did not take the identification as a compliment; but he made the
best of it, and turned it off with a laugh.  After sipping his
cognac to the end, he added:

``Yes, Miss Manette is going to be married.  But not to an Englishman;
to one who, like herself, is French by birth.  And speaking of Gaspard
(ah, poor Gaspard!  It was cruel, cruel!), it is a curious thing that
she is going to marry the nephew of Monsieur the Marquis, for whom
Gaspard was exalted to that height of so many feet; in other words,
the present Marquis.  But he lives unknown in England, he is no
Marquis there; he is Mr.\ Charles Darnay.  D'Aulnais is the name
of his mother's family.''

Madame Defarge knitted steadily, but the intelligence had a palpable
effect upon her husband.  Do what he would, behind the little counter,
as to the striking of a light and the lighting of his pipe, he was
troubled, and his hand was not trustworthy.  The spy would have been
no spy if he had failed to see it, or to record it in his mind.

Having made, at least, this one hit, whatever it might prove to be worth,
and no customers coming in to help him to any other, Mr.\ Barsad paid
for what he had drunk, and took his leave:  taking occasion to say, in a
genteel manner, before he departed, that he looked forward to the pleasure
of seeing Monsieur and Madame Defarge again.  For some minutes after he
had emerged into the outer presence of Saint Antoine, the husband and
wife remained exactly as he had left them, lest he should come back.

``Can it be true,'' said Defarge, in a low voice, looking down at his
wife as he stood smoking with his hand on the back of her chair:  ``what
he has said of Ma'amselle Manette?''

``As he has said it,'' returned madame, lifting her eyebrows a little,
``it is probably false.  But it may be true.''

``If it is---'' Defarge began, and stopped.

``If it is?'' repeated his wife.

``---And if it does come, while we live to see it triumph---I hope, for
her sake, Destiny will keep her husband out of France.''

``Her husband's destiny,'' said Madame Defarge, with her usual composure,
``will take him where he is to go, and will lead him to the end that is
to end him.  That is all I know.''

``But it is very strange---now, at least, is it not very strange''---said
Defarge, rather pleading with his wife to induce her to admit it,
``that, after all our sympathy for Monsieur her father, and herself,
her husband's name should be proscribed under your hand at this moment,
by the side of that infernal dog's who has just left us?''

``Stranger things than that will happen when it does come,'' answered
madame.  ``I have them both here, of a certainty; and they are both
here for their merits; that is enough.''

She roiled up her knitting when she had said those words, and presently
took the rose out of the handkerchief that was wound about her head.
Either Saint Antoine had an instinctive sense that the objectionable
decoration was gone, or Saint Antoine was on the watch for its
disappearance; howbeit, the Saint took courage to lounge in, very
shortly afterwards, and the wine-shop recovered its habitual aspect.

In the evening, at which season of all others Saint Antoine turned
himself inside out, and sat on door-steps and window-ledges, and
came to the corners of vile streets and courts, for a breath of air,
Madame Defarge with her work in her hand was accustomed to pass from
place to place and from group to group:  a Missionary---there were
many like her---such as the world will do well never to breed again.
All the women knitted.  They knitted worthless things; but, the
mechanical work was a mechanical substitute for eating and drinking;
the hands moved for the jaws and the digestive apparatus:  if the bony
fingers had been still, the stomachs would have been more famine-pinched.

But, as the fingers went, the eyes went, and the thoughts.  And as
Madame Defarge moved on from group to group, all three went quicker
and fiercer among every little knot of women that she had spoken with,
and left behind.

Her husband smoked at his door, looking after her with admiration.
``A great woman,'' said he, ``a strong woman, a grand woman, a frightfully
grand woman!''

Darkness closed around, and then came the ringing of church bells and
the distant beating of the military drums in the Palace Courtyard, as
the women sat knitting, knitting.  Darkness encompassed them.  Another
darkness was closing in as surely, when the church bells, then ringing
pleasantly in many an airy steeple over France, should be melted into
thundering cannon; when the military drums should be beating to drown
a wretched voice, that night all potent as the voice of Power and
Plenty, Freedom and Life.  So much was closing in about the women
who sat knitting, knitting, that they their very selves were closing
in around a structure yet unbuilt, where they were to sit knitting,
knitting, counting dropping heads.



\chapter{One Night}


Never did the sun go down with a brighter glory on the quiet corner
in Soho, than one memorable evening when the Doctor and his daughter
sat under the plane-tree together.  Never did the moon rise with a
milder radiance over great London, than on that night when it found
them still seated under the tree, and shone upon their faces
through its leaves.

Lucie was to be married to-morrow.  She had reserved this last
evening for her father, and they sat alone under the plane-tree.

``You are happy, my dear father?''

``Quite, my child.''

They had said little, though they had been there a long time.  When
it was yet light enough to work and read, she had neither engaged
herself in her usual work, nor had she read to him.  She had employed
herself in both ways, at his side under the tree, many and many a time;
but, this time was not quite like any other, and nothing could make it so.

``And I am very happy to-night, dear father.  I am deeply happy in the
love that Heaven has so blessed---my love for Charles, and Charles's
love for me.  But, if my life were not to be still consecrated to you,
or if my marriage were so arranged as that it would part us, even by
the length of a few of these streets, I should be more unhappy and
self-reproachful now than I can tell you.  Even as it is---''

Even as it was, she could not command her voice.

In the sad moonlight, she clasped him by the neck, and laid her face
upon his breast.  In the moonlight which is always sad, as the light
of the sun itself is---as the light called human life is---at its
coming and its going.

``Dearest dear!  Can you tell me, this last time, that you feel quite,
quite sure, no new affections of mine, and no new duties of mine,
will ever interpose between us?  \emph{I} know it well, but do you know it?
In your own heart, do you feel quite certain?''

Her father answered, with a cheerful firmness of conviction he could
scarcely have assumed, ``Quite sure, my darling!  More than that,''
he added, as he tenderly kissed her:  ``my future is far brighter,
Lucie,  seen through your marriage, than it could have been---nay,
than it  ever was---without it.''

``If I could hope \emph{that}, my father!---''

``Believe it, love!  Indeed it is so.  Consider how natural and how
plain it is, my dear, that it should be so.  You, devoted and young,
cannot fully appreciate the anxiety I have felt that your life
should not be wasted---''

She moved her hand towards his lips, but he took it in his,
and repeated the word.

``---wasted, my child---should not be wasted, struck aside from the
natural order of things---for my sake.  Your unselfishness cannot
entirely comprehend how much my mind has gone on this; but, only ask
yourself, how could my happiness be perfect, while yours was incomplete?''

``If I had never seen Charles, my father, I should have been quite
happy with you.''

He smiled at her unconscious admission that she would have been unhappy
without Charles, having seen him; and replied:

``My child, you did see him, and it is Charles.  If it had not been
Charles, it would have been another.  Or, if it had been no other,
I should have been the cause, and then the dark part of my life would
have cast its shadow beyond myself, and would have fallen on you.''

It was the first time, except at the trial, of her ever hearing him refer
to the period of his suffering.  It gave her a strange and new sensation
while his words were in her ears; and she remembered it long afterwards.

``See!'' said the Doctor of Beauvais, raising his hand towards the moon.
``I have looked at her from my prison-window, when I could not bear
her light.  I have looked at her when it has been such torture to me
to think of her shining upon what I had lost, that I have beaten my
head against my prison-walls.  I have looked at her, in a state so
dun and lethargic, that I have thought of nothing but the number of
horizontal lines I could draw across her at the full, and the number of
perpendicular lines with which I could intersect them.''  He added in his
inward and pondering manner, as he looked at the moon, ``It was twenty
either way, I remember, and the twentieth was difficult to squeeze in.''

The strange thrill with which she heard him go back to that time,
deepened as he dwelt upon it; but, there was nothing to shock her in
the manner of his reference.  He only seemed to contrast his present
cheerfulness and felicity with the dire endurance that was over.

``I have looked at her, speculating thousands of times upon the unborn
child from whom I had been rent.  Whether it was alive.  Whether it had
been born alive, or the poor mother's shock had killed it.  Whether it
was a son who would some day avenge his father. (There was a time in my
imprisonment, when my desire for vengeance was unbearable.)  Whether it
was a son who would never know his father's story; who might even live
to weigh the possibility of his father's having disappeared of his own
will and act.  Whether it was a daughter who would grow to be a woman.''

She drew closer to him, and kissed his cheek and his hand.

``I have pictured my daughter, to myself, as perfectly forgetful of me%
---rather, altogether ignorant of me, and unconscious of me.  I have
cast up the years of her age, year after year.  I have seen her married
to a man who knew nothing of my fate.  I have altogether perished from
the remembrance of the living, and in the next generation my place
was a blank.''

``My father!  Even to hear that you had such thoughts of a daughter
who never existed, strikes to my heart as if I had been that child.''

``You, Lucie?  It is out of the Consolation and restoration you have
brought to me, that these remembrances arise, and pass between us and
the moon on this last night.---What did I say just now?''

``She knew nothing of you.  She cared nothing for you.''

``So!  But on other moonlight nights, when the sadness and the silence
have touched me in a different way---have affected me with something as
like a sorrowful sense of peace, as any emotion that had pain for its
foundations could---I have imagined her as coming to me in my cell, and
leading me out into the freedom beyond the fortress.  I have seen her
image in the moonlight often, as I now see you; except that I never held
her in my arms; it stood between the little grated window and the door.
But, you understand that that was not the child I am speaking of?''

``The figure was not; the---the---image; the fancy?''

``No.  That was another thing.  It stood before my disturbed sense of
sight, but it never moved.  The phantom that my mind pursued, was
another and more real child.  Of her outward appearance I know no more
than that she was like her mother.  The other had that likeness too%
---as you have---but was not the same.  Can you follow me, Lucie?
Hardly, I think?  I doubt you must have been a solitary prisoner to
understand these perplexed distinctions.''

His collected and calm manner could not prevent her blood from running
cold, as he thus tried to anatomise his old condition.

``In that more peaceful state, I have imagined her, in the moonlight,
coming to me and taking me out to show me that the home of her married
life was full of her loving remembrance of her lost father.  My picture
was in her room, and I was in her prayers.  Her life was active,
cheerful, useful; but my poor history pervaded it all.''

``I was that child, my father, I was not half so good, but in my love
that was I.''

``And she showed me her children,'' said the Doctor of Beauvais, ``and
they had heard of me, and had been taught to pity me.  When they
passed a prison of the State, they kept far from its frowning walls,
and looked up at its bars, and spoke in whispers.  She could never
deliver me; I imagined that she always brought me back after showing
me such things.  But then, blessed with the relief of tears,
I fell upon my knees, and blessed her.''

``I am that child, I hope, my father.  O my dear, my dear, will you
bless me as fervently to-morrow?''

``Lucie, I recall these old troubles in the reason that I have to-night
for loving you better than words can tell, and thanking God for my
great happiness.  My thoughts, when they were wildest, never rose near
the happiness that I have known with you, and that we have before us.''

He embraced her, solemnly commended her to Heaven, and humbly thanked
Heaven for having bestowed her on him.  By-and-bye, they went
into the house.

There was no one bidden to the marriage but Mr.\ Lorry; there was even
to be no bridesmaid but the gaunt Miss Pross.  The marriage was to
make no change in their place of residence; they had been able to
extend it, by taking to themselves the upper rooms formerly belonging
to the apocryphal invisible lodger, and they desired nothing more.

Doctor Manette was very cheerful at the little supper.  They were
only three at table, and Miss Pross made the third.  He regretted that
Charles was not there; was more than half disposed to object to the
loving little plot that kept him away; and drank to him affectionately.

So, the time came for him to bid Lucie good night, and they separated.
But, in the stillness of the third hour of the morning, Lucie came
downstairs again, and stole into his room; not free from unshaped fears,
beforehand.

All things, however, were in their places; all was quiet; and he lay
asleep, his white hair picturesque on the untroubled pillow, and his
hands lying quiet on the coverlet.  She put her needless candle in the
shadow at a distance, crept up to his bed, and put her lips to his;
then, leaned over him, and looked at him.

Into his handsome face, the bitter waters of captivity had worn; but,
he covered up their tracks with a determination so strong, that he held
the mastery of them even in his sleep.  A more remarkable face in its
quiet, resolute, and guarded struggle with an unseen assailant, was
not to be beheld in all the wide dominions of sleep, that night.

She timidly laid her hand on his dear breast, and put up a prayer that
she might ever be as true to him as her love aspired to be, and as his
sorrows deserved.  Then, she withdrew her hand, and kissed his lips
once more, and went away.  So, the sunrise came, and the shadows of
the leaves of the plane-tree moved upon his face, as softly as her
lips had moved in praying for him.



\chapter{Nine Days}


The marriage-day was shining brightly, and they were ready outside
the closed door of the Doctor's room, where he was speaking with
Charles Darnay.  They were ready to go to church; the beautiful bride,
Mr.\ Lorry, and Miss Pross---to whom the event, through a gradual process
of reconcilement to the inevitable, would have been one of absolute
bliss, but for the yet lingering consideration that her brother
Solomon should have been the bridegroom.

``And so,'' said Mr.\ Lorry, who could not sufficiently admire the bride,
and who had been moving round her to take in every point of her quiet,
pretty dress; ``and so it was for this, my sweet Lucie, that I brought
you across the Channel, such a baby' Lord bless me' How little I
thought what I was doing!  How lightly I valued the obligation I was
conferring on my friend Mr.\ Charles!''

``You didn't mean it,'' remarked the matter-of-fact Miss Pross, ``and
therefore how could you know it?  Nonsense!''

``Really?  Well; but don't cry,'' said the gentle Mr.\ Lorry.

``I am not crying,'' said Miss Pross; ``\emph{you} are.''

``I, my Pross?'' (By this time, Mr.\ Lorry dared to be pleasant with
her, on occasion.)

``You were, just now; I saw you do it, and I don't wonder at it.  Such
a present of plate as you have made 'em, is enough to bring tears into
anybody's eyes.  There's not a fork or a spoon in the collection,''
said Miss Pross, ``that I didn't cry over, last night after the box came,
till I couldn't see it.''

``I am highly gratified,'' said Mr.\ Lorry, ``though, upon my honour, I
had no intention of rendering those trifling articles of remembrance
invisible to any one.  Dear me!  This is an occasion that makes a man
speculate on all he has lost.  Dear, dear, dear!  To think that there
might have been a Mrs.\ Lorry, any time these fifty years almost!''

``Not at all!''  From Miss Pross.

``You think there never might have been a Mrs.\ Lorry?'' asked the
gentleman of that name.

``Pooh!'' rejoined Miss Pross; ``you were a bachelor in your cradle.''

``Well!'' observed Mr.\ Lorry, beamingly adjusting his little wig,
``that seems probable, too.''

``And you were cut out for a bachelor,'' pursued Miss Pross, ``before
you were put in your cradle.''

``Then, I think,'' said Mr.\ Lorry, ``that I was very unhandsomely dealt
with, and that I ought to have had a voice in the selection of my
pattern.  Enough!  Now, my dear Lucie,'' drawing his arm soothingly
round her waist, ``I hear them moving in the next room, and Miss Pross
and I, as two formal folks of business, are anxious not to lose the
final opportunity of saying something to you that you wish to hear.
You leave your good father, my dear, in hands as earnest and as
loving as your own; he shall be taken every conceivable care of;
during the next fortnight, while you are in Warwickshire and thereabouts,
even Tellson's shall go to the wall (comparatively speaking) before him.
And when, at the fortnight's end, he comes to join you and your beloved
husband, on your other fortnight's trip in Wales, you shall say that
we have sent him to you in the best health and in the happiest frame.
Now, I hear Somebody's step coming to the door.  Let me kiss my dear
girl with an old-fashioned bachelor blessing, before Somebody comes
to claim his own.''

For a moment, he held the fair face from him to look at the
well-remembered expression on the forehead, and then laid the bright
golden hair against his little brown wig, with a genuine tenderness and
delicacy which, if such things be old-fashioned, were as old as Adam.

The door of the Doctor's room opened, and he came out with Charles
Darnay.  He was so deadly pale---which had not been the case when they
went in together---that no vestige of colour was to be seen in his face.
But, in the composure of his manner he was unaltered, except that to
the shrewd glance of Mr.\ Lorry it disclosed some shadowy indication
that the old air of avoidance and dread had lately passed over him,
like a cold wind.

He gave his arm to his daughter, and took her down-stairs to the chariot
which Mr.\ Lorry had hired in honour of the day.  The rest followed in
another carriage, and soon, in a neighbouring church, where no strange
eyes looked on, Charles Darnay and Lucie Manette were happily married.

Besides the glancing tears that shone among the smiles of the little
group when it was done, some diamonds, very bright and sparkling,
glanced on the bride's hand, which were newly released from the dark
obscurity of one of Mr.\ Lorry's pockets.  They returned home to
breakfast, and all went well, and in due course the golden hair that
had mingled with the poor shoemaker's white locks in the Paris garret,
were mingled with them again in the morning sunlight, on the threshold
of the door at parting.

It was a hard parting, though it was not for long.  But her father
cheered her, and said at last, gently disengaging himself from her
enfolding arms, ``Take her, Charles!  She is yours!''

And her agitated hand waved to them from a chaise window, and
she was gone.

The corner being out of the way of the idle and curious, and the
preparations having been very simple and few, the Doctor, Mr.\ Lorry,
and Miss Pross, were left quite alone.  It was when they turned into
the welcome shade of the cool old hall, that Mr.\ Lorry observed a
great change to have come over the Doctor; as if the golden arm
uplifted there, had struck him a poisoned blow.

He had naturally repressed much, and some revulsion might have been
expected in him when the occasion for repression was gone.  But, it
was the old scared lost look that troubled Mr.\ Lorry; and through
his absent manner of clasping his head and drearily wandering away
into his own room when they got up-stairs, Mr.\ Lorry was reminded of
Defarge the wine-shop keeper, and the starlight ride.

``I think,'' he whispered to Miss Pross, after anxious consideration,
``I think we had best not speak to him just now, or at all disturb him.
I must look in at Tellson's; so I will go there at once and come back
presently.  Then, we will take him a ride into the country, and dine
there, and all will be well.''

It was easier for Mr.\ Lorry to look in at Tellson's, than to look
out of Tellson's.  He was detained two hours.  When he came back,
he ascended the old staircase alone, having asked no question of
the servant; going thus into the Doctor's rooms, he was stopped by
a low sound of knocking.

``Good God!'' he said, with a start.  ``What's that?''

Miss Pross, with a terrified face, was at his ear.  ``O me, O me!
All is lost!'' cried she, wringing her hands.  ``What is to be told
to Ladybird?  He doesn't know me, and is making shoes!''

Mr.\ Lorry said what he could to calm her, and went himself into the
Doctor's room.  The bench was turned towards the light, as it had
been when he had seen the shoemaker at his work before, and his head
was bent down, and he was very busy.

``Doctor Manette.  My dear friend, Doctor Manette!''

The Doctor looked at him for a moment---half inquiringly, half as if
he were angry at being spoken to---and bent over his work again.

He had laid aside his coat and waistcoat; his shirt was open at the
throat, as it used to be when he did that work; and even the old
haggard, faded surface of face had come back to him.  He worked hard---%
impatiently---as if in some sense of having been interrupted.

Mr.\ Lorry glanced at the work in his hand, and observed that it was
a shoe of the old size and shape.  He took up another that was lying
by him, and asked what it was.

``A young lady's walking shoe,'' he muttered, without looking up.
``It ought to have been finished long ago.  Let it be.''

``But, Doctor Manette.  Look at me!''

He obeyed, in the old mechanically submissive manner, without
pausing in his work.

``You know me, my dear friend?  Think again.  This is not your proper
occupation.  Think, dear friend!''

Nothing would induce him to speak more.  He looked up, for an instant
at a time, when he was requested to do so; but, no persuasion would
extract a word from him.  He worked, and worked, and worked, in silence,
and words fell on him as they would have fallen on an echoless wall,
or on the air.  The only ray of hope that Mr.\ Lorry could discover,
was, that he sometimes furtively looked up without being asked.  In that,
there seemed a faint expression of curiosity or perplexity---as though
he were trying to reconcile some doubts in his mind.

Two things at once impressed themselves on Mr.\ Lorry, as important
above all others; the first, that this must be kept secret from Lucie;
the second, that it must be kept secret from all who knew him.  In
conjunction with Miss Pross, he took immediate steps towards the
latter precaution, by giving out that the Doctor was not well, and
required a few days of complete rest.  In aid of the kind deception
to be practised on his daughter, Miss Pross was to write, describing
his having been called away professionally, and referring to an
imaginary letter of two or three hurried lines in his own hand,
represented to have been addressed to her by the same post.

These measures, advisable to be taken in any case, Mr.\ Lorry took in
the hope of his coming to himself.  If that should happen soon, he kept
another course in reserve; which was, to have a certain opinion that he
thought the best, on the Doctor's case.

In the hope of his recovery, and of resort to this third course being
thereby rendered practicable, Mr.\ Lorry resolved to watch him
attentively, with as little appearance as possible of doing so.
He therefore made arrangements to absent himself from Tellson's for the
first time in his life, and took his post by the window in the same room.

He was not long in discovering that it was worse than useless to speak
to him, since, on being pressed, he became worried.  He abandoned that
attempt on the first day, and resolved merely to keep himself always
before him, as a silent protest against the delusion into which he had
fallen, or was falling.  He remained, therefore, in his seat near the
window, reading and writing, and expressing in as many pleasant and
natural ways as he could think of, that it was a free place.

Doctor Manette took what was given him to eat and drink, and worked on,
that first day, until it was too dark to see---worked on, half an hour
after Mr.\ Lorry could not have seen, for his life, to read or write.
When he put his tools aside as useless, until morning, Mr.\ Lorry rose
and said to him:

``Will you go out?''

He looked down at the floor on either side of him in the old manner,
looked up in the old manner, and repeated in the old low voice:

``Out?''

``Yes; for a walk with me.  Why not?''

He made no effort to say why not, and said not a word more.  But,
Mr.\ Lorry thought he saw, as he leaned forward on his bench in the
dusk, with his elbows on his knees and his head in his hands, that he
was in some misty way asking himself, ``Why not?''  The sagacity of the
man of business perceived an advantage here, and determined to hold it.

Miss Pross and he divided the night into two watches, and observed him
at intervals from the adjoining room.  He paced up and down for a long
time before he lay down; but, when he did finally lay himself down,
he fell asleep.  In the morning, he was up betimes, and went straight
to his bench and to work.

On this second day, Mr.\ Lorry saluted him cheerfully by his name, and
spoke to him on topics that had been of late familiar to them.  He
returned no reply, but it was evident that he heard what was said,
and that he thought about it, however confusedly.  This encouraged
Mr.\ Lorry to have Miss Pross in with her work, several times during the
day; at those times, they quietly spoke of Lucie, and of her father then
present, precisely in the usual manner, and as if there were nothing
amiss.  This was done without any demonstrative accompaniment, not long
enough, or often enough to harass him; and it lightened Mr.\ Lorry's
friendly heart to believe that he looked up oftener, and that he appeared
to be stirred by some perception of inconsistencies surrounding him.

When it fell dark again, Mr.\ Lorry asked him as before:

``Dear Doctor, will you go out?''

As before, he repeated, ``Out?''

``Yes; for a walk with me.  Why not?''

This time, Mr.\ Lorry feigned to go out when he could extract no answer
from him, and, after remaining absent for an hour, returned.  In the
meanwhile, the Doctor had removed to the seat in the window, and had
sat there looking down at the plane-tree; but, on Mr.\ Lorry's return,
be slipped away to his bench.

The time went very slowly on, and Mr.\ Lorry's hope darkened, and his
heart grew heavier again, and grew yet heavier and heavier every day.
The third day came and went, the fourth, the fifth.  Five days, six
days, seven days, eight days, nine days.

With a hope ever darkening, and with a heart always growing heavier
and heavier, Mr.\ Lorry passed through this anxious time.  The secret
was well kept, and Lucie was unconscious and happy; but he could not
fail to observe that the shoemaker, whose hand had been a little out
at first, was growing dreadfully skilful, and that he had never been
so intent on his work, and that his hands had never been so nimble and
expert, as in the dusk of the ninth evening.



\chapter{An Opinion}


Worn out by anxious watching, Mr.\ Lorry fell asleep at his post.  On
the tenth morning of his suspense, he was startled by the shining of
the sun into the room where a heavy slumber had overtaken him when it
was dark night.

He rubbed his eyes and roused himself; but he doubted, when he had
done so, whether he was not still asleep.  For, going to the door of
the Doctor's room and looking in, he perceived that the shoemaker's
bench and tools were put aside again, and that the Doctor himself sat
reading at the window.  He was in his usual morning dress, and his face
(which Mr.\ Lorry could distinctly see), though still very pale, was
calmly studious and attentive.

Even when he had satisfied himself that he was awake, Mr.\ Lorry felt
giddily uncertain for some few moments whether the late shoemaking
might not be a disturbed dream of his own; for, did not his eyes show
him his friend before him in his accustomed clothing and aspect, and
employed as usual; and was there any sign within their range, that the
change of which he had so strong an impression had actually happened?

It was but the inquiry of his first confusion and astonishment, the
answer being obvious.  If the impression were not produced by a real
corresponding and sufficient cause, how came he, Jarvis Lorry, there?
How came he to have fallen asleep, in his clothes, on the sofa in
Doctor Manette's consulting-room, and to be debating these points
outside the Doctor's bedroom door in the early morning?

Within a few minutes, Miss Pross stood whispering at his side.  If he
had had any particle of doubt left, her talk would of necessity have
resolved it; but he was by that time clear-headed, and had none.  He
advised that they should let the time go by until the regular
breakfast-hour, and should then meet the Doctor as if nothing unusual
had occurred.  If he appeared to be in his customary state of mind,
Mr.\ Lorry would then cautiously proceed to seek direction and guidance
from the opinion he had been, in his anxiety, so anxious to obtain.

Miss Pross, submitting herself to his judgment, the scheme was worked
out with care.  Having abundance of time for his usual methodical
toilette, Mr.\ Lorry presented himself at the breakfast-hour in his
usual white linen, and with his usual neat leg.  The Doctor was
summoned in the usual way, and came to breakfast.

So far as it was possible to comprehend him without overstepping
those delicate and gradual approaches which Mr.\ Lorry felt to be the
only safe advance, he at first supposed that his daughter's marriage
had taken place yesterday.  An incidental allusion, purposely thrown
out, to the day of the week, and the day of the month, set him thinking
and counting, and evidently made him uneasy.  In all other respects,
however, he was so composedly himself, that Mr.\ Lorry determined to
have the aid he sought.  And that aid was his own.

Therefore, when the breakfast was done and cleared away, and he and
the Doctor were left together, Mr.\ Lorry said, feelingly:

``My dear Manette, I am anxious to have your opinion, in confidence,
on a very curious case in which I am deeply interested; that is to say,
it is very curious to me; perhaps, to your better information it may
be less so.''

Glancing at his hands, which were discoloured by his late work, the
Doctor looked troubled, and listened attentively.  He had already
glanced at his hands more than once.

``Doctor Manette,'' said Mr.\ Lorry, touching him affectionately on the
arm, ``the case is the case of a particularly dear friend of mine.
Pray give your mind to it, and advise me well for his sake---and
above all, for his daughter's---his daughter's, my dear Manette.''

``If I understand,'' said the Doctor, in a subdued tone, ``some mental
shock---?''

``Yes!''

``Be explicit,'' said the Doctor.  ``Spare no detail.''

Mr.\ Lorry saw that they understood one another, and proceeded.

``My dear Manette, it is the case of an old and a prolonged shock, of
great acuteness and severity to the affections, the feelings,
the---the---as you express it---the mind.  The mind.  It is the case of
a shock under which the sufferer was borne down, one cannot say for
how long, because I believe he cannot calculate the time himself, and
there are no other means of getting at it.  It is the case of a shock
from which the sufferer recovered, by a process that he cannot trace
himself---as I once heard him publicly relate in a striking manner.
It is the case of a shock from which he has recovered, so completely,
as to be a highly intelligent man, capable of close application of mind,
and great exertion of body, and of constantly making fresh additions to
his stock of knowledge, which was already very large.  But, unfortunately,
there has been,'' he paused and took a deep breath---``a slight relapse.''

The Doctor, in a low voice, asked, ``Of how long duration?''

``Nine days and nights.''

``How did it show itself?  I infer,'' glancing at his hands again,
``in the resumption of some old pursuit connected with the shock?''

``That is the fact.''

``Now, did you ever see him,'' asked the Doctor, distinctly and
collectedly, though in the same low voice, ``engaged in that
pursuit originally?''

``Once.''

``And when the relapse fell on him, was he in most respects---or in
all respects---as he was then?''

``I think in all respects.''

``You spoke of his daughter.  Does his daughter know of the relapse?''

``No.  It has been kept from her, and I hope will always be kept from
her.  It is known only to myself, and to one other who may be trusted.''

The Doctor grasped his hand, and murmured, ``That was very kind.
That was very thoughtful!''  Mr.\ Lorry grasped his hand in return,
and neither of the two spoke for a little while.

``Now, my dear Manette,'' said Mr.\ Lorry, at length, in his most
considerate and most affectionate way, ``I am a mere man of business,
and unfit to cope with such intricate and difficult matters.  I do
not possess the kind of information necessary; I do not possess the
kind of intelligence; I want guiding.  There is no man in this world
on whom I could so rely for right guidance, as on you.  Tell me, how
does this relapse come about?  Is there danger of another?  Could a
repetition of it be prevented?  How should a repetition of it be
treated?  How does it come about at all?  What can I do for my friend?
No man ever can have been more desirous in his heart to serve a friend,
than I am to serve mine, if I knew how.

But I don't know how to originate, in such a case.  If your sagacity,
knowledge, and experience, could put me on the right track, I might be
able to do so much; unenlightened and undirected, I can do so little.
Pray discuss it with me; pray enable me to see it a little more clearly,
and teach me how to be a little more useful.``

Doctor Manette sat meditating after these earnest words were spoken,
and Mr.\ Lorry did not press him.

``I think it probable,'' said the Doctor, breaking silence with an
effort, ``that the relapse you have described, my dear friend, was
not quite unforeseen by its subject.''

``Was it dreaded by him?'' Mr.\ Lorry ventured to ask.

``Very much.''  He said it with an involuntary shudder.

``You have no idea how such an apprehension weighs on the sufferer's
mind, and how difficult---how almost impossible---it is, for him to force
himself to utter a word upon the topic that oppresses him.''

``Would he,'' asked Mr.\ Lorry, ``be sensibly relieved if he could
prevail upon himself to impart that secret brooding to any one,
when it is on him?''

``I think so.  But it is, as I have told you, next to impossible.
I even believe it---in some cases---to be quite impossible.''

``Now,'' said Mr.\ Lorry, gently laying his hand on the Doctor's arm
again, after a short silence on both sides, ``to what would you refer
this attack?''

``I believe,'' returned Doctor Manette, ``that there had been a strong
and extraordinary revival of the train of thought and remembrance that
was the first cause of the malady.  Some intense associations of a
most distressing nature were vividly recalled, I think.  It is probable
that there had long been a dread lurking in his mind, that those
associations would be recalled---say, under certain circumstances---say,
on a particular occasion.  He tried to prepare himself in vain; perhaps
the effort to prepare himself made him less able to bear it.''

``Would he remember what took place in the relapse?'' asked Mr.\ Lorry,
with natural hesitation.

The Doctor looked desolately round the room, shook his head, and
answered, in a low voice, ``Not at all.''

``Now, as to the future,'' hinted Mr.\ Lorry.

``As to the future,'' said the Doctor, recovering firmness, ``I should
have great hope.  As it pleased Heaven in its mercy to restore him so
soon, I should have great hope.  He, yielding under the pressure of a
complicated something, long dreaded and long vaguely foreseen and
contended against, and recovering after the cloud had burst and passed,
I should hope that the worst was over.''

``Well, well!  That's good comfort.  I am thankful!'' said Mr.\ Lorry.

``I am thankful!'' repeated the Doctor, bending his head with reverence.

``There are two other points,'' said Mr.\ Lorry, ``on which I am anxious
to be instructed.  I may go on?''

``You cannot do your friend a better service.''  The Doctor gave him
his hand.

``To the first, then.  He is of a studious habit, and unusually
energetic; he applies himself with great ardour to the acquisition
of professional knowledge, to the conducting of experiments, to
many things.  Now, does he do too much?''

``I think not.  It may be the character of his mind, to be always in
singular need of occupation.  That may be, in part, natural to it; in
part, the result of affliction.  The less it was occupied with healthy
things, the more it would be in danger of turning in the unhealthy
direction.  He may have observed himself, and made the discovery.''

``You are sure that he is not under too great a strain?''

``I think I am quite sure of it.''

``My dear Manette, if he were overworked now---''

``My dear Lorry, I doubt if that could easily be.  There has been a
violent stress in one direction, and it needs a counterweight.''

``Excuse me, as a persistent man of business.  Assuming for a moment,
that he \emph{was} overworked; it would show itself in some renewal of this disorder?''

``I do not think so.  I do not think,'' said Doctor Manette with the
firmness of self-conviction, ``that anything but the one train of
association would renew it.  I think that, henceforth, nothing but
some extraordinary jarring of that chord could renew it.  After what
has happened, and after his recovery, I find it difficult to imagine
any such violent sounding of that string again.  I trust, and I almost
believe, that the circumstances likely to renew it are exhausted.''

He spoke with the diffidence of a man who knew how slight a thing
would overset the delicate organisation of the mind, and yet with the
confidence of a man who had slowly won his assurance out of personal
endurance and distress.  It was not for his friend to abate that
confidence.  He professed himself more relieved and encouraged than he
really was, and approached his second and last point.  He felt it to
be the most difficult of all; but, remembering his old Sunday morning
conversation with Miss Pross, and remembering what he had seen in the
last nine days, he knew that he must face it.

``The occupation resumed under the influence of this passing affliction
so happily recovered from,'' said Mr.\ Lorry, clearing his throat, ``we will
call---Blacksmith's work, Blacksmith's work.  We will say, to put a case
and for the sake of illustration, that he had been used, in his bad time,
to work at a little forge.  We will say that he was unexpectedly found
at his forge again.  Is it not a pity that he should keep it by him?''

The Doctor shaded his forehead with his hand, and beat his foot nervously
on the ground.

``He has always kept it by him,'' said Mr.\ Lorry, with an anxious look
at his friend.  ``Now, would it not be better that he should let it go?''

Still, the Doctor, with shaded forehead, beat his foot nervously on
the ground.

``You do not find it easy to advise me?'' said Mr.\ Lorry.  ``I quite
understand it to be a nice question.  And yet I think---'' And there he
shook his head, and stopped.

``You see,'' said Doctor Manette, turning to him after an uneasy pause,
``it is very hard to explain, consistently, the innermost workings of
this poor man's mind.  He once yearned so frightfully for that
occupation, and it was so welcome when it came; no doubt it relieved
his pain so much, by substituting the perplexity of the fingers for
the perplexity of the brain, and by substituting, as he became more
practised, the ingenuity of the hands, for the ingenuity of the
mental torture; that he has never been able to bear the thought of
putting it quite out of his reach.  Even now, when I believe he is
more hopeful of himself than he has ever been, and even speaks of
himself with a kind of confidence, the idea that he might need that
old employment, and not find it, gives him a sudden sense of terror,
like that which one may fancy strikes to the heart of a lost child.''

He looked like his illustration, as he raised his eyes to
Mr.\ Lorry's face.

``But may not---mind!  I ask for information, as a plodding man of
business who only deals with such material objects as guineas,
shillings, and bank-notes---may not the retention of the thing involve
the retention of the idea?  If the thing were gone, my dear Manette,
might not the fear go with it?  In short, is it not a concession to
the misgiving, to keep the forge?''

There was another silence.

``You see, too,'' said the Doctor, tremulously, ``it is such an
old companion.''

``I would not keep it,'' said Mr.\ Lorry, shaking his head; for he gained
in firmness as he saw the Doctor disquieted.  ``I would recommend him
to sacrifice it.  I only want your authority.  I am sure it does no
good.  Come!  Give me your authority, like a dear good man.  For his
daughter's sake, my dear Manette!''

Very strange to see what a struggle there was within him!

``In her name, then, let it be done; I sanction it.  But, I would not
take it away while he was present.  Let it be removed when he is not
there; let him miss his old companion after an absence.''

Mr.\ Lorry readily engaged for that, and the conference was ended.
They passed the day in the country, and the Doctor was quite restored.
On the three following days he remained perfectly well, and on the
fourteenth day he went away to join Lucie and her husband.  The
precaution that had been taken to account for his silence, Mr.\ Lorry
had previously explained to him, and he had written to Lucie in
accordance with it, and she had no suspicions.

On the night of the day on which he left the house, Mr.\ Lorry went
into his room with a chopper, saw, chisel, and hammer, attended by
Miss Pross carrying a light.  There, with closed doors, and in a
mysterious and guilty manner, Mr.\ Lorry hacked the shoemaker's bench
to pieces, while Miss Pross held the candle as if she were assisting
at a murder---for which, indeed, in her grimness, she was no unsuitable
figure.  The burning of the body (previously reduced to pieces
convenient for the purpose) was commenced without delay in the kitchen
fire; and the tools, shoes, and leather, were buried in the garden.
So wicked do destruction and secrecy appear to honest minds, that
Mr.\ Lorry and Miss Pross, while engaged in the commission of their
deed and in the removal of its traces, almost felt, and almost looked,
like accomplices in a horrible crime.



\chapter{A Plea}


When the newly-married pair came home, the first person who appeared,
to offer his congratulations, was Sydney Carton.  They had not been
at home many hours, when he presented himself.  He was not improved in
habits, or in looks, or in manner; but there was a certain rugged air of
fidelity about him, which was new to the observation of Charles Darnay.

He watched his opportunity of taking Darnay aside into a window, and
of speaking to him when no one overheard.

``Mr.\ Darnay,'' said Carton, ``I wish we might be friends.''

``We are already friends, I hope.''

``You are good enough to say so, as a fashion of speech; but, I don't
mean any fashion of speech.  Indeed, when I say I wish we might be friends,
I scarcely mean quite that, either.''

Charles Darnay---as was natural---asked him, in all good-humour and
good-fellowship, what he did mean?

``Upon my life,'' said Carton, smiling, ``I find that easier to comprehend
in my own mind, than to convey to yours.  However, let me try.  You
remember a certain famous occasion when I was more drunk than---%
than usual?''

``I remember a certain famous occasion when you forced me to confess
that you had been drinking.''

``I remember it too.  The curse of those occasions is heavy upon me,
for I always remember them.  I hope it may be taken into account one
day, when all days are at an end for me!  Don't be alarmed;
I am not going to preach.''

``I am not at all alarmed.  Earnestness in you, is anything but
alarming to me.''

``Ah!'' said Carton, with a careless wave of his hand, as if he waved
that away.  ``On the drunken occasion in question (one of a large number,
as you know), I was insufferable about liking you, and not liking you.
I wish you would forget it.''

``I forgot it long ago.''

``Fashion of speech again!  But, Mr.\ Darnay, oblivion is not so easy to
me, as you represent it to be to you.  I have by no means forgotten it,
and a light answer does not help me to forget it.''

``If it was a light answer,'' returned Darnay, ``I beg your forgiveness
for it. I had no other object than to turn a slight thing, which,
to my surprise, seems to trouble you too much, aside.  I declare to you,
on the faith of a gentleman, that I have long dismissed it from my mind.
Good Heaven, what was there to dismiss!  Have I had nothing more
important to remember, in the great service you rendered me that day?''

``As to the great service,'' said Carton, ``I am bound to avow to you,
when you speak of it in that way, that it was mere professional
claptrap, I don't know that I cared what became of you, when I
rendered  it.---Mind!  I say when I rendered it; I am speaking of the past.''

``You make light of the obligation,'' returned Darnay, ``but I will not
quarrel with \emph{your} light answer.''

``Genuine truth, Mr.\ Darnay, trust me!  I have gone aside from my
purpose; I was speaking about our being friends.  Now, you know me;
you know I am incapable of all the higher and better flights of men.
If you doubt it, ask Stryver, and he'll tell you so.''

``I prefer to form my own opinion, without the aid of his.''

``Well!  At any rate you know me as a dissolute dog, who has never
done any good, and never will.''

``I don't know that you `never will.'\,''

``But I do, and you must take my word for it.  Well!  If you could
endure to have such a worthless fellow, and a fellow of such indifferent
reputation, coming and going at odd times, I should ask that I might be
permitted to come and go as a privileged person here; that I might be
regarded as an useless (and I would add, if it were not for the
resemblance I detected between you and me, an unornamental) piece of
furniture, tolerated for its old service, and taken no notice of.
I doubt if I should abuse the permission.  It is a hundred to one
if I should avail myself of it four times in a year.  It would satisfy me,
I dare say, to know that I had it.''

``Will you try?''

``That is another way of saying that I am placed on the footing I have
indicated.  I thank you, Darnay.  I may use that freedom with your name?''

``I think so, Carton, by this time.''

They shook hands upon it, and Sydney turned away.  Within a minute
afterwards, he was, to all outward appearance, as unsubstantial as ever.

When he was gone, and in the course of an evening passed with Miss Pross,
the Doctor, and Mr.\ Lorry, Charles Darnay made some mention of this
conversation in general terms, and spoke of Sydney Carton as a problem
of carelessness and recklessness.  He spoke of him, in short, not
bitterly or meaning to bear hard upon him, but as anybody might who
saw him as he showed himself.

He had no idea that this could dwell in the thoughts of his fair young
wife; but, when he afterwards joined her in their own rooms, he found
her waiting for him with the old pretty lifting of the forehead
strongly marked.

``We are thoughtful to-night!'' said Darnay, drawing his arm about her.

``Yes, dearest Charles,'' with her hands on his breast, and the
inquiring and attentive expression fixed upon him; ``we are rather
thoughtful to-night, for we have something on our mind to-night.''

``What is it, my Lucie?''

``Will you promise not to press one question on me, if I beg you
not to ask it?''

``Will I promise?  What will I not promise to my Love?''

What, indeed, with his hand putting aside the golden hair from the
cheek, and his other hand against the heart that beat for him!

``I think, Charles, poor Mr.\ Carton deserves more consideration and
respect than you expressed for him to-night.''

``Indeed, my own?  Why so?''

``That is what you are not to ask me.  But I think---I know---he does.''

``If you know it, it is enough.  What would you have me do, my Life?''

``I would ask you, dearest, to be very generous with him always, and
very lenient on his faults when he is not by.  I would ask you to
believe that he has a heart he very, very seldom reveals, and that there
are deep wounds in it.  My dear, I have seen it bleeding.''

``It is a painful reflection to me,'' said Charles Darnay, quite astounded,
``that I should have done him any wrong.  I never thought this of him.''

``My husband, it is so.  I fear he is not to be reclaimed; there is
scarcely a hope that anything in his character or fortunes is reparable
now.  But, I am sure that he is capable of good things, gentle things,
even magnanimous things.''

She looked so beautiful in the purity of her faith in this lost man,
that her husband could have looked at her as she was for hours.

``And, O my dearest Love!'' she urged, clinging nearer to him, laying
her head upon his breast, and raising her eyes to his, ``remember how
strong we are in our happiness, and how weak he is in his misery!''

The supplication touched him home.  ``I will always remember it, dear
Heart!  I will remember it as long as I live.''

He bent over the golden head, and put the rosy lips to his, and folded
her in his arms.  If one forlorn wanderer then pacing the dark streets,
could have heard her innocent disclosure, and could have seen the drops
of pity kissed away by her husband from the soft blue eyes so loving of
that husband, he might have cried to the night---and the words would not
have parted from his lips for the first time---%

``God bless her for her sweet compassion!''



\chapter{Echoing Footsteps}


A wonderful corner for echoes, it has been remarked, that corner where
the Doctor lived.  Ever busily winding the golden thread which bound
her husband, and her father, and herself, and her old directress and
companion, in a life of quiet bliss, Lucie sat in the still house in the
tranquilly resounding corner, listening to the echoing footsteps of years.

At first, there were times, though she was a perfectly happy young
wife, when her work would slowly fall from her hands, and her eyes
would be dimmed.  For, there was something coming in the echoes,
something light, afar off, and scarcely audible yet, that stirred
her heart too much.  Fluttering hopes and doubts---hopes, of a love as
yet unknown to her:  doubts, of her remaining upon earth, to enjoy that
new delight---divided her breast.  Among the echoes then, there would
arise the sound of footsteps at her own early grave; and thoughts of
the husband who would be left so desolate, and who would mourn for
her so much, swelled to her eyes, and broke like waves.

That time passed, and her little Lucie lay on her bosom.  Then,
among the advancing echoes, there was the tread of her tiny feet and
the sound of her prattling words.  Let greater echoes resound as they
would, the young mother at the cradle side could always hear those
coming.  They came, and the shady house was sunny with a child's laugh,
and the Divine friend of children, to whom in her trouble she had
confided hers, seemed to take her child in his arms, as He took the
child of old, and made it a sacred joy to her.

Ever busily winding the golden thread that bound them all together,
weaving the service of her happy influence through the tissue of all
their lives, and making it predominate nowhere, Lucie heard in the
echoes of years none but friendly and soothing sounds.  Her husband's
step was strong and prosperous among them; her father's firm and equal.
Lo, Miss Pross, in harness of string, awakening the echoes, as an
unruly charger, whip-corrected, snorting and pawing the earth under
the plane-tree in the garden!

Even when there were sounds of sorrow among the rest, they were not
harsh nor cruel.  Even when golden hair, like her own, lay in a halo
on a pillow round the worn face of a little boy, and he said, with a
radiant smile, ``Dear papa and mamma, I am very sorry to leave you both,
and to leave my pretty sister; but I am called, and I must go!''
those were not tears all of agony that wetted his young mother's cheek,
as the spirit departed from her embrace that had been entrusted to it.
Suffer them and forbid them not.  They see my Father's face.
O Father, blessed words!

Thus, the rustling of an Angel's wings got blended with the other
echoes, and they were not wholly of earth, but had in them that breath
of Heaven.  Sighs of the winds that blew over a little garden-tomb were
mingled with them also, and both were audible to Lucie, in a hushed
murmur---like the breathing of a summer sea asleep upon a sandy shore%
---as the little Lucie, comically studious at the task of the morning,
or dressing a doll at her mother's footstool, chattered in the
tongues of the Two Cities that were blended in her life.

The Echoes rarely answered to the actual tread of Sydney Carton.
Some half-dozen times a year, at most, he claimed his privilege of coming
in uninvited, and would sit among them through the evening, as he had
once done often.  He never came there heated with wine.  And one other
thing regarding him was whispered in the echoes, which has been
whispered by all true echoes for ages and ages.

No man ever really loved a woman, lost her, and knew her with a
blameless though an unchanged mind, when she was a wife and a mother,
but her children had a strange sympathy with him---an instinctive
delicacy of pity for him.  What fine hidden sensibilities are touched
in such a case, no echoes tell; but it is so, and it was so here.
Carton was the first stranger to whom little Lucie held out her chubby
arms, and he kept his place with her as she grew.  The little boy had
spoken of him, almost at the last.  ``Poor Carton!  Kiss him for me!''

Mr.\ Stryver shouldered his way through the law, like some great engine
forcing itself through turbid water, and dragged his useful friend in
his wake, like a boat towed astern.  As the boat so favoured is usually
in a rough plight, and mostly under water, so, Sydney had a swamped life
of it.  But, easy and strong custom, unhappily so much easier and
stronger in him than any stimulating sense of desert or disgrace, made
it the life he was to lead; and he no more thought of emerging from his
state of lion's jackal, than any real jackal may be supposed to think
of rising to be a lion.  Stryver was rich; had married a florid widow
with property and three boys, who had nothing particularly shining about
them but the straight hair of their dumpling heads.

These three young gentlemen, Mr.\ Stryver, exuding patronage of the most
offensive quality from every pore, had walked before him like three
sheep to the quiet corner in Soho, and had offered as pupils to Lucie's
husband:  delicately saying ``Halloa! here are three lumps of bread-and-%
cheese towards your matrimonial picnic, Darnay!''  The polite rejection
of the three lumps of bread-and-cheese had quite bloated Mr.\ Stryver
with indignation, which he afterwards turned to account in the training
of the young gentlemen, by directing them to beware of the pride of
Beggars, like that tutor-fellow.  He was also in the habit of declaiming
to Mrs.\ Stryver, over his full-bodied wine, on the arts Mrs.\ Darnay had
once put in practice to ``catch'' him, and on the diamond-cut-diamond
arts in himself, madam, which had rendered him ``not to be caught.''
Some of his King's Bench familiars, who were occasionally parties
to the full-bodied wine and the lie, excused him for the latter by saying
that he had told it so often, that he believed it himself---which is
surely such an incorrigible aggravation of an originally bad offence,
as to justify any such offender's being carried off to some suitably
retired spot, and there hanged out of the way.

These were among the echoes to which Lucie, sometimes pensive,
sometimes amused and laughing, listened in the echoing corner, until
her little daughter was six years old.  How near to her heart the echoes
of her child's tread came, and those of her own dear father's, always
active and self-possessed, and those of her dear husband's, need not
be told.  Nor, how the lightest echo of their united home, directed
by herself with such a wise and elegant thrift that it was more
abundant than any waste, was music to her.  Nor, how there were echoes
all about her, sweet in her ears, of the many times her father had
told her that he found her more devoted to him married (if that could be)
than single, and of the many times her husband had said to her that no
cares and duties seemed to divide her love for him or her help to him,
and asked her ``What is the magic secret, my darling, of your being
everything to all of us, as if there were only one of us,
yet never seeming to be hurried, or to have too much to do?''

But, there were other echoes, from a distance, that rumbled menacingly
in the corner all through this space of time.  And it was now, about
little Lucie's sixth birthday, that they began to have an awful sound,
as of a great storm in France with a dreadful sea rising.

On a night in mid-July, one thousand seven hundred and eighty-nine,
Mr.\ Lorry came in late, from Tellson's, and sat himself down by Lucie
and her husband in the dark window.  It was a hot, wild night, and
they were all three reminded of the old Sunday night when they had
looked at the lightning from the same place.

``I began to think,'' said Mr.\ Lorry, pushing his brown wig back, ``that
I should have to pass the night at Tellson's.  We have been so full of
business all day, that we have not known what to do first, or which
way to turn.  There is such an uneasiness in Paris, that we have
actually a run of confidence upon us!  Our customers over there, seem
not to be able to confide their property to us fast enough.  There is
positively a mania among some of them for sending it to England.''

``That has a bad look,'' said Darnay---%

``A bad look, you say, my dear Darnay?  Yes, but we don't know what
reason there is in it.  People are so unreasonable!  Some of us at
Tellson's are getting old, and we really can't be troubled out of
the ordinary course without due occasion.''

``Still,'' said Darnay, ``you know how gloomy and threatening the sky is.''

``I know that, to be sure,'' assented Mr.\ Lorry, trying to persuade
himself that his sweet temper was soured, and that he grumbled,
``but I am determined to be peevish after my long day's botheration.
Where is Manette?''

``Here he is,'' said the Doctor, entering the dark room at the moment.

``I am quite glad you are at home; for these hurries and forebodings by
which I have been surrounded all day long, have made me nervous
without reason.  You are not going out, I hope?''

``No; I am going to play backgammon with you, if you like,''
said the Doctor.

``I don't think I do like, if I may speak my mind.  I am not fit to
be pitted against you to-night.  Is the teaboard still there, Lucie?
I can't see.''

``Of course, it has been kept for you.''

``Thank ye, my dear.  The precious child is safe in bed?''

``And sleeping soundly.''

``That's right; all safe and well!  I don't know why anything should
be otherwise than safe and well here, thank God; but I have been so
put out all day, and I am not as young as I was!  My tea, my dear!
Thank ye.  Now, come and take your place in the circle, and let us
sit quiet, and hear the echoes about which you have your theory.''

``Not a theory; it was a fancy.''

``A fancy, then, my wise pet,'' said Mr.\ Lorry, patting her hand.  ``They
are very numerous and very loud, though, are they not?  Only hear them!''

Headlong, mad, and dangerous footsteps to force their way into anybody's
life, footsteps not easily made clean again if once stained red, the
footsteps raging in Saint Antoine afar off, as the little circle sat
in the dark London window.

Saint Antoine had been, that morning, a vast dusky mass of scarecrows
heaving to and fro, with frequent gleams of light above the billowy
heads, where steel blades and bayonets shone in the sun.  A tremendous
roar arose from the throat of Saint Antoine, and a forest of naked arms
struggled in the air like shrivelled branches of trees in a winter wind:
all the fingers convulsively clutching at every weapon or semblance of
a weapon that was thrown up from the depths below, no matter how far off.

Who gave them out, whence they last came, where they began, through
what agency they crookedly quivered and jerked, scores at a time, over
the heads of the crowd, like a kind of lightning, no eye in the throng
could have told; but, muskets were being distributed---so were
cartridges, powder, and ball, bars of iron and wood, knives, axes,
pikes, every weapon that distracted ingenuity could discover or devise.
People who could lay hold of nothing else, set themselves with bleeding
hands to force stones and bricks out of their places in walls.  Every
pulse and heart in Saint Antoine was on high-fever strain and at
high-fever heat.  Every living creature there held life as of no account,
and was demented with a passionate readiness to sacrifice it.

As a whirlpool of boiling waters has a centre point, so, all this raging
circled round Defarge's wine-shop, and every human drop in the caldron
had a tendency to be sucked towards the vortex where Defarge himself,
already begrimed with gunpowder and sweat, issued orders, issued arms,
thrust this man back, dragged this man forward, disarmed one to arm
another, laboured and strove in the thickest of the uproar.

``Keep near to me, Jacques Three,'' cried Defarge; ``and do you,
Jacques One and Two, separate and put yourselves at the head of
as many of these patriots as you can.  Where is my wife?''

``Eh, well!  Here you see me!'' said madame, composed as ever, but not
knitting to-day.  Madame's resolute right hand was occupied with an axe,
in place of the usual softer implements, and in her girdle were a pistol
and a cruel knife.

``Where do you go, my wife?''

``I go,'' said madame, ``with you at present.  You shall see me at the
head of women, by-and-bye.''

``Come, then!'' cried Defarge, in a resounding voice.  ``Patriots and
friends, we are ready!  The Bastille!''

With a roar that sounded as if all the breath in France had been
shaped into the detested word, the living sea rose, wave on wave,
depth on depth, and overflowed the city to that point.  Alarm-bells
ringing, drums beating, the sea raging and thundering on its new beach,
the attack began.

Deep ditches, double drawbridge, massive stone walls, eight great
towers, cannon, muskets, fire and smoke.  Through the fire and through
the smoke---in the fire and in the smoke, for the sea cast him up against
a cannon, and on the instant he became a cannonier---Defarge of the
wine-shop worked like a manful soldier, Two fierce hours.

Deep ditch, single drawbridge, massive stone walls, eight great towers,
cannon, muskets, fire and smoke.  One drawbridge down!  ``Work, comrades
all, work!  Work, Jacques One, Jacques Two, Jacques One Thousand,
Jacques Two Thousand, Jacques Five-and-Twenty Thousand; in the name of
all the Angels or the Devils---which you prefer---work!''  Thus Defarge
of the wine-shop, still at his gun, which had long grown hot.

``To me, women!'' cried madame his wife.  ``What!  We can kill as well as
the men when the place is taken!''  And to her, with a shrill thirsty cry,
trooping women variously armed, but all armed alike in hunger and revenge.

Cannon, muskets, fire and smoke; but, still the deep ditch, the single
drawbridge, the massive stone walls, and the eight great towers.  Slight
displacements of the raging sea, made by the falling wounded.  Flashing
weapons, blazing torches, smoking waggonloads of wet straw, hard work
at neighbouring barricades in all directions, shrieks, volleys,
execrations, bravery without stint, boom smash and rattle, and the
furious sounding of the living sea; but, still the deep ditch, and the
single drawbridge, and the massive stone walls, and the eight great
towers, and still Defarge of the wine-shop at his gun, grown doubly
hot by the service of Four fierce hours.

A white flag from within the fortress, and a parley---this dimly
perceptible through the raging storm, nothing audible in it---suddenly
the sea rose immeasurably wider and higher, and swept Defarge of the
wine-shop over the lowered drawbridge, past the massive stone outer
walls, in among the eight great towers surrendered!

So resistless was the force of the ocean bearing him on, that even
to draw his breath or turn his head was as impracticable as if he had
been struggling in the surf at the South Sea, until he was landed in
the outer courtyard of the Bastille.  There, against an angle of a
wall, he made a struggle to look about him.  Jacques Three was nearly
at his side; Madame Defarge, still heading some of her women, was
visible in the inner distance, and her knife was in her hand.  Everywhere
was tumult, exultation, deafening and maniacal bewilderment, astounding
noise, yet furious dumb-show.

``The Prisoners!''

``The Records!''

``The secret cells!''

``The instruments of torture!''

``The Prisoners!''

Of all these cries, and ten thousand incoherences, ``The Prisoners!''
was the cry most taken up by the sea that rushed in, as if there were
an eternity of people, as well as of time and space.  When the foremost
billows rolled past, bearing the prison officers with them, and
threatening them all with instant death if any secret nook remained
undisclosed, Defarge laid his strong hand on the breast of one of
these men---a man with a grey head, who had a lighted torch in his hand---%
separated him from the rest, and got him between himself and the wall.

``Show me the North Tower!'' said Defarge.  ``Quick!''

``I will faithfully,'' replied the man, ``if you will come with me.  But
there is no one there.''

``What is the meaning of One Hundred and Five, North Tower?''
asked Defarge.  ``Quick!''

``The meaning, monsieur?''

``Does it mean a captive, or a place of captivity?  Or do you mean that
I shall strike you dead?''

``Kill him!'' croaked Jacques Three, who had come close up.

``Monsieur, it is a cell.''

``Show it me!''

``Pass this way, then.''

Jacques Three, with his usual craving on him, and evidently
disappointed by the dialogue taking a turn that did not seem to promise
bloodshed, held by Defarge's arm as he held by the turnkey's.  Their
three heads had been close together during this brief discourse, and
it had been as much as they could do to hear one another, even then:
so tremendous was the noise of the living ocean, in its irruption into
the Fortress, and its inundation of the courts and passages and
staircases.  All around outside, too, it beat the walls with a deep,
hoarse roar, from which, occasionally, some partial shouts of tumult
broke and leaped into the air like spray.

Through gloomy vaults where the light of day had never shone, past
hideous doors of dark dens and cages, down cavernous flights of steps,
and again up steep rugged ascents of stone and brick, more like dry
waterfalls than staircases, Defarge, the turnkey, and Jacques Three,
linked hand and arm, went with all the speed they could make.  Here
and there, especially at first, the inundation started on them and
swept by; but when they had done descending, and were winding and
climbing up a tower, they were alone.  Hemmed in here by the massive
thickness of walls and arches, the storm within the fortress and without
was only audible to them in a dull, subdued way, as if the noise out of
which they had come had almost destroyed their sense of hearing.

The turnkey stopped at a low door, put a key in a clashing lock,
swung the door slowly open, and said, as they all bent their heads
and passed in:

``One hundred and five, North Tower!''

There was a small, heavily-grated, unglazed window high in the wall,
with a stone screen before it, so that the sky could be only seen by
stooping low and looking up.  There was a small chimney, heavily barred
across, a few feet within.  There was a heap of old feathery wood-ashes
on the hearth.  There was a stool, and table, and a straw bed.  There
were the four blackened walls, and a rusted iron ring in one of them.

``Pass that torch slowly along these walls, that I may see them,''
said Defarge to the turnkey.

The man obeyed, and Defarge followed the light closely with his eyes.

``Stop!---Look here, Jacques!''

``A. M.!'' croaked Jacques Three, as he read greedily.

``Alexandre Manette,'' said Defarge in his ear, following the letters
with his swart forefinger, deeply engrained with gunpowder.  ``And here
he wrote `a poor physician.' And it was he, without doubt, who scratched
a calendar on this stone.  What is that in your hand?  A crowbar?
Give it me!''

He had still the linstock of his gun in his own hand.  He made a
sudden exchange of the two instruments, and turning on the worm-eaten
stool and table, beat them to pieces in a few blows.

``Hold the light higher!'' he said, wrathfully, to the turnkey.
``Look among those fragments with care, Jacques.  And see!  Here is my knife,''
throwing it to him; ``rip open that bed, and search the straw.
Hold the light higher, you!''

With a menacing look at the turnkey he crawled upon the hearth,
and, peering up the chimney, struck and prised at its sides with the
crowbar, and worked at the iron grating across it.  In a few minutes,
some mortar and dust came dropping down, which he averted his face to
avoid; and in it, and in the old wood-ashes, and in a crevice in the
chimney into which his weapon had slipped or wrought itself, he groped
with a cautious touch.

``Nothing in the wood, and nothing in the straw, Jacques?''

``Nothing.''

``Let us collect them together, in the middle of the cell.  So!
Light them, you!''

The turnkey fired the little pile, which blazed high and hot.  Stooping
again to come out at the low-arched door, they left it burning, and
retraced their way to the courtyard; seeming to recover their sense of
hearing as they came down, until they were in the raging flood once more.

They found it surging and tossing, in quest of Defarge himself.
Saint Antoine was clamorous to have its wine-shop keeper foremost in
the guard upon the governor who had defended the Bastille and shot the
people.  Otherwise, the governor would not be marched to the Hotel de
Ville for judgment.  Otherwise, the governor would escape, and the
people's blood (suddenly of some value, after many years of
worthlessness) be unavenged.

In the howling universe of passion and contention that seemed to
encompass this grim old officer conspicuous in his grey coat and red
decoration, there was but one quite steady figure, and that was a
woman's.  ``See, there is my husband!'' she cried, pointing him out.
``See Defarge!''  She stood immovable close to the grim old officer,
and remained immovable close to him; remained immovable close to him
through the streets, as Defarge and the rest bore him along; remained
immovable close to him when he was got near his destination, and began
to be struck at from behind; remained immovable close to him when the
long-gathering rain of stabs and blows fell heavy; was so close to him
when he dropped dead under it, that, suddenly animated, she put her foot
upon his neck, and with her cruel knife---long ready---hewed off his head.

The hour was come, when Saint Antoine was to execute his horrible idea
of hoisting up men for lamps to show what he could be and do.  Saint
Antoine's blood was up, and the blood of tyranny and domination by
the iron hand was down---down on the steps of the Hotel de Ville where
the governor's body lay---down on the sole of the shoe of Madame Defarge
where she had trodden on the body to steady it for mutilation.
``Lower the lamp yonder!'' cried Saint Antoine, after glaring round for a
new means of death; ``here is one of his soldiers to be left on guard!''
The swinging sentinel was posted, and the sea rushed on.

The sea of black and threatening waters, and of destructive upheaving
of wave against wave, whose depths were yet unfathomed and whose
forces were yet unknown.  The remorseless sea of turbulently swaying
shapes, voices of vengeance, and faces hardened in the furnaces of
suffering until the touch of pity could make no mark on them.

But, in the ocean of faces where every fierce and furious expression
was in vivid life, there were two groups of faces---each seven in number%
---so fixedly contrasting with the rest, that never did sea roll which
bore more memorable wrecks with it.  Seven faces of prisoners, suddenly
released by the storm that had burst their tomb, were carried high
overhead:  all scared, all lost, all wondering and amazed, as if the
Last Day were come, and those who rejoiced around them were lost spirits.
Other seven faces there were, carried higher, seven dead faces, whose
drooping eyelids and half-seen eyes awaited the Last Day.  Impassive
faces, yet with a suspended---not an abolished---expression on them; faces,
rather, in a fearful pause, as having yet to raise the dropped lids of
the eyes, and bear witness with the bloodless lips, \emph{``Thou didst it!''}

Seven prisoners released, seven gory heads on pikes, the keys of the
accursed fortress of the eight strong towers, some discovered letters
and other memorials of prisoners of old time, long dead of broken
hearts,---such, and such---like, the loudly echoing footsteps of Saint
Antoine escort through the Paris streets in mid-July, one thousand seven
hundred and eighty-nine.  Now, Heaven defeat the fancy of Lucie Darnay,
and keep these feet far out of her life!  For, they are headlong, mad,
and dangerous; and in the years so long after the breaking of the cask
at Defarge's wine-shop door, they are not easily purified when once
stained red.



\chapter{The Sea Still Rises}


Haggard Saint Antoine had had only one exultant week, in which to
soften his modicum of hard and bitter bread to such extent as he
could, with the relish of fraternal embraces and congratulations,
when Madame Defarge sat at her counter, as usual, presiding over the
customers. Madame Defarge wore no rose in her head, for the great
brotherhood of Spies had become, even in one short week, extremely
chary of trusting themselves to the saint's mercies.  The lamps across
his streets had a portentously elastic swing with them.

Madame Defarge, with her arms folded, sat in the morning light and heat,
contemplating the wine-shop and the street.  In both, there were several
knots of loungers, squalid and miserable, but now with a manifest sense
of power enthroned on their distress.  The raggedest nightcap, awry on
the wretchedest head, had this crooked significance in it:  ``I know how
hard it has grown for me, the wearer of this, to support life in myself;
but do you know how easy it has grown for me, the wearer of this, to
destroy life in you?''  Every lean bare arm, that had been without work
before, had this work always ready for it now, that it could strike.
The fingers of the knitting women were vicious, with the experience that
they could tear.  There was a change in the appearance of Saint Antoine;
the image had been hammering into this for hundreds of years, and the
last finishing blows had told mightily on the expression.

Madame Defarge sat observing it, with such suppressed approval as was
to be desired in the leader of the Saint Antoine women.  One of her
sisterhood knitted beside her.  The short, rather plump wife of a
starved grocer, and the mother of two children withal, this lieutenant
had already earned the complimentary name of The Vengeance.

``Hark!'' said The Vengeance.  ``Listen, then!  Who comes?''

As if a train of powder laid from the outermost bound of Saint Antoine
Quarter to the wine-shop door, had been suddenly fired, a fast-spreading
murmur came rushing along.

``It is Defarge,'' said madame.  ``Silence, patriots!''

Defarge came in breathless, pulled off a red cap he wore, and looked
around him!  ``Listen, everywhere!'' said madame again.  ``Listen to him!''
Defarge stood, panting, against a background of eager eyes and open
mouths, formed outside the door; all those within the wine-shop had
sprung to their feet.

``Say then, my husband.  What is it?''

``News from the other world!''

``How, then?'' cried madame, contemptuously.  ``The other world?''

``Does everybody here recall old Foulon, who told the famished people
that they might eat grass, and who died, and went to Hell?''

``Everybody!'' from all throats.

``The news is of him.  He is among us!''

``Among us!'' from the universal throat again.  ``And dead?''

``Not dead!  He feared us so much---and with reason---that he caused
himself to be represented as dead, and had a grand mock-funeral.  But
they have found him alive, hiding in the country, and have brought him
in.  I have seen him but now, on his way to the Hotel de Ville, a
prisoner.  I have said that he had reason to fear us.  Say all!
\emph{Had} he reason?''

Wretched old sinner of more than threescore years and ten, if he had
never known it yet, he would have known it in his heart of hearts if
he could have heard the answering cry.

A moment of profound silence followed.  Defarge and his wife looked
steadfastly at one another.  The Vengeance stooped, and the jar of
a drum was heard as she moved it at her feet behind the counter.

``Patriots!'' said Defarge, in a determined voice, ``are we ready?''

Instantly Madame Defarge's knife was in her girdle; the drum was beating
in the streets, as if it and a drummer had flown together by magic; and
The Vengeance, uttering terrific shrieks, and flinging her arms about
her head like all the forty Furies at once, was tearing from house to
house, rousing the women.

The men were terrible, in the bloody-minded anger with which they looked
from windows, caught up what arms they had, and came pouring down into
the streets; but, the women were a sight to chill the boldest.  From
such household occupations as their bare poverty yielded, from their
children, from their aged and their sick crouching on the bare ground
famished and naked, they ran out with streaming hair, urging one
another, and themselves, to madness with the wildest cries and actions.
Villain Foulon taken, my sister!  Old Foulon taken, my mother!
Miscreant Foulon taken, my daughter!  Then, a score of others ran into
the midst of these, beating their breasts, tearing their hair, and
screaming, Foulon alive!  Foulon who told the starving people they
might eat grass!  Foulon who told my old father that he might eat
grass, when I had no bread to give him!  Foulon who told my baby it
might suck grass, when these breasts where dry with want!  O mother
of God, this Foulon!  O Heaven our suffering!  Hear me, my dead baby
and my withered father:  I swear on my knees, on these stones, to avenge
you on Foulon!  Husbands, and brothers, and young men, Give us the blood
of Foulon, Give us the head of Foulon, Give us the heart of Foulon,
Give us the body and soul of Foulon, Rend Foulon to pieces, and dig
him into the ground, that grass may grow from him!  With these cries,
numbers of the women, lashed into blind frenzy, whirled about, striking
and tearing at their own friends until they dropped into a passionate
swoon, and were only saved by the men belonging to them from being
trampled under foot.

Nevertheless, not a moment was lost; not a moment!  This Foulon was
at the Hotel de Ville, and might be loosed.  Never, if Saint Antoine
knew his own sufferings, insults, and wrongs!  Armed men and women
flocked out of the Quarter so fast, and drew even these last dregs
after them with such a force of suction, that within a quarter of an
hour there was not a human creature in Saint Antoine's bosom but a
few old crones and the wailing children.

No. They were all by that time choking the Hall of Examination where
this old man, ugly and wicked, was, and overflowing into the adjacent
open space and streets.  The Defarges, husband and wife, The Vengeance,
and Jacques Three, were in the first press, and at no great distance
from him in the Hall.

``See!'' cried madame, pointing with her knife.  ``See the old villain
bound with ropes.  That was well done to tie a bunch of grass upon
his back.  Ha, ha!  That was well done.  Let him eat it now!''  Madame
put her knife under her arm, and clapped her hands as at a play.

The people immediately behind Madame Defarge, explaining the cause of
her satisfaction to those behind them, and those again explaining
to others, and those to others, the neighbouring streets resounded with
the clapping of hands.  Similarly, during two or three hours of drawl,
and the winnowing of many bushels of words, Madame Defarge's frequent
expressions of impatience were taken up, with marvellous quickness,
at a distance:  the more readily, because certain men who had by some
wonderful exercise of agility climbed up the external architecture to
look in from the windows, knew Madame Defarge well, and acted as a
telegraph between her and the crowd outside the building.

At length the sun rose so high that it struck a kindly ray as of hope
or protection, directly down upon the old prisoner's head.  The favour
was too much to bear; in an instant the barrier of dust and chaff that
had stood surprisingly long, went to the winds, and Saint Antoine had
got him!

It was known directly, to the furthest confines of the crowd.  Defarge
had but sprung over a railing and a table, and folded the miserable
wretch in a deadly embrace---Madame Defarge had but followed and turned
her hand in one of the ropes with which he was tied---The Vengeance
and Jacques Three were not yet up with them, and the men at the windows
had not yet swooped into the Hall, like birds of prey from their high
perches---when the cry seemed to go up, all over the city, ``Bring him
out!  Bring him to the lamp!''

Down, and up, and head foremost on the steps of the building; now, on
his knees; now, on his feet; now, on his back; dragged, and struck at,
and stifled by the bunches of grass and straw that were thrust into his
face by hundreds of hands; torn, bruised, panting, bleeding, yet always
entreating and beseeching for mercy; now full of vehement agony of
action, with a small clear space about him as the people drew one
another back that they might see; now, a log of dead wood drawn through
a forest of legs; he was hauled to the nearest street corner where one
of the fatal lamps swung, and there Madame Defarge let him go---as a
cat might have done to a mouse---and silently and composedly looked
at him while they made ready, and while he besought her:  the women
passionately screeching at him all the time, and the men sternly
calling out to have him killed with grass in his mouth.  Once, he went
aloft, and the rope broke, and they caught him shrieking; twice, he went
aloft, and the rope broke, and they caught him shrieking; then, the rope
was merciful, and held him, and his head was soon upon a pike, with
grass enough in the mouth for all Saint Antoine to dance at the sight of.

Nor was this the end of the day's bad work, for Saint Antoine so
shouted and danced his angry blood up, that it boiled again, on
hearing when the day closed in that the son-in-law of the despatched,
another of the people's enemies and insulters, was coming into Paris
under a guard five hundred strong, in cavalry alone.  Saint Antoine
wrote his crimes on flaring sheets of paper, seized him---would have
torn him out of the breast of an army to bear Foulon company---set
his head and heart on pikes, and carried the three spoils of the day,
in Wolf-procession through the streets.

Not before dark night did the men and women come back to the children,
wailing and breadless.  Then, the miserable bakers' shops were beset
by long files of them, patiently waiting to buy bad bread; and while
they waited with stomachs faint and empty, they beguiled the time by
embracing one another on the triumphs of the day, and achieving them
again in gossip.  Gradually, these strings of ragged people shortened
and frayed away; and then poor lights began to shine in high windows,
and slender fires were made in the streets, at which neighbours cooked
in common, afterwards supping at their doors.

Scanty and insufficient suppers those, and innocent of meat, as of
most other sauce to wretched bread.  Yet, human fellowship infused
some nourishment into the flinty viands, and struck some sparks of
cheerfulness out of them.  Fathers and mothers who had had their full
share in the worst of the day, played gently with their meagre
children; and lovers, with such a world around them and before them,
loved and hoped.

It was almost morning, when Defarge's wine-shop parted with its last
knot of customers, and Monsieur Defarge said to madame his wife, in
husky tones, while fastening the door:

``At last it is come, my dear!''

``Eh well!'' returned madame.  ``Almost.''

Saint Antoine slept, the Defarges slept:  even The Vengeance slept with
her starved grocer, and the drum was at rest.  The drum's was the only
voice in Saint Antoine that blood and hurry had not changed.  The
Vengeance, as custodian of the drum, could have wakened him up and had
the same speech out of him as before the Bastille fell, or old Foulon
was seized; not so with the hoarse tones of the men and women in Saint
Antoine's bosom.



\chapter{Fire Rises}


There was a change on the village where the fountain fell, and where
the mender of roads went forth daily to hammer out of the stones on
the highway such morsels of bread as might serve for patches to hold
his poor ignorant soul and his poor reduced body together.  The prison
on the crag was not so dominant as of yore; there were soldiers to guard
it, but not many; there were officers to guard the soldiers, but not
one of them knew what his men would do---beyond this:  that it would
probably not be what he was ordered.

Far and wide lay a ruined country, yielding nothing but desolation.
Every green leaf, every blade of grass and blade of grain, was as
shrivelled and poor as the miserable people.  Everything was bowed
down, dejected, oppressed, and broken.  Habitations, fences,
domesticated animals, men, women, children, and the soil that bore
them---all worn out.

Monseigneur (often a most worthy individual gentleman) was a national
blessing, gave a chivalrous tone to things, was a polite example of
luxurious and shining fife, and a great deal more to equal purpose;
nevertheless, Monseigneur as a class had, somehow or other, brought
things to this.  Strange that Creation, designed expressly for
Monseigneur, should be so soon wrung dry and squeezed out!  There must
be something short-sighted in the eternal arrangements, surely!  Thus
it was, however; and the last drop of blood having been extracted from
the flints, and the last screw of the rack having been turned so often
that its purchase crumbled, and it now turned and turned with nothing
to bite, Monseigneur began to run away from a phenomenon so low
and unaccountable.

But, this was not the change on the village, and on many a village
like it.  For scores of years gone by, Monseigneur had squeezed it
and wrung it, and had seldom graced it with his presence except for
the pleasures of the chase---now, found in hunting the people; now,
found in hunting the beasts, for whose preservation Monseigneur made
edifying spaces of barbarous and barren wilderness.  No.  The change
consisted in the appearance of strange faces of low caste, rather than
in the disappearance of the high caste, chiselled, and otherwise
beautified and beautifying features of Monseigneur.

For, in these times, as the mender of roads worked, solitary, in the
dust, not often troubling himself to reflect that dust he was and to
dust he must return, being for the most part too much occupied in
thinking how little he had for supper and how much more he would eat
if he had it---in these times, as he raised his eyes from his lonely
labour, and viewed the prospect, he would see some rough figure
approaching on foot, the like of which was once a rarity in those
parts, but was now a frequent presence.  As it advanced, the mender
of roads would discern without surprise, that it was a shaggy-haired
man, of almost barbarian aspect, tall, in wooden shoes that were
clumsy even to the eyes of a mender of roads, grim, rough, swart,
steeped in the mud and dust of many highways, dank with the marshy
moisture of many low grounds, sprinkled with the thorns and leaves
and moss of many byways through woods.

Such a man came upon him, like a ghost, at noon in the July weather,
as he sat on his heap of stones under a bank, taking such shelter as
he could get from a shower of hail.

The man looked at him, looked at the village in the hollow, at the
mill, and at the prison on the crag.  When he had identified these
objects in what benighted mind he had, he said, in a dialect that
was just intelligible:

``How goes it, Jacques?''

``All well, Jacques.''

``Touch then!''

They joined hands, and the man sat down on the heap of stones.

``No dinner?''

``Nothing but supper now,'' said the mender of roads, with a hungry face.

``It is the fashion,'' growled the man.  ``I meet no dinner anywhere.''

He took out a blackened pipe, filled it, lighted it with flint and
steel, pulled at it until it was in a bright glow:  then, suddenly held
it from him and dropped something into it from between his finger and
thumb, that blazed and went out in a puff of smoke.

``Touch then.''  It was the turn of the mender of roads to say it this
time, after observing these operations.  They again joined hands.

``To-night?'' said the mender of roads.

``To-night,'' said the man, putting the pipe in his mouth.

``Where?''

``Here.''

He and the mender of roads sat on the heap of stones looking silently
at one another, with the hail driving in between them like a pigmy
charge of bayonets, until the sky began to clear over the village.

``Show me!'' said the traveller then, moving to the brow of the hill.

``See!'' returned the mender of roads, with extended finger.  ``You go
down here, and straight through the street, and past the fountain---''

``To the Devil with all that!'' interrupted the other, rolling his eye
over the landscape.  ``\emph{I} go through no streets and past no fountains.
Well?''

``Well!  About two leagues beyond the summit of that hill above
the village.''

``Good.  When do you cease to work?''

``At sunset.''

``Will you wake me, before departing?  I have walked two nights without
resting.  Let me finish my pipe, and I shall sleep like a child.  Will
you wake me?''

``Surely.''

The wayfarer smoked his pipe out, put it in his breast, slipped off
his great wooden shoes, and lay down on his back on the heap of stones.
He was fast asleep directly.

As the road-mender plied his dusty labour, and the hail-clouds, rolling
away, revealed bright bars and streaks of sky which were responded to
by silver gleams upon the landscape, the little man (who wore a red cap
now, in place of his blue one) seemed fascinated by the figure on the
heap of stones.  His eyes were so often turned towards it, that he
used his tools mechanically, and, one would have said, to very poor
account.  The bronze face, the shaggy black hair and beard, the coarse
woollen red cap, the rough medley dress of home-spun stuff and hairy
skins of beasts, the powerful frame attenuated by spare living, and
the sullen and desperate compression of the lips in sleep, inspired
the mender of roads with awe.  The traveller had travelled far, and
his feet were footsore, and his ankles chafed and bleeding; his great
shoes, stuffed with leaves and grass, had been heavy to drag over the
many long leagues, and his clothes were chafed into holes, as he himself
was into sores.  Stooping down beside him, the road-mender tried to
get a peep at secret weapons in his breast or where not; but, in vain,
for he slept with his arms crossed upon him, and set as resolutely as
his lips.  Fortified towns with their stockades, guard-houses, gates,
trenches, and drawbridges, seemed to the mender of roads, to be so much
air as against this figure.  And when he lifted his eyes from it to
the horizon and looked around, he saw in his small fancy similar figures,
stopped by no obstacle, tending to centres all over France.

The man slept on, indifferent to showers of hail and intervals of
brightness, to sunshine on his face and shadow, to the paltering lumps
of dull ice on his body and the diamonds into which the sun changed
them, until the sun was low in the west, and the sky was glowing.
Then, the mender of roads having got his tools together and all things
ready to go down into the village, roused him.

``Good!'' said the sleeper, rising on his elbow.  ``Two leagues beyond
the summit of the hill?''

``About.''

``About.  Good!''

The mender of roads went home, with the dust going on before him
according to the set of the wind, and was soon at the fountain,
squeezing himself in among the lean kine brought there to drink, and
appearing even to whisper to them in his whispering to all the village.
When the village had taken its poor supper, it did not creep to bed,
as it usually did, but came out of doors again, and remained there.
A curious contagion of whispering was upon it, and also, when it
gathered together at the fountain in the dark, another curious contagion
of looking expectantly at the sky in one direction only.  Monsieur
Gabelle, chief functionary of the place, became uneasy; went out on
his house-top alone, and looked in that direction too; glanced down
from behind his chimneys at the darkening faces by the fountain below,
and sent word to the sacristan who kept the keys of the church, that
there might be need to ring the tocsin by-and-bye.

The night deepened.  The trees environing the old chateau, keeping
its solitary state apart, moved in a rising wind, as though they
threatened the pile of building massive and dark in the gloom.  Up
the two terrace flights of steps the rain ran wildly, and beat at
the great door, like a swift messenger rousing those within; uneasy
rushes of wind went through the hall, among the old spears and knives,
and passed lamenting up the stairs, and shook the curtains of the bed
where the last Marquis had slept.  East, West, North, and South, through
the woods, four heavy-treading, unkempt figures crushed the high grass
and cracked the branches, striding on cautiously to come together in
the courtyard.  Four lights broke out there, and moved away in different
directions, and all was black again.

But, not for long.  Presently, the chateau began to make itself
strangely visible by some light of its own, as though it were growing
luminous.  Then, a flickering streak played behind the architecture
of the front, picking out transparent places, and showing where
balustrades, arches, and windows were.  Then it soared higher, and
grew broader and brighter.  Soon, from a score of the great windows,
flames burst forth, and the stone faces awakened, stared out of fire.

A faint murmur arose about the house from the few people who were left
there, and there was a saddling of a horse and riding away.  There was
spurring and splashing through the darkness, and bridle was drawn in
the space by the village fountain, and the horse in a foam stood at
Monsieur Gabelle's door.  ``Help, Gabelle!  Help, every one!''  The
tocsin rang impatiently, but other help (if that were any) there was
none.  The mender of roads, and two hundred and fifty particular
friends, stood with folded arms at the fountain, looking at the pillar
of fire in the sky.  ``It must be forty feet high,'' said they, grimly;
and never moved.

The rider from the chateau, and the horse in a foam, clattered away
through the village, and galloped up the stony steep, to the prison
on the crag.  At the gate, a group of officers were looking at the
fire; removed from them, a group of soldiers.  ``Help, gentlemen---%
officers!  The chateau is on fire; valuable objects may be saved from
the flames by timely aid!  Help, help!''  The officers looked towards
the soldiers who looked at the fire; gave no orders; and answered,
with shrugs and biting of lips, ``It must burn.''

As the rider rattled down the hill again and through the street, the
village was illuminating.  The mender of roads, and the two hundred
and fifty particular friends, inspired as one man and woman by the
idea of lighting up, had darted into their houses, and were putting
candles in every dull little pane of glass.  The general scarcity of
everything, occasioned candles to be borrowed in a rather peremptory
manner of Monsieur Gabelle; and in a moment of reluctance and hesitation
on that functionary's part, the mender of roads, once so submissive
to authority, had remarked that carriages were good to make bonfires
with, and that post-horses would roast.

The chateau was left to itself to flame and burn.  In the roaring and
raging of the conflagration, a red-hot wind, driving straight from
the infernal regions, seemed to be blowing the edifice away.  With the
rising and falling of the blaze, the stone faces showed as if they were
in torment.  When great masses of stone and timber fell, the face with
the two dints in the nose became obscured:  anon struggled out of the
smoke again, as if it were the face of the cruel Marquis, burning at
the stake and contending with the fire.

The chateau burned; the nearest trees, laid hold of by the fire,
scorched and shrivelled; trees at a distance, fired by the four fierce
figures, begirt the blazing edifice with a new forest of smoke.  Molten
lead and iron boiled in the marble basin of the fountain; the water
ran dry; the extinguisher tops of the towers vanished like ice before
the heat, and trickled down into four rugged wells of flame.  Great
rents and splits branched out in the solid walls, like crystallisation;
stupefied birds wheeled about and dropped into the furnace; four fierce
figures trudged away, East, West, North, and South, along the night-%
enshrouded roads, guided by the beacon they had lighted, towards their
next destination.  The illuminated village had seized hold of the
tocsin, and, abolishing the lawful ringer, rang for joy.

Not only that; but the village, light-headed with famine, fire, and
bell-ringing, and bethinking itself that Monsieur Gabelle had to do
with the collection of rent and taxes---though it was but a small
instalment of taxes, and no rent at all, that Gabelle had got in those
latter days---became impatient for an interview with him, and,
surrounding his house, summoned him to come forth for personal conference.
Whereupon, Monsieur Gabelle did heavily bar his door, and retire to
hold counsel with himself.  The result of that conference was, that
Gabelle again withdrew himself to his housetop behind his stack of
chimneys; this time resolved, if his door were broken in (he was a
small Southern man of retaliative temperament), to pitch himself head
foremost over the parapet, and crush a man or two below.

Probably, Monsieur Gabelle passed a long night up there, with the
distant chateau for fire and candle, and the beating at his door,
combined with the joy-ringing, for music; not to mention his having
an ill-omened lamp slung across the road before his posting-house gate,
which the village showed a lively inclination to displace in his favour.
A trying suspense, to be passing a whole summer night on the brink of
the black ocean, ready to take that plunge into it upon which Monsieur
Gabelle had resolved!  But, the friendly dawn appearing at last, and
the rush-candles of the village guttering out, the people happily
dispersed, and Monsieur Gabelle came down bringing his life with him
for that while.

Within a hundred miles, and in the light of other fires, there were
other functionaries less fortunate, that night and other nights, whom
the rising sun found hanging across once-peaceful streets, where they
had been born and bred; also, there were other villagers and townspeople
less fortunate than the mender of roads and his fellows, upon whom
the functionaries and soldiery turned with success, and whom they
strung up in their turn.  But, the fierce figures were steadily wending
East, West, North, and South, be that as it would; and whosoever hung,
fire burned.  The altitude of the gallows that would turn to water
and quench it, no functionary, by any stretch of mathematics, was
able to calculate successfully.



\chapter{Drawn to the Loadstone Rock}


In such risings of fire and risings of sea---the firm earth shaken by
the rushes of an angry ocean which had now no ebb, but was always on
the flow, higher and higher, to the terror and wonder of the beholders
on the shore---three years of tempest were consumed.  Three more
birthdays of little Lucie had been woven by the golden thread into
the peaceful tissue of the life of her home.

Many a night and many a day had its inmates listened to the echoes in
the corner, with hearts that failed them when they heard the thronging
feet.  For, the footsteps had become to their minds as the footsteps
of a people, tumultuous under a red flag and with their country declared
in danger, changed into wild beasts, by terrible enchantment long
persisted in.

Monseigneur, as a class, had dissociated himself from the phenomenon
of his not being appreciated:  of his being so little wanted in France,
as to incur considerable danger of receiving his dismissal from it,
and this life together.  Like the fabled rustic who raised the Devil
with infinite pains, and was so terrified at the sight of him that he
could ask the Enemy no question, but immediately fled; so, Monseigneur,
after boldly reading the Lord's Prayer backwards for a great number of
years, and performing many other potent spells for compelling the Evil
One, no sooner beheld him in his terrors than he took to his noble heels.

The shining Bull's Eye of the Court was gone, or it would have been
the mark for a hurricane of national bullets.  It had never been a
good eye to see with---had long had the mote in it of Lucifer's pride,
Sardana---palus's luxury, and a mole's blindness---but it had dropped
out and was gone.  The Court, from that exclusive inner circle to its
outermost rotten ring of intrigue, corruption, and dissimulation, was
all gone together.  Royalty was gone; had been besieged in its Palace
and ``suspended,'' when the last tidings came over.

The August of the year one thousand seven hundred and ninety-two was
come, and Monseigneur was by this time scattered far and wide.

As was natural, the head-quarters and great gathering-place of
Monseigneur, in London, was Tellson's Bank.  Spirits are supposed to
haunt the places where their bodies most resorted, and Monseigneur
without a guinea haunted the spot where his guineas used to be.
Moreover, it was the spot to which such French intelligence as was
most to be relied upon, came quickest.  Again:  Tellson's was a
munificent house, and extended great liberality to old customers who
had fallen from their high estate.  Again:  those nobles who had seen
the coming storm in time, and anticipating plunder or confiscation,
had made provident remittances to Tellson's, were always to be heard
of there by their needy brethren.  To which it must be added that every
new-comer from France reported himself and his tidings at Tellson's,
almost as a matter of course.  For such variety of reasons, Tellson's
was at that time, as to French intelligence, a kind of High Exchange;
and this was so well known to the public, and the inquiries made there
were in consequence so numerous, that Tellson's sometimes wrote the
latest news out in a line or so and posted it in the Bank windows,
for all who ran through Temple Bar to read.

On a steaming, misty afternoon, Mr.\ Lorry sat at his desk, and Charles
Darnay stood leaning on it, talking with him in a low voice.  The
penitential den once set apart for interviews with the House, was now
the news-Exchange, and was filled to overflowing.  It was within half
an hour or so of the time of closing.

``But, although you are the youngest man that ever lived,'' said Charles
Darnay, rather hesitating, ``I must still suggest to you---''

``I understand.  That I am too old?'' said Mr.\ Lorry.

``Unsettled weather, a long journey, uncertain means of travelling, a
disorganised country, a city that may not be even safe for you.''

``My dear Charles,'' said Mr.\ Lorry, with cheerful confidence, ``you
touch some of the reasons for my going:  not for my staying away.
It is safe enough for me; nobody will care to interfere with an old
fellow of hard upon fourscore when there are so many people there
much better worth interfering with.  As to its being a disorganised
city, if it were not a disorganised city there would be no occasion
to send somebody from our House here to our House there, who knows
the city and the business, of old, and is in Tellson's confidence.
As to the uncertain travelling, the long journey, and the winter
weather, if I were not prepared to submit myself to a few inconveniences
for the sake of Tellson's, after all these years, who ought to be?''

``I wish I were going myself,'' said Charles Darnay, somewhat restlessly,
and like one thinking aloud.

``Indeed!  You are a pretty fellow to object and advise!'' exclaimed
Mr.\ Lorry.  ``You wish you were going yourself?  And you a Frenchman
born?  You are a wise counsellor.''

``My dear Mr.\ Lorry, it is because I am a Frenchman born, that the
thought (which I did not mean to utter here, however) has passed
through my mind often.  One cannot help thinking, having had some
sympathy for the miserable people, and having abandoned something to
them,'' he spoke here in his former thoughtful manner, ``that one might
be listened to, and might have the power to persuade to some restraint.
Only last night, after you had left us, when I was talking to Lucie---''

``When you were talking to Lucie,'' Mr.\ Lorry repeated.  ``Yes.  I wonder
you are not ashamed to mention the name of Lucie!  Wishing you were
going to France at this time of day!''

``However, I am not going,'' said Charles Darnay, with a smile.  ``It is
more to the purpose that you say you are.''

``And I am, in plain reality.  The truth is, my dear Charles,'' Mr.\ Lorry
glanced at the distant House, and lowered his voice, ``you can have no
conception of the difficulty with which our business is transacted,
and of the peril in which our books and papers over yonder are involved.
The Lord above knows what the compromising consequences would be to
numbers of people, if some of our documents were seized or destroyed;
and they might be, at any time, you know, for who can say that Paris
is not set afire to-day, or sacked to-morrow!  Now, a judicious selection
from these with the least possible delay, and the burying of them,
or otherwise getting of them out of harm's way, is within the power
(without loss of precious time) of scarcely any one but myself,
if any one.  And shall I hang back, when Tellson's knows this and says
this---Tellson's, whose bread I have eaten these sixty years---because
I am a little stiff about the joints?  Why, I am a boy, sir, to half
a dozen old codgers here!''

``How I admire the gallantry of your youthful spirit, Mr.\ Lorry.''

``Tut!  Nonsense, sir!---And, my dear Charles,'' said Mr.\ Lorry, glancing
at the House again, ``you are to remember, that getting things out of
Paris at this present time, no matter what things, is next to an
impossibility.  Papers and precious matters were this very day brought
to us here (I speak in strict confidence; it is not business-like to
whisper it, even to you), by the strangest bearers you can imagine,
every one of whom had his head hanging on by a single hair as he
passed the Barriers.  At another time, our parcels would come and go,
as easily as in business-like Old England; but now, everything
is stopped.''

``And do you really go to-night?''

``I really go to-night, for the case has become too pressing to
admit of delay.''

``And do you take no one with you?''

``All sorts of people have been proposed to me, but I will have
nothing to say to any of them.  I intend to take Jerry.  Jerry has
been my bodyguard on Sunday nights for a long time past and I am used
to him.  Nobody will suspect Jerry of being anything but an English
bull-dog, or of having any design in his head but to fly at anybody
who touches his master.''

``I must say again that I heartily admire your gallantry and
youthfulness.''

``I must say again, nonsense, nonsense!  When I have executed this
little commission, I shall, perhaps, accept Tellson's proposal to retire
and live at my ease.  Time enough, then, to think about growing old.''

This dialogue had taken place at Mr.\ Lorry's usual desk, with Monseigneur
swarming within a yard or two of it, boastful of what he would do to
avenge himself on the rascal-people before long.  It was too much the
way of Monseigneur under his reverses as a refugee, and it was much
too much the way of native British orthodoxy, to talk of this terrible
Revolution as if it were the only harvest ever known under the skies
that had not been sown---as if nothing had ever been done, or omitted
to be done, that had led to it---as if observers of the wretched
millions in France, and of the misused and perverted resources that
should have made them prosperous, had not seen it inevitably coming,
years before, and had not in plain words recorded what they saw.  Such
vapouring, combined with the extravagant plots of Monseigneur for the
restoration of a state of things that had utterly exhausted itself,
and worn out Heaven and earth as well as itself, was hard to be endured
without some remonstrance by any sane man who knew the truth.  And it
was such vapouring all about his ears, like a troublesome confusion of
blood in his own head, added to a latent uneasiness in his mind, which
had already made Charles Darnay restless, and which still kept him so.

Among the talkers, was Stryver, of the King's Bench Bar, far on his
way to state promotion, and, therefore, loud on the theme:  broaching
to Monseigneur, his devices for blowing the people up and
exterminating them from the face of the earth, and doing without them:
and for accomplishing many similar objects akin in their nature to
the abolition of eagles by sprinkling salt on the tails of the race.
Him, Darnay heard with a particular feeling of objection; and Darnay
stood divided between going away that he might hear no more, and
remaining to interpose his word, when the thing that was to be, went
on to shape itself out.

The House approached Mr.\ Lorry, and laying a soiled and unopened
letter before him, asked if he had yet discovered any traces of the
person to whom it was addressed?  The House laid the letter down so
close to Darnay that he saw the direction---the more quickly because
it was his own right name.  The address, turned into English, ran:

``Very pressing.  To Monsieur heretofore the Marquis St. Evremonde,
of France.  Confided to the cares of Messrs. Tellson and Co., Bankers,
London, England.''

On the marriage morning, Doctor Manette had made it his one urgent
and express request to Charles Darnay, that the secret of this name
should be---unless he, the Doctor, dissolved the obligation---kept
inviolate between them.  Nobody else knew it to be his name; his own
wife had no suspicion of the fact; Mr.\ Lorry could have none.

``No,'' said Mr.\ Lorry, in reply to the House; ``I have referred it,
I think, to everybody now here, and no one can tell me where this
gentleman is to be found.''

The hands of the clock verging upon the hour of closing the Bank,
there was a general set of the current of talkers past Mr.\ Lorry's
desk.  He held the letter out inquiringly; and Monseigneur looked at
it, in the person of this plotting and indignant refugee; and
Monseigneur looked at it in the person of that plotting and indignant
refugee; and This, That, and The Other, all had something disparaging
to say, in French or in English, concerning the Marquis who was not
to be found.

``Nephew, I believe---but in any case degenerate successor---of the
polished Marquis who was murdered,'' said one.  ``Happy to say, I never
knew him.''

``A craven who abandoned his post,'' said another---this Monseigneur
had been got out of Paris, legs uppermost and half suffocated, in a
load of hay---``some years ago.''

``Infected with the new doctrines,'' said a third, eyeing the direction
through his glass in passing; ``set himself in opposition to the last
Marquis, abandoned the estates when he inherited them, and left them
to the ruffian herd.  They will recompense him now, I hope,
as he deserves.''

``Hey?'' cried the blatant Stryver.  ``Did he though?  Is that the sort
of fellow?  Let us look at his infamous name.  D---n the fellow!''

Darnay, unable to restrain himself any longer, touched Mr.\ Stryver on
the shoulder, and said:

``I know the fellow.''

``Do you, by Jupiter?'' said Stryver.  ``I am sorry for it.''

``Why?''

``Why, Mr.\ Darnay?  D'ye hear what he did?  Don't ask, why,
in these times.''

``But I do ask why?''

``Then I tell you again, Mr.\ Darnay, I am sorry for it.  I am sorry to
hear you putting any such extraordinary questions.  Here is a fellow,
who, infected by the most pestilent and blasphemous code of devilry
that ever was known, abandoned his property to the vilest scum of the
earth that ever did murder by wholesale, and you ask me why I am
sorry that a man who instructs youth knows him?  Well, but I'll
answer you.  I am sorry because I believe there is contamination in
such a scoundrel.  That's why.''

Mindful of the secret, Darnay with great difficulty checked himself,
and said:  ``You may not understand the gentleman.''

``I understand how to put \emph{you} in a corner, Mr.\ Darnay,'' said Bully
Stryver, ``and I'll do it.  If this fellow is a gentleman, I \emph{don't}
understand him.  You may tell him so, with my compliments.  You may
also tell him, from me, that after abandoning his worldly goods and
position to this butcherly mob, I wonder he is not at the head of them.
But, no, gentlemen,'' said Stryver, looking all round, and snapping his
fingers, ``I know something of human nature, and I tell you that you'll
never find a fellow like this fellow, trusting himself to the mercies
of such precious \emph{proteges}.  No, gentlemen; he'll always show 'em
a clean pair of heels very early in the scuffle, and sneak away.''

With those words, and a final snap of his fingers, Mr.\ Stryver
shouldered himself into Fleet-street, amidst the general approbation
of his hearers.  Mr.\ Lorry and Charles Darnay were left alone at the
desk, in the general departure from the Bank.

``Will you take charge of the letter?'' said Mr.\ Lorry.  ``You know
where to deliver it?''

``I do.''

``Will you undertake to explain, that we suppose it to have been
addressed here, on the chance of our knowing where to forward it,
and that it has been here some time?''

``I will do so.  Do you start for Paris from here?''

``From here, at eight.''

``I will come back, to see you off.''

Very ill at ease with himself, and with Stryver and most other men,
Darnay made the best of his way into the quiet of the Temple,
opened the letter, and read it.  These were its contents:


``Prison of the Abbaye, Paris.

``June 21, 1792.

``\emph{Monsieur heretofore the Marquis}.

``After having long been in danger of my life at the hands of the
village, I have been seized, with great violence and indignity, and
brought a long journey on foot to Paris.  On the road I have suffered
a great deal.  Nor is that all; my house has been destroyed---razed
to the ground.

``The crime for which I am imprisoned, Monsieur heretofore the
Marquis, and for which I shall be summoned before the tribunal, and
shall lose my life (without your so generous help), is, they tell me,
treason against the majesty of the people, in that I have acted
against them for an emigrant.  It is in vain I represent that I have
acted for them, and not against, according to your commands.  It is
in vain I represent that, before the sequestration of emigrant
property, I had remitted the imposts they had ceased to pay; that I
had collected no rent; that I had had recourse to no process.  The
only response is, that I have acted for an emigrant, and where is
that emigrant?

``Ah! most gracious Monsieur heretofore the Marquis, where is that
emigrant?  I cry in my sleep where is he?  I demand of Heaven, will
he not come to deliver me?  No answer.  Ah Monsieur heretofore the
Marquis, I send my desolate cry across the sea, hoping it may perhaps
reach your ears through the great bank of Tilson known at Paris!

``For the love of Heaven, of justice, of generosity, of the honour of
your noble name, I supplicate you, Monsieur heretofore the Marquis,
to succour and release me.  My fault is, that I have been true to you.
Oh Monsieur heretofore the Marquis, I pray you be you true to me!

``From this prison here of horror, whence I every hour tend nearer
and nearer to destruction, I send you, Monsieur heretofore the Marquis,
the assurance of my dolorous and unhappy service.


``Your afflicted,

``Gabelle.''


The latent uneasiness in Darnay's mind was roused to vigourous life
by this letter.  The peril of an old servant and a good one, whose
only crime was fidelity to himself and his family, stared him so
reproachfully in the face, that, as he walked to and fro in the Temple
considering what to do, he almost hid his face from the passersby.

He knew very well, that in his horror of the deed which had culminated
the bad deeds and bad reputation of the old family house, in his
resentful suspicions of his uncle, and in the aversion with which his
conscience regarded the crumbling fabric that he was supposed to
uphold, he had acted imperfectly.  He knew very well, that in his love
for Lucie, his renunciation of his social place, though by no means
new to his own mind, had been hurried and incomplete.  He knew that
he ought to have systematically worked it out and supervised it, and
that he had meant to do it, and that it had never been done.

The happiness of his own chosen English home, the necessity of being
always actively employed, the swift changes and troubles of the time
which had followed on one another so fast, that the events of this
week annihilated the immature plans of last week, and the events of
the week following made all new again; he knew very well, that to the
force of these circumstances he had yielded:---not without disquiet,
but still without continuous and accumulating resistance.  That he
had watched the times for a time of action, and that they had shifted
and struggled until the time had gone by, and the nobility were
trooping from France by every highway and byway, and their property
was in course of confiscation and destruction, and their very names
were blotting out, was as well known to himself as it could be to any
new authority in France that might impeach him for it.

But, he had oppressed no man, he had imprisoned no man; he was so far
from having harshly exacted payment of his dues, that he had
relinquished them of his own will, thrown himself on a world with no
favour in it, won his own private place there, and earned his own
bread.  Monsieur Gabelle had held the impoverished and involved estate
on written instructions, to spare the people, to give them what little
there was to give---such fuel as the heavy creditors would let them
have in the winter, and such produce as could be saved from the same
grip in the summer---and no doubt he had put the fact in plea and proof,
for his own safety, so that it could not but appear now.

This favoured the desperate resolution Charles Darnay had begun to make,
that he would go to Paris.

Yes.  Like the mariner in the old story, the winds and streams had
driven him within the influence of the Loadstone Rock, and it was
drawing him to itself, and he must go.  Everything that arose before
his mind drifted him on, faster and faster, more and more steadily,
to the terrible attraction.  His latent uneasiness had been, that bad
aims were being worked out in his own unhappy land by bad instruments,
and that he who could not fail to know that he was better than they,
was not there, trying to do something to stay bloodshed, and assert
the claims of mercy and humanity.  With this uneasiness half stifled,
and half reproaching him, he had been brought to the pointed comparison
of himself with the brave old gentleman in whom duty was so strong;
upon that comparison (injurious to himself) had instantly followed
the sneers of Monseigneur, which had stung him bitterly, and those of
Stryver, which above all were coarse and galling, for old reasons.
Upon those, had followed Gabelle's letter:  the appeal of an innocent
prisoner, in danger of death, to his justice, honour, and good name.

His resolution was made.  He must go to Paris.

Yes.  The Loadstone Rock was drawing him, and he must sail on, until
he struck.  He knew of no rock; he saw hardly any danger.  The
intention with which he had done what he had done, even although he
had left it incomplete, presented it before him in an aspect that
would be gratefully acknowledged in France on his presenting himself
to assert it.  Then, that glorious vision of doing good, which is so
often the sanguine mirage of so many good minds, arose before him,
and he even saw himself in the illusion with some influence to guide
this raging Revolution that was running so fearfully wild.

As he walked to and fro with his resolution made, he considered that
neither Lucie nor her father must know of it until he was gone.
Lucie should be spared the pain of separation; and her father, always
reluctant to turn his thoughts towards the dangerous ground of old,
should come to the knowledge of the step, as a step taken, and not in
the balance of suspense and doubt.  How much of the incompleteness of
his situation was referable to her father, through the painful
anxiety to avoid reviving old associations of France in his mind, he
did not discuss with himself.  But, that circumstance too,
had had its influence in his course.

He walked to and fro, with thoughts very busy, until it was time to
return to Tellson's and take leave of Mr.\ Lorry.  As soon as he
arrived in Paris he would present himself to this old friend, but he
must say nothing of his intention now.

A carriage with post-horses was ready at the Bank door, and Jerry
was booted and equipped.

``I have delivered that letter,'' said Charles Darnay to Mr.\ Lorry.
``I would not consent to your being charged with any written answer,
but perhaps you will take a verbal one?''

``That I will, and readily,'' said Mr.\ Lorry, ``if it is not dangerous.''

``Not at all.  Though it is to a prisoner in the Abbaye.''

``What is his name?'' said Mr.\ Lorry, with his open pocket-book in his hand.

``Gabelle.''

``Gabelle.  And what is the message to the unfortunate Gabelle in prison?''

``Simply, `that he has received the letter, and will come.'\,''

``Any time mentioned?''

``He will start upon his journey to-morrow night.''

``Any person mentioned?''

``No.''

He helped Mr.\ Lorry to wrap himself in a number of coats and cloaks,
and went out with him from the warm atmosphere of the old Bank, into
the misty air of Fleet-street.  ``My love to Lucie, and to little
Lucie,'' said Mr.\ Lorry at parting, ``and take precious care of them
till I come back.''  Charles Darnay shook his head and doubtfully smiled,
as the carriage rolled away.

That night---it was the fourteenth of August---he sat up late, and
wrote two fervent letters; one was to Lucie, explaining the strong
obligation he was under to go to Paris, and showing her, at length,
the reasons that he had, for feeling confident that he could become
involved in no personal danger there; the other was to the Doctor,
confiding Lucie and their dear child to his care, and dwelling on
the same topics with the strongest assurances.  To both, he wrote
that he would despatch letters in proof of his safety, immediately
after his arrival.

It was a hard day, that day of being among them, with the first
reservation of their joint lives on his mind.  It was a hard matter
to preserve the innocent deceit of which they were profoundly
unsuspicious.  But, an affectionate glance at his wife, so happy and
busy, made him resolute not to tell her what impended (he had been
half moved to do it, so strange it was to him to act in anything
without her quiet aid), and the day passed quickly.  Early in the
evening he embraced her, and her scarcely less dear namesake, pretending
that he would return by-and-bye (an imaginary engagement took him out,
and he had secreted a valise of clothes ready), and so he emerged
into the heavy mist of the heavy streets, with a heavier heart.

The unseen force was drawing him fast to itself, now, and all the
tides and winds were setting straight and strong towards it.  He left
his two letters with a trusty porter, to be delivered half an hour
before midnight, and no sooner; took horse for Dover; and began his
journey.  ``For the love of Heaven, of justice, of generosity, of the
honour of your noble name!'' was the poor prisoner's cry with which
he strengthened his sinking heart, as he left all that was dear on
earth behind him, and floated away for the Loadstone Rock.



% The end of the second book.




\cleartorecto
\part{Book the Third\\The Track of a Storm}




\chapter{In Secret}


The traveller fared slowly on his way, who fared towards Paris from
England in the autumn of the year one thousand seven hundred and
ninety-two.  More than enough of bad roads, bad equipages, and bad
horses, he would have encountered to delay him, though the fallen and
unfortunate King of France had been upon his throne in all his glory;
but, the changed times were fraught with other obstacles than these.
Every town-gate and village taxing-house had its band of citizen-%
patriots, with their national muskets in a most explosive state of
readiness, who stopped all comers and goers, cross-questioned them,
inspected their papers, looked for their names in lists of their own,
turned them back, or sent them on, or stopped them and laid them in
hold, as their capricious judgment or fancy deemed best for the
dawning Republic One and Indivisible, of Liberty, Equality,
Fraternity, or Death.

A very few French leagues of his journey were accomplished, when
Charles Darnay began to perceive that for him along these country
roads there was no hope of return until he should have been declared
a good citizen at Paris.  Whatever might befall now, he must on to
his journey's end.  Not a mean village closed upon him, not a common
barrier dropped across the road behind him, but he knew it to be
another iron door in the series that was barred between him and
England.  The universal watchfulness so encompassed him, that if he
had been taken in a net, or were being forwarded to his destination
in a cage, he could not have felt his freedom more completely gone.

This universal watchfulness not only stopped him on the highway
twenty times in a stage, but retarded his progress twenty times in a
day, by riding after him and taking him back, riding before him and
stopping him by anticipation, riding with him and keeping him in
charge.  He had been days upon his journey in France alone, when he
went to bed tired out, in a little town on the high road, still a
long way from Paris.

Nothing but the production of the afflicted Gabelle's letter from his
prison of the Abbaye would have got him on so far.  His difficulty at
the guard-house in this small place had been such, that he felt his
journey to have come to a crisis.  And he was, therefore, as little
surprised as a man could be, to find himself awakened at the small
inn to which he had been remitted until morning, in the middle of the
night.

Awakened by a timid local functionary and three armed patriots in
rough red caps and with pipes in their mouths, who sat down on the bed.

``Emigrant,'' said the functionary, ``I am going to send you on to Paris,
under an escort.''

``Citizen, I desire nothing more than to get to Paris, though I could
dispense with the escort.''

``Silence!'' growled a red-cap, striking at the coverlet with the
butt-end of his musket.  ``Peace, aristocrat!''

``It is as the good patriot says,'' observed the timid functionary.
``You are an aristocrat, and must have an escort---and must pay for it.''

``I have no choice,'' said Charles Darnay.

``Choice!  Listen to him!'' cried the same scowling red-cap.  ``As if it
was not a favour to be protected from the lamp-iron!''

``It is always as the good patriot says,'' observed the functionary.
``Rise and dress yourself, emigrant.''

Darnay complied, and was taken back to the guard-house, where other
patriots in rough red caps were smoking, drinking, and sleeping, by a
watch-fire.  Here he paid a heavy price for his escort, and hence he
started with it on the wet, wet roads at three o'clock in the morning.

The escort were two mounted patriots in red caps and tri-coloured
cockades, armed with national muskets and sabres, who rode one on
either side of him.

The escorted governed his own horse, but a loose line was attached to
his bridle, the end of which one of the patriots kept girded round
his wrist. In this state they set forth with the sharp rain driving
in their faces:  clattering at a heavy dragoon trot over the uneven
town pavement, and out upon the mire-deep roads.  In this state they
traversed without change, except of horses and pace, all the mire-%
deep leagues that lay between them and the capital.

They travelled in the night, halting an hour or two after daybreak,
and lying by until the twilight fell.  The escort were so wretchedly
clothed, that they twisted straw round their bare legs, and thatched
their ragged shoulders to keep the wet off.  Apart from the personal
discomfort of being so attended, and apart from such considerations
of present danger as arose from one of the patriots being chronically
drunk, and carrying his musket very recklessly, Charles Darnay did
not allow the restraint that was laid upon him to awaken any serious
fears in his breast; for, he reasoned with himself that it could have
no reference to the merits of an individual case that was not yet
stated, and of representations, confirmable by the prisoner in the
Abbaye, that were not yet made.

But when they came to the town of Beauvais---which they did at
eventide, when the streets were filled with people---he could not
conceal from himself that the aspect of affairs was very alarming.
An ominous crowd gathered to see him dismount of the posting-yard,
and many voices called out loudly, ``Down with the emigrant!''

He stopped in the act of swinging himself out of his saddle, and,
resuming it as his safest place, said:

``Emigrant, my friends!  Do you not see me here, in France, of my own will?''

``You are a cursed emigrant,'' cried a farrier, making at him in a
furious manner through the press, hammer in hand; ``and you are a
cursed aristocrat!''

The postmaster interposed himself between this man and the rider's
bridle (at which he was evidently making), and soothingly said,
``Let him be; let him be!  He will be judged at Paris.''

``Judged!'' repeated the farrier, swinging his hammer.
``Ay! and condemned as a traitor.''  At this the crowd roared approval.

Checking the postmaster, who was for turning his horse's head to the
yard (the drunken patriot sat composedly in his saddle looking on,
with the line round his wrist), Darnay said, as soon as he could make
his voice heard:

``Friends, you deceive yourselves, or you are deceived.  I am not a traitor.''

``He lies!'' cried the smith. ``He is a traitor since the decree.
His life is forfeit to the people.  His cursed life is not his own!''

At the instant when Darnay saw a rush in the eyes of the crowd,
which another instant would have brought upon him, the postmaster
turned his horse into the yard, the escort rode in close upon his
horse's flanks, and the postmaster shut and barred the crazy double
gates.  The farrier struck a blow upon them with his hammer, and the
crowd groaned; but, no more was done.

``What is this decree that the smith spoke of?'' Darnay asked the
postmaster, when he had thanked him, and stood beside him in the yard.

``Truly, a decree for selling the property of emigrants.''

``When passed?''

``On the fourteenth.''

``The day I left England!''

``Everybody says it is but one of several, and that there will be
others---if there are not already-banishing all emigrants, and
condemning all to death who return.  That is what he meant when he
said your life was not your own.''

``But there are no such decrees yet?''

``What do I know!'' said the postmaster, shrugging his shoulders;
``there may be, or there will be.  It is all the same.  What would
you have?''

They rested on some straw in a loft until the middle of the night,
and then rode forward again when all the town was asleep.  Among the
many wild changes observable on familiar things which made this wild
ride unreal, not the least was the seeming rarity of sleep.
After long and lonely spurring over dreary roads, they would come to
a cluster of poor cottages, not steeped in darkness, but all
glittering with lights, and would find the people, in a ghostly
manner in the dead of the night, circling hand in hand round a
shrivelled tree of Liberty, or all drawn up together singing a
Liberty song.  Happily, however, there was sleep in Beauvais that
night to help them out of it and they passed on once more into
solitude and loneliness:  jingling through the untimely cold and wet,
among impoverished fields that had yielded no fruits of the earth
that year, diversified by the blackened remains of burnt houses, and
by the sudden emergence from ambuscade, and sharp reining up across
their way, of patriot patrols on the watch on all the roads.

Daylight at last found them before the wall of Paris.  The barrier
was closed and strongly guarded when they rode up to it.

``Where are the papers of this prisoner?'' demanded a resolute-looking
man in authority, who was summoned out by the guard.

Naturally struck by the disagreeable word, Charles Darnay requested
the speaker to take notice that he was a free traveller and French
citizen, in charge of an escort which the disturbed state of the
country had imposed upon him, and which he had paid for.

``Where,'' repeated the same personage, without taking any heed of him
whatever, ``are the papers of this prisoner?''

The drunken patriot had them in his cap, and produced them.  Casting his
eyes over Gabelle's letter, the same personage in authority showed
some disorder and surprise, and looked at Darnay with a close attention.

He left escort and escorted without saying a word, however, and went
into the guard-room; meanwhile, they sat upon their horses outside
the gate.  Looking about him while in this state of suspense, Charles
Darnay observed that the gate was held by a mixed guard of soldiers
and patriots, the latter far outnumbering the former; and that while
ingress into the city for peasants' carts bringing in supplies, and
for similar traffic and traffickers, was easy enough, egress, even
for the homeliest people, was very difficult.  A numerous medley of
men and women, not to mention beasts and vehicles of various sorts,
was waiting to issue forth; but, the previous identification was so
strict, that they filtered through the barrier very slowly.  Some of
these people knew their turn for examination to be so far off, that
they lay down on the ground to sleep or smoke, while others talked
together, or loitered about.  The red cap and tri-colour cockade were
universal, both among men and women.

When he had sat in his saddle some half-hour, taking note of these
things, Darnay found himself confronted by the same man in authority,
who directed the guard to open the barrier.  Then he delivered to the
escort, drunk and sober, a receipt for the escorted, and requested him
to dismount.  He did so, and the two patriots, leading his tired horse,
turned and rode away without entering the city.

He accompanied his conductor into a guard-room, smelling of common
wine and tobacco, where certain soldiers and patriots, asleep and
awake, drunk and sober, and in various neutral states between
sleeping and waking, drunkenness and sobriety, were standing and
lying about. The light in the guard-house, half derived from the
waning oil-lamps of the night, and half from the overcast day, was in
a correspondingly uncertain condition.  Some registers were lying
open on a desk, and an officer of a coarse, dark aspect, presided
over these.

``Citizen Defarge,'' said he to Darnay's conductor, as he took a slip
of paper to write on.  ``Is this the emigrant Evremonde?''

``This is the man.''

``Your age, Evremonde?''

``Thirty-seven.''

``Married, Evremonde?''

``Yes.''

``Where married?''

``In England.''

``Without doubt.  Where is your wife, Evremonde?''

``In England.''

``Without doubt.  You are consigned, Evremonde, to the prison of La Force.''

``Just Heaven!'' exclaimed Darnay.  ``Under what law, and for what offence?''

The officer looked up from his slip of paper for a moment.

``We have new laws, Evremonde, and new offences, since you were here.''
He said it with a hard smile, and went on writing.

``I entreat you to observe that I have come here voluntarily, in response
to that written appeal of a fellow-countryman which lies before you.
I demand no more than the opportunity to do so without delay.
Is not that my right?''

``Emigrants have no rights, Evremonde,'' was the stolid reply.
The officer wrote until he had finished, read over to himself what he
had written, sanded it, and handed it to Defarge, with the words
``In secret.''

Defarge motioned with the paper to the prisoner that he must
accompany him.  The prisoner obeyed, and a guard of two armed
patriots attended them.

``Is it you,'' said Defarge, in a low voice, as they went down the
guardhouse steps and turned into Paris, ``who married the daughter of
Doctor Manette, once a prisoner in the Bastille that is no more?''

``Yes,'' replied Darnay, looking at him with surprise.

``My name is Defarge, and I keep a wine-shop in the Quarter Saint
Antoine.  Possibly you have heard of me.''

``My wife came to your house to reclaim her father?  Yes!''

The word ``wife'' seemed to serve as a gloomy reminder to Defarge,
to say with sudden impatience, ``In the name of that sharp female
newly-born, and called La Guillotine, why did you come to France?''

``You heard me say why, a minute ago.  Do you not believe it is the
truth?''

``A bad truth for you,'' said Defarge, speaking with knitted brows,
and looking straight before him.

``Indeed I am lost here.  All here is so unprecedented, so changed,
so sudden and unfair, that I am absolutely lost.  Will you render me
a little help?''

``None.''  Defarge spoke, always looking straight before him.

``Will you answer me a single question?''

``Perhaps.  According to its nature.  You can say what it is.''

``In this prison that I am going to so unjustly, shall I have some
free communication with the world outside?''

``You will see.''

``I am not to be buried there, prejudged, and without any means of
presenting my case?''

``You will see.  But, what then?  Other people have been similarly
buried in worse prisons, before now.''

``But never by me, Citizen Defarge.''

Defarge glanced darkly at him for answer, and walked on in a steady
and set silence.  The deeper he sank into this silence, the fainter
hope there was---or so Darnay thought---of his softening in any slight
degree. He, therefore, made haste to say:

``It is of the utmost importance to me (you know, Citizen, even better
than I, of how much importance), that I should be able to communicate
to Mr.\ Lorry of Tellson's Bank, an English gentleman who is now in
Paris, the simple fact, without comment, that I have been thrown into
the prison of La Force.  Will you cause that to be done for me?''

``I will do,'' Defarge doggedly rejoined, ``nothing for you.  My duty is
to my country and the People.  I am the sworn servant of both,
against you. I will do nothing for you.''

Charles Darnay felt it hopeless to entreat him further, and his pride
was touched besides.  As they walked on in silence, he could not but
see how used the people were to the spectacle of prisoners passing
along the streets.  The very children scarcely noticed him.  A few
passers turned their heads, and a few shook their fingers at him as
an aristocrat; otherwise, that a man in good clothes should be going
to prison, was no more remarkable than that a labourer in working
clothes should be going to work.  In one narrow, dark, and dirty
street through which they passed, an excited orator, mounted on a stool,
was addressing an excited audience on the crimes against the people,
of the king and the royal family.  The few words that he caught from
this man's lips, first made it known to Charles Darnay that the king
was in prison, and that the foreign ambassadors had one and all left
Paris.  On the road (except at Beauvais) he had heard absolutely nothing.
The escort and the universal watchfulness had completely isolated him.

That he had fallen among far greater dangers than those which had
developed themselves when he left England, he of course knew now.
That perils had thickened about him fast, and might thicken faster
and faster yet, he of course knew now.  He could not but admit to
himself that he might not have made this journey, if he could have
foreseen the events of a few days.  And yet his misgivings were not
so dark as, imagined by the light of this later time, they would appear.
Troubled as the future was, it was the unknown future, and in its
obscurity there was ignorant hope.  The horrible massacre, days and
nights long, which, within a few rounds of the clock, was to set a
great mark of blood upon the blessed garnering time of harvest, was
as far out of his knowledge as if it had been a hundred thousand
years away.  The ``sharp female newly-born, and called La Guillotine,''
was hardly known to him, or to the generality of people, by name.
The frightful deeds that were to be soon done, were probably
unimagined at that time in the brains of the doers.  How could they
have a place in the shadowy conceptions of a gentle mind?

Of unjust treatment in detention and hardship, and in cruel
separation from his wife and child, he foreshadowed the likelihood,
or the certainty; but, beyond this, he dreaded nothing distinctly.
With this on his mind, which was enough to carry into a dreary prison
courtyard, he arrived at the prison of La Force.

A man with a bloated face opened the strong wicket, to whom Defarge
presented ``The Emigrant Evremonde.''

``What the Devil!  How many more of them!'' exclaimed the man with
the bloated face.

Defarge took his receipt without noticing the exclamation,
and withdrew, with his two fellow-patriots.

``What the Devil, I say again!'' exclaimed the gaoler, left with his wife.
``How many more!''

The gaoler's wife, being provided with no answer to the question,
merely replied, ``One must have patience, my dear!''  Three turnkeys who
entered responsive to a bell she rang, echoed the sentiment, and one
added, ``For the love of Liberty;'' which sounded in that place like an
inappropriate conclusion.

The prison of La Force was a gloomy prison, dark and filthy, and with
a horrible smell of foul sleep in it.  Extraordinary how soon the
noisome flavour of imprisoned sleep, becomes manifest in all such
places that are ill cared for!

``In secret, too,'' grumbled the gaoler, looking at the written paper.
``As if I was not already full to bursting!''

He stuck the paper on a file, in an ill-humour, and Charles Darnay
awaited his further pleasure for half an hour:  sometimes, pacing to
and fro in the strong arched room:  sometimes, resting on a stone seat:
in either case detained to be imprinted on the memory of the chief
and his subordinates.

``Come!'' said the chief, at length taking up his keys, ``come with me, emigrant.''

Through the dismal prison twilight, his new charge accompanied him by
corridor and staircase, many doors clanging and locking behind them,
until they came into a large, low, vaulted chamber, crowded with
prisoners of both sexes.  The women were seated at a long table,
reading and writing, knitting, sewing, and embroidering; the men were
for the most part standing behind their chairs, or lingering up and
down the room.

In the instinctive association of prisoners with shameful crime and
disgrace, the new-comer recoiled from this company.  But the crowning
unreality of his long unreal ride, was, their all at once rising to
receive him, with every refinement of manner known to the time, and
with all the engaging graces and courtesies of life.

So strangely clouded were these refinements by the prison manners and
gloom, so spectral did they become in the inappropriate squalor and
misery through which they were seen, that Charles Darnay seemed to
stand in a company of the dead.  Ghosts all!  The ghost of beauty,
the ghost of stateliness, the ghost of elegance, the ghost of pride,
the ghost of frivolity, the ghost of wit, the ghost of youth, the
ghost of age, all waiting their dismissal from the desolate shore,
all turning on him eyes that were changed by the death they had died
in coming there.

It struck him motionless.  The gaoler standing at his side, and the
other gaolers moving about, who would have been well enough as to
appearance in the ordinary exercise of their functions, looked so
extravagantly coarse contrasted with sorrowing mothers and blooming
daughters who were there---with the apparitions of the coquette,
the young beauty, and the mature woman delicately bred---that the
inversion of all experience and likelihood which the scene of shadows
presented, was heightened to its utmost.  Surely, ghosts all.
Surely, the long unreal ride some progress of disease that had
brought him to these gloomy shades!

``In the name of the assembled companions in misfortune,'' said a
gentleman of courtly appearance and address, coming forward,
``I have the honour of giving you welcome to La Force, and of
condoling with you on the calamity that has brought you among us.
May it soon terminate happily!  It would be an impertinence elsewhere,
but it is not so here, to ask your name and condition?''

Charles Darnay roused himself, and gave the required information,
in words as suitable as he could find.

``But I hope,'' said the gentleman, following the chief gaoler with his
eyes, who moved across the room, ``that you are not in secret?''

``I do not understand the meaning of the term, but I have heard them
say so.''

``Ah, what a pity!  We so much regret it!  But take courage; several
members of our society have been in secret, at first, and it has
lasted but a short time.''  Then he added, raising his voice,
``I grieve to inform the society---in secret.''

There was a murmur of commiseration as Charles Darnay crossed the
room to a grated door where the gaoler awaited him, and many
voices---among which, the soft and compassionate voices of women were
conspicuous---gave him good wishes and encouragement.  He turned at
the grated door, to render the thanks of his heart; it closed under
the gaoler's hand; and the apparitions vanished from his sight forever.

The wicket opened on a stone staircase, leading upward.  When they
bad ascended forty steps (the prisoner of half an hour already
counted them), the gaoler opened a low black door, and they passed
into a solitary cell.  It struck cold and damp, but was not dark.

``Yours,'' said the gaoler.

``Why am I confined alone?''

``How do I know!''

``I can buy pen, ink, and paper?''

``Such are not my orders.  You will be visited, and can ask then.
At present, you may buy your food, and nothing more.''

There were in the cell, a chair, a table, and a straw mattress.
As the gaoler made a general inspection of these objects, and of the
four walls, before going out, a wandering fancy wandered through the
mind of the prisoner leaning against the wall opposite to him, that
this gaoler was so unwholesomely bloated, both in face and person,
as to look like a man who had been drowned and filled with water.
When the gaoler was gone, he thought in the same wandering way,
``Now am I left, as if I were dead.''  Stopping then, to look down at
the mattress, he turned from it with a sick feeling, and thought,
``And here in these crawling creatures is the first condition of the
body after death.''

``Five paces by four and a half, five paces by four and a half, five
paces by four and a half.''  The prisoner walked to and fro in his
cell, counting its measurement, and the roar of the city arose like
muffled drums with a wild swell of voices added to them.  ``He made
shoes, he made shoes, he made shoes.''  The prisoner counted the
measurement again, and paced faster, to draw his mind with him from
that latter repetition.  ``The ghosts that vanished when the wicket
closed.  There was one among them, the appearance of a lady dressed
in black, who was leaning in the embrasure of a window, and she had a
light shining upon her golden hair, and she looked like * * * * Let
us ride on again, for God's sake, through the illuminated villages
with the people all awake! * * * * He made shoes, he made shoes,
he made shoes. * * * * Five paces by four and a half.''  With such scraps
tossing and rolling upward from the depths of his mind, the prisoner
walked faster and faster, obstinately counting and counting; and the
roar of the city changed to this extent---that it still rolled in like
muffled drums, but with the wail of voices that he knew, in the swell
that rose above them.



\chapter{The Grindstone}


Tellson's Bank, established in the Saint Germain Quarter of Paris,
was in a wing of a large house, approached by a courtyard and shut
off from the street by a high wall and a strong gate.  The house
belonged to a great nobleman who had lived in it until he made a
flight from the troubles, in his own cook's dress, and got across the
borders.  A mere beast of the chase flying from hunters, he was still
in his metempsychosis no other than the same Monseigneur, the
preparation of whose chocolate for whose lips had once occupied three
strong men besides the cook in question.

Monseigneur gone, and the three strong men absolving themselves from
the sin of having drawn his high wages, by being more than ready and
willing to cut his throat on the altar of the dawning Republic one and
indivisible of Liberty, Equality, Fraternity, or Death, Monseigneur's
house had been first sequestrated, and then confiscated.  For, all
things moved so fast, and decree followed decree with that fierce
precipitation, that now upon the third night of the autumn month of
September, patriot emissaries of the law were in possession of
Monseigneur's house, and had marked it with the tri-colour, and were
drinking brandy in its state apartments.

A place of business in London like Tellson's place of business in
Paris, would soon have driven the House out of its mind and into the
Gazette. For, what would staid British responsibility and
respectability have said to orange-trees in boxes in a Bank courtyard,
and even to a Cupid over the counter?  Yet such things were.
Tellson's had whitewashed the Cupid, but he was still to be seen on
the ceiling, in the coolest linen, aiming (as he very often does) at
money from morning to night.  Bankruptcy must inevitably have come of
this young Pagan, in Lombard-street, London, and also of a curtained
alcove in the rear of the immortal boy, and also of a looking-glass
let into the wall, and also of clerks not at all old, who danced in
public on the slightest provocation.  Yet, a French Tellson's could
get on with these things exceedingly well, and, as long as the times
held together, no man had taken fright at them, and drawn out his money.

What money would be drawn out of Tellson's henceforth, and what would
lie there, lost and forgotten; what plate and jewels would tarnish in
Tellson's hiding-places, while the depositors rusted in prisons, and
when they should have violently perished; how many accounts with
Tellson's never to be balanced in this world, must be carried over
into the next; no man could have said, that night, any more than
Mr.\ Jarvis Lorry could, though he thought heavily of these questions.
He sat by a newly-lighted wood fire (the blighted and unfruitful year
was prematurely cold), and on his honest and courageous face there
was a deeper shade than the pendent lamp could throw, or any object
in the room distortedly reflect---a shade of horror.

He occupied rooms in the Bank, in his fidelity to the House of which
he had grown to be a part, lie strong root-ivy. it chanced that they
derived a kind of security from the patriotic occupation of the main
building, but the true-hearted old gentleman never calculated about
that.  All such circumstances were indifferent to him, so that he did
his duty.  On the opposite side of the courtyard, under a colonnade,
was extensive standing---for carriages---where, indeed, some carriages
of Monseigneur yet stood.  Against two of the pillars were fastened
two great flaring flambeaux, and in the light of these, standing out
in the open air, was a large grindstone:  a roughly mounted thing
which appeared to have hurriedly been brought there from some
neighbouring smithy, or other workshop.  Rising and looking out of
window at these harmless objects, Mr.\ Lorry shivered, and retired to
his seat by the fire.  He had opened, not only the glass window, but
the lattice blind outside it, and he had closed both again, and he
shivered through his frame.

From the streets beyond the high wall and the strong gate, there came
the usual night hum of the city, with now and then an indescribable
ring in it, weird and unearthly, as if some unwonted sounds of a
terrible nature were going up to Heaven.

``Thank God,'' said Mr.\ Lorry, clasping his hands, ``that no one near
and dear to me is in this dreadful town to-night.  May He have mercy
on all who are in danger!''

Soon afterwards, the bell at the great gate sounded, and he thought,
``They have come back!'' and sat listening.  But, there was no loud
irruption into the courtyard, as he had expected, and he heard the
gate clash again, and all was quiet.

The nervousness and dread that were upon him inspired that vague
uneasiness respecting the Bank, which a great change would naturally
awaken, with such feelings roused.  It was well guarded, and he got
up to go among the trusty people who were watching it, when his door
suddenly opened, and two figures rushed in, at sight of which he fell
back in amazement.

Lucie and her father!  Lucie with her arms stretched out to him, and
with that old look of earnestness so concentrated and intensified,
that it seemed as though it had been stamped upon her face expressly
to give force and power to it in this one passage of her life.

``What is this?'' cried Mr.\ Lorry, breathless and confused.
``What is the matter?  Lucie!  Manette!  What has happened?  What has
brought you here?  What is it?''

With the look fixed upon him, in her paleness and wildness,
she panted out in his arms, imploringly, ``O my dear friend!
My husband!''

``Your husband, Lucie?''

``Charles.''

``What of Charles?''

``Here.

``Here, in Paris?''

``Has been here some days---three or four---I don't know how many---%
I can't collect my thoughts.  An errand of generosity brought him
here unknown to us; he was stopped at the barrier, and sent to prison.''

The old man uttered an irrepressible cry.  Almost at the same moment,
the beg of the great gate rang again, and a loud noise of feet and
voices came pouring into the courtyard.

``What is that noise?'' said the Doctor, turning towards the window.

``Don't look!'' cried Mr.\ Lorry.  ``Don't look out!  Manette,
for your life, don't touch the blind!''

The Doctor turned, with his hand upon the fastening of the window,
and said, with a cool, bold smile:

``My dear friend, I have a charmed life in this city.  I have been a
Bastille prisoner.  There is no patriot in Paris---in Paris?  In
France---who, knowing me to have been a prisoner in the Bastille,
would touch me, except to overwhelm me with embraces, or carry me in
triumph.  My old pain has given me a power that has brought us
through the barrier, and gained us news of Charles there, and brought
us here.  I knew it would be so; I knew I could help Charles out of
all danger; I told Lucie so.---What is that noise?'' His hand was again
upon the window.

``Don't look!'' cried Mr.\ Lorry, absolutely desperate.  ``No, Lucie, my
dear, nor you!''  He got his arm round her, and held her.  ``Don't be so
terrified, my love.  I solemnly swear to you that I know of no harm
having happened to Charles; that I had no suspicion even of his being
in this fatal place.  What prison is he in?''

``La Force!''

``La Force!  Lucie, my child, if ever you were brave and serviceable in
your life---and you were always both---you will compose yourself now,
to do exactly as I bid you; for more depends upon it than you can think,
or I can say.  There is no help for you in any action on your part
to-night; you cannot possibly stir out.  I say this, because what I
must bid you to do for Charles's sake, is the hardest thing to do of all.
You must instantly be obedient, still, and quiet.  You must let me
put you in a room at the back here.  You must leave your father and
me alone for two minutes, and as there are Life and Death in the
world you must not delay.''

``I will be submissive to you.  I see in your face that you know I can
do nothing else than this.  I know you are true.''

The old man kissed her, and hurried her into his room, and turned the
key; then, came hurrying back to the Doctor, and opened the window
and partly opened the blind, and put his hand upon the Doctor's arm,
and looked out with him into the courtyard.

Looked out upon a throng of men and women:  not enough in number, or
near enough, to fill the courtyard:  not more than forty or fifty in
all.  The people in possession of the house had let them in at the
gate, and they had rushed in to work at the grindstone; it had
evidently been set up there for their purpose, as in a convenient and
retired spot.

But, such awful workers, and such awful work!

The grindstone had a double handle, and, turning at it madly were two
men, whose faces, as their long hair Rapped back when the whirlings
of the grindstone brought their faces up, were more horrible and
cruel than the visages of the wildest savages in their most barbarous
disguise. False eyebrows and false moustaches were stuck upon them,
and their hideous countenances were all bloody and sweaty, and all
awry with howling, and all staring and glaring with beastly
excitement and want of sleep.  As these ruffians turned and turned,
their matted locks now flung forward over their eyes, now flung
backward over their necks, some women held wine to their mouths that
they might drink; and what with dropping blood, and what with
dropping wine, and what with the stream of sparks struck out of the
stone, all their wicked atmosphere seemed gore and fire.  The eye
could not detect one creature in the group free from the smear of blood.
Shouldering one another to get next at the sharpening-stone, were men
stripped to the waist, with the stain all over their limbs and
bodies; men in all sorts of rags, with the stain upon those rags; men
devilishly set off with spoils of women's lace and silk and ribbon,
with the stain dyeing those trifles through and through.  Hatchets,
knives, bayonets, swords, all brought to be sharpened, were all red
with it. Some of the hacked swords were tied to the wrists of those
who carried them, with strips of linen and fragments of dress:
ligatures various in kind, but all deep of the one colour.  And as
the frantic wielders of these weapons snatched them from the stream
of sparks and tore away into the streets, the same red hue was red in
their frenzied eyes;---eyes which any unbrutalised beholder would have
given twenty years of life, to petrify with a well-directed gun.

All this was seen in a moment, as the vision of a drowning man, or of
any human creature at any very great pass, could see a world if it
were there.  They drew back from the window, and the Doctor looked
for explanation in his friend's ashy face.

``They are,'' Mr.\ Lorry whispered the words, glancing fearfully round
at the locked room, ``murdering the prisoners.  If you are sure of
what you say; if you really have the power you think you have---as I
believe you have---make yourself known to these devils, and get taken
to La Force.  It may be too late, I don't know, but let it not be a
minute later!''

Doctor Manette pressed his hand, hastened bareheaded out of the room,
and was in the courtyard when Mr.\ Lorry regained the blind.

His streaming white hair, his remarkable face, and the impetuous
confidence of his manner, as he put the weapons aside like water,
carried him in an instant to the heart of the concourse at the stone.
For a few moments there was a pause, and a hurry, and a murmur, and
the unintelligible sound of his voice; and then Mr.\ Lorry saw him,
surrounded by all, and in the midst of a line of twenty men long, all
linked shoulder to shoulder, and hand to shoulder, hurried out with
cries of---``Live the Bastille prisoner!  Help for the Bastille
prisoner's kindred in La Force!  Room for the Bastille prisoner in
front there!  Save the prisoner Evremonde at La Force!'' and a thousand
answering shouts.

He closed the lattice again with a fluttering heart, closed the
window and the curtain, hastened to Lucie, and told her that her
father was assisted by the people, and gone in search of her husband.
He found her child and Miss Pross with her; but, it never occurred to
him to be surprised by their appearance until a long time afterwards,
when he sat watching them in such quiet as the night knew.

Lucie had, by that time, fallen into a stupor on the floor at his feet,
clinging to his hand.  Miss Pross had laid the child down on his own bed,
and her head had gradually fallen on the pillow beside her pretty charge.
O the long, long night, with the moans of the poor wife!  And O the long,
long night, with no return of her father and no tidings!

Twice more in the darkness the bell at the great gate sounded,
and the irruption was repeated, and the grindstone whirled and
spluttered.  ``What is it?'' cried Lucie, affrighted.  ``Hush! The
soldiers' swords are sharpened there,'' said Mr.\ Lorry.  ``The place
is national property now, and used as a kind of armoury, my love.''

Twice more in all; but, the last spell of work was feeble and fitful.
Soon afterwards the day began to dawn, and he softly detached himself
from the clasping hand, and cautiously looked out again.  A man, so
besmeared that he might have been a sorely wounded soldier creeping
back to consciousness on a field of slain, was rising from the
pavement by the side of the grindstone, and looking about him with a
vacant air.  Shortly, this worn-out murderer descried in the imperfect
light one of the carriages of Monseigneur, and, staggering to that
gorgeous vehicle, climbed in at the door, and shut himself up to take
his rest on its dainty cushions.

The great grindstone, Earth, had turned when Mr.\ Lorry looked out again,
and the sun was red on the courtyard.  But, the lesser grindstone
stood alone there in the calm morning air, with a red upon it that
the sun had never given, and would never take away.



\chapter{The Shadow}


One of the first considerations which arose in the business mind of
Mr.\ Lorry when business hours came round, was this:---that he had no
right to imperil Tellson's by sheltering the wife of an emigrant
prisoner under the Bank roof, His own possessions, safety, life,
he would have hazarded for Lucie and her child, without a moment's
demur; but the great trust he held was not his own, and as to that
business charge he was a strict man of business.

At first, his mind reverted to Defarge, and he thought of finding out
the wine-shop again and taking counsel with its master in reference
to the safest dwelling-place in the distracted state of the city.
But, the same consideration that suggested him, repudiated him; he
lived in the most violent Quarter, and doubtless was influential
there, and deep in its dangerous workings.

Noon coming, and the Doctor not returning, and every minute's delay
tending to compromise Tellson's, Mr.\ Lorry advised with Lucie.
She said that her father had spoken of hiring a lodging for a short
term, in that Quarter, near the Banking-house.  As there was no
business objection to this, and as he foresaw that even if it were
all well with Charles, and he were to be released, he could not hope
to leave the city, Mr.\ Lorry went out in quest of such a lodging, and
found a suitable one, high up in a removed by-street where the closed
blinds in all the other windows of a high melancholy square of buildings
marked deserted homes.

To this lodging he at once removed Lucie and her child, and Miss
Pross:  giving them what comfort he could, and much more than he had
himself.  He left Jerry with them, as a figure to fill a doorway that
would bear considerable knocking on the head, and retained to his own
occupations.  A disturbed and doleful mind he brought to bear upon them,
and slowly and heavily the day lagged on with him.

It wore itself out, and wore him out with it, until the Bank closed.
He was again alone in his room of the previous night, considering
what to do next, when he heard a foot upon the stair.  In a few
moments, a man stood in his presence, who, with a keenly observant
look at him, addressed him by his name.

``Your servant,'' said Mr.\ Lorry.  ``Do you know me?''

He was a strongly made man with dark curling hair, from forty-five to
fifty years of age.  For answer he repeated, without any change of
emphasis, the words:

``Do you know me?''

``I have seen you somewhere.''

``Perhaps at my wine-shop?''

Much interested and agitated, Mr.\ Lorry said:  ``You come from Doctor
Manette?''

``Yes.  I come from Doctor Manette.''

``And what says he?  What does he send me?''

Defarge gave into his anxious hand, an open scrap of paper.  It bore
the words in the Doctor's writing:

\begin{quote}
    ``Charles is safe, but I cannot safely leave this place yet.
     I have obtained the favour that the bearer has a short note
     from Charles to his wife.  Let the bearer see his wife.''
\end{quote}

It was dated from La Force, within an hour.

``Will you accompany me,'' said Mr.\ Lorry, joyfully relieved after
reading this note aloud, ``to where his wife resides?''

``Yes,'' returned Defarge.

Scarcely noticing as yet, in what a curiously reserved and mechanical
way Defarge spoke, Mr.\ Lorry put on his hat and they went down into
the courtyard.  There, they found two women; one, knitting.

``Madame Defarge, surely!'' said Mr.\ Lorry, who had left her in exactly
the same attitude some seventeen years ago.

``It is she,'' observed her husband.

``Does Madame go with us?'' inquired Mr.\ Lorry, seeing that she moved
as they moved.

``Yes.  That she may be able to recognise the faces and know the persons.
It is for their safety.''

Beginning to be struck by Defarge's manner, Mr.\ Lorry looked
dubiously at him, and led the way.  Both the women followed; the
second woman being The Vengeance.

They passed through the intervening streets as quickly as they might,
ascended the staircase of the new domicile, were admitted by Jerry,
and found Lucie weeping, alone.  She was thrown into a transport by
the tidings Mr.\ Lorry gave her of her husband, and clasped the hand
that delivered his note---little thinking what it had been doing near
him in the night, and might, but for a chance, have done to him.

\begin{quote}
     ``\emph{Dearest},---Take courage.  I am well, and your father has
      influence around me.  You cannot answer this.
      Kiss our child for me.''
\end{quote}

That was all the writing.  It was so much, however, to her who
received it, that she turned from Defarge to his wife, and kissed one
of the hands that knitted.  It was a passionate, loving, thankful,
womanly action, but the hand made no response---dropped cold and
heavy, and took to its knitting again.

There was something in its touch that gave Lucie a check.
She stopped in the act of putting the note in her bosom, and,
with her hands yet at her neck, looked terrified at Madame Defarge.
Madame Defarge met the lifted eyebrows and forehead with a cold,
impassive stare.

``My dear,'' said Mr.\ Lorry, striking in to explain; ``there are
frequent risings in the streets; and, although it is not likely they
will ever trouble you, Madame Defarge wishes to see those whom she
has the power to protect at such times, to the end that she may know
them---that she may identify them.  I believe,'' said Mr.\ Lorry,
rather halting in his reassuring words, as the stony manner of all
the three impressed itself upon him more and more, ``I state the case,
Citizen Defarge?''

Defarge looked gloomily at his wife, and gave no other answer than a
gruff sound of acquiescence.

``You had better, Lucie,'' said Mr.\ Lorry, doing all he could to
propitiate, by tone and manner, ``have the dear child here, and our
good Pross.  Our good Pross, Defarge, is an English lady, and knows
no French.''

The lady in question, whose rooted conviction that she was more than
a match for any foreigner, was not to be shaken by distress and,
danger, appeared with folded arms, and observed in English to The
Vengeance, whom her eyes first encountered, ``Well, I am sure, Boldface!
I hope \emph{you} are pretty well!''  She also bestowed a British cough on
Madame Defarge; but, neither of the two took much heed of her.

``Is that his child?'' said Madame Defarge, stopping in her work for
the first time, and pointing her knitting-needle at little Lucie as
if it were the finger of Fate.

``Yes, madame,'' answered Mr.\ Lorry; ``this is our poor prisoner's
darling daughter, and only child.''

The shadow attendant on Madame Defarge and her party seemed to fall
so threatening and dark on the child, that her mother instinctively
kneeled on the ground beside her, and held her to her breast.  The
shadow attendant on Madame Defarge and her party seemed then to fall,
threatening and dark, on both the mother and the child.

``It is enough, my husband,'' said Madame Defarge.  ``I have seen them.
We may go.''

But, the suppressed manner had enough of menace in it---not visible
and presented, but indistinct and withheld---to alarm Lucie into
saying, as she laid her appealing hand on Madame Defarge's dress:

``You will be good to my poor husband.  You will do him no harm.
You will help me to see him if you can?''

``Your husband is not my business here,'' returned Madame Defarge,
looking down at her with perfect composure.  ``It is the daughter of
your father who is my business here.''

``For my sake, then, be merciful to my husband.  For my child's sake!
She will put her hands together and pray you to be merciful.  We are
more afraid of you than of these others.''

Madame Defarge received it as a compliment, and looked at her
husband.  Defarge, who had been uneasily biting his thumb-nail and
looking at her, collected his face into a sterner expression.

``What is it that your husband says in that little letter?''  asked
Madame Defarge, with a lowering smile.  ``Influence; he says something
touching influence?''

``That my father,'' said Lucie, hurriedly taking the paper from her
breast, but with her alarmed eyes on her questioner and not on it,
``has much influence around him.''

``Surely it will release him!'' said Madame Defarge.  ``Let it do so.''

``As a wife and mother,'' cried Lucie, most earnestly, ``I implore you
to have pity on me and not to exercise any power that you possess,
against my innocent husband, but to use it in his behalf.
O sister-woman, think of me.  As a wife and mother!''

Madame Defarge looked, coldly as ever, at the suppliant, and said,
turning to her friend The Vengeance:

``The wives and mothers we have been used to see, since we were as
little as this child, and much less, have not been greatly
considered?  We have known \emph{their} husbands and fathers laid in prison
and kept from them, often enough?  All our lives, we have seen our
sister-women suffer, in themselves and in their children, poverty,
nakedness, hunger, thirst, sickness, misery, oppression and neglect
of all kinds?''

``We have seen nothing else,'' returned The Vengeance.

``We have borne this a long time,'' said Madame Defarge, turning her
eyes again upon Lucie.  ``Judge you!  Is it likely that the trouble of
one wife and mother would be much to us now?''

She resumed her knitting and went out.  The Vengeance followed.
Defarge went last, and closed the door.

``Courage, my dear Lucie,'' said Mr.\ Lorry, as he raised her.
``Courage, courage!  So far all goes well with us---much, much better
than it has of late gone with many poor souls.  Cheer up, and have a
thankful heart.''

``I am not thankless, I hope, but that dreadful woman seems to throw a
shadow on me and on all my hopes.''

``Tut, tut!'' said Mr.\ Lorry; ``what is this despondency in the brave
little breast?  A shadow indeed!  No substance in it, Lucie.''

But the shadow of the manner of these Defarges was dark upon himself,
for all that, and in his secret mind it troubled him greatly.



\chapter{Calm in Storm}


Doctor Manette did not return until the morning of the fourth day of
his absence.  So much of what had happened in that dreadful time as
could be kept from the knowledge of Lucie was so well concealed from
her, that not until long afterwards, when France and she were far apart,
did she know that eleven hundred defenceless prisoners of both sexes
and all ages had been killed by the populace; that four days and
nights had been darkened by this deed of horror; and that the air
around her had been tainted by the slain.  She only knew that there
had been an attack upon the prisons, that all political prisoners had
been in danger, and that some had been dragged out by the crowd and
murdered.

To Mr.\ Lorry, the Doctor communicated under an injunction of secrecy
on which he had no need to dwell, that the crowd had taken him
through a scene of carnage to the prison of La Force.  That, in the
prison he had found a self-appointed Tribunal sitting, before which
the prisoners were brought singly, and by which they were rapidly
ordered to be put forth to be massacred, or to be released, or (in a
few cases) to be sent back to their cells.  That, presented by his
conductors to this Tribunal, he had announced himself by name and
profession as having been for eighteen years a secret and unaccused
prisoner in the Bastille; that, one of the body so sitting in
judgment had risen and identified him, and that this man was Defarge.

That, hereupon he had ascertained, through the registers on the table,
that his son-in-law was among the living prisoners, and had pleaded
hard to the Tribunal---of whom some members were asleep and some awake,
some dirty with murder and some clean, some sober and some not---for
his life and liberty.  That, in the first frantic greetings lavished
on himself as a notable sufferer under the overthrown system, it had
been accorded to him to have Charles Darnay brought before the lawless
Court, and examined.  That, he seemed on the point of being at once
released, when the tide in his favour met with some unexplained check
(not intelligible to the Doctor), which led to a few words of secret
conference.  That, the man sitting as President had then informed
Doctor Manette that the prisoner must remain in custody, but should,
for his sake, be held inviolate in safe custody.  That, immediately,
on a signal, the prisoner was removed to the interior of the prison
again; but, that he, the Doctor, had then so strongly pleaded for
permission to remain and assure himself that his son-in-law was,
through no malice or mischance, delivered to the concourse whose
murderous yells outside the gate had often drowned the proceedings,
that he had obtained the permission, and had remained in that Hall of
Blood until the danger was over.

The sights he had seen there, with brief snatches of food and sleep
by intervals, shall remain untold.  The mad joy over the prisoners
who were saved, had astounded him scarcely less than the mad ferocity
against those who were cut to pieces.  One prisoner there was, he
said, who had been discharged into the street free, but at whom a
mistaken savage had thrust a pike as he passed out.  Being besought
to go to him and dress the wound, the Doctor had passed out at the
same gate, and had found him in the arms of a company of Samaritans,
who were seated on the bodies of their victims.  With an inconsistency
as monstrous as anything in this awful nightmare, they had helped the
healer, and tended the wounded man with the gentlest solicitude---%
had made a litter for him and escorted him carefully from the spot---%
had then caught up their weapons and plunged anew into a butchery so
dreadful, that the Doctor had covered his eyes with his hands, and
swooned away in the midst of it.

As Mr.\ Lorry received these confidences, and as he watched the face
of his friend now sixty-two years of age, a misgiving arose within
him that such dread experiences would revive the old danger.

But, he had never seen his friend in his present aspect:  he had never
at all known him in his present character.  For the first time the
Doctor felt, now, that his suffering was strength and power.  For the
first time he felt that in that sharp fire, he had slowly forged the
iron which could break the prison door of his daughter's husband, and
deliver him.  ``It all tended to a good end, my friend; it was not
mere waste and ruin.  As my beloved child was helpful in restoring me
to myself, I will be helpful now in restoring the dearest part of
herself to her; by the aid of Heaven I will do it!''  Thus, Doctor
Manette.  And when Jarvis Lorry saw the kindled eyes, the resolute
face, the calm strong look and bearing of the man whose life always
seemed to him to have been stopped, like a clock, for so many years,
and then set going again with an energy which had lain dormant during
the cessation of its usefulness, he believed.

Greater things than the Doctor had at that time to contend with,
would have yielded before his persevering purpose.  While he kept
himself in his place, as a physician, whose business was with all
degrees of mankind, bond and free, rich and poor, bad and good, he
used his personal influence so wisely, that he was soon the inspecting
physician of three prisons, and among them of La Force.  He could now
assure Lucie that her husband was no longer confined alone, but was
mixed with the general body of prisoners; he saw her husband weekly,
and brought sweet messages to her, straight from his lips; sometimes
her husband himself sent a letter to her (though never by the Doctor's
hand), but she was not permitted to write to him:  for, among the many
wild suspicions of plots in the prisons, the wildest of all pointed
at emigrants who were known to have made friends or permanent
connections abroad.

This new life of the Doctor's was an anxious life, no doubt; still,
the sagacious Mr.\ Lorry saw that there was a new sustaining pride in it.
Nothing unbecoming tinged the pride; it was a natural and worthy one;
but he observed it as a curiosity.  The Doctor knew, that up to that
time, his imprisonment had been associated in the minds of his
daughter and his friend, with his personal affliction, deprivation,
and weakness.  Now that this was changed, and he knew himself to be
invested through that old trial with forces to which they both looked
for Charles's ultimate safety and deliverance, he became so far exalted
by the change, that he took the lead and direction, and required them
as the weak, to trust to him as the strong.  The preceding relative
positions of himself and Lucie were reversed, yet only as the
liveliest gratitude and affection could reverse them, for he could
have had no pride but in rendering some service to her who had
rendered so much to him.  ``All curious to see,'' thought Mr.\ Lorry,
in his amiably shrewd way, ``but all natural and right; so, take the
lead, my dear friend, and keep it; it couldn't be in better hands.''

But, though the Doctor tried hard, and never ceased trying, to get
Charles Darnay set at liberty, or at least to get him brought to trial,
the public current of the time set too strong and fast for him.
The new era began; the king was tried, doomed, and beheaded; the
Republic of Liberty, Equality, Fraternity, or Death, declared for
victory or death against the world in arms; the black flag waved
night and day from the great towers of Notre Dame; three hundred
thousand men, summoned to rise against the tyrants of the earth, rose
from all the varying soils of France, as if the dragon's teeth had
been sown broadcast, and had yielded fruit equally on hill and plain,
on rock, in gravel, and alluvial mud, under the bright sky of the
South and under the clouds of the North, in fell and forest, in the
vineyards and the olive-grounds and among the cropped grass and the
stubble of the corn, along the fruitful banks of the broad rivers,
and in the sand of the sea-shore.  What private solicitude could rear
itself against the deluge of the Year One of Liberty---the deluge
rising from below, not falling from above, and with the windows of
Heaven shut, not opened!

There was no pause, no pity, no peace, no interval of relenting rest,
no measurement of time.  Though days and nights circled as regularly
as when time was young, and the evening and morning were the first
day, other count of time there was none.  Hold of it was lost in the
raging fever of a nation, as it is in the fever of one patient.
Now, breaking the unnatural silence of a whole city, the executioner
showed the people the head of the king---and now, it seemed almost in
the same breath, the head of his fair wife which had had eight weary
months of imprisoned widowhood and misery, to turn it grey.

And yet, observing the strange law of contradiction which obtains in
all such cases, the time was long, while it flamed by so fast.
A revolutionary tribunal in the capital, and forty or fifty thousand
revolutionary committees all over the land; a law of the Suspected,
which struck away all security for liberty or life, and delivered
over any good and innocent person to any bad and guilty one; prisons
gorged with people who had committed no offence, and could obtain no
hearing; these things became the established order and nature of
appointed things, and seemed to be ancient usage before they were
many weeks old.  Above all, one hideous figure grew as familiar as if
it had been before the general gaze from the foundations of the
world---the figure of the sharp female called La Guillotine.

It was the popular theme for jests; it was the best cure for
headache, it infallibly prevented the hair from turning grey, it
imparted a peculiar delicacy to the complexion, it was the National
Razor which shaved close:  who kissed La Guillotine, looked through
the little window and sneezed into the sack.  It was the sign of the
regeneration of the human race.  It superseded the Cross.  Models of
it were worn on breasts from which the Cross was discarded, and it
was bowed down to and believed in where the Cross was denied.

It sheared off heads so many, that it, and the ground it most
polluted, were a rotten red.  It was taken to pieces, like a
toy-puzzle for a young Devil, and was put together again when the
occasion wanted it.  It hushed the eloquent, struck down the powerful,
abolished the beautiful and good.  Twenty-two friends of high public
mark, twenty-one living and one dead, it had lopped the heads off,
in one morning, in as many minutes.  The name of the strong man of
Old Scripture had descended to the chief functionary who worked it;
but, so armed, he was stronger than his namesake, and blinder, and
tore away the gates of God's own Temple every day.

Among these terrors, and the brood belonging to them, the Doctor
walked with a steady head:  confident in his power, cautiously
persistent in his end, never doubting that he would save Lucie's
husband at last. Yet the current of the time swept by, so strong and
deep, and carried the time away so fiercely, that Charles had lain in
prison one year and three months when the Doctor was thus steady and
confident.  So much more wicked and distracted had the Revolution
grown in that December month, that the rivers of the South were
encumbered with the bodies of the violently drowned by night, and
prisoners were shot in lines and squares under the southern wintry sun.
Still, the Doctor walked among the terrors with a steady head.
No man better known than he, in Paris at that day; no man in a
stranger situation.  Silent, humane, indispensable in hospital and
prison, using his art equally among assassins and victims, he was a
man apart.  In the exercise of his skill, the appearance and the
story of the Bastille Captive removed him from all other men.  He was
not suspected or brought in question, any more than if he had indeed
been recalled to life some eighteen years before, or were a Spirit
moving among mortals.



\chapter{The Wood-Sawyer}


One year and three months.  During all that time Lucie was never
sure, from hour to hour, but that the Guillotine would strike off her
husband's head next day.  Every day, through the stony streets, the
tumbrils now jolted heavily, filled with Condemned.  Lovely girls;
bright women, brown-haired, black-haired, and grey; youths; stalwart
men and old; gentle born and peasant born; all red wine for La
Guillotine, all daily brought into light from the dark cellars of the
loathsome prisons, and carried to her through the streets to slake
her devouring thirst.  Liberty, equality, fraternity, or death;---the
last, much the easiest to bestow, O Guillotine!

If the suddenness of her calamity, and the whirling wheels of the
time, had stunned the Doctor's daughter into awaiting the result in
idle despair, it would but have been with her as it was with many.
But, from the hour when she had taken the white head to her fresh
young bosom in the garret of Saint Antoine, she had been true to her
duties.  She was truest to them in the season of trial, as all the
quietly loyal and good will always be.

As soon as they were established in their new residence, and her
father had entered on the routine of his avocations, she arranged the
little household as exactly as if her husband had been there.
Everything had its appointed place and its appointed time.  Little
Lucie she taught, as regularly, as if they had all been united in
their English home.  The slight devices with which she cheated
herself into the show of a belief that they would soon be reunited---%
the little preparations for his speedy return, the setting aside of
his chair and his books---these, and the solemn prayer at night for
one dear prisoner especially, among the many unhappy souls in prison
and the shadow of death---were almost the only outspoken reliefs of
her heavy mind.

She did not greatly alter in appearance.  The plain dark dresses,
akin to mourning dresses, which she and her child wore, were as neat
and as well attended to as the brighter clothes of happy days.
She lost her colour, and the old and intent expression was a constant,
not an occasional, thing; otherwise, she remained very pretty and
comely.  Sometimes, at night on kissing her father, she would burst
into the grief she had repressed all day, and would say that her sole
reliance, under Heaven, was on him.  He always resolutely answered:
``Nothing can happen to him without my knowledge, and I know that I
can save him, Lucie.''

They had not made the round of their changed life many weeks,
when her father said to her, on coming home one evening:

``My dear, there is an upper window in the prison, to which Charles
can sometimes gain access at three in the afternoon.  When he can get
to it---which depends on many uncertainties and incidents---he might
see you in the street, he thinks, if you stood in a certain place
that I can show you.  But you will not be able to see him, my poor
child, and even if you could, it would be unsafe for you to make a
sign of recognition.''

``O show me the place, my father, and I will go there every day.''

From that time, in all weathers, she waited there two hours.
As the clock struck two, she was there, and at four she turned
resignedly away. When it was not too wet or inclement for her child
to be with her, they went together; at other times she was alone;
but, she never missed a single day.

It was the dark and dirty corner of a small winding street.
The hovel of a cutter of wood into lengths for burning, was the only
house at that end; all else was wall.  On the third day of her being
there, he noticed her.

``Good day, citizeness.''

``Good day, citizen.''

This mode of address was now prescribed by decree.  It had been
established voluntarily some time ago, among the more thorough
patriots; but, was now law for everybody.

``Walking here again, citizeness?''

``You see me, citizen!''

The wood-sawyer, who was a little man with a redundancy of gesture
(he had once been a mender of roads), cast a glance at the prison,
pointed at the prison, and putting his ten fingers before his face to
represent bars, peeped through them jocosely.

``But it's not my business,'' said he.  And went on sawing his wood.

Next day he was looking out for her, and accosted her the moment she
appeared.

``What?  Walking here again, citizeness?''

``Yes, citizen.''

``Ah!  A child too!  Your mother, is it not, my little citizeness?''

``Do I say yes, mamma?'' whispered little Lucie, drawing close to her.

``Yes, dearest.''

``Yes, citizen.''

``Ah!  But it's not my business.  My work is my business.  See my saw!
I call it my Little Guillotine.  La, la, la; La, la, la!  And off his
head comes!''

The billet fell as he spoke, and he threw it into a basket.

``I call myself the Samson of the firewood guillotine.  See here again!
Loo, loo, loo; Loo, loo, loo!  And off \emph{her} head comes!  Now, a child.
Tickle, tickle; Pickle, pickle!  And off \emph{its} head comes.  All the family!''

Lucie shuddered as he threw two more billets into his basket, but it
was impossible to be there while the wood-sawyer was at work, and not
be in his sight.  Thenceforth, to secure his good will, she always
spoke to him first, and often gave him drink-money, which he readily
received.

He was an inquisitive fellow, and sometimes when she had quite
forgotten him in gazing at the prison roof and grates, and in lifting
her heart up to her husband, she would come to herself to find him
looking at her, with his knee on his bench and his saw stopped in its
work.  ``But it's not my business!'' he would generally say at those
times, and would briskly fall to his sawing again.

In all weathers, in the snow and frost of winter, in the bitter winds
of spring, in the hot sunshine of summer, in the rains of autumn, and
again in the snow and frost of winter, Lucie passed two hours of
every day at this place; and every day on leaving it, she kissed the
prison wall.  Her husband saw her (so she learned from her father) it
might be once in five or six times:  it might be twice or thrice running:
it might be, not for a week or a fortnight together.  It was enough
that he could and did see her when the chances served, and on that
possibility she would have waited out the day, seven days a week.

These occupations brought her round to the December month, wherein
her father walked among the terrors with a steady head.  On a
lightly-snowing afternoon she arrived at the usual corner.  It was a
day of some wild rejoicing, and a festival.  She had seen the houses,
as she came along, decorated with little pikes, and with little red
caps stuck upon them; also, with tricoloured ribbons; also, with the
standard inscription (tricoloured letters were the favourite),
Republic One and Indivisible. Liberty, Equality, Fraternity, or Death!

The miserable shop of the wood-sawyer was so small, that its whole
surface furnished very indifferent space for this legend.  He had got
somebody to scrawl it up for him, however, who had squeezed Death in
with most inappropriate difficulty.  On his house-top, he displayed
pike and cap, as a good citizen must, and in a window he had
stationed his saw inscribed as his ``Little Sainte Guillotine''---%
for the great sharp female was by that time popularly canonised.
His shop was shut and he was not there, which was a relief to Lucie,
and left her quite alone.

But, he was not far off, for presently she heard a troubled movement
and a shouting coming along, which filled her with fear.  A moment
afterwards, and a throng of people came pouring round the corner by
the prison wall, in the midst of whom was the wood-sawyer hand in
hand with The Vengeance.  There could not be fewer than five hundred
people, and they were dancing like five thousand demons.  There was
no other music than their own singing.  They danced to the popular
Revolution song, keeping a ferocious time that was like a gnashing of
teeth in unison.  Men and women danced together, women danced
together, men danced together, as hazard had brought them together.
At first, they were a mere storm of coarse red caps and coarse
woollen rags; but, as they filled the place, and stopped to dance
about Lucie, some ghastly apparition of a dance-figure gone raving
mad arose among them.  They advanced, retreated, struck at one
another's hands, clutched at one another's heads, spun round alone,
caught one another and spun round in pairs, until many of them
dropped.  While those were down, the rest linked hand in hand, and
all spun round together:  then the ring broke, and in separate rings
of two and four they turned and turned until they all stopped at
once, began again, struck, clutched, and tore, and then reversed the
spin, and all spun round another way.  Suddenly they stopped again,
paused, struck out the time afresh, formed into lines the width of
the public way, and, with their heads low down and their hands high
up, swooped screaming off.  No fight could have been half so terrible
as this dance.  It was so emphatically a fallen sport---a something,
once innocent, delivered over to all devilry---a healthy pastime
changed into a means of angering the blood, bewildering the senses,
and steeling the heart.  Such grace as was visible in it, made it the
uglier, showing how warped and perverted all things good by nature
were become.  The maidenly bosom bared to this, the pretty
almost-child's head thus distracted, the delicate foot mincing in
this slough of blood and dirt, were types of the disjointed time.

This was the Carmagnole.  As it passed, leaving Lucie frightened and
bewildered in the doorway of the wood-sawyer's house, the feathery
snow fell as quietly and lay as white and soft, as if it had never been.

``O my father!'' for he stood before her when she lifted up the eyes
she had momentarily darkened with her hand; ``such a cruel, bad sight.''

``I know, my dear, I know.  I have seen it many times.  Don't be
frightened!  Not one of them would harm you.''

``I am not frightened for myself, my father.  But when I think of my
husband, and the mercies of these people---''

``We will set him above their mercies very soon.  I left him climbing
to the window, and I came to tell you.  There is no one here to see.
You may kiss your hand towards that highest shelving roof.''

``I do so, father, and I send him my Soul with it!''

``You cannot see him, my poor dear?''

``No, father,'' said Lucie, yearning and weeping as she kissed her hand,
``no.''

A footstep in the snow.  Madame Defarge.  ``I salute you, citizeness,''
from the Doctor.  ``I salute you, citizen.''  This in passing.  Nothing
more. Madame Defarge gone, like a shadow over the white road.

``Give me your arm, my love.  Pass from here with an air of cheerfulness
and courage, for his sake.  That was well done;'' they had left the spot;
``it shall not be in vain.  Charles is summoned for to-morrow.''

``For to-morrow!''

``There is no time to lose.  I am well prepared, but there are
precautions to be taken, that could not be taken until he was actually
summoned before the Tribunal.  He has not received the notice yet,
but I know that he will presently be summoned for to-morrow, and
removed to the Conciergerie; I have timely information.
You are not afraid?''

She could scarcely answer, ``I trust in you.''

``Do so, implicitly.  Your suspense is nearly ended, my darling; he
shall be restored to you within a few hours; I have encompassed him
with every protection.  I must see Lorry.''

He stopped.  There was a heavy lumbering of wheels within hearing.
They both knew too well what it meant.  One.  Two.  Three.  Three
tumbrils faring away with their dread loads over the hushing snow.

``I must see Lorry,'' the Doctor repeated, turning her another way.

The staunch old gentleman was still in his trust; had never left it.
He and his books were in frequent requisition as to property
confiscated and made national.  What he could save for the owners, he
saved.  No better man living to hold fast by what Tellson's had in
keeping, and to hold his peace.

A murky red and yellow sky, and a rising mist from the Seine, denoted
the approach of darkness.  It was almost dark when they arrived at
the Bank.  The stately residence of Monseigneur was altogether
blighted and deserted.  Above a heap of dust and ashes in the court,
ran the letters:  National Property.  Republic One and Indivisible.
Liberty, Equality, Fraternity, or Death!

Who could that be with Mr.\ Lorry---the owner of the riding-coat upon
the chair---who must not be seen?  From whom newly arrived, did he come
out, agitated and surprised, to take his favourite in his arms?  To
whom did he appear to repeat her faltering words, when, raising his
voice and turning his head towards the door of the room from which he
had issued, he said:  ``Removed to the Conciergerie, and summoned for
to-morrow?''



\chapter{Triumph}


The dread tribunal of five Judges, Public Prosecutor, and determined
Jury, sat every day.  Their lists went forth every evening, and were
read out by the gaolers of the various prisons to their prisoners.
The standard gaoler-joke was, ``Come out and listen to the Evening Paper,
you inside there!''

``Charles Evremonde, called Darnay!''

So at last began the Evening Paper at La Force.

When a name was called, its owner stepped apart into a spot reserved
for those who were announced as being thus fatally recorded.  Charles
Evremonde, called Darnay, had reason to know the usage; he had seen
hundreds pass away so.

His bloated gaoler, who wore spectacles to read with, glanced over
them to assure himself that he had taken his place, and went through
the list, making a similar short pause at each name.  There were
twenty-three names, but only twenty were responded to; for one of the
prisoners so summoned had died in gaol and been forgotten, and two
had already been guillotined and forgotten.  The list was read, in
the vaulted chamber where Darnay had seen the associated prisoners on
the night of his arrival.  Every one of those had perished in the
massacre; every human creature he had since cared for and parted with,
had died on the scaffold.

There were hurried words of farewell and kindness, but the parting
was soon over.  It was the incident of every day, and the society of
La Force were engaged in the preparation of some games of forfeits
and a little concert, for that evening.  They crowded to the grates
and shed tears there; but, twenty places in the projected
entertainments had to be refilled, and the time was, at best, short
to the lock-up hour, when the common rooms and corridors would be
delivered over to the great dogs who kept watch there through the
night.  The prisoners were far from insensible or unfeeling; their
ways arose out of the condition of the time.  Similarly, though with
a subtle difference, a species of fervour or intoxication, known,
without doubt, to have led some persons to brave the guillotine
unnecessarily, and to die by it, was not mere boastfulness, but a
wild infection of the wildly shaken public mind.  In seasons of
pestilence, some of us will have a secret attraction to the disease---%
a terrible passing inclination to die of it.  And all of us have like
wonders hidden in our breasts, only needing circumstances to evoke them.

The passage to the Conciergerie was short and dark; the night in its
vermin-haunted cells was long and cold.  Next day, fifteen prisoners
were put to the bar before Charles Darnay's name was called.  All the
fifteen were condemned, and the trials of the whole occupied an hour
and a half.

``Charles Evremonde, called Darnay,'' was at length arraigned.

His judges sat upon the Bench in feathered hats; but the rough red
cap and tricoloured cockade was the head-dress otherwise prevailing.
Looking at the Jury and the turbulent audience, he might have thought
that the usual order of things was reversed, and that the felons were
trying the honest men.  The lowest, cruelest, and worst populace of a
city, never without its quantity of low, cruel, and bad, were the
directing spirits of the scene:  noisily commenting, applauding,
disapproving, anticipating, and precipitating the result, without a
check.  Of the men, the greater part were armed in various ways; of
the women, some wore knives, some daggers, some ate and drank as they
looked on, many knitted.  Among these last, was one, with a spare
piece of knitting under her arm as she worked.  She was in a front
row, by the side of a man whom he had never seen since his arrival at
the Barrier, but whom he directly remembered as Defarge.  He noticed
that she once or twice whispered in his ear, and that she seemed to
be his wife; but, what he most noticed in the two figures was, that
although they were posted as close to himself as they could be, they
never looked towards him.  They seemed to be waiting for something
with a dogged determination, and they looked at the Jury, but at
nothing else.  Under the President sat Doctor Manette, in his usual
quiet dress.  As well as the prisoner could see, he and Mr.\ Lorry
were the only men there, unconnected with the Tribunal, who wore their
usual clothes, and had not assumed the coarse garb of the Carmagnole.

Charles Evremonde, called Darnay, was accused by the public
prosecutor as an emigrant, whose life was forfeit to the Republic,
under the decree which banished all emigrants on pain of Death.
It was nothing that the decree bore date since his return to France.
There he was, and there was the decree; he had been taken in France,
and his head was demanded.

``Take off his head!'' cried the audience.  ``An enemy to the Republic!''

The President rang his bell to silence those cries, and asked the
prisoner whether it was not true that he had lived many years in England?

Undoubtedly it was.

Was he not an emigrant then?  What did he call himself?

Not an emigrant, he hoped, within the sense and spirit of the law.

Why not?  the President desired to know.

Because he had voluntarily relinquished a title that was distasteful
to him, and a station that was distasteful to him, and had left his
country---he submitted before the word emigrant in the present
acceptation by the Tribunal was in use---to live by his own industry
in England, rather than on the industry of the overladen people of
France.

What proof had he of this?

He handed in the names of two witnesses; Theophile Gabelle, and
Alexandre Manette.

But he had married in England?  the President reminded him.

True, but not an English woman.

A citizeness of France?

Yes.  By birth.

Her name and family?

``Lucie Manette, only daughter of Doctor Manette, the good physician
who sits there.''

This answer had a happy effect upon the audience.  Cries in
exaltation of the well-known good physician rent the hall.  So
capriciously were the people moved, that tears immediately rolled
down several ferocious countenances which had been glaring at the
prisoner a moment before, as if with impatience to pluck him out into
the streets and kill him.

On these few steps of his dangerous way, Charles Darnay had set his
foot according to Doctor Manette's reiterated instructions.  The same
cautious counsel directed every step that lay before him, and had
prepared every inch of his road.

The President asked, why had he returned to France when he did,
and not sooner?

He had not returned sooner, he replied, simply because he had no
means of living in France, save those he had resigned; whereas, in
England, he lived by giving instruction in the French language and
literature. He had returned when he did, on the pressing and written
entreaty of a French citizen, who represented that his life was
endangered by his absence.  He had come back, to save a citizen's life,
and to bear his testimony, at whatever personal hazard, to the truth.
Was that criminal in the eyes of the Republic?

The populace cried enthusiastically, ``No!'' and the President rang his
bell to quiet them.  Which it did not, for they continued to cry
``No!'' until they left off, of their own will.

The President required the name of that citizen.  The accused
explained that the citizen was his first witness.  He also referred
with confidence to the citizen's letter, which had been taken from
him at the Barrier, but which he did not doubt would be found among
the papers then before the President.

The Doctor had taken care that it should be there---had assured him
that it would be there---and at this stage of the proceedings it was
produced and read.  Citizen Gabelle was called to confirm it, and did
so. Citizen Gabelle hinted, with infinite delicacy and politeness,
that in the pressure of business imposed on the Tribunal by the
multitude of enemies of the Republic with which it had to deal, he
had been slightly overlooked in his prison of the Abbaye---in fact,
had rather passed out of the Tribunal's patriotic remembrance---until
three days ago; when he had been summoned before it, and had been set
at liberty on the Jury's declaring themselves satisfied that the
accusation against him was answered, as to himself, by the surrender
of the citizen Evremonde, called Darnay.

Doctor Manette was next questioned.  His high personal popularity,
and the clearness of his answers, made a great impression; but, as he
proceeded, as he showed that the Accused was his first friend on his
release from his long imprisonment; that, the accused had remained in
England, always faithful and devoted to his daughter and himself in
their exile; that, so far from being in favour with the Aristocrat
government there, he had actually been tried for his life by it, as
the foe of England and friend of the United States---as he brought
these circumstances into view, with the greatest discretion and with
the straightforward force of truth and earnestness, the Jury and the
populace became one.  At last, when he appealed by name to Monsieur
Lorry, an English gentleman then and there present, who, like himself,
had been a witness on that English trial and could corroborate his
account of it, the Jury declared that they had heard enough, and that
they were ready with their votes if the President were content to
receive them.

At every vote (the Jurymen voted aloud and individually), the
populace set up a shout of applause.  All the voices were in the
prisoner's favour, and the President declared him free.

Then, began one of those extraordinary scenes with which the populace
sometimes gratified their fickleness, or their better impulses
towards generosity and mercy, or which they regarded as some set-off
against their swollen account of cruel rage.  No man can decide now
to which of these motives such extraordinary scenes were referable;
it is probable, to a blending of all the three, with the second
predominating.  No sooner was the acquittal pronounced, than tears
were shed as freely as blood at another time, and such fraternal
embraces were bestowed upon the prisoner by as many of both sexes as
could rush at him, that after his long and unwholesome confinement he
was in danger of fainting from exhaustion; none the less because he
knew very well, that the very same people, carried by another current,
would have rushed at him with the very same intensity, to rend him to
pieces and strew him over the streets.

His removal, to make way for other accused persons who were to be
tried, rescued him from these caresses for the moment.  Five were to
be tried together, next, as enemies of the Republic, forasmuch as
they had not assisted it by word or deed.  So quick was the Tribunal
to compensate itself and the nation for a chance lost, that these
five came down to him before he left the place, condemned to die
within twenty-four hours. The first of them told him so, with the
customary prison sign of Death---a raised finger---and they all added
in words, ``Long live the Republic!''

The five had had, it is true, no audience to lengthen their
proceedings, for when he and Doctor Manette emerged from the gate,
there was a great crowd about it, in which there seemed to be every
face he had seen in Court---except two, for which he looked in vain.
On his coming out, the concourse made at him anew, weeping,
embracing, and shouting, all by turns and all together, until the
very tide of the river on the bank of which the mad scene was acted,
seemed to run mad, like the people on the shore.

They put him into a great chair they had among them, and which they
had taken either out of the Court itself, or one of its rooms or
passages.  Over the chair they had thrown a red flag, and to the back
of it they had bound a pike with a red cap on its top.  In this car
of triumph, not even the Doctor's entreaties could prevent his being
carried to his home on men's shoulders, with a confused sea of red
caps heaving about him, and casting up to sight from the stormy deep
such wrecks of faces, that he more than once misdoubted his mind
being in confusion, and that he was in the tumbril on his way to the
Guillotine.

In wild dreamlike procession, embracing whom they met and pointing
him out, they carried him on.  Reddening the snowy streets with the
prevailing Republican colour, in winding and tramping through them,
as they had reddened them below the snow with a deeper dye, they
carried him thus into the courtyard of the building where he lived.
Her father had gone on before, to prepare her, and when her husband
stood upon his feet, she dropped insensible in his arms.

As he held her to his heart and turned her beautiful head between his
face and the brawling crowd, so that his tears and her lips might
come together unseen, a few of the people fell to dancing. Instantly,
all the rest fell to dancing, and the courtyard overflowed with the
Carmagnole. Then, they elevated into the vacant chair a young woman
from the crowd to be carried as the Goddess of Liberty, and then
swelling and overflowing out into the adjacent streets, and along the
river's bank, and over the bridge, the Carmagnole absorbed them every
one and whirled them away.

After grasping the Doctor's hand, as he stood victorious and proud
before him; after grasping the hand of Mr.\ Lorry, who came panting in
breathless from his struggle against the waterspout of the Carmagnole;
after kissing little Lucie, who was lifted up to clasp her arms round
his neck; and after embracing the ever zealous and faithful Pross who
lifted her; he took his wife in his arms, and carried her up to their
rooms.

``Lucie!  My own!  I am safe.''

``O dearest Charles, let me thank God for this on my knees as I have
prayed to Him.''

They all reverently bowed their heads and hearts.  When she was again
in his arms, he said to her:

``And now speak to your father, dearest.  No other man in all this
France could have done what he has done for me.''

She laid her head upon her father's breast, as she had laid his poor
head on her own breast, long, long ago.  He was happy in the return
he had made her, he was recompensed for his suffering, he was proud
of his strength.  ``You must not be weak, my darling,'' he remonstrated;
``don't tremble so.  I have saved him.''



\chapter{A Knock at the Door}


``I have saved him.''  It was not another of the dreams in which he had
often come back; he was really here.  And yet his wife trembled, and
a vague but heavy fear was upon her.

All the air round was so thick and dark, the people were so
passionately revengeful and fitful, the innocent were so constantly
put to death on vague suspicion and black malice, it was so
impossible to forget that many as blameless as her husband and as
dear to others as he was to her, every day shared the fate from which
he had been clutched, that her heart could not be as lightened of its
load as she felt it ought to be.  The shadows of the wintry afternoon
were beginning to fall, and even now the dreadful carts were rolling
through the streets.  Her mind pursued them, looking for him among
the Condemned; and then she clung closer to his real presence and
trembled more.

Her father, cheering her, showed a compassionate superiority to this
woman's weakness, which was wonderful to see.  No garret, no shoemaking,
no One Hundred and Five, North Tower, now!  He had accomplished the
task he had set himself, his promise was redeemed, he had saved Charles.
Let them all lean upon him.

Their housekeeping was of a very frugal kind:  not only because that
was the safest way of life, involving the least offence to the
people, but because they were not rich, and Charles, throughout his
imprisonment, had had to pay heavily for his bad food, and for his
guard, and towards the living of the poorer prisoners.  Partly on
this account, and partly to avoid a domestic spy, they kept no
servant; the citizen and citizeness who acted as porters at the
courtyard gate, rendered them occasional service; and Jerry (almost
wholly transferred to them by Mr.\ Lorry) had become their daily
retainer, and had his bed there every night.

It was an ordinance of the Republic One and Indivisible of Liberty,
Equality, Fraternity, or Death, that on the door or doorpost of every
house, the name of every inmate must be legibly inscribed in letters
of a certain size, at a certain convenient height from the ground.
Mr.\ Jerry Cruncher's name, therefore, duly embellished the doorpost
down below; and, as the afternoon shadows deepened, the owner of that
name himself appeared, from overlooking a painter whom Doctor Manette
had employed to add to the list the name of Charles Evremonde, called
Darnay.

In the universal fear and distrust that darkened the time, all the
usual harmless ways of life were changed.  In the Doctor's little
household, as in very many others, the articles of daily consumption
that were wanted were purchased every evening, in small quantities
and at various small shops.  To avoid attracting notice, and to give
as little occasion as possible for talk and envy, was the general desire.

For some months past, Miss Pross and Mr.\ Cruncher had discharged the
office of purveyors; the former carrying the money; the latter, the
basket.  Every afternoon at about the time when the public lamps were
lighted, they fared forth on this duty, and made and brought home
such purchases as were needful.  Although Miss Pross, through her
long association with a French family, might have known as much of
their language as of her own, if she had had a mind, she had no mind
in that direction; consequently she knew no more of that ``nonsense''
(as she was pleased to call it) than Mr.\ Cruncher did.  So her
manner of marketing was to plump a noun-substantive at the head of a
shopkeeper without any introduction in the nature of an article, and,
if it happened not to be the name of the thing she wanted, to look
round for that thing, lay hold of it, and hold on by it until the
bargain was concluded.  She always made a bargain for it, by holding
up, as a statement of its just price, one finger less than the merchant
held up, whatever his number might be.

``Now, Mr.\ Cruncher,'' said Miss Pross, whose eyes were red with
felicity; ``if you are ready, I am.''

Jerry hoarsely professed himself at Miss Pross's service.  He had worn
all his rust off long ago, but nothing would file his spiky head down.

``There's all manner of things wanted,'' said Miss Pross, ``and we shall
have a precious time of it.  We want wine, among the rest.
Nice toasts these Redheads will be drinking, wherever we buy it.''

``It will be much the same to your knowledge, miss, I should think,''
retorted Jerry, ``whether they drink your health or the Old Un's.''

``Who's he?'' said Miss Pross.

Mr.\ Cruncher, with some diffidence, explained himself as meaning ``Old
Nick's.''

``Ha!'' said Miss Pross, ``it doesn't need an interpreter to explain the
meaning of these creatures.  They have but one, and it's Midnight
Murder, and Mischief.''

``Hush, dear!  Pray, pray, be cautious!'' cried Lucie.

``Yes, yes, yes, I'll be cautious,'' said Miss Pross; ``but I may say
among ourselves, that I do hope there will be no oniony and tobaccoey
smotherings in the form of embracings all round, going on in the
streets.  Now, Ladybird, never you stir from that fire till I come
back!  Take care of the dear husband you have recovered, and don't
move your pretty head from his shoulder as you have it now, till you
see me again!  May I ask a question, Doctor Manette, before I go?''

``I think you may take that liberty,'' the Doctor answered, smiling.

``For gracious sake, don't talk about Liberty; we have quite enough of
that,'' said Miss Pross.

``Hush, dear!  Again?'' Lucie remonstrated.

``Well, my sweet,'' said Miss Pross, nodding her head emphatically,
``the short and the long of it is, that I am a subject of His Most
Gracious Majesty King George the Third;'' Miss Pross curtseyed at the
name; ``and as such, my maxim is, Confound their politics, Frustrate
their knavish tricks, On him our hopes we fix, God save the King!''

Mr.\ Cruncher, in an access of loyalty, growlingly repeated the words
after Miss Pross, like somebody at church.

``I am glad you have so much of the Englishman in you, though I wish
you had never taken that cold in your voice,'' said Miss Pross,
approvingly.  ``But the question, Doctor Manette.  Is there''---it was
the good creature's way to affect to make light of anything that was
a great anxiety with them all, and to come at it in this chance
manner---``is there any prospect yet, of our getting out of this place?''

``I fear not yet.  It would be dangerous for Charles yet.''

``Heigh-ho-hum!'' said Miss Pross, cheerfully repressing a sigh as she
glanced at her darling's golden hair in the light of the fire,
``then we must have patience and wait:  that's all.  We must hold up
our heads and fight low, as my brother Solomon used to say.
Now, Mr.\ Cruncher!---Don't you move, Ladybird!''

They went out, leaving Lucie, and her husband, her father, and the
child, by a bright fire.  Mr.\ Lorry was expected back presently from
the Banking House.  Miss Pross had lighted the lamp, but had put it
aside in a corner, that they might enjoy the fire-light undisturbed.
Little Lucie sat by her grandfather with her hands clasped through
his arm:  and he, in a tone not rising much above a whisper, began to
tell her a story of a great and powerful Fairy who had opened a
prison-wall and let out a captive who had once done the Fairy a
service.  All was subdued and quiet, and Lucie was more at ease than
she had been.

``What is that?'' she cried, all at once.

``My dear!'' said her father, stopping in his story, and laying his
hand on hers, ``command yourself.  What a disordered state you are in!
The least thing---nothing---startles you!  \emph{You}, your father's daughter!''

``I thought, my father,'' said Lucie, excusing herself, with a pale face
and in a faltering voice, ``that I heard strange feet upon the stairs.''

``My love, the staircase is as still as Death.''

As he said the word, a blow was struck upon the door.

``Oh father, father.  What can this be!  Hide Charles.  Save him!''

``My child,'' said the Doctor, rising, and laying his hand upon her
shoulder, ``I \emph{have} saved him.  What weakness is this, my dear!
Let me go to the door.''

He took the lamp in his hand, crossed the two intervening outer
rooms, and opened it.  A rude clattering of feet over the floor,
and four rough men in red caps, armed with sabres and pistols,
entered the room.

``The Citizen Evremonde, called Darnay,'' said the first.

``Who seeks him?'' answered Darnay.

``I seek him.  We seek him.  I know you, Evremonde; I saw you before
the Tribunal to-day.  You are again the prisoner of the Republic.''

The four surrounded him, where he stood with his wife and child
clinging to him.

``Tell me how and why am I again a prisoner?''

``It is enough that you return straight to the Conciergerie, and will
know to-morrow.  You are summoned for to-morrow.''

Doctor Manette, whom this visitation had so turned into stone, that
be stood with the lamp in his hand, as if be woe a statue made to
hold it, moved after these words were spoken, put the lamp down, and
confronting the speaker, and taking him, not ungently, by the loose
front of his red woollen shirt, said:

``You know him, you have said.  Do you know me?''

``Yes, I know you, Citizen Doctor.''

``We all know you, Citizen Doctor,'' said the other three.

He looked abstractedly from one to another, and said, in a lower
voice, after a pause:

``Will you answer his question to me then?  How does this happen?''

``Citizen Doctor,'' said the first, reluctantly, ``he has been denounced
to the Section of Saint Antoine.  This citizen,'' pointing out the
second who had entered, ``is from Saint Antoine.''

The citizen here indicated nodded his head, and added:

``He is accused by Saint Antoine.''

``Of what?'' asked the Doctor.

``Citizen Doctor,'' said the first, with his former reluctance, ``ask no
more.  If the Republic demands sacrifices from you, without doubt you
as a good patriot will be happy to make them.  The Republic goes
before all.  The People is supreme.  Evremonde, we are pressed.''

``One word,'' the Doctor entreated.  ``Will you tell me who denounced him?''

``It is against rule,'' answered the first; ``but you can ask Him of
Saint Antoine here.''

The Doctor turned his eyes upon that man.  Who moved uneasily on his
feet, rubbed his beard a little, and at length said:

``Well!  Truly it is against rule.  But he is denounced---and
gravely---by the Citizen and Citizeness Defarge.  And by one other.''

``What other?''

``Do \emph{you} ask, Citizen Doctor?''

``Yes.''

``Then,'' said he of Saint Antoine, with a strange look, ``you will be
answered to-morrow.  Now, I am dumb!''



\chapter{A Hand at Cards}


Happily unconscious of the new calamity at home, Miss Pross threaded
her way along the narrow streets and crossed the river by the bridge
of the Pont-Neuf, reckoning in her mind the number of indispensable
purchases she had to make.  Mr.\ Cruncher, with the basket, walked at
her side.  They both looked to the right and to the left into most of
the shops they passed, had a wary eye for all gregarious assemblages
of people, and turned out of their road to avoid any very excited
group of talkers.  It was a raw evening, and the misty river, blurred
to the eye with blazing lights and to the ear with harsh noises,
showed where the barges were stationed in which the smiths worked,
making guns for the Army of the Republic.  Woe to the man who played
tricks with \emph{that} Army, or got undeserved promotion in it!  Better
for him that his beard had never grown, for the National Razor shaved
him close.

Having purchased a few small articles of grocery, and a measure of
oil for the lamp, Miss Pross bethought herself of the wine they
wanted. After peeping into several wine-shops, she stopped at the
sign of the Good Republican Brutus of Antiquity, not far from the
National Palace, once (and twice) the Tuileries, where the aspect of
things rather took her fancy.  It had a quieter look than any other
place of the same description they had passed, and, though red with
patriotic caps, was not so red as the rest.  Sounding Mr.\ Cruncher,
and finding him of her opinion, Miss Pross resorted to the Good
Republican Brutus of Antiquity, attended by her cavalier.

Slightly observant of the smoky lights; of the people, pipe in mouth,
playing with limp cards and yellow dominoes; of the one bare-%
breasted, bare-armed, soot-begrimed workman reading a journal aloud,
and of the others listening to him; of the weapons worn, or laid
aside to be resumed; of the two or three customers fallen forward
asleep, who in the popular high-shouldered shaggy black spencer
looked, in that attitude, like slumbering bears or dogs; the two
outlandish customers approached the counter, and showed what they wanted.

As their wine was measuring out, a man parted from another man in a
corner, and rose to depart.  In going, he had to face Miss Pross.
No sooner did he face her, than Miss Pross uttered a scream, and
clapped her hands.

In a moment, the whole company were on their feet.  That somebody was
assassinated by somebody vindicating a difference of opinion was the
likeliest occurrence.  Everybody looked to see somebody fall, but
only saw a man and a woman standing staring at each other; the man
with all the outward aspect of a Frenchman and a thorough Republican;
the woman, evidently English.

What was said in this disappointing anti-climax, by the disciples of
the Good Republican Brutus of Antiquity, except that it was something
very voluble and loud, would have been as so much Hebrew or Chaldean
to Miss Pross and her protector, though they had been all ears.  But,
they had no ears for anything in their surprise.  For, it must be
recorded, that not only was Miss Pross lost in amazement and
agitation, but, Mr.\ Cruncher---though it seemed on his own separate
and individual account---was in a state of the greatest wonder.

``What is the matter?'' said the man who had caused Miss Pross to scream;
speaking in a vexed, abrupt voice (though in a low tone), and in
English.

``Oh, Solomon, dear Solomon!'' cried Miss Pross, clapping her hands
again.  ``After not setting eyes upon you or hearing of you for so
long a time, do I find you here!''

``Don't call me Solomon.  Do you want to be the death of me?''  asked
the man, in a furtive, frightened way.

``Brother, brother!'' cried Miss Pross, bursting into tears.  ``Have I
ever been so hard with you that you ask me such a cruel question?''

``Then hold your meddlesome tongue,'' said Solomon, ``and come out, if
you want to speak to me.  Pay for your wine, and come out.
Who's this man?''

Miss Pross, shaking her loving and dejected head at her by no means
affectionate brother, said through her tears, ``Mr.\ Cruncher.''

``Let him come out too,'' said Solomon.  ``Does he think me a ghost?''

Apparently, Mr.\ Cruncher did, to judge from his looks.  He said not a
word, however, and Miss Pross, exploring the depths of her reticule
through her tears with great difficulty paid for her wine.  As she
did so, Solomon turned to the followers of the Good Republican Brutus
of Antiquity, and offered a few words of explanation in the French
language, which caused them all to relapse into their former places
and pursuits.

``Now,'' said Solomon, stopping at the dark street corner, ``what do you want?''

``How dreadfully unkind in a brother nothing has ever turned my love
away from!'' cried Miss Pross, ``to give me such a greeting, and show
me no affection.''

``There.  Confound it!  There,'' said Solomon, making a dab at Miss
Pross's lips with his own.  ``Now are you content?''

Miss Pross only shook her head and wept in silence.

``If you expect me to be surprised,'' said her brother Solomon, ``I am
not surprised; I knew you were here; I know of most people who are
here.  If you really don't want to endanger my existence---which I half
believe you do---go your ways as soon as possible, and let me go mine.
I am busy.  I am an official.''

``My English brother Solomon,'' mourned Miss Pross, casting up her
tear-fraught eyes, ``that had the makings in him of one of the best
and greatest of men in his native country, an official among
foreigners, and such foreigners!  I would almost sooner have seen the
dear boy lying in his---''

``I said so!'' cried her brother, interrupting.  ``I knew it.  You want
to be the death of me.  I shall be rendered Suspected, by my own
sister.  Just as I am getting on!''

``The gracious and merciful Heavens forbid!'' cried Miss Pross.  ``Far
rather would I never see you again, dear Solomon, though I have ever
loved you truly, and ever shall.  Say but one affectionate word to
me, and tell me there is nothing angry or estranged between us, and I
will detain you no longer.''

Good Miss Pross!  As if the estrangement between them had come of any
culpability of hers.  As if Mr.\ Lorry had not known it for a fact,
years ago, in the quiet corner in Soho, that this precious brother
had spent her money and left her!

He was saying the affectionate word, however, with a far more
grudging condescension and patronage than he could have shown if
their relative merits and positions had been reversed (which is
invariably the case, all the world over), when Mr.\ Cruncher, touching
him on the shoulder, hoarsely and unexpectedly interposed with the
following singular question:

``I say!  Might I ask the favour?  As to whether your name is John
Solomon, or Solomon John?''

The official turned towards him with sudden distrust.  He had not
previously uttered a word.

``Come!'' said Mr.\ Cruncher.  ``Speak out, you know.''  (Which, by the
way, was more than he could do himself.) ``John Solomon, or Solomon
John?  She calls you Solomon, and she must know, being your sister.
And \emph{I} know you're John, you know. Which of the two goes first?
And regarding that name of Pross, likewise.  That warn't your name
over the water.''

``What do you mean?''

``Well, I don't know all I mean, for I can't call to mind what your
name was, over the water.''

``No?''

``No.  But I'll swear it was a name of two syllables.''

``Indeed?''

``Yes.  T'other one's was one syllable.  I know you.  You was a spy---%
witness at the Bailey.  What, in the name of the Father of Lies,
own father to yourself, was you called at that time?''

``Barsad,'' said another voice, striking in.

``That's the name for a thousand pound!'' cried Jerry.

The speaker who struck in, was Sydney Carton.  He had his hands
behind him under the skirts of his riding-coat, and he stood at
Mr.\ Cruncher's elbow as negligently as he might have stood at the Old
Bailey itself.

``Don't be alarmed, my dear Miss Pross.  I arrived at Mr.\ Lorry's,
to his surprise, yesterday evening; we agreed that I would not
present myself elsewhere until all was well, or unless I could be
useful; I present myself here, to beg a little talk with your brother.
I wish you had a better employed brother than Mr.\ Barsad.  I wish
for your sake Mr.\ Barsad was not a Sheep of the Prisons.''

Sheep was a cant word of the time for a spy, under the gaolers.
The spy, who was pale, turned paler, and asked him how he dared---%

``I'll tell you,'' said Sydney.  ``I lighted on you, Mr.\ Barsad, coming
out of the prison of the Conciergerie while I was contemplating the
walls, an hour or more ago.  You have a face to be remembered, and I
remember faces well.  Made curious by seeing you in that connection,
and having a reason, to which you are no stranger, for associating
you with the misfortunes of a friend now very unfortunate, I walked
in your direction.  I walked into the wine-shop here, close after you,
and sat near you.  I had no difficulty in deducing from your unreserved
conversation, and the rumour openly going about among your admirers,
the nature of your calling.  And gradually, what I had done at random,
seemed to shape itself into a purpose, Mr.\ Barsad.''

``What purpose?'' the spy asked.

``It would be troublesome, and might be dangerous, to explain in the
street.  Could you favour me, in confidence, with some minutes of
your company---at the office of Tellson's Bank, for instance?''

``Under a threat?''

``Oh!  Did I say that?''

``Then, why should I go there?''

``Really, Mr.\ Barsad, I can't say, if you can't.''

``Do you mean that you won't say, sir?'' the spy irresolutely asked.

``You apprehend me very clearly, Mr.\ Barsad.  I won't.''

Carton's negligent recklessness of manner came powerfully in aid of
his quickness and skill, in such a business as he had in his secret
mind, and with such a man as he had to do with.  His practised eye
saw it, and made the most of it.

``Now, I told you so,'' said the spy, casting a reproachful look at his
sister; ``if any trouble comes of this, it's your doing.''

``Come, come, Mr.\ Barsad!'' exclaimed Sydney.  ``Don't be
ungrateful. But for my great respect for your sister, I might not
have led up so pleasantly to a little proposal that I wish to make
for our mutual satisfaction.  Do you go with me to the Bank?''

``I'll hear what you have got to say.  Yes, I'll go with you.''

``I propose that we first conduct your sister safely to the corner of
her own street.  Let me take your arm, Miss Pross.  This is not a
good city, at this time, for you to be out in, unprotected; and as
your escort knows Mr.\ Barsad, I will invite him to Mr.\ Lorry's with us.
Are we ready?  Come then!''

Miss Pross recalled soon afterwards, and to the end of her life
remembered, that as she pressed her hands on Sydney's arm and looked
up in his face, imploring him to do no hurt to Solomon, there was a
braced purpose in the arm and a kind of inspiration in the eyes,
which not only contradicted his light manner, but changed and raised
the man.  She was too much occupied then with fears for the brother
who so little deserved her affection, and with Sydney's friendly
reassurances, adequately to heed what she observed.

They left her at the corner of the street, and Carton led the way to
Mr.\ Lorry's, which was within a few minutes' walk.  John Barsad, or
Solomon Pross, walked at his side.

Mr.\ Lorry had just finished his dinner, and was sitting before a
cheery little log or two of fire---perhaps looking into their blaze
for the picture of that younger elderly gentleman from Tellson's, who
had looked into the red coals at the Royal George at Dover, now a
good many years ago.  He turned his head as they entered, and showed
the surprise with which he saw a stranger.

``Miss Pross's brother, sir,'' said Sydney.  ``Mr.\ Barsad.''

``Barsad?'' repeated the old gentleman, ``Barsad?  I have an association
with the name---and with the face.''

``I told you you had a remarkable face, Mr.\ Barsad,'' observed Carton,
coolly.  ``Pray sit down.''

As he took a chair himself, he supplied the link that Mr.\ Lorry
wanted, by saying to him with a frown, ``Witness at that trial.''
Mr.\ Lorry immediately remembered, and regarded his new visitor with
an undisguised look of abhorrence.

``Mr.\ Barsad has been recognised by Miss Pross as the affectionate
brother you have heard of,'' said Sydney, ``and has acknowledged the
relationship.  I pass to worse news.  Darnay has been arrested again.''

Struck with consternation, the old gentleman exclaimed, ``What do you
tell me!  I left him safe and free within these two hours, and am
about to return to him!''

``Arrested for all that.  When was it done, Mr.\ Barsad?''

``Just now, if at all.''

``Mr.\ Barsad is the best authority possible, sir,'' said Sydney, ``and I
have it from Mr.\ Barsad's communication to a friend and brother Sheep
over a bottle of wine, that the arrest has taken place.  He left the
messengers at the gate, and saw them admitted by the porter.
There is no earthly doubt that he is retaken.''

Mr.\ Lorry's business eye read in the speaker's face that it was loss
of time to dwell upon the point.  Confused, but sensible that
something might depend on his presence of mind, he commanded himself,
and was silently attentive.

``Now, I trust,'' said Sydney to him, ``that the name and influence of
Doctor Manette may stand him in as good stead to-morrow---you said he
would be before the Tribunal again to-morrow, Mr.\ Barsad?---''

``Yes; I believe so.''

``---In as good stead to-morrow as to-day.  But it may not be so.
I own to you, I am shaken, Mr.\ Lorry, by Doctor Manette's not having
had the power to prevent this arrest.''

``He may not have known of it beforehand,'' said Mr.\ Lorry.

``But that very circumstance would be alarming, when we remember how
identified he is with his son-in-law.''

``That's true,'' Mr.\ Lorry acknowledged, with his troubled hand at his
chin, and his troubled eyes on Carton.

``In short,'' said Sydney, ``this is a desperate time, when desperate
games are played for desperate stakes.  Let the Doctor play the
winning game; I will play the losing one.  No man's life here is
worth purchase. Any one carried home by the people to-day, may be
condemned tomorrow.  Now, the stake I have resolved to play for, in
case of the worst, is a friend in the Conciergerie.  And the friend I
purpose to myself to win, is Mr.\ Barsad.''

``You need have good cards, sir,'' said the spy.

``I'll run them over.  I'll see what I hold,---Mr.\ Lorry, you know
what a brute I am; I wish you'd give me a little brandy.''

It was put before him, and he drank off a glassful---drank off another
glassful---pushed the bottle thoughtfully away.

``Mr.\ Barsad,'' he went on, in the tone of one who really was looking
over a hand at cards:  ``Sheep of the prisons, emissary of Republican
committees, now turnkey, now prisoner, always spy and secret
informer, so much the more valuable here for being English that an
Englishman is less open to suspicion of subornation in those
characters than a Frenchman, represents himself to his employers
under a false name. That's a very good card.  Mr.\ Barsad, now in the
employ of the republican French government, was formerly in the
employ of the aristocratic English government, the enemy of France
and freedom.  That's an excellent card.  Inference clear as day in
this region of suspicion, that Mr.\ Barsad, still in the pay of the
aristocratic English government, is the spy of Pitt, the treacherous
foe of the Republic crouching in its bosom, the English traitor and
agent of all mischief so much spoken of and so difficult to find.
That's a card not to be beaten.  Have you followed my hand, Mr.\ Barsad?''

``Not to understand your play,'' returned the spy, somewhat uneasily.

``I play my Ace, Denunciation of Mr.\ Barsad to the nearest Section
Committee.  Look over your hand, Mr.\ Barsad, and see what you have.
Don't hurry.''

He drew the bottle near, poured out another glassful of brandy,
and drank it off.  He saw that the spy was fearful of his drinking
himself into a fit state for the immediate denunciation of him.
Seeing it, he poured out and drank another glassful.

``Look over your hand carefully, Mr.\ Barsad.  Take time.''

It was a poorer hand than he suspected.  Mr.\ Barsad saw losing cards
in it that Sydney Carton knew nothing of.  Thrown out of his
honourable employment in England, through too much unsuccessful hard
swearing there---not because he was not wanted there; our English
reasons for vaunting our superiority to secrecy and spies are of very
modern date---he knew that he had crossed the Channel, and accepted
service in France:  first, as a tempter and an eavesdropper among his
own countrymen there:  gradually, as a tempter and an eavesdropper
among the natives.  He knew that under the overthrown government he
had been a spy upon Saint Antoine and Defarge's wine-shop; had
received from the watchful police such heads of information
concerning Doctor Manette's imprisonment, release, and history, as
should serve him for an introduction to familiar conversation with
the Defarges; and tried them on Madame Defarge, and had broken down
with them signally.  He always remembered with fear and trembling,
that that terrible woman had knitted when he talked with her, and had
looked ominously at him as her fingers moved.  He had since seen her,
in the Section of Saint Antoine, over and over again produce her
knitted registers, and denounce people whose lives the guillotine
then surely swallowed up.  He knew, as every one employed as he was
did, that he was never safe; that flight was impossible; that he was
tied fast under the shadow of the axe; and that in spite of his
utmost tergiversation and treachery in furtherance of the reigning
terror, a word might bring it down upon him.  Once denounced, and on
such grave grounds as had just now been suggested to his mind, he
foresaw that the dreadful woman of whose unrelenting character he had
seen many proofs, would produce against him that fatal register, and
would quash his last chance of life.  Besides that all secret men are
men soon terrified, here were surely cards enough of one black suit,
to justify the holder in growing rather livid as he turned them over.

``You scarcely seem to like your hand,'' said Sydney, with the greatest
composure.  ``Do you play?''

``I think, sir,'' said the spy, in the meanest manner, as he turned to
Mr.\ Lorry, ``I may appeal to a gentleman of your years and benevolence,
to put it to this other gentleman, so much your junior, whether he
can under any circumstances reconcile it to his station to play that
Ace of which he has spoken.  I admit that \emph{I} am a spy, and that it
is considered a discreditable station---though it must be filled by
somebody; but this gentleman is no spy, and why should he so demean
himself as to make himself one?''

``I play my Ace, Mr.\ Barsad,'' said Carton, taking the answer on himself,
and looking at his watch, ``without any scruple, in a very few minutes.''

``I should have hoped, gentlemen both,'' said the spy, always striving
to hook Mr.\ Lorry into the discussion, ``that your respect for my
sister---''

``I could not better testify my respect for your sister than by
finally relieving her of her brother,'' said Sydney Carton.

``You think not, sir?''

``I have thoroughly made up my mind about it.''

The smooth manner of the spy, curiously in dissonance with his
ostentatiously rough dress, and probably with his usual demeanour,
received such a check from the inscrutability of Carton,---who was a
mystery to wiser and honester men than he,---that it faltered here and
failed him.  While he was at a loss, Carton said, resuming his former
air of contemplating cards:

``And indeed, now I think again, I have a strong impression that I
have another good card here, not yet enumerated.  That friend and
fellow-Sheep, who spoke of himself as pasturing in the country prisons;
who was he?''

``French.  You don't know him,'' said the spy, quickly.

``French, eh?'' repeated Carton, musing, and not appearing to notice
him at all, though he echoed his word.  ``Well; he may be.''

``Is, I assure you,'' said the spy; ``though it's not important.''

``Though it's not important,'' repeated Carton, in the same mechanical
way---``though it's not important---No, it's not important.  No. Yet I
know the face.''

``I think not.  I am sure not.  It can't be,'' said the spy.

``It-can't-be,'' muttered Sydney Carton, retrospectively, and idling
his glass (which fortunately was a small one) again.  ``Can't-be.
Spoke good French.  Yet like a foreigner, I thought?''

``Provincial,'' said the spy.

``No.  Foreign!'' cried Carton, striking his open hand on the table, as
a light broke clearly on his mind.  ``Cly!  Disguised, but the same man.
We had that man before us at the Old Bailey.''

``Now, there you are hasty, sir,'' said Barsad, with a smile that gave
his aquiline nose an extra inclination to one side; ``there you really
give me an advantage over you.  Cly (who I will unreservedly admit,
at this distance of time, was a partner of mine) has been dead
several years.  I attended him in his last illness.  He was buried in
London, at the church of Saint Pancras-in-the-Fields.  His unpopularity
with the blackguard multitude at the moment prevented my following
his remains, but I helped to lay him in his coffin.''

Here, Mr.\ Lorry became aware, from where he sat, of a most remarkable
goblin shadow on the wall.  Tracing it to its source, he discovered
it to be caused by a sudden extraordinary rising and stiffening of
all the risen and stiff hair on Mr.\ Cruncher's head.

``Let us be reasonable,'' said the spy, ``and let us be fair.  To show
you how mistaken you are, and what an unfounded assumption yours is,
I will lay before you a certificate of Cly's burial, which I happened
to have carried in my pocket-book,'' with a hurried hand he produced
and opened it, ``ever since.  There it is.  Oh, look at it, look at it!
You may take it in your hand; it's no forgery.''

Here, Mr.\ Lorry perceived the reflection on the wall to elongate, and
Mr.\ Cruncher rose and stepped forward.  His hair could not have been
more violently on end, if it had been that moment dressed by the Cow
with the crumpled horn in the house that Jack built.

Unseen by the spy, Mr.\ Cruncher stood at his side, and touched him on
the shoulder like a ghostly bailiff.

``That there Roger Cly, master,'' said Mr.\ Cruncher, with a taciturn
and iron-bound visage.  ``So \emph{you} put him in his coffin?''

``I did.''

``Who took him out of it?''

Barsad leaned back in his chair, and stammered, ``What do you mean?''

``I mean,'' said Mr.\ Cruncher, ``that he warn't never in it.  No!  Not he!
I'll have my head took off, if he was ever in it.''

The spy looked round at the two gentlemen; they both looked in
unspeakable astonishment at Jerry.

``I tell you,'' said Jerry, ``that you buried paving-stones and earth in
that there coffin.  Don't go and tell me that you buried Cly.  It was
a take in. Me and two more knows it.''

``How do you know it?''

``What's that to you?  Ecod!'' growled Mr.\ Cruncher, ``it's you I have got
a old grudge again, is it, with your shameful impositions upon tradesmen!
I'd catch hold of your throat and choke you for half a guinea.''

Sydney Carton, who, with Mr.\ Lorry, had been lost in amazement at
this turn of the business, here requested Mr.\ Cruncher to moderate
and explain himself.

``At another time, sir,'' he returned, evasively, ``the present time is
ill-conwenient for explainin'.  What I stand to, is, that he knows
well wot that there Cly was never in that there coffin.  Let him say
he was, in so much as a word of one syllable, and I'll either catch
hold of his throat and choke him for half a guinea;'' Mr.\ Cruncher
dwelt upon this as quite a liberal offer; ``or I'll out and announce him.''

``Humph!  I see one thing,'' said Carton.  ``I hold another card,
Mr.\ Barsad.  Impossible, here in raging Paris, with Suspicion filling
the air, for you to outlive denunciation, when you are in communication
with another aristocratic spy of the same antecedents as yourself,
who, moreover, has the mystery about him of having feigned death and
come to life again!  A plot in the prisons, of the foreigner against
the Republic. A strong card---a certain Guillotine card!  Do you play?''

``No!'' returned the spy.  ``I throw up.  I confess that we were so
unpopular with the outrageous mob, that I only got away from England
at the risk of being ducked to death, and that Cly was so ferreted up
and down, that he never would have got away at all but for that sham.
Though how this man knows it was a sham, is a wonder of wonders to me.''

``Never you trouble your head about this man,'' retorted the
contentious Mr.\ Cruncher; ``you'll have trouble enough with giving
your attention to that gentleman.  And look here!  Once more!''---%
Mr.\ Cruncher could not be restrained from making rather an ostentatious
parade of his liberality---``I'd catch hold of your throat and choke
you for half a guinea.''

The Sheep of the prisons turned from him to Sydney Carton, and said,
with more decision, ``It has come to a point.  I go on duty soon, and
can't overstay my time.  You told me you had a proposal; what is it?
Now, it is of no use asking too much of me.  Ask me to do anything in
my office, putting my head in great extra danger, and I had better
trust my life to the chances of a refusal than the chances of consent.
In short, I should make that choice.  You talk of desperation.
We are all desperate here.  Remember!  I may denounce you if I think
proper, and I can swear my way through stone walls, and so can others.
Now, what do you want with me?''

``Not very much.  You are a turnkey at the Conciergerie?''

``I tell you once for all, there is no such thing as an escape possible,''
said the spy, firmly.

``Why need you tell me what I have not asked?  You are a turnkey at the
Conciergerie?''

``I am sometimes.''

``You can be when you choose?''

``I can pass in and out when I choose.''

Sydney Carton filled another glass with brandy, poured it slowly out
upon the hearth, and watched it as it dropped.  It being all spent,
he said, rising:

``So far, we have spoken before these two, because it was as well that
the merits of the cards should not rest solely between you and me.
Come into the dark room here, and let us have one final word alone.''



\chapter{The Game Made}


While Sydney Carton and the Sheep of the prisons were in the
adjoining dark room, speaking so low that not a sound was heard,
Mr.\ Lorry looked at Jerry in considerable doubt and mistrust.  That
honest tradesman's manner of receiving the look, did not inspire
confidence; he changed the leg on which he rested, as often as if he
had fifty of those limbs, and were trying them all; he examined his
finger-nails with a very questionable closeness of attention; and
whenever Mr.\ Lorry's eye caught his, he was taken with that peculiar
kind of short cough requiring the hollow of a hand before it, which
is seldom, if ever, known to be an infirmity attendant on perfect
openness of character.

``Jerry,'' said Mr.\ Lorry.  ``Come here.''

Mr.\ Cruncher came forward sideways, with one of his shoulders in
advance of him.

``What have you been, besides a messenger?''

After some cogitation, accompanied with an intent look at his patron,
Mr.\ Cruncher conceived the luminous idea of replying, ``Agicultooral
character.''

``My mind misgives me much,'' said Mr.\ Lorry, angrily shaking a
forefinger at him, ``that you have used the respectable and great
house of Tellson's as a blind, and that you have had an unlawful
occupation of an infamous description.  If you have, don't expect me
to befriend you when you get back to England.  If you have, don't
expect me to keep your secret.  Tellson's shall not be imposed upon.''

``I hope, sir,'' pleaded the abashed Mr.\ Cruncher, ``that a gentleman
like yourself wot I've had the honour of odd jobbing till I'm grey at
it, would think twice about harming of me, even if it wos so---I don't
say it is, but even if it wos.  And which it is to be took into
account that if it wos, it wouldn't, even then, be all o' one side.
There'd be two sides to it. There might be medical doctors at the
present hour, a picking up their guineas where a honest tradesman
don't pick up his fardens---fardens! no, nor yet his half fardens---%
half fardens! no, nor yet his quarter---a banking away like smoke at
Tellson's, and a cocking their medical eyes at that tradesman on the
sly, a going in and going out to their own carriages---ah! equally
like smoke, if not more so.  Well, that 'ud be imposing, too, on
Tellson's.  For you cannot sarse the goose and not the gander.
And here's Mrs.\ Cruncher, or leastways wos in the Old England times,
and would be to-morrow, if cause given, a floppin' again the business
to that degree as is ruinating---stark ruinating!  Whereas them medical
doctors' wives don't flop---catch 'em at it!  Or, if they flop, their
toppings goes in favour of more patients, and how can you rightly
have one without t'other?  Then, wot with undertakers, and wot with
parish clerks, and wot with sextons, and wot with private watchmen
(all awaricious and all in it), a man wouldn't get much by it, even
if it wos so.  And wot little a man did get, would never prosper with
him, Mr.\ Lorry.  He'd never have no good of it; he'd want all along
to be out of the line, if he, could see his way out, being once in---%
even if it wos so.''

``Ugh!'' cried Mr.\ Lorry, rather relenting, nevertheless, ``I am shocked
at the sight of you.''

``Now, what I would humbly offer to you, sir,'' pursued Mr.\ Cruncher,
``even if it wos so, which I don't say it is---''

``Don't prevaricate,'' said Mr.\ Lorry.

``No, I will \emph{not}, sir,'' returned Mr.\ Crunches as if nothing were
further from his thoughts or practice---``which I don't say it is---wot
I would humbly offer to you, sir, would be this.  Upon that there
stool, at that there Bar, sets that there boy of mine, brought up and
growed up to be a man, wot will errand you, message you, general-%
light-job you, till your heels is where your head is, if such should
be your wishes.  If it wos so, which I still don't say it is (for I
will not prewaricate to you, sir), let that there boy keep his
father's place, and take care of his mother; don't blow upon that
boy's father---do not do it, sir---and let that father go into the line
of the reg'lar diggin', and make amends for what he would have
undug---if it wos so-by diggin' of 'em in with a will, and with
conwictions respectin' the futur' keepin' of 'em safe.  That,
Mr.\ Lorry,'' said Mr.\ Cruncher, wiping his forehead with his arm, as
an announcement that he had arrived at the peroration of his
discourse, ``is wot I would respectfully offer to you, sir.  A man
don't see all this here a goin' on dreadful round him, in the way of
Subjects without heads, dear me, plentiful enough fur to bring the
price down to porterage and hardly that, without havin' his serious
thoughts of things.  And these here would be mine, if it wos so,
entreatin' of you fur to bear in mind that wot I said just now, I up
and said in the good cause when I might have kep' it back.''

``That at least is true,'' said Mr.\ Lorry.  ``Say no more now.  It may be
that I shall yet stand your friend, if you deserve it, and repent in
action---not in words.  I want no more words.''

Mr.\ Cruncher knuckled his forehead, as Sydney Carton and the spy
returned from the dark room.  ``Adieu, Mr.\ Barsad,'' said the former;
``our arrangement thus made, you have nothing to fear from me.''

He sat down in a chair on the hearth, over against Mr.\ Lorry.
When they were alone, Mr.\ Lorry asked him what he had done?

``Not much.  If it should go ill with the prisoner, I have ensured
access to him, once.''

Mr.\ Lorry's countenance fell.

``It is all I could do,'' said Carton.  ``To propose too much, would be
to put this man's head under the axe, and, as he himself said,
nothing worse could happen to him if he were denounced.  It was
obviously the weakness of the position.  There is no help for it.''

``But access to him,'' said Mr.\ Lorry, ``if it should go ill before the
Tribunal, will not save him.''

``I never said it would.''

Mr.\ Lorry's eyes gradually sought the fire; his sympathy with his
darling, and the heavy disappointment of his second arrest, gradually
weakened them; he was an old man now, overborne with anxiety of late,
and his tears fell.

``You are a good man and a true friend,'' said Carton, in an altered
voice.  ``Forgive me if I notice that you are affected.  I could not
see my father weep, and sit by, careless.  And I could not respect
your sorrow more, if you were my father.  You are free from that
misfortune, however.''

Though he said the last words, with a slip into his usual manner,
there was a true feeling and respect both in his tone and in his
touch, that Mr.\ Lorry, who had never seen the better side of him,
was wholly unprepared for.  He gave him his hand, and Carton gently
pressed it.

``To return to poor Darnay,'' said Carton.  ``Don't tell Her of this
interview, or this arrangement.  It would not enable Her to go to see
him. She might think it was contrived, in case of the worse, to
convey to him the means of anticipating the sentence.''

Mr.\ Lorry had not thought of that, and he looked quickly at Carton to
see if it were in his mind.  It seemed to be; he returned the look,
and evidently understood it.

``She might think a thousand things,'' Carton said, ``and any of them
would only add to her trouble.  Don't speak of me to her.  As I said
to you when I first came, I had better not see her.  I can put my
hand out, to do any little helpful work for her that my hand can find
to do, without that. You are going to her, I hope?  She must be very
desolate to-night.''

``I am going now, directly.''

``I am glad of that.  She has such a strong attachment to you and
reliance on you.  How does she look?''

``Anxious and unhappy, but very beautiful.''

``Ah!''

It was a long, grieving sound, like a sigh---almost like a sob.  It
attracted Mr.\ Lorry's eyes to Carton's face, which was turned to the
fire.  A light, or a shade (the old gentleman could not have said
which), passed from it as swiftly as a change will sweep over a
hill-side on a wild bright day, and he lifted his foot to put back
one of the little flaming logs, which was tumbling forward.  He wore
the white riding-coat and top-boots, then in vogue, and the light of
the fire touching their light surfaces made him look very pale, with
his long brown hair, all untrimmed, hanging loose about him.  His
indifference to fire was sufficiently remarkable to elicit a word of
remonstrance from Mr.\ Lorry; his boot was still upon the hot embers
of the flaming log, when it had broken under the weight of his foot.

``I forgot it,'' he said.

Mr.\ Lorry's eyes were again attracted to his face.  Taking note of
the wasted air which clouded the naturally handsome features, and
having the expression of prisoners' faces fresh in his mind, he was
strongly reminded of that expression.

``And your duties here have drawn to an end, sir?'' said Carton,
turning to him.

``Yes.  As I was telling you last night when Lucie came in so
unexpectedly, I have at length done all that I can do here.  I hoped
to have left them in perfect safety, and then to have quitted Paris.
I have my Leave to Pass.  I was ready to go.''

They were both silent.

``Yours is a long life to look back upon, sir?'' said Carton, wistfully.

``I am in my seventy-eighth year.''

``You have been useful all your life; steadily and constantly occupied;
trusted, respected, and looked up to?''

``I have been a man of business, ever since I have been a man.
indeed, I may say that I was a man of business when a boy.''

``See what a place you fill at seventy-eight.  How many people will
miss you when you leave it empty!''

``A solitary old bachelor,'' answered Mr.\ Lorry, shaking his
head. ``There is nobody to weep for me.''

``How can you say that?  Wouldn't She weep for you?  Wouldn't her child?''

``Yes, yes, thank God.  I didn't quite mean what I said.''

``It \emph{is} a thing to thank God for; is it not?''

``Surely, surely.''

``If you could say, with truth, to your own solitary heart, to-night,
'I have secured to myself the love and attachment, the gratitude or
respect, of no human creature; I have won myself a tender place in no
regard; I have done nothing good or serviceable to be remembered by!'
your seventy-eight years would be seventy-eight heavy curses; would
they not?''

``You say truly, Mr.\ Carton; I think they would be.''

Sydney turned his eyes again upon the fire, and, after a silence of a
few moments, said:

``I should like to ask you:---Does your childhood seem far off?  Do the
days when you sat at your mother's knee, seem days of very long ago?''

Responding to his softened manner, Mr.\ Lorry answered:

``Twenty years back, yes; at this time of my life, no.  For, as I draw
closer and closer to the end, I travel in the circle, nearer and
nearer to the beginning.  It seems to be one of the kind smoothings
and preparings of the way.  My heart is touched now, by many
remembrances that had long fallen asleep, of my pretty young mother
(and I so old!), and by many associations of the days when what we
call the World was not so real with me, and my faults were not
confirmed in me.''

``I understand the feeling!'' exclaimed Carton, with a bright flush.
``And you are the better for it?''

``I hope so.''

Carton terminated the conversation here, by rising to help him on
with his outer coat; ``But you,'' said Mr.\ Lorry, reverting to the theme,
``you are young.''

``Yes,'' said Carton.  ``I am not old, but my young way was never the
way to age.  Enough of me.''

``And of me, I am sure,'' said Mr.\ Lorry.  ``Are you going out?''

``I'll walk with you to her gate.  You know my vagabond and restless
habits.  If I should prowl about the streets a long time, don't be
uneasy; I shall reappear in the morning.  You go to the Court to-morrow?''

``Yes, unhappily.''

``I shall be there, but only as one of the crowd.  My Spy will find a
place for me.  Take my arm, sir.''

Mr.\ Lorry did so, and they went down-stairs and out in the streets.
A few minutes brought them to Mr.\ Lorry's destination.  Carton left
him there; but lingered at a little distance, and turned back to the
gate again when it was shut, and touched it.  He had heard of her
going to the prison every day.  ``She came out here,'' he said, looking
about him, ``turned this way, must have trod on these stones often.
Let me follow in her steps.''

It was ten o'clock at night when he stood before the prison of La
Force, where she had stood hundreds of times.  A little wood-sawyer,
having closed his shop, was smoking his pipe at his shop-door.

``Good night, citizen,'' said Sydney Carton, pausing in going by;
for, the man eyed him inquisitively.

``Good night, citizen.''

``How goes the Republic?''

``You mean the Guillotine.  Not ill.  Sixty-three to-day.  We shall
mount to a hundred soon.  Samson and his men complain sometimes, of
being exhausted.  Ha, ha, ha!  He is so droll, that Samson.
Such a Barber!''

``Do you often go to see him---''

``Shave?  Always.  Every day.  What a barber!  You have seen him at work?''

``Never.''

``Go and see him when he has a good batch.  Figure this to yourself,
citizen; he shaved the sixty-three to-day, in less than two pipes!
Less than two pipes.  Word of honour!''

As the grinning little man held out the pipe he was smoking, to
explain how he timed the executioner, Carton was so sensible of a
rising desire to strike the life out of him, that he turned away.

``But you are not English,'' said the wood-sawyer, ``though you wear
English dress?''

``Yes,'' said Carton, pausing again, and answering over his shoulder.

``You speak like a Frenchman.''

``I am an old student here.''

``Aha, a perfect Frenchman!  Good night, Englishman.''

``Good night, citizen.''

``But go and see that droll dog,'' the little man persisted, calling
after him.  ``And take a pipe with you!''

Sydney had not gone far out of sight, when he stopped in the middle
of the street under a glimmering lamp, and wrote with his pencil on a
scrap of paper.  Then, traversing with the decided step of one who
remembered the way well, several dark and dirty streets---much dirtier
than usual, for the best public thoroughfares remained uncleansed in
those times of terror---he stopped at a chemist's shop, which the
owner was closing with his own hands.  A small, dim, crooked shop,
kept in a tortuous, up-hill thoroughfare, by a small, dim, crooked man.

Giving this citizen, too, good night, as he confronted him at his
counter, he laid the scrap of paper before him.  ``Whew!'' the chemist
whistled softly, as he read it.  ``Hi! hi! hi!''

Sydney Carton took no heed, and the chemist said:

``For you, citizen?''

``For me.''

``You will be careful to keep them separate, citizen?  You know the
consequences of mixing them?''

``Perfectly.''

Certain small packets were made and given to him.  He put them, one
by one, in the breast of his inner coat, counted out the money for
them, and deliberately left the shop.  ``There is nothing more to do,''
said he, glancing upward at the moon, ``until to-morrow.  I can't sleep.''

It was not a reckless manner, the manner in which he said these words
aloud under the fast-sailing clouds, nor was it more expressive of
negligence than defiance.  It was the settled manner of a tired man,
who had wandered and struggled and got lost, but who at length struck
into his road and saw its end.

Long ago, when he had been famous among his earliest competitors as a
youth of great promise, he had followed his father to the grave.
His mother had died, years before.  These solemn words, which had
been read at his father's grave, arose in his mind as he went down
the dark streets, among the heavy shadows, with the moon and the
clouds sailing on high above him.  ``I am the resurrection and the
life, saith the Lord:  he that believeth in me, though he were dead,
yet shall he live:  and whosoever liveth and believeth in me, shall
never die.''

In a city dominated by the axe, alone at night, with natural sorrow
rising in him for the sixty-three who had been that day put to death,
and for to-morrow's victims then awaiting their doom in the prisons,
and still of to-morrow's and to-morrow's, the chain of association
that brought the words home, like a rusty old ship's anchor from the
deep, might have been easily found.  He did not seek it, but repeated
them and went on.

With a solemn interest in the lighted windows where the people were
going to rest, forgetful through a few calm hours of the horrors
surrounding them; in the towers of the churches, where no prayers
were said, for the popular revulsion had even travelled that length
of self-destruction from years of priestly impostors, plunderers, and
profligates; in the distant burial-places, reserved, as they wrote
upon the gates, for Eternal Sleep; in the abounding gaols; and in the
streets along which the sixties rolled to a death which had become so
common and material, that no sorrowful story of a haunting Spirit
ever arose among the people out of all the working of the Guillotine;
with a solemn interest in the whole life and death of the city
settling down to its short nightly pause in fury; Sydney Carton
crossed the Seine again for the lighter streets.

Few coaches were abroad, for riders in coaches were liable to be
suspected, and gentility hid its head in red nightcaps, and put on
heavy shoes, and trudged.  But, the theatres were all well filled,
and the people poured cheerfully out as he passed, and went chatting
home.  At one of the theatre doors, there was a little girl with a
mother, looking for a way across the street through the mud.
He carried the child over, and before, the timid arm was loosed from
his neck asked her for a kiss.

``I am the resurrection and the life, saith the Lord:  he that
believeth in me, though he were dead, yet shall he live:  and
whosoever liveth and believeth in me, shall never die.''

Now, that the streets were quiet, and the night wore on, the words
were in the echoes of his feet, and were in the air.  Perfectly calm
and steady, he sometimes repeated them to himself as he walked; but,
he heard them always.

The night wore out, and, as he stood upon the bridge listening to the
water as it splashed the river-walls of the Island of Paris, where
the picturesque confusion of houses and cathedral shone bright in the
light of the moon, the day came coldly, looking like a dead face out
of the sky. Then, the night, with the moon and the stars, turned pale
and died, and for a little while it seemed as if Creation were
delivered over to Death's dominion.

But, the glorious sun, rising, seemed to strike those words, that
burden of the night, straight and warm to his heart in its long
bright rays.  And looking along them, with reverently shaded eyes,
a bridge of light appeared to span the air between him and the sun,
while the river sparkled under it.

The strong tide, so swift, so deep, and certain, was like a congenial
friend, in the morning stillness.  He walked by the stream, far from
the houses, and in the light and warmth of the sun fell asleep on the
bank. When he awoke and was afoot again, he lingered there yet a
little longer, watching an eddy that turned and turned purposeless,
until the stream absorbed it, and carried it on to the sea.---``Like me.''

A trading-boat, with a sail of the softened colour of a dead leaf,
then glided into his view, floated by him, and died away.  As its
silent track in the water disappeared, the prayer that had broken up
out of his heart for a merciful consideration of all his poor
blindnesses and errors, ended in the words, ``I am the resurrection
and the life.''

Mr.\ Lorry was already out when he got back, and it was easy to
surmise where the good old man was gone.  Sydney Carton drank nothing
but a little coffee, ate some bread, and, having washed and changed
to refresh himself, went out to the place of trial.

The court was all astir and a-buzz, when the black sheep---whom many
fell away from in dread---pressed him into an obscure corner among the
crowd.  Mr.\ Lorry was there, and Doctor Manette was there.  She was
there, sitting beside her father.

When her husband was brought in, she turned a look upon him, so
sustaining, so encouraging, so full of admiring love and pitying
tenderness, yet so courageous for his sake, that it called the
healthy blood into his face, brightened his glance, and animated his
heart.  If there had been any eyes to notice the influence of her
look, on Sydney Carton, it would have been seen to be the same
influence exactly.

Before that unjust Tribunal, there was little or no order of
procedure, ensuring to any accused person any reasonable hearing.
There could have been no such Revolution, if all laws, forms, and
ceremonies, had not first been so monstrously abused, that the
suicidal vengeance of the Revolution was to scatter them all to the
winds.

Every eye was turned to the jury.  The same determined patriots and
good republicans as yesterday and the day before, and to-morrow and
the day after.  Eager and prominent among them, one man with a
craving face, and his fingers perpetually hovering about his lips,
whose appearance gave great satisfaction to the spectators.  A life-%
thirsting, cannibal-looking, bloody-minded juryman, the Jacques Three
of St. Antoine.  The whole jury, as a jury of dogs empannelled to try
the deer.

Every eye then turned to the five judges and the public prosecutor.
No favourable leaning in that quarter to-day.  A fell, uncompromising,
murderous business-meaning there.  Every eye then sought some other
eye in the crowd, and gleamed at it approvingly; and heads nodded at
one another, before bending forward with a strained attention.

Charles Evremonde, called Darnay.  Released yesterday.  Reaccused and
retaken yesterday.  Indictment delivered to him last night. Suspected
and Denounced enemy of the Republic, Aristocrat, one of a family of
tyrants, one of a race proscribed, for that they had used their
abolished privileges to the infamous oppression of the people.
Charles Evremonde, called Darnay, in right of such proscription,
absolutely Dead in Law.

To this effect, in as few or fewer words, the Public Prosecutor.

The President asked, was the Accused openly denounced or secretly?

``Openly, President.''

``By whom?''

``Three voices.  Ernest Defarge, wine-vendor of St. Antoine.''

``Good.''

``Therese Defarge, his wife.''

``Good.''

``Alexandre Manette, physician.''

A great uproar took place in the court, and in the midst of it,
Doctor Manette was seen, pale and trembling, standing where he had
been seated.

``President, I indignantly protest to you that this is a forgery and a
fraud.  You know the accused to be the husband of my daughter.  My
daughter, and those dear to her, are far dearer to me than my life.
Who and where is the false conspirator who says that I denounce the
husband of my child!''

``Citizen Manette, be tranquil.  To fail in submission to the
authority of the Tribunal would be to put yourself out of Law.
As to what is dearer to you than life, nothing can be so dear to a
good citizen as the Republic.''

Loud acclamations hailed this rebuke.  The President rang his bell,
and with warmth resumed.

``If the Republic should demand of you the sacrifice of your child
herself, you would have no duty but to sacrifice her.  Listen to what
is to follow.  In the meanwhile, be silent!''

Frantic acclamations were again raised.  Doctor Manette sat down,
with his eyes looking around, and his lips trembling; his daughter
drew closer to him.  The craving man on the jury rubbed his hands
together, and restored the usual hand to his mouth.

Defarge was produced, when the court was quiet enough to admit of his
being heard, and rapidly expounded the story of the imprisonment, and
of his having been a mere boy in the Doctor's service, and of the
release, and of the state of the prisoner when released and delivered
to him.  This short examination followed, for the court was quick
with its work.

``You did good service at the taking of the Bastille, citizen?''

``I believe so.''

Here, an excited woman screeched from the crowd:  ``You were one of the
best patriots there.  Why not say so?  You were a cannoneer that day
there, and you were among the first to enter the accursed fortress
when it fell.  Patriots, I speak the truth!''

It was The Vengeance who, amidst the warm commendations of the
audience, thus assisted the proceedings.  The President rang his
bell; but, The Vengeance, warming with encouragement, shrieked,
``I defy that bell!'' wherein she was likewise much commended.

``Inform the Tribunal of what you did that day within the Bastille,
citizen.''

``I knew,'' said Defarge, looking down at his wife, who stood at the
bottom of the steps on which he was raised, looking steadily up at
him; ``I knew that this prisoner, of whom I speak, had been confined
in a cell known as One Hundred and Five, North Tower.  I knew it from
himself. He knew himself by no other name than One Hundred and Five,
North Tower, when he made shoes under my care.  As I serve my gun
that day, I resolve, when the place shall fall, to examine that cell.
It falls.  I mount to the cell, with a fellow-citizen who is one of
the Jury, directed by a gaoler.  I examine it, very closely.  In a
hole in the chimney, where a stone has been worked out and replaced,
I find a written paper.  This is that written paper.  I have made it
my business to examine some specimens of the writing of Doctor
Manette.  This is the writing of Doctor Manette.  I confide this
paper, in the writing of Doctor Manette, to the hands of the President.''

``Let it be read.''

In a dead silence and stillness---the prisoner under trial looking
lovingly at his wife, his wife only looking from him to look with
solicitude at her father, Doctor Manette keeping his eyes fixed on
the reader, Madame Defarge never taking hers from the prisoner,
Defarge never taking his from his feasting wife, and all the other
eyes there intent upon the Doctor, who saw none of them---the paper
was read, as follows.



\chapter{The Substance of the Shadow}


``I, Alexandre Manette, unfortunate physician, native of Beauvais,
and afterwards resident in Paris, write this melancholy paper in my
doleful cell in the Bastille, during the last month of the year,
1767. I write it at stolen intervals, under every difficulty.
I design to secrete it in the wall of the chimney, where I have
slowly and laboriously made a place of concealment for it.  Some
pitying hand may find it there, when I and my sorrows are dust.

``These words are formed by the rusty iron point with which I write
with difficulty in scrapings of soot and charcoal from the chimney,
mixed with blood, in the last month of the tenth year of my captivity.
Hope has quite departed from my breast.  I know from terrible
warnings I have noted in myself that my reason will not long remain
unimpaired, but I solemnly declare that I am at this time in the
possession of my right mind---that my memory is exact and
circumstantial---and that I write the truth as I shall answer for
these my last recorded words, whether they be ever read by men or not,
at the Eternal Judgment-seat.

``One cloudy moonlight night, in the third week of December (I think
the twenty-second of the month) in the year 1757, I was walking on a
retired part of the quay by the Seine for the refreshment of the
frosty air, at an hour's distance from my place of residence in the
Street of the School of Medicine, when a carriage came along behind
me, driven very fast.  As I stood aside to let that carriage pass,
apprehensive that it might otherwise run me down, a head was put out
at the window, and a voice called to the driver to stop.

``The carriage stopped as soon as the driver could rein in his horses,
and the same voice called to me by my name.  I answered.  The carriage
was then so far in advance of me that two gentlemen had time to open
the door and alight before I came up with it.

I observed that they were both wrapped in cloaks, and appeared to
conceal themselves.  As they stood side by side near the carriage
door, I also observed that they both looked of about my own age, or
rather younger, and that they were greatly alike, in stature, manner,
voice, and (as far as I could see) face too.

``\,`You are Doctor Manette?' said one.

``I am.''

``\,`Doctor Manette, formerly of Beauvais,' said the other; `the young
physician, originally an expert surgeon, who within the last year or
two has made a rising reputation in Paris?'

``\,`Gentlemen,' I returned, `I am that Doctor Manette of whom you speak
so graciously.'

``\,`We have been to your residence,' said the first, `and not being so
fortunate as to find you there, and being informed that you were
probably walking in this direction, we followed, in the hope of
overtaking you. Will you please to enter the carriage?'

``The manner of both was imperious, and they both moved, as these
words were spoken, so as to place me between themselves and the
carriage door.  They were armed.  I was not.

``\,`Gentlemen,' said I, `pardon me; but I usually inquire who does me
the honour to seek my assistance, and what is the nature of the case
to which I am summoned.'

``The reply to this was made by him who had spoken second.
'Doctor, your clients are people of condition.  As to the nature of
the case, our confidence in your skill assures us that you will
ascertain it for yourself better than we can describe it.  Enough.
Will you please to enter the carriage?'

``I could do nothing but comply, and I entered it in silence.  They
both entered after me---the last springing in, after putting up the
steps.  The carriage turned about, and drove on at its former speed.

``I repeat this conversation exactly as it occurred.  I have no doubt
that it is, word for word, the same.  I describe everything exactly
as it took place, constraining my mind not to wander from the task.
Where I make the broken marks that follow here, I leave off for the
time, and put my paper in its hiding-place.

* * * *

``The carriage left the streets behind, passed the North Barrier, and
emerged upon the country road.  At two-thirds of a league from the
Barrier---I did not estimate the distance at that time, but afterwards
when I traversed it---it struck out of the main avenue, and presently
stopped at a solitary house, We all three alighted, and walked, by a
damp soft footpath in a garden where a neglected fountain had
overflowed, to the door of the house.  It was not opened immediately,
in answer to the ringing of the bell, and one of my two conductors
struck the man who opened it, with his heavy riding glove, across the
face.

``There was nothing in this action to attract my particular attention,
for I had seen common people struck more commonly than dogs.
But, the other of the two, being angry likewise, struck the man in
like manner with his arm; the look and bearing of the brothers were
then so exactly alike, that I then first perceived them to be twin
brothers.

``From the time of our alighting at the outer gate (which we found
locked, and which one of the brothers had opened to admit us, and had
relocked), I had heard cries proceeding from an upper chamber.  I was
conducted to this chamber straight, the cries growing louder as we
ascended the stairs, and I found a patient in a high fever of the brain,
lying on a bed.

``The patient was a woman of great beauty, and young; assuredly not
much past twenty.  Her hair was torn and ragged, and her arms were
bound to her sides with sashes and handkerchiefs.  I noticed that
these bonds were all portions of a gentleman's dress.  On one of
them, which was a fringed scarf for a dress of ceremony, I saw the
armorial bearings of a Noble, and the letter E.

``I saw this, within the first minute of my contemplation of the
patient; for, in her restless strivings she had turned over on her
face on the edge of the bed, had drawn the end of the scarf into her
mouth, and was in danger of suffocation.  My first act was to put out
my hand to relieve her breathing; and in moving the scarf aside, the
embroidery in the corner caught my sight.

``I turned her gently over, placed my hands upon her breast to calm
her and keep her down, and looked into her face.  Her eyes were
dilated and wild, and she constantly uttered piercing shrieks, and
repeated the words, `My husband, my father, and my brother!'  and
then counted up to twelve, and said, `Hush!' For an instant, and no
more, she would pause to listen, and then the piercing shrieks would
begin again, and she would repeat the cry, `My husband, my father,
and my brother!' and would count up to twelve, and say, `Hush!' There
was no variation in the order, or the manner.  There was no cessation,
but the regular moment's pause, in the utterance of these sounds.

``\,`How long,' I asked, `has this lasted?'

``To distinguish the brothers, I will call them the elder and the
younger; by the elder, I mean him who exercised the most authority.
It was the elder who replied, `Since about this hour last night.'

``\,`She has a husband, a father, and a brother?'

``\,`A brother.'

``\,`I do not address her brother?'

``He answered with great contempt, `No.'

``\,`She has some recent association with the number twelve?'

``The younger brother impatiently rejoined, `With twelve o'clock?'

``\,`See, gentlemen,' said I, still keeping my hands upon her breast,
'how useless I am, as you have brought me!  If I had known what I was
coming to see, I could have come provided.  As it is, time must be
lost.  There are no medicines to be obtained in this lonely place.'

``The elder brother looked to the younger, who said haughtily, `There
is a case of medicines here;' and brought it from a closet, and put
it on the table.

* * * *

``I opened some of the bottles, smelt them, and put the stoppers to my
lips.  If I had wanted to use anything save narcotic medicines that
were poisons in themselves, I would not have administered any of those.

``\,`Do you doubt them?' asked the younger brother.

``\,`You see, monsieur, I am going to use them,' I replied, and said no
more.

``I made the patient swallow, with great difficulty, and after many
efforts, the dose that I desired to give.  As I intended to repeat it
after a while, and as it was necessary to watch its influence, I then
sat down by the side of the bed.  There was a timid and suppressed
woman in attendance (wife of the man down-stairs), who had retreated
into a corner. The house was damp and decayed, indifferently
furnished---evidently, recently occupied and temporarily used.
Some thick old hangings had been nailed up before the windows, to
deaden the sound of the shrieks.  They continued to be uttered in
their regular succession, with the cry, `My husband, my father, and
my brother!'  the counting up to twelve, and `Hush!' The frenzy was
so violent, that I had not unfastened the bandages restraining the
arms; but, I had looked to them, to see that they were not painful.
The only spark of encouragement in the case, was, that my hand upon
the sufferer's breast had this much soothing influence, that for
minutes at a time it tranquillised the figure.  It had no effect upon
the cries; no pendulum could be more regular.

``For the reason that my hand had this effect (I assume), I had sat by
the side of the bed for half an hour, with the two brothers looking
on, before the elder said:

``\,`There is another patient.'

``I was startled, and asked, `Is it a pressing case?'

``\,`You had better see,' he carelessly answered; and took up a light.

* * * *

``The other patient lay in a back room across a second staircase,
which was a species of loft over a stable.  There was a low plastered
ceiling to a part of it; the rest was open, to the ridge of the tiled
roof, and there were beams across.  Hay and straw were stored in that
portion of the place, fagots for firing, and a heap of apples in sand.
I had to pass through that part, to get at the other.  My memory is
circumstantial and unshaken.  I try it with these details, and I see
them all, in this my cell in the Bastille, near the close of the
tenth year of my captivity, as I saw them all that night.

``On some hay on the ground, with a cushion thrown under his head, lay
a handsome peasant boy---a boy of not more than seventeen at the most.
He lay on his back, with his teeth set, his right hand clenched on
his breast, and his glaring eyes looking straight upward.  I could
not see where his wound was, as I kneeled on one knee over him;
but, I could see that he was dying of a wound from a sharp point.

``\,`I am a doctor, my poor fellow,' said I. `Let me examine it.'

``\,`I do not want it examined,' he answered; `let it be.'

``It was under his hand, and I soothed him to let me move his hand
away.  The wound was a sword-thrust, received from twenty to twenty-%
four hours before, but no skill could have saved him if it had been
looked to without delay.  He was then dying fast.  As I turned my
eyes to the elder brother, I saw him looking down at this handsome
boy whose life was ebbing out, as if he were a wounded bird, or hare,
or rabbit; not at all as if he were a fellow-creature.

``\,`How has this been done, monsieur?' said I.

``\,`A crazed young common dog!  A serf!  Forced my brother to draw upon him,
and has fallen by my brother's sword---like a gentleman.'

``There was no touch of pity, sorrow, or kindred humanity, in this
answer.  The speaker seemed to acknowledge that it was inconvenient
to have that different order of creature dying there, and that it
would have been better if he had died in the usual obscure routine of
his vermin kind. He was quite incapable of any compassionate feeling
about the boy, or about his fate.

``The boy's eyes had slowly moved to him as he had spoken, and they
now slowly moved to me.

``\,`Doctor, they are very proud, these Nobles; but we common dogs are
proud too, sometimes.  They plunder us, outrage us, beat us, kill us;
but we have a little pride left, sometimes.  She---have you seen her, Doctor?'

``The shrieks and the cries were audible there, though subdued by the
distance.  He referred to them, as if she were lying in our presence.

``I said, `I have seen her.'

``\,`She is my sister, Doctor.  They have had their shameful rights,
these Nobles, in the modesty and virtue of our sisters, many years,
but we have had good girls among us.  I know it, and have heard my
father say so. She was a good girl.  She was betrothed to a good
young man, too:  a tenant of his.  We were all tenants of his---that man's
who stands there. The other is his brother, the worst of a bad race.'

``It was with the greatest difficulty that the boy gathered bodily
force to speak; but, his spirit spoke with a dreadful emphasis.

``\,`We were so robbed by that man who stands there, as all we common
dogs are by those superior Beings---taxed by him without mercy, obliged
to work for him without pay, obliged to grind our corn at his mill,
obliged to feed scores of his tame birds on our wretched crops, and
forbidden for our lives to keep a single tame bird of our own,
pillaged and plundered to that degree that when we chanced to have a
bit of meat, we ate it in fear, with the door barred and the shutters
closed, that his people should not see it and take it from us---I say,
we were so robbed, and hunted, and were made so poor, that our father
told us it was a dreadful thing to bring a child into the world, and
that what we should most pray for, was, that our women might be barren
and our miserable race die out!'

``I had never before seen the sense of being oppressed, bursting forth
like a fire.  I had supposed that it must be latent in the people
somewhere; but, I had never seen it break out, until I saw it in the
dying boy.

``\,`Nevertheless, Doctor, my sister married.  He was ailing at that
time, poor fellow, and she married her lover, that she might tend and
comfort him in our cottage---our dog-hut, as that man would call it.
She had not been married many weeks, when that man's brother saw her
and admired her, and asked that man to lend her to him---for what are
husbands among us!  He was willing enough, but my sister was good and
virtuous, and hated his brother with a hatred as strong as mine.
What did the two then, to persuade her husband to use his influence
with her, to make her willing?'

``The boy's eyes, which had been fixed on mine, slowly turned to the
looker-on, and I saw in the two faces that all he said was true.
The two opposing kinds of pride confronting one another, I can see,
even in this Bastille; the gentleman's, all negligent indifference;
the peasants, all trodden-down sentiment, and passionate revenge.

``\,`You know, Doctor, that it is among the Rights of these Nobles to
harness us common dogs to carts, and drive us.  They so harnessed him
and drove him.  You know that it is among their Rights to keep us in
their grounds all night, quieting the frogs, in order that their
noble sleep may not be disturbed.  They kept him out in the unwholesome
mists at night, and ordered him back into his harness in the day.
But he was not persuaded.  No!  Taken out of harness one day at noon,
to feed---if he could find food---he sobbed twelve times, once for
every stroke of the bell, and died on her bosom.'

``Nothing human could have held life in the boy but his determination
to tell all his wrong.  He forced back the gathering shadows of death,
as he forced his clenched right hand to remain clenched, and to cover
his wound.

``\,`Then, with that man's permission and even with his aid, his brother
took her away; in spite of what I know she must have told his
brother---and what that is, will not be long unknown to you, Doctor,
if it is now---his brother took her away---for his pleasure and
diversion, for a little while.  I saw her pass me on the road.
When I took the tidings home, our father's heart burst; he never
spoke one of the words that filled it.  I took my young sister (for
I have another) to a place beyond the reach of this man, and where,
at least, she will never be \emph{his} vassal.  Then, I tracked the
brother here, and last night climbed in---a common dog, but sword in
hand.---Where is the loft window?  It was somewhere here?'

``The room was darkening to his sight; the world was narrowing around
him.  I glanced about me, and saw that the hay and straw were
trampled over the floor, as if there had been a struggle.

``\,`She heard me, and ran in.  I told her not to come near us till he
was dead.  He came in and first tossed me some pieces of money; then
struck at me with a whip.  But I, though a common dog, so struck at
him as to make him draw.  Let him break into as many pieces as he
will, the sword that he stained with my common blood; he drew to
defend himself---thrust at me with all his skill for his life.'

``My glance had fallen, but a few moments before, on the fragments of
a broken sword, lying among the hay.  That weapon was a gentleman's.
In another place, lay an old sword that seemed to have been a soldier's.

``\,`Now, lift me up, Doctor; lift me up.  Where is he?'

``\,`He is not here,' I said, supporting the boy, and thinking that he
referred to the brother.

``\,`He!  Proud as these nobles are, he is afraid to see me.  Where is
the man who was here?  turn my face to him.'

``I did so, raising the boy's head against my knee.  But, invested for
the moment with extraordinary power, he raised himself completely:
obliging me to rise too, or I could not have still supported him.

``\,`Marquis,' said the boy, turned to him with his eyes opened wide,
and his right hand raised, `in the days when all these things are to
be answered for, I summon you and yours, to the last of your bad race,
to answer for them.  I mark this cross of blood upon you, as a sign
that I do it.  In the days when all these things are to be answered
for, I summon your brother, the worst of the bad race, to answer for
them separately. I mark this cross of blood upon him, as a sign that
I do it.'

``Twice, he put his hand to the wound in his breast, and with his
forefinger drew a cross in the air.  He stood for an instant with the
finger yet raised, and as it dropped, he dropped with it, and I laid
him down dead.

* * * *

``When I returned to the bedside of the young woman, I found her
raving in precisely the same order of continuity.  I knew that this
might last for many hours, and that it would probably end in the
silence of the grave.

``I repeated the medicines I had given her, and I sat at the side of
the bed until the night was far advanced.  She never abated the
piercing quality of her shrieks, never stumbled in the distinctness
or the order of her words.  They were always `My husband, my father,
and my brother!  One, two, three, four, five, six, seven, eight, nine,
ten, eleven, twelve. Hush!'

``This lasted twenty-six hours from the time when I first saw her.  I
had come and gone twice, and was again sitting by her, when she began
to falter.  I did what little could be done to assist that opportunity,
and by-and-bye she sank into a lethargy, and lay like the dead.

``It was as if the wind and rain had lulled at last, after a long and
fearful storm.  I released her arms, and called the woman to assist
me to compose her figure and the dress she had to.  It was then that
I knew her condition to be that of one in whom the first expectations
of being a mother have arisen; and it was then that I lost the little
hope I had had of her.

``\,`Is she dead?' asked the Marquis, whom I will still describe as the
elder brother, coming booted into the room from his horse.

``\,`Not dead,' said I; `but like to die.'

``\,`What strength there is in these common bodies!' he said, looking
down at her with some curiosity.

``\,`There is prodigious strength,' I answered him, `in sorrow and despair.'

``He first laughed at my words, and then frowned at them.  He moved a
chair with his foot near to mine, ordered the woman away, and said in
a subdued voice,

``\,`Doctor, finding my brother in this difficulty with these hinds,
I recommended that your aid should be invited.  Your reputation is
high, and, as a young man with your fortune to make, you are probably
mindful of your interest.  The things that you see here, are things
to be seen, and not spoken of.'

``I listened to the patient's breathing, and avoided answering.

``\,`Do you honour me with your attention, Doctor?'

``\,`Monsieur,' said I, `in my profession, the communications of
patients are always received in confidence.' I was guarded in my
answer, for I was troubled in my mind with what I had heard and seen.

``Her breathing was so difficult to trace, that I carefully tried the
pulse and the heart.  There was life, and no more.  Looking round as
I resumed my seat, I found both the brothers intent upon me.

* * * *

``I write with so much difficulty, the cold is so severe, I am so
fearful of being detected and consigned to an underground cell and
total darkness, that I must abridge this narrative.  There is no
confusion or failure in my memory; it can recall, and could detail,
every word that was ever spoken between me and those brothers.

``She lingered for a week.  Towards the last, I could understand some
few syllables that she said to me, by placing my ear close to her lips.
She asked me where she was, and I told her; who I was, and I told her.
It was in vain that I asked her for her family name.  She faintly
shook her head upon the pillow, and kept her secret, as the boy had done.

``I had no opportunity of asking her any question, until I had told
the brothers she was sinking fast, and could not live another day.
Until then, though no one was ever presented to her consciousness
save the woman and myself, one or other of them had always jealously
sat behind the curtain at the head of the bed when I was there.
But when it came to that, they seemed careless what communication I
might hold with her; as if---the thought passed through my mind---I
were dying too.

``I always observed that their pride bitterly resented the younger
brother's (as I call him) having crossed swords with a peasant, and
that peasant a boy.  The only consideration that appeared to affect
the mind of either of them was the consideration that this was highly
degrading to the family, and was ridiculous.  As often as I caught
the younger brother's eyes, their expression reminded me that he
disliked me deeply, for knowing what I knew from the boy.  He was
smoother and more polite to me than the elder; but I saw this.
I also saw that I was an incumbrance in the mind of the elder, too.

``My patient died, two hours before midnight---at a time, by my watch,
answering almost to the minute when I had first seen her.  I was
alone with her, when her forlorn young head drooped gently on one
side, and all her earthly wrongs and sorrows ended.

``The brothers were waiting in a room down-stairs, impatient to ride
away.  I had heard them, alone at the bedside, striking their boots
with their riding-whips, and loitering up and down.

``\,`At last she is dead?' said the elder, when I went in.

``\,`She is dead,' said I.

``\,`I congratulate you, my brother,' were his words as he turned round.

``He had before offered me money, which I had postponed taking.  He
now gave me a rouleau of gold.  I took it from his hand, but laid it
on the table.  I had considered the question, and had resolved to
accept nothing.

``\,`Pray excuse me,' said I. `Under the circumstances, no.'

``They exchanged looks, but bent their heads to me as I bent mine to
them, and we parted without another word on either side.

* * * *

``I am weary, weary, weary-worn down by misery.  I cannot read what I
have written with this gaunt hand.

``Early in the morning, the rouleau of gold was left at my door in a
little box, with my name on the outside.  From the first, I had
anxiously considered what I ought to do.  I decided, that day, to
write privately to the Minister, stating the nature of the two cases
to which I had been summoned, and the place to which I had gone:  in
effect, stating all the circumstances.  I knew what Court influence
was, and what the immunities of the Nobles were, and I expected that
the matter would never be heard of; but, I wished to relieve my own
mind.  I had kept the matter a profound secret, even from my wife;
and this, too, I resolved to state in my letter.  I had no apprehension
whatever of my real danger; but I was conscious that there might be
danger for others, if others were compromised by possessing the
knowledge that I possessed.

``I was much engaged that day, and could not complete my letter that
night.  I rose long before my usual time next morning to finish it.
It was the last day of the year.  The letter was lying before me just
completed, when I was told that a lady waited, who wished to see me.

* * * *

``I am growing more and more unequal to the task I have set myself.
It is so cold, so dark, my senses are so benumbed, and the gloom upon
me is so dreadful.

``The lady was young, engaging, and handsome, but not marked for long
life.  She was in great agitation.  She presented herself to me as
the wife of the Marquis St. Evremonde.  I connected the title by
which the boy had addressed the elder brother, with the initial
letter embroidered on the scarf, and had no difficulty in arriving at
the conclusion that I had seen that nobleman very lately.

``My memory is still accurate, but I cannot write the words of our
conversation.  I suspect that I am watched more closely than I was,
and I know not at what times I may be watched.  She had in part
suspected, and in part discovered, the main facts of the cruel story,
of her husband's share in it, and my being resorted to.  She did not
know that the girl was dead.  Her hope had been, she said in great
distress, to show her, in secret, a woman's sympathy.  Her hope had
been to avert the wrath of Heaven from a House that had long been
hateful to the suffering many.

``She had reasons for believing that there was a young sister living,
and her greatest desire was, to help that sister.  I could tell her
nothing but that there was such a sister; beyond that, I knew nothing.
Her inducement to come to me, relying on my confidence, had been the
hope that I could tell her the name and place of abode.  Whereas,
to this wretched hour I am ignorant of both.

* * * *

``These scraps of paper fail me.  One was taken from me, with a
warning, yesterday.  I must finish my record to-day.

``She was a good, compassionate lady, and not happy in her marriage.
How could she be!  The brother distrusted and disliked her, and his
influence was all opposed to her; she stood in dread of him, and in
dread of her husband too.  When I handed her down to the door, there
was a child, a pretty boy from two to three years old, in her carriage.

``\,`For his sake, Doctor,' she said, pointing to him in tears, `I would
do all I can to make what poor amends I can.  He will never prosper
in his inheritance otherwise.  I have a presentiment that if no other
innocent atonement is made for this, it will one day be required of
him.  What I have left to call my own---it is little beyond the worth
of a few jewels---I will make it the first charge of his life to
bestow, with the compassion and lamenting of his dead mother, on this
injured family, if the sister can be discovered.'

``She kissed the boy, and said, caressing him, `It is for thine own
dear sake.  Thou wilt be faithful, little Charles?' The child
answered her bravely, `Yes!' I kissed her hand, and she took him in
her arms, and went away caressing him.  I never saw her more.

``As she had mentioned her husband's name in the faith that I knew it,
I added no mention of it to my letter.  I sealed my letter, and, not
trusting it out of my own hands, delivered it myself that day.

``That night, the last night of the year, towards nine o'clock, a man
in a black dress rang at my gate, demanded to see me, and softly
followed my servant, Ernest Defarge, a youth, up-stairs.  When my
servant came into the room where I sat with my wife---O my wife,
beloved of my heart!  My fair young English wife!---we saw the man,
who was supposed to be at the gate, standing silent behind him.

``An urgent case in the Rue St. Honore, he said.  It would not detain
me, he had a coach in waiting.

``It brought me here, it brought me to my grave.  When I was clear of
the house, a black muffler was drawn tightly over my mouth from
behind, and my arms were pinioned.  The two brothers crossed the road
from a dark corner, and identified me with a single gesture.  The
Marquis took from his pocket the letter I had written, showed it me,
burnt it in the light of a lantern that was held, and extinguished
the ashes with his foot.  Not a word was spoken.  I was brought here,
I was brought to my living grave.

``If it had pleased \emph{God} to put it in the hard heart of either of the
brothers, in all these frightful years, to grant me any tidings of my
dearest wife---so much as to let me know by a word whether alive or
dead---I might have thought that He had not quite abandoned them.
But, now I believe that the mark of the red cross is fatal to them,
and that they have no part in His mercies.  And them and their
descendants, to the last of their race, I, Alexandre Manette, unhappy
prisoner, do this last night of the year 1767, in my unbearable agony,
denounce to the times when all these things shall be answered for.
I denounce them to Heaven and to earth.''

A terrible sound arose when the reading of this document was done. A
sound of craving and eagerness that had nothing articulate in it but
blood.  The narrative called up the most revengeful passions of the
time, and there was not a head in the nation but must have dropped
before it.

Little need, in presence of that tribunal and that auditory, to show
how the Defarges had not made the paper public, with the other
captured Bastille memorials borne in procession, and had kept it,
biding their time.  Little need to show that this detested family
name had long been anathematised by Saint Antoine, and was wrought
into the fatal register.  The man never trod ground whose virtues and
services would have sustained him in that place that day, against
such denunciation.

And all the worse for the doomed man, that the denouncer was a
well-known citizen, his own attached friend, the father of his wife.
One of the frenzied aspirations of the populace was, for imitations
of the questionable public virtues of antiquity, and for sacrifices
and self-immolations on the people's altar.  Therefore when the
President said (else had his own head quivered on his shoulders),
that the good physician of the Republic would deserve better still of
the Republic by rooting out an obnoxious family of Aristocrats, and
would doubtless feel a sacred glow and joy in making his daughter a
widow and her child an orphan, there was wild excitement, patriotic
fervour, not a touch of human sympathy.

``Much influence around him, has that Doctor?'' murmured Madame Defarge,
smiling to The Vengeance.  ``Save him now, my Doctor, save him!''

At every juryman's vote, there was a roar.  Another and another.
Roar and roar.

Unanimously voted.  At heart and by descent an Aristocrat, an enemy
of the Republic, a notorious oppressor of the People.  Back to the
Conciergerie, and Death within four-and-twenty hours!



\chapter{Dusk}


The wretched wife of the innocent man thus doomed to die, fell under
the sentence, as if she had been mortally stricken.  But, she uttered
no sound; and so strong was the voice within her, representing that
it was she of all the world who must uphold him in his misery and not
augment it, that it quickly raised her, even from that shock.

The Judges having to take part in a public demonstration out of
doors, the Tribunal adjourned.  The quick noise and movement of the
court's emptying itself by many passages had not ceased, when Lucie
stood stretching out her arms towards her husband, with nothing in
her face but love and consolation.

``If I might touch him!  If I might embrace him once!  O, good citizens,
if you would have so much compassion for us!''

There was but a gaoler left, along with two of the four men who had
taken him last night, and Barsad.  The people had all poured out to
the show in the streets.  Barsad proposed to the rest, ``Let her
embrace him then; it is but a moment.''  It was silently acquiesced in,
and they passed her over the seats in the hall to a raised place,
where he, by leaning over the dock, could fold her in his arms.

``Farewell, dear darling of my soul.  My parting blessing on my love.
We shall meet again, where the weary are at rest!''

They were her husband's words, as he held her to his bosom.

``I can bear it, dear Charles.  I am supported from above:  don't
suffer for me.  A parting blessing for our child.''

``I send it to her by you.  I kiss her by you.  I say farewell to her by you.''

``My husband.  No! A moment!''  He was tearing himself apart from her.
``We shall not be separated long.  I feel that this will break my heart
by-and-bye; but I will do my duty while I can, and when I leave her,
God will raise up friends for her, as He did for me.''

Her father had followed her, and would have fallen on his knees to
both of them, but that Darnay put out a hand and seized him, crying:

``No, no!  What have you done, what have you done, that you should
kneel to us!  We know now, what a struggle you made of old.  We know,
now what you underwent when you suspected my descent, and when you
knew it.  We know now, the natural antipathy you strove against, and
conquered, for her dear sake.  We thank you with all our hearts, and
all our love and duty.  Heaven be with you!''

Her father's only answer was to draw his hands through his white hair,
and wring them with a shriek of anguish.

``It could not be otherwise,'' said the prisoner.  ``All things have
worked together as they have fallen out. it was the always-vain
endeavour to discharge my poor mother's trust that first brought my
fatal presence near you.  Good could never come of such evil,
a happier end was not in nature to so unhappy a beginning.  Be comforted,
and forgive me.  Heaven bless you!''

As he was drawn away, his wife released him, and stood looking after
him with her hands touching one another in the attitude of prayer,
and with a radiant look upon her face, in which there was even a
comforting smile.  As he went out at the prisoners' door, she turned,
laid her head lovingly on her father's breast, tried to speak to him,
and fell at his feet.

Then, issuing from the obscure corner from which he had never moved,
Sydney Carton came and took her up.  Only her father and Mr.\ Lorry
were with her.  His arm trembled as it raised her, and supported her head.
Yet, there was an air about him that was not all of pity---that had a flush
of pride in it.

``Shall I take her to a coach?  I shall never feel her weight.''

He carried her lightly to the door, and laid her tenderly down in a
coach.  Her father and their old friend got into it, and he took his
seat beside the driver.

When they arrived at the gateway where he had paused in the dark not
many hours before, to picture to himself on which of the rough stones
of the street her feet had trodden, he lifted her again, and carried
her up the staircase to their rooms.  There, he laid her down on a
couch, where her child and Miss Pross wept over her.

``Don't recall her to herself,'' he said, softly, to the latter, ``she is
better so.  Don't revive her to consciousness, while she only faints.''

``Oh, Carton, Carton, dear Carton!'' cried little Lucie, springing up
and throwing her arms passionately round him, in a burst of grief.
``Now that you have come, I think you will do something to help mamma,
something to save papa!  O, look at her, dear Carton!  Can you, of all
the people who love her, bear to see her so?''

He bent over the child, and laid her blooming cheek against his face.
He put her gently from him, and looked at her unconscious mother.

``Before I go,'' he said, and paused---``I may kiss her?''

It was remembered afterwards that when he bent down and touched her
face with his lips, he murmured some words.  The child, who was
nearest to him, told them afterwards, and told her grandchildren when
she was a handsome old lady, that she heard him say, ``A life you love.''

When he had gone out into the next room, he turned suddenly on
Mr.\ Lorry and her father, who were following, and said to the latter:

``You had great influence but yesterday, Doctor Manette; let it at
least be tried.  These judges, and all the men in power, are very
friendly to you, and very recognisant of your services; are they not?''

``Nothing connected with Charles was concealed from me.  I had the
strongest assurances that I should save him; and I did.''  He returned
the answer in great trouble, and very slowly.

``Try them again.  The hours between this and to-morrow afternoon are
few and short, but try.''

``I intend to try.  I will not rest a moment.''

``That's well.  I have known such energy as yours do great things
before now---though never,'' he added, with a smile and a sigh together,
``such great things as this.  But try!  Of little worth as life is when
we misuse it, it is worth that effort.  It would cost nothing to lay
down if it were not.''

``I will go,'' said Doctor Manette, ``to the Prosecutor and the President
straight, and I will go to others whom it is better not to name.
I will write too, and---But stay!  There is a Celebration in the streets,
and no one will be accessible until dark.''

``That's true.  Well!  It is a forlorn hope at the best, and not much
the forlorner for being delayed till dark.  I should like to know how
you speed; though, mind!  I expect nothing!  When are you likely to
have seen these dread powers, Doctor Manette?''

``Immediately after dark, I should hope.  Within an hour or two from this.''

``It will be dark soon after four.  Let us stretch the hour or two.
If I go to Mr.\ Lorry's at nine, shall I hear what you have done,
either from our friend or from yourself?''

``Yes.''

``May you prosper!''

Mr.\ Lorry followed Sydney to the outer door, and, touching him on the
shoulder as he was going away, caused him to turn.

``I have no hope,'' said Mr.\ Lorry, in a low and sorrowful whisper.

``Nor have I.''

``If any one of these men, or all of these men, were disposed to spare
him---which is a large supposition; for what is his life, or any man's
to them!---I doubt if they durst spare him after the demonstration in
the court.''

``And so do I. I heard the fall of the axe in that sound.''

Mr.\ Lorry leaned his arm upon the door-post, and bowed his face upon it.

``Don't despond,'' said Carton, very gently; ``don't grieve.
I encouraged Doctor Manette in this idea, because I felt that it
might one day be consolatory to her.  Otherwise, she might think `his
life was want only thrown away or wasted,' and that might trouble her.''

``Yes, yes, yes,'' returned Mr.\ Lorry, drying his eyes, ``you are
right. But he will perish; there is no real hope.''

``Yes.  He will perish:  there is no real hope,'' echoed Carton.

And walked with a settled step, down-stairs.



\chapter{Darkness}


Sydney Carton paused in the street, not quite decided where to go.
``At Tellson's banking-house at nine,'' he said, with a musing face.
``Shall I do well, in the mean time, to show myself?  I think so.
It is best that these people should know there is such a man as I
here; it is a sound precaution, and may be a necessary preparation.
But care, care, care!  Let me think it out!''

Checking his steps which had begun to tend towards an object, he took
a turn or two in the already darkening street, and traced the thought
in his mind to its possible consequences.  His first impression was
confirmed.  ``It is best,'' he said, finally resolved, ``that these
people should know there is such a man as I here.''  And he turned his
face towards Saint Antoine.

Defarge had described himself, that day, as the keeper of a wine-shop
in the Saint Antoine suburb.  It was not difficult for one who knew
the city well, to find his house without asking any question.  Having
ascertained its situation, Carton came out of those closer streets
again, and dined at a place of refreshment and fell sound asleep
after dinner.  For the first time in many years, he had no strong drink.
Since last night he had taken nothing but a little light thin wine,
and last night he had dropped the brandy slowly down on Mr.\ Lorry's
hearth like a man who had done with it.

It was as late as seven o'clock when he awoke refreshed, and went out
into the streets again.  As he passed along towards Saint Antoine, he
stopped at a shop-window where there was a mirror, and slightly
altered the disordered arrangement of his loose cravat, and his coat-%
collar, and his wild hair.  This done, he went on direct to Defarge's,
and went in.

There happened to be no customer in the shop but Jacques Three,
of the restless fingers and the croaking voice.  This man, whom he
had seen upon the Jury, stood drinking at the little counter, in
conversation with the Defarges, man and wife.  The Vengeance assisted
in the conversation, like a regular member of the establishment.

As Carton walked in, took his seat and asked (in very indifferent
French) for a small measure of wine, Madame Defarge cast a careless
glance at him, and then a keener, and then a keener, and then
advanced to him herself, and asked him what it was he had ordered.

He repeated what he had already said.

``English?'' asked Madame Defarge, inquisitively raising her dark eyebrows.

After looking at her, as if the sound of even a single French word
were slow to express itself to him, he answered, in his former strong
foreign accent.  ``Yes, madame, yes.  I am English!''

Madame Defarge returned to her counter to get the wine, and, as he
took up a Jacobin journal and feigned to pore over it puzzling out
its meaning, he heard her say, ``I swear to you, like Evremonde!''

Defarge brought him the wine, and gave him Good Evening.

``How?''

``Good evening.''

``Oh!  Good evening, citizen,'' filling his glass.  ``Ah! and good wine.
I drink to the Republic.''

Defarge went back to the counter, and said, ``Certainly, a little
like.''  Madame sternly retorted, ``I tell you a good deal like.''
Jacques Three pacifically remarked, ``He is so much in your mind,
see you, madame.''  The amiable Vengeance added, with a laugh, ``Yes,
my faith!  And you are looking forward with so much pleasure to seeing
him once more to-morrow!''

Carton followed the lines and words of his paper, with a slow
forefinger, and with a studious and absorbed face.  They were all
leaning their arms on the counter close together, speaking low.
After a silence of a few moments, during which they all looked
towards him without disturbing his outward attention from the Jacobin
editor, they resumed their conversation.

``It is true what madame says,'' observed Jacques Three.  ``Why stop?
There is great force in that.  Why stop?''

``Well, well,'' reasoned Defarge, ``but one must stop somewhere.
After all, the question is still where?''

``At extermination,'' said madame.

``Magnificent!'' croaked Jacques Three.  The Vengeance, also, highly approved.

``Extermination is good doctrine, my wife,'' said Defarge, rather
troubled; ``in general, I say nothing against it.  But this Doctor has
suffered much; you have seen him to-day; you have observed his face
when the paper was read.''

``I have observed his face!'' repeated madame, contemptuously and
angrily.  ``Yes.  I have observed his face.  I have observed his face
to be not the face of a true friend of the Republic.  Let him take
care of his face!''

``And you have observed, my wife,'' said Defarge, in a deprecatory
manner, ``the anguish of his daughter, which must be a dreadful
anguish to him!''

``I have observed his daughter,'' repeated madame; ``yes, I have
observed his daughter, more times than one.  I have observed her
to-day, and I have observed her other days.  I have observed her
in the court, and I have observed her in the street by the prison.
Let me but lift my finger---!''  She seemed to raise it (the listener's
eyes were always on his paper), and to let it fall with a rattle on
the ledge before her, as if the axe had dropped.

``The citizeness is superb!'' croaked the Juryman.

``She is an Angel!'' said The Vengeance, and embraced her.

``As to thee,'' pursued madame, implacably, addressing her husband,
``if it depended on thee---which, happily, it does not---thou wouldst
rescue this man even now.''

``No!'' protested Defarge.  ``Not if to lift this glass would do it!
But I would leave the matter there.  I say, stop there.''

``See you then, Jacques,'' said Madame Defarge, wrathfully; ``and see
you, too, my little Vengeance; see you both!  Listen!  For other crimes
as tyrants and oppressors, I have this race a long time on my register,
doomed to destruction and extermination.  Ask my husband, is that so.''

``It is so,'' assented Defarge, without being asked.

``In the beginning of the great days, when the Bastille falls, he
finds this paper of to-day, and he brings it home, and in the middle
of the night when this place is clear and shut, we read it, here on
this spot, by the light of this lamp.  Ask him, is that so.''

``It is so,'' assented Defarge.

``That night, I tell him, when the paper is read through, and the lamp
is burnt out, and the day is gleaming in above those shutters and
between those iron bars, that I have now a secret to communicate.
Ask him, is that so.''

``It is so,'' assented Defarge again.

``I communicate to him that secret.  I smite this bosom with these two
hands as I smite it now, and I tell him, `Defarge, I was brought up
among the fishermen of the sea-shore, and that peasant family so
injured by the two Evremonde brothers, as that Bastille paper describes,
is my family.  Defarge, that sister of the mortally wounded boy upon
the ground was my sister, that husband was my sister's husband, that
unborn child was their child, that brother was my brother, that
father was my father, those dead are my dead, and that summons to
answer for those things descends to me!' Ask him, is that so.''

``It is so,'' assented Defarge once more.

``Then tell Wind and Fire where to stop,'' returned madame; ``but don't tell me.''

Both her hearers derived a horrible enjoyment from the deadly nature
of her wrath---the listener could feel how white she was, without
seeing her---and both highly commended it.  Defarge, a weak minority,
interposed a few words for the memory of the compassionate wife of
the Marquis; but only elicited from his own wife a repetition of her
last reply.  ``Tell the Wind and the Fire where to stop; not me!''

Customers entered, and the group was broken up.  The English customer
paid for what he had had, perplexedly counted his change, and asked,
as a stranger, to be directed towards the National Palace.
Madame Defarge took him to the door, and put her arm on his, in
pointing out the road.  The English customer was not without his
reflections then, that it might be a good deed to seize that arm,
lift it, and strike under it sharp and deep.

But, he went his way, and was soon swallowed up in the shadow of the
prison wall.  At the appointed hour, he emerged from it to present
himself in Mr.\ Lorry's room again, where he found the old gentleman
walking to and fro in restless anxiety.  He said he had been with
Lucie until just now, and had only left her for a few minutes, to
come and keep his appointment.  Her father had not been seen, since
he quitted the banking-house towards four o'clock.  She had some
faint hopes that his mediation might save Charles, but they were very
slight.  He had been more than five hours gone:  where could he be?

Mr.\ Lorry waited until ten; but, Doctor Manette not returning, and he
being unwilling to leave Lucie any longer, it was arranged that he
should go back to her, and come to the banking-house again at midnight.
In the meanwhile, Carton would wait alone by the fire for the Doctor.

He waited and waited, and the clock struck twelve; but Doctor Manette
did not come back.  Mr.\ Lorry returned, and found no tidings of him,
and brought none.  Where could he be?

They were discussing this question, and were almost building up some
weak structure of hope on his prolonged absence, when they heard him
on the stairs.  The instant he entered the room, it was plain that
all was lost.

Whether he had really been to any one, or whether he had been all
that time traversing the streets, was never known.  As he stood
staring at them, they asked him no question, for his face told them
everything.

``I cannot find it,'' said he, ``and I must have it.  Where is it?''

His head and throat were bare, and, as he spoke with a helpless look
straying all around, he took his coat off, and let it drop on the floor.

``Where is my bench?  I have been looking everywhere for my bench, and
I can't find it.  What have they done with my work?  Time presses:
I must finish those shoes.''

They looked at one another, and their hearts died within them.

``Come, come!'' said he, in a whimpering miserable way; ``let me get to work.
Give me my work.''

Receiving no answer, he tore his hair, and beat his feet upon the ground,
like a distracted child.

``Don't torture a poor forlorn wretch,'' he implored them, with a dreadful cry;
``but give me my work!  What is to become of us, if those shoes are not done
to-night?''

Lost, utterly lost!

It was so clearly beyond hope to reason with him, or try to restore him,
that---as if by agreement---they each put a hand upon his shoulder,
and soothed him to sit down before the fire, with a promise that he
should have his work presently.  He sank into the chair, and brooded
over the embers, and shed tears.  As if all that had happened since
the garret time were a momentary fancy, or a dream, Mr.\ Lorry saw him
shrink into the exact figure that Defarge had had in keeping.

Affected, and impressed with terror as they both were, by this
spectacle of ruin, it was not a time to yield to such emotions.
His lonely daughter, bereft of her final hope and reliance, appealed
to them both too strongly. Again, as if by agreement, they looked at
one another with one meaning in their faces.
Carton was the first to speak:

``The last chance is gone:  it was not much.  Yes; he had better be
taken to her.  But, before you go, will you, for a moment, steadily
attend to me?  Don't ask me why I make the stipulations I am going to
make, and exact the promise I am going to exact; I have a reason---%
a good one.''

``I do not doubt it,'' answered Mr.\ Lorry.  ``Say on.''

The figure in the chair between them, was all the time monotonously
rocking itself to and fro, and moaning.  They spoke in such a tone as
they would have used if they had been watching by a sick-bed in the night.

Carton stooped to pick up the coat, which lay almost entangling his feet.
As he did so, a small case in which the Doctor was accustomed to
carry the lists of his day's duties, fell lightly on the floor.
Carton took it up, and there was a folded paper in it.  ``We should
look at this!'' he said. Mr.\ Lorry nodded his consent.  He opened it,
and exclaimed, ``Thank \emph{God}!''

``What is it?'' asked Mr.\ Lorry, eagerly.

``A moment!  Let me speak of it in its place.  First,'' he put his hand
in his coat, and took another paper from it, ``that is the certificate
which enables me to pass out of this city.  Look at it.  You see---%
Sydney Carton, an Englishman?''

Mr.\ Lorry held it open in his hand, gazing in his earnest face.

``Keep it for me until to-morrow.  I shall see him to-morrow,
you remember, and I had better not take it into the prison.''

``Why not?''

``I don't know; I prefer not to do so.  Now, take this paper that
Doctor Manette has carried about him.  It is a similar certificate,
enabling him and his daughter and her child, at any time, to pass the
barrier and the frontier!  You see?''

``Yes!''

``Perhaps he obtained it as his last and utmost precaution against
evil, yesterday.  When is it dated?  But no matter; don't stay to look;
put it up carefully with mine and your own.  Now, observe!  I never
doubted until within this hour or two, that he had, or could have
such a paper. It is good, until recalled.  But it may be soon recalled,
and, I have reason to think, will be.''

``They are not in danger?''

``They are in great danger.  They are in danger of denunciation by
Madame Defarge.  I know it from her own lips.  I have overheard words
of that woman's, to-night, which have presented their danger to me in
strong colours.  I have lost no time, and since then, I have seen the
spy. He confirms me.  He knows that a wood-sawyer, living by the
prison wall, is under the control of the Defarges, and has been
rehearsed by Madame Defarge as to his having seen Her''---he never
mentioned Lucie's name---``making signs and signals to prisoners.
It is easy to foresee that the pretence will be the common one, a
prison plot, and that it will involve her life---and perhaps her
child's---and perhaps her father's---for both have been seen with her
at that place.  Don't look so horrified.  You will save them all.''

``Heaven grant I may, Carton!  But how?''

``I am going to tell you how.  It will depend on you, and it could
depend on no better man.  This new denunciation will certainly not
take place until after to-morrow; probably not until two or three
days afterwards; more probably a week afterwards.  You know it is a
capital crime, to mourn for, or sympathise with, a victim of the
Guillotine.  She and her father would unquestionably be guilty of
this crime, and this woman (the inveteracy of whose pursuit cannot
be described) would wait to add that strength to her case, and make
herself doubly sure. You follow me?''

``So attentively, and with so much confidence in what you say, that
for the moment I lose sight,'' touching the back of the Doctor's
chair, even of this distress.``

``You have money, and can buy the means of travelling to the seacoast
as quickly as the journey can be made.  Your preparations have been
completed for some days, to return to England.  Early to-morrow have
your horses ready, so that they may be in starting trim at two o'clock
in the afternoon.''

``It shall be done!''

His manner was so fervent and inspiring, that Mr.\ Lorry caught the
flame, and was as quick as youth.

``You are a noble heart.  Did I say we could depend upon no better man?
Tell her, to-night, what you know of her danger as involving her
child and her father.  Dwell upon that, for she would lay her own
fair head beside her husband's cheerfully.''  He faltered for an instant;
then went on as before.  ``For the sake of her child and her father,
press upon her the necessity of leaving Paris, with them and you,
at that hour.  Tell her that it was her husband's last arrangement.
Tell her that more depends upon it than she dare believe, or hope.
You think that her father, even in this sad state, will submit
himself to her; do you not?''

``I am sure of it.''

``I thought so.  Quietly and steadily have all these arrangements made
in the courtyard here, even to the taking of your own seat in the
carriage. The moment I come to you, take me in, and drive away.''

``I understand that I wait for you under all circumstances?''

``You have my certificate in your hand with the rest, you know,
and will reserve my place.  Wait for nothing but to have my place
occupied, and then for England!''

``Why, then,'' said Mr.\ Lorry, grasping his eager but so firm and
steady hand, ``it does not all depend on one old man, but I shall have
a young and ardent man at my side.''

``By the help of Heaven you shall!  Promise me solemnly that nothing
will influence you to alter the course on which we now stand pledged
to one another.''

``Nothing, Carton.''

``Remember these words to-morrow:  change the course, or delay in it---%
for any reason---and no life can possibly be saved, and many lives
must inevitably be sacrificed.''

``I will remember them.  I hope to do my part faithfully.''

``And I hope to do mine.  Now, good bye!''

Though he said it with a grave smile of earnestness, and though he
even put the old man's hand to his lips, he did not part from him
then. He helped him so far to arouse the rocking figure before the
dying embers, as to get a cloak and hat put upon it, and to tempt it
forth to find where the bench and work were hidden that it still
moaningly besought to have.  He walked on the other side of it and
protected it to the courtyard of the house where the afflicted
heart---so happy in the memorable time when he had revealed his own
desolate heart to it---outwatched the awful night.  He entered the
courtyard and remained there for a few moments alone, looking up at
the light in the window of her room.  Before he went away, he
breathed a blessing towards it, and a Farewell.



\chapter{Fifty-two}


In the black prison of the Conciergerie, the doomed of the day
awaited their fate.  They were in number as the weeks of the year.
Fifty-two were to roll that afternoon on the life-tide of the city to
the boundless everlasting sea.  Before their cells were quit of them,
new occupants were appointed; before their blood ran into the blood
spilled yesterday, the blood that was to mingle with theirs to-morrow
was already set apart.

Two score and twelve were told off.  From the farmer-general of seventy,
whose riches could not buy his life, to the seamstress of twenty,
whose poverty and obscurity could not save her.  Physical diseases,
engendered in the vices and neglects of men, will seize on victims
of all degrees; and the frightful moral disorder, born of unspeakable
suffering, intolerable oppression, and heartless indifference,
smote equally without distinction.

Charles Darnay, alone in a cell, had sustained himself with
no flattering delusion since he came to it from the Tribunal.
In every line of the narrative he had heard, he had heard his condemnation.
He had fully comprehended that no personal influence could possibly save him,
that he was virtually sentenced by the millions, and that units could
avail him nothing.

Nevertheless, it was not easy, with the face of his beloved wife
fresh before him, to compose his mind to what it must bear.  His hold
on life was strong, and it was very, very hard, to loosen; by gradual
efforts and degrees unclosed a little here, it clenched the tighter
there; and when he brought his strength to bear on that hand and it
yielded, this was closed again.  There was a hurry, too, in all his
thoughts, a turbulent and heated working of his heart, that contended
against resignation.  If, for a moment, he did feel resigned, then
his wife and child who had to live after him, seemed to protest and
to make it a selfish thing.

But, all this was at first.  Before long, the consideration that
there was no disgrace in the fate he must meet, and that numbers went
the same road wrongfully, and trod it firmly every day, sprang up to
stimulate him.  Next followed the thought that much of the future
peace of mind enjoyable by the dear ones, depended on his quiet
fortitude.  So, by degrees he calmed into the better state, when he
could raise his thoughts much higher, and draw comfort down.

Before it had set in dark on the night of his condemnation, he had
travelled thus far on his last way.  Being allowed to purchase the
means of writing, and a light, he sat down to write until such time
as the prison lamps should be extinguished.

He wrote a long letter to Lucie, showing her that he had known
nothing of her father's imprisonment, until he had heard of it from
herself, and that he had been as ignorant as she of his father's and
uncle's responsibility for that misery, until the paper had been read.
He had already explained to her that his concealment from herself of
the name he had relinquished, was the one condition---fully
intelligible now---that her father had attached to their betrothal,
and was the one promise he had still exacted on the morning of their
marriage.  He entreated her, for her father's sake, never to seek to
know whether her father had become oblivious of the existence of the
paper, or had had it recalled to him (for the moment, or for good),
by the story of the Tower, on that old Sunday under the dear old
plane-tree in the garden.  If he had preserved any definite remembrance
of it, there could be no doubt that he had supposed it destroyed with
the Bastille, when he had found no mention of it among the relics of
prisoners which the populace had discovered there, and which had been
described to all the world.  He besought her---though he added that he
knew it was needless---to console her father, by impressing him
through every tender means she could think of, with the truth that he
had done nothing for which he could justly reproach himself, but had
uniformly forgotten himself for their joint sakes.  Next to her
preservation of his own last grateful love and blessing, and her
overcoming of her sorrow, to devote herself to their dear child,
he adjured her, as they would meet in Heaven, to comfort her father.

To her father himself, he wrote in the same strain; but, he told her
father that he expressly confided his wife and child to his care.
And he told him this, very strongly, with the hope of rousing him
from any despondency or dangerous retrospect towards which he foresaw
he might be tending.

To Mr.\ Lorry, he commended them all, and explained his worldly affairs.
That done, with many added sentences of grateful friendship and warm
attachment, all was done.  He never thought of Carton.  His mind was
so full of the others, that he never once thought of him.

He had time to finish these letters before the lights were put out.
When he lay down on his straw bed, he thought he had done with this world.

But, it beckoned him back in his sleep, and showed itself in shining
forms.  Free and happy, back in the old house in Soho (though it had
nothing in it like the real house), unaccountably released and light
of heart, he was with Lucie again, and she told him it was all a dream,
and he had never gone away.  A pause of forgetfulness, and then he
had even suffered, and had come back to her, dead and at peace, and yet
there was no difference in him.  Another pause of oblivion, and he
awoke in the sombre morning, unconscious where he was or what had
happened, until it flashed upon his mind, ``this is the day of my death!''

Thus, had he come through the hours, to the day when the fifty-two
heads were to fall.  And now, while he was composed, and hoped that
he could meet the end with quiet heroism, a new action began in his
waking thoughts, which was very difficult to master.

He had never seen the instrument that was to terminate his life.
How high it was from the ground, how many steps it had, where he
would be stood, how he would be touched, whether the touching hands
would be dyed red, which way his face would be turned, whether he
would be the first, or might be the last:  these and many similar
questions, in nowise directed by his will, obtruded themselves over
and over again, countless times.  Neither were they connected with
fear:  he was conscious of no fear.  Rather, they originated in a
strange besetting desire to know what to do when the time came;
a desire gigantically disproportionate to the few swift moments to
which it referred; a wondering that was more like the wondering of
some other spirit within his, than his own.

The hours went on as he walked to and fro, and the clocks struck the
numbers he would never hear again.  Nine gone for ever, ten gone for
ever, eleven gone for ever, twelve coming on to pass away.  After a
hard contest with that eccentric action of thought which had last
perplexed him, he had got the better of it.  He walked up and down,
softly repeating their names to himself.  The worst of the strife was
over.  He could walk up and down, free from distracting fancies,
praying for himself and for them.

Twelve gone for ever.

He had been apprised that the final hour was Three, and he knew he
would be summoned some time earlier, inasmuch as the tumbrils jolted
heavily and slowly through the streets.  Therefore, he resolved to keep
Two before his mind, as the hour, and so to strengthen himself in the
interval that he might be able, after that time, to strengthen others.

Walking regularly to and fro with his arms folded on his breast,
a very different man from the prisoner, who had walked to and fro at
La Force, he heard One struck away from him, without surprise.
The hour had measured like most other hours.  Devoutly thankful to
Heaven for his recovered self-possession, he thought, ``There is but
another now,'' and turned to walk again.

Footsteps in the stone passage outside the door.  He stopped.

The key was put in the lock, and turned.  Before the door was opened,
or as it opened, a man said in a low voice, in English:  ``He has never
seen me here; I have kept out of his way.  Go you in alone; I wait near.
Lose no time!''

The door was quickly opened and closed, and there stood before him
face to face, quiet, intent upon him, with the light of a smile on
his features, and a cautionary finger on his lip, Sydney Carton.

There was something so bright and remarkable in his look, that, for
the first moment, the prisoner misdoubted him to be an apparition of
his own imagining.  But, he spoke, and it was his voice; he took the
prisoner's hand, and it was his real grasp.

``Of all the people upon earth, you least expected to see me?'' he said.

``I could not believe it to be you.  I can scarcely believe it now.
You are not''---the apprehension came suddenly into his mind---``a prisoner?''

``No.  I am accidentally possessed of a power over one of the keepers
here, and in virtue of it I stand before you.  I come from her---%
your wife, dear Darnay.''

The prisoner wrung his hand.

``I bring you a request from her.''

``What is it?''

``A most earnest, pressing, and emphatic entreaty, addressed to you in
the most pathetic tones of the voice so dear to you, that you well remember.''

The prisoner turned his face partly aside.

``You have no time to ask me why I bring it, or what it means; I have
no time to tell you.  You must comply with it---take off those boots
you wear, and draw on these of mine.''

There was a chair against the wall of the cell, behind the
prisoner. Carton, pressing forward, had already, with the speed of
lightning, got him down into it, and stood over him, barefoot.

``Draw on these boots of mine.  Put your hands to them;
put your will to them.  Quick!''

``Carton, there is no escaping from this place; it never can be done.
You will only die with me.  It is madness.''

``It would be madness if I asked you to escape; but do I?  When I ask
you to pass out at that door, tell me it is madness and remain here.
Change that cravat for this of mine, that coat for this of mine.
While you do it, let me take this ribbon from your hair, and shake
out your hair like this of mine!''

With wonderful quickness, and with a strength both of will and action,
that appeared quite supernatural, he forced all these changes upon him.
The prisoner was like a young child in his hands.

``Carton!  Dear Carton!  It is madness.  It cannot be accomplished,
it never can be done, it has been attempted, and has always failed.
I implore you not to add your death to the bitterness of mine.''

``Do I ask you, my dear Darnay, to pass the door?  When I ask that,
refuse.  There are pen and ink and paper on this table.  Is your hand
steady enough to write?''

``It was when you came in.''

``Steady it again, and write what I shall dictate.  Quick, friend, quick!''

Pressing his hand to his bewildered head, Darnay sat down at the table.
Carton, with his right hand in his breast, stood close beside him.

``Write exactly as I speak.''

``To whom do I address it?''

``To no one.''  Carton still had his hand in his breast.

``Do I date it?''

``No.''

The prisoner looked up, at each question.  Carton, standing over him
with his hand in his breast, looked down.

``\,`If you remember,'\,'' said Carton, dictating, ``\,`the words that passed
between us, long ago, you will readily comprehend this when you see it.
You do remember them, I know.  It is not in your nature to forget them.'\,''

He was drawing his hand from his breast; the prisoner chancing to
look up in his hurried wonder as he wrote, the hand stopped, closing
upon something.

``Have you written `forget them'?'' Carton asked.

``I have.  Is that a weapon in your hand?''

``No; I am not armed.''

``What is it in your hand?''

``You shall know directly.  Write on; there are but a few words more.''
He dictated again.  ``\,`I am thankful that the time has come, when I
can prove them.  That I do so is no subject for regret or grief.'\,''
As he said these words with his eyes fixed on the writer, his hand
slowly and softly moved down close to the writer's face.

The pen dropped from Darnay's fingers on the table, and he looked
about him vacantly.

``What vapour is that?'' he asked.

``Vapour?''

``Something that crossed me?''

``I am conscious of nothing; there can be nothing here.  Take up the
pen and finish.  Hurry, hurry!''

As if his memory were impaired, or his faculties disordered, the
prisoner made an effort to rally his attention.  As he looked at
Carton with clouded eyes and with an altered manner of breathing,
Carton---his hand again in his breast---looked steadily at him.

``Hurry, hurry!''

The prisoner bent over the paper, once more.

``\,`If it had been otherwise;'\,'' Carton's hand was again watchfully
and softly stealing down; ``\,`I never should have used the longer
opportunity. If it had been otherwise;'\,'' the hand was at the
prisoner's face; ``\,`I should but have had so much the more to answer
for.  If it had been otherwise---'\,'' Carton looked at the pen and saw
it was trailing off into unintelligible signs.

Carton's hand moved back to his breast no more.  The prisoner sprang
up with a reproachful look, but Carton's hand was close and firm at
his nostrils, and Carton's left arm caught him round the waist.
For a few seconds he faintly struggled with the man who had come
to lay down his life for him; but, within a minute or so, he was
stretched insensible on the ground.

Quickly, but with hands as true to the purpose as his heart was,
Carton dressed himself in the clothes the prisoner had laid aside,
combed back his hair, and tied it with the ribbon the prisoner had
worn.  Then, he softly called, ``Enter there!  Come in!'' and the Spy
presented himself.

``You see?'' said Carton, looking up, as he kneeled on one knee beside
the insensible figure, putting the paper in the breast:  ``is your
hazard very great?''

``Mr.\ Carton,'' the Spy answered, with a timid snap of his fingers,
``my hazard is not \emph{that}, in the thick of business here, if you are
true to the whole of your bargain.''

``Don't fear me.  I will be true to the death.''

``You must be, Mr.\ Carton, if the tale of fifty-two is to be right.
Being made right by you in that dress, I shall have no fear.''

``Have no fear!  I shall soon be out of the way of harming you, and the
rest will soon be far from here, please God!  Now, get assistance and
take me to the coach.''

``You?'' said the Spy nervously.

``Him, man, with whom I have exchanged.  You go out at the gate by
which you brought me in?''

``Of course.''

``I was weak and faint when you brought me in, and I am fainter now
you take me out.  The parting interview has overpowered me.  Such a
thing has happened here, often, and too often.  Your life is in your
own hands. Quick!  Call assistance!''

``You swear not to betray me?'' said the trembling Spy, as he paused
for a last moment.

``Man, man!'' returned Carton, stamping his foot; ``have I sworn by no
solemn vow already, to go through with this, that you waste the
precious moments now?  Take him yourself to the courtyard you know of,
place him yourself in the carriage, show him yourself to Mr.\ Lorry,
tell him yourself to give him no restorative but air, and to remember
my words of last night, and his promise of last night, and drive away!''

The Spy withdrew, and Carton seated himself at the table, resting his
forehead on his hands.  The Spy returned immediately, with two men.

``How, then?'' said one of them, contemplating the fallen figure.  ``So
afflicted to find that his friend has drawn a prize in the lottery of
Sainte Guillotine?''

``A good patriot,'' said the other, ``could hardly have been more
afflicted if the Aristocrat had drawn a blank.''

They raised the unconscious figure, placed it on a litter they had
brought to the door, and bent to carry it away.

``The time is short, Evremonde,'' said the Spy, in a warning voice.

``I know it well,'' answered Carton.  ``Be careful of my friend, I
entreat you, and leave me.''

``Come, then, my children,'' said Barsad.  ``Lift him, and come away!''

The door closed, and Carton was left alone.  Straining his powers of
listening to the utmost, he listened for any sound that might denote
suspicion or alarm.  There was none.  Keys turned, doors clashed,
footsteps passed along distant passages:  no cry was raised, or hurry
made, that seemed unusual.  Breathing more freely in a little while,
he sat down at the table, and listened again until the clock struck Two.

Sounds that he was not afraid of, for he divined their meaning, then
began to be audible.  Several doors were opened in succession, and
finally his own.  A gaoler, with a list in his hand, looked in,
merely saying, ``Follow me, Evremonde!'' and he followed into a large
dark room, at a distance.  It was a dark winter day, and what with
the shadows within, and what with the shadows without, he could but
dimly discern the others who were brought there to have their arms
bound.  Some were standing; some seated.  Some were lamenting, and in
restless motion; but, these were few. The great majority were silent
and still, looking fixedly at the ground.

As he stood by the wall in a dim corner, while some of the fifty-two
were brought in after him, one man stopped in passing, to embrace
him, as having a knowledge of him.  It thrilled him with a great
dread of discovery; but the man went on.  A very few moments after
that, a young woman, with a slight girlish form, a sweet spare face
in which there was no vestige of colour, and large widely opened
patient eyes, rose from the seat where he had observed her sitting,
and came to speak to him.

``Citizen Evremonde,'' she said, touching him with her cold hand.
``I am a poor little seamstress, who was with you in La Force.''

He murmured for answer:  ``True.  I forget what you were accused of?''

``Plots.  Though the just Heaven knows that I am innocent of any.
Is it likely?  Who would think of plotting with a poor little weak
creature like me?''

The forlorn smile with which she said it, so touched him, that tears
started from his eyes.

``I am not afraid to die, Citizen Evremonde, but I have done nothing.
I am not unwilling to die, if the Republic which is to do so much
good to us poor, will profit by my death; but I do not know how that
can be, Citizen Evremonde.  Such a poor weak little creature!''

As the last thing on earth that his heart was to warm and soften to,
it warmed and softened to this pitiable girl.

``I heard you were released, Citizen Evremonde.  I hoped it was true?''

``It was.  But, I was again taken and condemned.''

``If I may ride with you, Citizen Evremonde, will you let me hold your
hand?  I am not afraid, but I am little and weak, and it will give me
more courage.''

As the patient eyes were lifted to his face, he saw a sudden doubt in
them, and then astonishment.  He pressed the work-worn, hunger-worn
young fingers, and touched his lips.

``Are you dying for him?'' she whispered.

``And his wife and child.  Hush!  Yes.''

``O you will let me hold your brave hand, stranger?''

``Hush!  Yes, my poor sister; to the last.''


The same shadows that are falling on the prison, are falling, in that
same hour of the early afternoon, on the Barrier with the crowd about it,
when a coach going out of Paris drives up to be examined.

``Who goes here?  Whom have we within?  Papers!''

The papers are handed out, and read.

``Alexandre Manette.  Physician.  French. Which is he?''

This is he; this helpless, inarticulately murmuring, wandering old
man pointed out.

``Apparently the Citizen-Doctor is not in his right mind?
The Revolution-fever will have been too much for him?''

Greatly too much for him.

``Hah!  Many suffer with it.  Lucie.  His daughter.  French.  Which is she?''

This is she.

``Apparently it must be.  Lucie, the wife of Evremonde; is it not?''

It is.

``Hah!  Evremonde has an assignation elsewhere.  Lucie, her child.
English.  This is she?''

She and no other.

``Kiss me, child of Evremonde.  Now, thou hast kissed a good
Republican; something new in thy family; remember it!  Sydney Carton.
Advocate.  English. Which is he?''

He lies here, in this corner of the carriage.  He, too, is pointed out.

``Apparently the English advocate is in a swoon?''

It is hoped he will recover in the fresher air. It is represented
that he is not in strong health, and has separated sadly from a
friend who is under the displeasure of the Republic.

``Is that all?  It is not a great deal, that!  Many are under the
displeasure of the Republic, and must look out at the little window.
Jarvis Lorry. Banker.  English.  Which is he?''

``I am he.  Necessarily, being the last.''

It is Jarvis Lorry who has replied to all the previous questions.
It is Jarvis Lorry who has alighted and stands with his hand on the
coach door, replying to a group of officials.  They leisurely walk
round the carriage and leisurely mount the box, to look at what
little luggage it carries on the roof; the country-people hanging
about, press nearer to the coach doors and greedily stare in; a
little child, carried by its mother, has its short arm held out for
it, that it may touch the wife of an aristocrat who has gone to the
Guillotine.

``Behold your papers, Jarvis Lorry, countersigned.''

``One can depart, citizen?''

``One can depart.  Forward, my postilions!  A good journey!''

``I salute you, citizens.---And the first danger passed!''

These are again the words of Jarvis Lorry, as he clasps his hands,
and looks upward.  There is terror in the carriage, there is weeping,
there is the heavy breathing of the insensible traveller.

``Are we not going too slowly?  Can they not be induced to go faster?''
asks Lucie, clinging to the old man.

``It would seem like flight, my darling.  I must not urge them too much;
it would rouse suspicion.''

``Look back, look back, and see if we are pursued!''

``The road is clear, my dearest.  So far, we are not pursued.''

Houses in twos and threes pass by us, solitary farms, ruinous
buildings, dye-works, tanneries, and the like, open country, avenues
of leafless trees.  The hard uneven pavement is under us, the soft
deep mud is on either side.  Sometimes, we strike into the skirting
mud, to avoid the stones that clatter us and shake us; sometimes, we
stick in ruts and sloughs there.  The agony of our impatience is then
so great, that in our wild alarm and hurry we are for getting out and
running---hiding---doing anything but stopping.

Out of the open country, in again among ruinous buildings, solitary
farms, dye-works, tanneries, and the like, cottages in twos and
threes, avenues of leafless trees.  Have these men deceived us, and
taken us back by another road?  Is not this the same place twice over?
Thank Heaven, no. A village.  Look back, look back, and see if we are
pursued!  Hush! the posting-house.

Leisurely, our four horses are taken out; leisurely, the coach stands
in the little street, bereft of horses, and with no likelihood upon
it of ever moving again; leisurely, the new horses come into visible
existence, one by one; leisurely, the new postilions follow, sucking
and plaiting the lashes of their whips; leisurely, the old postilions
count their money, make wrong additions, and arrive at dissatisfied
results.  All the time, our overfraught hearts are beating at a rate
that would far outstrip the fastest gallop of the fastest horses ever
foaled.

At length the new postilions are in their saddles, and the old are
left behind.  We are through the village, up the hill, and down the
hill, and on the low watery grounds.  Suddenly, the postilions
exchange speech with animated gesticulation, and the horses are
pulled up, almost on their haunches.  We are pursued?

``Ho!  Within the carriage there.  Speak then!''

``What is it?'' asks Mr.\ Lorry, looking out at window.

``How many did they say?''

``I do not understand you.''

``---At the last post.  How many to the Guillotine to-day?''

``Fifty-two.''

``I said so!  A brave number!  My fellow-citizen here would have it
forty-two; ten more heads are worth having.  The Guillotine goes
handsomely.  I love it.  Hi forward.  Whoop!''

The night comes on dark.  He moves more; he is beginning to revive,
and to speak intelligibly; he thinks they are still together; he asks
him, by his name, what he has in his hand. O pity us, kind Heaven,
and help us!  Look out, look out, and see if we are pursued.

The wind is rushing after us, and the clouds are flying after us, and
the moon is plunging after us, and the whole wild night is in pursuit
of us; but, so far, we are pursued by nothing else.



\chapter{The Knitting Done}


In that same juncture of time when the Fifty-Two awaited their fate
Madame Defarge held darkly ominous council with The Vengeance and
Jacques Three of the Revolutionary Jury.  Not in the wine-shop did
Madame Defarge confer with these ministers, but in the shed of the
wood-sawyer, erst a mender of roads.  The sawyer himself did not
participate in the conference, but abided at a little distance,
like an outer satellite who was not to speak until required, or to
offer an opinion until invited.

``But our Defarge,'' said Jacques Three, ``is undoubtedly a good
Republican?  Eh?''

``There is no better,'' the voluble Vengeance protested in her shrill
notes, ``in France.''

``Peace, little Vengeance,'' said Madame Defarge, laying her hand with
a slight frown on her lieutenant's lips, ``hear me speak.  My husband,
fellow-citizen, is a good Republican and a bold man; he has deserved
well of the Republic, and possesses its confidence.  But my husband
has his weaknesses, and he is so weak as to relent towards this Doctor.''

``It is a great pity,'' croaked Jacques Three, dubiously shaking his
head, with his cruel fingers at his hungry mouth; ``it is not quite
like a good citizen; it is a thing to regret.''

``See you,'' said madame, ``I care nothing for this Doctor, I.  He may
wear his head or lose it, for any interest I have in him; it is all
one to me. But, the Evremonde people are to be exterminated, and the
wife and child must follow the husband and father.''

``She has a fine head for it,'' croaked Jacques Three.  ``I have seen
blue eyes and golden hair there, and they looked charming when Samson
held them up.''  Ogre that he was, he spoke like an epicure.

Madame Defarge cast down her eyes, and reflected a little.

``The child also,'' observed Jacques Three, with a meditative enjoyment
of his words, ``has golden hair and blue eyes.  And we seldom have a
child there.  It is a pretty sight!''

``In a word,'' said Madame Defarge, coming out of her short abstraction,
``I cannot trust my husband in this matter.  Not only do I feel, since
last night, that I dare not confide to him the details of my projects;
but also I feel that if I delay, there is danger of his giving warning,
and then they might escape.''

``That must never be,'' croaked Jacques Three; ``no one must escape.
We have not half enough as it is.  We ought to have six score a day.''

``In a word,'' Madame Defarge went on, ``my husband has not my reason
for pursuing this family to annihilation, and I have not his reason
for regarding this Doctor with any sensibility.  I must act for myself,
therefore.  Come hither, little citizen.''

The wood-sawyer, who held her in the respect, and himself in the
submission, of mortal fear, advanced with his hand to his red cap.

``Touching those signals, little citizen,'' said Madame Defarge,
sternly, ``that she made to the prisoners; you are ready to bear
witness to them this very day?''

``Ay, ay, why not!'' cried the sawyer.  ``Every day, in all weathers,
from two to four, always signalling, sometimes with the little one,
sometimes without.  I know what I know.  I have seen with my eyes.''

He made all manner of gestures while he spoke, as if in incidental
imitation of some few of the great diversity of signals that he had
never seen.

``Clearly plots,'' said Jacques Three.  ``Transparently!''

``There is no doubt of the Jury?'' inquired Madame Defarge, letting her
eyes turn to him with a gloomy smile.

``Rely upon the patriotic Jury, dear citizeness.  I answer for my
fellow-Jurymen.''

``Now, let me see,'' said Madame Defarge, pondering again.  ``Yet once more!
Can I spare this Doctor to my husband?  I have no feeling either way.
Can I spare him?''

``He would count as one head,'' observed Jacques Three, in a low voice.
``We really have not heads enough; it would be a pity, I think.''

``He was signalling with her when I saw her,'' argued Madame Defarge;
``I cannot speak of one without the other; and I must not be silent,
and trust the case wholly to him, this little citizen here.
For, I am not a bad witness.''

The Vengeance and Jacques Three vied with each other in their fervent
protestations that she was the most admirable and marvellous of
witnesses.  The little citizen, not to be outdone, declared her to be
a celestial witness.

``He must take his chance,'' said Madame Defarge.  ``No, I cannot spare
him!  You are engaged at three o'clock; you are going to see the batch
of to-day executed.---You?''

The question was addressed to the wood-sawyer, who hurriedly replied
in the affirmative:  seizing the occasion to add that he was the most
ardent of Republicans, and that he would be in effect the most
desolate of Republicans, if anything prevented him from enjoying the
pleasure of smoking his afternoon pipe in the contemplation of the
droll national barber.  He was so very demonstrative herein, that he
might have been suspected (perhaps was, by the dark eyes that looked
contemptuously at him out of Madame Defarge's head) of having his small
individual fears for his own personal safety, every hour in the day.

``I,'' said madame, ``am equally engaged at the same place.  After it is
over-say at eight to-night---come you to me, in Saint Antoine, and we
will give information against these people at my Section.''

The wood-sawyer said he would be proud and flattered to attend the
citizeness.  The citizeness looking at him, he became embarrassed,
evaded her glance as a small dog would have done, retreated among
his wood, and hid his confusion over the handle of his saw.

Madame Defarge beckoned the Juryman and The Vengeance a little nearer
to the door, and there expounded her further views to them thus:

``She will now be at home, awaiting the moment of his death.  She will
be mourning and grieving.  She will be in a state of mind to impeach
the justice of the Republic.  She will be full of sympathy with its
enemies.  I will go to her.''

``What an admirable woman; what an adorable woman!'' exclaimed
Jacques Three, rapturously.  ``Ah, my cherished!'' cried The Vengeance;
and embraced her.

``Take you my knitting,'' said Madame Defarge, placing it in her
lieutenant's hands, ``and have it ready for me in my usual seat.
Keep me my usual chair.  Go you there, straight, for there will
probably be a greater concourse than usual, to-day.''

``I willingly obey the orders of my Chief,'' said The Vengeance with
alacrity, and kissing her cheek.  ``You will not be late?''

``I shall be there before the commencement.''

``And before the tumbrils arrive.  Be sure you are there, my soul,''
said The Vengeance, calling after her, for she had already turned
into the street, ``before the tumbrils arrive!''

Madame Defarge slightly waved her hand, to imply that she heard, and
might be relied upon to arrive in good time, and so went through the
mud, and round the corner of the prison wall.  The Vengeance and the
Juryman, looking after her as she walked away, were highly appreciative
of her fine figure, and her superb moral endowments.

There were many women at that time, upon whom the time laid a
dreadfully disfiguring hand; but, there was not one among them more
to be dreaded than this ruthless woman, now taking her way along the
streets.  Of a strong and fearless character, of shrewd sense and
readiness, of great determination, of that kind of beauty which not
only seems to impart to its possessor firmness and animosity, but to
strike into others an instinctive recognition of those qualities; the
troubled time would have heaved her up, under any circumstances.
But, imbued from her childhood with a brooding sense of wrong, and an
inveterate hatred of a class, opportunity had developed her into a
tigress.  She was absolutely without pity.  If she had ever had the
virtue in her, it had quite gone out of her.

It was nothing to her, that an innocent man was to die for the sins
of his forefathers; she saw, not him, but them.  It was nothing to her,
that his wife was to be made a widow and his daughter an orphan; that
was insufficient punishment, because they were her natural enemies
and her prey, and as such had no right to live.  To appeal to her,
was made hopeless by her having no sense of pity, even for herself.
If she had been laid low in the streets, in any of the many encounters
in which she had been engaged, she would not have pitied herself;
nor, if she had been ordered to the axe to-morrow, would she have
gone to it with any softer feeling than a fierce desire to change
places with the man who sent here there.

Such a heart Madame Defarge carried under her rough robe.  Carelessly
worn, it was a becoming robe enough, in a certain weird way, and her
dark hair looked rich under her coarse red cap.  Lying hidden in her
bosom, was a loaded pistol.  Lying hidden at her waist, was a sharpened
dagger.  Thus accoutred, and walking with the confident tread of such
a character, and with the supple freedom of a woman who had habitually
walked in her girlhood, bare-foot and bare-legged, on the brown
sea-sand, Madame Defarge took her way along the streets.

Now, when the journey of the travelling coach, at that very moment
waiting for the completion of its load, had been planned out last
night, the difficulty of taking Miss Pross in it had much engaged
Mr.\ Lorry's attention.  It was not merely desirable to avoid
overloading the coach, but it was of the highest importance that the
time occupied in examining it and its passengers, should be reduced
to the utmost; since their escape might depend on the saving of only
a few seconds here and there.  Finally, he had proposed, after anxious
consideration, that Miss Pross and Jerry, who were at liberty to
leave the city, should leave it at three o'clock in the lightest-%
wheeled conveyance known to that period.  Unencumbered with luggage,
they would soon overtake the coach, and, passing it and preceding it
on the road, would order its horses in advance, and greatly facilitate
its progress during the precious hours of the night, when delay was
the most to be dreaded.

Seeing in this arrangement the hope of rendering real service in that
pressing emergency, Miss Pross hailed it with joy.  She and Jerry had
beheld the coach start, had known who it was that Solomon brought,
had passed some ten minutes in tortures of suspense, and were now
concluding their arrangements to follow the coach, even as Madame
Defarge, taking her way through the streets, now drew nearer and
nearer to the else-deserted lodging in which they held their consultation.

``Now what do you think, Mr.\ Cruncher,'' said Miss Pross, whose
agitation was so great that she could hardly speak, or stand,
or move, or live:  ``what do you think of our not starting from this
courtyard?  Another carriage having already gone from here to-day,
it might awaken suspicion.''

``My opinion, miss,'' returned Mr.\ Cruncher, ``is as you're right.
Likewise wot I'll stand by you, right or wrong.''

``I am so distracted with fear and hope for our precious creatures,''
said Miss Pross, wildly crying, ``that I am incapable of forming any
plan. Are \emph{you} capable of forming any plan, my dear good Mr.\ Cruncher?''

``Respectin' a future spear o' life, miss,'' returned Mr.\ Cruncher,
``I hope so.  Respectin' any present use o' this here blessed old head
o' mind, I think not.  Would you do me the favour, miss, to take
notice o' two promises and wows wot it is my wishes fur to record in
this here crisis?''

``Oh, for gracious sake!'' cried Miss Pross, still wildly crying,
``record them at once, and get them out of the way, like an excellent man.''

``First,'' said Mr.\ Cruncher, who was all in a tremble, and who spoke
with an ashy and solemn visage, ``them poor things well out o' this,
never no more will I do it, never no more!''

``I am quite sure, Mr.\ Cruncher,'' returned Miss Pross, ``that you never
will do it again, whatever it is, and I beg you not to think it
necessary to mention more particularly what it is.''

``No, miss,'' returned Jerry, ``it shall not be named to you.  Second:
them poor things well out o' this, and never no more will I interfere
with Mrs.\ Cruncher's flopping, never no more!''

``Whatever housekeeping arrangement that may be,'' said Miss Pross,
striving to dry her eyes and compose herself, ``I have no doubt it
is best that Mrs.\ Cruncher should have it entirely under her own
superintendence.---O my poor darlings!''

``I go so far as to say, miss, moreover,'' proceeded Mr.\ Cruncher, with
a most alarming tendency to hold forth as from a pulpit---``and let my
words be took down and took to Mrs.\ Cruncher through yourself---that
wot my opinions respectin' flopping has undergone a change, and that
wot I only hope with all my heart as Mrs.\ Cruncher may be a flopping
at the present time.''

``There, there, there!  I hope she is, my dear man,'' cried the distracted
Miss Pross, ``and I hope she finds it answering her expectations.''

``Forbid it,'' proceeded Mr.\ Cruncher, with additional solemnity,
additional slowness, and additional tendency to hold forth and hold
out, ``as anything wot I have ever said or done should be wisited on
my earnest wishes for them poor creeturs now!  Forbid it as we shouldn't
all flop (if it was anyways conwenient) to get 'em out o' this here
dismal risk!  Forbid it, miss!  Wot I say, for-\emph{bid} it!''  This was
Mr.\ Cruncher's conclusion after a protracted but vain endeavour
to find a better one.

And still Madame Defarge, pursuing her way along the streets, came
nearer and nearer.

``If we ever get back to our native land,'' said Miss Pross, ``you may
rely upon my telling Mrs.\ Cruncher as much as I may be able to remember
and understand of what you have so impressively said; and at all
events you may be sure that I shall bear witness to your being
thoroughly in earnest at this dreadful time.  Now, pray let us think!
My esteemed Mr.\ Cruncher, let us think!''

Still, Madame Defarge, pursuing her way along the streets, came
nearer and nearer.

``If you were to go before,'' said Miss Pross, ``and stop the vehicle
and horses from coming here, and were to wait somewhere for me;
wouldn't that be best?''

Mr.\ Cruncher thought it might be best.

``Where could you wait for me?'' asked Miss Pross.

Mr.\ Cruncher was so bewildered that he could think of no locality but
Temple Bar.  Alas!  Temple Bar was hundreds of miles away, and Madame
Defarge was drawing very near indeed.

``By the cathedral door,'' said Miss Pross.  ``Would it be much out of the
way, to take me in, near the great cathedral door between the two towers?''

``No, miss,'' answered Mr.\ Cruncher.

``Then, like the best of men,'' said Miss Pross, ``go to the posting-%
house straight, and make that change.''

``I am doubtful,'' said Mr.\ Cruncher, hesitating and shaking his head,
``about leaving of you, you see.  We don't know what may happen.''

``Heaven knows we don't,'' returned Miss Pross, ``but have no fear for
me. Take me in at the cathedral, at Three o'Clock, or as near it as
you can, and I am sure it will be better than our going from here.
I feel certain of it.  There!  Bless you, Mr.\ Cruncher!  Think-not of
me, but of the lives that may depend on both of us!''

This exordium, and Miss Pross's two hands in quite agonised entreaty
clasping his, decided Mr.\ Cruncher.  With an encouraging nod or two,
he immediately went out to alter the arrangements, and left her by
herself to follow as she had proposed.

The having originated a precaution which was already in course of
execution, was a great relief to Miss Pross.  The necessity of
composing her appearance so that it should attract no special notice
in the streets, was another relief.  She looked at her watch, and it
was twenty minutes past two.  She had no time to lose, but must get
ready at once.

Afraid, in her extreme perturbation, of the loneliness of the
deserted rooms, and of half-imagined faces peeping from behind every
open door in them, Miss Pross got a basin of cold water and began
laving her eyes, which were swollen and red.  Haunted by her feverish
apprehensions, she could not bear to have her sight obscured for a
minute at a time by the dripping water, but constantly paused and
looked round to see that there was no one watching her.  In one of
those pauses she recoiled and cried out, for she saw a figure
standing in the room.

The basin fell to the ground broken, and the water flowed to the feet
of Madame Defarge.  By strange stern ways, and through much staining
blood, those feet had come to meet that water.

Madame Defarge looked coldly at her, and said, ``The wife of Evremonde;
where is she?''

It flashed upon Miss Pross's mind that the doors were all standing
open, and would suggest the flight.  Her first act was to shut them.
There were four in the room, and she shut them all.  She then placed
herself before the door of the chamber which Lucie had occupied.

Madame Defarge's dark eyes followed her through this rapid movement,
and rested on her when it was finished.  Miss Pross had nothing
beautiful about her; years had not tamed the wildness, or softened
the grimness, of her appearance; but, she too was a determined woman
in her different way, and she measured Madame Defarge with her eyes,
every inch.

``You might, from your appearance, be the wife of Lucifer,'' said Miss
Pross, in her breathing.  ``Nevertheless, you shall not get the better
of me. I am an Englishwoman.''

Madame Defarge looked at her scornfully, but still with something of
Miss Pross's own perception that they two were at bay.  She saw a
tight, hard, wiry woman before her, as Mr.\ Lorry had seen in the same
figure a woman with a strong hand, in the years gone by.  She knew
full well that Miss Pross was the family's devoted friend; Miss Pross
knew full well that Madame Defarge was the family's malevolent enemy.

``On my way yonder,'' said Madame Defarge, with a slight movement of
her hand towards the fatal spot, ``where they reserve my chair and my
knitting for me, I am come to make my compliments to her in passing.
I wish to see her.''

``I know that your intentions are evil,'' said Miss Pross, ``and you may
depend upon it, I'll hold my own against them.''

Each spoke in her own language; neither understood the other's words;
both were very watchful, and intent to deduce from look and manner,
what the unintelligible words meant.

``It will do her no good to keep herself concealed from me at this
moment,'' said Madame Defarge.  ``Good patriots will know what that means.
Let me see her.  Go tell her that I wish to see her.  Do you hear?''

``If those eyes of yours were bed-winches,'' returned Miss Pross, ``and
I was an English four-poster, they shouldn't loose a splinter of me.
No, you wicked foreign woman; I am your match.''

Madame Defarge was not likely to follow these idiomatic remarks in
detail; but, she so far understood them as to perceive that she was
set at naught.

``Woman imbecile and pig-like!'' said Madame Defarge, frowning.
``I take no answer from you.  I demand to see her.  Either tell her
that I demand to see her, or stand out of the way of the door and let
me go to her!''  This, with an angry explanatory wave of her right arm.

``I little thought,'' said Miss Pross, ``that I should ever want to
understand your nonsensical language; but I would give all I have,
except the clothes I wear, to know whether you suspect the truth, or
any part of it.''

Neither of them for a single moment released the other's eyes.
Madame Defarge had not moved from the spot where she stood when Miss
Pross first became aware of her; but, she now advanced one step.

``I am a Briton,'' said Miss Pross, ``I am desperate.  I don't care an
English Twopence for myself.  I know that the longer I keep you here,
the greater hope there is for my Ladybird.  I'll not leave a handful
of that dark hair upon your head, if you lay a finger on me!''

Thus Miss Pross, with a shake of her head and a flash of her eyes
between every rapid sentence, and every rapid sentence a whole breath.
Thus Miss Pross, who had never struck a blow in her life.

But, her courage was of that emotional nature that it brought the
irrepressible tears into her eyes.  This was a courage that Madame
Defarge so little comprehended as to mistake for weakness.  ``Ha, ha!''
she laughed, ``you poor wretch!  What are you worth!  I address myself
to that Doctor.''  Then she raised her voice and called out, ``Citizen
Doctor!  Wife of Evremonde!  Child of Evremonde!  Any person but this
miserable fool, answer the Citizeness Defarge!''

Perhaps the following silence, perhaps some latent disclosure in the
expression of Miss Pross's face, perhaps a sudden misgiving apart from
either suggestion, whispered to Madame Defarge that they were gone.
Three of the doors she opened swiftly, and looked in.

``Those rooms are all in disorder, there has been hurried packing,
there are odds and ends upon the ground.  There is no one in that
room behind you!  Let me look.''

``Never!'' said Miss Pross, who understood the request as perfectly as
Madame Defarge understood the answer.

``If they are not in that room, they are gone, and can be pursued and
brought back,'' said Madame Defarge to herself.

``As long as you don't know whether they are in that room or not, you
are uncertain what to do,'' said Miss Pross to herself; ``and you shall
not know that, if I can prevent your knowing it; and know that, or
not know that, you shall not leave here while I can hold you.''

``I have been in the streets from the first, nothing has stopped me,
I will tear you to pieces, but I will have you from that door,'' said
Madame Defarge.

``We are alone at the top of a high house in a solitary courtyard,
we are not likely to be heard, and I pray for bodily strength to keep
you here, while every minute you are here is worth a hundred thousand
guineas to my darling,'' said Miss Pross.

Madame Defarge made at the door.  Miss Pross, on the instinct of the
moment, seized her round the waist in both her arms, and held her
tight. It was in vain for Madame Defarge to struggle and to strike;
Miss Pross, with the vigorous tenacity of love, always so much
stronger than hate, clasped her tight, and even lifted her from the
floor in the struggle that they had.  The two hands of Madame Defarge
buffeted and tore her face; but, Miss Pross, with her head down, held
her round the waist, and clung to her with more than the hold of a
drowning woman.

Soon, Madame Defarge's hands ceased to strike, and felt at her
encircled waist.  ``It is under my arm,'' said Miss Pross, in smothered
tones, ``you shall not draw it.  I am stronger than you, I bless
Heaven for it.  I hold you till one or other of us faints or dies!''

Madame Defarge's hands were at her bosom.  Miss Pross looked up, saw
what it was, struck at it, struck out a flash and a crash, and stood
alone---blinded with smoke.

All this was in a second.  As the smoke cleared, leaving an awful
stillness, it passed out on the air, like the soul of the furious
woman whose body lay lifeless on the ground.

In the first fright and horror of her situation, Miss Pross passed
the body as far from it as she could, and ran down the stairs to call
for fruitless help.  Happily, she bethought herself of the
consequences of what she did, in time to check herself and go back.
It was dreadful to go in at the door again; but, she did go in, and
even went near it, to get the bonnet and other things that she must
wear.  These she put on, out on the staircase, first shutting and
locking the door and taking away the key.  She then sat down on the
stairs a few moments to breathe and to cry, and then got up and
hurried away.

By good fortune she had a veil on her bonnet, or she could hardly
have gone along the streets without being stopped.  By good fortune,
too, she was naturally so peculiar in appearance as not to show
disfigurement like any other woman.  She needed both advantages, for
the marks of gripping fingers were deep in her face, and her hair was
torn, and her dress (hastily composed with unsteady hands) was
clutched and dragged a hundred ways.

In crossing the bridge, she dropped the door key in the river.
Arriving at the cathedral some few minutes before her escort, and
waiting there, she thought, what if the key were already taken in a
net, what if it were identified, what if the door were opened and the
remains discovered, what if she were stopped at the gate, sent to
prison, and charged with murder!  In the midst of these fluttering
thoughts, the escort appeared, took her in, and took her away.

``Is there any noise in the streets?'' she asked him.

``The usual noises,'' Mr.\ Cruncher replied; and looked surprised by the
question and by her aspect.

``I don't hear you,'' said Miss Pross.  ``What do you say?''

It was in vain for Mr.\ Cruncher to repeat what he said; Miss Pross
could not hear him.  ``So I'll nod my head,'' thought Mr.\ Cruncher,
amazed, ``at all events she'll see that.''  And she did.

``Is there any noise in the streets now?'' asked Miss Pross again,
presently.

Again Mr.\ Cruncher nodded his head.

``I don't hear it.''

``Gone deaf in an hour?'' said Mr.\ Cruncher, ruminating, with his mind
much disturbed; ``wot's come to her?''

``I feel,'' said Miss Pross, ``as if there had been a flash and a crash,
and that crash was the last thing I should ever hear in this life.''

``Blest if she ain't in a queer condition!'' said Mr.\ Cruncher, more
and more disturbed.  ``Wot can she have been a takin', to keep her
courage up?  Hark!  There's the roll of them dreadful carts!  You can
hear that, miss?''

``I can hear,'' said Miss Pross, seeing that he spoke to her,
``nothing. O, my good man, there was first a great crash, and then a
great stillness, and that stillness seems to be fixed and
unchangeable, never to be broken any more as long as my life lasts.''

``If she don't hear the roll of those dreadful carts, now very nigh
their journey's end,'' said Mr.\ Cruncher, glancing over his shoulder,
``it's my opinion that indeed she never will hear anything else in
this world.''

And indeed she never did.



\chapter{The Footsteps Die Out For Ever}


Along the Paris streets, the death-carts rumble, hollow and harsh.
Six tumbrils carry the day's wine to La Guillotine.  All the
devouring and insatiate Monsters imagined since imagination could
record itself, are fused in the one realisation, Guillotine.  And yet
there is not in France, with its rich variety of soil and climate,
a blade, a leaf, a root, a sprig, a peppercorn, which will grow to
maturity under conditions more certain than those that have produced
this horror.  Crush humanity out of shape once more, under similar
hammers, and it will twist itself into the same tortured forms.
Sow the same seed of rapacious license and oppression over again,
and it will surely yield the same fruit according to its kind.

Six tumbrils roll along the streets.  Change these back again to what
they were, thou powerful enchanter, Time, and they shall be seen to
be the carriages of absolute monarchs, the equipages of feudal nobles,
the toilettes of flaring Jezebels, the churches that are not my
father's house but dens of thieves, the huts of millions of starving
peasants!  No; the great magician who majestically works out the
appointed order of the Creator, never reverses his transformations.
``If thou be changed into this shape by the will of God,'' say the
seers to the enchanted, in the wise Arabian stories, ``then remain so!
But, if thou wear this form through mere passing conjuration, then resume
thy former aspect!''  Changeless and hopeless, the tumbrils roll along.

As the sombre wheels of the six carts go round, they seem to plough
up a long crooked furrow among the populace in the streets.  Ridges
of faces are thrown to this side and to that, and the ploughs go
steadily onward.  So used are the regular inhabitants of the houses
to the spectacle, that in many windows there are no people,
and in some the occupation of the hands is not so much as suspended,
while the eyes survey the faces in the tumbrils.  Here and there,
the inmate has visitors to see the sight; then he points his finger,
with something of the complacency of a curator or authorised exponent,
to this cart and to this, and seems to tell who sat here yesterday,
and who there the day before.

Of the riders in the tumbrils, some observe these things, and all
things on their last roadside, with an impassive stare; others, with
a lingering interest in the ways of life and men.  Some, seated with
drooping heads, are sunk in silent despair; again, there are some so
heedful of their looks that they cast upon the multitude such glances
as they have seen in theatres, and in pictures.  Several close their
eyes, and think, or try to get their straying thoughts together.
Only one, and he a miserable creature, of a crazed aspect, is so
shattered and made drunk by horror, that he sings, and tries to
dance.  Not one of the whole number appeals by look or gesture, to
the pity of the people.

There is a guard of sundry horsemen riding abreast of the tumbrils,
and faces are often turned up to some of them, and they are asked
some question.  It would seem to be always the same question, for,
it is always followed by a press of people towards the third cart.
The horsemen abreast of that cart, frequently point out one man in it
with their swords. The leading curiosity is, to know which is he;
he stands at the back of the tumbril with his head bent down,
to converse with a mere girl who sits on the side of the cart,
and holds his hand.  He has no curiosity or care for the scene about him,
and always speaks to the girl.  Here and there in the long street
of St. Honore, cries are raised against him.  If they move him at all,
it is only to a quiet smile, as he shakes his hair a little more
loosely about his face.  He cannot easily touch his face, his arms
being bound.

On the steps of a church, awaiting the coming-up of the tumbrils,
stands the Spy and prison-sheep.  He looks into the first of them:
not there.  He looks into the second:  not there.  He already asks
himself, ``Has he sacrificed me?'' when his face clears, as he looks
into the third.

``Which is Evremonde?'' says a man behind him.

``That.  At the back there.''

``With his hand in the girl's?''

``Yes.''

The man cries, ``Down, Evremonde!  To the Guillotine all aristocrats!
Down, Evremonde!''

``Hush, hush!'' the Spy entreats him, timidly.

``And why not, citizen?''

``He is going to pay the forfeit:  it will be paid in five minutes more.
Let him be at peace.''

But the man continuing to exclaim, ``Down, Evremonde!'' the face of
Evremonde is for a moment turned towards him.  Evremonde then sees
the Spy, and looks attentively at him, and goes his way.

The clocks are on the stroke of three, and the furrow ploughed among
the populace is turning round, to come on into the place of execution,
and end.  The ridges thrown to this side and to that, now crumble in
and close behind the last plough as it passes on, for all are following
to the Guillotine.  In front of it, seated in chairs, as in a garden
of public diversion, are a number of women, busily knitting.  On one
of the fore-most chairs, stands The Vengeance, looking about for her friend.

``Therese!'' she cries, in her shrill tones.  ``Who has seen her?
Therese Defarge!''

``She never missed before,'' says a knitting-woman of the sisterhood.

``No; nor will she miss now,'' cries The Vengeance, petulantly.  ``Therese.''

``Louder,'' the woman recommends.

Ay!  Louder, Vengeance, much louder, and still she will scarcely hear
thee.  Louder yet, Vengeance, with a little oath or so added, and yet
it will hardly bring her.  Send other women up and down to seek her,
lingering somewhere; and yet, although the messengers have done dread
deeds, it is questionable whether of their own wills they will go far
enough to find her!

``Bad Fortune!'' cries The Vengeance, stamping her foot in the chair,
``and here are the tumbrils!  And Evremonde will be despatched in a
wink, and she not here!  See her knitting in my hand, and her empty
chair ready for her.  I cry with vexation and disappointment!''

As The Vengeance descends from her elevation to do it, the tumbrils
begin to discharge their loads.  The ministers of Sainte Guillotine
are robed and ready.  Crash!---A head is held up, and the knitting-%
women who scarcely lifted their eyes to look at it a moment ago when
it could think and speak, count One.

The second tumbril empties and moves on; the third comes up.  Crash!%
---And the knitting-women, never faltering or pausing in their Work,
count Two.

The supposed Evremonde descends, and the seamstress is lifted out
next after him.  He has not relinquished her patient hand in getting
out, but still holds it as he promised.  He gently places her with
her back to the crashing engine that constantly whirrs up and falls,
and she looks into his face and thanks him.

``But for you, dear stranger, I should not be so composed, for I am
naturally a poor little thing, faint of heart; nor should I have been
able to raise my thoughts to Him who was put to death, that we might
have hope and comfort here to-day.  I think you were sent to me by Heaven.''

``Or you to me,'' says Sydney Carton.  ``Keep your eyes upon me, dear child,
and mind no other object.''

``I mind nothing while I hold your hand.  I shall mind nothing when
I let it go, if they are rapid.''

``They will be rapid.  Fear not!''

The two stand in the fast-thinning throng of victims, but they speak
as if they were alone.  Eye to eye, voice to voice, hand to hand,
heart to heart, these two children of the Universal Mother, else so
wide apart and differing, have come together on the dark highway,
to repair home together, and to rest in her bosom.

``Brave and generous friend, will you let me ask you one last
question?  I am very ignorant, and it troubles me---just a little.''

``Tell me what it is.''

``I have a cousin, an only relative and an orphan, like myself, whom I
love very dearly.  She is five years younger than I, and she lives in
a farmer's house in the south country.  Poverty parted us, and she
knows nothing of my fate---for I cannot write---and if I could, how
should I tell her!  It is better as it is.''

``Yes, yes:  better as it is.''

``What I have been thinking as we came along, and what I am still
thinking now, as I look into your kind strong face which gives me so
much support, is this:---If the Republic really does good to the poor,
and they come to be less hungry, and in all ways to suffer less, she
may live a long time:  she may even live to be old.''

``What then, my gentle sister?''

``Do you think:'' the uncomplaining eyes in which there is so much
endurance, fill with tears, and the lips part a little more and
tremble:  ``that it will seem long to me, while I wait for her in the
better land where I trust both you and I will be mercifully sheltered?''

``It cannot be, my child; there is no Time there, and no trouble
there.''

``You comfort me so much!  I am so ignorant.  Am I to kiss you now?
Is the moment come?''

``Yes.''

She kisses his lips; he kisses hers; they solemnly bless each other.
The spare hand does not tremble as he releases it; nothing worse than
a sweet, bright constancy is in the patient face.  She goes next
before him---is gone; the knitting-women count Twenty-Two.

``I am the Resurrection and the Life, saith the Lord:
he that believeth in me, though he were dead, yet shall he live:
and whosoever liveth and believeth in me shall never die.''

The murmuring of many voices, the upturning of many faces,
the pressing on of many footsteps in the outskirts of the crowd,
so that it swells forward in a mass, like one great heave of water,
all flashes away. Twenty-Three.



They said of him, about the city that night, that it was the
peacefullest man's face ever beheld there.  Many added that he looked
sublime and prophetic.

One of the most remarkable sufferers by the same axe---a woman---had
asked at the foot of the same scaffold, not long before, to be
allowed to write down the thoughts that were inspiring her.  If he
had given any utterance to his, and they were prophetic, they would
have been these:

``I see Barsad, and Cly, Defarge, The Vengeance, the Juryman, the
Judge, long ranks of the new oppressors who have risen on the
destruction of the old, perishing by this retributive instrument,
before it shall cease out of its present use.  I see a beautiful city
and a brilliant people rising from this abyss, and, in their struggles
to be truly free, in their triumphs and defeats, through long years
to come, I see the evil of this time and of the previous time of
which this is the natural birth, gradually making expiation for
itself and wearing out.

``I see the lives for which I lay down my life, peaceful, useful,
prosperous and happy, in that England which I shall see no more.
I see Her with a child upon her bosom, who bears my name.  I see her
father, aged and bent, but otherwise restored, and faithful to all
men in his healing office, and at peace.  I see the good old man, so
long their friend, in ten years' time enriching them with all he has,
and passing tranquilly to his reward.

``I see that I hold a sanctuary in their hearts, and in the hearts of
their descendants, generations hence.  I see her, an old woman,
weeping for me on the anniversary of this day.  I see her and her
husband, their course done, lying side by side in their last earthly
bed, and I know that each was not more honoured and held sacred in
the other's soul, than I was in the souls of both.

``I see that child who lay upon her bosom and who bore my name, a man
winning his way up in that path of life which once was mine.  I see
him winning it so well, that my name is made illustrious there by the
light of his.  I see the blots I threw upon it, faded away.  I see
him, fore-most of just judges and honoured men, bringing a boy of my
name, with a forehead that I know and golden hair, to this place---%
then fair to look upon, with not a trace of this day's disfigurement%
---and I hear him tell the child my story, with a tender and a faltering voice.

``It is a far, far better thing that I do, than I have ever done;
it is a far, far better rest that I go to than I have ever known.''

\end{document}

% End of The Project Gutenberg Etext of A Tale of Two Cities

