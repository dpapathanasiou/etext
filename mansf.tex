% Please take a look at the important information in this header.
% We encourage you to keep this file on your own disk, keeping an
% electronic path open for the next readers.  Do not remove this.
% 
% 
% **Welcome To The World of Free Plain Vanilla Electronic Texts**
% 
% **Etexts Readable By Both Humans and By Computers, Since 1971**
% 
% *These Etexts Prepared By Hundreds of Volunteers and Donations*
% 
% Information on contacting Project Gutenberg to get Etexts, and
% further information is included below.  We need your donations.
% 
% 
% Mansfield Park, by Jane Austen (1775-1817)
% 
% June, 1994  [Etext #141]
% 
% [Date last updated: March 28, 2002]
% 
% ***The Project Gutenberg Etext of Mansfield Park, by Austen***
% *****This file should be named mansf10.txt or mansf10.zip*****
% 
% Corrected EDITIONS of our etexts get a new NUMBER, mansf11.txt.
% VERSIONS based on separate sources get new LETTER, mansf10a.txt.
% 
% We are now trying to release all our books one month in advance
% of the official release dates, for time for better editing.  We
% have this as a goal to accomplish by the end of the year but we
% cannot guarantee to stay that far ahead every month after that.
% 
% Please note:  neither this list nor its contents are final till
% midnight of the last day of the month of any such announcement.
% The official release date of all Project Gutenberg Etexts is at
% Midnight, Central Time, of the last day of the stated month.  A
% preliminary version may often be posted for suggestion, comment
% and editing by those who wish to do so.  To be sure you have an
% up to date first edition [xxxxx10x.xxx] please check file sizes
% in the first week of the next month.  Since our ftp program has
% a bug in it that scrambles the date [tried to fix and failed] a
% look at the file size will have to do, but we will try to see a
% new copy has at least one byte more or less.
% 
% 
% Information about Project Gutenberg (one page)
% 
% We produce about two million dollars for each hour we work.  The
% fifty hours is one conservative estimate for how long it we take
% to get any etext selected, entered, proofread, edited, copyright
% searched and analyzed, the copyright letters written, etc.  This
% projected audience is one hundred million readers.  If our value
% per text is nominally estimated at one dollar then we produce $4
% million dollars per hour this year as we release some eight text
% files per month:  thus upping our productivity from $2 million.
% 
% The Goal of Project Gutenberg is to Give Away One Trillion Etext
% Files by the December 31, 2001.  [10,000 x 100,000,000=Trillion]
% This is ten thousand titles each to one hundred million readers,
% which is 10% of the expected number of computer users by the end
% of the year 2001.
% 
% We need your donations more than ever!
% 
% All donations should be made to "Project Gutenberg/IBC", and are
% tax deductible to the extent allowable by law ("IBC" is Illinois
% Benedictine College).  (Subscriptions to our paper newsletter go
% to IBC, too)
% 
% For these and other matters, please mail to:
% 
% Project Gutenberg
% P. O. Box  2782
% Champaign, IL 61825
% 
% When all other email fails try our Michael S. Hart, Executive
% Director:
% hart@vmd.cso.uiuc.edu (internet)   hart@uiucvmd   (bitnet)
% 
% We would prefer to send you this information by email
% (Internet, Bitnet, Compuserve, ATTMAIL or MCImail).
% 
% ******
% If you have an FTP program (or emulator), please
% FTP directly to the Project Gutenberg archives:
% [Mac users, do NOT point and click. . .type]
% 
% ftp mrcnext.cso.uiuc.edu
% login:  anonymous
% password:  your@login
% cd etext/etext91
% or cd etext92
% or cd etext93
% or cd etext94 [for new books]
% or cd etext/articles [get suggest gut for more information]
% dir [to see files]
% get or mget [to get files. . .set bin for zip files]
% GET 0INDEX.GUT
% for a list of books
% and
% GET NEW GUT for general information
% and
% MGET GUT* for newsletters.
% 
% **Information prepared by the Project Gutenberg legal advisor**
% (Three Pages)
% 
% 
% ***START**THE SMALL PRINT!**FOR PUBLIC DOMAIN ETEXTS**START***
% Why is this "Small Print!" statement here?  You know: lawyers.
% They tell us you might sue us if there is something wrong with
% your copy of this etext, even if you got it for free from
% someone other than us, and even if what's wrong is not our
% fault.  So, among other things, this "Small Print!" statement
% disclaims most of our liability to you.  It also tells you how
% you can distribute copies of this etext if you want to.
% 
% *BEFORE!* YOU USE OR READ THIS ETEXT
% By using or reading any part of this PROJECT GUTENBERG-tm
% etext, you indicate that you understand, agree to and accept
% this "Small Print!" statement.  If you do not, you can receive
% a refund of the money (if any) you paid for this etext by
% sending a request within 30 days of receiving it to the person
% you got it from.  If you received this etext on a physical
% medium (such as a disk), you must return it with your request.
% 
% ABOUT PROJECT GUTENBERG-TM ETEXTS
% This PROJECT GUTENBERG-tm etext, like most PROJECT GUTENBERG-
% tm etexts, is a "public domain" work distributed by Professor
% Michael S. Hart through the Project Gutenberg Association at
% Illinois Benedictine College (the "Project").  Among other
% things, this means that no one owns a United States copyright
% on or for this work, so the Project (and you!) can copy and
% distribute it in the United States without permission and
% without paying copyright royalties.  Special rules, set forth
% below, apply if you wish to copy and distribute this etext
% under the Project's "PROJECT GUTENBERG" trademark.
% 
% To create these etexts, the Project expends considerable
% efforts to identify, transcribe and proofread public domain
% works.  Despite these efforts, the Project's etexts and any
% medium they may be on may contain "Defects".  Among other
% things, Defects may take the form of incomplete, inaccurate or
% corrupt data, transcription errors, a copyright or other
% intellectual property infringement, a defective or damaged
% disk or other etext medium, a computer virus, or computer
% codes that damage or cannot be read by your equipment.
% 
% LIMITED WARRANTY; DISCLAIMER OF DAMAGES
% But for the "Right of Replacement or Refund" described below,
% [1] the Project (and any other party you may receive this
% etext from as a PROJECT GUTENBERG-tm etext) disclaims all
% liability to you for damages, costs and expenses, including
% legal fees, and [2] YOU HAVE NO REMEDIES FOR NEGLIGENCE OR
% UNDER STRICT LIABILITY, OR FOR BREACH OF WARRANTY OR CONTRACT,
% INCLUDING BUT NOT LIMITED TO INDIRECT, CONSEQUENTIAL, PUNITIVE
% OR INCIDENTAL DAMAGES, EVEN IF YOU GIVE NOTICE OF THE
% POSSIBILITY OF SUCH DAMAGES.
% 
% If you discover a Defect in this etext within 90 days of
% receiving it, you can receive a refund of the money (if any)
% you paid for it by sending an explanatory note within that
% time to the person you received it from.  If you received it
% on a physical medium, you must return it with your note, and
% such person may choose to alternatively give you a replacement
% copy.  If you received it electronically, such person may
% choose to alternatively give you a second opportunity to
% receive it electronically.
% 
% THIS ETEXT IS OTHERWISE PROVIDED TO YOU "AS-IS".  NO OTHER
% WARRANTIES OF ANY KIND, EXPRESS OR IMPLIED, ARE MADE TO YOU AS
% TO THE ETEXT OR ANY MEDIUM IT MAY BE ON, INCLUDING BUT NOT
% LIMITED TO WARRANTIES OF MERCHANTABILITY OR FITNESS FOR A
% PARTICULAR PURPOSE.
% 
% Some states do not allow disclaimers of implied warranties or
% the exclusion or limitation of consequential damages, so the
% above disclaimers and exclusions may not apply to you, and you
% may have other legal rights.
% 
% INDEMNITY
% You will indemnify and hold the Project, its directors,
% officers, members and agents harmless from all liability, cost
% and expense, including legal fees, that arise directly or
% indirectly from any of the following that you do or cause:
% [1] distribution of this etext, [2] alteration, modification,
% or addition to the etext, or [3] any Defect.
% 
% DISTRIBUTION UNDER "PROJECT GUTENBERG-tm"
% You may distribute copies of this etext electronically, or by
% disk, book or any other medium if you either delete this
% "Small Print!" and all other references to Project Gutenberg,
% or:
% 
% [1]  Only give exact copies of it.  Among other things, this
%      requires that you do not remove, alter or modify the
%      etext or this "small print!" statement.  You may however,
%      if you wish, distribute this etext in machine readable
%      binary, compressed, mark-up, or proprietary form,
%      including any form resulting from conversion by word pro-
%      cessing or hypertext software, but only so long as
%      *EITHER*:
% 
%      [*]  The etext, when displayed, is clearly readable, and
%           does *not* contain characters other than those
%           intended by the author of the work, although tilde
%           (~), asterisk (*) and underline (_) characters may
%           be used to convey punctuation intended by the
%           author, and additional characters may be used to
%           indicate hypertext links; OR
% 
%      [*]  The etext may be readily converted by the reader at
%           no expense into plain ASCII, EBCDIC or equivalent
%           form by the program that displays the etext (as is
%           the case, for instance, with most word processors);
%           OR
% 
%      [*]  You provide, or agree to also provide on request at
%           no additional cost, fee or expense, a copy of the
%           etext in its original plain ASCII form (or in EBCDIC
%           or other equivalent proprietary form).
% 
% [2]  Honor the etext refund and replacement provisions of this
%      "Small Print!" statement.
% 
% [3]  Pay a trademark license fee to the Project of 20% of the
%      net profits you derive calculated using the method you
%      already use to calculate your applicable taxes.  If you
%      don't derive profits, no royalty is due.  Royalties are
%      payable to "Project Gutenberg Association / Illinois
%      Benedictine College" within the 60 days following each
%      date you prepare (or were legally required to prepare)
%      your annual (or equivalent periodic) tax return.
% 
% WHAT IF YOU *WANT* TO SEND MONEY EVEN IF YOU DON'T HAVE TO?
% The Project gratefully accepts contributions in money, time,
% scanning machines, OCR software, public domain etexts, royalty
% free copyright licenses, and every other sort of contribution
% you can think of.  Money should be paid to "Project Gutenberg
% Association / Illinois Benedictine College".
% 
% This "Small Print!" by Charles B. Kramer, Attorney
% Internet (72600.2026@compuserve.com); TEL: (212-254-5093)
% *END*THE SMALL PRINT! FOR PUBLIC DOMAIN ETEXTS*Ver.04.29.93*END*

%
% converted to LaTeX by Peter Monta <pmonta@pmonta.com>
% June 2002
%

\input gutenberg-simple.tex

\begin{document}

\gtitle{Mansfield Park}
% (1814)

% by

\gauthor{Jane Austen}




\chapter{Chapter 1}

\gintro{About thirty years ago} Miss Maria Ward, of Huntingdon,
with only seven thousand pounds, had the good luck
to captivate Sir Thomas Bertram, of Mansfield Park,
in the county of Northampton, and to be thereby raised
to the rank of a baronet's lady, with all the comforts
and consequences of an handsome house and large income.
All Huntingdon exclaimed on the greatness of the match,
and her uncle, the lawyer, himself, allowed her to be at least
three thousand pounds short of any equitable claim to it.
She had two sisters to be benefited by her elevation;
and such of their acquaintance as thought Miss Ward and Miss
Frances quite as handsome as Miss Maria, did not scruple
to predict their marrying with almost equal advantage.
But there certainly are not so many men of large fortune
in the world as there are pretty women to deserve them.
Miss Ward, at the end of half a dozen years, found
herself obliged to be attached to the Rev.\ Mr.\ Norris,
a friend of her brother-in-law, with scarcely any
private fortune, and Miss Frances fared yet worse.
Miss Ward's match, indeed, when it came to the point,
was not contemptible:  Sir Thomas being happily able
to give his friend an income in the living of Mansfield;
and Mr.\ and Mrs.\ Norris began their career of conjugal
felicity with very little less than a thousand a year.
But Miss Frances married, in the common phrase,
to disoblige her family, and by fixing on a lieutenant
of marines, without education, fortune, or connexions,
did it very thoroughly.  She could hardly have made
a more untoward choice.  Sir Thomas Bertram had interest,
which, from principle as well as pride---from a general
wish of doing right, and a desire of seeing all that were
connected with him in situations of respectability,
he would have been glad to exert for the advantage
of Lady Bertram's sister; but her husband's profession
was such as no interest could reach; and before he
had time to devise any other method of assisting them,
an absolute breach between the sisters had taken place.
It was the natural result of the conduct of each party,
and such as a very imprudent marriage almost always produces.
To save herself from useless remonstrance, Mrs.\ Price never
wrote to her family on the subject till actually married.
Lady Bertram, who was a woman of very tranquil feelings,
and a temper remarkably easy and indolent, would have
contented herself with merely giving up her sister,
and thinking no more of the matter; but Mrs.\ Norris
had a spirit of activity, which could not be satisfied
till she had written a long and angry letter to Fanny,
to point out the folly of her conduct, and threaten
her with all its possible ill consequences.  Mrs.\ Price,
in her turn, was injured and angry; and an answer,
which comprehended each sister in its bitterness, and bestowed
such very disrespectful reflections on the pride of Sir
Thomas as Mrs.\ Norris could not possibly keep to herself,
put an end to all intercourse between them for a considerable
period.

Their homes were so distant, and the circles in which they
moved so distinct, as almost to preclude the means of ever
hearing of each other's existence during the eleven
following years, or, at least, to make it very wonderful
to Sir Thomas that Mrs.\ Norris should ever have it
in her power to tell them, as she now and then did,
in an angry voice, that Fanny had got another child.
By the end of eleven years, however, Mrs.\ Price could no
longer afford to cherish pride or resentment, or to lose one
connexion that might possibly assist her.  A large and still
increasing family, an husband disabled for active service,
but not the less equal to company and good liquor, and a
very small income to supply their wants, made her eager
to regain the friends she had so carelessly sacrificed;
and she addressed Lady Bertram in a letter which spoke
so much contrition and despondence, such a superfluity
of children, and such a want of almost everything else,
as could not but dispose them all to a reconciliation.
She was preparing for her ninth lying-in; and after
bewailing the circumstance, and imploring their countenance
as sponsors to the expected child, she could not conceal
how important she felt they might be to the future
maintenance of the eight already in being.  Her eldest
was a boy of ten years old, a fine spirited fellow,
who longed to be out in the world; but what could she do?
Was there any chance of his being hereafter useful to Sir
Thomas in the concerns of his West Indian property?
No situation would be beneath him; or what did Sir Thomas
think of Woolwich? or how could a boy be sent out to
the East?

The letter was not unproductive.  It re-established
peace and kindness.  Sir Thomas sent friendly
advice and professions, Lady Bertram dispatched
money and baby-linen, and Mrs.\ Norris wrote the letters.

Such were its immediate effects, and within a twelvemonth
a more important advantage to Mrs.\ Price resulted from it.
Mrs.\ Norris was often observing to the others that she
could not get her poor sister and her family out of
her head, and that, much as they had all done for her,
she seemed to be wanting to do more; and at length she
could not but own it to be her wish that poor Mrs.\ Price
should be relieved from the charge and expense of one child
entirely out of her great number.  ``What if they were
among them to undertake the care of her eldest daughter,
a girl now nine years old, of an age to require more
attention than her poor mother could possibly give?
The trouble and expense of it to them would be nothing,
compared with the benevolence of the action.''  Lady Bertram
agreed with her instantly.  ``I think we cannot do better,''
said she; ``let us send for the child.''

Sir Thomas could not give so instantaneous and unqualified
a consent.  He debated and hesitated;---it was a serious charge;---%
a girl so brought up must be adequately provided for,
or there would be cruelty instead of kindness in taking
her from her family.  He thought of his own four children,
of his two sons, of cousins in love, etc.;---but no sooner
had he deliberately begun to state his objections,
than Mrs.\ Norris interrupted him with a reply to them all,
whether stated or not.

``My dear Sir Thomas, I perfectly comprehend you, and do
justice to the generosity and delicacy of your notions,
which indeed are quite of a piece with your general conduct;
and I entirely agree with you in the main as to the propriety
of doing everything one could by way of providing for a
child one had in a manner taken into one's own hands;
and I am sure I should be the last person in the world to
withhold my mite upon such an occasion.  Having no children
of my own, who should I look to in any little matter I
may ever have to bestow, but the children of my sisters?---%
and I am sure Mr.\ Norris is too just---but you know I am
a woman of few words and professions.  Do not let us
be frightened from a good deed by a trifle.  Give a girl
an education, and introduce her properly into the world,
and ten to one but she has the means of settling well,
without farther expense to anybody.  A niece of ours,
Sir Thomas, I may say, or at least of \emph{yours}, would not
grow up in this neighbourhood without many advantages.
I don't say she would be so handsome as her cousins.
I dare say she would not; but she would be introduced into
the society of this country under such very favourable
circumstances as, in all human probability, would get her
a creditable establishment.  You are thinking of your sons---%
but do not you know that, of all things upon earth,
\emph{that} is the least likely to happen, brought up as they
would be, always together like brothers and sisters?
It is morally impossible.  I never knew an instance of it.
It is, in fact, the only sure way of providing against
the connexion.  Suppose her a pretty girl, and seen by Tom
or Edmund for the first time seven years hence, and I dare
say there would be mischief.  The very idea of her having
been suffered to grow up at a distance from us all in poverty
and neglect, would be enough to make either of the dear,
sweet-tempered boys in love with her.  But breed her up
with them from this time, and suppose her even to have the
beauty of an angel, and she will never be more to either than
a sister.''

``There is a great deal of truth in what you say,''
replied Sir Thomas, ``and far be it from me to throw any
fanciful impediment in the way of a plan which would be
so consistent with the relative situations of each.  I only
meant to observe that it ought not to be lightly engaged in,
and that to make it really serviceable to Mrs.\ Price,
and creditable to ourselves, we must secure to the child,
or consider ourselves engaged to secure to her hereafter,
as circumstances may arise, the provision of a gentlewoman,
if no such establishment should offer as you are so sanguine
in expecting.''

``I thoroughly understand you,'' cried Mrs.\ Norris,
``you are everything that is generous and considerate,
and I am sure we shall never disagree on this point.
Whatever I can do, as you well know, I am always ready
enough to do for the good of those I love; and, though I
could never feel for this little girl the hundredth
part of the regard I bear your own dear children,
nor consider her, in any respect, so much my own,
I should hate myself if I were capable of neglecting her.
Is not she a sister's child? and could I bear to see
her want while I had a bit of bread to give her?
My dear Sir Thomas, with all my faults I have a warm heart;
and, poor as I am, would rather deny myself the necessaries
of life than do an ungenerous thing.  So, if you are not
against it, I will write to my poor sister tomorrow,
and make the proposal; and, as soon as matters are settled,
\emph{I} will engage to get the child to Mansfield; \emph{you} shall
have no trouble about it.  My own trouble, you know,
I never regard.  I will send Nanny to London on purpose,
and she may have a bed at her cousin the saddler's, and the
child be appointed to meet her there.  They may easily get
her from Portsmouth to town by the coach, under the care
of any creditable person that may chance to be going.
I dare say there is always some reputable tradesman's wife
or other going up.''

Except to the attack on Nanny's cousin, Sir Thomas no longer
made any objection, and a more respectable, though less
economical rendezvous being accordingly substituted,
everything was considered as settled, and the pleasures
of so benevolent a scheme were already enjoyed.
The division of gratifying sensations ought not,
in strict justice, to have been equal; for Sir Thomas was
fully resolved to be the real and consistent patron of the
selected child, and Mrs.\ Norris had not the least intention
of being at any expense whatever in her maintenance.
As far as walking, talking, and contriving reached,
she was thoroughly benevolent, and nobody knew better
how to dictate liberality to others; but her love of money
was equal to her love of directing, and she knew quite as
well how to save her own as to spend that of her friends.
Having married on a narrower income than she had been
used to look forward to, she had, from the first,
fancied a very strict line of economy necessary;
and what was begun as a matter of prudence, soon grew
into a matter of choice, as an object of that needful
solicitude which there were no children to supply.
Had there been a family to provide for, Mrs.\ Norris might
never have saved her money; but having no care of that kind,
there was nothing to impede her frugality, or lessen the
comfort of making a yearly addition to an income which they
had never lived up to.  Under this infatuating principle,
counteracted by no real affection for her sister,
it was impossible for her to aim at more than the credit
of projecting and arranging so expensive a charity;
though perhaps she might so little know herself as to
walk home to the Parsonage, after this conversation,
in the happy belief of being the most liberal-minded
sister and aunt in the world.

When the subject was brought forward again, her views
were more fully explained; and, in reply to Lady Bertram's
calm inquiry of ``Where shall the child come to first,
sister, to you or to us?''  Sir Thomas heard with some
surprise that it would be totally out of Mrs.\ Norris's
power to take any share in the personal charge of her.
He had been considering her as a particularly welcome
addition at the Parsonage, as a desirable companion
to an aunt who had no children of her own; but he found
himself wholly mistaken.  Mrs.\ Norris was sorry to say
that the little girl's staying with them, at least
as things then were, was quite out of the question.
Poor Mr.\ Norris's indifferent state of health made it
an impossibility:  he could no more bear the noise of a child
than he could fly; if, indeed, he should ever get well
of his gouty complaints, it would be a different matter:
she should then be glad to take her turn, and think nothing
of the inconvenience; but just now, poor Mr.\ Norris
took up every moment of her time, and the very mention
of such a thing she was sure would distract him.

``Then she had better come to us,'' said Lady Bertram,
with the utmost composure.  After a short pause Sir Thomas
added with dignity, ``Yes, let her home be in this house.
We will endeavour to do our duty by her, and she will,
at least, have the advantage of companions of her own age,
and of a regular instructress.''

``Very true,'' cried Mrs.\ Norris, ``which are both very
important considerations; and it will be just the same
to Miss Lee whether she has three girls to teach,
or only two---there can be no difference.  I only wish I
could be more useful; but you see I do all in my power.
I am not one of those that spare their own trouble;
and Nanny shall fetch her, however it may put me
to inconvenience to have my chief counsellor away for
three days.  I suppose, sister, you will put the child
in the little white attic, near the old nurseries.
It will be much the best place for her, so near Miss Lee,
and not far from the girls, and close by the housemaids,
who could either of them help to dress her, you know,
and take care of her clothes, for I suppose you would not
think it fair to expect Ellis to wait on her as well as
the others.  Indeed, I do not see that you could possibly
place her anywhere else.''

Lady Bertram made no opposition.

``I hope she will prove a well-disposed girl,''
continued Mrs.\ Norris, ``and be sensible of her uncommon
good fortune in having such friends.''

``Should her disposition be really bad,'' said Sir Thomas,
``we must not, for our own children's sake, continue her
in the family; but there is no reason to expect so great
an evil.  We shall probably see much to wish altered
in her, and must prepare ourselves for gross ignorance,
some meanness of opinions, and very distressing vulgarity
of manner; but these are not incurable faults; nor, I trust,
can they be dangerous for her associates.  Had my daughters
been \emph{younger} than herself, I should have considered
the introduction of such a companion as a matter of very
serious moment; but, as it is, I hope there can be nothing
to fear for \emph{them}, and everything to hope for \emph{her},
from the association.''

``That is exactly what I think,'' cried Mrs.\ Norris,
``and what I was saying to my husband this morning.
It will be an education for the child, said I, only being
with her cousins; if Miss Lee taught her nothing, she would
learn to be good and clever from \emph{them}.''

``I hope she will not tease my poor pug,'' said Lady Bertram;
``I have but just got Julia to leave it alone.''

``There will be some difficulty in our way, Mrs.\ Norris,''
observed Sir Thomas, ``as to the distinction proper to be made
between the girls as they grow up:  how to preserve in the
minds of my \emph{daughters} the consciousness of what they are,
without making them think too lowly of their cousin;
and how, without depressing her spirits too far,
to make her remember that she is not a \emph{Miss Bertram}.
I should wish to see them very good friends, and would,
on no account, authorise in my girls the smallest degree
of arrogance towards their relation; but still they cannot
be equals.  Their rank, fortune, rights, and expectations
will always be different.  It is a point of great delicacy,
and you must assist us in our endeavours to choose exactly
the right line of conduct.''

Mrs.\ Norris was quite at his service; and though she
perfectly agreed with him as to its being a most
difficult thing, encouraged him to hope that between
them it would be easily managed.

It will be readily believed that Mrs.\ Norris did not write
to her sister in vain.  Mrs.\ Price seemed rather surprised
that a girl should be fixed on, when she had so many fine boys,
but accepted the offer most thankfully, assuring them of her
daughter's being a very well-disposed, good-humoured girl,
and trusting they would never have cause to throw her off.
She spoke of her farther as somewhat delicate and puny,
but was sanguine in the hope of her being materially better
for change of air.  Poor woman! she probably thought
change of air might agree with many of her children.



\chapter{Chapter 2}

\gintro{The little girl} performed her long journey in safety;
and at Northampton was met by Mrs.\ Norris, who thus
regaled in the credit of being foremost to welcome her,
and in the importance of leading her in to the others,
and recommending her to their kindness.

Fanny Price was at this time just ten years old,
and though there might not be much in her first appearance
to captivate, there was, at least, nothing to disgust
her relations.  She was small of her age, with no
glow of complexion, nor any other striking beauty;
exceedingly timid and shy, and shrinking from notice;
but her air, though awkward, was not vulgar, her voice
was sweet, and when she spoke her countenance was pretty.
Sir Thomas and Lady Bertram received her very kindly;
and Sir Thomas, seeing how much she needed encouragement,
tried to be all that was conciliating:  but he had
to work against a most untoward gravity of deportment;
and Lady Bertram, without taking half so much trouble,
or speaking one word where he spoke ten, by the mere aid
of a good-humoured smile, became immediately the less awful
character of the two.

The young people were all at home, and sustained their
share in the introduction very well, with much good humour,
and no embarrassment, at least on the part of the sons, who,
at seventeen and sixteen, and tall of their age, had all
the grandeur of men in the eyes of their little cousin.
The two girls were more at a loss from being younger
and in greater awe of their father, who addressed them
on the occasion with rather an injudicious particularity.
But they were too much used to company and praise to have
anything like natural shyness; and their confidence
increasing from their cousin's total want of it,
they were soon able to take a full survey of her face
and her frock in easy indifference.

They were a remarkably fine family, the sons very well-looking,
the daughters decidedly handsome, and all of them well-grown
and forward of their age, which produced as striking
a difference between the cousins in person, as education
had given to their address; and no one would have supposed
the girls so nearly of an age as they really were.  There were
in fact but two years between the youngest and Fanny.
Julia Bertram was only twelve, and Maria but a year older.
The little visitor meanwhile was as unhappy as possible.
Afraid of everybody, ashamed of herself, and longing
for the home she had left, she knew not how to look up,
and could scarcely speak to be heard, or without crying.
Mrs.\ Norris had been talking to her the whole way from
Northampton of her wonderful good fortune, and the
extraordinary degree of gratitude and good behaviour
which it ought to produce, and her consciousness of
misery was therefore increased by the idea of its being
a wicked thing for her not to be happy.  The fatigue,
too, of so long a journey, became soon no trifling evil.
In vain were the well-meant condescensions of Sir Thomas,
and all the officious prognostications of Mrs.\ Norris
that she would be a good girl; in vain did Lady Bertram
smile and make her sit on the sofa with herself and pug,
and vain was even the sight of a gooseberry tart towards
giving her comfort; she could scarcely swallow two mouthfuls
before tears interrupted her, and sleep seeming to be her
likeliest friend, she was taken to finish her sorrows in bed.

``This is not a very promising beginning,'' said Mrs.\ Norris,
when Fanny had left the room.  ``After all that I said to her
as we came along, I thought she would have behaved better;
I told her how much might depend upon her acquitting
herself well at first.  I wish there may not be a little
sulkiness of temper---her poor mother had a good deal;
but we must make allowances for such a child---and I
do not know that her being sorry to leave her home is
really against her, for, with all its faults, it \emph{was}
her home, and she cannot as yet understand how much she
has changed for the better; but then there is moderation
in all things.''

It required a longer time, however, than Mrs.\ Norris
was inclined to allow, to reconcile Fanny to the novelty
of Mansfield Park, and the separation from everybody
she had been used to.  Her feelings were very acute,
and too little understood to be properly attended to.
Nobody meant to be unkind, but nobody put themselves out
of their way to secure her comfort.

The holiday allowed to the Miss Bertrams the next day,
on purpose to afford leisure for getting acquainted with,
and entertaining their young cousin, produced little union.
They could not but hold her cheap on finding that she
had but two sashes, and had never learned French; and when
they perceived her to be little struck with the duet they
were so good as to play, they could do no more than make
her a generous present of some of their least valued toys,
and leave her to herself, while they adjourned to whatever
might be the favourite holiday sport of the moment,
making artificial flowers or wasting gold paper.

Fanny, whether near or from her cousins, whether in
the schoolroom, the drawing-room, or the shrubbery,
was equally forlorn, finding something to fear in
every person and place.  She was disheartened by Lady
Bertram's silence, awed by Sir Thomas's grave looks,
and quite overcome by Mrs.\ Norris's admonitions.
Her elder cousins mortified her by reflections on her size,
and abashed her by noticing her shyness:  Miss Lee
wondered at her ignorance, and the maid-servants sneered
at her clothes; and when to these sorrows was added the idea
of the brothers and sisters among whom she had always
been important as playfellow, instructress, and nurse,
the despondence that sunk her little heart was severe.

The grandeur of the house astonished, but could not console her.
The rooms were too large for her to move in with ease:
whatever she touched she expected to injure, and she
crept about in constant terror of something or other;
often retreating towards her own chamber to cry;
and the little girl who was spoken of in the drawing-room
when she left it at night as seeming so desirably sensible
of her peculiar good fortune, ended every day's sorrows
by sobbing herself to sleep.  A week had passed in this way,
and no suspicion of it conveyed by her quiet passive manner,
when she was found one morning by her cousin Edmund,
the youngest of the sons, sitting crying on the attic stairs.

``My dear little cousin,'' said he, with all the gentleness
of an excellent nature, ``what can be the matter?''  And sitting
down by her, he was at great pains to overcome her shame
in being so surprised, and persuade her to speak openly.
Was she ill? or was anybody angry with her? or had she
quarrelled with Maria and Julia? or was she puzzled
about anything in her lesson that he could explain?
Did she, in short, want anything he could possibly get her,
or do for her?  For a long while no answer could be
obtained beyond a ``no, no---not at all---no, thank you'';
but he still persevered; and no sooner had he begun to
revert to her own home, than her increased sobs explained
to him where the grievance lay.  He tried to console her.

``You are sorry to leave Mama, my dear little Fanny,''
said he, ``which shows you to be a very good girl; but you
must remember that you are with relations and friends,
who all love you, and wish to make you happy.  Let us walk
out in the park, and you shall tell me all about your
brothers and sisters.''

On pursuing the subject, he found that, dear as all
these brothers and sisters generally were, there was one
among them who ran more in her thoughts than the rest.
It was William whom she talked of most, and wanted most
to see.  William, the eldest, a year older than herself,
her constant companion and friend; her advocate with her
mother (of whom he was the darling) in every distress.
``William did not like she should come away; he had told
her he should miss her very much indeed.''  ``But William will
write to you, I dare say.''  ``Yes, he had promised he would,
but he had told \emph{her} to write first.''  ``And when shall
you do it?''  She hung her head and answered hesitatingly,
``she did not know; she had not any paper.''

``If that be all your difficulty, I will furnish you
with paper and every other material, and you may write
your letter whenever you choose.  Would it make you
happy to write to William?''

``Yes, very.''

``Then let it be done now.  Come with me into the
breakfast-room, we shall find everything there,
and be sure of having the room to ourselves.''

``But, cousin, will it go to the post?''

``Yes, depend upon me it shall:  it shall go with the
other letters; and, as your uncle will frank it,
it will cost William nothing.''

``My uncle!'' repeated Fanny, with a frightened look.

``Yes, when you have written the letter, I will take it
to my father to frank.''

Fanny thought it a bold measure, but offered no further
resistance; and they went together into the breakfast-room,
where Edmund prepared her paper, and ruled her lines
with all the goodwill that her brother could himself
have felt, and probably with somewhat more exactness.
He continued with her the whole time of her writing,
to assist her with his penknife or his orthography,
as either were wanted; and added to these attentions,
which she felt very much, a kindness to her brother which
delighted her beyond all the rest.  He wrote with his own
hand his love to his cousin William, and sent him half
a guinea under the seal.  Fanny's feelings on the occasion
were such as she believed herself incapable of expressing;
but her countenance and a few artless words fully
conveyed all their gratitude and delight, and her cousin
began to find her an interesting object.  He talked
to her more, and, from all that she said, was convinced
of her having an affectionate heart, and a strong desire
of doing right; and he could perceive her to be farther
entitled to attention by great sensibility of her situation,
and great timidity.  He had never knowingly given her pain,
but he now felt that she required more positive kindness;
and with that view endeavoured, in the first place,
to lessen her fears of them all, and gave her especially
a great deal of good advice as to playing with Maria
and Julia, and being as merry as possible.

From this day Fanny grew more comfortable.  She felt
that she had a friend, and the kindness of her cousin
Edmund gave her better spirits with everybody else.
The place became less strange, and the people less formidable;
and if there were some amongst them whom she could not
cease to fear, she began at least to know their ways,
and to catch the best manner of conforming to them.
The little rusticities and awkwardnesses which had at
first made grievous inroads on the tranquillity of all,
and not least of herself, necessarily wore away, and she
was no longer materially afraid to appear before her uncle,
nor did her aunt Norris's voice make her start very much.
To her cousins she became occasionally an acceptable companion.
Though unworthy, from inferiority of age and strength,
to be their constant associate, their pleasures and schemes
were sometimes of a nature to make a third very useful,
especially when that third was of an obliging,
yielding temper; and they could not but own, when their
aunt inquired into her faults, or their brother Edmund
urged her claims to their kindness, that ``Fanny was
good-natured enough.''

Edmund was uniformly kind himself; and she had nothing
worse to endure on the part of Tom than that sort
of merriment which a young man of seventeen will always
think fair with a child of ten.  He was just entering
into life, full of spirits, and with all the liberal
dispositions of an eldest son, who feels born only
for expense and enjoyment.  His kindness to his little
cousin was consistent with his situation and rights:
he made her some very pretty presents, and laughed at her.

As her appearance and spirits improved, Sir Thomas and Mrs.\ Norris
thought with greater satisfaction of their benevolent plan;
and it was pretty soon decided between them that,
though far from clever, she showed a tractable disposition,
and seemed likely to give them little trouble.  A mean
opinion of her abilities was not confined to \emph{them}.
Fanny could read, work, and write, but she had been taught
nothing more; and as her cousins found her ignorant
of many things with which they had been long familiar,
they thought her prodigiously stupid, and for the first
two or three weeks were continually bringing some fresh
report of it into the drawing-room. ``Dear mama, only think,
my cousin cannot put the map of Europe together---%
or my cousin cannot tell the principal rivers in Russia---%
or, she never heard of Asia Minor---or she does not know
the difference between water-colours and crayons!---%
How strange!---Did you ever hear anything so stupid?''

``My dear,'' their considerate aunt would reply,
``it is very bad, but you must not expect everybody
to be as forward and quick at learning as yourself.''

``But, aunt, she is really so very ignorant!---Do you know,
we asked her last night which way she would go to get
to Ireland; and she said, she should cross to the Isle
of Wight.  She thinks of nothing but the Isle of Wight,
and she calls it \emph{the Island}, as if there were no
other island in the world.  I am sure I should have been
ashamed of myself, if I had not known better long before I
was so old as she is.  I cannot remember the time when I
did not know a great deal that she has not the least
notion of yet.  How long ago it is, aunt, since we used
to repeat the chronological order of the kings of England,
with the dates of their accession, and most of the principal
events of their reigns!''

``Yes,'' added the other; ``and of the Roman emperors
as low as Severus; besides a great deal of the heathen
mythology, and all the metals, semi-metals, planets,
and distinguished philosophers.''

``Very true indeed, my dears, but you are blessed with
wonderful memories, and your poor cousin has probably none
at all.  There is a vast deal of difference in memories,
as well as in everything else, and therefore you must
make allowance for your cousin, and pity her deficiency.
And remember that, if you are ever so forward and clever
yourselves, you should always be modest; for, much as you
know already, there is a great deal more for you to learn.''

``Yes, I know there is, till I am seventeen.  But I must
tell you another thing of Fanny, so odd and so stupid.
Do you know, she says she does not want to learn either
music or drawing.''

``To be sure, my dear, that is very stupid indeed,
and shows a great want of genius and emulation.
But, all things considered, I do not know whether it is
not as well that it should be so, for, though you know
(owing to me) your papa and mama are so good as to bring
her up with you, it is not at all necessary that she
should be as accomplished as you are;---on the contrary,
it is much more desirable that there should be a difference.''

Such were the counsels by which Mrs.\ Norris assisted to form
her nieces' minds; and it is not very wonderful that,
with all their promising talents and early information,
they should be entirely deficient in the less common
acquirements of self-knowledge, generosity and humility.
In everything but disposition they were admirably taught.
Sir Thomas did not know what was wanting, because, though a
truly anxious father, he was not outwardly affectionate,
and the reserve of his manner repressed all the flow of their
spirits before him.

To the education of her daughters Lady Bertram paid not
the smallest attention.  She had not time for such cares.
She was a woman who spent her days in sitting, nicely dressed,
on a sofa, doing some long piece of needlework, of little use
and no beauty, thinking more of her pug than her children,
but very indulgent to the latter when it did not put
herself to inconvenience, guided in everything important
by Sir Thomas, and in smaller concerns by her sister.
Had she possessed greater leisure for the service of her girls,
she would probably have supposed it unnecessary, for they
were under the care of a governess, with proper masters,
and could want nothing more.  As for Fanny's being stupid
at learning, ``she could only say it was very unlucky,
but some people \emph{were} stupid, and Fanny must take more pains:
she did not know what else was to be done; and, except her
being so dull, she must add she saw no harm in the poor
little thing, and always found her very handy and quick
in carrying messages, and fetching, what she wanted.''

Fanny, with all her faults of ignorance and timidity,
was fixed at Mansfield Park, and learning to transfer
in its favour much of her attachment to her former home,
grew up there not unhappily among her cousins.  There was
no positive ill-nature in Maria or Julia; and though
Fanny was often mortified by their treatment of her,
she thought too lowly of her own claims to feel injured
by it.

From about the time of her entering the family,
Lady Bertram, in consequence of a little ill-health,
and a great deal of indolence, gave up the house in town,
which she had been used to occupy every spring,
and remained wholly in the country, leaving Sir Thomas
to attend his duty in Parliament, with whatever increase
or diminution of comfort might arise from her absence.
In the country, therefore, the Miss Bertrams continued
to exercise their memories, practise their duets, and grow
tall and womanly:  and their father saw them becoming
in person, manner, and accomplishments, everything that
could satisfy his anxiety.  His eldest son was careless
and extravagant, and had already given him much uneasiness;
but his other children promised him nothing but good.
His daughters, he felt, while they retained the name
of Bertram, must be giving it new grace, and in quitting it,
he trusted, would extend its respectable alliances;
and the character of Edmund, his strong good sense
and uprightness of mind, bid most fairly for utility,
honour, and happiness to himself and all his connexions.
He was to be a clergyman.

Amid the cares and the complacency which his own
children suggested, Sir Thomas did not forget to do what
he could for the children of Mrs.\ Price:  he assisted
her liberally in the education and disposal of her sons
as they became old enough for a determinate pursuit;
and Fanny, though almost totally separated from her family,
was sensible of the truest satisfaction in hearing of any
kindness towards them, or of anything at all promising
in their situation or conduct.  Once, and once only,
in the course of many years, had she the happiness
of being with William.  Of the rest she saw nothing:
nobody seemed to think of her ever going amongst them again,
even for a visit, nobody at home seemed to want her;
but William determining, soon after her removal,
to be a sailor, was invited to spend a week with his
sister in Northamptonshire before he went to sea.
Their eager affection in meeting, their exquisite
delight in being together, their hours of happy mirth,
and moments of serious conference, may be imagined;
as well as the sanguine views and spirits of the boy even
to the last, and the misery of the girl when he left her.
Luckily the visit happened in the Christmas holidays,
when she could directly look for comfort to her cousin Edmund;
and he told her such charming things of what William was
to do, and be hereafter, in consequence of his profession,
as made her gradually admit that the separation might
have some use.  Edmund's friendship never failed her:
his leaving Eton for Oxford made no change in his kind
dispositions, and only afforded more frequent opportunities
of proving them.  Without any display of doing more than
the rest, or any fear of doing too much, he was always
true to her interests, and considerate of her feelings,
trying to make her good qualities understood, and to conquer
the diffidence which prevented their being more apparent;
giving her advice, consolation, and encouragement.

Kept back as she was by everybody else, his single support
could not bring her forward; but his attentions were otherwise
of the highest importance in assisting the improvement
of her mind, and extending its pleasures.  He knew her to
be clever, to have a quick apprehension as well as good sense,
and a fondness for reading, which, properly directed,
must be an education in itself.  Miss Lee taught her French,
and heard her read the daily portion of history; but he
recommended the books which charmed her leisure hours,
he encouraged her taste, and corrected her judgment:
he made reading useful by talking to her of what she read,
and heightened its attraction by judicious praise.
In return for such services she loved him better than
anybody in the world except William:  her heart was divided
between the two.



\chapter{Chapter 3}

\gintro{The first event} of any importance in the family was
the death of Mr.\ Norris, which happened when Fanny was
about fifteen, and necessarily introduced alterations
and novelties.  Mrs.\ Norris, on quitting the Parsonage,
removed first to the Park, and afterwards to a small house
of Sir Thomas's in the village, and consoled herself
for the loss of her husband by considering that she
could do very well without him; and for her reduction
of income by the evident necessity of stricter economy.

The living was hereafter for Edmund; and, had his uncle
died a few years sooner, it would have been duly given
to some friend to hold till he were old enough for orders.
But Tom's extravagance had, previous to that event,
been so great as to render a different disposal of the
next presentation necessary, and the younger brother
must help to pay for the pleasures of the elder.
There was another family living actually held for Edmund;
but though this circumstance had made the arrangement
somewhat easier to Sir Thomas's conscience, he could not
but feel it to be an act of injustice, and he earnestly
tried to impress his eldest son with the same conviction,
in the hope of its producing a better effect than anything he
had yet been able to say or do.

``I blush for you, Tom,'' said he, in his most dignified manner;
``I blush for the expedient which I am driven on, and I trust
I may pity your feelings as a brother on the occasion.
You have robbed Edmund for ten, twenty, thirty years,
perhaps for life, of more than half the income which ought
to be his.  It may hereafter be in my power, or in yours
(I hope it will), to procure him better preferment;
but it must not be forgotten that no benefit of that
sort would have been beyond his natural claims on us,
and that nothing can, in fact, be an equivalent for the
certain advantage which he is now obliged to forego
through the urgency of your debts.''

Tom listened with some shame and some sorrow;
but escaping as quickly as possible, could soon with
cheerful selfishness reflect, firstly, that he had
not been half so much in debt as some of his friends;
secondly, that his father had made a most tiresome piece
of work of it; and, thirdly, that the future incumbent,
whoever he might be, would, in all probability, die very soon.

On Mr.\ Norris's death the presentation became the right of
a Dr.\ Grant, who came consequently to reside at Mansfield;
and on proving to be a hearty man of forty-five, seemed
likely to disappoint Mr.\ Bertram's calculations.
But ``no, he was a short-necked, apoplectic sort of fellow,
and, plied well with good things, would soon pop off.''

He had a wife about fifteen years his junior, but no children;
and they entered the neighbourhood with the usual fair
report of being very respectable, agreeable people.

The time was now come when Sir Thomas expected his
sister-in-law to claim her share in their niece,
the change in Mrs.\ Norris's situation, and the improvement
in Fanny's age, seeming not merely to do away any former
objection to their living together, but even to give it
the most decided eligibility; and as his own circumstances
were rendered less fair than heretofore, by some recent
losses on his West India estate, in addition to his eldest
son's extravagance, it became not undesirable to himself to be
relieved from the expense of her support, and the obligation
of her future provision.  In the fullness of his belief
that such a thing must be, he mentioned its probability
to his wife; and the first time of the subject's occurring
to her again happening to be when Fanny was present,
she calmly observed to her, ``So, Fanny, you are going
to leave us, and live with my sister.  How shall you like it?''

Fanny was too much surprised to do more than repeat
her aunt's words, ``Going to leave you?''

``Yes, my dear; why should you be astonished?
You have been five years with us, and my sister
always meant to take you when Mr.\ Norris died.
But you must come up and tack on my patterns all the same.''

The news was as disagreeable to Fanny as it had been unexpected.
She had never received kindness from her aunt Norris,
and could not love her.

``I shall be very sorry to go away,'' said she, with a
faltering voice.

``Yes, I dare say you will; \emph{that's} natural enough.
I suppose you have had as little to vex you since you came
into this house as any creature in the world.''

``I hope I am not ungrateful, aunt,'' said Fanny modestly.

``No, my dear; I hope not.  I have always found you
a very good girl.''

``And am I never to live here again?''

``Never, my dear; but you are sure of a comfortable home.
It can make very little difference to you, whether you are
in one house or the other.''

Fanny left the room with a very sorrowful heart; she could
not feel the difference to be so small, she could not think
of living with her aunt with anything like satisfaction.
As soon as she met with Edmund she told him her distress.

``Cousin,'' said she, ``something is going to happen which I
do not like at all; and though you have often persuaded me
into being reconciled to things that I disliked at first,
you will not be able to do it now.  I am going to live
entirely with my aunt Norris.''

``Indeed!''

``Yes; my aunt Bertram has just told me so.  It is quite settled.
I am to leave Mansfield Park, and go to the White House,
I suppose, as soon as she is removed there.''

``Well, Fanny, and if the plan were not unpleasant to you,
I should call it an excellent one.''

``Oh, cousin!''

``It has everything else in its favour.  My aunt is
acting like a sensible woman in wishing for you.  She is
choosing a friend and companion exactly where she ought,
and I am glad her love of money does not interfere.
You will be what you ought to be to her.  I hope it does
not distress you very much, Fanny?''

``Indeed it does:  I cannot like it.  I love this house
and everything in it:  I shall love nothing there.
You know how uncomfortable I feel with her.''

``I can say nothing for her manner to you as a child;
but it was the same with us all, or nearly so.  She never
knew how to be pleasant to children.  But you are now
of an age to be treated better; I think she is behaving
better already; and when you are her only companion,
you \emph{must} be important to her.''

``I can never be important to any one.''

``What is to prevent you?''

``Everything.  My situation, my foolishness and awkwardness.''

``As to your foolishness and awkwardness, my dear Fanny,
believe me, you never have a shadow of either, but in using
the words so improperly.  There is no reason in the world
why you should not be important where you are known.
You have good sense, and a sweet temper, and I am sure you
have a grateful heart, that could never receive kindness
without wishing to return it.  I do not know any better
qualifications for a friend and companion.''

``You are too kind,'' said Fanny, colouring at such praise;
``how shall I ever thank you as I ought, for thinking
so well of me.  Oh! cousin, if I am to go away, I shall
remember your goodness to the last moment of my life.''

``Why, indeed, Fanny, I should hope to be remembered at
such a distance as the White House.  You speak as if you
were going two hundred miles off instead of only across
the park; but you will belong to us almost as much as ever.
The two families will be meeting every day in the year.
The only difference will be that, living with your aunt,
you will necessarily be brought forward as you ought to be.
\emph{Here} there are too many whom you can hide behind; but with
\emph{her} you will be forced to speak for yourself.''

``Oh!  I do not say so.''

``I must say it, and say it with pleasure.  Mrs.\ Norris
is much better fitted than my mother for having the charge
of you now.  She is of a temper to do a great deal
for anybody she really interests herself about, and she
will force you to do justice to your natural powers.''

Fanny sighed, and said, ``I cannot see things as you do;
but I ought to believe you to be right rather than myself,
and I am very much obliged to you for trying to reconcile
me to what must be.  If I could suppose my aunt really
to care for me, it would be delightful to feel myself
of consequence to anybody.  \emph{Here}, I know, I am of none,
and yet I love the place so well.''

``The place, Fanny, is what you will not quit, though you
quit the house.  You will have as free a command of the
park and gardens as ever.  Even \emph{your} constant little
heart need not take fright at such a nominal change.
You will have the same walks to frequent, the same library
to choose from, the same people to look at, the same horse
to ride.''

``Very true.  Yes, dear old grey pony!  Ah! cousin, when I
remember how much I used to dread riding, what terrors
it gave me to hear it talked of as likely to do me good
(oh! how I have trembled at my uncle's opening his lips
if horses were talked of), and then think of the kind
pains you took to reason and persuade me out of my fears,
and convince me that I should like it after a little while,
and feel how right you proved to be, I am inclined to hope
you may always prophesy as well.''

``And I am quite convinced that your being with Mrs.\ Norris
will be as good for your mind as riding has been for
your health, and as much for your ultimate happiness too.''

So ended their discourse, which, for any very appropriate
service it could render Fanny, might as well have been spared,
for Mrs.\ Norris had not the smallest intention of taking her.
It had never occurred to her, on the present occasion,
but as a thing to be carefully avoided.  To prevent its
being expected, she had fixed on the smallest habitation
which could rank as genteel among the buildings of Mansfield
parish, the White House being only just large enough to
receive herself and her servants, and allow a spare room
for a friend, of which she made a very particular point.
The spare rooms at the Parsonage had never been wanted,
but the absolute necessity of a spare room for a friend
was now never forgotten.  Not all her precautions, however,
could save her from being suspected of something better;
or, perhaps, her very display of the importance of a
spare room might have misled Sir Thomas to suppose it
really intended for Fanny.  Lady Bertram soon brought
the matter to a certainty by carelessly observing to Mrs.\ Norris---%

``I think, sister, we need not keep Miss Lee any longer,
when Fanny goes to live with you.''

Mrs.\ Norris almost started.  ``Live with me, dear Lady
Bertram! what do you mean?''

``Is she not to live with you?  I thought you had settled
it with Sir Thomas.''

``Me! never.  I never spoke a syllable about it to Sir Thomas,
nor he to me.  Fanny live with me! the last thing in the
world for me to think of, or for anybody to wish that really
knows us both.  Good heaven! what could I do with Fanny?
Me! a poor, helpless, forlorn widow, unfit for anything,
my spirits quite broke down; what could I do with a girl
at her time of life?  A girl of fifteen! the very age
of all others to need most attention and care, and put
the cheerfullest spirits to the test!  Sure Sir Thomas
could not seriously expect such a thing!  Sir Thomas is too
much my friend.  Nobody that wishes me well, I am sure,
would propose it.  How came Sir Thomas to speak to you
about it?''

``Indeed, I do not know.  I suppose he thought it best.''

``But what did he say?  He could not say he \emph{wished} me
to take Fanny.  I am sure in his heart he could not wish
me to do it.''

``No; he only said he thought it very likely; and I thought
so too.  We both thought it would be a comfort to you.
But if you do not like it, there is no more to be said.
She is no encumbrance here.''

``Dear sister, if you consider my unhappy state, how can she
be any comfort to me?  Here am I, a poor desolate widow,
deprived of the best of husbands, my health gone in attending
and nursing him, my spirits still worse, all my peace
in this world destroyed, with hardly enough to support
me in the rank of a gentlewoman, and enable me to live
so as not to disgrace the memory of the dear departed---%
what possible comfort could I have in taking such a charge
upon me as Fanny?  If I could wish it for my own sake,
I would not do so unjust a thing by the poor girl.
She is in good hands, and sure of doing well.  I must
struggle through my sorrows and difficulties as I can.''

``Then you will not mind living by yourself quite alone?''

``Lady Bertram, I do not complain.  I know I cannot
live as I have done, but I must retrench where I can,
and learn to be a better manager.  I \emph{have been}
a liberal housekeeper enough, but I shall not be ashamed
to practise economy now.  My situation is as much
altered as my income.  A great many things were due
from poor Mr.\ Norris, as clergyman of the parish,
that cannot be expected from me.  It is unknown how much
was consumed in our kitchen by odd comers and goers.
At the White House, matters must be better looked after.
I \emph{must} live within my income, or I shall be miserable;
and I own it would give me great satisfaction to be able
to do rather more, to lay by a little at the end of
the year.''

``I dare say you will.  You always do, don't you?''

``My object, Lady Bertram, is to be of use to those that
come after me.  It is for your children's good that I
wish to be richer.  I have nobody else to care for,
but I should be very glad to think I could leave a little
trifle among them worth their having.''

``You are very good, but do not trouble yourself about them.
They are sure of being well provided for.  Sir Thomas
will take care of that.''

``Why, you know, Sir Thomas's means will be rather straitened
if the Antigua estate is to make such poor returns.''

``Oh! \emph{that} will soon be settled.  Sir Thomas has been
writing about it, I know.''

``Well, Lady Bertram,'' said Mrs.\ Norris, moving to go,
``I can only say that my sole desire is to be of use
to your family:  and so, if Sir Thomas should ever speak
again about my taking Fanny, you will be able to say that
my health and spirits put it quite out of the question;
besides that, I really should not have a bed to give her,
for I must keep a spare room for a friend.''

Lady Bertram repeated enough of this conversation
to her husband to convince him how much he had mistaken
his sister-in-law's views; and she was from that moment
perfectly safe from all expectation, or the slightest
allusion to it from him.  He could not but wonder at her
refusing to do anything for a niece whom she had been so
forward to adopt; but, as she took early care to make him,
as well as Lady Bertram, understand that whatever she
possessed was designed for their family, he soon grew
reconciled to a distinction which, at the same time
that it was advantageous and complimentary to them,
would enable him better to provide for Fanny himself.

Fanny soon learnt how unnecessary had been her fears of a removal;
and her spontaneous, untaught felicity on the discovery,
conveyed some consolation to Edmund for his disappointment
in what he had expected to be so essentially serviceable
to her.  Mrs.\ Norris took possession of the White House,
the Grants arrived at the Parsonage, and these events over,
everything at Mansfield went on for some time as usual.

The Grants showing a disposition to be friendly and sociable,
gave great satisfaction in the main among their new acquaintance.
They had their faults, and Mrs.\ Norris soon found them out.
The Doctor was very fond of eating, and would have a good
dinner every day; and Mrs.\ Grant, instead of contriving
to gratify him at little expense, gave her cook as high
wages as they did at Mansfield Park, and was scarcely ever
seen in her offices.  Mrs.\ Norris could not speak with any
temper of such grievances, nor of the quantity of butter
and eggs that were regularly consumed in the house.
``Nobody loved plenty and hospitality more than herself;
nobody more hated pitiful doings; the Parsonage,
she believed, had never been wanting in comforts of any sort,
had never borne a bad character in \emph{her time}, but this
was a way of going on that she could not understand.
A fine lady in a country parsonage was quite out of place.
\emph{Her} store-room, she thought, might have been good enough
for Mrs.\ Grant to go into.  Inquire where she would,
she could not find out that Mrs.\ Grant had ever had more
than five thousand pounds.''

Lady Bertram listened without much interest to this
sort of invective.  She could not enter into the wrongs
of an economist, but she felt all the injuries of beauty
in Mrs.\ Grant's being so well settled in life without
being handsome, and expressed her astonishment on
that point almost as often, though not so diffusely,
as Mrs.\ Norris discussed the other.

These opinions had been hardly canvassed a year before
another event arose of such importance in the family,
as might fairly claim some place in the thoughts and
conversation of the ladies.  Sir Thomas found it expedient
to go to Antigua himself, for the better arrangement
of his affairs, and he took his eldest son with him,
in the hope of detaching him from some bad connexions
at home.  They left England with the probability of being
nearly a twelvemonth absent.

The necessity of the measure in a pecuniary light,
and the hope of its utility to his son, reconciled Sir
Thomas to the effort of quitting the rest of his family,
and of leaving his daughters to the direction of others
at their present most interesting time of life.
He could not think Lady Bertram quite equal to supply his
place with them, or rather, to perform what should have
been her own; but, in Mrs.\ Norris's watchful attention,
and in Edmund's judgment, he had sufficient confidence
to make him go without fears for their conduct.

Lady Bertram did not at all like to have her husband leave her;
but she was not disturbed by any alarm for his safety,
or solicitude for his comfort, being one of those persons
who think nothing can be dangerous, or difficult,
or fatiguing to anybody but themselves.

The Miss Bertrams were much to be pitied on the occasion:
not for their sorrow, but for their want of it.
Their father was no object of love to them; he had never
seemed the friend of their pleasures, and his absence
was unhappily most welcome.  They were relieved by it from
all restraint; and without aiming at one gratification
that would probably have been forbidden by Sir Thomas,
they felt themselves immediately at their own disposal,
and to have every indulgence within their reach.
Fanny's relief, and her consciousness of it, were quite
equal to her cousins'; but a more tender nature suggested
that her feelings were ungrateful, and she really
grieved because she could not grieve.  ``Sir Thomas,
who had done so much for her and her brothers, and who was
gone perhaps never to return! that she should see him
go without a tear! it was a shameful insensibility.''
He had said to her, moreover, on the very last morning,
that he hoped she might see William again in the course
of the ensuing winter, and had charged her to write
and invite him to Mansfield as soon as the squadron
to which he belonged should be known to be in England.
``This was so thoughtful and kind!'' and would he only
have smiled upon her, and called her ``my dear Fanny,''
while he said it, every former frown or cold address
might have been forgotten.  But he had ended his speech
in a way to sink her in sad mortification, by adding,
``If William does come to Mansfield, I hope you may be able
to convince him that the many years which have passed
since you parted have not been spent on your side entirely
without improvement; though, I fear, he must find his sister
at sixteen in some respects too much like his sister at ten.''
She cried bitterly over this reflection when her uncle
was gone; and her cousins, on seeing her with red eyes,
set her down as a hypocrite.



\chapter{Chapter 4}

\gintro{Tom Bertram} had of late spent so little of his time at
home that he could be only nominally missed; and Lady
Bertram was soon astonished to find how very well they
did even without his father, how well Edmund could
supply his place in carving, talking to the steward,
writing to the attorney, settling with the servants,
and equally saving her from all possible fatigue or exertion
in every particular but that of directing her letters.

The earliest intelligence of the travellers' safe arrival
at Antigua, after a favourable voyage, was received;
though not before Mrs.\ Norris had been indulging in very
dreadful fears, and trying to make Edmund participate them
whenever she could get him alone; and as she depended
on being the first person made acquainted with any
fatal catastrophe, she had already arranged the manner of
breaking it to all the others, when Sir Thomas's assurances
of their both being alive and well made it necessary to lay
by her agitation and affectionate preparatory speeches for a while.

The winter came and passed without their being
called for; the accounts continued perfectly good;
and Mrs.\ Norris, in promoting gaieties for her nieces,
assisting their toilets, displaying their accomplishments,
and looking about for their future husbands, had so much
to do as, in addition to all her own household cares,
some interference in those of her sister, and Mrs.\ Grant's
wasteful doings to overlook, left her very little occasion
to be occupied in fears for the absent.

The Miss Bertrams were now fully established among the
belles of the neighbourhood; and as they joined to beauty
and brilliant acquirements a manner naturally easy,
and carefully formed to general civility and obligingness,
they possessed its favour as well as its admiration.
Their vanity was in such good order that they seemed
to be quite free from it, and gave themselves no airs;
while the praises attending such behaviour, secured and
brought round by their aunt, served to strengthen them in
believing they had no faults.

Lady Bertram did not go into public with her daughters.
She was too indolent even to accept a mother's gratification
in witnessing their success and enjoyment at the expense
of any personal trouble, and the charge was made over
to her sister, who desired nothing better than a post
of such honourable representation, and very thoroughly
relished the means it afforded her of mixing in society
without having horses to hire.

Fanny had no share in the festivities of the season;
but she enjoyed being avowedly useful as her aunt's companion
when they called away the rest of the family; and, as Miss
Lee had left Mansfield, she naturally became everything
to Lady Bertram during the night of a ball or a party.
She talked to her, listened to her, read to her;
and the tranquillity of such evenings, her perfect security
in such a \emph{t\^{e}te-\`{a}-t\^{e}te} from any sound of unkindness,
was unspeakably welcome to a mind which had seldom
known a pause in its alarms or embarrassments.  As to
her cousins' gaieties, she loved to hear an account of them,
especially of the balls, and whom Edmund had danced with;
but thought too lowly of her own situation to imagine
she should ever be admitted to the same, and listened,
therefore, without an idea of any nearer concern in them.
Upon the whole, it was a comfortable winter to her;
for though it brought no William to England, the never-failing
hope of his arrival was worth much.

The ensuing spring deprived her of her valued friend,
the old grey pony; and for some time she was in danger of
feeling the loss in her health as well as in her affections;
for in spite of the acknowledged importance of her riding
on horse-back, no measures were taken for mounting
her again, ``because,'' as it was observed by her aunts,
``she might ride one of her cousin's horses at any time
when they did not want them,'' and as the Miss Bertrams
regularly wanted their horses every fine day, and had no
idea of carrying their obliging manners to the sacrifice
of any real pleasure, that time, of course, never came.
They took their cheerful rides in the fine mornings
of April and May; and Fanny either sat at home the whole
day with one aunt, or walked beyond her strength at the
instigation of the other:  Lady Bertram holding exercise
to be as unnecessary for everybody as it was unpleasant
to herself; and Mrs.\ Norris, who was walking all day,
thinking everybody ought to walk as much.  Edmund was absent
at this time, or the evil would have been earlier remedied.
When he returned, to understand how Fanny was situated,
and perceived its ill effects, there seemed with him but
one thing to be done; and that ``Fanny must have a horse''
was the resolute declaration with which he opposed
whatever could be urged by the supineness of his mother,
or the economy of his aunt, to make it appear unimportant.
Mrs.\ Norris could not help thinking that some steady
old thing might be found among the numbers belonging
to the Park that would do vastly well; or that one might
be borrowed of the steward; or that perhaps Dr.\ Grant
might now and then lend them the pony he sent to the post.
She could not but consider it as absolutely unnecessary,
and even improper, that Fanny should have a regular
lady's horse of her own, in the style of her cousins.
She was sure Sir Thomas had never intended it:  and she
must say that, to be making such a purchase in his absence,
and adding to the great expenses of his stable,
at a time when a large part of his income was unsettled,
seemed to her very unjustifiable.  ``Fanny must have
a horse,'' was Edmund's only reply.  Mrs.\ Norris could
not see it in the same light.  Lady Bertram did:
she entirely agreed with her son as to the necessity of it,
and as to its being considered necessary by his father;
she only pleaded against there being any hurry; she only
wanted him to wait till Sir Thomas's return, and then Sir
Thomas might settle it all himself.  He would be at home
in September, and where would be the harm of only waiting
till September?

Though Edmund was much more displeased with his aunt than
with his mother, as evincing least regard for her niece,
he could not help paying more attention to what she said;
and at length determined on a method of proceeding
which would obviate the risk of his father's thinking he
had done too much, and at the same time procure for Fanny
the immediate means of exercise, which he could not bear
she should be without.  He had three horses of his own,
but not one that would carry a woman.  Two of them
were hunters; the third, a useful road-horse: this third he
resolved to exchange for one that his cousin might ride;
he knew where such a one was to be met with; and having once
made up his mind, the whole business was soon completed.
The new mare proved a treasure; with a very little
trouble she became exactly calculated for the purpose,
and Fanny was then put in almost full possession of her.
She had not supposed before that anything could ever suit
her like the old grey pony; but her delight in Edmund's
mare was far beyond any former pleasure of the sort;
and the addition it was ever receiving in the consideration
of that kindness from which her pleasure sprung,
was beyond all her words to express.  She regarded
her cousin as an example of everything good and great,
as possessing worth which no one but herself could
ever appreciate, and as entitled to such gratitude
from her as no feelings could be strong enough to pay.
Her sentiments towards him were compounded of all that
was respectful, grateful, confiding, and tender.

As the horse continued in name, as well as fact,
the property of Edmund, Mrs.\ Norris could tolerate its being
for Fanny's use; and had Lady Bertram ever thought about
her own objection again, he might have been excused in her
eyes for not waiting till Sir Thomas's return in September,
for when September came Sir Thomas was still abroad,
and without any near prospect of finishing his business.
Unfavourable circumstances had suddenly arisen at a moment
when he was beginning to turn all his thoughts towards England;
and the very great uncertainty in which everything was then
involved determined him on sending home his son, and waiting
the final arrangement by himself Tom arrived safely,
bringing an excellent account of his father's health;
but to very little purpose, as far as Mrs.\ Norris
was concerned.  Sir Thomas's sending away his son seemed
to her so like a parent's care, under the influence of a
foreboding of evil to himself, that she could not help
feeling dreadful presentiments; and as the long evenings
of autumn came on, was so terribly haunted by these ideas,
in the sad solitariness of her cottage, as to be obliged
to take daily refuge in the dining-room of the Park.
The return of winter engagements, however, was not
without its effect; and in the course of their progress,
her mind became so pleasantly occupied in superintending
the fortunes of her eldest niece, as tolerably to quiet
her nerves.  ``If poor Sir Thomas were fated never to return,
it would be peculiarly consoling to see their dear Maria
well married,'' she very often thought; always when they
were in the company of men of fortune, and particularly on
the introduction of a young man who had recently succeeded
to one of the largest estates and finest places in the country.

Mr.\ Rushworth was from the first struck with the beauty
of Miss Bertram, and, being inclined to marry, soon fancied
himself in love.  He was a heavy young man, with not more
than common sense; but as there was nothing disagreeable
in his figure or address, the young lady was well pleased
with her conquest.  Being now in her twenty-first year,
Maria Bertram was beginning to think matrimony a duty;
and as a marriage with Mr.\ Rushworth would give her the
enjoyment of a larger income than her father's, as well as
ensure her the house in town, which was now a prime object,
it became, by the same rule of moral obligation,
her evident duty to marry Mr.\ Rushworth if she could.
Mrs.\ Norris was most zealous in promoting the match,
by every suggestion and contrivance likely to enhance
its desirableness to either party; and, among other means,
by seeking an intimacy with the gentleman's mother,
who at present lived with him, and to whom she even forced
Lady Bertram to go through ten miles of indifferent road
to pay a morning visit.  It was not long before a good
understanding took place between this lady and herself.
Mrs.\ Rushworth acknowledged herself very desirous that
her son should marry, and declared that of all the young
ladies she had ever seen, Miss Bertram seemed, by her
amiable qualities and accomplishments, the best adapted
to make him happy.  Mrs.\ Norris accepted the compliment,
and admired the nice discernment of character which
could so well distinguish merit.  Maria was indeed
the pride and delight of them all---perfectly faultless---%
an angel; and, of course, so surrounded by admirers, must be
difficult in her choice:  but yet, as far as Mrs.\ Norris
could allow herself to decide on so short an acquaintance,
Mr.\ Rushworth appeared precisely the young man to deserve
and attach her.

After dancing with each other at a proper number of balls,
the young people justified these opinions, and an engagement,
with a due reference to the absent Sir Thomas, was entered into,
much to the satisfaction of their respective families,
and of the general lookers-on of the neighbourhood,
who had, for many weeks past, felt the expediency
of Mr.\ Rushworth's marrying Miss Bertram.

It was some months before Sir Thomas's consent could
be received; but, in the meanwhile, as no one felt
a doubt of his most cordial pleasure in the connexion,
the intercourse of the two families was carried on
without restraint, and no other attempt made at secrecy
than Mrs.\ Norris's talking of it everywhere as a matter
not to be talked of at present.

Edmund was the only one of the family who could see a fault
in the business; but no representation of his aunt's could
induce him to find Mr.\ Rushworth a desirable companion.
He could allow his sister to be the best judge of her
own happiness, but he was not pleased that her happiness
should centre in a large income; nor could he refrain
from often saying to himself, in Mr.\ Rushworth's company---%
``If this man had not twelve thousand a year, he would be
a very stupid fellow.''

Sir Thomas, however, was truly happy in the prospect of an
alliance so unquestionably advantageous, and of which he
heard nothing but the perfectly good and agreeable.
It was a connexion exactly of the right sort---%
in the same county, and the same interest---and his most
hearty concurrence was conveyed as soon as possible.
He only conditioned that the marriage should not take
place before his return, which he was again looking
eagerly forward to.  He wrote in April, and had strong
hopes of settling everything to his entire satisfaction,
and leaving Antigua before the end of the summer.

Such was the state of affairs in the month of July;
and Fanny had just reached her eighteenth year, when the
society of the village received an addition in the brother
and sister of Mrs.\ Grant, a Mr.\ and Miss Crawford,
the children of her mother by a second marriage.
They were young people of fortune.  The son had a good
estate in Norfolk, the daughter twenty thousand pounds.
As children, their sister had been always very fond
of them; but, as her own marriage had been soon followed
by the death of their common parent, which left them
to the care of a brother of their father, of whom
Mrs.\ Grant knew nothing, she had scarcely seen them since.
In their uncle's house they had found a kind home.
Admiral and Mrs.\ Crawford, though agreeing in nothing else,
were united in affection for these children, or, at least,
were no farther adverse in their feelings than that each
had their favourite, to whom they showed the greatest
fondness of the two.  The Admiral delighted in the boy,
Mrs.\ Crawford doted on the girl; and it was the lady's
death which now obliged her \emph{protegee}, after some months'
further trial at her uncle's house, to find another home.
Admiral Crawford was a man of vicious conduct, who chose,
instead of retaining his niece, to bring his mistress
under his own roof; and to this Mrs.\ Grant was indebted
for her sister's proposal of coming to her, a measure quite
as welcome on one side as it could be expedient on the other;
for Mrs.\ Grant, having by this time run through the usual
resources of ladies residing in the country without a
family of children---having more than filled her favourite
sitting-room with pretty furniture, and made a choice
collection of plants and poultry---was very much in want
of some variety at home.  The arrival, therefore, of a sister
whom she had always loved, and now hoped to retain with
her as long as she remained single, was highly agreeable;
and her chief anxiety was lest Mansfield should not satisfy
the habits of a young woman who had been mostly used
to London.

Miss Crawford was not entirely free from similar
apprehensions, though they arose principally from doubts
of her sister's style of living and tone of society;
and it was not till after she had tried in vain to persuade
her brother to settle with her at his own country house,
that she could resolve to hazard herself among her
other relations.  To anything like a permanence of abode,
or limitation of society, Henry Crawford had, unluckily,
a great dislike:  he could not accommodate his sister
in an article of such importance; but he escorted her,
with the utmost kindness, into Northamptonshire,
and as readily engaged to fetch her away again, at half
an hour's notice, whenever she were weary of the place.

The meeting was very satisfactory on each side.
Miss Crawford found a sister without preciseness
or rusticity, a sister's husband who looked the gentleman,
and a house commodious and well fitted up; and Mrs.\ Grant
received in those whom she hoped to love better than ever
a young man and woman of very prepossessing appearance.
Mary Crawford was remarkably pretty; Henry, though not handsome,
had air and countenance; the manners of both were lively
and pleasant, and Mrs.\ Grant immediately gave them credit
for everything else.  She was delighted with each,
but Mary was her dearest object; and having never been
able to glory in beauty of her own, she thoroughly enjoyed
the power of being proud of her sister's. She had not waited
her arrival to look out for a suitable match for her:
she had fixed on Tom Bertram; the eldest son of a baronet
was not too good for a girl of twenty thousand pounds,
with all the elegance and accomplishments which Mrs.\ Grant
foresaw in her; and being a warm-hearted, unreserved woman,
Mary had not been three hours in the house before she
told her what she had planned.

Miss Crawford was glad to find a family of such consequence
so very near them, and not at all displeased either at
her sister's early care, or the choice it had fallen on.
Matrimony was her object, provided she could marry well:
and having seen Mr.\ Bertram in town, she knew that
objection could no more be made to his person than to
his situation in life.  While she treated it as a joke,
therefore, she did not forget to think of it seriously.
The scheme was soon repeated to Henry.

``And now,'' added Mrs.\ Grant, ``I have thought of something
to make it complete.  I should dearly love to settle you
both in this country; and therefore, Henry, you shall
marry the youngest Miss Bertram, a nice, handsome,
good-humoured, accomplished girl, who will make you very happy.''

Henry bowed and thanked her.

``My dear sister,'' said Mary, ``if you can persuade him
into anything of the sort, it will be a fresh matter of
delight to me to find myself allied to anybody so clever,
and I shall only regret that you have not half a dozen
daughters to dispose of.  If you can persuade Henry
to marry, you must have the address of a Frenchwoman.
All that English abilities can do has been tried already.
I have three very particular friends who have been all
dying for him in their turn; and the pains which they,
their mothers (very clever women), as well as my dear
aunt and myself, have taken to reason, coax, or trick
him into marrying, is inconceivable!  He is the most
horrible flirt that can be imagined.  If your Miss
Bertrams do not like to have their hearts broke, let them
avoid Henry.''

``My dear brother, I will not believe this of you.''

``No, I am sure you are too good.  You will be kinder than Mary.
You will allow for the doubts of youth and inexperience.
I am of a cautious temper, and unwilling to risk my
happiness in a hurry.  Nobody can think more highly of
the matrimonial state than myself I consider the blessing
of a wife as most justly described in those discreet
lines of the poet---'Heaven's \emph{last} best gift.'\,''

``There, Mrs.\ Grant, you see how he dwells on one word,
and only look at his smile.  I assure you he is very detestable;
the Admiral's lessons have quite spoiled him.''

``I pay very little regard,'' said Mrs.\ Grant, ``to what
any young person says on the subject of marriage.
If they profess a disinclination for it, I only set it
down that they have not yet seen the right person.''

Dr.\ Grant laughingly congratulated Miss Crawford
on feeling no disinclination to the state herself.

``Oh yes!  I am not at all ashamed of it.  I would
have everybody marry if they can do it properly:
I do not like to have people throw themselves away;
but everybody should marry as soon as they can do it
to advantage.''



\chapter{Chapter 5}

\gintro{The young people} were pleased with each other from
the first.  On each side there was much to attract,
and their acquaintance soon promised as early an intimacy
as good manners would warrant.  Miss Crawford's
beauty did her no disservice with the Miss Bertrams.
They were too handsome themselves to dislike any woman
for being so too, and were almost as much charmed as their
brothers with her lively dark eye, clear brown complexion,
and general prettiness.  Had she been tall, full formed,
and fair, it might have been more of a trial:  but as it was,
there could be no comparison; and she was most allowably
a sweet, pretty girl, while they were the finest young
women in the country.

Her brother was not handsome:  no, when they first saw him
he was absolutely plain, black and plain; but still he
was the gentleman, with a pleasing address.  The second
meeting proved him not so very plain:  he was plain,
to be sure, but then he had so much countenance, and his
teeth were so good, and he was so well made, that one
soon forgot he was plain; and after a third interview,
after dining in company with him at the Parsonage,
he was no longer allowed to be called so by anybody.
He was, in fact, the most agreeable young man the sisters
had ever known, and they were equally delighted with him.
Miss Bertram's engagement made him in equity the property
of Julia, of which Julia was fully aware; and before he had
been at Mansfield a week, she was quite ready to be fallen
in love with.

Maria's notions on the subject were more confused
and indistinct.  She did not want to see or understand.
``There could be no harm in her liking an agreeable man---%
everybody knew her situation---Mr.\ Crawford must take care
of himself.''  Mr.\ Crawford did not mean to be in any danger!
the Miss Bertrams were worth pleasing, and were ready
to be pleased; and he began with no object but of making
them like him.  He did not want them to die of love;
but with sense and temper which ought to have made him
judge and feel better, he allowed himself great latitude
on such points.

``I like your Miss Bertrams exceedingly, sister,'' said he,
as he returned from attending them to their carriage
after the said dinner visit; ``they are very elegant,
agreeable girls.''

``So they are indeed, and I am delighted to hear you say it.
But you like Julia best.''

``Oh yes!  I like Julia best.''

``But do you really? for Miss Bertram is in general thought
the handsomest.''

``So I should suppose.  She has the advantage in every feature,
and I prefer her countenance; but I like Julia best;
Miss Bertram is certainly the handsomest, and I have found
her the most agreeable, but I shall always like Julia best,
because you order me.''

``I shall not talk to you, Henry, but I know you \emph{will}
like her best at last.''

``Do not I tell you that I like her best \emph{at first}?''

``And besides, Miss Bertram is engaged.  Remember that,
my dear brother.  Her choice is made.''

``Yes, and I like her the better for it.  An engaged
woman is always more agreeable than a disengaged.
She is satisfied with herself.  Her cares are over,
and she feels that she may exert all her powers of pleasing
without suspicion.  All is safe with a lady engaged:
no harm can be done.''

``Why, as to that, Mr.\ Rushworth is a very good sort
of young man, and it is a great match for her.''

``But Miss Bertram does not care three straws for him;
\emph{that} is your opinion of your intimate friend.  \emph{I} do
not subscribe to it.  I am sure Miss Bertram is very much
attached to Mr.\ Rushworth.  I could see it in her eyes,
when he was mentioned.  I think too well of Miss Bertram
to suppose she would ever give her hand without her heart.''

``Mary, how shall we manage him?''

``We must leave him to himself, I believe.  Talking does
no good.  He will be taken in at last.''

``But I would not have him \emph{taken in}; I would not have
him duped; I would have it all fair and honourable.''

``Oh dear! let him stand his chance and be taken in.
It will do just as well.  Everybody is taken in at some
period or other.''

``Not always in marriage, dear Mary.''

``In marriage especially.  With all due respect to such
of the present company as chance to be married, my dear
Mrs.\ Grant, there is not one in a hundred of either sex
who is not taken in when they marry.  Look where I will,
I see that it \emph{is} so; and I feel that it \emph{must} be so,
when I consider that it is, of all transactions, the one
in which people expect most from others, and are least
honest themselves.''

``Ah!  You have been in a bad school for matrimony,
in Hill Street.''

``My poor aunt had certainly little cause to love
the state; but, however, speaking from my own observation,
it is a manoeuvring business.  I know so many who
have married in the full expectation and confidence
of some one particular advantage in the connexion,
or accomplishment, or good quality in the person, who have
found themselves entirely deceived, and been obliged
to put up with exactly the reverse.  What is this but a take in?''

``My dear child, there must be a little imagination here.
I beg your pardon, but I cannot quite believe you.
Depend upon it, you see but half.  You see the evil,
but you do not see the consolation.  There will be
little rubs and disappointments everywhere, and we
are all apt to expect too much; but then, if one scheme
of happiness fails, human nature turns to another;
if the first calculation is wrong, we make a second better:
we find comfort somewhere---and those evil-minded observers,
dearest Mary, who make much of a little, are more taken
in and deceived than the parties themselves.''

``Well done, sister!  I honour your \emph{esprit du corps}.
When I am a wife, I mean to be just as staunch myself;
and I wish my friends in general would be so too.  It would
save me many a heartache.''

``You are as bad as your brother, Mary; but we will cure
you both.  Mansfield shall cure you both, and without
any taking in.  Stay with us, and we will cure you.''

The Crawfords, without wanting to be cured, were very
willing to stay.  Mary was satisfied with the Parsonage
as a present home, and Henry equally ready to lengthen
his visit.  He had come, intending to spend only a few
days with them; but Mansfield promised well, and there
was nothing to call him elsewhere.  It delighted Mrs.\ Grant
to keep them both with her, and Dr.\ Grant was exceedingly
well contented to have it so:  a talking pretty young
woman like Miss Crawford is always pleasant society
to an indolent, stay-at-home man; and Mr.\ Crawford's
being his guest was an excuse for drinking claret every day.

The Miss Bertrams' admiration of Mr.\ Crawford was more
rapturous than anything which Miss Crawford's habits made
her likely to feel.  She acknowledged, however, that the
Mr.\ Bertrams were very fine young men, that two such
young men were not often seen together even in London,
and that their manners, particularly those of the eldest,
were very good.  \emph{He} had been much in London,
and had more liveliness and gallantry than Edmund,
and must, therefore, be preferred; and, indeed, his being
the eldest was another strong claim.  She had felt an early
presentiment that she \emph{should} like the eldest best.
She knew it was her way.

Tom Bertram must have been thought pleasant, indeed, at any rate;
he was the sort of young man to be generally liked,
his agreeableness was of the kind to be oftener found
agreeable than some endowments of a higher stamp, for he
had easy manners, excellent spirits, a large acquaintance,
and a great deal to say; and the reversion of Mansfield Park,
and a baronetcy, did no harm to all this.  Miss Crawford
soon felt that he and his situation might do.  She looked
about her with due consideration, and found almost everything
in his favour:  a park, a real park, five miles round,
a spacious modern-built house, so well placed and well
screened as to deserve to be in any collection of engravings
of gentlemen's seats in the kingdom, and wanting only to be
completely new furnished---pleasant sisters, a quiet mother,
and an agreeable man himself---with the advantage of
being tied up from much gaming at present by a promise
to his father, and of being Sir Thomas hereafter.
It might do very well; she believed she should accept him;
and she began accordingly to interest herself a little
about the horse which he had to run at the B\gdash{} races.

These races were to call him away not long after their
acquaintance began; and as it appeared that the family
did not, from his usual goings on, expect him back
again for many weeks, it would bring his passion to an
early proof.  Much was said on his side to induce her
to attend the races, and schemes were made for a large
party to them, with all the eagerness of inclination,
but it would only do to be talked of.

And Fanny, what was \emph{she} doing and thinking all this
while? and what was \emph{her} opinion of the newcomers?
Few young ladies of eighteen could be less called on
to speak their opinion than Fanny.  In a quiet way,
very little attended to, she paid her tribute of admiration
to Miss Crawford's beauty; but as she still continued
to think Mr.\ Crawford very plain, in spite of her two
cousins having repeatedly proved the contrary, she never
mentioned \emph{him}.  The notice, which she excited herself,
was to this effect.  ``I begin now to understand you all,
except Miss Price,'' said Miss Crawford, as she was
walking with the Mr.\ Bertrams.  ``Pray, is she out,
or is she not?  I am puzzled.  She dined at the Parsonage,
with the rest of you, which seemed like being \emph{out};
and yet she says so little, that I can hardly suppose
she \emph{is}.''

Edmund, to whom this was chiefly addressed, replied, ``I believe
I know what you mean, but I will not undertake to answer
the question.  My cousin is grown up.  She has the age
and sense of a woman, but the outs and not outs are beyond me.''

``And yet, in general, nothing can be more easily ascertained.
The distinction is so broad.  Manners as well as
appearance are, generally speaking, so totally different.
Till now, I could not have supposed it possible to be
mistaken as to a girl's being out or not.  A girl not
out has always the same sort of dress:  a close bonnet,
for instance; looks very demure, and never says a word.
You may smile, but it is so, I assure you; and except
that it is sometimes carried a little too far, it is
all very proper.  Girls should be quiet and modest.
The most objectionable part is, that the alteration
of manners on being introduced into company is frequently
too sudden.  They sometimes pass in such very little
time from reserve to quite the opposite---to confidence!
\emph{That} is the faulty part of the present system.
One does not like to see a girl of eighteen or nineteen
so immediately up to every thing---and perhaps when one
has seen her hardly able to speak the year before.
Mr.\ Bertram, I dare say \emph{you} have sometimes met with
such changes.''

``I believe I have, but this is hardly fair; I see what you
are at.  You are quizzing me and Miss Anderson.''

``No, indeed.  Miss Anderson!  I do not know who or what
you mean.  I am quite in the dark.  But I \emph{will} quiz you
with a great deal of pleasure, if you will tell me what about.''

``Ah! you carry it off very well, but I cannot be quite
so far imposed on.  You must have had Miss Anderson
in your eye, in describing an altered young lady.
You paint too accurately for mistake.  It was exactly so.
The Andersons of Baker Street.  We were speaking of them
the other day, you know.  Edmund, you have heard me mention
Charles Anderson.  The circumstance was precisely as this
lady has represented it.  When Anderson first introduced
me to his family, about two years ago, his sister was
not \emph{out}, and I could not get her to speak to me.
I sat there an hour one morning waiting for Anderson,
with only her and a little girl or two in the room,
the governess being sick or run away, and the mother
in and out every moment with letters of business, and I
could hardly get a word or a look from the young lady---%
nothing like a civil answer---she screwed up her mouth,
and turned from me with such an air!  I did not see
her again for a twelvemonth.  She was then \emph{out}.
I met her at Mrs.\ Holford's, and did not recollect her.
She came up to me, claimed me as an acquaintance, stared me
out of countenance; and talked and laughed till I did not
know which way to look.  I felt that I must be the jest
of the room at the time, and Miss Crawford, it is plain,
has heard the story.''

``And a very pretty story it is, and with more truth
in it, I dare say, than does credit to Miss Anderson.
It is too common a fault.  Mothers certainly have not yet
got quite the right way of managing their daughters.
I do not know where the error lies.  I do not pretend to set
people right, but I do see that they are often wrong.''

``Those who are showing the world what female manners
\emph{should} be,'' said Mr.\ Bertram gallantly, ``are doing
a great deal to set them right.''

``The error is plain enough,'' said the less courteous Edmund;
``such girls are ill brought up.  They are given wrong notions
from the beginning.  They are always acting upon motives
of vanity, and there is no more real modesty in their
behaviour \emph{before} they appear in public than afterwards.''

``I do not know,'' replied Miss Crawford hesitatingly.
``Yes, I cannot agree with you there.  It is certainly
the modestest part of the business.  It is much worse to
have girls not out give themselves the same airs and take
the same liberties as if they were, which I have seen done.
That is worse than anything---quite disgusting!''

``Yes, \emph{that} is very inconvenient indeed,'' said Mr.\ Bertram.
``It leads one astray; one does not know what to do.
The close bonnet and demure air you describe so well (and
nothing was ever juster), tell one what is expected;
but I got into a dreadful scrape last year from the want
of them.  I went down to Ramsgate for a week with a friend
last September, just after my return from the West Indies.
My friend Sneyd---you have heard me speak of Sneyd, Edmund---%
his father, and mother, and sisters, were there, all new
to me.  When we reached Albion Place they were out;
we went after them, and found them on the pier:  Mrs.\ and
the two Miss Sneyds, with others of their acquaintance.
I made my bow in form; and as Mrs.\ Sneyd was surrounded
by men, attached myself to one of her daughters,
walked by her side all the way home, and made myself
as agreeable as I could; the young lady perfectly easy
in her manners, and as ready to talk as to listen.
I had not a suspicion that I could be doing anything wrong.
They looked just the same:  both well-dressed, with veils
and parasols like other girls; but I afterwards found
that I had been giving all my attention to the youngest,
who was not \emph{out}, and had most excessively offended
the eldest.  Miss Augusta ought not to have been noticed
for the next six months; and Miss Sneyd, I believe, has never
forgiven me.''

``That was bad indeed.  Poor Miss Sneyd.  Though I have no
younger sister, I feel for her.  To be neglected before
one's time must be very vexatious; but it was entirely
the mother's fault.  Miss Augusta should have been with
her governess.  Such half-and-half doings never prosper.
But now I must be satisfied about Miss Price.
Does she go to balls?  Does she dine out every where,
as well as at my sister's?''

``No,'' replied Edmund; ``I do not think she has ever been
to a ball.  My mother seldom goes into company herself,
and dines nowhere but with Mrs.\ Grant, and Fanny stays at
home with \emph{her}.''

``Oh! then the point is clear.  Miss Price is not out.''



\chapter{Chapter 6}

\gintro{Mr.\ Bertram} set off for \gdash{}, and Miss Crawford
was prepared to find a great chasm in their society,
and to miss him decidedly in the meetings which were now
becoming almost daily between the families; and on their
all dining together at the Park soon after his going,
she retook her chosen place near the bottom of the table,
fully expecting to feel a most melancholy difference in
the change of masters.  It would be a very flat business,
she was sure.  In comparison with his brother, Edmund would
have nothing to say.  The soup would be sent round in a
most spiritless manner, wine drank without any smiles
or agreeable trifling, and the venison cut up without
supplying one pleasant anecdote of any former haunch,
or a single entertaining story, about ``my friend such a one.''
She must try to find amusement in what was passing at the
upper end of the table, and in observing Mr.\ Rushworth,
who was now making his appearance at Mansfield for the first
time since the Crawfords' arrival.  He had been visiting
a friend in the neighbouring county, and that friend
having recently had his grounds laid out by an improver,
Mr.\ Rushworth was returned with his head full of the subject,
and very eager to be improving his own place in the same way;
and though not saying much to the purpose, could talk
of nothing else.  The subject had been already handled
in the drawing-room; it was revived in the dining-parlour.
Miss Bertram's attention and opinion was evidently
his chief aim; and though her deportment showed rather
conscious superiority than any solicitude to oblige him,
the mention of Sotherton Court, and the ideas attached
to it, gave her a feeling of complacency, which prevented
her from being very ungracious.

``I wish you could see Compton,'' said he; ``it is the most
complete thing!  I never saw a place so altered in my life.
I told Smith I did not know where I was.  The approach \emph{now},
is one of the finest things in the country:  you see the
house in the most surprising manner.  I declare, when I
got back to Sotherton yesterday, it looked like a prison---%
quite a dismal old prison.''

``Oh, for shame!'' cried Mrs.\ Norris.  ``A prison indeed?
Sotherton Court is the noblest old place in the world.''

``It wants improvement, ma'am, beyond anything.  I never
saw a place that wanted so much improvement in my life;
and it is so forlorn that I do not know what can be done
with it.''

``No wonder that Mr.\ Rushworth should think so at present,''
said Mrs.\ Grant to Mrs.\ Norris, with a smile; ``but depend
upon it, Sotherton will have \emph{every} improvement in time
which his heart can desire.''

``I must try to do something with it,'' said Mr.\ Rushworth,
``but I do not know what.  I hope I shall have some good
friend to help me.''

``Your best friend upon such an occasion,'' said Miss
Bertram calmly, ``would be Mr.\ Repton, I imagine.''

``That is what I was thinking of.  As he has done so
well by Smith, I think I had better have him at once.
His terms are five guineas a day.''

``Well, and if they were \emph{ten},'' cried Mrs.\ Norris,
``I am sure \emph{you} need not regard it.  The expense need
not be any impediment.  If I were you, I should not
think of the expense.  I would have everything done
in the best style, and made as nice as possible.
Such a place as Sotherton Court deserves everything that
taste and money can do.  You have space to work upon there,
and grounds that will well reward you.  For my own part,
if I had anything within the fiftieth part of the size
of Sotherton, I should be always planting and improving,
for naturally I am excessively fond of it.  It would be
too ridiculous for me to attempt anything where I am now,
with my little half acre.  It would be quite a burlesque.
But if I had more room, I should take a prodigious delight
in improving and planting.  We did a vast deal in that way
at the Parsonage:  we made it quite a different place
from what it was when we first had it.  You young ones
do not remember much about it, perhaps; but if dear Sir
Thomas were here, he could tell you what improvements
we made:  and a great deal more would have been done,
but for poor Mr.\ Norris's sad state of health.  He could
hardly ever get out, poor man, to enjoy anything, and \emph{that}
disheartened me from doing several things that Sir Thomas
and I used to talk of.  If it had not been for \emph{that},
we should have carried on the garden wall, and made the
plantation to shut out the churchyard, just as Dr.\ Grant
has done.  We were always doing something as it was.
It was only the spring twelvemonth before Mr.\ Norris's
death that we put in the apricot against the stable wall,
which is now grown such a noble tree, and getting
to such perfection, sir,'' addressing herself then to
Dr.\ Grant.

``The tree thrives well, beyond a doubt, madam,'' replied Dr.\ Grant.
``The soil is good; and I never pass it without regretting
that the fruit should be so little worth the trouble of gathering.''

``Sir, it is a Moor Park, we bought it as a Moor Park,
and it cost us---that is, it was a present from Sir Thomas,
but I saw the bill---and I know it cost seven shillings,
and was charged as a Moor Park.''

``You were imposed on, ma'am,'' replied Dr.\ Grant:
``these potatoes have as much the flavour of a Moor Park
apricot as the fruit from that tree.  It is an insipid
fruit at the best; but a good apricot is eatable,
which none from my garden are.''

``The truth is, ma'am,'' said Mrs.\ Grant, pretending to
whisper across the table to Mrs.\ Norris, ``that Dr.\ Grant
hardly knows what the natural taste of our apricot is:
he is scarcely ever indulged with one, for it is so
valuable a fruit; with a little assistance, and ours is
such a remarkably large, fair sort, that what with early
tarts and preserves, my cook contrives to get them all.''

Mrs.\ Norris, who had begun to redden, was appeased;
and, for a little while, other subjects took place of the
improvements of Sotherton.  Dr.\ Grant and Mrs.\ Norris
were seldom good friends; their acquaintance had begun
in dilapidations, and their habits were totally dissimilar.

After a short interruption Mr.\ Rushworth began again.
``Smith's place is the admiration of all the country;
and it was a mere nothing before Repton took it in hand.
I think I shall have Repton.''

``Mr.\ Rushworth,'' said Lady Bertram, ``if I were you,
I would have a very pretty shrubbery.  One likes to get
out into a shrubbery in fine weather.''

Mr.\ Rushworth was eager to assure her ladyship of his
acquiescence, and tried to make out something complimentary;
but, between his submission to \emph{her} taste, and his having
always intended the same himself, with the superadded
objects of professing attention to the comfort of ladies
in general, and of insinuating that there was one only whom
he was anxious to please, he grew puzzled, and Edmund was
glad to put an end to his speech by a proposal of wine.
Mr.\ Rushworth, however, though not usually a great talker,
had still more to say on the subject next his heart.
``Smith has not much above a hundred acres altogether
in his grounds, which is little enough, and makes it more
surprising that the place can have been so improved.
Now, at Sotherton we have a good seven hundred,
without reckoning the water meadows; so that I think,
if so much could be done at Compton, we need not despair.
There have been two or three fine old trees cut down, that grew
too near the house, and it opens the prospect amazingly,
which makes me think that Repton, or anybody of that sort,
would certainly have the avenue at Sotherton down:  the avenue
that leads from the west front to the top of the hill,
you know,'' turning to Miss Bertram particularly as he spoke.
But Miss Bertram thought it most becoming to reply---%

``The avenue!  Oh!  I do not recollect it.  I really know
very little of Sotherton.''

Fanny, who was sitting on the other side of Edmund,
exactly opposite Miss Crawford, and who had been attentively
listening, now looked at him, and said in a low voice---%

``Cut down an avenue!  What a pity!  Does it not make you
think of Cowper?  `Ye fallen avenues, once more I mourn
your fate unmerited.'\,''

He smiled as he answered, ``I am afraid the avenue stands
a bad chance, Fanny.''

``I should like to see Sotherton before it is cut down,
to see the place as it is now, in its old state; but I do
not suppose I shall.''

``Have you never been there?  No, you never can;
and, unluckily, it is out of distance for a ride.
I wish we could contrive it.''

``Oh! it does not signify.  Whenever I do see it,
you will tell me how it has been altered.''

``I collect,'' said Miss Crawford, ``that Sotherton
is an old place, and a place of some grandeur.
In any particular style of building?''

``The house was built in Elizabeth's time, and is a large,
regular, brick building; heavy, but respectable looking,
and has many good rooms.  It is ill placed.  It stands
in one of the lowest spots of the park; in that respect,
unfavourable for improvement.  But the woods are fine,
and there is a stream, which, I dare say, might be made
a good deal of.  Mr.\ Rushworth is quite right, I think,
in meaning to give it a modern dress, and I have no doubt
that it will be all done extremely well.''

Miss Crawford listened with submission, and said to herself,
``He is a well-bred man; he makes the best of it.''

``I do not wish to influence Mr.\ Rushworth,'' he continued;
``but, had I a place to new fashion, I should not put
myself into the hands of an improver.  I would rather
have an inferior degree of beauty, of my own choice,
and acquired progressively.  I would rather abide by my own
blunders than by his.''

``\emph{You} would know what you were about, of course;
but that would not suit \emph{me}.  I have no eye or
ingenuity for such matters, but as they are before me;
and had I a place of my own in the country, I should be
most thankful to any Mr.\ Repton who would undertake it,
and give me as much beauty as he could for my money;
and I should never look at it till it was complete.''

``It would be delightful to \emph{me} to see the progress
of it all,'' said Fanny.

``Ay, you have been brought up to it.  It was no part of
my education; and the only dose I ever had, being administered
by not the first favourite in the world, has made me consider
improvements \emph{in hand} as the greatest of nuisances.
Three years ago the Admiral, my honoured uncle, bought a
cottage at Twickenham for us all to spend our summers in;
and my aunt and I went down to it quite in raptures;
but it being excessively pretty, it was soon found
necessary to be improved, and for three months we were
all dirt and confusion, without a gravel walk to step on,
or a bench fit for use.  I would have everything as complete
as possible in the country, shrubberies and flower-gardens,
and rustic seats innumerable:  but it must all be done
without my care.  Henry is different; he loves to be doing.''

Edmund was sorry to hear Miss Crawford, whom he was much
disposed to admire, speak so freely of her uncle.
It did not suit his sense of propriety, and he was silenced,
till induced by further smiles and liveliness to put
the matter by for the present.

``Mr.\ Bertram,'' said she, ``I have tidings of my harp at last.
I am assured that it is safe at Northampton; and there it
has probably been these ten days, in spite of the solemn
assurances we have so often received to the contrary.''
Edmund expressed his pleasure and surprise.  ``The truth is,
that our inquiries were too direct; we sent a servant,
we went ourselves:  this will not do seventy miles from London;
but this morning we heard of it in the right way.
It was seen by some farmer, and he told the miller,
and the miller told the butcher, and the butcher's
son-in-law left word at the shop.''

``I am very glad that you have heard of it, by whatever means,
and hope there will be no further delay.''

``I am to have it to-morrow; but how do you think it
is to be conveyed?  Not by a wagon or cart:  oh no!
nothing of that kind could be hired in the village.
I might as well have asked for porters and a handbarrow.''

``You would find it difficult, I dare say, just now,
in the middle of a very late hay harvest, to hire a horse
and cart?''

``I was astonished to find what a piece of work was made of it!
To want a horse and cart in the country seemed impossible,
so I told my maid to speak for one directly; and as I cannot
look out of my dressing-closet without seeing one farmyard,
nor walk in the shrubbery without passing another,
I thought it would be only ask and have, and was rather
grieved that I could not give the advantage to all.
Guess my surprise, when I found that I had been asking
the most unreasonable, most impossible thing in the world;
had offended all the farmers, all the labourers,
all the hay in the parish!  As for Dr.\ Grant's bailiff,
I believe I had better keep out of \emph{his} way; and my
brother-in-law himself, who is all kindness in general,
looked rather black upon me when he found what I had
been at.''

``You could not be expected to have thought on the subject before;
but when you \emph{do} think of it, you must see the importance
of getting in the grass.  The hire of a cart at any time
might not be so easy as you suppose:  our farmers are
not in the habit of letting them out; but, in harvest,
it must be quite out of their power to spare a horse.''

``I shall understand all your ways in time; but, coming down
with the true London maxim, that everything is to be
got with money, I was a little embarrassed at first
by the sturdy independence of your country customs.
However, I am to have my harp fetched to-morrow. Henry,
who is good-nature itself, has offered to fetch
it in his barouche.  Will it not be honourably conveyed?''

Edmund spoke of the harp as his favourite instrument,
and hoped to be soon allowed to hear her.  Fanny had never
heard the harp at all, and wished for it very much.

``I shall be most happy to play to you both,'' said Miss
Crawford; ``at least as long as you can like to listen:
probably much longer, for I dearly love music myself,
and where the natural taste is equal the player must
always be best off, for she is gratified in more ways
than one.  Now, Mr.\ Bertram, if you write to your brother,
I entreat you to tell him that my harp is come:
he heard so much of my misery about it.  And you may say,
if you please, that I shall prepare my most plaintive
airs against his return, in compassion to his feelings,
as I know his horse will lose.''

``If I write, I will say whatever you wish me; but I do not,
at present, foresee any occasion for writing.''

``No, I dare say, nor if he were to be gone a twelvemonth,
would you ever write to him, nor he to you, if it could
be helped.  The occasion would never be foreseen.
What strange creatures brothers are!  You would not write
to each other but upon the most urgent necessity in the world;
and when obliged to take up the pen to say that such a horse
is ill, or such a relation dead, it is done in the fewest
possible words.  You have but one style among you.
I know it perfectly.  Henry, who is in every other respect
exactly what a brother should be, who loves me, consults me,
confides in me, and will talk to me by the hour together,
has never yet turned the page in a letter; and very often
it is nothing more than---'Dear Mary, I am just arrived.
Bath seems full, and everything as usual.  Yours sincerely.'
That is the true manly style; that is a complete
brother's letter.''

``When they are at a distance from all their family,''
said Fanny, colouring for William's sake, ``they can write
long letters.''

``Miss Price has a brother at sea,'' said Edmund,
``whose excellence as a correspondent makes her think
you too severe upon us.''

``At sea, has she?  In the king's service, of course?''

Fanny would rather have had Edmund tell the story,
but his determined silence obliged her to relate her
brother's situation:  her voice was animated in speaking
of his profession, and the foreign stations he had been on;
but she could not mention the number of years that he
had been absent without tears in her eyes.  Miss Crawford
civilly wished him an early promotion.

``Do you know anything of my cousin's captain?'' said Edmund;
``Captain Marshall?  You have a large acquaintance in the navy,
I conclude?''

``Among admirals, large enough; but,'' with an air of grandeur,
``we know very little of the inferior ranks.  Post-captains may
be very good sort of men, but they do not belong to \emph{us}.
Of various admirals I could tell you a great deal:
of them and their flags, and the gradation of their pay,
and their bickerings and jealousies.  But, in general,
I can assure you that they are all passed over, and all
very ill used.  Certainly, my home at my uncle's brought
me acquainted with a circle of admirals.  Of \emph{Rears} and
\emph{Vices} I saw enough.  Now do not be suspecting me of a pun,
I entreat.''

Edmund again felt grave, and only replied, ``It is
a noble profession.''

``Yes, the profession is well enough under two circumstances:
if it make the fortune, and there be discretion in spending it;
but, in short, it is not a favourite profession of mine.
It has never worn an amiable form to \emph{me}.''

Edmund reverted to the harp, and was again very happy
in the prospect of hearing her play.

The subject of improving grounds, meanwhile, was still
under consideration among the others; and Mrs.\ Grant could
not help addressing her brother, though it was calling
his attention from Miss Julia Bertram.

``My dear Henry, have \emph{you} nothing to say?  You have been
an improver yourself, and from what I hear of Everingham,
it may vie with any place in England.  Its natural beauties,
I am sure, are great.  Everingham, as it \emph{used} to be,
was perfect in my estimation:  such a happy fall of ground,
and such timber!  What would I not give to see it again?''

``Nothing could be so gratifying to me as to hear your
opinion of it,'' was his answer; ``but I fear there would
be some disappointment:  you would not find it equal
to your present ideas.  In extent, it is a mere nothing;
you would be surprised at its insignificance; and,
as for improvement, there was very little for me to do---%
too little:  I should like to have been busy much longer.''

``You are fond of the sort of thing?'' said Julia.

``Excessively; but what with the natural advantages of
the ground, which pointed out, even to a very young eye,
what little remained to be done, and my own consequent
resolutions, I had not been of age three months before
Everingham was all that it is now.  My plan was laid
at Westminster, a little altered, perhaps, at Cambridge,
and at one-and-twenty executed.  I am inclined to envy
Mr.\ Rushworth for having so much happiness yet before him.
I have been a devourer of my own.''

``Those who see quickly, will resolve quickly, and act quickly,''
said Julia.  ``\emph{You} can never want employment.
Instead of envying Mr.\ Rushworth, you should assist
him with your opinion.''

Mrs.\ Grant, hearing the latter part of this speech,
enforced it warmly, persuaded that no judgment could
be equal to her brother's; and as Miss Bertram caught
at the idea likewise, and gave it her full support,
declaring that, in her opinion, it was infinitely better
to consult with friends and disinterested advisers,
than immediately to throw the business into the hands of a
professional man, Mr.\ Rushworth was very ready to request
the favour of Mr.\ Crawford's assistance; and Mr.\ Crawford,
after properly depreciating his own abilities, was quite at
his service in any way that could be useful.  Mr.\ Rushworth
then began to propose Mr.\ Crawford's doing him the honour
of coming over to Sotherton, and taking a bed there;
when Mrs.\ Norris, as if reading in her two nieces'
minds their little approbation of a plan which was to take
Mr.\ Crawford away, interposed with an amendment.

``There can be no doubt of Mr.\ Crawford's willingness;
but why should not more of us go?  Why should not we
make a little party?  Here are many that would be
interested in your improvements, my dear Mr.\ Rushworth,
and that would like to hear Mr.\ Crawford's opinion on
the spot, and that might be of some small use to you with
\emph{their} opinions; and, for my own part, I have been long
wishing to wait upon your good mother again; nothing but
having no horses of my own could have made me so remiss;
but now I could go and sit a few hours with Mrs.\ Rushworth,
while the rest of you walked about and settled things,
and then we could all return to a late dinner here,
or dine at Sotherton, just as might be most agreeable to
your mother, and have a pleasant drive home by moonlight.
I dare say Mr.\ Crawford would take my two nieces and me
in his barouche, and Edmund can go on horseback, you know,
sister, and Fanny will stay at home with you.''

Lady Bertram made no objection; and every one concerned in
the going was forward in expressing their ready concurrence,
excepting Edmund, who heard it all and said nothing.



\chapter{Chapter 7}

\gintro{``Well, Fanny,} and how do you like Miss Crawford \emph{now}?''
said Edmund the next day, after thinking some time on the
subject himself.  ``How did you like her yesterday?''

``Very well---very much.  I like to hear her talk.
She entertains me; and she is so extremely pretty, that I
have great pleasure in looking at her.''

``It is her countenance that is so attractive.  She has
a wonderful play of feature!  But was there nothing in her
conversation that struck you, Fanny, as not quite right?''

``Oh yes! she ought not to have spoken of her uncle as she did.
I was quite astonished.  An uncle with whom she has been
living so many years, and who, whatever his faults may be,
is so very fond of her brother, treating him, they say,
quite like a son.  I could not have believed it!''

``I thought you would be struck.  It was very wrong;
very indecorous.''

``And very ungrateful, I think.''

``Ungrateful is a strong word.  I do not know that her uncle
has any claim to her \emph{gratitude}; his wife certainly had;
and it is the warmth of her respect for her aunt's memory
which misleads her here.  She is awkwardly circumstanced.
With such warm feelings and lively spirits it must be
difficult to do justice to her affection for Mrs.\ Crawford,
without throwing a shade on the Admiral.  I do not pretend
to know which was most to blame in their disagreements,
though the Admiral's present conduct might incline one
to the side of his wife; but it is natural and amiable
that Miss Crawford should acquit her aunt entirely.
I do not censure her \emph{opinions}; but there certainly \emph{is}
impropriety in making them public.''

``Do not you think,'' said Fanny, after a little consideration,
``that this impropriety is a reflection itself upon
Mrs.\ Crawford, as her niece has been entirely brought
up by her?  She cannot have given her right notions
of what was due to the Admiral.''

``That is a fair remark.  Yes, we must suppose the faults
of the niece to have been those of the aunt; and it makes
one more sensible of the disadvantages she has been under.
But I think her present home must do her good.
Mrs.\ Grant's manners are just what they ought to be.
She speaks of her brother with a very pleasing affection.''

``Yes, except as to his writing her such short letters.
She made me almost laugh; but I cannot rate so very highly
the love or good-nature of a brother who will not give
himself the trouble of writing anything worth reading
to his sisters, when they are separated.  I am sure William
would never have used \emph{me} so, under any circumstances.
And what right had she to suppose that \emph{you} would not write
long letters when you were absent?''

``The right of a lively mind, Fanny, seizing whatever
may contribute to its own amusement or that of others;
perfectly allowable, when untinctured by ill-humour
or roughness; and there is not a shadow of either in the
countenance or manner of Miss Crawford:  nothing sharp,
or loud, or coarse.  She is perfectly feminine, except in
the instances we have been speaking of.  There she cannot
be justified.  I am glad you saw it all as I did.''

Having formed her mind and gained her affections, he had a
good chance of her thinking like him; though at this period,
and on this subject, there began now to be some danger
of dissimilarity, for he was in a line of admiration
of Miss Crawford, which might lead him where Fanny could
not follow.  Miss Crawford's attractions did not lessen.
The harp arrived, and rather added to her beauty, wit,
and good-humour; for she played with the greatest obligingness,
with an expression and taste which were peculiarly becoming,
and there was something clever to be said at the close
of every air.  Edmund was at the Parsonage every day,
to be indulged with his favourite instrument:
one morning secured an invitation for the next;
for the lady could not be unwilling to have a listener,
and every thing was soon in a fair train.

A young woman, pretty, lively, with a harp as
elegant as herself, and both placed near a window,
cut down to the ground, and opening on a little lawn,
surrounded by shrubs in the rich foliage of summer,
was enough to catch any man's heart.  The season, the scene,
the air, were all favourable to tenderness and sentiment.
Mrs.\ Grant and her tambour frame were not without their use:
it was all in harmony; and as everything will turn to account
when love is once set going, even the sandwich tray,
and Dr.\ Grant doing the honours of it, were worth looking at.
Without studying the business, however, or knowing
what he was about, Edmund was beginning, at the end
of a week of such intercourse, to be a good deal in love;
and to the credit of the lady it may be added that,
without his being a man of the world or an elder brother,
without any of the arts of flattery or the gaieties of
small talk, he began to be agreeable to her.  She felt it
to be so, though she had not foreseen, and could hardly
understand it; for he was not pleasant by any common rule:
he talked no nonsense; he paid no compliments; his opinions
were unbending, his attentions tranquil and simple.
There was a charm, perhaps, in his sincerity, his steadiness,
his integrity, which Miss Crawford might be equal
to feel, though not equal to discuss with herself.
She did not think very much about it, however:  he pleased
her for the present; she liked to have him near her;
it was enough.

Fanny could not wonder that Edmund was at the Parsonage
every morning; she would gladly have been there too,
might she have gone in uninvited and unnoticed, to hear
the harp; neither could she wonder that, when the evening
stroll was over, and the two families parted again,
he should think it right to attend Mrs.\ Grant and her
sister to their home, while Mr.\ Crawford was devoted
to the ladies of the Park; but she thought it a very
bad exchange; and if Edmund were not there to mix the wine
and water for her, would rather go without it than not.
She was a little surprised that he could spend so many
hours with Miss Crawford, and not see more of the sort
of fault which he had already observed, and of which \emph{she}
was almost always reminded by a something of the same
nature whenever she was in her company; but so it was.
Edmund was fond of speaking to her of Miss Crawford,
but he seemed to think it enough that the Admiral had
since been spared; and she scrupled to point out her own
remarks to him, lest it should appear like ill-nature.
The first actual pain which Miss Crawford occasioned her
was the consequence of an inclination to learn to ride,
which the former caught, soon after her being settled
at Mansfield, from the example of the young ladies at the Park,
and which, when Edmund's acquaintance with her increased,
led to his encouraging the wish, and the offer of his own
quiet mare for the purpose of her first attempts, as the best
fitted for a beginner that either stable could furnish.
No pain, no injury, however, was designed by him to his
cousin in this offer:  \emph{she} was not to lose a day's exercise
by it.  The mare was only to be taken down to the Parsonage
half an hour before her ride were to begin; and Fanny,
on its being first proposed, so far from feeling slighted,
was almost over-powered with gratitude that he should be
asking her leave for it.

Miss Crawford made her first essay with great credit
to herself, and no inconvenience to Fanny.  Edmund,
who had taken down the mare and presided at the whole,
returned with it in excellent time, before either Fanny
or the steady old coachman, who always attended her when
she rode without her cousins, were ready to set forward.
The second day's trial was not so guiltless.  Miss Crawford's
enjoyment of riding was such that she did not know how to
leave off.  Active and fearless, and though rather small,
strongly made, she seemed formed for a horsewoman; and to
the pure genuine pleasure of the exercise, something was
probably added in Edmund's attendance and instructions,
and something more in the conviction of very much surpassing
her sex in general by her early progress, to make her
unwilling to dismount.  Fanny was ready and waiting,
and Mrs.\ Norris was beginning to scold her for not being gone,
and still no horse was announced, no Edmund appeared.
To avoid her aunt, and look for him, she went out.

The houses, though scarcely half a mile apart, were not
within sight of each other; but, by walking fifty yards
from the hall door, she could look down the park,
and command a view of the Parsonage and all its demesnes,
gently rising beyond the village road; and in Dr.\ Grant's
meadow she immediately saw the group---Edmund and Miss
Crawford both on horse-back, riding side by side, Dr.\ and
Mrs.\ Grant, and Mr.\ Crawford, with two or three grooms,
standing about and looking on.  A happy party it appeared
to her, all interested in one object:  cheerful beyond
a doubt, for the sound of merriment ascended even to her.
It was a sound which did not make \emph{her} cheerful;
she wondered that Edmund should forget her, and felt
a pang.  She could not turn her eyes from the meadow;
she could not help watching all that passed.  At first Miss
Crawford and her companion made the circuit of the field,
which was not small, at a foot's pace; then, at \emph{her}
apparent suggestion, they rose into a canter; and to Fanny's
timid nature it was most astonishing to see how well
she sat.  After a few minutes they stopped entirely.
Edmund was close to her; he was speaking to her;
he was evidently directing her management of the bridle;
he had hold of her hand; she saw it, or the imagination
supplied what the eye could not reach.  She must not
wonder at all this; what could be more natural than that
Edmund should be making himself useful, and proving his
good-nature by any one?  She could not but think, indeed,
that Mr.\ Crawford might as well have saved him the trouble;
that it would have been particularly proper and becoming
in a brother to have done it himself; but Mr.\ Crawford,
with all his boasted good-nature, and all his coachmanship,
probably knew nothing of the matter, and had no active
kindness in comparison of Edmund. She began to think it
rather hard upon the mare to have such double duty;
if she were forgotten, the poor mare should be remembered.

Her feelings for one and the other were soon a little
tranquillised by seeing the party in the meadow disperse,
and Miss Crawford still on horseback, but attended by Edmund
on foot, pass through a gate into the lane, and so into
the park, and make towards the spot where she stood.
She began then to be afraid of appearing rude and impatient;
and walked to meet them with a great anxiety to avoid
the suspicion.

``My dear Miss Price,'' said Miss Crawford, as soon as she
was at all within hearing, ``I am come to make my own
apologies for keeping you waiting; but I have nothing
in the world to say for myself---I knew it was very late,
and that I was behaving extremely ill; and therefore,
if you please, you must forgive me.  Selfishness must
always be forgiven, you know, because there is no hope
of a cure.''

Fanny's answer was extremely civil, and Edmund added
his conviction that she could be in no hurry.  ``For there
is more than time enough for my cousin to ride twice
as far as she ever goes,'' said he, ``and you have been
promoting her comfort by preventing her from setting off
half an hour sooner:  clouds are now coming up, and she
will not suffer from the heat as she would have done then.
I wish \emph{you} may not be fatigued by so much exercise.
I wish you had saved yourself this walk home.''

``No part of it fatigues me but getting off this horse,
I assure you,'' said she, as she sprang down with his help;
``I am very strong.  Nothing ever fatigues me but doing
what I do not like.  Miss Price, I give way to you with
a very bad grace; but I sincerely hope you will have
a pleasant ride, and that I may have nothing but good
to hear of this dear, delightful, beautiful animal.''

The old coachman, who had been waiting about with his
own horse, now joining them, Fanny was lifted on hers,
and they set off across another part of the park;
her feelings of discomfort not lightened by seeing,
as she looked back, that the others were walking down
the hill together to the village; nor did her attendant
do her much good by his comments on Miss Crawford's great
cleverness as a horse-woman, which he had been watching
with an interest almost equal to her own.

``It is a pleasure to see a lady with such a good heart
for riding!'' said he.  ``I never see one sit a horse better.
She did not seem to have a thought of fear.  Very different
from you, miss, when you first began, six years ago come
next Easter.  Lord bless you! how you did tremble when Sir
Thomas first had you put on!''

In the drawing-room Miss Crawford was also celebrated.
Her merit in being gifted by Nature with strength
and courage was fully appreciated by the Miss Bertrams;
her delight in riding was like their own; her early
excellence in it was like their own, and they had great
pleasure in praising it.

``I was sure she would ride well,'' said Julia; ``she has
the make for it.  Her figure is as neat as her brother's.''

``Yes,'' added Maria, ``and her spirits are as good, and she
has the same energy of character.  I cannot but think
that good horsemanship has a great deal to do with the mind.''

When they parted at night Edmund asked Fanny whether she
meant to ride the next day.

``No, I do not know---not if you want the mare,'' was her answer.

``I do not want her at all for myself,'' said he;
``but whenever you are next inclined to stay at home,
I think Miss Crawford would be glad to have her a longer time---%
for a whole morning, in short.  She has a great desire to get
as far as Mansfield Common:  Mrs.\ Grant has been telling
her of its fine views, and I have no doubt of her being
perfectly equal to it.  But any morning will do for this.
She would be extremely sorry to interfere with you.
It would be very wrong if she did.  \emph{She} rides only
for pleasure; \emph{you} for health.''

``I shall not ride to-morrow, certainly,'' said Fanny;
``I have been out very often lately, and would rather
stay at home.  You know I am strong enough now to walk
very well.''

Edmund looked pleased, which must be Fanny's comfort,
and the ride to Mansfield Common took place the next morning:
the party included all the young people but herself,
and was much enjoyed at the time, and doubly enjoyed
again in the evening discussion.  A successful scheme
of this sort generally brings on another; and the having
been to Mansfield Common disposed them all for going
somewhere else the day after.  There were many other
views to be shewn; and though the weather was hot,
there were shady lanes wherever they wanted to go.
A young party is always provided with a shady lane.
Four fine mornings successively were spent in this manner,
in shewing the Crawfords the country, and doing the
honours of its finest spots.  Everything answered;
it was all gaiety and good-humour, the heat only supplying
inconvenience enough to be talked of with pleasure---%
till the fourth day, when the happiness of one of the party
was exceedingly clouded.  Miss Bertram was the one.
Edmund and Julia were invited to dine at the Parsonage,
and \emph{she} was excluded.  It was meant and done by Mrs.\ Grant,
with perfect good-humour, on Mr.\ Rushworth's account,
who was partly expected at the Park that day; but it was felt
as a very grievous injury, and her good manners were severely
taxed to conceal her vexation and anger till she reached home.
As Mr.\ Rushworth did \emph{not} come, the injury was increased,
and she had not even the relief of shewing her power over him;
she could only be sullen to her mother, aunt, and cousin,
and throw as great a gloom as possible over their dinner
and dessert.

Between ten and eleven Edmund and Julia walked into the
drawing-room, fresh with the evening air, glowing and cheerful,
the very reverse of what they found in the three ladies
sitting there, for Maria would scarcely raise her eyes
from her book, and Lady Bertram was half-asleep; and even
Mrs.\ Norris, discomposed by her niece's ill-humour,
and having asked one or two questions about the dinner,
which were not immediately attended to, seemed almost
determined to say no more.  For a few minutes the brother
and sister were too eager in their praise of the night
and their remarks on the stars, to think beyond themselves;
but when the first pause came, Edmund, looking around,
said, ``But where is Fanny?  Is she gone to bed?''

``No, not that I know of,'' replied Mrs.\ Norris; ``she was
here a moment ago.''

Her own gentle voice speaking from the other end
of the room, which was a very long one, told them
that she was on the sofa.  Mrs.\ Norris began scolding.

``That is a very foolish trick, Fanny, to be idling away all
the evening upon a sofa.  Why cannot you come and sit here,
and employ yourself as \emph{we} do?  If you have no work
of your own, I can supply you from the poor basket.
There is all the new calico, that was bought last week,
not touched yet.  I am sure I almost broke my back
by cutting it out.  You should learn to think of
other people; and, take my word for it, it is a shocking
trick for a young person to be always lolling upon a sofa.''

Before half this was said, Fanny was returned to her
seat at the table, and had taken up her work again;
and Julia, who was in high good-humour, from the pleasures
of the day, did her the justice of exclaiming, ``I must say,
ma'am, that Fanny is as little upon the sofa as anybody
in the house.''

``Fanny,'' said Edmund, after looking at her attentively,
``I am sure you have the headache.''

She could not deny it, but said it was not very bad.

``I can hardly believe you,'' he replied; ``I know your looks
too well.  How long have you had it?''

``Since a little before dinner.  It is nothing but the heat.''

``Did you go out in the heat?''

``Go out! to be sure she did,'' said Mrs.\ Norris:
``would you have her stay within such a fine day as this?
Were not we \emph{all} out?  Even your mother was out to-day
for above an hour.''

``Yes, indeed, Edmund,'' added her ladyship, who had been
thoroughly awakened by Mrs.\ Norris's sharp reprimand
to Fanny; ``I was out above an hour.  I sat three-quarters
of an hour in the flower-garden, while Fanny cut the roses;
and very pleasant it was, I assure you, but very hot.
It was shady enough in the alcove, but I declare I quite
dreaded the coming home again.''

``Fanny has been cutting roses, has she?''

``Yes, and I am afraid they will be the last this year.
Poor thing!  \emph{She} found it hot enough; but they were so
full-blown that one could not wait.''

``There was no help for it, certainly,'' rejoined Mrs.\ Norris,
in a rather softened voice; ``but I question whether her
headache might not be caught \emph{then}, sister.  There is
nothing so likely to give it as standing and stooping
in a hot sun; but I dare say it will be well to-morrow.
Suppose you let her have your aromatic vinegar; I always
forget to have mine filled.''

``She has got it,'' said Lady Bertram; ``she has had it ever
since she came back from your house the second time.''

``What!'' cried Edmund; ``has she been walking as well as
cutting roses; walking across the hot park to your house,
and doing it twice, ma'am?  No wonder her head aches.''

Mrs.\ Norris was talking to Julia, and did not hear.

``I was afraid it would be too much for her,'' said Lady Bertram;
``but when the roses were gathered, your aunt wished
to have them, and then you know they must be taken home.''

``But were there roses enough to oblige her to go twice?''

``No; but they were to be put into the spare room to dry;
and, unluckily, Fanny forgot to lock the door of the room
and bring away the key, so she was obliged to go again.''

Edmund got up and walked about the room, saying, ``And could
nobody be employed on such an errand but Fanny?  Upon my word,
ma'am, it has been a very ill-managed business.''

``I am sure I do not know how it was to have been done better,''
cried Mrs.\ Norris, unable to be longer deaf; ``unless I had
gone myself, indeed; but I cannot be in two places at once;
and I was talking to Mr.\ Green at that very time about
your mother's dairymaid, by \emph{her} desire, and had promised
John Groom to write to Mrs.\ Jefferies about his son,
and the poor fellow was waiting for me half an hour.
I think nobody can justly accuse me of sparing myself upon
any occasion, but really I cannot do everything at once.
And as for Fanny's just stepping down to my house for me---%
it is not much above a quarter of a mile---I cannot think I
was unreasonable to ask it.  How often do I pace it three
times a day, early and late, ay, and in all weathers too,
and say nothing about it?''

``I wish Fanny had half your strength, ma'am.''

``If Fanny would be more regular in her exercise, she would
not be knocked up so soon.  She has not been out on
horseback now this long while, and I am persuaded that,
when she does not ride, she ought to walk.  If she had
been riding before, I should not have asked it of her.
But I thought it would rather do her good after being
stooping among the roses; for there is nothing so
refreshing as a walk after a fatigue of that kind;
and though the sun was strong, it was not so very hot.
Between ourselves, Edmund,'' nodding significantly at
his mother, ``it was cutting the roses, and dawdling
about in the flower-garden, that did the mischief.''

``I am afraid it was, indeed,'' said the more candid
Lady Bertram, who had overheard her; ``I am very much afraid
she caught the headache there, for the heat was enough
to kill anybody.  It was as much as I could bear myself.
Sitting and calling to Pug, and trying to keep him from
the flower-beds, was almost too much for me.''

Edmund said no more to either lady; but going quietly
to another table, on which the supper-tray yet remained,
brought a glass of Madeira to Fanny, and obliged her to drink
the greater part.  She wished to be able to decline it;
but the tears, which a variety of feelings created,
made it easier to swallow than to speak.

Vexed as Edmund was with his mother and aunt, he was still
more angry with himself.  His own forgetfulness of her was
worse than anything which they had done.  Nothing of this
would have happened had she been properly considered;
but she had been left four days together without any choice
of companions or exercise, and without any excuse for
avoiding whatever her unreasonable aunts might require.
He was ashamed to think that for four days together she had
not had the power of riding, and very seriously resolved,
however unwilling he must be to check a pleasure of Miss
Crawford's, that it should never happen again.

Fanny went to bed with her heart as full as on the first
evening of her arrival at the Park.  The state of her
spirits had probably had its share in her indisposition;
for she had been feeling neglected, and been struggling
against discontent and envy for some days past.
As she leant on the sofa, to which she had retreated
that she might not be seen, the pain of her mind
had been much beyond that in her head; and the sudden
change which Edmund's kindness had then occasioned,
made her hardly know how to support herself.



\chapter{Chapter 8}

\gintro{Fanny's} rides recommenced the very next day; and as it
was a pleasant fresh-feeling morning, less hot than the
weather had lately been, Edmund trusted that her losses,
both of health and pleasure, would be soon made good.
While she was gone Mr.\ Rushworth arrived, escorting his mother,
who came to be civil and to shew her civility especially,
in urging the execution of the plan for visiting Sotherton,
which had been started a fortnight before, and which,
in consequence of her subsequent absence from home,
had since lain dormant.  Mrs.\ Norris and her nieces were all
well pleased with its revival, and an early day was named
and agreed to, provided Mr.\ Crawford should be disengaged:
the young ladies did not forget that stipulation, and though
Mrs.\ Norris would willingly have answered for his being so,
they would neither authorise the liberty nor run the risk;
and at last, on a hint from Miss Bertram, Mr.\ Rushworth
discovered that the properest thing to be done was for
him to walk down to the Parsonage directly, and call on
Mr.\ Crawford, and inquire whether Wednesday would suit him
or not.

Before his return Mrs.\ Grant and Miss Crawford came in.
Having been out some time, and taken a different route
to the house, they had not met him.  Comfortable hopes,
however, were given that he would find Mr.\ Crawford
at home.  The Sotherton scheme was mentioned of course.
It was hardly possible, indeed, that anything else should
be talked of, for Mrs.\ Norris was in high spirits about it;
and Mrs.\ Rushworth, a well-meaning, civil, prosing,
pompous woman, who thought nothing of consequence, but as it
related to her own and her son's concerns, had not yet
given over pressing Lady Bertram to be of the party.
Lady Bertram constantly declined it; but her placid manner
of refusal made Mrs.\ Rushworth still think she wished
to come, till Mrs.\ Norris's more numerous words and louder
tone convinced her of the truth.

``The fatigue would be too much for my sister, a great
deal too much, I assure you, my dear Mrs.\ Rushworth.
Ten miles there, and ten back, you know.  You must
excuse my sister on this occasion, and accept of our
two dear girls and myself without her.  Sotherton is
the only place that could give her a \emph{wish} to go so far,
but it cannot be, indeed.  She will have a companion
in Fanny Price, you know, so it will all do very well;
and as for Edmund, as he is not here to speak for himself,
I will answer for his being most happy to join the party.
He can go on horseback, you know.''

Mrs.\ Rushworth being obliged to yield to Lady Bertram's
staying at home, could only be sorry.  ``The loss of her
ladyship's company would be a great drawback, and she
should have been extremely happy to have seen the young
lady too, Miss Price, who had never been at Sotherton yet,
and it was a pity she should not see the place.''

``You are very kind, you are all kindness, my dear madam,''
cried Mrs.\ Norris; ``but as to Fanny, she will have
opportunities in plenty of seeing Sotherton.  She has
time enough before her; and her going now is quite out
of the question.  Lady Bertram could not possibly spare her.''

``Oh no!  I cannot do without Fanny.''

Mrs.\ Rushworth proceeded next, under the conviction that
everybody must be wanting to see Sotherton, to include
Miss Crawford in the invitation; and though Mrs.\ Grant,
who had not been at the trouble of visiting Mrs.\ Rushworth,
on her coming into the neighbourhood, civilly declined it
on her own account, she was glad to secure any pleasure
for her sister; and Mary, properly pressed and persuaded,
was not long in accepting her share of the civility.
Mr.\ Rushworth came back from the Parsonage successful;
and Edmund made his appearance just in time to learn what
had been settled for Wednesday, to attend Mrs.\ Rushworth
to her carriage, and walk half-way down the park with the two
other ladies.

On his return to the breakfast-room, he found Mrs.\ Norris
trying to make up her mind as to whether Miss Crawford's
being of the party were desirable or not, or whether
her brother's barouche would not be full without her.
The Miss Bertrams laughed at the idea, assuring her
that the barouche would hold four perfectly well,
independent of the box, on which \emph{one} might go with him.

``But why is it necessary,'' said Edmund, ``that Crawford's carriage,
or his \emph{only}, should be employed?  Why is no use to be
made of my mother's chaise?  I could not, when the scheme
was first mentioned the other day, understand why a visit
from the family were not to be made in the carriage of the family.''

``What!'' cried Julia:  ``go boxed up three in a postchaise
in this weather, when we may have seats in a barouche!
No, my dear Edmund, that will not quite do.''

``Besides,'' said Maria, ``I know that Mr.\ Crawford depends
upon taking us.  After what passed at first, he would
claim it as a promise.''

``And, my dear Edmund,'' added Mrs.\ Norris, ``taking out \emph{two}
carriages when \emph{one} will do, would be trouble for nothing;
and, between ourselves, coachman is not very fond of the
roads between this and Sotherton:  he always complains
bitterly of the narrow lanes scratching his carriage,
and you know one should not like to have dear Sir Thomas,
when he comes home, find all the varnish scratched off.''

``That would not be a very handsome reason for using
Mr.\ Crawford's,'' said Maria; ``but the truth is, that Wilcox
is a stupid old fellow, and does not know how to drive.
I will answer for it that we shall find no inconvenience
from narrow roads on Wednesday.''

``There is no hardship, I suppose, nothing unpleasant,''
said Edmund, ``in going on the barouche box.''

``Unpleasant!'' cried Maria:  ``oh dear!  I believe it would
be generally thought the favourite seat.  There can
be no comparison as to one's view of the country.
Probably Miss Crawford will choose the barouche-box herself.''

``There can be no objection, then, to Fanny's going with you;
there can be no doubt of your having room for her.''

``Fanny!'' repeated Mrs.\ Norris; ``my dear Edmund, there is
no idea of her going with us.  She stays with her aunt.
I told Mrs.\ Rushworth so.  She is not expected.''

``You can have no reason, I imagine, madam,'' said he,
addressing his mother, ``for wishing Fanny \emph{not}
to be of the party, but as it relates to yourself,
to your own comfort.  If you could do without her,
you would not wish to keep her at home?''

``To be sure not, but I \emph{cannot} do without her.''

``You can, if I stay at home with you, as I mean to do.''

There was a general cry out at this.  ``Yes,'' he continued,
``there is no necessity for my going, and I mean to stay
at home.  Fanny has a great desire to see Sotherton.
I know she wishes it very much.  She has not often a
gratification of the kind, and I am sure, ma'am, you would
be glad to give her the pleasure now?''

``Oh yes! very glad, if your aunt sees no objection.''

Mrs.\ Norris was very ready with the only objection which
could remain---their having positively assured Mrs.\ Rushworth
that Fanny could not go, and the very strange appearance
there would consequently be in taking her, which seemed
to her a difficulty quite impossible to be got over.
It must have the strangest appearance!  It would be
something so very unceremonious, so bordering on disrespect
for Mrs.\ Rushworth, whose own manners were such a pattern
of good-breeding and attention, that she really did not
feel equal to it.  Mrs.\ Norris had no affection for Fanny,
and no wish of procuring her pleasure at any time;
but her opposition to Edmund \emph{now}, arose more from
partiality for her own scheme, because it \emph{was} her own,
than from anything else.  She felt that she had arranged
everything extremely well, and that any alteration must be
for the worse.  When Edmund, therefore, told her in reply,
as he did when she would give him the hearing, that she
need not distress herself on Mrs.\ Rushworth's account,
because he had taken the opportunity, as he walked with
her through the hall, of mentioning Miss Price as one
who would probably be of the party, and had directly
received a very sufficient invitation for his cousin,
Mrs.\ Norris was too much vexed to submit with a very
good grace, and would only say, ``Very well, very well,
just as you chuse, settle it your own way, I am sure I
do not care about it.''

``It seems very odd,'' said Maria, ``that you should be
staying at home instead of Fanny.''

``I am sure she ought to be very much obliged to you,''
added Julia, hastily leaving the room as she spoke,
from a consciousness that she ought to offer to stay at
home herself.

``Fanny will feel quite as grateful as the occasion requires,''
was Edmund's only reply, and the subject dropt.

Fanny's gratitude, when she heard the plan, was, in fact,
much greater than her pleasure.  She felt Edmund's kindness
with all, and more than all, the sensibility which he,
unsuspicious of her fond attachment, could be aware of;
but that he should forego any enjoyment on her account gave
her pain, and her own satisfaction in seeing Sotherton would
be nothing without him.

The next meeting of the two Mansfield families produced
another alteration in the plan, and one that was admitted
with general approbation.  Mrs.\ Grant offered herself as
companion for the day to Lady Bertram in lieu of her son,
and Dr.\ Grant was to join them at dinner.  Lady Bertram
was very well pleased to have it so, and the young ladies
were in spirits again.  Even Edmund was very thankful for an
arrangement which restored him to his share of the party;
and Mrs.\ Norris thought it an excellent plan, and had it
at her tongue's end, and was on the point of proposing it,
when Mrs.\ Grant spoke.

Wednesday was fine, and soon after breakfast the barouche
arrived, Mr.\ Crawford driving his sisters; and as everybody
was ready, there was nothing to be done but for Mrs.\ Grant
to alight and the others to take their places.  The place
of all places, the envied seat, the post of honour,
was unappropriated.  To whose happy lot was it to fall?
While each of the Miss Bertrams were meditating how best,
and with the most appearance of obliging the others,
to secure it, the matter was settled by Mrs.\ Grant's saying,
as she stepped from the carriage, ``As there are five
of you, it will be better that one should sit with Henry;
and as you were saying lately that you wished you
could drive, Julia, I think this will be a good opportunity
for you to take a lesson.''

Happy Julia!  Unhappy Maria!  The former was on the
barouche-box in a moment, the latter took her seat within,
in gloom and mortification; and the carriage drove
off amid the good wishes of the two remaining ladies,
and the barking of Pug in his mistress's arms.

Their road was through a pleasant country; and Fanny,
whose rides had never been extensive, was soon beyond
her knowledge, and was very happy in observing all that
was new, and admiring all that was pretty.  She was not
often invited to join in the conversation of the others,
nor did she desire it.  Her own thoughts and reflections
were habitually her best companions; and, in observing
the appearance of the country, the bearings of the roads,
the difference of soil, the state of the harvest, the cottages,
the cattle, the children, she found entertainment
that could only have been heightened by having Edmund
to speak to of what she felt.  That was the only point
of resemblance between her and the lady who sat by her:
in everything but a value for Edmund, Miss Crawford was
very unlike her.  She had none of Fanny's delicacy of taste,
of mind, of feeling; she saw Nature, inanimate Nature,
with little observation; her attention was all for men
and women, her talents for the light and lively.
In looking back after Edmund, however, when there was
any stretch of road behind them, or when he gained on
them in ascending a considerable hill, they were united,
and a ``there he is'' broke at the same moment from them both,
more than once.

For the first seven miles Miss Bertram had very little
real comfort:  her prospect always ended in Mr.\ Crawford
and her sister sitting side by side, full of conversation
and merriment; and to see only his expressive profile
as he turned with a smile to Julia, or to catch the laugh
of the other, was a perpetual source of irritation,
which her own sense of propriety could but just smooth over.
When Julia looked back, it was with a countenance of delight,
and whenever she spoke to them, it was in the highest spirits:
``her view of the country was charming, she wished they
could all see it,'' etc.; but her only offer of exchange
was addressed to Miss Crawford, as they gained the summit
of a long hill, and was not more inviting than this:
``Here is a fine burst of country.  I wish you had my seat,
but I dare say you will not take it, let me press you ever
so much;'' and Miss Crawford could hardly answer before they
were moving again at a good pace.

When they came within the influence of Sotherton associations,
it was better for Miss Bertram, who might be said to have
two strings to her bow.  She had Rushworth feelings,
and Crawford feelings, and in the vicinity of Sotherton
the former had considerable effect.  Mr.\ Rushworth's
consequence was hers.  She could not tell Miss Crawford
that ``those woods belonged to Sotherton,'' she could not
carelessly observe that ``she believed that it was now
all Mr.\ Rushworth's property on each side of the road,''
without elation of heart; and it was a pleasure to increase
with their approach to the capital freehold mansion,
and ancient manorial residence of the family, with all
its rights of court-leet and court-baron.

``Now we shall have no more rough road, Miss Crawford;
our difficulties are over.  The rest of the way is such
as it ought to be.  Mr.\ Rushworth has made it since he
succeeded to the estate.  Here begins the village.
Those cottages are really a disgrace.  The church spire
is reckoned remarkably handsome.  I am glad the church
is not so close to the great house as often happens in
old places.  The annoyance of the bells must be terrible.
There is the parsonage:  a tidy-looking house, and I
understand the clergyman and his wife are very decent people.
Those are almshouses, built by some of the family.
To the right is the steward's house; he is a very
respectable man.  Now we are coming to the lodge-gates;
but we have nearly a mile through the park still.
It is not ugly, you see, at this end; there is some
fine timber, but the situation of the house is dreadful.
We go down hill to it for half a mile, and it is a pity,
for it would not be an ill-looking place if it had a
better approach.''

Miss Crawford was not slow to admire; she pretty well guessed
Miss Bertram's feelings, and made it a point of honour
to promote her enjoyment to the utmost.  Mrs.\ Norris was
all delight and volubility; and even Fanny had something
to say in admiration, and might be heard with complacency.
Her eye was eagerly taking in everything within her reach;
and after being at some pains to get a view of the house,
and observing that ``it was a sort of building which she
could not look at but with respect,'' she added, ``Now, where
is the avenue?  The house fronts the east, I perceive.
The avenue, therefore, must be at the back of it.
Mr.\ Rushworth talked of the west front.''

``Yes, it is exactly behind the house; begins at a little
distance, and ascends for half a mile to the extremity
of the grounds.  You may see something of it here---%
something of the more distant trees.  It is oak entirely.''

Miss Bertram could now speak with decided information
of what she had known nothing about when Mr.\ Rushworth
had asked her opinion; and her spirits were in as happy
a flutter as vanity and pride could furnish, when they drove
up to the spacious stone steps before the principal entrance.



\chapter{Chapter 9}

\gintro{Mr.\ Rushworth} was at the door to receive his fair lady;
and the whole party were welcomed by him with due attention.
In the drawing-room they were met with equal cordiality
by the mother, and Miss Bertram had all the distinction
with each that she could wish.  After the business
of arriving was over, it was first necessary to eat,
and the doors were thrown open to admit them through one
or two intermediate rooms into the appointed dining-parlour,
where a collation was prepared with abundance and elegance.
Much was said, and much was ate, and all went well.
The particular object of the day was then considered.
How would Mr.\ Crawford like, in what manner would he chuse,
to take a survey of the grounds?  Mr.\ Rushworth mentioned
his curricle.  Mr.\ Crawford suggested the greater desirableness
of some carriage which might convey more than two.
``To be depriving themselves of the advantage of other eyes
and other judgments, might be an evil even beyond the loss
of present pleasure.''

Mrs.\ Rushworth proposed that the chaise should be taken also;
but this was scarcely received as an amendment:  the young
ladies neither smiled nor spoke.  Her next proposition,
of shewing the house to such of them as had not been
there before, was more acceptable, for Miss Bertram was
pleased to have its size displayed, and all were glad
to be doing something.

The whole party rose accordingly, and under Mrs.\ Rushworth's
guidance were shewn through a number of rooms, all lofty,
and many large, and amply furnished in the taste of fifty
years back, with shining floors, solid mahogany, rich damask,
marble, gilding, and carving, each handsome in its way.
Of pictures there were abundance, and some few good,
but the larger part were family portraits, no longer
anything to anybody but Mrs.\ Rushworth, who had been at
great pains to learn all that the housekeeper could teach,
and was now almost equally well qualified to shew the house.
On the present occasion she addressed herself chiefly
to Miss Crawford and Fanny, but there was no comparison
in the willingness of their attention; for Miss Crawford,
who had seen scores of great houses, and cared for none
of them, had only the appearance of civilly listening,
while Fanny, to whom everything was almost as interesting
as it was new, attended with unaffected earnestness to all
that Mrs.\ Rushworth could relate of the family in former times,
its rise and grandeur, regal visits and loyal efforts,
delighted to connect anything with history already known,
or warm her imagination with scenes of the past.

The situation of the house excluded the possibility
of much prospect from any of the rooms; and while Fanny
and some of the others were attending Mrs.\ Rushworth,
Henry Crawford was looking grave and shaking his head
at the windows.  Every room on the west front looked
across a lawn to the beginning of the avenue immediately
beyond tall iron palisades and gates.

Having visited many more rooms than could be supposed to be
of any other use than to contribute to the window-tax, and
find employment for housemaids, ``Now,'' said Mrs.\ Rushworth,
``we are coming to the chapel, which properly we ought
to enter from above, and look down upon; but as we
are quite among friends, I will take you in this way,
if you will excuse me.''

They entered.  Fanny's imagination had prepared her
for something grander than a mere spacious, oblong room,
fitted up for the purpose of devotion:  with nothing more
striking or more solemn than the profusion of mahogany,
and the crimson velvet cushions appearing over the ledge
of the family gallery above.  ``I am disappointed,''
said she, in a low voice, to Edmund.  ``This is not
my idea of a chapel.  There is nothing awful here,
nothing melancholy, nothing grand.  Here are no aisles,
no arches, no inscriptions, no banners.  No banners,
cousin, to be `blown by the night wind of heaven.'
No signs that a `Scottish monarch sleeps below.'\,''

``You forget, Fanny, how lately all this has been built,
and for how confined a purpose, compared with the old
chapels of castles and monasteries.  It was only for
the private use of the family.  They have been buried,
I suppose, in the parish church.  \emph{There} you must look
for the banners and the achievements.''

``It was foolish of me not to think of all that; but I
am disappointed.''

Mrs.\ Rushworth began her relation.  ``This chapel was fitted up
as you see it, in James the Second's time.  Before that period,
as I understand, the pews were only wainscot; and there
is some reason to think that the linings and cushions
of the pulpit and family seat were only purple cloth;
but this is not quite certain.  It is a handsome chapel,
and was formerly in constant use both morning and evening.
Prayers were always read in it by the domestic chaplain,
within the memory of many; but the late Mr.\ Rushworth left
it off.''

``Every generation has its improvements,'' said Miss Crawford,
with a smile, to Edmund.

Mrs.\ Rushworth was gone to repeat her lesson to Mr.\ Crawford;
and Edmund, Fanny, and Miss Crawford remained in a cluster
together.

``It is a pity,'' cried Fanny, ``that the custom should have
been discontinued.  It was a valuable part of former times.
There is something in a chapel and chaplain so much
in character with a great house, with one's ideas of what
such a household should be!  A whole family assembling
regularly for the purpose of prayer is fine!''

``Very fine indeed,'' said Miss Crawford, laughing.  ``It must
do the heads of the family a great deal of good to force
all the poor housemaids and footmen to leave business
and pleasure, and say their prayers here twice a day,
while they are inventing excuses themselves for staying
away.''

``\emph{That} is hardly Fanny's idea of a family assembling,''
said Edmund.  ``If the master and mistress do \emph{not}
attend themselves, there must be more harm than good
in the custom.''

``At any rate, it is safer to leave people to their own
devices on such subjects.  Everybody likes to go their
own way---to chuse their own time and manner of devotion.
The obligation of attendance, the formality, the restraint,
the length of time---altogether it is a formidable thing,
and what nobody likes; and if the good people who used
to kneel and gape in that gallery could have foreseen
that the time would ever come when men and women might lie
another ten minutes in bed, when they woke with a headache,
without danger of reprobation, because chapel was missed,
they would have jumped with joy and envy.  Cannot you
imagine with what unwilling feelings the former belles
of the house of Rushworth did many a time repair to
this chapel?  The young Mrs.\ Eleanors and Mrs.\ Bridgets---%
starched up into seeming piety, but with heads full
of something very different---especially if the poor
chaplain were not worth looking at---and, in those days,
I fancy parsons were very inferior even to what they
are now.''

For a few moments she was unanswered.  Fanny coloured
and looked at Edmund, but felt too angry for speech;
and he needed a little recollection before he could say,
``Your lively mind can hardly be serious even on serious subjects.
You have given us an amusing sketch, and human nature
cannot say it was not so.  We must all feel \emph{at} \emph{times}
the difficulty of fixing our thoughts as we could wish;
but if you are supposing it a frequent thing, that is to say,
a weakness grown into a habit from neglect, what could
be expected from the \emph{private} devotions of such persons?
Do you think the minds which are suffered, which are
indulged in wanderings in a chapel, would be more collected
in a closet?''

``Yes, very likely.  They would have two chances at least
in their favour.  There would be less to distract the
attention from without, and it would not be tried so long.''

``The mind which does not struggle against itself under
\emph{one} circumstance, would find objects to distract it
in the \emph{other}, I believe; and the influence of the place
and of example may often rouse better feelings than are
begun with.  The greater length of the service, however,
I admit to be sometimes too hard a stretch upon the mind.
One wishes it were not so; but I have not yet left
Oxford long enough to forget what chapel prayers are.''

While this was passing, the rest of the party being scattered
about the chapel, Julia called Mr.\ Crawford's attention to
her sister, by saying, ``Do look at Mr.\ Rushworth and Maria,
standing side by side, exactly as if the ceremony were
going to be performed.  Have not they completely the air of it?''

Mr.\ Crawford smiled his acquiescence, and stepping forward
to Maria, said, in a voice which she only could hear,
``I do not like to see Miss Bertram so near the altar.''

Starting, the lady instinctively moved a step or two,
but recovering herself in a moment, affected to laugh,
and asked him, in a tone not much louder, ``If he would give
her away?''

``I am afraid I should do it very awkwardly,'' was his reply,
with a look of meaning.

Julia, joining them at the moment, carried on the joke.

``Upon my word, it is really a pity that it should not
take place directly, if we had but a proper licence,
for here we are altogether, and nothing in the world
could be more snug and pleasant.''  And she talked and
laughed about it with so little caution as to catch the
comprehension of Mr.\ Rushworth and his mother, and expose
her sister to the whispered gallantries of her lover,
while Mrs.\ Rushworth spoke with proper smiles and dignity
of its being a most happy event to her whenever it took place.

``If Edmund were but in orders!'' cried Julia, and running
to where he stood with Miss Crawford and Fanny:
``My dear Edmund, if you were but in orders now, you might
perform the ceremony directly.  How unlucky that you
are not ordained; Mr.\ Rushworth and Maria are quite ready.''

Miss Crawford's countenance, as Julia spoke, might have
amused a disinterested observer.  She looked almost aghast
under the new idea she was receiving.  Fanny pitied her.
``How distressed she will be at what she said just now,''
passed across her mind.

``Ordained!'' said Miss Crawford; ``what, are you to be
a clergyman?''

``Yes; I shall take orders soon after my father's return---%
probably at Christmas.''

Miss Crawford, rallying her spirits, and recovering
her complexion, replied only, ``If I had known this before,
I would have spoken of the cloth with more respect,''
and turned the subject.

The chapel was soon afterwards left to the silence and stillness
which reigned in it, with few interruptions, throughout the year.
Miss Bertram, displeased with her sister, led the way,
and all seemed to feel that they had been there long enough.

The lower part of the house had been now entirely shewn,
and Mrs.\ Rushworth, never weary in the cause, would have
proceeded towards the principal staircase, and taken
them through all the rooms above, if her son had not
interposed with a doubt of there being time enough.
``For if,'' said he, with the sort of self-evident proposition
which many a clearer head does not always avoid, ``we are
\emph{too} long going over the house, we shall not have time
for what is to be done out of doors.  It is past two,
and we are to dine at five.''

Mrs.\ Rushworth submitted; and the question of surveying
the grounds, with the who and the how, was likely to be more
fully agitated, and Mrs.\ Norris was beginning to arrange
by what junction of carriages and horses most could be done,
when the young people, meeting with an outward door,
temptingly open on a flight of steps which led immediately
to turf and shrubs, and all the sweets of pleasure-grounds,
as by one impulse, one wish for air and liberty, all walked out.

``Suppose we turn down here for the present,'' said Mrs.\ Rushworth,
civilly taking the hint and following them.  ``Here are the
greatest number of our plants, and here are the curious pheasants.''

``Query,'' said Mr.\ Crawford, looking round him,
``whether we may not find something to employ us here
before we go farther?  I see walls of great promise.
Mr.\ Rushworth, shall we summon a council on this lawn?''

``James,'' said Mrs.\ Rushworth to her son, ``I believe
the wilderness will be new to all the party.  The Miss
Bertrams have never seen the wilderness yet.''

No objection was made, but for some time there seemed
no inclination to move in any plan, or to any distance.
All were attracted at first by the plants or the pheasants,
and all dispersed about in happy independence.
Mr.\ Crawford was the first to move forward to examine
the capabilities of that end of the house.  The lawn,
bounded on each side by a high wall, contained beyond
the first planted area a bowling-green, and beyond
the bowling-green a long terrace walk, backed by iron
palisades, and commanding a view over them into the tops
of the trees of the wilderness immediately adjoining.
It was a good spot for fault-finding. Mr.\ Crawford was soon
followed by Miss Bertram and Mr.\ Rushworth; and when,
after a little time, the others began to form into parties,
these three were found in busy consultation on the terrace
by Edmund, Miss Crawford, and Fanny, who seemed as naturally
to unite, and who, after a short participation of their
regrets and difficulties, left them and walked on.
The remaining three, Mrs.\ Rushworth, Mrs.\ Norris,
and Julia, were still far behind; for Julia, whose happy
star no longer prevailed, was obliged to keep by the side
of Mrs.\ Rushworth, and restrain her impatient feet to that
lady's slow pace, while her aunt, having fallen in with
the housekeeper, who was come out to feed the pheasants,
was lingering behind in gossip with her.  Poor Julia,
the only one out of the nine not tolerably satisfied
with their lot, was now in a state of complete penance,
and as different from the Julia of the barouche-box
as could well be imagined.  The politeness which she had
been brought up to practise as a duty made it impossible
for her to escape; while the want of that higher species
of self-command, that just consideration of others,
that knowledge of her own heart, that principle of right,
which had not formed any essential part of her education,
made her miserable under it.

``This is insufferably hot,'' said Miss Crawford, when they
had taken one turn on the terrace, and were drawing
a second time to the door in the middle which opened to
the wilderness.  ``Shall any of us object to being comfortable?
Here is a nice little wood, if one can but get into it.
What happiness if the door should not be locked! but of
course it is; for in these great places the gardeners
are the only people who can go where they like.''

The door, however, proved not to be locked, and they were
all agreed in turning joyfully through it, and leaving
the unmitigated glare of day behind.  A considerable
flight of steps landed them in the wilderness, which was
a planted wood of about two acres, and though chiefly
of larch and laurel, and beech cut down, and though laid
out with too much regularity, was darkness and shade,
and natural beauty, compared with the bowling-green
and the terrace.  They all felt the refreshment of it,
and for some time could only walk and admire.  At length,
after a short pause, Miss Crawford began with, ``So you
are to be a clergyman, Mr.\ Bertram.  This is rather
a surprise to me.''

``Why should it surprise you?  You must suppose me designed
for some profession, and might perceive that I am neither
a lawyer, nor a soldier, nor a sailor.''

``Very true; but, in short, it had not occurred to me.
And you know there is generally an uncle or a grandfather
to leave a fortune to the second son.''

``A very praiseworthy practice,'' said Edmund,
``but not quite universal.  I am one of the exceptions,
and \emph{being} one, must do something for myself.''

``But why are you to be a clergyman?  I thought \emph{that}
was always the lot of the youngest, where there were
many to chuse before him.''

``Do you think the church itself never chosen, then?''

``\emph{Never} is a black word.  But yes, in the \emph{never}
of conversation, which means \emph{not} \emph{very} \emph{often},
I do think it.  For what is to be done in the church?
Men love to distinguish themselves, and in either of the other
lines distinction may be gained, but not in the church.
A clergyman is nothing.''

``The \emph{nothing} of conversation has its gradations, I hope,
as well as the \emph{never}.  A clergyman cannot be high in
state or fashion.  He must not head mobs, or set the ton
in dress.  But I cannot call that situation nothing which
has the charge of all that is of the first importance
to mankind, individually or collectively considered,
temporally and eternally, which has the guardianship
of religion and morals, and consequently of the manners
which result from their influence.  No one here can call
the \emph{office} nothing.  If the man who holds it is so,
it is by the neglect of his duty, by foregoing its
just importance, and stepping out of his place to appear
what he ought not to appear.''

``\emph{You} assign greater consequence to the clergyman than one
has been used to hear given, or than I can quite comprehend.
One does not see much of this influence and importance
in society, and how can it be acquired where they are
so seldom seen themselves?  How can two sermons a week,
even supposing them worth hearing, supposing the preacher
to have the sense to prefer Blair's to his own, do all
that you speak of? govern the conduct and fashion the
manners of a large congregation for the rest of the week?
One scarcely sees a clergyman out of his pulpit.''

``\emph{You} are speaking of London, \emph{I} am speaking of the
nation at large.''

``The metropolis, I imagine, is a pretty fair sample
of the rest.''

``Not, I should hope, of the proportion of virtue to vice
throughout the kingdom.  We do not look in great cities
for our best morality.  It is not there that respectable
people of any denomination can do most good; and it
certainly is not there that the influence of the clergy can
be most felt.  A fine preacher is followed and admired;
but it is not in fine preaching only that a good clergyman
will be useful in his parish and his neighbourhood,
where the parish and neighbourhood are of a size capable
of knowing his private character, and observing his
general conduct, which in London can rarely be the case.
The clergy are lost there in the crowds of their parishioners.
They are known to the largest part only as preachers.
And with regard to their influencing public manners,
Miss Crawford must not misunderstand me, or suppose I mean
to call them the arbiters of good-breeding, the regulators
of refinement and courtesy, the masters of the ceremonies
of life.  The \emph{manners} I speak of might rather be
called \emph{conduct}, perhaps, the result of good principles;
the effect, in short, of those doctrines which it
is their duty to teach and recommend; and it will,
I believe, be everywhere found, that as the clergy are,
or are not what they ought to be, so are the rest of
the nation.''

``Certainly,'' said Fanny, with gentle earnestness.

``There,'' cried Miss Crawford, ``you have quite convinced
Miss Price already.''

``I wish I could convince Miss Crawford too.''

``I do not think you ever will,'' said she, with an arch smile;
``I am just as much surprised now as I was at first
that you should intend to take orders.  You really are
fit for something better.  Come, do change your mind.
It is not too late.  Go into the law.''

``Go into the law!  With as much ease as I was told to go
into this wilderness.''

``Now you are going to say something about law being
the worst wilderness of the two, but I forestall you;
remember, I have forestalled you.''

``You need not hurry when the object is only to prevent
my saying a \emph{bon} \emph{mot}, for there is not the least wit in
my nature.  I am a very matter-of-fact, plain-spoken being,
and may blunder on the borders of a repartee for half
an hour together without striking it out.''

A general silence succeeded.  Each was thoughtful.
Fanny made the first interruption by saying, ``I wonder
that I should be tired with only walking in this sweet wood;
but the next time we come to a seat, if it is not disagreeable
to you, I should be glad to sit down for a little while.''

``My dear Fanny,'' cried Edmund, immediately drawing her arm
within his, ``how thoughtless I have been!  I hope you
are not very tired.  Perhaps,'' turning to Miss Crawford,
``my other companion may do me the honour of taking an arm.''

``Thank you, but I am not at all tired.''  She took it,
however, as she spoke, and the gratification of having
her do so, of feeling such a connexion for the first time,
made him a little forgetful of Fanny.  ``You scarcely
touch me,'' said he.  ``You do not make me of any use.
What a difference in the weight of a woman's arm from
that of a man!  At Oxford I have been a good deal used
to have a man lean on me for the length of a street,
and you are only a fly in the comparison.''

``I am really not tired, which I almost wonder at;
for we must have walked at least a mile in this wood.
Do not you think we have?''

``Not half a mile,'' was his sturdy answer; for he was not yet
so much in love as to measure distance, or reckon time,
with feminine lawlessness.

``Oh! you do not consider how much we have wound about.
We have taken such a very serpentine course, and the wood
itself must be half a mile long in a straight line,
for we have never seen the end of it yet since we left
the first great path.''

``But if you remember, before we left that first great path,
we saw directly to the end of it.  We looked down the
whole vista, and saw it closed by iron gates, and it
could not have been more than a furlong in length.''

``Oh!  I know nothing of your furlongs, but I am sure
it is a very long wood, and that we have been winding
in and out ever since we came into it; and therefore,
when I say that we have walked a mile in it, I must speak
within compass.''

``We have been exactly a quarter of an hour here,''
said Edmund, taking out his watch.  ``Do you think we
are walking four miles an hour?''

``Oh! do not attack me with your watch.  A watch is always
too fast or too slow.  I cannot be dictated to by a watch.''

A few steps farther brought them out at the bottom of the
very walk they had been talking of; and standing back,
well shaded and sheltered, and looking over a ha-ha into
the park, was a comfortable-sized bench, on which they
all sat down.

``I am afraid you are very tired, Fanny,'' said Edmund,
observing her; ``why would not you speak sooner?  This will be
a bad day's amusement for you if you are to be knocked up.
Every sort of exercise fatigues her so soon, Miss Crawford,
except riding.''

``How abominable in you, then, to let me engross her horse
as I did all last week!  I am ashamed of you and of myself,
but it shall never happen again.''

``\emph{Your} attentiveness and consideration makes me more
sensible of my own neglect.  Fanny's interest seems
in safer hands with you than with me.''

``That she should be tired now, however, gives me no surprise;
for there is nothing in the course of one's duties
so fatiguing as what we have been doing this morning:
seeing a great house, dawdling from one room to another,
straining one's eyes and one's attention, hearing what one
does not understand, admiring what one does not care for.
It is generally allowed to be the greatest bore in the world,
and Miss Price has found it so, though she did not
know it.''

``I shall soon be rested,'' said Fanny; ``to sit
in the shade on a fine day, and look upon verdure,
is the most perfect refreshment.''

After sitting a little while Miss Crawford was up again.
``I must move,'' said she; ``resting fatigues me.
I have looked across the ha-ha till I am weary.  I must
go and look through that iron gate at the same view,
without being able to see it so well.''

Edmund left the seat likewise.  ``Now, Miss Crawford,
if you will look up the walk, you will convince yourself
that it cannot be half a mile long, or half half a mile.''

``It is an immense distance,'' said she; ``I see \emph{that}
with a glance.''

He still reasoned with her, but in vain.  She would
not calculate, she would not compare.  She would only
smile and assert.  The greatest degree of rational
consistency could not have been more engaging, and they
talked with mutual satisfaction.  At last it was agreed
that they should endeavour to determine the dimensions
of the wood by walking a little more about it.  They would
go to one end of it, in the line they were then in---%
for there was a straight green walk along the bottom
by the side of the ha-ha---and perhaps turn a little way
in some other direction, if it seemed likely to assist them,
and be back in a few minutes.  Fanny said she was rested,
and would have moved too, but this was not suffered.
Edmund urged her remaining where she was with an
earnestness which she could not resist, and she was left
on the bench to think with pleasure of her cousin's care,
but with great regret that she was not stronger.
She watched them till they had turned the corner,
and listened till all sound of them had ceased.



\chapter{Chapter 10}

\gintro{A quarter of an hour,} twenty minutes, passed away,
and Fanny was still thinking of Edmund, Miss Crawford,
and herself, without interruption from any one.  She began
to be surprised at being left so long, and to listen
with an anxious desire of hearing their steps and their
voices again.  She listened, and at length she heard;
she heard voices and feet approaching; but she had just
satisfied herself that it was not those she wanted,
when Miss Bertram, Mr.\ Rushworth, and Mr.\ Crawford issued
from the same path which she had trod herself, and were
before her.

``Miss Price all alone'' and ``My dear Fanny, how comes this?''
were the first salutations.  She told her story.
``Poor dear Fanny,'' cried her cousin, ``how ill you have been
used by them!  You had better have staid with us.''

Then seating herself with a gentleman on each side,
she resumed the conversation which had engaged them before,
and discussed the possibility of improvements with
much animation.  Nothing was fixed on; but Henry Crawford
was full of ideas and projects, and, generally speaking,
whatever he proposed was immediately approved, first by her,
and then by Mr.\ Rushworth, whose principal business
seemed to be to hear the others, and who scarcely risked
an original thought of his own beyond a wish that they
had seen his friend Smith's place.

After some minutes spent in this way, Miss Bertram,
observing the iron gate, expressed a wish of passing
through it into the park, that their views and their
plans might be more comprehensive.  It was the very thing
of all others to be wished, it was the best, it was
the only way of proceeding with any advantage, in Henry
Crawford's opinion; and he directly saw a knoll not half
a mile off, which would give them exactly the requisite
command of the house.  Go therefore they must to that knoll,
and through that gate; but the gate was locked.
Mr.\ Rushworth wished he had brought the key; he had been
very near thinking whether he should not bring the key;
he was determined he would never come without the key again;
but still this did not remove the present evil.  They could
not get through; and as Miss Bertram's inclination for so
doing did by no means lessen, it ended in Mr.\ Rushworth's
declaring outright that he would go and fetch the key.
He set off accordingly.

``It is undoubtedly the best thing we can do now, as we
are so far from the house already,'' said Mr.\ Crawford,
when he was gone.

``Yes, there is nothing else to be done.  But now, sincerely,
do not you find the place altogether worse than you expected?''

``No, indeed, far otherwise.  I find it better, grander, more
complete in its style, though that style may not be the best.
And to tell you the truth,'' speaking rather lower, ``I do not
think that \emph{I} shall ever see Sotherton again with so much
pleasure as I do now.  Another summer will hardly improve it to
me.''

After a moment's embarrassment the lady replied, ``You are
too much a man of the world not to see with the eyes
of the world.  If other people think Sotherton improved,
I have no doubt that you will.''

``I am afraid I am not quite so much the man of the world
as might be good for me in some points.  My feelings
are not quite so evanescent, nor my memory of the past
under such easy dominion as one finds to be the case
with men of the world.''

This was followed by a short silence.  Miss Bertram
began again.  ``You seemed to enjoy your drive here very much
this morning.  I was glad to see you so well entertained.
You and Julia were laughing the whole way.''

``Were we?  Yes, I believe we were; but I have not
the least recollection at what.  Oh!  I believe
I was relating to her some ridiculous stories
of an old Irish groom of my uncle's. Your sister loves to laugh.''

``You think her more light-hearted than I am?''

``More easily amused,'' he replied; ``consequently, you know,''
smiling, ``better company.  I could not have hoped
to entertain you with Irish anecdotes during a ten miles' drive.''

``Naturally, I believe, I am as lively as Julia, but I
have more to think of now.''

``You have, undoubtedly; and there are situations in
which very high spirits would denote insensibility.
Your prospects, however, are too fair to justify want
of spirits.  You have a very smiling scene before you.''

``Do you mean literally or figuratively?  Literally,
I conclude.  Yes, certainly, the sun shines, and the park
looks very cheerful.  But unluckily that iron gate,
that ha-ha, give me a feeling of restraint and hardship.
`I cannot get out,' as the starling said.''  As she spoke,
and it was with expression, she walked to the gate:
he followed her.  ``Mr.\ Rushworth is so long fetching
this key!''

``And for the world you would not get out without the key
and without Mr.\ Rushworth's authority and protection,
or I think you might with little difficulty pass round
the edge of the gate, here, with my assistance; I think it
might be done, if you really wished to be more at large,
and could allow yourself to think it not prohibited.''

``Prohibited! nonsense!  I certainly can get out that way,
and I will.  Mr.\ Rushworth will be here in a moment,
you know; we shall not be out of sight.''

``Or if we are, Miss Price will be so good as to tell him
that he will find us near that knoll:  the grove of oak
on the knoll.''

Fanny, feeling all this to be wrong, could not help
making an effort to prevent it.  ``You will hurt yourself,
Miss Bertram,'' she cried; ``you will certainly hurt
yourself against those spikes; you will tear your gown;
you will be in danger of slipping into the ha-ha. You had
better not go.''

Her cousin was safe on the other side while these words
were spoken, and, smiling with all the good-humour
of success, she said, ``Thank you, my dear Fanny,
but I and my gown are alive and well, and so good-bye.''

Fanny was again left to her solitude, and with no increase
of pleasant feelings, for she was sorry for almost all
that she had seen and heard, astonished at Miss Bertram,
and angry with Mr.\ Crawford.  By taking a circuitous
route, and, as it appeared to her, very unreasonable
direction to the knoll, they were soon beyond her eye;
and for some minutes longer she remained without sight
or sound of any companion.  She seemed to have the little
wood all to herself.  She could almost have thought
that Edmund and Miss Crawford had left it, but that
it was impossible for Edmund to forget her so entirely.

She was again roused from disagreeable musings by sudden footsteps:
somebody was coming at a quick pace down the principal walk.
She expected Mr.\ Rushworth, but it was Julia, who,
hot and out of breath, and with a look of disappointment,
cried out on seeing her, ``Heyday!  Where are the others?
I thought Maria and Mr.\ Crawford were with you.''

Fanny explained.

``A pretty trick, upon my word!  I cannot see them anywhere,''
looking eagerly into the park.  ``But they cannot be very
far off, and I think I am equal to as much as Maria,
even without help.''

``But, Julia, Mr.\ Rushworth will be here in a moment
with the key.  Do wait for Mr.\ Rushworth.''

``Not I, indeed.  I have had enough of the family for
one morning.  Why, child, I have but this moment escaped from
his horrible mother.  Such a penance as I have been enduring,
while you were sitting here so composed and so happy!
It might have been as well, perhaps, if you had been in
my place, but you always contrive to keep out of these scrapes.''

This was a most unjust reflection, but Fanny could allow
for it, and let it pass:  Julia was vexed, and her
temper was hasty; but she felt that it would not last,
and therefore, taking no notice, only asked her if she
had not seen Mr.\ Rushworth.

``Yes, yes, we saw him.  He was posting away as if upon
life and death, and could but just spare time to tell us
his errand, and where you all were.''

``It is a pity he should have so much trouble for nothing.''

``\emph{That} is Miss Maria's concern.  I am not obliged
to punish myself for \emph{her} sins.  The mother I could
not avoid, as long as my tiresome aunt was dancing about
with the housekeeper, but the son I \emph{can} get away from.''

And she immediately scrambled across the fence,
and walked away, not attending to Fanny's last question of
whether she had seen anything of Miss Crawford and Edmund.
The sort of dread in which Fanny now sat of seeing
Mr.\ Rushworth prevented her thinking so much of their
continued absence, however, as she might have done.
She felt that he had been very ill-used, and was quite
unhappy in having to communicate what had passed.
He joined her within five minutes after Julia's exit;
and though she made the best of the story, he was evidently
mortified and displeased in no common degree.  At first
he scarcely said anything; his looks only expressed his
extreme surprise and vexation, and he walked to the gate
and stood there, without seeming to know what to do.

``They desired me to stay---my cousin Maria charged me to say
that you would find them at that knoll, or thereabouts.''

``I do not believe I shall go any farther,'' said he sullenly;
``I see nothing of them.  By the time I get to the knoll they
may be gone somewhere else.  I have had walking enough.''

And he sat down with a most gloomy countenance by Fanny.

``I am very sorry,'' said she; ``it is very unlucky.''  And she
longed to be able to say something more to the purpose.

After an interval of silence, ``I think they might as well
have staid for me,'' said he.

``Miss Bertram thought you would follow her.''

``I should not have had to follow her if she had staid.''

This could not be denied, and Fanny was silenced.
After another pause, he went on---``Pray, Miss Price,
are you such a great admirer of this Mr.\ Crawford as some
people are?  For my part, I can see nothing in him.''

``I do not think him at all handsome.''

``Handsome!  Nobody can call such an undersized man handsome.
He is not five foot nine.  I should not wonder if he is not more
than five foot eight.  I think he is an ill-looking fellow.
In my opinion, these Crawfords are no addition at all.
We did very well without them.''

A small sigh escaped Fanny here, and she did not know
how to contradict him.

``If I had made any difficulty about fetching the key,
there might have been some excuse, but I went the very
moment she said she wanted it.''

``Nothing could be more obliging than your manner, I am sure,
and I dare say you walked as fast as you could; but still
it is some distance, you know, from this spot to the house,
quite into the house; and when people are waiting,
they are bad judges of time, and every half minute seems
like five.''

He got up and walked to the gate again, and ``wished he
had had the key about him at the time.''  Fanny thought she
discerned in his standing there an indication of relenting,
which encouraged her to another attempt, and she said,
therefore, ``It is a pity you should not join them.
They expected to have a better view of the house from
that part of the park, and will be thinking how it
may be improved; and nothing of that sort, you know,
can be settled without you.''

She found herself more successful in sending away than
in retaining a companion.  Mr.\ Rushworth was worked on.
``Well,'' said he, ``if you really think I had better go:
it would be foolish to bring the key for nothing.''
And letting himself out, he walked off without farther
ceremony.

Fanny's thoughts were now all engrossed by the two who
had left her so long ago, and getting quite impatient,
she resolved to go in search of them.  She followed
their steps along the bottom walk, and had just turned
up into another, when the voice and the laugh of Miss
Crawford once more caught her ear; the sound approached,
and a few more windings brought them before her.
They were just returned into the wilderness from the park,
to which a sidegate, not fastened, had tempted them very
soon after their leaving her, and they had been across
a portion of the park into the very avenue which Fanny
had been hoping the whole morning to reach at last,
and had been sitting down under one of the trees.
This was their history.  It was evident that they had been
spending their time pleasantly, and were not aware of the
length of their absence.  Fanny's best consolation was
in being assured that Edmund had wished for her very much,
and that he should certainly have come back for her,
had she not been tired already; but this was not quite
sufficient to do away with the pain of having been left
a whole hour, when he had talked of only a few minutes,
nor to banish the sort of curiosity she felt to know
what they had been conversing about all that time;
and the result of the whole was to her disappointment
and depression, as they prepared by general agreement to
return to the house.

On reaching the bottom of the steps to the terrace,
Mrs.\ Rushworth and Mrs.\ Norris presented themselves
at the top, just ready for the wilderness, at the end
of an hour and a half from their leaving the house.
Mrs.\ Norris had been too well employed to move faster.
Whatever cross-accidents had occurred to intercept the pleasures
of her nieces, she had found a morning of complete enjoyment;
for the housekeeper, after a great many courtesies on
the subject of pheasants, had taken her to the dairy,
told her all about their cows, and given her the receipt
for a famous cream cheese; and since Julia's leaving them
they had been met by the gardener, with whom she had made
a most satisfactory acquaintance, for she had set him
right as to his grandson's illness, convinced him that it
was an ague, and promised him a charm for it; and he,
in return, had shewn her all his choicest nursery of plants,
and actually presented her with a very curious specimen
of heath.

On this \emph{rencontre} they all returned to the house together,
there to lounge away the time as they could with sofas,
and chit-chat, and Quarterly Reviews, till the return
of the others, and the arrival of dinner.  It was late
before the Miss Bertrams and the two gentlemen came in,
and their ramble did not appear to have been more than
partially agreeable, or at all productive of anything
useful with regard to the object of the day.  By their
own accounts they had been all walking after each other,
and the junction which had taken place at last seemed,
to Fanny's observation, to have been as much too late
for re-establishing harmony, as it confessedly had
been for determining on any alteration.  She felt,
as she looked at Julia and Mr.\ Rushworth, that hers
was not the only dissatisfied bosom amongst them:
there was gloom on the face of each.  Mr.\ Crawford
and Miss Bertram were much more gay, and she thought
that he was taking particular pains, during dinner,
to do away any little resentment of the other two,
and restore general good-humour.

Dinner was soon followed by tea and coffee, a ten miles'
drive home allowed no waste of hours; and from the time
of their sitting down to table, it was a quick succession
of busy nothings till the carriage came to the door,
and Mrs.\ Norris, having fidgeted about, and obtained a
few pheasants' eggs and a cream cheese from the housekeeper,
and made abundance of civil speeches to Mrs.\ Rushworth,
was ready to lead the way.  At the same moment Mr.\ Crawford,
approaching Julia, said, ``I hope I am not to lose
my companion, unless she is afraid of the evening air
in so exposed a seat.''  The request had not been foreseen,
but was very graciously received, and Julia's day was
likely to end almost as well as it began.  Miss Bertram
had made up her mind to something different, and was a
little disappointed; but her conviction of being really
the one preferred comforted her under it, and enabled her
to receive Mr.\ Rushworth's parting attentions as she ought.
He was certainly better pleased to hand her into
the barouche than to assist her in ascending the box,
and his complacency seemed confirmed by the arrangement.

``Well, Fanny, this has been a fine day for you, upon my word,''
said Mrs.\ Norris, as they drove through the park.
``Nothing but pleasure from beginning to end!  I am sure
you ought to be very much obliged to your aunt Bertram
and me for contriving to let you go.  A pretty good day's
amusement you have had!''

Maria was just discontented enough to say directly, ``I think
\emph{you} have done pretty well yourself, ma'am. Your lap seems
full of good things, and here is a basket of something
between us which has been knocking my elbow unmercifully.''

``My dear, it is only a beautiful little heath,
which that nice old gardener would make me take; but if
it is in your way, I will have it in my lap directly.
There, Fanny, you shall carry that parcel for me;
take great care of it:  do not let it fall; it is a
cream cheese, just like the excellent one we had at dinner.
Nothing would satisfy that good old Mrs.\ Whitaker,
but my taking one of the cheeses.  I stood out as long
as I could, till the tears almost came into her eyes,
and I knew it was just the sort that my sister would
be delighted with.  That Mrs.\ Whitaker is a treasure!
She was quite shocked when I asked her whether wine was allowed
at the second table, and she has turned away two housemaids
for wearing white gowns.  Take care of the cheese, Fanny.
Now I can manage the other parcel and the basket very well.''

``What else have you been spunging?'' said Maria,
half-pleased that Sotherton should be so complimented.

``Spunging, my dear!  It is nothing but four of those
beautiful pheasants' eggs, which Mrs.\ Whitaker would
quite force upon me:  she would not take a denial.
She said it must be such an amusement to me, as she
understood I lived quite alone, to have a few living
creatures of that sort; and so to be sure it will.
I shall get the dairymaid to set them under the first
spare hen, and if they come to good I can have them moved
to my own house and borrow a coop; and it will be a great
delight to me in my lonely hours to attend to them.
And if I have good luck, your mother shall have some.''

It was a beautiful evening, mild and still, and the
drive was as pleasant as the serenity of Nature
could make it; but when Mrs.\ Norris ceased speaking,
it was altogether a silent drive to those within.
Their spirits were in general exhausted; and to determine
whether the day had afforded most pleasure or pain,
might occupy the meditations of almost all.



\chapter{Chapter 11}

\gintro{The day at Sotherton,} with all its imperfections,
afforded the Miss Bertrams much more agreeable feelings
than were derived from the letters from Antigua,
which soon afterwards reached Mansfield.  It was much
pleasanter to think of Henry Crawford than of their father;
and to think of their father in England again within
a certain period, which these letters obliged them to do,
was a most unwelcome exercise.

November was the black month fixed for his return.
Sir Thomas wrote of it with as much decision as experience
and anxiety could authorise.  His business was so nearly
concluded as to justify him in proposing to take his
passage in the September packet, and he consequently
looked forward with the hope of being with his beloved
family again early in November.

Maria was more to be pitied than Julia; for to her the
father brought a husband, and the return of the friend most
solicitous for her happiness would unite her to the lover,
on whom she had chosen that happiness should depend.
It was a gloomy prospect, and all she could do was to
throw a mist over it, and hope when the mist cleared
away she should see something else.  It would hardly
be \emph{early} in November, there were generally delays,
a bad passage or \emph{something}; that favouring \emph{something}
which everybody who shuts their eyes while they look,
or their understandings while they reason, feels the
comfort of.  It would probably be the middle of November
at least; the middle of November was three months off.
Three months comprised thirteen weeks.  Much might happen
in thirteen weeks.

Sir Thomas would have been deeply mortified by a suspicion
of half that his daughters felt on the subject of his return,
and would hardly have found consolation in a knowledge of the
interest it excited in the breast of another young lady.
Miss Crawford, on walking up with her brother to spend
the evening at Mansfield Park, heard the good news;
and though seeming to have no concern in the affair
beyond politeness, and to have vented all her feelings
in a quiet congratulation, heard it with an attention
not so easily satisfied.  Mrs.\ Norris gave the particulars
of the letters, and the subject was dropt; but after tea,
as Miss Crawford was standing at an open window with
Edmund and Fanny looking out on a twilight scene,
while the Miss Bertrams, Mr.\ Rushworth, and Henry Crawford
were all busy with candles at the pianoforte, she suddenly
revived it by turning round towards the group, and saying,
``How happy Mr.\ Rushworth looks!  He is thinking of November.''

Edmund looked round at Mr.\ Rushworth too, but had nothing
to say.

``Your father's return will be a very interesting event.''

``It will, indeed, after such an absence; an absence
not only long, but including so many dangers.''

``It will be the forerunner also of other interesting events:
your sister's marriage, and your taking orders.''

``Yes.''

``Don't be affronted,'' said she, laughing, ``but it does
put me in mind of some of the old heathen heroes, who,
after performing great exploits in a foreign land,
offered sacrifices to the gods on their safe return.''

``There is no sacrifice in the case,'' replied Edmund,
with a serious smile, and glancing at the pianoforte again;
``it is entirely her own doing.''

``Oh yes I know it is.  I was merely joking.  She has
done no more than what every young woman would do;
and I have no doubt of her being extremely happy.
My other sacrifice, of course, you do not understand.''

``My taking orders, I assure you, is quite as voluntary
as Maria's marrying.''

``It is fortunate that your inclination and your father's
convenience should accord so well.  There is a very good
living kept for you, I understand, hereabouts.''

``Which you suppose has biassed me?''

``But \emph{that} I am sure it has not,'' cried Fanny.

``Thank you for your good word, Fanny, but it is more than
I would affirm myself.  On the contrary, the knowing
that there was such a provision for me probably did
bias me.  Nor can I think it wrong that it should.
There was no natural disinclination to be overcome,
and I see no reason why a man should make a worse clergyman
for knowing that he will have a competence early in life.
I was in safe hands.  I hope I should not have been
influenced myself in a wrong way, and I am sure my father
was too conscientious to have allowed it.  I have no doubt
that I was biased, but I think it was blamelessly.''

``It is the same sort of thing,'' said Fanny, after a
short pause, ``as for the son of an admiral to go into
the navy, or the son of a general to be in the army,
and nobody sees anything wrong in that.  Nobody wonders
that they should prefer the line where their friends can
serve them best, or suspects them to be less in earnest
in it than they appear.''

``No, my dear Miss Price, and for reasons good.  The profession,
either navy or army, is its own justification.  It has
everything in its favour:  heroism, danger, bustle, fashion.
Soldiers and sailors are always acceptable in society.
Nobody can wonder that men are soldiers and sailors.''

``But the motives of a man who takes orders with the certainty
of preferment may be fairly suspected, you think?''
said Edmund.  ``To be justified in your eyes, he must
do it in the most complete uncertainty of any provision.''

``What! take orders without a living!  No; that is
madness indeed; absolute madness.''

``Shall I ask you how the church is to be filled, if a man
is neither to take orders with a living nor without?
No; for you certainly would not know what to say.
But I must beg some advantage to the clergyman from
your own argument.  As he cannot be influenced by those
feelings which you rank highly as temptation and reward
to the soldier and sailor in their choice of a profession,
as heroism, and noise, and fashion, are all against him,
he ought to be less liable to the suspicion of wanting
sincerity or good intentions in the choice of his.''

``Oh! no doubt he is very sincere in preferring an income
ready made, to the trouble of working for one; and has
the best intentions of doing nothing all the rest of his
days but eat, drink, and grow fat.  It is indolence,
Mr.\ Bertram, indeed.  Indolence and love of ease; a want
of all laudable ambition, of taste for good company,
or of inclination to take the trouble of being agreeable,
which make men clergymen.  A clergyman has nothing
to do but be slovenly and selfish---read the newspaper,
watch the weather, and quarrel with his wife.  His curate
does all the work, and the business of his own life is
to dine.''

``There are such clergymen, no doubt, but I think they
are not so common as to justify Miss Crawford in esteeming
it their general character.  I suspect that in this
comprehensive and (may I say) commonplace censure, you are
not judging from yourself, but from prejudiced persons,
whose opinions you have been in the habit of hearing.
It is impossible that your own observation can have given
you much knowledge of the clergy.  You can have been
personally acquainted with very few of a set of men you
condemn so conclusively.  You are speaking what you have
been told at your uncle's table.''

``I speak what appears to me the general opinion;
and where an opinion is general, it is usually correct.
Though \emph{I} have not seen much of the domestic lives
of clergymen, it is seen by too many to leave any deficiency
of information.''

``Where any one body of educated men, of whatever denomination,
are condemned indiscriminately, there must be a deficiency
of information, or (smiling) of something else.
Your uncle, and his brother admirals, perhaps knew little
of clergymen beyond the chaplains whom, good or bad,
they were always wishing away.''

``Poor William!  He has met with great kindness from
the chaplain of the Antwerp,'' was a tender apostrophe
of Fanny's, very much to the purpose of her own feelings
if not of the conversation.

``I have been so little addicted to take my opinions from
my uncle,'' said Miss Crawford, ``that I can hardly suppose---%
and since you push me so hard, I must observe, that I am
not entirely without the means of seeing what clergymen are,
being at this present time the guest of my own brother,
Dr.\ Grant.  And though Dr.\ Grant is most kind and obliging
to me, and though he is really a gentleman, and, I dare say,
a good scholar and clever, and often preaches good sermons,
and is very respectable, \emph{I} see him to be an indolent,
selfish \emph{bon} \emph{vivant}, who must have his palate consulted
in everything; who will not stir a finger for the convenience
of any one; and who, moreover, if the cook makes a blunder,
is out of humour with his excellent wife.  To own the truth,
Henry and I were partly driven out this very evening
by a disappointment about a green goose, which he could
not get the better of.  My poor sister was forced to stay
and bear it.''

``I do not wonder at your disapprobation, upon my word.
It is a great defect of temper, made worse by a very faulty
habit of self-indulgence; and to see your sister suffering
from it must be exceedingly painful to such feelings
as yours.  Fanny, it goes against us.  We cannot attempt
to defend Dr.\ Grant.''

``No,'' replied Fanny, ``but we need not give up his profession
for all that; because, whatever profession Dr.\ Grant
had chosen, he would have taken a---not a good temper into it;
and as he must, either in the navy or army, have had a
great many more people under his command than he has now,
I think more would have been made unhappy by him as a
sailor or soldier than as a clergyman.  Besides, I cannot
but suppose that whatever there may be to wish otherwise
in Dr.\ Grant would have been in a greater danger of
becoming worse in a more active and worldly profession,
where he would have had less time and obligation---%
where he might have escaped that knowledge of himself,
the \emph{frequency}, at least, of that knowledge which it
is impossible he should escape as he is now.  A man---%
a sensible man like Dr.\ Grant, cannot be in the habit
of teaching others their duty every week, cannot go
to church twice every Sunday, and preach such very good
sermons in so good a manner as he does, without being
the better for it himself.  It must make him think;
and I have no doubt that he oftener endeavours to restrain
himself than he would if he had been anything but a clergyman.''

``We cannot prove to the contrary, to be sure; but I wish
you a better fate, Miss Price, than to be the wife of a man
whose amiableness depends upon his own sermons; for though
he may preach himself into a good-humour every Sunday,
it will be bad enough to have him quarrelling about green
geese from Monday morning till Saturday night.''

``I think the man who could often quarrel with Fanny,''
said Edmund affectionately, ``must be beyond the reach
of any sermons.''

Fanny turned farther into the window; and Miss
Crawford had only time to say, in a pleasant manner,
``I fancy Miss Price has been more used to deserve
praise than to hear it''; when, being earnestly invited
by the Miss Bertrams to join in a glee, she tripped off
to the instrument, leaving Edmund looking after her
in an ecstasy of admiration of all her many virtues,
from her obliging manners down to her light and graceful tread.

``There goes good-humour, I am sure,'' said he presently.
``There goes a temper which would never give pain!
How well she walks! and how readily she falls in with the
inclination of others! joining them the moment she is asked.
What a pity,'' he added, after an instant's reflection,
``that she should have been in such hands!''

Fanny agreed to it, and had the pleasure of seeing him continue
at the window with her, in spite of the expected glee;
and of having his eyes soon turned, like hers, towards the
scene without, where all that was solemn, and soothing,
and lovely, appeared in the brilliancy of an unclouded night,
and the contrast of the deep shade of the woods.  Fanny spoke
her feelings.  ``Here's harmony!'' said she; ``here's repose!
Here's what may leave all painting and all music behind,
and what poetry only can attempt to describe!  Here's what
may tranquillise every care, and lift the heart to rapture!
When I look out on such a night as this, I feel as if there
could be neither wickedness nor sorrow in the world;
and there certainly would be less of both if the sublimity
of Nature were more attended to, and people were carried
more out of themselves by contemplating such a scene.''

``I like to hear your enthusiasm, Fanny.  It is a lovely night,
and they are much to be pitied who have not been taught
to feel, in some degree, as you do; who have not,
at least, been given a taste for Nature in early life.
They lose a great deal.''

``\emph{You} taught me to think and feel on the subject, cousin.''

``I had a very apt scholar.  There's Arcturus looking
very bright.''

``Yes, and the Bear.  I wish I could see Cassiopeia.''

``We must go out on the lawn for that.  Should you be afraid?''

``Not in the least.  It is a great while since we have
had any star-gazing.''

``Yes; I do not know how it has happened.''  The glee began.
``We will stay till this is finished, Fanny,'' said he,
turning his back on the window; and as it advanced,
she had the mortification of seeing him advance too,
moving forward by gentle degrees towards the instrument,
and when it ceased, he was close by the singers, among the most
urgent in requesting to hear the glee again.

Fanny sighed alone at the window till scolded away
by Mrs.\ Norris's threats of catching cold.



\chapter{Chapter 12}

\gintro{Sir Thomas} was to return in November, and his eldest
son had duties to call him earlier home.  The approach
of September brought tidings of Mr.\ Bertram, first in a
letter to the gamekeeper and then in a letter to Edmund;
and by the end of August he arrived himself, to be gay,
agreeable, and gallant again as occasion served,
or Miss Crawford demanded; to tell of races and Weymouth,
and parties and friends, to which she might have listened
six weeks before with some interest, and altogether
to give her the fullest conviction, by the power
of actual comparison, of her preferring his younger brother.

It was very vexatious, and she was heartily sorry for it;
but so it was; and so far from now meaning to marry
the elder, she did not even want to attract him beyond
what the simplest claims of conscious beauty required:
his lengthened absence from Mansfield, without anything
but pleasure in view, and his own will to consult, made it
perfectly clear that he did not care about her; and his
indifference was so much more than equalled by her own,
that were he now to step forth the owner of Mansfield Park,
the Sir Thomas complete, which he was to be in time, she did
not believe she could accept him.

The season and duties which brought Mr.\ Bertram back to
Mansfield took Mr.\ Crawford into Norfolk.  Everingham could
not do without him in the beginning of September.  He went
for a fortnight---a fortnight of such dullness to the Miss
Bertrams as ought to have put them both on their guard,
and made even Julia admit, in her jealousy of her sister,
the absolute necessity of distrusting his attentions,
and wishing him not to return; and a fortnight of sufficient
leisure, in the intervals of shooting and sleeping, to have
convinced the gentleman that he ought to keep longer away,
had he been more in the habit of examining his own motives,
and of reflecting to what the indulgence of his idle vanity
was tending; but, thoughtless and selfish from prosperity
and bad example, he would not look beyond the present moment.
The sisters, handsome, clever, and encouraging, were an
amusement to his sated mind; and finding nothing in Norfolk
to equal the social pleasures of Mansfield, he gladly
returned to it at the time appointed, and was welcomed
thither quite as gladly by those whom he came to trifle with
further.

Maria, with only Mr.\ Rushworth to attend to her, and doomed
to the repeated details of his day's sport, good or bad,
his boast of his dogs, his jealousy of his neighbours,
his doubts of their qualifications, and his zeal after poachers,
subjects which will not find their way to female feelings
without some talent on one side or some attachment on
the other, had missed Mr.\ Crawford grievously; and Julia,
unengaged and unemployed, felt all the right of missing him
much more.  Each sister believed herself the favourite.
Julia might be justified in so doing by the hints
of Mrs.\ Grant, inclined to credit what she wished,
and Maria by the hints of Mr.\ Crawford himself.
Everything returned into the same channel as before his absence;
his manners being to each so animated and agreeable
as to lose no ground with either, and just stopping short
of the consistence, the steadiness, the solicitude,
and the warmth which might excite general notice.

Fanny was the only one of the party who found anything
to dislike; but since the day at Sotherton, she could never
see Mr.\ Crawford with either sister without observation,
and seldom without wonder or censure; and had her
confidence in her own judgment been equal to her exercise
of it in every other respect, had she been sure that she
was seeing clearly, and judging candidly, she would
probably have made some important communications to her
usual confidant.  As it was, however, she only hazarded
a hint, and the hint was lost.  ``I am rather surprised,''
said she, ``that Mr.\ Crawford should come back again so soon,
after being here so long before, full seven weeks;
for I had understood he was so very fond of change and
moving about, that I thought something would certainly
occur, when he was once gone, to take him elsewhere.
He is used to much gayer places than Mansfield.''

``It is to his credit,'' was Edmund's answer; ``and I dare
say it gives his sister pleasure.  She does not like his
unsettled habits.''

``What a favourite he is with my cousins!''

``Yes, his manners to women are such as must please.
Mrs.\ Grant, I believe, suspects him of a preference for Julia;
I have never seen much symptom of it, but I wish it may
be so.  He has no faults but what a serious attachment
would remove.''

``If Miss Bertram were not engaged,'' said Fanny cautiously,
``I could sometimes almost think that he admired her more
than Julia.''

``Which is, perhaps, more in favour of his liking
Julia best, than you, Fanny, may be aware; for I believe
it often happens that a man, before he has quite made up
his own mind, will distinguish the sister or intimate
friend of the woman he is really thinking of more than
the woman herself Crawford has too much sense to stay
here if he found himself in any danger from Maria;
and I am not at all afraid for her, after such a proof
as she has given that her feelings are not strong.''

Fanny supposed she must have been mistaken, and meant to
think differently in future; but with all that submission
to Edmund could do, and all the help of the coinciding
looks and hints which she occasionally noticed in some
of the others, and which seemed to say that Julia was
Mr.\ Crawford's choice, she knew not always what to think.
She was privy, one evening, to the hopes of her aunt
Norris on the subject, as well as to her feelings,
and the feelings of Mrs.\ Rushworth, on a point of some
similarity, and could not help wondering as she listened;
and glad would she have been not to be obliged to listen,
for it was while all the other young people were dancing,
and she sitting, most unwillingly, among the chaperons at
the fire, longing for the re-entrance of her elder cousin,
on whom all her own hopes of a partner then depended.
It was Fanny's first ball, though without the preparation
or splendour of many a young lady's first ball, being the
thought only of the afternoon, built on the late acquisition
of a violin player in the servants' hall, and the possibility
of raising five couple with the help of Mrs.\ Grant and a new
intimate friend of Mr.\ Bertram's just arrived on a visit.
It had, however, been a very happy one to Fanny through
four dances, and she was quite grieved to be losing
even a quarter of an hour.  While waiting and wishing,
looking now at the dancers and now at the door, this dialogue
between the two above-mentioned ladies was forced on her---%

``I think, ma'am,'' said Mrs.\ Norris, her eyes directed
towards Mr.\ Rushworth and Maria, who were partners for
the second time, ``we shall see some happy faces again now.''

``Yes, ma'am, indeed,'' replied the other, with a stately simper,
``there will be some satisfaction in looking on \emph{now},
and I think it was rather a pity they should have been
obliged to part.  Young folks in their situation
should be excused complying with the common forms.
I wonder my son did not propose it.''

``I dare say he did, ma'am. Mr.\ Rushworth is never remiss.
But dear Maria has such a strict sense of propriety, so much
of that true delicacy which one seldom meets with nowadays,
Mrs.\ Rushworth---that wish of avoiding particularity!
Dear ma'am, only look at her face at this moment;
how different from what it was the two last dances!''

Miss Bertram did indeed look happy, her eyes were
sparkling with pleasure, and she was speaking with
great animation, for Julia and her partner, Mr.\ Crawford,
were close to her; they were all in a cluster together.
How she had looked before, Fanny could not recollect,
for she had been dancing with Edmund herself, and had
not thought about her.

Mrs.\ Norris continued, ``It is quite delightful, ma'am, to
see young people so properly happy, so well suited,
and so much the thing!  I cannot but think of dear Sir
Thomas's delight.  And what do you say, ma'am, to the chance
of another match?  Mr.\ Rushworth has set a good example,
and such things are very catching.''

Mrs.\ Rushworth, who saw nothing but her son, was quite
at a loss.

``The couple above, ma'am. Do you see no symptoms there?''

``Oh dear!  Miss Julia and Mr.\ Crawford.  Yes, indeed,
a very pretty match.  What is his property?''

``Four thousand a year.''

``Very well.  Those who have not more must be satisfied with
what they have.  Four thousand a year is a pretty estate,
and he seems a very genteel, steady young man, so I hope
Miss Julia will be very happy.''

``It is not a settled thing, ma'am, yet.  We only speak of it
among friends.  But I have very little doubt it \emph{will} be.
He is growing extremely particular in his attentions.''

Fanny could listen no farther.  Listening and wondering were all
suspended for a time, for Mr.\ Bertram was in the room again;
and though feeling it would be a great honour to be asked
by him, she thought it must happen.  He came towards
their little circle; but instead of asking her to dance,
drew a chair near her, and gave her an account of the present
state of a sick horse, and the opinion of the groom,
from whom he had just parted.  Fanny found that it was
not to be, and in the modesty of her nature immediately
felt that she had been unreasonable in expecting it.
When he had told of his horse, he took a newspaper from
the table, and looking over it, said in a languid way,
``If you want to dance, Fanny, I will stand up with you.''
With more than equal civility the offer was declined;
she did not wish to dance.  ``I am glad of it,'' said he,
in a much brisker tone, and throwing down the newspaper
again, ``for I am tired to death.  I only wonder how
the good people can keep it up so long.  They had need
be \emph{all} in love, to find any amusement in such folly;
and so they are, I fancy.  If you look at them you may
see they are so many couple of lovers---all but Yates
and Mrs.\ Grant---and, between ourselves, she, poor woman,
must want a lover as much as any one of them.  A desperate
dull life hers must be with the doctor,'' making a sly face
as he spoke towards the chair of the latter, who proving,
however, to be close at his elbow, made so instantaneous
a change of expression and subject necessary, as Fanny,
in spite of everything, could hardly help laughing at.
``A strange business this in America, Dr.\ Grant!  What is
your opinion?  I always come to you to know what I am to
think of public matters.''

``My dear Tom,'' cried his aunt soon afterwards, ``as you
are not dancing, I dare say you will have no objection
to join us in a rubber; shall you?''  Then leaving her seat,
and coming to him to enforce the proposal, added in
a whisper, ``We want to make a table for Mrs.\ Rushworth,
you know.  Your mother is quite anxious about it,
but cannot very well spare time to sit down herself,
because of her fringe.  Now, you and I and Dr.\ Grant will
just do; and though \emph{we} play but half-crowns, you know,
you may bet half-guineas with \emph{him}.''

``I should be most happy,'' replied he aloud, and jumping up
with alacrity, ``it would give me the greatest pleasure;
but that I am this moment going to dance.''  Come, Fanny,
taking her hand, ``do not be dawdling any longer,
or the dance will be over.''

Fanny was led off very willingly, though it was impossible
for her to feel much gratitude towards her cousin,
or distinguish, as he certainly did, between the selfishness
of another person and his own.

``A pretty modest request upon my word,'' he indignantly
exclaimed as they walked away.  ``To want to nail me
to a card-table for the next two hours with herself and
Dr.\ Grant, who are always quarrelling, and that poking
old woman, who knows no more of whist than of algebra.
I wish my good aunt would be a little less busy!  And to ask
me in such a way too! without ceremony, before them all,
so as to leave me no possibility of refusing.  \emph{That} is
what I dislike most particularly.  It raises my spleen
more than anything, to have the pretence of being asked,
of being given a choice, and at the same time addressed
in such a way as to oblige one to do the very thing,
whatever it be!  If I had not luckily thought of standing
up with you I could not have got out of it.  It is a great
deal too bad.  But when my aunt has got a fancy in her head,
nothing can stop her.''



\chapter{Chapter 13}

\gintro{The Honourable John Yates,} this new friend, had not much
to recommend him beyond habits of fashion and expense,
and being the younger son of a lord with a tolerable
independence; and Sir Thomas would probably have thought
his introduction at Mansfield by no means desirable.
Mr.\ Bertram's acquaintance with him had begun at Weymouth,
where they had spent ten days together in the same society,
and the friendship, if friendship it might be called,
had been proved and perfected by Mr.\ Yates's being invited
to take Mansfield in his way, whenever he could, and by his
promising to come; and he did come rather earlier than had
been expected, in consequence of the sudden breaking-up
of a large party assembled for gaiety at the house
of another friend, which he had left Weymouth to join.
He came on the wings of disappointment, and with his head
full of acting, for it had been a theatrical party;
and the play in which he had borne a part was within
two days of representation, when the sudden death
of one of the nearest connexions of the family had
destroyed the scheme and dispersed the performers.
To be so near happiness, so near fame, so near the long
paragraph in praise of the private theatricals at Ecclesford,
the seat of the Right Hon. Lord Ravenshaw, in Cornwall,
which would of course have immortalised the whole party
for at least a twelvemonth! and being so near, to lose
it all, was an injury to be keenly felt, and Mr.\ Yates
could talk of nothing else.  Ecclesford and its theatre,
with its arrangements and dresses, rehearsals and jokes,
was his never-failing subject, and to boast of the past his
only consolation.

Happily for him, a love of the theatre is so general,
an itch for acting so strong among young people, that he
could hardly out-talk the interest of his hearers.
From the first casting of the parts to the epilogue
it was all bewitching, and there were few who did
not wish to have been a party concerned, or would have
hesitated to try their skill.  The play had been Lovers'
Vows, and Mr.\ Yates was to have been Count Cassel.
``A trifling part,'' said he, ``and not at all to my taste,
and such a one as I certainly would not accept again;
but I was determined to make no difficulties.
Lord Ravenshaw and the duke had appropriated the only two
characters worth playing before I reached Ecclesford;
and though Lord Ravenshaw offered to resign his to me,
it was impossible to take it, you know.  I was sorry
for \emph{him} that he should have so mistaken his powers,
for he was no more equal to the Baron---a little man
with a weak voice, always hoarse after the first
ten minutes.  It must have injured the piece materially;
but \emph{I} was resolved to make no difficulties.
Sir Henry thought the duke not equal to Frederick,
but that was because Sir Henry wanted the part himself;
whereas it was certainly in the best hands of the two.
I was surprised to see Sir Henry such a stick.
Luckily the strength of the piece did not depend upon him.
Our Agatha was inimitable, and the duke was thought very great
by many.  And upon the whole, it would certainly have gone
off wonderfully.''

``It was a hard case, upon my word''; and, ``I do think you
were very much to be pitied,'' were the kind responses
of listening sympathy.

``It is not worth complaining about; but to be sure the
poor old dowager could not have died at a worse time;
and it is impossible to help wishing that the news could
have been suppressed for just the three days we wanted.
It was but three days; and being only a grandmother,
and all happening two hundred miles off, I think there would
have been no great harm, and it was suggested, I know;
but Lord Ravenshaw, who I suppose is one of the most correct
men in England, would not hear of it.''

``An afterpiece instead of a comedy,'' said Mr.\ Bertram.
``Lovers' Vows were at an end, and Lord and Lady Ravenshaw
left to act My Grandmother by themselves.  Well, the jointure
may comfort \emph{him}; and perhaps, between friends, he began
to tremble for his credit and his lungs in the Baron,
and was not sorry to withdraw; and to make \emph{you} amends,
Yates, I think we must raise a little theatre at Mansfield,
and ask you to be our manager.''

This, though the thought of the moment, did not end
with the moment; for the inclination to act was awakened,
and in no one more strongly than in him who was now
master of the house; and who, having so much leisure
as to make almost any novelty a certain good, had likewise
such a degree of lively talents and comic taste,
as were exactly adapted to the novelty of acting.
The thought returned again and again.  ``Oh for the
Ecclesford theatre and scenery to try something with.''
Each sister could echo the wish; and Henry Crawford,
to whom, in all the riot of his gratifications it was
yet an untasted pleasure, was quite alive at the idea.
``I really believe,'' said he, ``I could be fool enough
at this moment to undertake any character that ever
was written, from Shylock or Richard III down to the singing
hero of a farce in his scarlet coat and cocked hat.
I feel as if I could be anything or everything;
as if I could rant and storm, or sigh or cut capers,
in any tragedy or comedy in the English language.  Let us
be doing something.  Be it only half a play, an act, a scene;
what should prevent us?  Not these countenances, I am sure,''
looking towards the Miss Bertrams; ``and for a theatre,
what signifies a theatre?  We shall be only amusing ourselves.
Any room in this house might suffice.''

``We must have a curtain,'' said Tom Bertram; ``a few yards
of green baize for a curtain, and perhaps that may be enough.''

``Oh, quite enough,'' cried Mr.\ Yates, ``with only just a side wing
or two run up, doors in flat, and three or four scenes to be
let down; nothing more would be necessary on such a plan as this.
For mere amusement among ourselves we should want nothing more.''

``I believe we must be satisfied with \emph{less},'' said Maria.
``There would not be time, and other difficulties
would arise.  We must rather adopt Mr.\ Crawford's views,
and make the \emph{performance}, not the \emph{theatre}, our object.
Many parts of our best plays are independent of scenery.''

``Nay,'' said Edmund, who began to listen with alarm.
``Let us do nothing by halves.  If we are to act, let it be in
a theatre completely fitted up with pit, boxes, and gallery,
and let us have a play entire from beginning to end; so as it
be a German play, no matter what, with a good tricking,
shifting afterpiece, and a figure-dance, and a hornpipe,
and a song between the acts.  If we do not outdo Ecclesford,
we do nothing.''

``Now, Edmund, do not be disagreeable,'' said Julia.
``Nobody loves a play better than you do, or can have gone
much farther to see one.''

``True, to see real acting, good hardened real acting;
but I would hardly walk from this room to the next to
look at the raw efforts of those who have not been
bred to the trade:  a set of gentlemen and ladies,
who have all the disadvantages of education and decorum
to struggle through.''

After a short pause, however, the subject still continued,
and was discussed with unabated eagerness, every one's
inclination increasing by the discussion, and a knowledge
of the inclination of the rest; and though nothing was settled
but that Tom Bertram would prefer a comedy, and his sisters
and Henry Crawford a tragedy, and that nothing in the world
could be easier than to find a piece which would please
them all, the resolution to act something or other seemed
so decided as to make Edmund quite uncomfortable.  He was
determined to prevent it, if possible, though his mother,
who equally heard the conversation which passed at table,
did not evince the least disapprobation.

The same evening afforded him an opportunity of trying
his strength.  Maria, Julia, Henry Crawford, and Mr.\ Yates
were in the billiard-room. Tom, returning from them into
the drawing-room, where Edmund was standing thoughtfully
by the fire, while Lady Bertram was on the sofa at a
little distance, and Fanny close beside her arranging
her work, thus began as he entered---``Such a horribly vile
billiard-table as ours is not to be met with, I believe,
above ground.  I can stand it no longer, and I think,
I may say, that nothing shall ever tempt me to it again;
but one good thing I have just ascertained:  it is the very
room for a theatre, precisely the shape and length for it;
and the doors at the farther end, communicating with each other,
as they may be made to do in five minutes, by merely moving
the bookcase in my father's room, is the very thing we
could have desired, if we had sat down to wish for it;
and my father's room will be an excellent greenroom.
It seems to join the billiard-room on purpose.''

``You are not serious, Tom, in meaning to act?'' said Edmund,
in a low voice, as his brother approached the fire.

``Not serious! never more so, I assure you.  What is there
to surprise you in it?''

``I think it would be very wrong.  In a \emph{general} light,
private theatricals are open to some objections, but as \emph{we}
are circumstanced, I must think it would be highly injudicious,
and more than injudicious to attempt anything of the kind.
It would shew great want of feeling on my father's account,
absent as he is, and in some degree of constant danger;
and it would be imprudent, I think, with regard to Maria,
whose situation is a very delicate one, considering everything,
extremely delicate.''

``You take up a thing so seriously! as if we were going
to act three times a week till my father's return,
and invite all the country.  But it is not to be a
display of that sort.  We mean nothing but a little
amusement among ourselves, just to vary the scene,
and exercise our powers in something new.  We want
no audience, no publicity.  We may be trusted, I think,
in chusing some play most perfectly unexceptionable;
and I can conceive no greater harm or danger to any of us
in conversing in the elegant written language of some
respectable author than in chattering in words of our own.
I have no fears and no scruples.  And as to my father's
being absent, it is so far from an objection, that I
consider it rather as a motive; for the expectation
of his return must be a very anxious period to my mother;
and if we can be the means of amusing that anxiety,
and keeping up her spirits for the next few weeks, I shall
think our time very well spent, and so, I am sure, will he.
It is a \emph{very} anxious period for her.''

As he said this, each looked towards their mother.
Lady Bertram, sunk back in one corner of the sofa,
the picture of health, wealth, ease, and tranquillity,
was just falling into a gentle doze, while Fanny was getting
through the few difficulties of her work for her.

Edmund smiled and shook his head.

``By Jove! this won't do,'' cried Tom, throwing himself into
a chair with a hearty laugh.  ``To be sure, my dear mother,
your anxiety---I was unlucky there.''

``What is the matter?'' asked her ladyship, in the heavy
tone of one half-roused; ``I was not asleep.''

``Oh dear, no, ma'am, nobody suspected you!  Well, Edmund,''
he continued, returning to the former subject, posture,
and voice, as soon as Lady Bertram began to nod again,
``but \emph{this} I \emph{will} maintain, that we shall be doing
no harm.''

``I cannot agree with you; I am convinced that my father
would totally disapprove it.''

``And I am convinced to the contrary.  Nobody is fonder of
the exercise of talent in young people, or promotes it more,
than my father, and for anything of the acting, spouting,
reciting kind, I think he has always a decided taste.
I am sure he encouraged it in us as boys.  How many a time
have we mourned over the dead body of Julius Caesar,
and to \emph{be'd} and not \emph{to} \emph{be'd}, in this very room,
for his amusement?  And I am sure, \emph{my} \emph{name} \emph{was} \emph{Norval},
every evening of my life through one Christmas holidays.''

``It was a very different thing.  You must see the
difference yourself.  My father wished us, as schoolboys,
to speak well, but he would never wish his grown-up
daughters to be acting plays.  His sense of decorum is strict.''

``I know all that,'' said Tom, displeased.  ``I know my father
as well as you do; and I'll take care that his daughters
do nothing to distress him.  Manage your own concerns,
Edmund, and I'll take care of the rest of the family.''

``If you are resolved on acting,'' replied the persevering Edmund,
``I must hope it will be in a very small and quiet way;
and I think a theatre ought not to be attempted.
It would be taking liberties with my father's house
in his absence which could not be justified.''

``For everything of that nature I will be answerable,''
said Tom, in a decided tone.  ``His house shall not be hurt.
I have quite as great an interest in being careful
of his house as you can have; and as to such alterations
as I was suggesting just now, such as moving a bookcase,
or unlocking a door, or even as using the billiard-room
for the space of a week without playing at billiards in it,
you might just as well suppose he would object to our sitting
more in this room, and less in the breakfast-room, than
we did before he went away, or to my sister's pianoforte
being moved from one side of the room to the other.
Absolute nonsense!''

``The innovation, if not wrong as an innovation, will be
wrong as an expense.''

``Yes, the expense of such an undertaking would be prodigious!
Perhaps it might cost a whole twenty pounds.  Something of
a theatre we must have undoubtedly, but it will be on the
simplest plan:  a green curtain and a little carpenter's work,
and that's all; and as the carpenter's work may be all
done at home by Christopher Jackson himself, it will be
too absurd to talk of expense; and as long as Jackson
is employed, everything will be right with Sir Thomas.
Don't imagine that nobody in this house can see or judge
but yourself.  Don't act yourself, if you do not like it,
but don't expect to govern everybody else.''

``No, as to acting myself,'' said Edmund, ``\emph{that} I
absolutely protest against.''

Tom walked out of the room as he said it, and Edmund was
left to sit down and stir the fire in thoughtful vexation.

Fanny, who had heard it all, and borne Edmund company
in every feeling throughout the whole, now ventured to say,
in her anxiety to suggest some comfort, ``Perhaps they may
not be able to find any play to suit them.  Your brother's
taste and your sisters' seem very different.''

``I have no hope there, Fanny.  If they persist in the scheme,
they will find something.  I shall speak to my sisters
and try to dissuade \emph{them}, and that is all I can do.''

``I should think my aunt Norris would be on your side.''

``I dare say she would, but she has no influence with
either Tom or my sisters that could be of any use;
and if I cannot convince them myself, I shall let things
take their course, without attempting it through her.
Family squabbling is the greatest evil of all, and we had
better do anything than be altogether by the ears.''

His sisters, to whom he had an opportunity of speaking
the next morning, were quite as impatient of his advice,
quite as unyielding to his representation, quite as determined
in the cause of pleasure, as Tom.  Their mother had no
objection to the plan, and they were not in the least afraid
of their father's disapprobation.  There could be no harm
in what had been done in so many respectable families,
and by so many women of the first consideration; and it
must be scrupulousness run mad that could see anything to
censure in a plan like theirs, comprehending only brothers
and sisters and intimate friends, and which would never
be heard of beyond themselves.  Julia \emph{did} seem inclined
to admit that Maria's situation might require particular
caution and delicacy---but that could not extend to \emph{her}---%
she was at liberty; and Maria evidently considered her
engagement as only raising her so much more above restraint,
and leaving her less occasion than Julia to consult
either father or mother.  Edmund had little to hope,
but he was still urging the subject when Henry Crawford
entered the room, fresh from the Parsonage, calling out,
``No want of hands in our theatre, Miss Bertram.
No want of understrappers:  my sister desires her love,
and hopes to be admitted into the company, and will be happy
to take the part of any old duenna or tame confidante,
that you may not like to do yourselves.''

Maria gave Edmund a glance, which meant, ``What say you now?
Can we be wrong if Mary Crawford feels the same?''
And Edmund, silenced, was obliged to acknowledge that the
charm of acting might well carry fascination to the mind
of genius; and with the ingenuity of love, to dwell more
on the obliging, accommodating purport of the message
than on anything else.

The scheme advanced.  Opposition was vain; and as to
Mrs.\ Norris, he was mistaken in supposing she would wish
to make any.  She started no difficulties that were
not talked down in five minutes by her eldest nephew
and niece, who were all-powerful with her; and as the
whole arrangement was to bring very little expense
to anybody, and none at all to herself, as she foresaw
in it all the comforts of hurry, bustle, and importance,
and derived the immediate advantage of fancying herself
obliged to leave her own house, where she had been living
a month at her own cost, and take up her abode in theirs,
that every hour might be spent in their service, she was,
in fact, exceedingly delighted with the project.



\chapter{Chapter 14}

\gintro{Fanny} seemed nearer being right than Edmund had supposed.
The business of finding a play that would suit everybody
proved to be no trifle; and the carpenter had received
his orders and taken his measurements, had suggested
and removed at least two sets of difficulties, and having
made the necessity of an enlargement of plan and expense
fully evident, was already at work, while a play was
still to seek.  Other preparations were also in hand.
An enormous roll of green baize had arrived from Northampton,
and been cut out by Mrs.\ Norris (with a saving by her
good management of full three-quarters of a yard), and
was actually forming into a curtain by the housemaids,
and still the play was wanting; and as two or three days
passed away in this manner, Edmund began almost to hope
that none might ever be found.

There were, in fact, so many things to be attended to,
so many people to be pleased, so many best characters
required, and, above all, such a need that the play
should be at once both tragedy and comedy, that there
did seem as little chance of a decision as anything
pursued by youth and zeal could hold out.

On the tragic side were the Miss Bertrams, Henry Crawford,
and Mr.\ Yates; on the comic, Tom Bertram, not \emph{quite} alone,
because it was evident that Mary Crawford's wishes,
though politely kept back, inclined the same way:  but his
determinateness and his power seemed to make allies unnecessary;
and, independent of this great irreconcilable difference,
they wanted a piece containing very few characters
in the whole, but every character first-rate, and three
principal women.  All the best plays were run over in vain.
Neither Hamlet, nor Macbeth, nor Othello, nor Douglas,
nor The Gamester, presented anything that could satisfy
even the tragedians; and The Rivals, The School for Scandal,
Wheel of Fortune, Heir at Law, and a long et cetera,
were successively dismissed with yet warmer objections.
No piece could be proposed that did not supply somebody
with a difficulty, and on one side or the other it was
a continual repetition of, ``Oh no, \emph{that} will never do!
Let us have no ranting tragedies.  Too many characters.
Not a tolerable woman's part in the play.  Anything but \emph{that},
my dear Tom.  It would be impossible to fill it up.
One could not expect anybody to take such a part.
Nothing but buffoonery from beginning to end.
\emph{That} might do, perhaps, but for the low parts.  If I
\emph{must} give my opinion, I have always thought it the most
insipid play in the English language.  \emph{I} do not wish
to make objections; I shall be happy to be of any use, but I
think we could not chuse worse.''

Fanny looked on and listened, not unamused to observe
the selfishness which, more or less disguised, seemed to
govern them all, and wondering how it would end.  For her
own gratification she could have wished that something
might be acted, for she had never seen even half a play,
but everything of higher consequence was against it.

``This will never do,'' said Tom Bertram at last.  ``We are
wasting time most abominably.  Something must be fixed on.
No matter what, so that something is chosen.  We must not be
so nice.  A few characters too many must not frighten us.
We must \emph{double} them.  We must descend a little.
If a part is insignificant, the greater our credit in making
anything of it.  From this moment I make no difficulties.
I take any part you chuse to give me, so as it be comic.
Let it but be comic, I condition for nothing more.''

For about the fifth time he then proposed the Heir at Law,
doubting only whether to prefer Lord Duberley or Dr.\ Pangloss
for himself; and very earnestly, but very unsuccessfully,
trying to persuade the others that there were some fine
tragic parts in the rest of the dramatis personae.

The pause which followed this fruitless effort
was ended by the same speaker, who, taking up one
of the many volumes of plays that lay on the table,
and turning it over, suddenly exclaimed---``Lovers' Vows!
And why should not Lovers' Vows do for \emph{us} as well
as for the Ravenshaws?  How came it never to be thought
of before?  It strikes me as if it would do exactly.
What say you all?  Here are two capital tragic parts for
Yates and Crawford, and here is the rhyming Butler for me,
if nobody else wants it; a trifling part, but the sort
of thing I should not dislike, and, as I said before,
I am determined to take anything and do my best.
And as for the rest, they may be filled up by anybody.
It is only Count Cassel and Anhalt.''

The suggestion was generally welcome.  Everybody was growing
weary of indecision, and the first idea with everybody was,
that nothing had been proposed before so likely to suit
them all.  Mr.\ Yates was particularly pleased:  he had
been sighing and longing to do the Baron at Ecclesford,
had grudged every rant of Lord Ravenshaw's, and been forced
to re-rant it all in his own room.  The storm through Baron
Wildenheim was the height of his theatrical ambition;
and with the advantage of knowing half the scenes by
heart already, he did now, with the greatest alacrity,
offer his services for the part.  To do him justice,
however, he did not resolve to appropriate it;
for remembering that there was some very good ranting-ground
in Frederick, he professed an equal willingness for that.
Henry Crawford was ready to take either.  Whichever Mr.\ Yates
did not chuse would perfectly satisfy him, and a short
parley of compliment ensued.  Miss Bertram, feeling all
the interest of an Agatha in the question, took on her
to decide it, by observing to Mr.\ Yates that this was a
point in which height and figure ought to be considered,
and that \emph{his} being the tallest, seemed to fit him
peculiarly for the Baron.  She was acknowledged to be
quite right, and the two parts being accepted accordingly,
she was certain of the proper Frederick.  Three of the
characters were now cast, besides Mr.\ Rushworth, who was
always answered for by Maria as willing to do anything;
when Julia, meaning, like her sister, to be Agatha,
began to be scrupulous on Miss Crawford's account.

``This is not behaving well by the absent,'' said she.
``Here are not women enough.  Amelia and Agatha may do
for Maria and me, but here is nothing for your sister,
Mr.\ Crawford.''

Mr.\ Crawford desired \emph{that} might not be thought of:
he was very sure his sister had no wish of acting
but as she might be useful, and that she would not
allow herself to be considered in the present case.
But this was immediately opposed by Tom Bertram,
who asserted the part of Amelia to be in every respect
the property of Miss Crawford, if she would accept it.
``It falls as naturally, as necessarily to her,'' said he,
``as Agatha does to one or other of my sisters.  It can be no
sacrifice on their side, for it is highly comic.''

A short silence followed.  Each sister looked anxious;
for each felt the best claim to Agatha, and was hoping
to have it pressed on her by the rest.  Henry Crawford,
who meanwhile had taken up the play, and with seeming
carelessness was turning over the first act, soon settled
the business.

``I must entreat Miss \emph{Julia} Bertram,'' said he, ``not to
engage in the part of Agatha, or it will be the ruin
of all my solemnity.  You must not, indeed you must not''
(turning to her). ``I could not stand your countenance
dressed up in woe and paleness.  The many laughs we have
had together would infallibly come across me, and Frederick
and his knapsack would be obliged to run away.''

Pleasantly, courteously, it was spoken; but the
manner was lost in the matter to Julia's feelings.
She saw a glance at Maria which confirmed the injury
to herself:  it was a scheme, a trick; she was slighted,
Maria was preferred; the smile of triumph which Maria
was trying to suppress shewed how well it was understood;
and before Julia could command herself enough to speak,
her brother gave his weight against her too, by saying,
``Oh yes!  Maria must be Agatha.  Maria will be the
best Agatha.  Though Julia fancies she prefers tragedy,
I would not trust her in it.  There is nothing of tragedy
about her.  She has not the look of it.  Her features
are not tragic features, and she walks too quick,
and speaks too quick, and would not keep her countenance.
She had better do the old countrywoman:  the Cottager's wife;
you had, indeed, Julia.  Cottager's wife is a very pretty part,
I assure you.  The old lady relieves the high-flown
benevolence of her husband with a good deal of spirit.
You shall be Cottager's wife.''

``Cottager's wife!'' cried Mr.\ Yates.  ``What are you
talking of?  The most trivial, paltry, insignificant part;
the merest commonplace; not a tolerable speech in the whole.
Your sister do that!  It is an insult to propose it.
At Ecclesford the governess was to have done it.
We all agreed that it could not be offered to anybody else.
A little more justice, Mr.\ Manager, if you please.
You do not deserve the office, if you cannot appreciate
the talents of your company a little better.''

``Why, as to \emph{that}, my good friend, till I and my company
have really acted there must be some guesswork; but I mean
no disparagement to Julia.  We cannot have two Agathas,
and we must have one Cottager's wife; and I am sure I set
her the example of moderation myself in being satisfied
with the old Butler.  If the part is trifling she will
have more credit in making something of it; and if she
is so desperately bent against everything humorous,
let her take Cottager's speeches instead of Cottager's
wife's, and so change the parts all through; \emph{he} is
solemn and pathetic enough, I am sure.  It could make
no difference in the play, and as for Cottager himself,
when he has got his wife's speeches, \emph{I} would undertake
him with all my heart.''

``With all your partiality for Cottager's wife,''
said Henry Crawford, ``it will be impossible to make
anything of it fit for your sister, and we must not suffer
her good-nature to be imposed on.  We must not \emph{allow}
her to accept the part.  She must not be left to her
own complaisance.  Her talents will be wanted in Amelia.
Amelia is a character more difficult to be well represented
than even Agatha.  I consider Amelia is the most difficult
character in the whole piece.  It requires great powers,
great nicety, to give her playfulness and simplicity
without extravagance.  I have seen good actresses fail
in the part.  Simplicity, indeed, is beyond the reach
of almost every actress by profession.  It requires
a delicacy of feeling which they have not.  It requires
a gentlewoman---a Julia Bertram.  You \emph{will} undertake it,
I hope?'' turning to her with a look of anxious entreaty,
which softened her a little; but while she hesitated
what to say, her brother again interposed with Miss
Crawford's better claim.

``No, no, Julia must not be Amelia.  It is not at
all the part for her.  She would not like it.
She would not do well.  She is too tall and robust.
Amelia should be a small, light, girlish, skipping figure.
It is fit for Miss Crawford, and Miss Crawford only.
She looks the part, and I am persuaded will do it admirably.''

Without attending to this, Henry Crawford continued
his supplication.  ``You must oblige us,'' said he,
``indeed you must.  When you have studied the character, I am
sure you will feel it suit you.  Tragedy may be your choice,
but it will certainly appear that comedy chuses \emph{you}.
You will be to visit me in prison with a basket of provisions;
you will not refuse to visit me in prison?  I think I
see you coming in with your basket''

The influence of his voice was felt.  Julia wavered;
but was he only trying to soothe and pacify her, and make
her overlook the previous affront?  She distrusted him.
The slight had been most determined.  He was, perhaps,
but at treacherous play with her.  She looked suspiciously
at her sister; Maria's countenance was to decide it:
if she were vexed and alarmed---but Maria looked all
serenity and satisfaction, and Julia well knew that on
this ground Maria could not be happy but at her expense.
With hasty indignation, therefore, and a tremulous voice,
she said to him, ``You do not seem afraid of not
keeping your countenance when I come in with a basket
of provisions---though one might have supposed---but it
is only as Agatha that I was to be so overpowering!''
She stopped---Henry Crawford looked rather foolish,
and as if he did not know what to say.  Tom Bertram
began again---%

``Miss Crawford must be Amelia.  She will be an excellent Amelia.''

``Do not be afraid of \emph{my} wanting the character,''
cried Julia, with angry quickness:  ``I am \emph{not} to be Agatha,
and I am sure I will do nothing else; and as to Amelia,
it is of all parts in the world the most disgusting to me.
I quite detest her.  An odious, little, pert, unnatural,
impudent girl.  I have always protested against comedy,
and this is comedy in its worst form.''  And so saying,
she walked hastily out of the room, leaving awkward feelings
to more than one, but exciting small compassion in any
except Fanny, who had been a quiet auditor of the whole,
and who could not think of her as under the agitations of
\emph{jealousy} without great pity.

A short silence succeeded her leaving them; but her brother
soon returned to business and Lovers' Vows, and was
eagerly looking over the play, with Mr.\ Yates's help,
to ascertain what scenery would be necessary---while Maria
and Henry Crawford conversed together in an under-voice,
and the declaration with which she began of, ``I am
sure I would give up the part to Julia most willingly,
but that though I shall probably do it very ill,
I feel persuaded \emph{she} would do it worse,'' was doubtless
receiving all the compliments it called for.

When this had lasted some time, the division of the party
was completed by Tom Bertram and Mr.\ Yates walking off
together to consult farther in the room now beginning
to be called \emph{the} \emph{Theatre}, and Miss Bertram's resolving
to go down to the Parsonage herself with the offer
of Amelia to Miss Crawford; and Fanny remained alone.

The first use she made of her solitude was to take up
the volume which had been left on the table, and begin
to acquaint herself with the play of which she had heard
so much.  Her curiosity was all awake, and she ran
through it with an eagerness which was suspended only
by intervals of astonishment, that it could be chosen
in the present instance, that it could be proposed
and accepted in a private theatre!  Agatha and Amelia
appeared to her in their different ways so totally
improper for home representation---the situation of one,
and the language of the other, so unfit to be expressed
by any woman of modesty, that she could hardly suppose
her cousins could be aware of what they were engaging in;
and longed to have them roused as soon as possible
by the remonstrance which Edmund would certainly make.



\chapter{Chapter 15}

\gintro{Miss Crawford} accepted the part very readily; and soon after
Miss Bertram's return from the Parsonage, Mr.\ Rushworth
arrived, and another character was consequently cast.
He had the offer of Count Cassel and Anhalt, and at first
did not know which to chuse, and wanted Miss Bertram
to direct him; but upon being made to understand the
different style of the characters, and which was which,
and recollecting that he had once seen the play in London,
and had thought Anhalt a very stupid fellow, he soon
decided for the Count.  Miss Bertram approved the decision,
for the less he had to learn the better; and though she
could not sympathise in his wish that the Count and
Agatha might be to act together, nor wait very patiently
while he was slowly turning over the leaves with the hope
of still discovering such a scene, she very kindly took
his part in hand, and curtailed every speech that admitted
being shortened; besides pointing out the necessity
of his being very much dressed, and chusing his colours.
Mr.\ Rushworth liked the idea of his finery very well,
though affecting to despise it; and was too much
engaged with what his own appearance would be to think
of the others, or draw any of those conclusions, or feel
any of that displeasure which Maria had been half prepared for.

Thus much was settled before Edmund, who had been out all
the morning, knew anything of the matter; but when he
entered the drawing-room before dinner, the buzz of
discussion was high between Tom, Maria, and Mr.\ Yates;
and Mr.\ Rushworth stepped forward with great alacrity
to tell him the agreeable news.

``We have got a play,'' said he.  ``It is to be Lovers'
Vows; and I am to be Count Cassel, and am to come
in first with a blue dress and a pink satin cloak,
and afterwards am to have another fine fancy suit,
by way of a shooting-dress. I do not know how I shall like it.''

Fanny's eyes followed Edmund, and her heart beat for him
as she heard this speech, and saw his look, and felt
what his sensations must be.

``Lovers' Vows!'' in a tone of the greatest amazement,
was his only reply to Mr.\ Rushworth, and he turned
towards his brother and sisters as if hardly doubting
a contradiction.

``Yes,'' cried Mr.\ Yates.  ``After all our debatings
and difficulties, we find there is nothing that will
suit us altogether so well, nothing so unexceptionable,
as Lovers' Vows.  The wonder is that it should not have been
thought of before.  My stupidity was abominable, for here
we have all the advantage of what I saw at Ecclesford;
and it is so useful to have anything of a model!
We have cast almost every part.''

``But what do you do for women?'' said Edmund gravely,
and looking at Maria.

Maria blushed in spite of herself as she answered,
``I take the part which Lady Ravenshaw was to have done,
and'' (with a bolder eye) ``Miss Crawford is to be Amelia.''

``I should not have thought it the sort of play to be so
easily filled up, with \emph{us},'' replied Edmund, turning away
to the fire, where sat his mother, aunt, and Fanny,
and seating himself with a look of great vexation.

Mr.\ Rushworth followed him to say, ``I come in three times,
and have two-and-forty speeches.  That's something,
is not it?  But I do not much like the idea of being so fine.
I shall hardly know myself in a blue dress and a pink
satin cloak.''

Edmund could not answer him.  In a few minutes Mr.\ Bertram
was called out of the room to satisfy some doubts
of the carpenter; and being accompanied by Mr.\ Yates,
and followed soon afterwards by Mr.\ Rushworth, Edmund almost
immediately took the opportunity of saying, ``I cannot,
before Mr.\ Yates, speak what I feel as to this play,
without reflecting on his friends at Ecclesford;
but I must now, my dear Maria, tell \emph{you}, that I
think it exceedingly unfit for private representation,
and that I hope you will give it up.  I cannot but suppose
you \emph{will} when you have read it carefully over.
Read only the first act aloud to either your mother or aunt,
and see how you can approve it.  It will not be necessary
to send you to your \emph{father's} judgment, I am convinced.''

``We see things very differently,'' cried Maria.
``I am perfectly acquainted with the play, I assure you;
and with a very few omissions, and so forth, which will
be made, of course, I can see nothing objectionable in it;
and \emph{I} am not the \emph{only} young woman you find who thinks
it very fit for private representation.''

``I am sorry for it,'' was his answer; ``but in this matter
it is \emph{you} who are to lead.  \emph{You} must set the example.
If others have blundered, it is your place to put
them right, and shew them what true delicacy is.
In all points of decorum \emph{your} conduct must be law
to the rest of the party.''

This picture of her consequence had some effect, for no
one loved better to lead than Maria; and with far more
good-humour she answered, ``I am much obliged to you, Edmund;
you mean very well, I am sure:  but I still think you
see things too strongly; and I really cannot undertake
to harangue all the rest upon a subject of this kind.
\emph{There} would be the greatest indecorum, I think.''

``Do you imagine that I could have such an idea in
my head?  No; let your conduct be the only harangue.
Say that, on examining the part, you feel yourself
unequal to it; that you find it requiring more exertion
and confidence than you can be supposed to have.
Say this with firmness, and it will be quite enough.
All who can distinguish will understand your motive.
The play will be given up, and your delicacy honoured as
it ought.''

``Do not act anything improper, my dear,'' said Lady Bertram.
``Sir Thomas would not like it.---Fanny, ring the bell;
I must have my dinner.---To be sure, Julia is dressed by
this time.''

``I am convinced, madam,'' said Edmund, preventing Fanny,
``that Sir Thomas would not like it.''

``There, my dear, do you hear what Edmund says?''

``If I were to decline the part,'' said Maria,
with renewed zeal, ``Julia would certainly take it.''

``What!'' cried Edmund, ``if she knew your reasons!''

``Oh! she might think the difference between us---%
the difference in our situations---that \emph{she} need
not be so scrupulous as \emph{I} might feel necessary.
I am sure she would argue so.  No; you must excuse me;
I cannot retract my consent; it is too far settled,
everybody would be so disappointed, Tom would be quite angry;
and if we are so very nice, we shall never act anything.''

``I was just going to say the very same thing,'' said Mrs.\ Norris.
``If every play is to be objected to, you will act nothing,
and the preparations will be all so much money thrown away,
and I am sure \emph{that} would be a discredit to us all.
I do not know the play; but, as Maria says, if there
is anything a little too warm (and it is so with most
of them) it can be easily left out.  We must not be
over-precise, Edmund.  As Mr.\ Rushworth is to act too,
there can be no harm.  I only wish Tom had known his own
mind when the carpenters began, for there was the loss
of half a day's work about those side-doors. The curtain
will be a good job, however.  The maids do their work
very well, and I think we shall be able to send back
some dozens of the rings.  There is no occasion to put
them so very close together.  I \emph{am} of some use, I hope,
in preventing waste and making the most of things.
There should always be one steady head to superintend
so many young ones.  I forgot to tell Tom of something
that happened to me this very day.  I had been looking
about me in the poultry-yard, and was just coming out,
when who should I see but Dick Jackson making up
to the servants' hall-door with two bits of deal board
in his hand, bringing them to father, you may be sure;
mother had chanced to send him of a message to father,
and then father had bid him bring up them two bits of board,
for he could not no how do without them.  I knew what all
this meant, for the servants' dinner-bell was ringing
at the very moment over our heads; and as I hate such
encroaching people (the Jacksons are very encroaching,
I have always said so:  just the sort of people to get
all they can), I said to the boy directly (a great lubberly
fellow of ten years old, you know, who ought to be ashamed
of himself), `\emph{I'll} take the boards to your father,
Dick, so get you home again as fast as you can.'
The boy looked very silly, and turned away without
offering a word, for I believe I might speak pretty sharp;
and I dare say it will cure him of coming marauding
about the house for one while.  I hate such greediness---%
so good as your father is to the family, employing the man
all the year round!''

Nobody was at the trouble of an answer; the others
soon returned; and Edmund found that to have endeavoured
to set them right must be his only satisfaction.

Dinner passed heavily.  Mrs.\ Norris related again
her triumph over Dick Jackson, but neither play nor
preparation were otherwise much talked of, for Edmund's
disapprobation was felt even by his brother, though he
would not have owned it.  Maria, wanting Henry Crawford's
animating support, thought the subject better avoided.
Mr.\ Yates, who was trying to make himself agreeable to Julia,
found her gloom less impenetrable on any topic than
that of his regret at her secession from their company;
and Mr.\ Rushworth, having only his own part and his own
dress in his head, had soon talked away all that could
be said of either.

But the concerns of the theatre were suspended only for an
hour or two:  there was still a great deal to be settled;
and the spirits of evening giving fresh courage, Tom, Maria,
and Mr.\ Yates, soon after their being reassembled
in the drawing-room, seated themselves in committee
at a separate table, with the play open before them,
and were just getting deep in the subject when a most
welcome interruption was given by the entrance of Mr.\ and
Miss Crawford, who, late and dark and dirty as it was,
could not help coming, and were received with the most grateful
joy.

``Well, how do you go on?'' and ``What have you settled?''
and ``Oh! we can do nothing without you,'' followed the
first salutations; and Henry Crawford was soon seated
with the other three at the table, while his sister made
her way to Lady Bertram, and with pleasant attention
was complimenting \emph{her}.  ``I must really congratulate
your ladyship,'' said she, ``on the play being chosen;
for though you have borne it with exemplary patience, I am
sure you must be sick of all our noise and difficulties.
The actors may be glad, but the bystanders must be infinitely
more thankful for a decision; and I do sincerely give
you joy, madam, as well as Mrs.\ Norris, and everybody else
who is in the same predicament,'' glancing half fearfully,
half slyly, beyond Fanny to Edmund.

She was very civilly answered by Lady Bertram,
but Edmund said nothing.  His being only a bystander was
not disclaimed.  After continuing in chat with the party
round the fire a few minutes, Miss Crawford returned
to the party round the table; and standing by them,
seemed to interest herself in their arrangements till,
as if struck by a sudden recollection, she exclaimed,
``My good friends, you are most composedly at work upon
these cottages and alehouses, inside and out; but pray let
me know my fate in the meanwhile.  Who is to be Anhalt?
What gentleman among you am I to have the pleasure of making
love to?''

For a moment no one spoke; and then many spoke together
to tell the same melancholy truth, that they had not yet
got any Anhalt.  ``Mr.\ Rushworth was to be Count Cassel,
but no one had yet undertaken Anhalt.''

``I had my choice of the parts,'' said Mr.\ Rushworth;
``but I thought I should like the Count best, though I do
not much relish the finery I am to have.''

``You chose very wisely, I am sure,'' replied Miss Crawford,
with a brightened look; ``Anhalt is a heavy part.''

``\emph{The} \emph{Count} has two-and-forty speeches,''
returned Mr.\ Rushworth, ``which is no trifle.''

``I am not at all surprised,'' said Miss Crawford,
after a short pause, ``at this want of an Anhalt.
Amelia deserves no better.  Such a forward young lady
may well frighten the men.''

``I should be but too happy in taking the part, if it
were possible,'' cried Tom; ``but, unluckily, the Butler
and Anhalt are in together.  I will not entirely give
it up, however; I will try what can be done---I will look
it over again.''

``Your \emph{brother} should take the part,'' said Mr.\ Yates,
in a low voice.  ``Do not you think he would?''

``\emph{I} shall not ask him,'' replied Tom, in a cold,
determined manner.

Miss Crawford talked of something else, and soon afterwards
rejoined the party at the fire.

``They do not want me at all,'' said she, seating herself.
``I only puzzle them, and oblige them to make civil speeches.
Mr.\ Edmund Bertram, as you do not act yourself,
you will be a disinterested adviser; and, therefore,
I apply to \emph{you}.  What shall we do for an Anhalt?
Is it practicable for any of the others to double it?
What is your advice?''

``My advice,'' said he calmly, ``is that you change the play.''

``\emph{I} should have no objection,'' she replied; ``for though
I should not particularly dislike the part of Amelia
if well supported, that is, if everything went well,
I shall be sorry to be an inconvenience; but as they
do not chuse to hear your advice at \emph{that} \emph{table}''
(looking round), ``it certainly will not be taken.''

Edmund said no more.

``If \emph{any} part could tempt \emph{you} to act, I suppose it would
be Anhalt,'' observed the lady archly, after a short pause;
``for he is a clergyman, you know.''

``\emph{That} circumstance would by no means tempt me,''
he replied, ``for I should be sorry to make the character
ridiculous by bad acting.  It must be very difficult
to keep Anhalt from appearing a formal, solemn lecturer;
and the man who chuses the profession itself is, perhaps,
one of the last who would wish to represent it on the stage.''

Miss Crawford was silenced, and with some feelings of resentment
and mortification, moved her chair considerably nearer the
tea-table, and gave all her attention to Mrs.\ Norris, who was
presiding there.

``Fanny,'' cried Tom Bertram, from the other table,
where the conference was eagerly carrying on, and the
conversation incessant, ``we want your services''

Fanny was up in a moment, expecting some errand; for the
habit of employing her in that way was not yet overcome,
in spite of all that Edmund could do.

``Oh! we do not want to disturb you from your seat.
We do not want your \emph{present} services.  We shall only want
you in our play.  You must be Cottager's wife.''

``Me!'' cried Fanny, sitting down again with a most frightened look.
``Indeed you must excuse me.  I could not act anything
if you were to give me the world.  No, indeed, I cannot act.''

``Indeed, but you must, for we cannot excuse you.
It need not frighten you:  it is a nothing of a part,
a mere nothing, not above half a dozen speeches altogether,
and it will not much signify if nobody hears a word you say;
so you may be as creep-mouse as you like, but we must have
you to look at.''

``If you are afraid of half a dozen speeches,'' cried Mr.\ Rushworth,
``what would you do with such a part as mine?  I have forty-two to
learn.''

``It is not that I am afraid of learning by heart,''
said Fanny, shocked to find herself at that moment the
only speaker in the room, and to feel that almost every
eye was upon her; ``but I really cannot act.''

``Yes, yes, you can act well enough for \emph{us}.
Learn your part, and we will teach you all the rest.
You have only two scenes, and as I shall be Cottager,
I'll put you in and push you about, and you will do it
very well, I'll answer for it.''

``No, indeed, Mr.\ Bertram, you must excuse me.  You cannot
have an idea.  It would be absolutely impossible for me.
If I were to undertake it, I should only disappoint you.''

``Phoo!  Phoo!  Do not be so shamefaced.  You'll do it
very well.  Every allowance will be made for you.
We do not expect perfection.  You must get a brown gown,
and a white apron, and a mob cap, and we must make
you a few wrinkles, and a little of the crowsfoot at
the corner of your eyes, and you will be a very proper,
little old woman.''

``You must excuse me, indeed you must excuse me,'' cried Fanny,
growing more and more red from excessive agitation,
and looking distressfully at Edmund, who was kindly
observing her; but unwilling to exasperate his brother
by interference, gave her only an encouraging smile.
Her entreaty had no effect on Tom:  he only said again
what he had said before; and it was not merely Tom,
for the requisition was now backed by Maria, and Mr.\ Crawford,
and Mr.\ Yates, with an urgency which differed from
his but in being more gentle or more ceremonious,
and which altogether was quite overpowering to Fanny;
and before she could breathe after it, Mrs.\ Norris completed
the whole by thus addressing her in a whisper at once angry
and audible---``What a piece of work here is about nothing:
I am quite ashamed of you, Fanny, to make such a difficulty
of obliging your cousins in a trifle of this sort---so kind
as they are to you!  Take the part with a good grace,
and let us hear no more of the matter, I entreat.''

``Do not urge her, madam,'' said Edmund.  ``It is not fair to
urge her in this manner.  You see she does not like to act.
Let her chuse for herself, as well as the rest of us.
Her judgment may be quite as safely trusted.  Do not urge
her any more.''

``I am not going to urge her,'' replied Mrs.\ Norris sharply;
``but I shall think her a very obstinate, ungrateful girl,
if she does not do what her aunt and cousins wish her---%
very ungrateful, indeed, considering who and what she is.''

Edmund was too angry to speak; but Miss Crawford,
looking for a moment with astonished eyes at Mrs.\ Norris,
and then at Fanny, whose tears were beginning to shew
themselves, immediately said, with some keenness, ``I do
not like my situation:  this \emph{place} is too hot for me,''
and moved away her chair to the opposite side of the table,
close to Fanny, saying to her, in a kind, low whisper,
as she placed herself, ``Never mind, my dear Miss Price,
this is a cross evening:  everybody is cross and teasing,
but do not let us mind them''; and with pointed attention
continued to talk to her and endeavour to raise her spirits,
in spite of being out of spirits herself.  By a look at
her brother she prevented any farther entreaty from the
theatrical board, and the really good feelings by which she
was almost purely governed were rapidly restoring her
to all the little she had lost in Edmund's favour.

Fanny did not love Miss Crawford; but she felt very much
obliged to her for her present kindness; and when,
from taking notice of her work, and wishing \emph{she} could
work as well, and begging for the pattern, and supposing
Fanny was now preparing for her \emph{appearance}, as of
course she would come out when her cousin was married,
Miss Crawford proceeded to inquire if she had heard lately
from her brother at sea, and said that she had quite
a curiosity to see him, and imagined him a very fine
young man, and advised Fanny to get his picture drawn
before he went to sea again---she could not help admitting
it to be very agreeable flattery, or help listening,
and answering with more animation than she had intended.

The consultation upon the play still went on, and Miss
Crawford's attention was first called from Fanny by Tom
Bertram's telling her, with infinite regret, that he
found it absolutely impossible for him to undertake the
part of Anhalt in addition to the Butler:  he had been
most anxiously trying to make it out to be feasible,
but it would not do; he must give it up.  ``But there will
not be the smallest difficulty in filling it,'' he added.
``We have but to speak the word; we may pick and chuse.
I could name, at this moment, at least six young men within
six miles of us, who are wild to be admitted into our company,
and there are one or two that would not disgrace us:
I should not be afraid to trust either of the Olivers
or Charles Maddox.  Tom Oliver is a very clever fellow,
and Charles Maddox is as gentlemanlike a man as you will
see anywhere, so I will take my horse early to-morrow
morning and ride over to Stoke, and settle with one
of them.''

While he spoke, Maria was looking apprehensively round
at Edmund in full expectation that he must oppose such
an enlargement of the plan as this:  so contrary to all
their first protestations; but Edmund said nothing.
After a moment's thought, Miss Crawford calmly replied,
``As far as I am concerned, I can have no objection to
anything that you all think eligible.  Have I ever seen
either of the gentlemen?  Yes, Mr.\ Charles Maddox dined
at my sister's one day, did not he, Henry?  A quiet-looking
young man.  I remember him.  Let \emph{him} be applied to,
if you please, for it will be less unpleasant to me than
to have a perfect stranger.''

Charles Maddox was to be the man.  Tom repeated his resolution
of going to him early on the morrow; and though Julia,
who had scarcely opened her lips before, observed, in a
sarcastic manner, and with a glance first at Maria and then
at Edmund, that ``the Mansfield theatricals would enliven
the whole neighbourhood exceedingly,'' Edmund still held
his peace, and shewed his feelings only by a determined gravity.

``I am not very sanguine as to our play,'' said Miss Crawford,
in an undervoice to Fanny, after some consideration;
``and I can tell Mr.\ Maddox that I shall shorten some
of \emph{his} speeches, and a great many of \emph{my} \emph{own},
before we rehearse together.  It will be very disagreeable,
and by no means what I expected.''



\chapter{Chapter 16}

\gintro{It was not} in Miss Crawford's power to talk Fanny into any
real forgetfulness of what had passed.  When the evening
was over, she went to bed full of it, her nerves still
agitated by the shock of such an attack from her cousin Tom,
so public and so persevered in, and her spirits sinking
under her aunt's unkind reflection and reproach.
To be called into notice in such a manner, to hear that it
was but the prelude to something so infinitely worse,
to be told that she must do what was so impossible as to act;
and then to have the charge of obstinacy and ingratitude
follow it, enforced with such a hint at the dependence
of her situation, had been too distressing at the time
to make the remembrance when she was alone much less so,
especially with the superadded dread of what the
morrow might produce in continuation of the subject.
Miss Crawford had protected her only for the time;
and if she were applied to again among themselves with all
the authoritative urgency that Tom and Maria were capable of,
and Edmund perhaps away, what should she do?  She fell
asleep before she could answer the question, and found
it quite as puzzling when she awoke the next morning.
The little white attic, which had continued her sleeping-room
ever since her first entering the family, proving incompetent
to suggest any reply, she had recourse, as soon as she
was dressed, to another apartment more spacious and more
meet for walking about in and thinking, and of which she
had now for some time been almost equally mistress.
It had been their school-room; so called till the Miss
Bertrams would not allow it to be called so any longer,
and inhabited as such to a later period.  There Miss
Lee had lived, and there they had read and written,
and talked and laughed, till within the last three years,
when she had quitted them.  The room had then become useless,
and for some time was quite deserted, except by Fanny,
when she visited her plants, or wanted one of the books,
which she was still glad to keep there, from the deficiency
of space and accommodation in her little chamber above:
but gradually, as her value for the comforts of it increased,
she had added to her possessions, and spent more of her
time there; and having nothing to oppose her, had so
naturally and so artlessly worked herself into it, that it
was now generally admitted to be hers.  The East room,
as it had been called ever since Maria Bertram was sixteen,
was now considered Fanny's, almost as decidedly as the
white attic:  the smallness of the one making the use of
the other so evidently reasonable that the Miss Bertrams,
with every superiority in their own apartments which their
own sense of superiority could demand, were entirely
approving it; and Mrs.\ Norris, having stipulated for there
never being a fire in it on Fanny's account, was tolerably
resigned to her having the use of what nobody else wanted,
though the terms in which she sometimes spoke of the
indulgence seemed to imply that it was the best room in
the house.

The aspect was so favourable that even without a fire
it was habitable in many an early spring and late
autumn morning to such a willing mind as Fanny's;
and while there was a gleam of sunshine she hoped not
to be driven from it entirely, even when winter came.
The comfort of it in her hours of leisure was extreme.
She could go there after anything unpleasant below,
and find immediate consolation in some pursuit,
or some train of thought at hand.  Her plants, her books---%
of which she had been a collector from the first hour
of her commanding a shilling---her writing-desk, and her
works of charity and ingenuity, were all within her reach;
or if indisposed for employment, if nothing but musing
would do, she could scarcely see an object in that room
which had not an interesting remembrance connected with it.
Everything was a friend, or bore her thoughts to a friend;
and though there had been sometimes much of suffering
to her; though her motives had often been misunderstood,
her feelings disregarded, and her comprehension undervalued;
though she had known the pains of tyranny, of ridicule,
and neglect, yet almost every recurrence of either had led
to something consolatory:  her aunt Bertram had spoken
for her, or Miss Lee had been encouraging, or, what was yet
more frequent or more dear, Edmund had been her champion
and her friend:  he had supported her cause or explained
her meaning, he had told her not to cry, or had given her
some proof of affection which made her tears delightful;
and the whole was now so blended together, so harmonised
by distance, that every former affliction had its charm.
The room was most dear to her, and she would not have
changed its furniture for the handsomest in the house,
though what had been originally plain had suffered all
the ill-usage of children; and its greatest elegancies
and ornaments were a faded footstool of Julia's work,
too ill done for the drawing-room, three transparencies,
made in a rage for transparencies, for the three lower
panes of one window, where Tintern Abbey held its station
between a cave in Italy and a moonlight lake in Cumberland,
a collection of family profiles, thought unworthy of being
anywhere else, over the mantelpiece, and by their side,
and pinned against the wall, a small sketch of a ship
sent four years ago from the Mediterranean by William,
with H.M.S. Antwerp at the bottom, in letters as tall as the
mainmast.

To this nest of comforts Fanny now walked down to try
its influence on an agitated, doubting spirit, to see
if by looking at Edmund's profile she could catch any of
his counsel, or by giving air to her geraniums she might
inhale a breeze of mental strength herself.  But she had
more than fears of her own perseverance to remove:  she had
begun to feel undecided as to what she \emph{ought} \emph{to} \emph{do};
and as she walked round the room her doubts were increasing.
Was she \emph{right} in refusing what was so warmly asked,
so strongly wished for---what might be so essential
to a scheme on which some of those to whom she owed the
greatest complaisance had set their hearts?  Was it not
ill-nature, selfishness, and a fear of exposing herself?
And would Edmund's judgment, would his persuasion of Sir
Thomas's disapprobation of the whole, be enough to justify
her in a determined denial in spite of all the rest?
It would be so horrible to her to act that she was inclined
to suspect the truth and purity of her own scruples;
and as she looked around her, the claims of her cousins
to being obliged were strengthened by the sight of
present upon present that she had received from them.
The table between the windows was covered with work-boxes
and netting-boxes which had been given her at different times,
principally by Tom; and she grew bewildered as to the amount
of the debt which all these kind remembrances produced.
A tap at the door roused her in the midst of this attempt
to find her way to her duty, and her gentle ``Come in''
was answered by the appearance of one, before whom all her
doubts were wont to be laid.  Her eyes brightened at the
sight of Edmund.

``Can I speak with you, Fanny, for a few minutes?''
said he.

``Yes, certainly.''

``I want to consult.  I want your opinion.''

``My opinion!'' she cried, shrinking from such a compliment,
highly as it gratified her.

``Yes, your advice and opinion.  I do not know what to do.
This acting scheme gets worse and worse, you see.
They have chosen almost as bad a play as they could,
and now, to complete the business, are going to ask the
help of a young man very slightly known to any of us.
This is the end of all the privacy and propriety which was
talked about at first.  I know no harm of Charles Maddox;
but the excessive intimacy which must spring from his being
admitted among us in this manner is highly objectionable,
the \emph{more} than intimacy---the familiarity.  I cannot think
of it with any patience; and it does appear to me an evil
of such magnitude as must, \emph{if} \emph{possible}, be prevented.
Do not you see it in the same light?''

``Yes; but what can be done?  Your brother is so determined.''

``There is but \emph{one} thing to be done, Fanny.  I must
take Anhalt myself.  I am well aware that nothing else
will quiet Tom.''

Fanny could not answer him.

``It is not at all what I like,'' he continued.  ``No man can
like being driven into the \emph{appearance} of such inconsistency.
After being known to oppose the scheme from the beginning,
there is absurdity in the face of my joining them \emph{now},
when they are exceeding their first plan in every respect;
but I can think of no other alternative.  Can you, Fanny?''

``No,'' said Fanny slowly, ``not immediately, but---''

``But what?  I see your judgment is not with me.  Think it
a little over.  Perhaps you are not so much aware as I am
of the mischief that \emph{may}, of the unpleasantness that \emph{must}
arise from a young man's being received in this manner:
domesticated among us; authorised to come at all hours,
and placed suddenly on a footing which must do away
all restraints.  To think only of the licence which every
rehearsal must tend to create.  It is all very bad!
Put yourself in Miss Crawford's place, Fanny.
Consider what it would be to act Amelia with a stranger.
She has a right to be felt for, because she evidently
feels for herself.  I heard enough of what she said to you
last night to understand her unwillingness to be acting
with a stranger; and as she probably engaged in the part
with different expectations---perhaps without considering
the subject enough to know what was likely to be---%
it would be ungenerous, it would be really wrong to
expose her to it.  Her feelings ought to be respected.
Does it not strike you so, Fanny?  You hesitate.''

``I am sorry for Miss Crawford; but I am more sorry to see
you drawn in to do what you had resolved against, and what
you are known to think will be disagreeable to my uncle.
It will be such a triumph to the others!''

``They will not have much cause of triumph when they
see how infamously I act.  But, however, triumph there
certainly will be, and I must brave it.  But if I can be
the means of restraining the publicity of the business,
of limiting the exhibition, of concentrating our folly,
I shall be well repaid.  As I am now, I have no influence,
I can do nothing:  I have offended them, and they will
not hear me; but when I have put them in good-humour
by this concession, I am not without hopes of persuading
them to confine the representation within a much
smaller circle than they are now in the high road for.
This will be a material gain.  My object is to confine
it to Mrs.\ Rushworth and the Grants.  Will not this be
worth gaining?''

``Yes, it will be a great point.''

``But still it has not your approbation.  Can you mention
any other measure by which I have a chance of doing
equal good?''

``No, I cannot think of anything else.''

``Give me your approbation, then, Fanny.  I am not
comfortable without it.''

``Oh, cousin!''

``If you are against me, I ought to distrust myself,
and yet---But it is absolutely impossible to let Tom
go on in this way, riding about the country in quest
of anybody who can be persuaded to act---no matter whom:
the look of a gentleman is to be enough.  I thought \emph{you}
would have entered more into Miss Crawford's feelings.''

``No doubt she will be very glad.  It must be a great relief
to her,'' said Fanny, trying for greater warmth of manner.

``She never appeared more amiable than in her behaviour
to you last night.  It gave her a very strong claim
on my goodwill.''

``She \emph{was} very kind, indeed, and I am glad to have her
spared''\ldots

She could not finish the generous effusion.  Her conscience
stopt her in the middle, but Edmund was satisfied.

``I shall walk down immediately after breakfast,'' said he,
``and am sure of giving pleasure there.  And now, dear Fanny,
I will not interrupt you any longer.  You want to be reading.
But I could not be easy till I had spoken to you,
and come to a decision.  Sleeping or waking, my head
has been full of this matter all night.  It is an evil,
but I am certainly making it less than it might be.
If Tom is up, I shall go to him directly and get it over,
and when we meet at breakfast we shall be all in high
good-humour at the prospect of acting the fool together
with such unanimity.  \emph{You}, in the meanwhile, will be taking
a trip into China, I suppose.  How does Lord Macartney
go on?''---opening a volume on the table and then taking up
some others.  ``And here are Crabbe's Tales, and the Idler,
at hand to relieve you, if you tire of your great book.
I admire your little establishment exceedingly; and as
soon as I am gone, you will empty your head of all this
nonsense of acting, and sit comfortably down to your table.
But do not stay here to be cold.''

He went; but there was no reading, no China, no composure
for Fanny.  He had told her the most extraordinary,
the most inconceivable, the most unwelcome news;
and she could think of nothing else.  To be acting!
After all his objections---objections so just and so public!
After all that she had heard him say, and seen him look,
and known him to be feeling.  Could it be possible?
Edmund so inconsistent!  Was he not deceiving himself?
Was he not wrong?  Alas! it was all Miss Crawford's doing.
She had seen her influence in every speech, and was miserable.
The doubts and alarms as to her own conduct, which had previously
distressed her, and which had all slept while she listened
to him, were become of little consequence now.  This deeper
anxiety swallowed them up.  Things should take their course;
she cared not how it ended.  Her cousins might attack,
but could hardly tease her.  She was beyond their reach;
and if at last obliged to yield---no matter---it was all
misery now.



\chapter{Chapter 17}

\gintro{It was,} indeed, a triumphant day to Mr.\ Bertram and Maria.
Such a victory over Edmund's discretion had been beyond
their hopes, and was most delightful.  There was no
longer anything to disturb them in their darling project,
and they congratulated each other in private on the
jealous weakness to which they attributed the change,
with all the glee of feelings gratified in every way.
Edmund might still look grave, and say he did not like the
scheme in general, and must disapprove the play in particular;
their point was gained:  he was to act, and he was
driven to it by the force of selfish inclinations only.
Edmund had descended from that moral elevation which he
had maintained before, and they were both as much the better
as the happier for the descent.

They behaved very well, however, to \emph{him} on the occasion,
betraying no exultation beyond the lines about the corners
of the mouth, and seemed to think it as great an escape
to be quit of the intrusion of Charles Maddox, as if they
had been forced into admitting him against their inclination.
``To have it quite in their own family circle was what
they had particularly wished.  A stranger among them
would have been the destruction of all their comfort'';
and when Edmund, pursuing that idea, gave a hint of his hope
as to the limitation of the audience, they were ready,
in the complaisance of the moment, to promise anything.
It was all good-humour and encouragement.  Mrs.\ Norris
offered to contrive his dress, Mr.\ Yates assured him
that Anhalt's last scene with the Baron admitted a good
deal of action and emphasis, and Mr.\ Rushworth undertook
to count his speeches.

``Perhaps,'' said Tom, ``Fanny may be more disposed to oblige
us now.  Perhaps you may persuade \emph{her}.''

``No, she is quite determined.  She certainly will not act.''

``Oh! very well.''  And not another word was said; but Fanny
felt herself again in danger, and her indifference
to the danger was beginning to fail her already.

There were not fewer smiles at the Parsonage than at the Park
on this change in Edmund; Miss Crawford looked very lovely
in hers, and entered with such an instantaneous renewal
of cheerfulness into the whole affair as could have but
one effect on him.  ``He was certainly right in respecting
such feelings; he was glad he had determined on it.''
And the morning wore away in satisfactions very sweet,
if not very sound.  One advantage resulted from it
to Fanny:  at the earnest request of Miss Crawford,
Mrs.\ Grant had, with her usual good-humour, agreed
to undertake the part for which Fanny had been wanted;
and this was all that occurred to gladden \emph{her} heart
during the day; and even this, when imparted by Edmund,
brought a pang with it, for it was Miss Crawford to
whom she was obliged---it was Miss Crawford whose kind
exertions were to excite her gratitude, and whose merit
in making them was spoken of with a glow of admiration.
She was safe; but peace and safety were unconnected here.
Her mind had been never farther from peace.  She could
not feel that she had done wrong herself, but she was
disquieted in every other way.  Her heart and her judgment
were equally against Edmund's decision:  she could not
acquit his unsteadiness, and his happiness under it made
her wretched.  She was full of jealousy and agitation.
Miss Crawford came with looks of gaiety which seemed
an insult, with friendly expressions towards herself
which she could hardly answer calmly.  Everybody around
her was gay and busy, prosperous and important; each had
their object of interest, their part, their dress,
their favourite scene, their friends and confederates:
all were finding employment in consultations and comparisons,
or diversion in the playful conceits they suggested.
She alone was sad and insignificant:  she had no share
in anything; she might go or stay; she might be in the
midst of their noise, or retreat from it to the solitude
of the East room, without being seen or missed.  She could
almost think anything would have been preferable to this.
Mrs.\ Grant was of consequence:  \emph{her} good-nature had
honourable mention; her taste and her time were considered;
her presence was wanted; she was sought for, and attended,
and praised; and Fanny was at first in some danger
of envying her the character she had accepted.
But reflection brought better feelings, and shewed her
that Mrs.\ Grant was entitled to respect, which could never
have belonged to \emph{her}; and that, had she received even
the greatest, she could never have been easy in joining
a scheme which, considering only her uncle, she must
condemn altogether.

Fanny's heart was not absolutely the only saddened one
amongst them, as she soon began to acknowledge to herself.
Julia was a sufferer too, though not quite so blamelessly.

Henry Crawford had trifled with her feelings; but she
had very long allowed and even sought his attentions,
with a jealousy of her sister so reasonable as ought
to have been their cure; and now that the conviction
of his preference for Maria had been forced on her,
she submitted to it without any alarm for Maria's situation,
or any endeavour at rational tranquillity for herself.
She either sat in gloomy silence, wrapt in such gravity
as nothing could subdue, no curiosity touch, no wit amuse;
or allowing the attentions of Mr.\ Yates, was talking with
forced gaiety to him alone, and ridiculing the acting of
the others.

For a day or two after the affront was given,
Henry Crawford had endeavoured to do it away by the usual
attack of gallantry and compliment, but he had not cared
enough about it to persevere against a few repulses;
and becoming soon too busy with his play to have time
for more than one flirtation, he grew indifferent to
the quarrel, or rather thought it a lucky occurrence,
as quietly putting an end to what might ere long
have raised expectations in more than Mrs.\ Grant.
She was not pleased to see Julia excluded from the play,
and sitting by disregarded; but as it was not a matter
which really involved her happiness, as Henry must be the
best judge of his own, and as he did assure her, with a
most persuasive smile, that neither he nor Julia had ever
had a serious thought of each other, she could only renew
her former caution as to the elder sister, entreat him
not to risk his tranquillity by too much admiration there,
and then gladly take her share in anything that brought
cheerfulness to the young people in general, and that did
so particularly promote the pleasure of the two so dear to her.

``I rather wonder Julia is not in love with Henry,''
was her observation to Mary.

``I dare say she is,'' replied Mary coldly.  ``I imagine
both sisters are.''

``Both! no, no, that must not be.  Do not give him a hint
of it.  Think of Mr.\ Rushworth!''

``You had better tell Miss Bertram to think of Mr.\ Rushworth.
It may do \emph{her} some good.  I often think of Mr.\ Rushworth's
property and independence, and wish them in other hands;
but I never think of him.  A man might represent the county
with such an estate; a man might escape a profession
and represent the county.''

``I dare say he \emph{will} be in parliament soon.  When Sir
Thomas comes, I dare say he will be in for some borough,
but there has been nobody to put him in the way of doing
anything yet.''

``Sir Thomas is to achieve many mighty things when he
comes home,'' said Mary, after a pause.  ``Do you remember
Hawkins Browne's `Address to Tobacco,' in imitation
of Pope?---%

\begin{verse}
     Blest leaf! whose aromatic gales dispense\\
     To Templars modesty, to Parsons sense.
\end{verse}

I will parody them---%

\begin{verse}
     Blest Knight! whose dictatorial looks dispense\\
     To Children affluence, to Rushworth sense.
\end{verse}

Will not that do, Mrs.\ Grant?  Everything seems to depend
upon Sir Thomas's return.``

``You will find his consequence very just and reasonable
when you see him in his family, I assure you.  I do not think
we do so well without him.  He has a fine dignified manner,
which suits the head of such a house, and keeps everybody
in their place.  Lady Bertram seems more of a cipher
now than when he is at home; and nobody else can keep
Mrs.\ Norris in order.  But, Mary, do not fancy that Maria
Bertram cares for Henry.  I am sure \emph{Julia} does not,
or she would not have flirted as she did last night with
Mr.\ Yates; and though he and Maria are very good friends,
I think she likes Sotherton too well to be inconstant.''

``I would not give much for Mr.\ Rushworth's chance if Henry
stept in before the articles were signed.''

``If you have such a suspicion, something must be done;
and as soon as the play is all over, we will talk to him
seriously and make him know his own mind; and if he
means nothing, we will send him off, though he is Henry,
for a time.''

Julia \emph{did} suffer, however, though Mrs.\ Grant discerned
it not, and though it escaped the notice of many of her
own family likewise.  She had loved, she did love still,
and she had all the suffering which a warm temper and a
high spirit were likely to endure under the disappointment
of a dear, though irrational hope, with a strong sense
of ill-usage. Her heart was sore and angry, and she
was capable only of angry consolations.  The sister
with whom she was used to be on easy terms was now become
her greatest enemy:  they were alienated from each other;
and Julia was not superior to the hope of some distressing
end to the attentions which were still carrying on there,
some punishment to Maria for conduct so shameful towards
herself as well as towards Mr.\ Rushworth.  With no material
fault of temper, or difference of opinion, to prevent
their being very good friends while their interests
were the same, the sisters, under such a trial as this,
had not affection or principle enough to make them merciful
or just, to give them honour or compassion.  Maria felt
her triumph, and pursued her purpose, careless of Julia;
and Julia could never see Maria distinguished by Henry
Crawford without trusting that it would create jealousy,
and bring a public disturbance at last.

Fanny saw and pitied much of this in Julia; but there
was no outward fellowship between them.  Julia made
no communication, and Fanny took no liberties.  They were
two solitary sufferers, or connected only by Fanny's consciousness.

The inattention of the two brothers and the aunt to
Julia's discomposure, and their blindness to its true cause,
must be imputed to the fullness of their own minds.
They were totally preoccupied.  Tom was engrossed by
the concerns of his theatre, and saw nothing that did
not immediately relate to it.  Edmund, between his
theatrical and his real part, between Miss Crawford's
claims and his own conduct, between love and consistency,
was equally unobservant; and Mrs.\ Norris was too busy
in contriving and directing the general little matters
of the company, superintending their various dresses
with economical expedient, for which nobody thanked her,
and saving, with delighted integrity, half a crown here and
there to the absent Sir Thomas, to have leisure for watching
the behaviour, or guarding the happiness of his daughters.



\chapter{Chapter 18}

\gintro{Everything} was now in a regular train:  theatre, actors,
actresses, and dresses, were all getting forward;
but though no other great impediments arose, Fanny found,
before many days were past, that it was not all uninterrupted
enjoyment to the party themselves, and that she had
not to witness the continuance of such unanimity and
delight as had been almost too much for her at first.
Everybody began to have their vexation.  Edmund had many.
Entirely against \emph{his} judgment, a scene-painter arrived
from town, and was at work, much to the increase
of the expenses, and, what was worse, of the eclat of
their proceedings; and his brother, instead of being really
guided by him as to the privacy of the representation,
was giving an invitation to every family who came in his way.
Tom himself began to fret over the scene-painter's
slow progress, and to feel the miseries of waiting.
He had learned his part---all his parts, for he took
every trifling one that could be united with the Butler,
and began to be impatient to be acting; and every day
thus unemployed was tending to increase his sense of
the insignificance of all his parts together, and make
him more ready to regret that some other play had not been chosen.

Fanny, being always a very courteous listener, and often
the only listener at hand, came in for the complaints
and the distresses of most of them.  \emph{She} knew that
Mr.\ Yates was in general thought to rant dreadfully;
that Mr.\ Yates was disappointed in Henry Crawford;
that Tom Bertram spoke so quick he would be unintelligible;
that Mrs.\ Grant spoiled everything by laughing; that Edmund
was behindhand with his part, and that it was misery
to have anything to do with Mr.\ Rushworth, who was wanting
a prompter through every speech.  She knew, also, that poor
Mr.\ Rushworth could seldom get anybody to rehearse with him:
\emph{his} complaint came before her as well as the rest;
and so decided to her eye was her cousin Maria's
avoidance of him, and so needlessly often the rehearsal
of the first scene between her and Mr.\ Crawford, that she
had soon all the terror of other complaints from \emph{him}.
So far from being all satisfied and all enjoying,
she found everybody requiring something they had not,
and giving occasion of discontent to the others.
Everybody had a part either too long or too short;
nobody would attend as they ought; nobody would remember on
which side they were to come in; nobody but the complainer
would observe any directions.

Fanny believed herself to derive as much innocent enjoyment
from the play as any of them; Henry Crawford acted well,
and it was a pleasure to \emph{her} to creep into the theatre,
and attend the rehearsal of the first act, in spite of the
feelings it excited in some speeches for Maria.  Maria, she
also thought, acted well, too well; and after the first
rehearsal or two, Fanny began to be their only audience;
and sometimes as prompter, sometimes as spectator,
was often very useful.  As far as she could judge,
Mr.\ Crawford was considerably the best actor of all:
he had more confidence than Edmund, more judgment than Tom,
more talent and taste than Mr.\ Yates.  She did not like him
as a man, but she must admit him to be the best actor,
and on this point there were not many who differed from her.
Mr.\ Yates, indeed, exclaimed against his tameness
and insipidity; and the day came at last, when Mr.\ Rushworth
turned to her with a black look, and said, ``Do you think
there is anything so very fine in all this?  For the life
and soul of me, I cannot admire him; and, between ourselves,
to see such an undersized, little, mean-looking man,
set up for a fine actor, is very ridiculous in my opinion.''

From this moment there was a return of his former jealousy,
which Maria, from increasing hopes of Crawford, was at
little pains to remove; and the chances of Mr.\ Rushworth's
ever attaining to the knowledge of his two-and-forty
speeches became much less.  As to his ever making anything
\emph{tolerable} of them, nobody had the smallest idea of that
except his mother; \emph{she}, indeed, regretted that his part
was not more considerable, and deferred coming over to
Mansfield till they were forward enough in their rehearsal
to comprehend all his scenes; but the others aspired at
nothing beyond his remembering the catchword, and the first
line of his speech, and being able to follow the prompter
through the rest.  Fanny, in her pity and kindheartedness,
was at great pains to teach him how to learn, giving him
all the helps and directions in her power, trying to make
an artificial memory for him, and learning every word
of his part herself, but without his being much the forwarder.

Many uncomfortable, anxious, apprehensive feelings she
certainly had; but with all these, and other claims
on her time and attention, she was as far from finding
herself without employment or utility amongst them,
as without a companion in uneasiness; quite as far from
having no demand on her leisure as on her compassion.
The gloom of her first anticipations was proved to have
been unfounded.  She was occasionally useful to all;
she was perhaps as much at peace as any.

There was a great deal of needlework to be done, moreover,
in which her help was wanted; and that Mrs.\ Norris
thought her quite as well off as the rest, was evident
by the manner in which she claimed it---``Come, Fanny,''
she cried, ``these are fine times for you, but you must
not be always walking from one room to the other,
and doing the lookings-on at your ease, in this way;
I want you here.  I have been slaving myself till I
can hardly stand, to contrive Mr.\ Rushworth's cloak
without sending for any more satin; and now I think
you may give me your help in putting it together.
There are but three seams; you may do them in a trice.
It would be lucky for me if I had nothing but the executive
part to do.  \emph{You} are best off, I can tell you:
but if nobody did more than \emph{you}, we should not get on
very fast''

Fanny took the work very quietly, without attempting
any defence; but her kinder aunt Bertram observed on her behalf---%

``One cannot wonder, sister, that Fanny \emph{should} be delighted:
it is all new to her, you know; you and I used to be
very fond of a play ourselves, and so am I still;
and as soon as I am a little more at leisure, \emph{I} mean
to look in at their rehearsals too.  What is the play about,
Fanny? you have never told me.''

``Oh! sister, pray do not ask her now; for Fanny is not
one of those who can talk and work at the same time.
It is about Lovers' Vows.''

``I believe,'' said Fanny to her aunt Bertram, ``there will
be three acts rehearsed to-morrow evening, and that will
give you an opportunity of seeing all the actors at once.''

``You had better stay till the curtain is hung,'' interposed
Mrs.\ Norris; ``the curtain will be hung in a day or two---%
there is very little sense in a play without a curtain---%
and I am much mistaken if you do not find it draw up
into very handsome festoons.''

Lady Bertram seemed quite resigned to waiting.  Fanny did
not share her aunt's composure:  she thought of the morrow
a great deal, for if the three acts were rehearsed,
Edmund and Miss Crawford would then be acting together
for the first time; the third act would bring a scene
between them which interested her most particularly,
and which she was longing and dreading to see how they
would perform.  The whole subject of it was love---%
a marriage of love was to be described by the gentleman,
and very little short of a declaration of love be made by
the lady.

She had read and read the scene again with many painful,
many wondering emotions, and looked forward to their
representation of it as a circumstance almost too interesting.
She did not \emph{believe} they had yet rehearsed it,
even in private.

The morrow came, the plan for the evening continued,
and Fanny's consideration of it did not become less agitated.
She worked very diligently under her aunt's directions,
but her diligence and her silence concealed a very absent,
anxious mind; and about noon she made her escape with her
work to the East room, that she might have no concern
in another, and, as she deemed it, most unnecessary
rehearsal of the first act, which Henry Crawford was
just proposing, desirous at once of having her time
to herself, and of avoiding the sight of Mr.\ Rushworth.
A glimpse, as she passed through the hall, of the two
ladies walking up from the Parsonage made no change
in her wish of retreat, and she worked and meditated
in the East room, undisturbed, for a quarter of an hour,
when a gentle tap at the door was followed by the entrance
of Miss Crawford.

``Am I right?  Yes; this is the East room.  My dear
Miss Price, I beg your pardon, but I have made my way
to you on purpose to entreat your help.''

Fanny, quite surprised, endeavoured to shew herself
mistress of the room by her civilities, and looked
at the bright bars of her empty grate with concern.

``Thank you; I am quite warm, very warm.  Allow me to stay
here a little while, and do have the goodness to hear me
my third act.  I have brought my book, and if you would
but rehearse it with me, I should be \emph{so} obliged!
I came here to-day intending to rehearse it with Edmund---%
by ourselves---against the evening, but he is not in the way;
and if he \emph{were}, I do not think I could go through
it with \emph{him}, till I have hardened myself a little;
for really there is a speech or two.  You will be so good,
won't you?''

Fanny was most civil in her assurances, though she could
not give them in a very steady voice.

``Have you ever happened to look at the part I mean?''
continued Miss Crawford, opening her book.  ``Here it is.
I did not think much of it at first---but, upon my word.
There, look at \emph{that} speech, and \emph{that}, and \emph{that}.
How am I ever to look him in the face and say such things?
Could you do it?  But then he is your cousin, which makes
all the difference.  You must rehearse it with me, that I
may fancy \emph{you} him, and get on by degrees.  You \emph{have} a look
of \emph{his} sometimes.''

``Have I?  I will do my best with the greatest readiness;
but I must \emph{read} the part, for I can say very little
of it.''

``\emph{None} of it, I suppose.  You are to have the book,
of course.  Now for it.  We must have two chairs at hand
for you to bring forward to the front of the stage.
There---very good school-room chairs, not made for a theatre,
I dare say; much more fitted for little girls to sit and
kick their feet against when they are learning a lesson.
What would your governess and your uncle say to see them
used for such a purpose?  Could Sir Thomas look in upon us
just now, he would bless himself, for we are rehearsing
all over the house.  Yates is storming away in the
dining-room. I heard him as I came upstairs, and the theatre
is engaged of course by those indefatigable rehearsers,
Agatha and Frederick.  If \emph{they} are not perfect,
I \emph{shall} be surprised.  By the bye, I looked in upon
them five minutes ago, and it happened to be exactly at
one of the times when they were trying \emph{not} to embrace,
and Mr.\ Rushworth was with me.  I thought he began to look
a little queer, so I turned it off as well as I could,
by whispering to him, `We shall have an excellent Agatha;
there is something so \emph{maternal} in her manner,
so completely \emph{maternal} in her voice and countenance.'
Was not that well done of me?  He brightened up directly.
Now for my soliloquy.''

She began, and Fanny joined in with all the modest feeling
which the idea of representing Edmund was so strongly
calculated to inspire; but with looks and voice so truly
feminine as to be no very good picture of a man.  With such
an Anhalt, however, Miss Crawford had courage enough;
and they had got through half the scene, when a tap at
the door brought a pause, and the entrance of Edmund,
the next moment, suspended it all.

Surprise, consciousness, and pleasure appeared in each
of the three on this unexpected meeting; and as Edmund
was come on the very same business that had brought
Miss Crawford, consciousness and pleasure were likely
to be more than momentary in \emph{them}.  He too had his book,
and was seeking Fanny, to ask her to rehearse with him,
and help him to prepare for the evening, without knowing
Miss Crawford to be in the house; and great was the joy and
animation of being thus thrown together, of comparing schemes,
and sympathising in praise of Fanny's kind offices.

\emph{She} could not equal them in their warmth.  \emph{Her} spirits
sank under the glow of theirs, and she felt herself becoming
too nearly nothing to both to have any comfort in having
been sought by either.  They must now rehearse together.
Edmund proposed, urged, entreated it, till the lady,
not very unwilling at first, could refuse no longer,
and Fanny was wanted only to prompt and observe them.
She was invested, indeed, with the office of judge and critic,
and earnestly desired to exercise it and tell them all
their faults; but from doing so every feeling within
her shrank---she could not, would not, dared not attempt it:
had she been otherwise qualified for criticism, her conscience
must have restrained her from venturing at disapprobation.
She believed herself to feel too much of it in the aggregate
for honesty or safety in particulars.  To prompt them must
be enough for her; and it was sometimes \emph{more} than enough;
for she could not always pay attention to the book.
In watching them she forgot herself; and, agitated by the
increasing spirit of Edmund's manner, had once closed
the page and turned away exactly as he wanted help.
It was imputed to very reasonable weariness, and she was
thanked and pitied; but she deserved their pity more than
she hoped they would ever surmise.  At last the scene
was over, and Fanny forced herself to add her praise to
the compliments each was giving the other; and when again
alone and able to recall the whole, she was inclined
to believe their performance would, indeed, have such
nature and feeling in it as must ensure their credit,
and make it a very suffering exhibition to herself.
Whatever might be its effect, however, she must stand
the brunt of it again that very day.

The first regular rehearsal of the three first acts
was certainly to take place in the evening:  Mrs.\ Grant
and the Crawfords were engaged to return for that purpose
as soon as they could after dinner; and every one concerned
was looking forward with eagerness.  There seemed
a general diffusion of cheerfulness on the occasion.
Tom was enjoying such an advance towards the end;
Edmund was in spirits from the morning's rehearsal,
and little vexations seemed everywhere smoothed away.
All were alert and impatient; the ladies moved soon,
the gentlemen soon followed them, and with the exception
of Lady Bertram, Mrs.\ Norris, and Julia, everybody was
in the theatre at an early hour; and having lighted it up
as well as its unfinished state admitted, were waiting only
the arrival of Mrs.\ Grant and the Crawfords to begin.

They did not wait long for the Crawfords, but there
was no Mrs.\ Grant.  She could not come.  Dr.\ Grant,
professing an indisposition, for which he had little credit
with his fair sister-in-law, could not spare his wife.

``Dr.\ Grant is ill,'' said she, with mock solemnity.
``He has been ill ever since he did not eat any of the
pheasant today.  He fancied it tough, sent away his plate,
and has been suffering ever since''.

Here was disappointment!  Mrs.\ Grant's non-attendance
was sad indeed.  Her pleasant manners and cheerful
conformity made her always valuable amongst them;
but \emph{now} she was absolutely necessary.  They could not act,
they could not rehearse with any satisfaction without her.
The comfort of the whole evening was destroyed.
What was to be done?  Tom, as Cottager, was in despair.
After a pause of perplexity, some eyes began to be
turned towards Fanny, and a voice or two to say,
``If Miss Price would be so good as to \emph{read} the part.''
She was immediately surrounded by supplications;
everybody asked it; even Edmund said, ``Do, Fanny, if it is
not \emph{very} disagreeable to you.''

But Fanny still hung back.  She could not endure the idea
of it.  Why was not Miss Crawford to be applied to as well?
Or why had not she rather gone to her own room,
as she had felt to be safest, instead of attending
the rehearsal at all?  She had known it would irritate
and distress her; she had known it her duty to keep away.
She was properly punished.

``You have only to \emph{read} the part,'' said Henry Crawford,
with renewed entreaty.

``And I do believe she can say every word of it,''
added Maria, ``for she could put Mrs.\ Grant right the other
day in twenty places.  Fanny, I am sure you know the part.''

Fanny could not say she did \emph{not}; and as they all persevered,
as Edmund repeated his wish, and with a look of even
fond dependence on her good-nature, she must yield.
She would do her best.  Everybody was satisfied; and she
was left to the tremors of a most palpitating heart,
while the others prepared to begin.

They \emph{did} begin; and being too much engaged in their
own noise to be struck by an unusual noise in the other
part of the house, had proceeded some way when the door
of the room was thrown open, and Julia, appearing at it,
with a face all aghast, exclaimed, ``My father is come!
He is in the hall at this moment.''



\chapter{Chapter 19}

\gintro{How is the consternation} of the party to be described?
To the greater number it was a moment of absolute horror.
Sir Thomas in the house!  All felt the instantaneous conviction.
Not a hope of imposition or mistake was harboured anywhere.
Julia's looks were an evidence of the fact that made
it indisputable; and after the first starts and exclamations,
not a word was spoken for half a minute:  each with
an altered countenance was looking at some other,
and almost each was feeling it a stroke the most unwelcome,
most ill-timed, most appalling!  Mr.\ Yates might consider
it only as a vexatious interruption for the evening,
and Mr.\ Rushworth might imagine it a blessing; but every
other heart was sinking under some degree of self-condemnation
or undefined alarm, every other heart was suggesting,
``What will become of us? what is to be done now?''
It was a terrible pause; and terrible to every ear were the
corroborating sounds of opening doors and passing footsteps.

Julia was the first to move and speak again.  Jealousy and
bitterness had been suspended:  selfishness was lost
in the common cause; but at the moment of her appearance,
Frederick was listening with looks of devotion to
Agatha's narrative, and pressing her hand to his heart;
and as soon as she could notice this, and see that,
in spite of the shock of her words, he still kept his
station and retained her sister's hand, her wounded
heart swelled again with injury, and looking as red
as she had been white before, she turned out of the room,
saying, ``\emph{I} need not be afraid of appearing before him.''

Her going roused the rest; and at the same moment
the two brothers stepped forward, feeling the necessity
of doing something.  A very few words between them
were sufficient.  The case admitted no difference
of opinion:  they must go to the drawing-room directly.
Maria joined them with the same intent, just then the
stoutest of the three; for the very circumstance which
had driven Julia away was to her the sweetest support.
Henry Crawford's retaining her hand at such a moment,
a moment of such peculiar proof and importance,
was worth ages of doubt and anxiety.  She hailed it
as an earnest of the most serious determination, and was
equal even to encounter her father.  They walked off,
utterly heedless of Mr.\ Rushworth's repeated question of,
``Shall I go too?  Had not I better go too?  Will not it
be right for me to go too?'' but they were no sooner
through the door than Henry Crawford undertook to answer
the anxious inquiry, and, encouraging him by all means
to pay his respects to Sir Thomas without delay,
sent him after the others with delighted haste.

Fanny was left with only the Crawfords and Mr.\ Yates.
She had been quite overlooked by her cousins; and as her
own opinion of her claims on Sir Thomas's affection
was much too humble to give her any idea of classing
herself with his children, she was glad to remain
behind and gain a little breathing-time. Her agitation
and alarm exceeded all that was endured by the rest,
by the right of a disposition which not even innocence
could keep from suffering.  She was nearly fainting:
all her former habitual dread of her uncle was returning,
and with it compassion for him and for almost every one
of the party on the development before him, with solicitude
on Edmund's account indescribable.  She had found a seat,
where in excessive trembling she was enduring all these
fearful thoughts, while the other three, no longer under
any restraint, were giving vent to their feelings of vexation,
lamenting over such an unlooked-for premature arrival
as a most untoward event, and without mercy wishing
poor Sir Thomas had been twice as long on his passage,
or were still in Antigua.

The Crawfords were more warm on the subject than Mr.\ Yates,
from better understanding the family, and judging more
clearly of the mischief that must ensue.  The ruin of
the play was to them a certainty:  they felt the total
destruction of the scheme to be inevitably at hand;
while Mr.\ Yates considered it only as a temporary interruption,
a disaster for the evening, and could even suggest the
possibility of the rehearsal being renewed after tea,
when the bustle of receiving Sir Thomas were over,
and he might be at leisure to be amused by it.
The Crawfords laughed at the idea; and having soon
agreed on the propriety of their walking quietly home
and leaving the family to themselves, proposed Mr.\ Yates's
accompanying them and spending the evening at the Parsonage.
But Mr.\ Yates, having never been with those who thought much
of parental claims, or family confidence, could not perceive
that anything of the kind was necessary; and therefore,
thanking them, said, ``he preferred remaining where he was,
that he might pay his respects to the old gentleman
handsomely since he \emph{was} come; and besides, he did not
think it would be fair by the others to have everybody run away.''

Fanny was just beginning to collect herself,
and to feel that if she staid longer behind it might
seem disrespectful, when this point was settled, and being
commissioned with the brother and sister's apology,
saw them preparing to go as she quitted the room herself
to perform the dreadful duty of appearing before her uncle.

Too soon did she find herself at the drawing-room door;
and after pausing a moment for what she knew would not come,
for a courage which the outside of no door had ever supplied
to her, she turned the lock in desperation, and the lights
of the drawing-room, and all the collected family,
were before her.  As she entered, her own name caught
her ear.  Sir Thomas was at that moment looking round him,
and saying, ``But where is Fanny?  Why do not I see
my little Fanny?''---and on perceiving her, came forward
with a kindness which astonished and penetrated her,
calling her his dear Fanny, kissing her affectionately,
and observing with decided pleasure how much she was grown!
Fanny knew not how to feel, nor where to look.  She was
quite oppressed.  He had never been so kind, so \emph{very}
kind to her in his life.  His manner seemed changed,
his voice was quick from the agitation of joy; and all that
had been awful in his dignity seemed lost in tenderness.
He led her nearer the light and looked at her again---%
inquired particularly after her health, and then,
correcting himself, observed that he need not inquire,
for her appearance spoke sufficiently on that point.  A fine
blush having succeeded the previous paleness of her face,
he was justified in his belief of her equal improvement
in health and beauty.  He inquired next after her family,
especially William:  and his kindness altogether was such
as made her reproach herself for loving him so little,
and thinking his return a misfortune; and when, on having
courage to lift her eyes to his face, she saw that he
was grown thinner, and had the burnt, fagged, worn look
of fatigue and a hot climate, every tender feeling
was increased, and she was miserable in considering
how much unsuspected vexation was probably ready to burst
on him.

Sir Thomas was indeed the life of the party, who at
his suggestion now seated themselves round the fire.
He had the best right to be the talker; and the delight
of his sensations in being again in his own house,
in the centre of his family, after such a separation,
made him communicative and chatty in a very unusual degree;
and he was ready to give every information as to his voyage,
and answer every question of his two sons almost before
it was put.  His business in Antigua had latterly been
prosperously rapid, and he came directly from Liverpool,
having had an opportunity of making his passage thither
in a private vessel, instead of waiting for the packet;
and all the little particulars of his proceedings and events,
his arrivals and departures, were most promptly delivered,
as he sat by Lady Bertram and looked with heartfelt
satisfaction on the faces around him---interrupting himself
more than once, however, to remark on his good fortune
in finding them all at home---coming unexpectedly as he did---%
all collected together exactly as he could have wished,
but dared not depend on.  Mr.\ Rushworth was not forgotten:
a most friendly reception and warmth of hand-shaking
had already met him, and with pointed attention he was
now included in the objects most intimately connected
with Mansfield.  There was nothing disagreeable in
Mr.\ Rushworth's appearance, and Sir Thomas was liking
him already.

By not one of the circle was he listened to with such unbroken,
unalloyed enjoyment as by his wife, who was really
extremely happy to see him, and whose feelings were
so warmed by his sudden arrival as to place her nearer
agitation than she had been for the last twenty years.
She had been \emph{almost} fluttered for a few minutes,
and still remained so sensibly animated as to put away
her work, move Pug from her side, and give all her
attention and all the rest of her sofa to her husband.
She had no anxieties for anybody to cloud \emph{her} pleasure:
her own time had been irreproachably spent during his absence:
she had done a great deal of carpet-work, and made many
yards of fringe; and she would have answered as freely
for the good conduct and useful pursuits of all the young
people as for her own.  It was so agreeable to her to see
him again, and hear him talk, to have her ear amused
and her whole comprehension filled by his narratives,
that she began particularly to feel how dreadfully she
must have missed him, and how impossible it would have
been for her to bear a lengthened absence.

Mrs.\ Norris was by no means to be compared in happiness
to her sister.  Not that \emph{she} was incommoded by many
fears of Sir Thomas's disapprobation when the present
state of his house should be known, for her judgment
had been so blinded that, except by the instinctive
caution with which she had whisked away Mr.\ Rushworth's
pink satin cloak as her brother-in-law entered,
she could hardly be said to shew any sign of alarm;
but she was vexed by the \emph{manner} of his return.
It had left her nothing to do.  Instead of being sent
for out of the room, and seeing him first, and having
to spread the happy news through the house, Sir Thomas,
with a very reasonable dependence, perhaps, on the nerves
of his wife and children, had sought no confidant but
the butler, and had been following him almost instantaneously
into the drawing-room. Mrs.\ Norris felt herself defrauded
of an office on which she had always depended, whether his
arrival or his death were to be the thing unfolded;
and was now trying to be in a bustle without having
anything to bustle about, and labouring to be important
where nothing was wanted but tranquillity and silence.
Would Sir Thomas have consented to eat, she might have gone
to the housekeeper with troublesome directions, and insulted
the footmen with injunctions of despatch; but Sir Thomas
resolutely declined all dinner:  he would take nothing,
nothing till tea came---he would rather wait for tea.
Still Mrs.\ Norris was at intervals urging something different;
and in the most interesting moment of his passage to England,
when the alarm of a French privateer was at the height,
she burst through his recital with the proposal of soup.
``Sure, my dear Sir Thomas, a basin of soup would be
a much better thing for you than tea.  Do have a basin
of soup.''

Sir Thomas could not be provoked.  ``Still the same
anxiety for everybody's comfort, my dear Mrs.\ Norris,''
was his answer.  ``But indeed I would rather have nothing
but tea.''

``Well, then, Lady Bertram, suppose you speak for
tea directly; suppose you hurry Baddeley a little;
he seems behindhand to-night.'' She carried this point,
and Sir Thomas's narrative proceeded.

At length there was a pause.  His immediate communications
were exhausted, and it seemed enough to be looking joyfully
around him, now at one, now at another of the beloved circle;
but the pause was not long:  in the elation of her
spirits Lady Bertram became talkative, and what were
the sensations of her children upon hearing her say,
``How do you think the young people have been amusing
themselves lately, Sir Thomas?  They have been acting.
We have been all alive with acting.''

``Indeed! and what have you been acting?''

``Oh! they'll tell you all about it.''

``The \emph{all} will soon be told,'' cried Tom hastily,
and with affected unconcern; ``but it is not worth
while to bore my father with it now.  You will hear
enough of it to-morrow, sir.  We have just been trying,
by way of doing something, and amusing my mother,
just within the last week, to get up a few scenes,
a mere trifle.  We have had such incessant rains almost
since October began, that we have been nearly confined
to the house for days together.  I have hardly taken out
a gun since the 3rd.  Tolerable sport the first three days,
but there has been no attempting anything since.
The first day I went over Mansfield Wood, and Edmund took
the copses beyond Easton, and we brought home six brace
between us, and might each have killed six times as many,
but we respect your pheasants, sir, I assure you,
as much as you could desire.  I do not think you will find
your woods by any means worse stocked than they were.
\emph{I} never saw Mansfield Wood so full of pheasants in my
life as this year.  I hope you will take a day's sport
there yourself, sir, soon.''

For the present the danger was over, and Fanny's sick
feelings subsided; but when tea was soon afterwards
brought in, and Sir Thomas, getting up, said that he found
that he could not be any longer in the house without
just looking into his own dear room, every agitation
was returning.  He was gone before anything had been
said to prepare him for the change he must find there;
and a pause of alarm followed his disappearance.
Edmund was the first to speak---%

``Something must be done,'' said he.

``It is time to think of our visitors,'' said Maria,
still feeling her hand pressed to Henry Crawford's heart,
and caring little for anything else.  ``Where did you leave
Miss Crawford, Fanny?''

Fanny told of their departure, and delivered their message.

``Then poor Yates is all alone,'' cried Tom.  ``I will go
and fetch him.  He will be no bad assistant when it
all comes out.''

To the theatre he went, and reached it just in time to
witness the first meeting of his father and his friend.
Sir Thomas had been a good deal surprised to find candles
burning in his room; and on casting his eye round it,
to see other symptoms of recent habitation and a general
air of confusion in the furniture.  The removal of the
bookcase from before the billiard-room door struck
him especially, but he had scarcely more than time
to feel astonished at all this, before there were sounds
from the billiard-room to astonish him still farther.
Some one was talking there in a very loud accent; he did
not know the voice---more than talking---almost hallooing.
He stepped to the door, rejoicing at that moment in having
the means of immediate communication, and, opening it,
found himself on the stage of a theatre, and opposed
to a ranting young man, who appeared likely to knock him
down backwards.  At the very moment of Yates perceiving
Sir Thomas, and giving perhaps the very best start he
had ever given in the whole course of his rehearsals,
Tom Bertram entered at the other end of the room;
and never had he found greater difficulty in keeping
his countenance.  His father's looks of solemnity and
amazement on this his first appearance on any stage,
and the gradual metamorphosis of the impassioned Baron
Wildenheim into the well-bred and easy Mr.\ Yates,
making his bow and apology to Sir Thomas Bertram, was such
an exhibition, such a piece of true acting, as he would
not have lost upon any account.  It would be the last---%
in all probability---the last scene on that stage; but he
was sure there could not be a finer.  The house would
close with the greatest eclat.

There was little time, however, for the indulgence
of any images of merriment.  It was necessary for him
to step forward, too, and assist the introduction,
and with many awkward sensations he did his best.
Sir Thomas received Mr.\ Yates with all the appearance
of cordiality which was due to his own character,
but was really as far from pleased with the necessity of
the acquaintance as with the manner of its commencement.
Mr.\ Yates's family and connexions were sufficiently known
to him to render his introduction as the ``particular friend,''
another of the hundred particular friends of his son,
exceedingly unwelcome; and it needed all the felicity of being
again at home, and all the forbearance it could supply,
to save Sir Thomas from anger on finding himself thus
bewildered in his own house, making part of a ridiculous
exhibition in the midst of theatrical nonsense, and forced
in so untoward a moment to admit the acquaintance of a young
man whom he felt sure of disapproving, and whose easy
indifference and volubility in the course of the first
five minutes seemed to mark him the most at home of the two.

Tom understood his father's thoughts, and heartily
wishing he might be always as well disposed to give them
but partial expression, began to see, more clearly than
he had ever done before, that there might be some ground
of offence, that there might be some reason for the glance
his father gave towards the ceiling and stucco of the room;
and that when he inquired with mild gravity after the fate
of the billiard-table, he was not proceeding beyond
a very allowable curiosity.  A few minutes were enough
for such unsatisfactory sensations on each side; and Sir
Thomas having exerted himself so far as to speak a few
words of calm approbation in reply to an eager appeal
of Mr.\ Yates, as to the happiness of the arrangement,
the three gentlemen returned to the drawing-room together,
Sir Thomas with an increase of gravity which was not
lost on all.

``I come from your theatre,'' said he composedly, as he
sat down; ``I found myself in it rather unexpectedly.
Its vicinity to my own room---but in every respect, indeed,
it took me by surprise, as I had not the smallest suspicion
of your acting having assumed so serious a character.
It appears a neat job, however, as far as I could judge
by candlelight, and does my friend Christopher Jackson credit.''
And then he would have changed the subject, and sipped
his coffee in peace over domestic matters of a calmer hue;
but Mr.\ Yates, without discernment to catch Sir Thomas's meaning,
or diffidence, or delicacy, or discretion enough to allow
him to lead the discourse while he mingled among the others
with the least obtrusiveness himself, would keep him on
the topic of the theatre, would torment him with questions
and remarks relative to it, and finally would make him hear
the whole history of his disappointment at Ecclesford.
Sir Thomas listened most politely, but found much to
offend his ideas of decorum, and confirm his ill-opinion
of Mr.\ Yates's habits of thinking, from the beginning
to the end of the story; and when it was over, could give
him no other assurance of sympathy than what a slight bow conveyed.

``This was, in fact, the origin of \emph{our} acting,'' said Tom,
after a moment's thought.  ``My friend Yates brought the
infection from Ecclesford, and it spread---as those things
always spread, you know, sir---the faster, probably,
from \emph{your} having so often encouraged the sort of thing
in us formerly.  It was like treading old ground again.''

Mr.\ Yates took the subject from his friend as soon as possible,
and immediately gave Sir Thomas an account of what they
had done and were doing:  told him of the gradual
increase of their views, the happy conclusion of their
first difficulties, and present promising state of affairs;
relating everything with so blind an interest as made him
not only totally unconscious of the uneasy movements of many
of his friends as they sat, the change of countenance,
the fidget, the hem! of unquietness, but prevented him
even from seeing the expression of the face on which his
own eyes were fixed---from seeing Sir Thomas's dark brow
contract as he looked with inquiring earnestness at his
daughters and Edmund, dwelling particularly on the latter,
and speaking a language, a remonstrance, a reproof,
which \emph{he} felt at his heart.  Not less acutely was it
felt by Fanny, who had edged back her chair behind her
aunt's end of the sofa, and, screened from notice herself,
saw all that was passing before her.  Such a look
of reproach at Edmund from his father she could never
have expected to witness; and to feel that it was in any
degree deserved was an aggravation indeed.  Sir Thomas's
look implied, ``On your judgment, Edmund, I depended;
what have you been about?''  She knelt in spirit to her uncle,
and her bosom swelled to utter, ``Oh, not to \emph{him}!
Look so to all the others, but not to \emph{him}!''

Mr.\ Yates was still talking.  ``To own the truth, Sir Thomas,
we were in the middle of a rehearsal when you arrived
this evening.  We were going through the three first acts,
and not unsuccessfully upon the whole.  Our company is
now so dispersed, from the Crawfords being gone home,
that nothing more can be done to-night; but if you will
give us the honour of your company to-morrow evening,
I should not be afraid of the result.  We bespeak
your indulgence, you understand, as young performers;
we bespeak your indulgence.''

``My indulgence shall be given, sir,'' replied Sir
Thomas gravely, ``but without any other rehearsal.''
And with a relenting smile, he added, ``I come home
to be happy and indulgent.''  Then turning away towards
any or all of the rest, he tranquilly said, ``Mr.\ and Miss
Crawford were mentioned in my last letters from Mansfield.
Do you find them agreeable acquaintance?''

Tom was the only one at all ready with an answer, but he
being entirely without particular regard for either,
without jealousy either in love or acting, could speak
very handsomely of both.  ``Mr.\ Crawford was a most pleasant,
gentleman-like man; his sister a sweet, pretty, elegant,
lively girl.''

Mr.\ Rushworth could be silent no longer.  ``I do not say
he is not gentleman-like, considering; but you should
tell your father he is not above five feet eight,
or he will be expecting a well-looking man.''

Sir Thomas did not quite understand this, and looked
with some surprise at the speaker.

``If I must say what I think,'' continued Mr.\ Rushworth, ``in my
opinion it is very disagreeable to be always rehearsing.
It is having too much of a good thing.  I am not so fond
of acting as I was at first.  I think we are a great deal
better employed, sitting comfortably here among ourselves,
and doing nothing.''

Sir Thomas looked again, and then replied with an approving
smile, ``I am happy to find our sentiments on this subject
so much the same.  It gives me sincere satisfaction.
That I should be cautious and quick-sighted, and feel many
scruples which my children do \emph{not} feel, is perfectly natural;
and equally so that my value for domestic tranquillity,
for a home which shuts out noisy pleasures, should much
exceed theirs.  But at your time of life to feel all this,
is a most favourable circumstance for yourself,
and for everybody connected with you; and I am sensible
of the importance of having an ally of such weight.''

Sir Thomas meant to be giving Mr.\ Rushworth's opinion
in better words than he could find himself.  He was
aware that he must not expect a genius in Mr.\ Rushworth;
but as a well-judging, steady young man, with better notions
than his elocution would do justice to, he intended to value
him very highly.  It was impossible for many of the others
not to smile.  Mr.\ Rushworth hardly knew what to do
with so much meaning; but by looking, as he really felt,
most exceedingly pleased with Sir Thomas's good opinion,
and saying scarcely anything, he did his best towards
preserving that good opinion a little longer.



\chapter{Chapter 20}

\gintro{Edmund's} first object the next morning was to see his
father alone, and give him a fair statement of the whole
acting scheme, defending his own share in it as far only
as he could then, in a soberer moment, feel his motives
to deserve, and acknowledging, with perfect ingenuousness,
that his concession had been attended with such partial
good as to make his judgment in it very doubtful.
He was anxious, while vindicating himself, to say nothing
unkind of the others:  but there was only one amongst them
whose conduct he could mention without some necessity
of defence or palliation.  ``We have all been more or less
to blame,'' said he, ``every one of us, excepting Fanny.
Fanny is the only one who has judged rightly throughout;
who has been consistent.  \emph{Her} feelings have been steadily
against it from first to last.  She never ceased to think
of what was due to you.  You will find Fanny everything you
could wish.''

Sir Thomas saw all the impropriety of such a scheme among
such a party, and at such a time, as strongly as his son
had ever supposed he must; he felt it too much, indeed,
for many words; and having shaken hands with Edmund,
meant to try to lose the disagreeable impression,
and forget how much he had been forgotten himself as soon
as he could, after the house had been cleared of every
object enforcing the remembrance, and restored to its
proper state.  He did not enter into any remonstrance with
his other children:  he was more willing to believe they
felt their error than to run the risk of investigation.
The reproof of an immediate conclusion of everything,
the sweep of every preparation, would be sufficient.

There was one person, however, in the house, whom he could
not leave to learn his sentiments merely through his conduct.
He could not help giving Mrs.\ Norris a hint of his having
hoped that her advice might have been interposed to prevent
what her judgment must certainly have disapproved.  The young
people had been very inconsiderate in forming the plan;
they ought to have been capable of a better decision themselves;
but they were young; and, excepting Edmund, he believed,
of unsteady characters; and with greater surprise, therefore,
he must regard her acquiescence in their wrong measures,
her countenance of their unsafe amusements, than that such
measures and such amusements should have been suggested.
Mrs.\ Norris was a little confounded and as nearly being
silenced as ever she had been in her life; for she
was ashamed to confess having never seen any of the
impropriety which was so glaring to Sir Thomas, and would
not have admitted that her influence was insufficient---%
that she might have talked in vain.  Her only resource
was to get out of the subject as fast as possible, and turn
the current of Sir Thomas's ideas into a happier channel.
She had a great deal to insinuate in her own praise
as to \emph{general} attention to the interest and comfort
of his family, much exertion and many sacrifices to glance
at in the form of hurried walks and sudden removals from
her own fireside, and many excellent hints of distrust
and economy to Lady Bertram and Edmund to detail,
whereby a most considerable saving had always arisen,
and more than one bad servant been detected.  But her chief
strength lay in Sotherton.  Her greatest support and glory
was in having formed the connexion with the Rushworths.
\emph{There} she was impregnable.  She took to herself all
the credit of bringing Mr.\ Rushworth's admiration of Maria
to any effect.  ``If I had not been active,'' said she,
``and made a point of being introduced to his mother,
and then prevailed on my sister to pay the first visit,
I am as certain as I sit here that nothing would have
come of it; for Mr.\ Rushworth is the sort of amiable
modest young man who wants a great deal of encouragement,
and there were girls enough on the catch for him if we
had been idle.  But I left no stone unturned.  I was
ready to move heaven and earth to persuade my sister,
and at last I did persuade her.  You know the distance
to Sotherton; it was in the middle of winter, and the roads
almost impassable, but I did persuade her.''

``I know how great, how justly great, your influence
is with Lady Bertram and her children, and am the more
concerned that it should not have been.''

``My dear Sir Thomas, if you had seen the state of the
roads \emph{that} day!  I thought we should never have got
through them, though we had the four horses of course;
and poor old coachman would attend us, out of his great love
and kindness, though he was hardly able to sit the box
on account of the rheumatism which I had been doctoring
him for ever since Michaelmas.  I cured him at last;
but he was very bad all the winter---and this was such a day,
I could not help going to him up in his room before we set
off to advise him not to venture:  he was putting on his wig;
so I said, `Coachman, you had much better not go; your Lady
and I shall be very safe; you know how steady Stephen is,
and Charles has been upon the leaders so often now,
that I am sure there is no fear.'  But, however, I soon
found it would not do; he was bent upon going, and as I
hate to be worrying and officious, I said no more; but my
heart quite ached for him at every jolt, and when we got
into the rough lanes about Stoke, where, what with frost
and snow upon beds of stones, it was worse than anything
you can imagine, I was quite in an agony about him.
And then the poor horses too!  To see them straining away!
You know how I always feel for the horses.  And when we got
to the bottom of Sandcroft Hill, what do you think I did?
You will laugh at me; but I got out and walked up.
I did indeed.  It might not be saving them much, but it
was something, and I could not bear to sit at my ease
and be dragged up at the expense of those noble animals.
I caught a dreadful cold, but \emph{that} I did not regard.
My object was accomplished in the visit.''

``I hope we shall always think the acquaintance worth
any trouble that might be taken to establish it.
There is nothing very striking in Mr.\ Rushworth's manners,
but I was pleased last night with what appeared to be his
opinion on one subject:  his decided preference of a quiet
family party to the bustle and confusion of acting.
He seemed to feel exactly as one could wish.''

``Yes, indeed, and the more you know of him the better
you will like him.  He is not a shining character,
but he has a thousand good qualities; and is so disposed
to look up to you, that I am quite laughed at about it,
for everybody considers it as my doing.  `Upon my word,
Mrs.\ Norris,' said Mrs.\ Grant the other day, `if Mr.\ Rushworth
were a son of your own, he could not hold Sir Thomas
in greater respect.'\,''

Sir Thomas gave up the point, foiled by her evasions,
disarmed by her flattery; and was obliged to rest
satisfied with the conviction that where the present
pleasure of those she loved was at stake, her kindness
did sometimes overpower her judgment.

It was a busy morning with him.  Conversation with any
of them occupied but a small part of it.  He had to
reinstate himself in all the wonted concerns of his
Mansfield life:  to see his steward and his bailiff;
to examine and compute, and, in the intervals
of business, to walk into his stables and his gardens,
and nearest plantations; but active and methodical,
he had not only done all this before he resumed his seat
as master of the house at dinner, he had also set the
carpenter to work in pulling down what had been so lately
put up in the billiard-room, and given the scene-painter
his dismissal long enough to justify the pleasing belief
of his being then at least as far off as Northampton.
The scene-painter was gone, having spoilt only the
floor of one room, ruined all the coachman's sponges,
and made five of the under-servants idle and dissatisfied;
and Sir Thomas was in hopes that another day or two would
suffice to wipe away every outward memento of what had been,
even to the destruction of every unbound copy of Lovers'
Vows in the house, for he was burning all that met his eye.

Mr.\ Yates was beginning now to understand Sir Thomas's intentions,
though as far as ever from understanding their source.
He and his friend had been out with their guns the chief of
the morning, and Tom had taken the opportunity of explaining,
with proper apologies for his father's particularity,
what was to be expected.  Mr.\ Yates felt it as acutely
as might be supposed.  To be a second time disappointed
in the same way was an instance of very severe ill-luck;
and his indignation was such, that had it not been for delicacy
towards his friend, and his friend's youngest sister,
he believed he should certainly attack the baronet on
the absurdity of his proceedings, and argue him into a
little more rationality.  He believed this very stoutly
while he was in Mansfield Wood, and all the way home;
but there was a something in Sir Thomas, when they sat
round the same table, which made Mr.\ Yates think it wiser
to let him pursue his own way, and feel the folly of it
without opposition.  He had known many disagreeable
fathers before, and often been struck with the inconveniences
they occasioned, but never, in the whole course of his life,
had he seen one of that class so unintelligibly moral,
so infamously tyrannical as Sir Thomas.  He was not a man
to be endured but for his children's sake, and he might
be thankful to his fair daughter Julia that Mr.\ Yates
did yet mean to stay a few days longer under his roof.

The evening passed with external smoothness, though almost
every mind was ruffled; and the music which Sir Thomas
called for from his daughters helped to conceal the want
of real harmony.  Maria was in a good deal of agitation.
It was of the utmost consequence to her that Crawford
should now lose no time in declaring himself, and she
was disturbed that even a day should be gone by without
seeming to advance that point.  She had been expecting
to see him the whole morning, and all the evening, too,
was still expecting him.  Mr.\ Rushworth had set off early
with the great news for Sotherton; and she had fondly hoped
for such an immediate \emph{eclaircissement} as might save him
the trouble of ever coming back again.  But they had seen
no one from the Parsonage, not a creature, and had heard
no tidings beyond a friendly note of congratulation
and inquiry from Mrs.\ Grant to Lady Bertram.  It was
the first day for many, many weeks, in which the families
had been wholly divided.  Four-and-twenty hours had never
passed before, since August began, without bringing them
together in some way or other.  It was a sad, anxious day;
and the morrow, though differing in the sort of evil,
did by no means bring less.  A few moments of feverish
enjoyment were followed by hours of acute suffering.
Henry Crawford was again in the house:  he walked up
with Dr.\ Grant, who was anxious to pay his respects to
Sir Thomas, and at rather an early hour they were ushered
into the breakfast-room, where were most of the family.
Sir Thomas soon appeared, and Maria saw with delight
and agitation the introduction of the man she loved to
her father.  Her sensations were indefinable, and so were
they a few minutes afterwards upon hearing Henry Crawford,
who had a chair between herself and Tom, ask the latter
in an undervoice whether there were any plans for resuming
the play after the present happy interruption (with
a courteous glance at Sir Thomas), because, in that case,
he should make a point of returning to Mansfield at any time
required by the party:  he was going away immediately,
being to meet his uncle at Bath without delay; but if there
were any prospect of a renewal of Lovers' Vows, he should
hold himself positively engaged, he should break through
every other claim, he should absolutely condition with his
uncle for attending them whenever he might be wanted.
The play should not be lost by \emph{his} absence.

``From Bath, Norfolk, London, York, wherever I may be,''
said he; ``I will attend you from any place in England,
at an hour's notice.''

It was well at that moment that Tom had to speak, and not
his sister.  He could immediately say with easy fluency,
``I am sorry you are going; but as to our play, \emph{that} is
all over---entirely at an end'' (looking significantly
at his father). ``The painter was sent off yesterday,
and very little will remain of the theatre to-morrow. I knew
how \emph{that} would be from the first.  It is early for Bath.
You will find nobody there.''

``It is about my uncle's usual time.''

``When do you think of going?''

``I may, perhaps, get as far as Banbury to-day.''

``Whose stables do you use at Bath?'' was the next question;
and while this branch of the subject was under discussion,
Maria, who wanted neither pride nor resolution, was preparing
to encounter her share of it with tolerable calmness.

To her he soon turned, repeating much of what he had
already said, with only a softened air and stronger
expressions of regret.  But what availed his expressions
or his air?  He was going, and, if not voluntarily going,
voluntarily intending to stay away; for, excepting what might
be due to his uncle, his engagements were all self-imposed.
He might talk of necessity, but she knew his independence.
The hand which had so pressed hers to his heart! the hand
and the heart were alike motionless and passive now!
Her spirit supported her, but the agony of her mind was severe.
She had not long to endure what arose from listening
to language which his actions contradicted, or to bury
the tumult of her feelings under the restraint of society;
for general civilities soon called his notice from her,
and the farewell visit, as it then became openly acknowledged,
was a very short one.  He was gone---he had touched her
hand for the last time, he had made his parting bow,
and she might seek directly all that solitude could do
for her.  Henry Crawford was gone, gone from the house,
and within two hours afterwards from the parish;
and so ended all the hopes his selfish vanity had raised
in Maria and Julia Bertram.

Julia could rejoice that he was gone.  His presence was
beginning to be odious to her; and if Maria gained him not,
she was now cool enough to dispense with any other revenge.
She did not want exposure to be added to desertion.
Henry Crawford gone, she could even pity her sister.

With a purer spirit did Fanny rejoice in the intelligence.
She heard it at dinner, and felt it a blessing.
By all the others it was mentioned with regret;
and his merits honoured with due gradation of feeling---%
from the sincerity of Edmund's too partial regard,
to the unconcern of his mother speaking entirely by rote.
Mrs.\ Norris began to look about her, and wonder that
his falling in love with Julia had come to nothing;
and could almost fear that she had been remiss herself
in forwarding it; but with so many to care for, how was
it possible for even \emph{her} activity to keep pace with
her wishes?

Another day or two, and Mr.\ Yates was gone likewise.
In \emph{his} departure Sir Thomas felt the chief interest:
wanting to be alone with his family, the presence of a
stranger superior to Mr.\ Yates must have been irksome;
but of him, trifling and confident, idle and expensive,
it was every way vexatious.  In himself he was wearisome,
but as the friend of Tom and the admirer of Julia he
became offensive.  Sir Thomas had been quite indifferent
to Mr.\ Crawford's going or staying:  but his good
wishes for Mr.\ Yates's having a pleasant journey,
as he walked with him to the hall-door, were given with
genuine satisfaction.  Mr.\ Yates had staid to see the
destruction of every theatrical preparation at Mansfield,
the removal of everything appertaining to the play:
he left the house in all the soberness of its general
character; and Sir Thomas hoped, in seeing him out of it,
to be rid of the worst object connected with the scheme,
and the last that must be inevitably reminding him of
its existence.

Mrs.\ Norris contrived to remove one article from his sight
that might have distressed him.  The curtain, over which
she had presided with such talent and such success,
went off with her to her cottage, where she happened
to be particularly in want of green baize.



\chapter{Chapter 21}

\gintro{Sir Thomas's} return made a striking change in the ways of
the family, independent of Lovers' Vows.  Under his government,
Mansfield was an altered place.  Some members of their
society sent away, and the spirits of many others saddened---%
it was all sameness and gloom compared with the past---%
a sombre family party rarely enlivened.  There was little
intercourse with the Parsonage.  Sir Thomas, drawing back
from intimacies in general, was particularly disinclined,
at this time, for any engagements but in one quarter.
The Rushworths were the only addition to his own domestic
circle which he could solicit.

Edmund did not wonder that such should be his father's feelings,
nor could he regret anything but the exclusion of the Grants.
``But they,'' he observed to Fanny, ``have a claim.  They seem
to belong to us; they seem to be part of ourselves.
I could wish my father were more sensible of their very
great attention to my mother and sisters while he was away.
I am afraid they may feel themselves neglected.
But the truth is, that my father hardly knows them.
They had not been here a twelvemonth when he left England.
If he knew them better, he would value their society
as it deserves; for they are in fact exactly the sort
of people he would like.  We are sometimes a little
in want of animation among ourselves:  my sisters seem
out of spirits, and Tom is certainly not at his ease.
Dr.\ and Mrs.\ Grant would enliven us, and make our evenings
pass away with more enjoyment even to my father.''

``Do you think so?'' said Fanny:  ``in my opinion,
my uncle would not like \emph{any} addition.  I think he
values the very quietness you speak of, and that the
repose of his own family circle is all he wants.
And it does not appear to me that we are more serious
than we used to be---I mean before my uncle went abroad.
As well as I can recollect, it was always much the same.
There was never much laughing in his presence; or,
if there is any difference, it is not more, I think,
than such an absence has a tendency to produce at first.
There must be a sort of shyness; but I cannot recollect
that our evenings formerly were ever merry, except when
my uncle was in town.  No young people's are, I suppose,
when those they look up to are at home''.

``I believe you are right, Fanny,'' was his reply, after a
short consideration.  ``I believe our evenings are rather
returned to what they were, than assuming a new character.
The novelty was in their being lively.  Yet, how strong
the impression that only a few weeks will give!
I have been feeling as if we had never lived so before.''

``I suppose I am graver than other people,'' said Fanny.
``The evenings do not appear long to me.  I love to hear
my uncle talk of the West Indies.  I could listen to him
for an hour together.  It entertains \emph{me} more than many
other things have done; but then I am unlike other people,
I dare say.''

``Why should you dare say \emph{that}?'' (smiling). ``Do you
want to be told that you are only unlike other people
in being more wise and discreet?  But when did you,
or anybody, ever get a compliment from me, Fanny?
Go to my father if you want to be complimented.
He will satisfy you.  Ask your uncle what he thinks,
and you will hear compliments enough:  and though they
may be chiefly on your person, you must put up with it,
and trust to his seeing as much beauty of mind in time.''

Such language was so new to Fanny that it quite embarrassed her.

``Your uncle thinks you very pretty, dear Fanny---%
and that is the long and the short of the matter.
Anybody but myself would have made something more of it,
and anybody but you would resent that you had not been
thought very pretty before; but the truth is, that your
uncle never did admire you till now---and now he does.
Your complexion is so improved!---and you have gained
so much countenance!---and your figure---nay, Fanny, do not
turn away about it---it is but an uncle.  If you cannot
bear an uncle's admiration, what is to become of you?
You must really begin to harden yourself to the idea of
being worth looking at.  You must try not to mind growing
up into a pretty woman.''

``Oh! don't talk so, don't talk so,'' cried Fanny,
distressed by more feelings than he was aware of; but seeing
that she was distressed, he had done with the subject,
and only added more seriously---%

``Your uncle is disposed to be pleased with you in
every respect; and I only wish you would talk to him more.
You are one of those who are too silent in the evening circle.''

``But I do talk to him more than I used.  I am sure I do.
Did not you hear me ask him about the slave-trade
last night?''

``I did---and was in hopes the question would be followed
up by others.  It would have pleased your uncle to be
inquired of farther.''

``And I longed to do it---but there was such a dead silence!
And while my cousins were sitting by without speaking a word,
or seeming at all interested in the subject, I did not like---%
I thought it would appear as if I wanted to set myself
off at their expense, by shewing a curiosity and pleasure
in his information which he must wish his own daughters
to feel.''

``Miss Crawford was very right in what she said of you
the other day:  that you seemed almost as fearful of notice
and praise as other women were of neglect.  We were talking
of you at the Parsonage, and those were her words.
She has great discernment.  I know nobody who distinguishes
characters better.  For so young a woman it is remarkable!
She certainly understands \emph{you} better than you are
understood by the greater part of those who have known you
so long; and with regard to some others, I can perceive,
from occasional lively hints, the unguarded expressions
of the moment, that she could define \emph{many} as accurately,
did not delicacy forbid it.  I wonder what she thinks
of my father!  She must admire him as a fine-looking man,
with most gentlemanlike, dignified, consistent manners;
but perhaps, having seen him so seldom, his reserve
may be a little repulsive.  Could they be much together,
I feel sure of their liking each other.  He would enjoy
her liveliness and she has talents to value his powers.
I wish they met more frequently!  I hope she does not suppose
there is any dislike on his side.''

``She must know herself too secure of the regard of all
the rest of you,'' said Fanny, with half a sigh, ``to have
any such apprehension.  And Sir Thomas's wishing just at
first to be only with his family, is so very natural,
that she can argue nothing from that.  After a little while,
I dare say, we shall be meeting again in the same sort
of way, allowing for the difference of the time of year.''

``This is the first October that she has passed in the country
since her infancy.  I do not call Tunbridge or Cheltenham
the country; and November is a still more serious month,
and I can see that Mrs.\ Grant is very anxious for her
not finding Mansfield dull as winter comes on.''

Fanny could have said a great deal, but it was safer to
say nothing, and leave untouched all Miss Crawford's resources---%
her accomplishments, her spirits, her importance,
her friends, lest it should betray her into any observations
seemingly unhandsome.  Miss Crawford's kind opinion
of herself deserved at least a grateful forbearance,
and she began to talk of something else.

``To-morrow, I think, my uncle dines at Sotherton, and you
and Mr.\ Bertram too.  We shall be quite a small party at home.
I hope my uncle may continue to like Mr.\ Rushworth.''

``That is impossible, Fanny.  He must like him less
after to-morrow's visit, for we shall be five hours
in his company.  I should dread the stupidity of the day,
if there were not a much greater evil to follow---%
the impression it must leave on Sir Thomas.  He cannot much
longer deceive himself.  I am sorry for them all, and would
give something that Rushworth and Maria had never met.''

In this quarter, indeed, disappointment was impending
over Sir Thomas.  Not all his good-will for Mr.\ Rushworth,
not all Mr.\ Rushworth's deference for him, could prevent
him from soon discerning some part of the truth---%
that Mr.\ Rushworth was an inferior young man, as ignorant
in business as in books, with opinions in general unfixed,
and without seeming much aware of it himself.

He had expected a very different son-in-law; and beginning
to feel grave on Maria's account, tried to understand
\emph{her} feelings.  Little observation there was necessary
to tell him that indifference was the most favourable
state they could be in.  Her behaviour to Mr.\ Rushworth
was careless and cold.  She could not, did not like him.
Sir Thomas resolved to speak seriously to her.
Advantageous as would be the alliance, and long standing
and public as was the engagement, her happiness must not be
sacrificed to it.  Mr.\ Rushworth had, perhaps, been accepted
on too short an acquaintance, and, on knowing him better,
she was repenting.

With solemn kindness Sir Thomas addressed her:  told her
his fears, inquired into her wishes, entreated her to be
open and sincere, and assured her that every inconvenience
should be braved, and the connexion entirely given up,
if she felt herself unhappy in the prospect of it.
He would act for her and release her.  Maria had a moment's
struggle as she listened, and only a moment's: when her
father ceased, she was able to give her answer immediately,
decidedly, and with no apparent agitation.  She thanked
him for his great attention, his paternal kindness, but he
was quite mistaken in supposing she had the smallest desire
of breaking through her engagement, or was sensible of any
change of opinion or inclination since her forming it.
She had the highest esteem for Mr.\ Rushworth's character
and disposition, and could not have a doubt of her happiness with
him.

Sir Thomas was satisfied; too glad to be satisfied,
perhaps, to urge the matter quite so far as his judgment
might have dictated to others.  It was an alliance which
he could not have relinquished without pain; and thus
he reasoned.  Mr.\ Rushworth was young enough to improve.
Mr.\ Rushworth must and would improve in good society;
and if Maria could now speak so securely of her happiness
with him, speaking certainly without the prejudice,
the blindness of love, she ought to be believed.
Her feelings, probably, were not acute; he had never
supposed them to be so; but her comforts might not
be less on that account; and if she could dispense
with seeing her husband a leading, shining character,
there would certainly be everything else in her favour.
A well-disposed young woman, who did not marry for love,
was in general but the more attached to her own family;
and the nearness of Sotherton to Mansfield must naturally hold
out the greatest temptation, and would, in all probability,
be a continual supply of the most amiable and innocent enjoyments.
Such and such-like were the reasonings of Sir Thomas,
happy to escape the embarrassing evils of a rupture,
the wonder, the reflections, the reproach that must
attend it; happy to secure a marriage which would bring
him such an addition of respectability and influence,
and very happy to think anything of his daughter's
disposition that was most favourable for the purpose.

To her the conference closed as satisfactorily as to him.
She was in a state of mind to be glad that she had secured
her fate beyond recall:  that she had pledged herself
anew to Sotherton; that she was safe from the possibility
of giving Crawford the triumph of governing her actions,
and destroying her prospects; and retired in proud resolve,
determined only to behave more cautiously to Mr.\ Rushworth
in future, that her father might not be again suspecting her.

Had Sir Thomas applied to his daughter within the first
three or four days after Henry Crawford's leaving Mansfield,
before her feelings were at all tranquillised, before she
had given up every hope of him, or absolutely resolved on
enduring his rival, her answer might have been different;
but after another three or four days, when there was no return,
no letter, no message, no symptom of a softened heart,
no hope of advantage from separation, her mind became
cool enough to seek all the comfort that pride and self
revenge could give.

Henry Crawford had destroyed her happiness, but he
should not know that he had done it; he should not
destroy her credit, her appearance, her prosperity, too.
He should not have to think of her as pining in the
retirement of Mansfield for \emph{him}, rejecting Sotherton
and London, independence and splendour, for \emph{his} sake.
Independence was more needful than ever; the want of it
at Mansfield more sensibly felt.  She was less and less
able to endure the restraint which her father imposed.
The liberty which his absence had given was now become
absolutely necessary.  She must escape from him and Mansfield
as soon as possible, and find consolation in fortune
and consequence, bustle and the world, for a wounded spirit.
Her mind was quite determined, and varied not.

To such feelings delay, even the delay of much preparation,
would have been an evil, and Mr.\ Rushworth could hardly
be more impatient for the marriage than herself.
In all the important preparations of the mind she
was complete:  being prepared for matrimony by an hatred
of home, restraint, and tranquillity; by the misery
of disappointed affection, and contempt of the man she
was to marry.  The rest might wait.  The preparations
of new carriages and furniture might wait for London
and spring, when her own taste could have fairer play.

The principals being all agreed in this respect, it soon
appeared that a very few weeks would be sufficient
for such arrangements as must precede the wedding.

Mrs.\ Rushworth was quite ready to retire, and make way for
the fortunate young woman whom her dear son had selected;
and very early in November removed herself, her maid,
her footman, and her chariot, with true dowager propriety,
to Bath, there to parade over the wonders of Sotherton
in her evening parties; enjoying them as thoroughly,
perhaps, in the animation of a card-table, as she had
ever done on the spot; and before the middle of the same
month the ceremony had taken place which gave Sotherton
another mistress.

It was a very proper wedding.  The bride was elegantly dressed;
the two bridesmaids were duly inferior; her father gave
her away; her mother stood with salts in her hand,
expecting to be agitated; her aunt tried to cry;
and the service was impressively read by Dr.\ Grant.
Nothing could be objected to when it came under the
discussion of the neighbourhood, except that the carriage
which conveyed the bride and bridegroom and Julia
from the church-door to Sotherton was the same chaise
which Mr.\ Rushworth had used for a twelvemonth before.
In everything else the etiquette of the day might stand
the strictest investigation.

It was done, and they were gone.  Sir Thomas felt as an
anxious father must feel, and was indeed experiencing much
of the agitation which his wife had been apprehensive
of for herself, but had fortunately escaped.  Mrs.\ Norris,
most happy to assist in the duties of the day,
by spending it at the Park to support her sister's spirits,
and drinking the health of Mr.\ and Mrs.\ Rushworth in
a supernumerary glass or two, was all joyous delight;
for she had made the match; she had done everything;
and no one would have supposed, from her confident triumph,
that she had ever heard of conjugal infelicity in her life,
or could have the smallest insight into the disposition
of the niece who had been brought up under her eye.

The plan of the young couple was to proceed,
after a few days, to Brighton, and take a house there
for some weeks.  Every public place was new to Maria,
and Brighton is almost as gay in winter as in summer.
When the novelty of amusement there was over, it would
be time for the wider range of London.

Julia was to go with them to Brighton.  Since rivalry
between the sisters had ceased, they had been gradually
recovering much of their former good understanding;
and were at least sufficiently friends to make each of them
exceedingly glad to be with the other at such a time.
Some other companion than Mr.\ Rushworth was of the first
consequence to his lady; and Julia was quite as eager
for novelty and pleasure as Maria, though she might not
have struggled through so much to obtain them, and could
better bear a subordinate situation.

Their departure made another material change at Mansfield,
a chasm which required some time to fill up.  The family
circle became greatly contracted; and though the Miss
Bertrams had latterly added little to its gaiety,
they could not but be missed.  Even their mother missed them;
and how much more their tenderhearted cousin, who wandered
about the house, and thought of them, and felt for them,
with a degree of affectionate regret which they had never
done much to deserve!



\chapter{Chapter 22}

\gintro{Fanny's} consequence increased on the departure of
her cousins.  Becoming, as she then did, the only young
woman in the drawing-room, the only occupier of that
interesting division of a family in which she had hitherto
held so humble a third, it was impossible for her not
to be more looked at, more thought of and attended to,
than she had ever been before; and ``Where is Fanny?''
became no uncommon question, even without her being
wanted for any one's convenience.

Not only at home did her value increase, but at the
Parsonage too.  In that house, which she had hardly
entered twice a year since Mr.\ Norris's death, she became
a welcome, an invited guest, and in the gloom and dirt
of a November day, most acceptable to Mary Crawford.
Her visits there, beginning by chance, were continued
by solicitation.  Mrs.\ Grant, really eager to get any
change for her sister, could, by the easiest self-deceit,
persuade herself that she was doing the kindest thing
by Fanny, and giving her the most important opportunities
of improvement in pressing her frequent calls.

Fanny, having been sent into the village on some errand
by her aunt Norris, was overtaken by a heavy shower close
to the Parsonage; and being descried from one of the
windows endeavouring to find shelter under the branches
and lingering leaves of an oak just beyond their premises,
was forced, though not without some modest reluctance on
her part, to come in.  A civil servant she had withstood;
but when Dr.\ Grant himself went out with an umbrella,
there was nothing to be done but to be very much ashamed,
and to get into the house as fast as possible; and to poor
Miss Crawford, who had just been contemplating the dismal
rain in a very desponding state of mind, sighing over
the ruin of all her plan of exercise for that morning,
and of every chance of seeing a single creature beyond
themselves for the next twenty-four hours, the sound of
a little bustle at the front door, and the sight of Miss
Price dripping with wet in the vestibule, was delightful.
The value of an event on a wet day in the country was
most forcibly brought before her.  She was all alive
again directly, and among the most active in being useful
to Fanny, in detecting her to be wetter than she would at
first allow, and providing her with dry clothes; and Fanny,
after being obliged to submit to all this attention,
and to being assisted and waited on by mistresses
and maids, being also obliged, on returning downstairs,
to be fixed in their drawing-room for an hour while
the rain continued, the blessing of something fresh
to see and think of was thus extended to Miss Crawford,
and might carry on her spirits to the period of dressing
and dinner.

The two sisters were so kind to her, and so pleasant,
that Fanny might have enjoyed her visit could she have
believed herself not in the way, and could she have
foreseen that the weather would certainly clear at the
end of the hour, and save her from the shame of having
Dr.\ Grant's carriage and horses out to take her home,
with which she was threatened.  As to anxiety for any alarm
that her absence in such weather might occasion at home,
she had nothing to suffer on that score; for as her being
out was known only to her two aunts, she was perfectly
aware that none would be felt, and that in whatever cottage
aunt Norris might chuse to establish her during the rain,
her being in such cottage would be indubitable to aunt Bertram.

It was beginning to look brighter, when Fanny,
observing a harp in the room, asked some questions about it,
which soon led to an acknowledgment of her wishing very
much to hear it, and a confession, which could hardly
be believed, of her having never yet heard it since its
being in Mansfield.  To Fanny herself it appeared a very
simple and natural circumstance.  She had scarcely ever
been at the Parsonage since the instrument's arrival,
there had been no reason that she should; but Miss Crawford,
calling to mind an early expressed wish on the subject,
was concerned at her own neglect; and ``Shall I play
to you now?'' and ``What will you have?'' were questions
immediately following with the readiest good-humour.

She played accordingly; happy to have a new listener,
and a listener who seemed so much obliged, so full
of wonder at the performance, and who shewed herself
not wanting in taste.  She played till Fanny's eyes,
straying to the window on the weather's being evidently fair,
spoke what she felt must be done.

``Another quarter of an hour,'' said Miss Crawford, ``and we
shall see how it will be.  Do not run away the first
moment of its holding up.  Those clouds look alarming.''

``But they are passed over,'' said Fanny.  ``I have been
watching them.  This weather is all from the south.''

``South or north, I know a black cloud when I see it;
and you must not set forward while it is so threatening.
And besides, I want to play something more to you---a very
pretty piece---and your cousin Edmund's prime favourite.
You must stay and hear your cousin's favourite.''

Fanny felt that she must; and though she had not
waited for that sentence to be thinking of Edmund,
such a memento made her particularly awake to his idea,
and she fancied him sitting in that room again and again,
perhaps in the very spot where she sat now, listening with
constant delight to the favourite air, played, as it
appeared to her, with superior tone and expression;
and though pleased with it herself, and glad to like whatever
was liked by him, she was more sincerely impatient to go
away at the conclusion of it than she had been before;
and on this being evident, she was so kindly asked to
call again, to take them in her walk whenever she could,
to come and hear more of the harp, that she felt it
necessary to be done, if no objection arose at home.

Such was the origin of the sort of intimacy which took
place between them within the first fortnight after
the Miss Bertrams' going away---an intimacy resulting
principally from Miss Crawford's desire of something new,
and which had little reality in Fanny's feelings.
Fanny went to her every two or three days:  it seemed a kind
of fascination:  she could not be easy without going,
and yet it was without loving her, without ever thinking
like her, without any sense of obligation for being
sought after now when nobody else was to be had;
and deriving no higher pleasure from her conversation
than occasional amusement, and \emph{that} often at the expense
of her judgment, when it was raised by pleasantry on
people or subjects which she wished to be respected.
She went, however, and they sauntered about together
many an half-hour in Mrs.\ Grant's shrubbery, the weather
being unusually mild for the time of year, and venturing
sometimes even to sit down on one of the benches now
comparatively unsheltered, remaining there perhaps till,
in the midst of some tender ejaculation of Fanny's on
the sweets of so protracted an autumn, they were forced,
by the sudden swell of a cold gust shaking down the last few
yellow leaves about them, to jump up and walk for warmth.

``This is pretty, very pretty,'' said Fanny, looking around
her as they were thus sitting together one day; ``every time
I come into this shrubbery I am more struck with its
growth and beauty.  Three years ago, this was nothing
but a rough hedgerow along the upper side of the field,
never thought of as anything, or capable of becoming anything;
and now it is converted into a walk, and it would be
difficult to say whether most valuable as a convenience
or an ornament; and perhaps, in another three years,
we may be forgetting---almost forgetting what it was before.
How wonderful, how very wonderful the operations of time,
and the changes of the human mind!''  And following
the latter train of thought, she soon afterwards added:
``If any one faculty of our nature may be called \emph{more}
wonderful than the rest, I do think it is memory.
There seems something more speakingly incomprehensible
in the powers, the failures, the inequalities
of memory, than in any other of our intelligences.
The memory is sometimes so retentive, so serviceable,
so obedient; at others, so bewildered and so weak;
and at others again, so tyrannic, so beyond control!
We are, to be sure, a miracle every way; but our powers
of recollecting and of forgetting do seem peculiarly past
finding out.''

Miss Crawford, untouched and inattentive, had nothing
to say; and Fanny, perceiving it, brought back her own
mind to what she thought must interest.

``It may seem impertinent in \emph{me} to praise, but I must
admire the taste Mrs.\ Grant has shewn in all this.
There is such a quiet simplicity in the plan of the walk!
Not too much attempted!''

``Yes,'' replied Miss Crawford carelessly, ``it does
very well for a place of this sort.  One does not think
of extent \emph{here}; and between ourselves, till I came
to Mansfield, I had not imagined a country parson
ever aspired to a shrubbery, or anything of the kind.''

``I am so glad to see the evergreens thrive!'' said Fanny,
in reply.  ``My uncle's gardener always says the soil here
is better than his own, and so it appears from the growth
of the laurels and evergreens in general.  The evergreen!
How beautiful, how welcome, how wonderful the evergreen!
When one thinks of it, how astonishing a variety of nature!
In some countries we know the tree that sheds its leaf
is the variety, but that does not make it less amazing
that the same soil and the same sun should nurture plants
differing in the first rule and law of their existence.
You will think me rhapsodising; but when I am out of doors,
especially when I am sitting out of doors, I am very apt
to get into this sort of wondering strain.  One cannot fix
one's eyes on the commonest natural production without
finding food for a rambling fancy.''

``To say the truth,'' replied Miss Crawford, ``I am something
like the famous Doge at the court of Lewis XIV.;
and may declare that I see no wonder in this shrubbery
equal to seeing myself in it.  If anybody had told
me a year ago that this place would be my home,
that I should be spending month after month here, as I
have done, I certainly should not have believed them.
I have now been here nearly five months; and, moreover,
the quietest five months I ever passed.''

``\emph{Too} quiet for you, I believe.''

``I should have thought so \emph{theoretically} myself, but,''
and her eyes brightened as she spoke, ``take it all
and all, I never spent so happy a summer.  But then,''
with a more thoughtful air and lowered voice, ``there is
no saying what it may lead to.''

Fanny's heart beat quick, and she felt quite unequal
to surmising or soliciting anything more.  Miss Crawford,
however, with renewed animation, soon went on---%

``I am conscious of being far better reconciled to a country
residence than I had ever expected to be.  I can even
suppose it pleasant to spend \emph{half} the year in the country,
under certain circumstances, very pleasant.  An elegant,
moderate-sized house in the centre of family connexions;
continual engagements among them; commanding the first society
in the neighbourhood; looked up to, perhaps, as leading
it even more than those of larger fortune, and turning
from the cheerful round of such amusements to nothing
worse than a \emph{tete-a-tete} with the person one feels
most agreeable in the world.  There is nothing frightful
in such a picture, is there, Miss Price?  One need not
envy the new Mrs.\ Rushworth with such a home as \emph{that}.''
``Envy Mrs.\ Rushworth!'' was all that Fanny attempted to say.
``Come, come, it would be very un-handsome in us to be
severe on Mrs.\ Rushworth, for I look forward to our owing
her a great many gay, brilliant, happy hours.  I expect
we shall be all very much at Sotherton another year.
Such a match as Miss Bertram has made is a public blessing;
for the first pleasures of Mr.\ Rushworth's wife must be to
fill her house, and give the best balls in the country.''

Fanny was silent, and Miss Crawford relapsed into
thoughtfulness, till suddenly looking up at the end
of a few minutes, she exclaimed, ``Ah! here he is.''
It was not Mr.\ Rushworth, however, but Edmund,
who then appeared walking towards them with Mrs.\ Grant.
``My sister and Mr.\ Bertram.  I am so glad your eldest
cousin is gone, that he may be Mr.\ Bertram again.  There is
something in the sound of Mr.\ \emph{Edmund} Bertram so formal,
so pitiful, so younger-brother-like, that I detest it.''

``How differently we feel!'' cried Fanny.  ``To me,
the sound of \emph{Mr.} Bertram is so cold and nothing-meaning,
so entirely without warmth or character!  It just stands
for a gentleman, and that's all.  But there is nobleness
in the name of Edmund.  It is a name of heroism and renown;
of kings, princes, and knights; and seems to breathe
the spirit of chivalry and warm affections.''

``I grant you the name is good in itself, and \emph{Lord} Edmund
or \emph{Sir} Edmund sound delightfully; but sink it under the chill,
the annihilation of a Mr., and Mr.\ Edmund is no more than
Mr.\ John or Mr.\ Thomas.  Well, shall we join and disappoint
them of half their lecture upon sitting down out of doors
at this time of year, by being up before they can begin?''

Edmund met them with particular pleasure.  It was the
first time of his seeing them together since the beginning
of that better acquaintance which he had been hearing
of with great satisfaction.  A friendship between two so
very dear to him was exactly what he could have wished:
and to the credit of the lover's understanding, be it stated,
that he did not by any means consider Fanny as the only,
or even as the greater gainer by such a friendship.

``Well,'' said Miss Crawford, ``and do you not scold us for
our imprudence?  What do you think we have been sitting
down for but to be talked to about it, and entreated
and supplicated never to do so again?''

``Perhaps I might have scolded,'' said Edmund, ``if either
of you had been sitting down alone; but while you
do wrong together, I can overlook a great deal.''

``They cannot have been sitting long,'' cried Mrs.\ Grant,
``for when I went up for my shawl I saw them from the
staircase window, and then they were walking.''

``And really,'' added Edmund, ``the day is so mild,
that your sitting down for a few minutes can be hardly
thought imprudent.  Our weather must not always be judged
by the calendar.  We may sometimes take greater liberties
in November than in May.''

``Upon my word,'' cried Miss Crawford, ``you are two of the most
disappointing and unfeeling kind friends I ever met with!
There is no giving you a moment's uneasiness.  You do not
know how much we have been suffering, nor what chills
we have felt!  But I have long thought Mr.\ Bertram one
of the worst subjects to work on, in any little manoeuvre
against common sense, that a woman could be plagued with.
I had very little hope of \emph{him} from the first; but you,
Mrs.\ Grant, my sister, my own sister, I think I had a right
to alarm you a little.''

``Do not flatter yourself, my dearest Mary.  You have not
the smallest chance of moving me.  I have my alarms,
but they are quite in a different quarter; and if I could
have altered the weather, you would have had a good sharp
east wind blowing on you the whole time---for here are
some of my plants which Robert \emph{will} leave out because
the nights are so mild, and I know the end of it will be,
that we shall have a sudden change of weather, a hard frost
setting in all at once, taking everybody (at least Robert)
by surprise, and I shall lose every one; and what is worse,
cook has just been telling me that the turkey, which I
particularly wished not to be dressed till Sunday,
because I know how much more Dr.\ Grant would enjoy it
on Sunday after the fatigues of the day, will not keep
beyond to-morrow. These are something like grievances,
and make me think the weather most unseasonably close.''

``The sweets of housekeeping in a country village!''
said Miss Crawford archly.  ``Commend me to the nurseryman
and the poulterer.''

``My dear child, commend Dr.\ Grant to the deanery
of Westminster or St. Paul's, and I should be as glad
of your nurseryman and poulterer as you could be.  But we
have no such people in Mansfield.  What would you have me do?''

``Oh! you can do nothing but what you do already:
be plagued very often, and never lose your temper.''

``Thank you; but there is no escaping these little vexations,
Mary, live where we may; and when you are settled in town
and I come to see you, I dare say I shall find you
with yours, in spite of the nurseryman and the poulterer,
perhaps on their very account.  Their remoteness
and unpunctuality, or their exorbitant charges and frauds,
will be drawing forth bitter lamentations.''

``I mean to be too rich to lament or to feel anything
of the sort.  A large income is the best recipe for
happiness I ever heard of.  It certainly may secure
all the myrtle and turkey part of it.''

``You intend to be very rich?'' said Edmund, with a look which,
to Fanny's eye, had a great deal of serious meaning.

``To be sure.  Do not you?  Do not we all?''

``I cannot intend anything which it must be so completely
beyond my power to command.  Miss Crawford may chuse her
degree of wealth.  She has only to fix on her number of
thousands a year, and there can be no doubt of their coming.
My intentions are only not to be poor.''

``By moderation and economy, and bringing down your wants
to your income, and all that.  I understand you---and a
very proper plan it is for a person at your time of life,
with such limited means and indifferent connexions.
What can \emph{you} want but a decent maintenance?  You have
not much time before you; and your relations are in no
situation to do anything for you, or to mortify you
by the contrast of their own wealth and consequence.
Be honest and poor, by all means---but I shall not
envy you; I do not much think I shall even respect you.
I have a much greater respect for those that are honest
and rich.''

``Your degree of respect for honesty, rich or poor,
is precisely what I have no manner of concern with.
I do not mean to be poor.  Poverty is exactly what I have
determined against.  Honesty, in the something between,
in the middle state of worldly circumstances, is all that I
am anxious for your not looking down on.''

``But I do look down upon it, if it might have been higher.
I must look down upon anything contented with obscurity
when it might rise to distinction.''

``But how may it rise?  How may my honesty at least rise
to any distinction?''

This was not so very easy a question to answer,
and occasioned an ``Oh!'' of some length from the fair lady
before she could add, ``You ought to be in parliament,
or you should have gone into the army ten years ago.''

``\emph{That} is not much to the purpose now; and as to my being
in parliament, I believe I must wait till there is an
especial assembly for the representation of younger sons
who have little to live on.  No, Miss Crawford,'' he added,
in a more serious tone, ``there \emph{are} distinctions which I
should be miserable if I thought myself without any chance---%
absolutely without chance or possibility of obtaining---%
but they are of a different character.''

A look of consciousness as he spoke, and what seemed
a consciousness of manner on Miss Crawford's side
as she made some laughing answer, was sorrowfull food
for Fanny's observation; and finding herself quite
unable to attend as she ought to Mrs.\ Grant, by whose
side she was now following the others, she had nearly
resolved on going home immediately, and only waited
for courage to say so, when the sound of the great clock
at Mansfield Park, striking three, made her feel that she
had really been much longer absent than usual, and brought
the previous self-inquiry of whether she should take
leave or not just then, and how, to a very speedy issue.
With undoubting decision she directly began her adieus;
and Edmund began at the same time to recollect that
his mother had been inquiring for her, and that he
had walked down to the Parsonage on purpose to bring her back.

Fanny's hurry increased; and without in the least expecting
Edmund's attendance, she would have hastened away alone;
but the general pace was quickened, and they all accompanied
her into the house, through which it was necessary to pass.
Dr.\ Grant was in the vestibule, and as they stopt to
speak to him she found, from Edmund's manner, that he
\emph{did} mean to go with her.  He too was taking leave.
She could not but be thankful.  In the moment of parting,
Edmund was invited by Dr.\ Grant to eat his mutton
with him the next day; and Fanny had barely time for an
unpleasant feeling on the occasion, when Mrs.\ Grant,
with sudden recollection, turned to her and asked for the
pleasure of her company too.  This was so new an attention,
so perfectly new a circumstance in the events of
Fanny's life, that she was all surprise and embarrassment;
and while stammering out her great obligation, and her
``but she did not suppose it would be in her power,''
was looking at Edmund for his opinion and help.  But Edmund,
delighted with her having such an happiness offered,
and ascertaining with half a look, and half a sentence,
that she had no objection but on her aunt's account,
could not imagine that his mother would make any difficulty
of sparing her, and therefore gave his decided open advice
that the invitation should be accepted; and though Fanny
would not venture, even on his encouragement, to such
a flight of audacious independence, it was soon settled,
that if nothing were heard to the contrary, Mrs.\ Grant
might expect her.

``And you know what your dinner will be,''
said Mrs.\ Grant, smiling---``the turkey, and I assure you
a very fine one; for, my dear,'' turning to her husband,
``cook insists upon the turkey's being dressed to-morrow.''

``Very well, very well,'' cried Dr.\ Grant, ``all the better;
I am glad to hear you have anything so good in the house.
But Miss Price and Mr.\ Edmund Bertram, I dare say, would take
their chance.  We none of us want to hear the bill of fare.
A friendly meeting, and not a fine dinner, is all we
have in view.  A turkey, or a goose, or a leg of mutton,
or whatever you and your cook chuse to give us.''

The two cousins walked home together; and, except in the
immediate discussion of this engagement, which Edmund
spoke of with the warmest satisfaction, as so particularly
desirable for her in the intimacy which he saw with
so much pleasure established, it was a silent walk;
for having finished that subject, he grew thoughtful
and indisposed for any other.



\chapter{Chapter 23}

\gintro{``But why should Mrs.\ Grant ask Fanny?''} said Lady Bertram.
``How came she to think of asking Fanny?  Fanny never
dines there, you know, in this sort of way.  I cannot
spare her, and I am sure she does not want to go.
Fanny, you do not want to go, do you?''

``If you put such a question to her,'' cried Edmund,
preventing his cousin's speaking, ``Fanny will immediately
say No; but I am sure, my dear mother, she would like to go;
and I can see no reason why she should not.''

``I cannot imagine why Mrs.\ Grant should think of asking her?
She never did before.  She used to ask your sisters now
and then, but she never asked Fanny.''

``If you cannot do without me, ma'am---'' said Fanny,
in a self-denying tone.

``But my mother will have my father with her all the evening.''

``To be sure, so I shall.''

``Suppose you take my father's opinion, ma'am.''

``That's well thought of.  So I will, Edmund.  I will
ask Sir Thomas, as soon as he comes in, whether I can
do without her.''

``As you please, ma'am, on that head; but I meant my
father's opinion as to the \emph{propriety} of the invitation's
being accepted or not; and I think he will consider
it a right thing by Mrs.\ Grant, as well as by Fanny,
that being the \emph{first} invitation it should be accepted.''

``I do not know.  We will ask him.  But he will be very
much surprised that Mrs.\ Grant should ask Fanny at all.''

There was nothing more to be said, or that could be
said to any purpose, till Sir Thomas were present;
but the subject involving, as it did, her own evening's
comfort for the morrow, was so much uppermost in Lady
Bertram's mind, that half an hour afterwards, on his
looking in for a minute in his way from his plantation
to his dressing-room, she called him back again,
when he had almost closed the door, with ``Sir Thomas,
stop a moment---I have something to say to you.''

Her tone of calm languor, for she never took the trouble
of raising her voice, was always heard and attended to;
and Sir Thomas came back.  Her story began; and Fanny
immediately slipped out of the room; for to hear herself
the subject of any discussion with her uncle was more
than her nerves could bear.  She was anxious, she knew---%
more anxious perhaps than she ought to be---for what was
it after all whether she went or staid? but if her uncle
were to be a great while considering and deciding,
and with very grave looks, and those grave looks directed
to her, and at last decide against her, she might not
be able to appear properly submissive and indifferent.
Her cause, meanwhile, went on well.  It began, on Lady
Bertram's part, with---``I have something to tell you
that will surprise you.  Mrs.\ Grant has asked Fanny
to dinner.''

``Well,'' said Sir Thomas, as if waiting more to accomplish
the surprise.

``Edmund wants her to go.  But how can I spare her?''

``She will be late,'' said Sir Thomas, taking out his watch;
``but what is your difficulty?''

Edmund found himself obliged to speak and fill up
the blanks in his mother's story.  He told the whole;
and she had only to add, ``So strange! for Mrs.\ Grant
never used to ask her.''

``But is it not very natural,'' observed Edmund,
``that Mrs.\ Grant should wish to procure so agreeable
a visitor for her sister?''

``Nothing can be more natural,'' said Sir Thomas, after a
short deliberation; ``nor, were there no sister in the case,
could anything, in my opinion, be more natural.
Mrs.\ Grant's shewing civility to Miss Price, to Lady
Bertram's niece, could never want explanation.  The only
surprise I can feel is, that this should be the \emph{first}
time of its being paid.  Fanny was perfectly right in
giving only a conditional answer.  She appears to feel
as she ought.  But as I conclude that she must wish to go,
since all young people like to be together, I can see
no reason why she should be denied the indulgence.''

``But can I do without her, Sir Thomas?''

``Indeed I think you may.''

``She always makes tea, you know, when my sister is not here.''

``Your sister, perhaps, may be prevailed on to spend
the day with us, and I shall certainly be at home.''

``Very well, then, Fanny may go, Edmund.''

The good news soon followed her.  Edmund knocked at her
door in his way to his own.

``Well, Fanny, it is all happily settled, and without
the smallest hesitation on your uncle's side.
He had but one opinion.  You are to go.''

``Thank you, I am \emph{so} glad,'' was Fanny's instinctive reply;
though when she had turned from him and shut the door,
she could not help feeling, ``And yet why should I be glad?
for am I not certain of seeing or hearing something there
to pain me?''

In spite of this conviction, however, she was glad.
Simple as such an engagement might appear in other eyes,
it had novelty and importance in hers, for excepting the
day at Sotherton, she had scarcely ever dined out before;
and though now going only half a mile, and only to
three people, still it was dining out, and all the little
interests of preparation were enjoyments in themselves.
She had neither sympathy nor assistance from those who ought
to have entered into her feelings and directed her taste;
for Lady Bertram never thought of being useful to anybody,
and Mrs.\ Norris, when she came on the morrow, in consequence
of an early call and invitation from Sir Thomas, was in
a very ill humour, and seemed intent only on lessening
her niece's pleasure, both present and future, as much
as possible.

``Upon my word, Fanny, you are in high luck to meet
with such attention and indulgence!  You ought to be
very much obliged to Mrs.\ Grant for thinking of you,
and to your aunt for letting you go, and you ought to look
upon it as something extraordinary; for I hope you are
aware that there is no real occasion for your going into
company in this sort of way, or ever dining out at all;
and it is what you must not depend upon ever being repeated.
Nor must you be fancying that the invitation is meant
as any particular compliment to \emph{you}; the compliment
is intended to your uncle and aunt and me.  Mrs.\ Grant
thinks it a civility due to \emph{us} to take a little notice
of you, or else it would never have come into her head,
and you may be very certain that, if your cousin Julia
had been at home, you would not have been asked at all.''

Mrs.\ Norris had now so ingeniously done away all
Mrs.\ Grant's part of the favour, that Fanny, who found
herself expected to speak, could only say that she was
very much obliged to her aunt Bertram for sparing her,
and that she was endeavouring to put her aunt's evening
work in such a state as to prevent her being missed.

``Oh! depend upon it, your aunt can do very well without you,
or you would not be allowed to go.  \emph{I} shall be here, so you
may be quite easy about your aunt.  And I hope you will have
a very \emph{agreeable} day, and find it all mighty \emph{delightful}.
But I must observe that five is the very awkwardest of
all possible numbers to sit down to table; and I cannot
but be surprised that such an \emph{elegant} lady as Mrs.\ Grant
should not contrive better!  And round their enormous great
wide table, too, which fills up the room so dreadfully!
Had the doctor been contented to take my dining-table when I
came away, as anybody in their senses would have done,
instead of having that absurd new one of his own,
which is wider, literally wider than the dinner-table here,
how infinitely better it would have been! and how much
more he would have been respected! for people are never
respected when they step out of their proper sphere.
Remember that, Fanny.  Five---only five to be sitting
round that table.  However, you will have dinner enough
on it for ten, I dare say.''

Mrs.\ Norris fetched breath, and went on again.

``The nonsense and folly of people's stepping out of their
rank and trying to appear above themselves, makes me
think it right to give \emph{you} a hint, Fanny, now that you
are going into company without any of us; and I do beseech
and entreat you not to be putting yourself forward,
and talking and giving your opinion as if you were one of
your cousins---as if you were dear Mrs.\ Rushworth or Julia.
\emph{That} will never do, believe me.  Remember, wherever you are,
you must be the lowest and last; and though Miss Crawford
is in a manner at home at the Parsonage, you are not to
be taking place of her.  And as to coming away at night,
you are to stay just as long as Edmund chuses.
Leave him to settle \emph{that}.''

``Yes, ma'am, I should not think of anything else.''

``And if it should rain, which I think exceedingly likely,
for I never saw it more threatening for a wet evening
in my life, you must manage as well as you can, and not be
expecting the carriage to be sent for you.  I certainly
do not go home to-night, and, therefore, the carriage will
not be out on my account; so you must make up your mind
to what may happen, and take your things accordingly.''

Her niece thought it perfectly reasonable.  She rated her
own claims to comfort as low even as Mrs.\ Norris could;
and when Sir Thomas soon afterwards, just opening
the door, said, ``Fanny, at what time would you have the
carriage come round?'' she felt a degree of astonishment
which made it impossible for her to speak.

``My dear Sir Thomas!'' cried Mrs.\ Norris, red with anger,
``Fanny can walk.''

``Walk!'' repeated Sir Thomas, in a tone of most unanswerable
dignity, and coming farther into the room.  ``My niece
walk to a dinner engagement at this time of the year!
Will twenty minutes after four suit you?''

``Yes, sir,'' was Fanny's humble answer, given with the
feelings almost of a criminal towards Mrs.\ Norris;
and not bearing to remain with her in what might seem
a state of triumph, she followed her uncle out of the room,
having staid behind him only long enough to hear these
words spoken in angry agitation---%

``Quite unnecessary! a great deal too kind!
But Edmund goes; true, it is upon Edmund's account.
I observed he was hoarse on Thursday night.''

But this could not impose on Fanny.  She felt that
the carriage was for herself, and herself alone:
and her uncle's consideration of her, coming immediately
after such representations from her aunt, cost her
some tears of gratitude when she was alone.

The coachman drove round to a minute; another minute
brought down the gentleman; and as the lady had, with a
most scrupulous fear of being late, been many minutes
seated in the drawing-room, Sir Thomas saw them off
in as good time as his own correctly punctual habits required.

``Now I must look at you, Fanny,'' said Edmund, with the
kind smile of an affectionate brother, ``and tell you
how I like you; and as well as I can judge by this light,
you look very nicely indeed.  What have you got on?''

``The new dress that my uncle was so good as to give me
on my cousin's marriage.  I hope it is not too fine; but I
thought I ought to wear it as soon as I could, and that I
might not have such another opportunity all the winter.
I hope you do not think me too fine.''

``A woman can never be too fine while she is all in white.  No, I
see
no finery about you; nothing but what is perfectly proper.
Your gown seems very pretty.  I like these glossy spots.
Has not Miss Crawford a gown something the same?''

In approaching the Parsonage they passed close by the
stable-yard and coach-house.

``Heyday!'' said Edmund, ``here's company, here's a carriage!
who have they got to meet us?''  And letting down the side-glass
to distinguish, ``\,'Tis Crawford's, Crawford's barouche,
I protest!  There are his own two men pushing it back
into its old quarters.  He is here, of course.  This is
quite a surprise, Fanny.  I shall be very glad to see him.''

There was no occasion, there was no time for Fanny
to say how very differently she felt; but the idea
of having such another to observe her was a great
increase of the trepidation with which she performed
the very awful ceremony of walking into the drawing-room.

In the drawing-room Mr.\ Crawford certainly was, having been
just long enough arrived to be ready for dinner; and the
smiles and pleased looks of the three others standing
round him, shewed how welcome was his sudden resolution
of coming to them for a few days on leaving Bath.
A very cordial meeting passed between him and Edmund;
and with the exception of Fanny, the pleasure was general;
and even to \emph{her} there might be some advantage in
his presence, since every addition to the party must
rather forward her favourite indulgence of being suffered
to sit silent and unattended to.  She was soon aware
of this herself; for though she must submit, as her
own propriety of mind directed, in spite of her aunt
Norris's opinion, to being the principal lady in company,
and to all the little distinctions consequent thereon,
she found, while they were at table, such a happy flow
of conversation prevailing, in which she was not required
to take any part---there was so much to be said between
the brother and sister about Bath, so much between
the two young men about hunting, so much of politics
between Mr.\ Crawford and Dr.\ Grant, and of everything
and all together between Mr.\ Crawford and Mrs.\ Grant,
as to leave her the fairest prospect of having only to
listen in quiet, and of passing a very agreeable day.
She could not compliment the newly arrived gentleman,
however, with any appearance of interest, in a scheme
for extending his stay at Mansfield, and sending for his
hunters from Norfolk, which, suggested by Dr.\ Grant,
advised by Edmund, and warmly urged by the two sisters,
was soon in possession of his mind, and which he seemed
to want to be encouraged even by her to resolve on.
Her opinion was sought as to the probable continuance
of the open weather, but her answers were as short
and indifferent as civility allowed.  She could not wish
him to stay, and would much rather not have him speak
to her.

Her two absent cousins, especially Maria, were much in her
thoughts on seeing him; but no embarrassing remembrance
affected \emph{his} spirits.  Here he was again on the same
ground where all had passed before, and apparently as
willing to stay and be happy without the Miss Bertrams,
as if he had never known Mansfield in any other state.
She heard them spoken of by him only in a general way,
till they were all re-assembled in the drawing-room,
when Edmund, being engaged apart in some matter of business
with Dr.\ Grant, which seemed entirely to engross them,
and Mrs.\ Grant occupied at the tea-table, he began talking
of them with more particularity to his other sister.
With a significant smile, which made Fanny quite hate him,
he said, ``So!  Rushworth and his fair bride are at Brighton,
I understand; happy man!''

``Yes, they have been there about a fortnight, Miss Price,
have they not?  And Julia is with them.''

``And Mr.\ Yates, I presume, is not far off.''

``Mr.\ Yates!  Oh! we hear nothing of Mr.\ Yates.  I do not
imagine he figures much in the letters to Mansfield Park;
do you, Miss Price?  I think my friend Julia knows better
than to entertain her father with Mr.\ Yates.''

``Poor Rushworth and his two-and-forty speeches!''
continued Crawford.  ``Nobody can ever forget them.
Poor fellow!  I see him now---his toil and his despair.
Well, I am much mistaken if his lovely Maria will ever
want him to make two-and-forty speeches to her''; adding,
with a momentary seriousness, ``She is too good for him---%
much too good.''  And then changing his tone again to one
of gentle gallantry, and addressing Fanny, he said,
``You were Mr.\ Rushworth's best friend.  Your kindness and
patience can never be forgotten, your indefatigable patience
in trying to make it possible for him to learn his part---%
in trying to give him a brain which nature had denied---%
to mix up an understanding for him out of the superfluity
of your own!  \emph{He} might not have sense enough himself
to estimate your kindness, but I may venture to say that it
had honour from all the rest of the party.''

Fanny coloured, and said nothing.

``It is as a dream, a pleasant dream!'' he exclaimed,
breaking forth again, after a few minutes' musing.  ``I shall
always look back on our theatricals with exquisite pleasure.
There was such an interest, such an animation, such a
spirit diffused.  Everybody felt it.  We were all alive.
There was employment, hope, solicitude, bustle, for every
hour of the day.  Always some little objection,
some little doubt, some little anxiety to be got over.
I never was happier.''

With silent indignation Fanny repeated to herself,
``Never happier!---never happier than when doing what
you must know was not justifiable!---never happier
than when behaving so dishonourably and unfeelingly!
Oh! what a corrupted mind!''

``We were unlucky, Miss Price,'' he continued, in a lower tone,
to avoid the possibility of being heard by Edmund,
and not at all aware of her feelings, ``we certainly
were very unlucky.  Another week, only one other week,
would have been enough for us.  I think if we had had the
disposal of events---if Mansfield Park had had the government
of the winds just for a week or two, about the equinox,
there would have been a difference.  Not that we would
have endangered his safety by any tremendous weather---%
but only by a steady contrary wind, or a calm.  I think,
Miss Price, we would have indulged ourselves with a week's
calm in the Atlantic at that season.''

He seemed determined to be answered; and Fanny,
averting her face, said, with a firmer tone than usual,
``As far as \emph{I} am concerned, sir, I would not have
delayed his return for a day.  My uncle disapproved it
all so entirely when he did arrive, that in my opinion
everything had gone quite far enough.''

She had never spoken so much at once to him in her life before,
and never so angrily to any one; and when her speech was over,
she trembled and blushed at her own daring.  He was surprised;
but after a few moments' silent consideration of her,
replied in a calmer, graver tone, and as if the candid
result of conviction, ``I believe you are right.  It was
more pleasant than prudent.  We were getting too noisy.''
And then turning the conversation, he would have engaged
her on some other subject, but her answers were so shy
and reluctant that he could not advance in any.

Miss Crawford, who had been repeatedly eyeing Dr.\ Grant
and Edmund, now observed, ``Those gentlemen must have
some very interesting point to discuss.''

``The most interesting in the world,'' replied her brother---%
``how to make money; how to turn a good income into a better.
Dr.\ Grant is giving Bertram instructions about the living
he is to step into so soon.  I find he takes orders
in a few weeks.  They were at it in the dining-parlour.
I am glad to hear Bertram will be so well off.  He will
have a very pretty income to make ducks and drakes with,
and earned without much trouble.  I apprehend he will
not have less than seven hundred a year.  Seven hundred
a year is a fine thing for a younger brother; and as of
course he will still live at home, it will be all for his
\emph{menus} \emph{plaisirs}; and a sermon at Christmas and Easter,
I suppose, will be the sum total of sacrifice.''

His sister tried to laugh off her feelings by saying,
``Nothing amuses me more than the easy manner with which everybody
settles the abundance of those who have a great deal less
than themselves.  You would look rather blank, Henry, if your
\emph{menus} \emph{plaisirs} were to be limited to seven hundred a year.''

``Perhaps I might; but all \emph{that} you know is
entirely comparative.  Birthright and habit must settle
the business.  Bertram is certainly well off for a cadet
of even a baronet's family.  By the time he is four or five
and twenty he will have seven hundred a year, and nothing to do for
it.''

Miss Crawford \emph{could} have said that there would
be a something to do and to suffer for it, which she
could not think lightly of; but she checked herself
and let it pass; and tried to look calm and unconcerned
when the two gentlemen shortly afterwards joined them.

``Bertram,'' said Henry Crawford, ``I shall make a point of
coming to Mansfield to hear you preach your first sermon.
I shall come on purpose to encourage a young beginner.
When is it to be?  Miss Price, will not you join me in
encouraging your cousin?  Will not you engage to attend
with your eyes steadily fixed on him the whole time---%
as I shall do---not to lose a word; or only looking off
just to note down any sentence preeminently beautiful?
We will provide ourselves with tablets and a pencil.
When will it be?  You must preach at Mansfield, you know,
that Sir Thomas and Lady Bertram may hear you.''

``I shall keep clear of you, Crawford, as long as I can,''
said Edmund; ``for you would be more likely to disconcert me,
and I should be more sorry to see you trying at it than
almost any other man.''

``Will he not feel this?'' thought Fanny.  ``No, he can feel
nothing as he ought.''

The party being now all united, and the chief talkers
attracting each other, she remained in tranquillity;
and as a whist-table was formed after tea---formed really
for the amusement of Dr.\ Grant, by his attentive wife,
though it was not to be supposed so---and Miss Crawford
took her harp, she had nothing to do but to listen;
and her tranquillity remained undisturbed the rest
of the evening, except when Mr.\ Crawford now and then
addressed to her a question or observation, which she
could not avoid answering.  Miss Crawford was too much
vexed by what had passed to be in a humour for anything
but music.  With that she soothed herself and amused
her friend.

The assurance of Edmund's being so soon to take orders,
coming upon her like a blow that had been suspended,
and still hoped uncertain and at a distance, was felt
with resentment and mortification.  She was very angry
with him.  She had thought her influence more.
She \emph{had} begun to think of him; she felt that she had,
with great regard, with almost decided intentions;
but she would now meet him with his own cool feelings.
It was plain that he could have no serious views, no true
attachment, by fixing himself in a situation which he must
know she would never stoop to.  She would learn to match
him in his indifference.  She would henceforth admit his
attentions without any idea beyond immediate amusement.
If \emph{he} could so command his affections, \emph{hers} should do her
no harm.



\chapter{Chapter 24}

\gintro{Henry Crawford} had quite made up his mind by the
next morning to give another fortnight to Mansfield,
and having sent for his hunters, and written a few
lines of explanation to the Admiral, he looked round at
his sister as he sealed and threw the letter from him,
and seeing the coast clear of the rest of the family,
said, with a smile, ``And how do you think I mean to
amuse myself, Mary, on the days that I do not hunt?
I am grown too old to go out more than three times a week;
but I have a plan for the intermediate days, and what do you think
it is?''

``To walk and ride with me, to be sure.''

``Not exactly, though I shall be happy to do both, but \emph{that}
would be exercise only to my body, and I must take care
of my mind.  Besides, \emph{that} would be all recreation
and indulgence, without the wholesome alloy of labour,
and I do not like to eat the bread of idleness.  No, my plan
is to make Fanny Price in love with me.''

``Fanny Price!  Nonsense!  No, no.  You ought to be
satisfied with her two cousins.''

``But I cannot be satisfied without Fanny Price,
without making a small hole in Fanny Price's heart.
You do not seem properly aware of her claims to notice.
When we talked of her last night, you none of you
seemed sensible of the wonderful improvement that has
taken place in her looks within the last six weeks.
You see her every day, and therefore do not notice it;
but I assure you she is quite a different creature
from what she was in the autumn.  She was then merely
a quiet, modest, not plain-looking girl, but she is now
absolutely pretty.  I used to think she had neither
complexion nor countenance; but in that soft skin of hers,
so frequently tinged with a blush as it was yesterday,
there is decided beauty; and from what I observed of her
eyes and mouth, I do not despair of their being capable
of expression enough when she has anything to express.
And then, her air, her manner, her \emph{tout} \emph{ensemble},
is so indescribably improved!  She must be grown two inches,
at least, since October.''

``Phoo! phoo!  This is only because there were no tall women
to compare her with, and because she has got a new gown,
and you never saw her so well dressed before.  She is
just what she was in October, believe me.  The truth is,
that she was the only girl in company for you to notice,
and you must have a somebody.  I have always thought
her pretty---not strikingly pretty---but `pretty enough,'
as people say; a sort of beauty that grows on one.
Her eyes should be darker, but she has a sweet smile;
but as for this wonderful degree of improvement, I am
sure it may all be resolved into a better style of dress,
and your having nobody else to look at; and therefore,
if you do set about a flirtation with her, you never
will persuade me that it is in compliment to her beauty,
or that it proceeds from anything but your own idleness
and folly.''

Her brother gave only a smile to this accusation,
and soon afterwards said, ``I do not quite know what to make
of Miss Fanny.  I do not understand her.  I could not tell
what she would be at yesterday.  What is her character?
Is she solemn?  Is she queer?  Is she prudish?  Why did
she draw back and look so grave at me?  I could hardly get
her to speak.  I never was so long in company with a girl
in my life, trying to entertain her, and succeed so ill!
Never met with a girl who looked so grave on me!
I must try to get the better of this.  Her looks say,
`I will not like you, I am determined not to like you';
and I say she shall.''

``Foolish fellow!  And so this is her attraction after all!
This it is, her not caring about you, which gives
her such a soft skin, and makes her so much taller,
and produces all these charms and graces!  I do desire
that you will not be making her really unhappy;
a \emph{little} love, perhaps, may animate and do her good,
but I will not have you plunge her deep, for she is as
good a little creature as ever lived, and has a great
deal of feeling.''

``It can be but for a fortnight,'' said Henry; ``and if a
fortnight can kill her, she must have a constitution
which nothing could save.  No, I will not do her any harm,
dear little soul! only want her to look kindly on me,
to give me smiles as well as blushes, to keep a chair
for me by herself wherever we are, and be all animation
when I take it and talk to her; to think as I think,
be interested in all my possessions and pleasures,
try to keep me longer at Mansfield, and feel when I
go away that she shall be never happy again.  I want
nothing more.''

``Moderation itself!'' said Mary.  ``I can have no scruples now.
Well, you will have opportunities enough of endeavouring
to recommend yourself, for we are a great deal together.''

And without attempting any farther remonstrance, she left
Fanny to her fate, a fate which, had not Fanny's heart
been guarded in a way unsuspected by Miss Crawford,
might have been a little harder than she deserved;
for although there doubtless are such unconquerable young
ladies of eighteen (or one should not read about them)
as are never to be persuaded into love against their judgment
by all that talent, manner, attention, and flattery can do,
I have no inclination to believe Fanny one of them,
or to think that with so much tenderness of disposition,
and so much taste as belonged to her, she could have
escaped heart-whole from the courtship (though the
courtship only of a fortnight) of such a man as Crawford,
in spite of there being some previous ill opinion of him
to be overcome, had not her affection been engaged elsewhere.
With all the security which love of another and disesteem
of him could give to the peace of mind he was attacking,
his continued attentions---continued, but not obtrusive,
and adapting themselves more and more to the gentleness
and delicacy of her character---obliged her very soon
to dislike him less than formerly.  She had by no means
forgotten the past, and she thought as ill of him as ever;
but she felt his powers:  he was entertaining; and his
manners were so improved, so polite, so seriously and
blamelessly polite, that it was impossible not to be civil
to him in return.

A very few days were enough to effect this; and at the end
of those few days, circumstances arose which had a tendency
rather to forward his views of pleasing her, inasmuch as
they gave her a degree of happiness which must dispose
her to be pleased with everybody.  William, her brother,
the so long absent and dearly loved brother, was in
England again.  She had a letter from him herself, a few
hurried happy lines, written as the ship came up Channel,
and sent into Portsmouth with the first boat that left
the Antwerp at anchor in Spithead; and when Crawford walked
up with the newspaper in his hand, which he had hoped
would bring the first tidings, he found her trembling
with joy over this letter, and listening with a glowing,
grateful countenance to the kind invitation which her
uncle was most collectedly dictating in reply.

It was but the day before that Crawford had made himself
thoroughly master of the subject, or had in fact become
at all aware of her having such a brother, or his being
in such a ship, but the interest then excited had been
very properly lively, determining him on his return to
town to apply for information as to the probable period
of the Antwerp's return from the Mediterranean, etc.;
and the good luck which attended his early examination
of ship news the next morning seemed the reward of his
ingenuity in finding out such a method of pleasing her,
as well as of his dutiful attention to the Admiral,
in having for many years taken in the paper esteemed
to have the earliest naval intelligence.  He proved,
however, to be too late.  All those fine first feelings,
of which he had hoped to be the exciter, were already given.
But his intention, the kindness of his intention,
was thankfully acknowledged:  quite thankfully and warmly,
for she was elevated beyond the common timidity of her
mind by the flow of her love for William.

This dear William would soon be amongst them.  There could
be no doubt of his obtaining leave of absence immediately,
for he was still only a midshipman; and as his parents,
from living on the spot, must already have seen him,
and be seeing him perhaps daily, his direct holidays
might with justice be instantly given to the sister,
who had been his best correspondent through a period of
seven years, and the uncle who had done most for his support
and advancement; and accordingly the reply to her reply,
fixing a very early day for his arrival, came as soon
as possible; and scarcely ten days had passed since Fanny
had been in the agitation of her first dinner-visit,
when she found herself in an agitation of a higher nature,
watching in the hall, in the lobby, on the stairs,
for the first sound of the carriage which was to bring her
a brother.

It came happily while she was thus waiting; and there
being neither ceremony nor fearfulness to delay the moment
of meeting, she was with him as he entered the house,
and the first minutes of exquisite feeling had no interruption
and no witnesses, unless the servants chiefly intent
upon opening the proper doors could be called such.
This was exactly what Sir Thomas and Edmund had been
separately conniving at, as each proved to the other
by the sympathetic alacrity with which they both advised
Mrs.\ Norris's continuing where she was, instead of rushing
out into the hall as soon as the noises of the arrival
reached them.

William and Fanny soon shewed themselves; and Sir Thomas
had the pleasure of receiving, in his protege, certainly a
very different person from the one he had equipped seven
years ago, but a young man of an open, pleasant countenance,
and frank, unstudied, but feeling and respectful manners,
and such as confirmed him his friend.

It was long before Fanny could recover from the agitating
happiness of such an hour as was formed by the last
thirty minutes of expectation, and the first of fruition;
it was some time even before her happiness could be said
to make her happy, before the disappointment inseparable
from the alteration of person had vanished, and she could
see in him the same William as before, and talk to him,
as her heart had been yearning to do through many
a past year.  That time, however, did gradually come,
forwarded by an affection on his side as warm as her own,
and much less encumbered by refinement or self-distrust.
She was the first object of his love, but it was a love
which his stronger spirits, and bolder temper, made it
as natural for him to express as to feel.  On the morrow
they were walking about together with true enjoyment,
and every succeeding morrow renewed a \emph{tete-a-tete}
which Sir Thomas could not but observe with complacency,
even before Edmund had pointed it out to him.

Excepting the moments of peculiar delight, which any marked
or unlooked-for instance of Edmund's consideration of her
in the last few months had excited, Fanny had never known
so much felicity in her life, as in this unchecked, equal,
fearless intercourse with the brother and friend who was opening
all his heart to her, telling her all his hopes and fears,
plans, and solicitudes respecting that long thought of,
dearly earned, and justly valued blessing of promotion;
who could give her direct and minute information of the
father and mother, brothers and sisters, of whom she
very seldom heard; who was interested in all the comforts
and all the little hardships of her home at Mansfield;
ready to think of every member of that home as she directed,
or differing only by a less scrupulous opinion, and more
noisy abuse of their aunt Norris, and with whom (perhaps
the dearest indulgence of the whole) all the evil and
good of their earliest years could be gone over again,
and every former united pain and pleasure retraced
with the fondest recollection.  An advantage this,
a strengthener of love, in which even the conjugal tie
is beneath the fraternal.  Children of the same family,
the same blood, with the same first associations and habits,
have some means of enjoyment in their power, which no
subsequent connexions can supply; and it must be by a
long and unnatural estrangement, by a divorce which no
subsequent connexion can justify, if such precious remains
of the earliest attachments are ever entirely outlived.
Too often, alas! it is so.  Fraternal love, sometimes
almost everything, is at others worse than nothing.
But with William and Fanny Price it was still a sentiment
in all its prime and freshness, wounded by no opposition
of interest, cooled by no separate attachment, and feeling
the influence of time and absence only in its increase.

An affection so amiable was advancing each in the opinion
of all who had hearts to value anything good.  Henry Crawford
was as much struck with it as any.  He honoured the
warm-hearted, blunt fondness of the young sailor, which led
him to say, with his hands stretched towards Fanny's head,
``Do you know, I begin to like that queer fashion already,
though when I first heard of such things being done
in England, I could not believe it; and when Mrs.\ Brown,
and the other women at the Commissioner's at Gibraltar,
appeared in the same trim, I thought they were mad; but Fanny
can reconcile me to anything''; and saw, with lively admiration,
the glow of Fanny's cheek, the brightness of her eye,
the deep interest, the absorbed attention, while her
brother was describing any of the imminent hazards,
or terrific scenes, which such a period at sea must supply.

It was a picture which Henry Crawford had moral taste enough
to value.  Fanny's attractions increased---increased twofold;
for the sensibility which beautified her complexion and
illumined her countenance was an attraction in itself.
He was no longer in doubt of the capabilities of her heart.
She had feeling, genuine feeling.  It would be something
to be loved by such a girl, to excite the first ardours
of her young unsophisticated mind!  She interested him
more than he had foreseen.  A fortnight was not enough.
His stay became indefinite.

William was often called on by his uncle to be the talker.
His recitals were amusing in themselves to Sir Thomas,
but the chief object in seeking them was to understand
the reciter, to know the young man by his histories;
and he listened to his clear, simple, spirited details with
full satisfaction, seeing in them the proof of good principles,
professional knowledge, energy, courage, and cheerfulness,
everything that could deserve or promise well.
Young as he was, William had already seen a great deal.
He had been in the Mediterranean; in the West Indies;
in the Mediterranean again; had been often taken on shore
by the favour of his captain, and in the course of seven
years had known every variety of danger which sea and war
together could offer.  With such means in his power he
had a right to be listened to; and though Mrs.\ Norris could
fidget about the room, and disturb everybody in quest
of two needlefuls of thread or a second-hand shirt button,
in the midst of her nephew's account of a shipwreck
or an engagement, everybody else was attentive; and even
Lady Bertram could not hear of such horrors unmoved,
or without sometimes lifting her eyes from her work to say,
``Dear me! how disagreeable!  I wonder anybody can ever go
to sea.''

To Henry Crawford they gave a different feeling.  He longed
to have been at sea, and seen and done and suffered as much.
His heart was warmed, his fancy fired, and he felt
the highest respect for a lad who, before he was twenty,
had gone through such bodily hardships and given such
proofs of mind.  The glory of heroism, of usefulness,
of exertion, of endurance, made his own habits of selfish
indulgence appear in shameful contrast; and he wished
he had been a William Price, distinguishing himself and
working his way to fortune and consequence with so much
self-respect and happy ardour, instead of what he was!

The wish was rather eager than lasting.  He was roused from
the reverie of retrospection and regret produced by it,
by some inquiry from Edmund as to his plans for the next
day's hunting; and he found it was as well to be a man
of fortune at once with horses and grooms at his command.
In one respect it was better, as it gave him the means
of conferring a kindness where he wished to oblige.
With spirits, courage, and curiosity up to anything,
William expressed an inclination to hunt; and Crawford could
mount him without the slightest inconvenience to himself,
and with only some scruples to obviate in Sir Thomas,
who knew better than his nephew the value of such a loan,
and some alarms to reason away in Fanny.  She feared
for William; by no means convinced by all that he could
relate of his own horsemanship in various countries,
of the scrambling parties in which he had been engaged,
the rough horses and mules he had ridden, or his many narrow
escapes from dreadful falls, that he was at all equal to the
management of a high-fed hunter in an English fox-chase;
nor till he returned safe and well, without accident
or discredit, could she be reconciled to the risk,
or feel any of that obligation to Mr.\ Crawford for lending
the horse which he had fully intended it should produce.
When it was proved, however, to have done William no harm,
she could allow it to be a kindness, and even reward
the owner with a smile when the animal was one minute
tendered to his use again; and the next, with the
greatest cordiality, and in a manner not to be resisted,
made over to his use entirely so long as he remained
in Northamptonshire.

%                        [End volume one of this edition.
%                        Printed by T. and A. Constable,
%                        Printers to Her Majesty at
%                        the Edinburgh University Press]



\chapter{Chapter 25}

\gintro{The intercourse} of the two families was at this period
more nearly restored to what it had been in the autumn,
than any member of the old intimacy had thought ever
likely to be again.  The return of Henry Crawford,
and the arrival of William Price, had much to do with it,
but much was still owing to Sir Thomas's more than toleration
of the neighbourly attempts at the Parsonage.  His mind,
now disengaged from the cares which had pressed on him
at first, was at leisure to find the Grants and their
young inmates really worth visiting; and though infinitely
above scheming or contriving for any the most advantageous
matrimonial establishment that could be among the apparent
possibilities of any one most dear to him, and disdaining
even as a littleness the being quick-sighted on such points,
he could not avoid perceiving, in a grand and careless way,
that Mr.\ Crawford was somewhat distinguishing his niece---%
nor perhaps refrain (though unconsciously) from giving a
more willing assent to invitations on that account.

His readiness, however, in agreeing to dine at the Parsonage,
when the general invitation was at last hazarded,
after many debates and many doubts as to whether it were
worth while, ``because Sir Thomas seemed so ill inclined,
and Lady Bertram was so indolent!'' proceeded from
good-breeding and goodwill alone, and had nothing to do
with Mr.\ Crawford, but as being one in an agreeable group:
for it was in the course of that very visit that he first
began to think that any one in the habit of such idle
observations \emph{would} \emph{have} \emph{thought} that Mr.\ Crawford
was the admirer of Fanny Price.

The meeting was generally felt to be a pleasant one,
being composed in a good proportion of those who would talk
and those who would listen; and the dinner itself was elegant
and plentiful, according to the usual style of the Grants,
and too much according to the usual habits of all to raise
any emotion except in Mrs.\ Norris, who could never behold
either the wide table or the number of dishes on it
with patience, and who did always contrive to experience
some evil from the passing of the servants behind her chair,
and to bring away some fresh conviction of its being
impossible among so many dishes but that some must be cold.

In the evening it was found, according to the predetermination
of Mrs.\ Grant and her sister, that after making up
the whist-table there would remain sufficient for a
round game, and everybody being as perfectly complying
and without a choice as on such occasions they always are,
speculation was decided on almost as soon as whist;
and Lady Bertram soon found herself in the critical situation
of being applied to for her own choice between the games,
and being required either to draw a card for whist or not.
She hesitated.  Luckily Sir Thomas was at hand.

``What shall I do, Sir Thomas?  Whist and speculation;
which will amuse me most?''

Sir Thomas, after a moment's thought, recommended speculation.
He was a whist player himself, and perhaps might feel
that it would not much amuse him to have her for a partner.

``Very well,'' was her ladyship's contented answer;
``then speculation, if you please, Mrs.\ Grant.  I know
nothing about it, but Fanny must teach me.''

Here Fanny interposed, however, with anxious protestations
of her own equal ignorance; she had never played the
game nor seen it played in her life; and Lady Bertram
felt a moment's indecision again; but upon everybody's
assuring her that nothing could be so easy, that it
was the easiest game on the cards, and Henry Crawford's
stepping forward with a most earnest request to be allowed
to sit between her ladyship and Miss Price, and teach
them both, it was so settled; and Sir Thomas, Mrs.\ Norris,
and Dr.\ and Mrs.\ Grant being seated at the table of prime
intellectual state and dignity, the remaining six,
under Miss Crawford's direction, were arranged round
the other.  It was a fine arrangement for Henry Crawford,
who was close to Fanny, and with his hands full of business,
having two persons' cards to manage as well as his own;
for though it was impossible for Fanny not to feel herself
mistress of the rules of the game in three minutes,
he had yet to inspirit her play, sharpen her avarice,
and harden her heart, which, especially in any competition
with William, was a work of some difficulty; and as for
Lady Bertram, he must continue in charge of all her fame
and fortune through the whole evening; and if quick enough
to keep her from looking at her cards when the deal began,
must direct her in whatever was to be done with them
to the end of it.

He was in high spirits, doing everything with happy ease,
and preeminent in all the lively turns, quick resources,
and playful impudence that could do honour to the game;
and the round table was altogether a very comfortable
contrast to the steady sobriety and orderly silence of
the other.

Twice had Sir Thomas inquired into the enjoyment and
success of his lady, but in vain; no pause was long enough
for the time his measured manner needed; and very little
of her state could be known till Mrs.\ Grant was able,
at the end of the first rubber, to go to her and pay
her compliments.

``I hope your ladyship is pleased with the game.''

``Oh dear, yes! very entertaining indeed.  A very odd game.
I do not know what it is all about.  I am never to see
my cards; and Mr.\ Crawford does all the rest.''

``Bertram,'' said Crawford, some time afterwards, taking the
opportunity of a little languor in the game, ``I have never
told you what happened to me yesterday in my ride home.''
They had been hunting together, and were in the midst of a
good run, and at some distance from Mansfield, when his horse
being found to have flung a shoe, Henry Crawford had been
obliged to give up, and make the best of his way back.
``I told you I lost my way after passing that old farmhouse
with the yew-trees, because I can never bear to ask;
but I have not told you that, with my usual luck---for I
never do wrong without gaining by it---I found myself in due
time in the very place which I had a curiosity to see.
I was suddenly, upon turning the corner of a steepish
downy field, in the midst of a retired little village
between gently rising hills; a small stream before me to
be forded, a church standing on a sort of knoll to my right---%
which church was strikingly large and handsome for
the place, and not a gentleman or half a gentleman's house
to be seen excepting one---to be presumed the Parsonage---%
within a stone's throw of the said knoll and church.
I found myself, in short, in Thornton Lacey.''

``It sounds like it,'' said Edmund; ``but which way did you
turn after passing Sewell's farm?''

``I answer no such irrelevant and insidious questions;
though were I to answer all that you could put in the course
of an hour, you would never be able to prove that it
was \emph{not} Thornton Lacey---for such it certainly was.''

``You inquired, then?''

``No, I never inquire.  But I \emph{told} a man mending a hedge
that it was Thornton Lacey, and he agreed to it.''

``You have a good memory.  I had forgotten having ever
told you half so much of the place.''

Thornton Lacey was the name of his impending living,
as Miss Crawford well knew; and her interest in a negotiation
for William Price's knave increased.

``Well,'' continued Edmund, ``and how did you like what
you saw?''

``Very much indeed.  You are a lucky fellow.  There will be
work for five summers at least before the place is liveable.''

``No, no, not so bad as that.  The farmyard must be moved,
I grant you; but I am not aware of anything else.
The house is by no means bad, and when the yard is removed,
there may be a very tolerable approach to it.''

``The farmyard must be cleared away entirely, and planted
up to shut out the blacksmith's shop.  The house must
be turned to front the east instead of the north---%
the entrance and principal rooms, I mean, must be on
that side, where the view is really very pretty; I am
sure it may be done.  And \emph{there} must be your approach,
through what is at present the garden.  You must make
a new garden at what is now the back of the house;
which will be giving it the best aspect in the world,
sloping to the south-east. The ground seems precisely
formed for it.  I rode fifty yards up the lane,
between the church and the house, in order to look about me;
and saw how it might all be.  Nothing can be easier.
The meadows beyond what \emph{will} \emph{be} the garden, as well
as what now \emph{is}, sweeping round from the lane I stood
in to the north-east, that is, to the principal road
through the village, must be all laid together, of course;
very pretty meadows they are, finely sprinkled with timber.
They belong to the living, I suppose; if not, you must
purchase them.  Then the stream---something must be done
with the stream; but I could not quite determine what.
I had two or three ideas.''

``And I have two or three ideas also,'' said Edmund,
``and one of them is, that very little of your plan
for Thornton Lacey will ever be put in practice.
I must be satisfied with rather less ornament and beauty.
I think the house and premises may be made comfortable,
and given the air of a gentleman's residence, without any
very heavy expense, and that must suffice me; and, I hope,
may suffice all who care about me.''

Miss Crawford, a little suspicious and resentful of a
certain tone of voice, and a certain half-look attending
the last expression of his hope, made a hasty finish
of her dealings with William Price; and securing his knave
at an exorbitant rate, exclaimed, ``There, I will stake
my last like a woman of spirit.  No cold prudence for me.
I am not born to sit still and do nothing.  If I lose
the game, it shall not be from not striving for it.''

The game was hers, and only did not pay her for what
she had given to secure it.  Another deal proceeded,
and Crawford began again about Thornton Lacey.

``My plan may not be the best possible:  I had not many
minutes to form it in; but you must do a good deal.
The place deserves it, and you will find yourself not
satisfied with much less than it is capable of.  (Excuse me,
your ladyship must not see your cards.  There, let them
lie just before you.) The place deserves it, Bertram.
You talk of giving it the air of a gentleman's residence.
\emph{That} will be done by the removal of the farmyard;
for, independent of that terrible nuisance, I never saw
a house of the kind which had in itself so much the air
of a gentleman's residence, so much the look of a something
above a mere parsonage-house---above the expenditure of a few
hundreds a year.  It is not a scrambling collection of low
single rooms, with as many roofs as windows; it is not
cramped into the vulgar compactness of a square farmhouse:
it is a solid, roomy, mansion-like looking house, such as one
might suppose a respectable old country family had lived
in from generation to generation, through two centuries
at least, and were now spending from two to three thousand
a year in.''  Miss Crawford listened, and Edmund agreed
to this.  ``The air of a gentleman's residence, therefore,
you cannot but give it, if you do anything.  But it is
capable of much more.  (Let me see, Mary; Lady Bertram
bids a dozen for that queen; no, no, a dozen is more
than it is worth.  Lady Bertram does not bid a dozen.
She will have nothing to say to it.  Go on, go on.)
By some such improvements as I have suggested (I do not really
require you to proceed upon my plan, though, by the bye,
I doubt anybody's striking out a better) you may give it
a higher character.  You may raise it into a \emph{place}.
From being the mere gentleman's residence, it becomes,
by judicious improvement, the residence of a man
of education, taste, modern manners, good connexions.
All this may be stamped on it; and that house receive
such an air as to make its owner be set down as the great
landholder of the parish by every creature travelling
the road; especially as there is no real squire's house
to dispute the point---a circumstance, between ourselves,
to enhance the value of such a situation in point
of privilege and independence beyond all calculation.
\emph{You} think with me, I hope'' (turning with a softened
voice to Fanny). ``Have you ever seen the place?''

Fanny gave a quick negative, and tried to hide her interest
in the subject by an eager attention to her brother,
who was driving as hard a bargain, and imposing on her
as much as he could; but Crawford pursued with ``No, no,
you must not part with the queen.  You have bought
her too dearly, and your brother does not offer half
her value.  No, no, sir, hands off, hands off.  Your sister
does not part with the queen.  She is quite determined.
The game will be yours,'' turning to her again; ``it will
certainly be yours.''

``And Fanny had much rather it were William's,'' said Edmund,
smiling at her.  ``Poor Fanny! not allowed to cheat herself
as she wishes!''

``Mr.\ Bertram,'' said Miss Crawford, a few minutes afterwards,
``you know Henry to be such a capital improver, that you
cannot possibly engage in anything of the sort at Thornton
Lacey without accepting his help.  Only think how useful
he was at Sotherton!  Only think what grand things were
produced there by our all going with him one hot day
in August to drive about the grounds, and see his genius
take fire.  There we went, and there we came home again;
and what was done there is not to be told!''

Fanny's eyes were turned on Crawford for a moment
with an expression more than grave---even reproachful;
but on catching his, were instantly withdrawn.
With something of consciousness he shook his head at
his sister, and laughingly replied, ``I cannot say there
was much done at Sotherton; but it was a hot day, and we
were all walking after each other, and bewildered.''
As soon as a general buzz gave him shelter, he added,
in a low voice, directed solely at Fanny, ``I should be
sorry to have my powers of \emph{planning} judged of by the
day at Sotherton.  I see things very differently now.
Do not think of me as I appeared then.''

Sotherton was a word to catch Mrs.\ Norris, and being
just then in the happy leisure which followed securing
the odd trick by Sir Thomas's capital play and her own
against Dr.\ and Mrs.\ Grant's great hands, she called out,
in high good-humour, ``Sotherton!  Yes, that is a place,
indeed, and we had a charming day there.  William, you are
quite out of luck; but the next time you come, I hope dear
Mr.\ and Mrs.\ Rushworth will be at home, and I am sure
I can answer for your being kindly received by both.
Your cousins are not of a sort to forget their relations,
and Mr.\ Rushworth is a most amiable man.  They are at
Brighton now, you know; in one of the best houses there,
as Mr.\ Rushworth's fine fortune gives them a right to be.
I do not exactly know the distance, but when you get back
to Portsmouth, if it is not very far off, you ought to go
over and pay your respects to them; and I could send
a little parcel by you that I want to get conveyed to
your cousins.''

``I should be very happy, aunt; but Brighton is almost
by Beachey Head; and if I could get so far, I could
not expect to be welcome in such a smart place as that---%
poor scrubby midshipman as I am.''

Mrs.\ Norris was beginning an eager assurance of the
affability he might depend on, when she was stopped
by Sir Thomas's saying with authority, ``I do not advise
your going to Brighton, William, as I trust you may soon
have more convenient opportunities of meeting; but my
daughters would be happy to see their cousins anywhere;
and you will find Mr.\ Rushworth most sincerely disposed
to regard all the connexions of our family as his own.''

``I would rather find him private secretary to the First
Lord than anything else,'' was William's only answer,
in an undervoice, not meant to reach far, and the
subject dropped.

As yet Sir Thomas had seen nothing to remark in Mr.\ Crawford's
behaviour; but when the whist-table broke up at the end
of the second rubber, and leaving Dr.\ Grant and Mrs.\ Norris
to dispute over their last play, he became a looker-on
at the other, he found his niece the object of attentions,
or rather of professions, of a somewhat pointed character.

Henry Crawford was in the first glow of another scheme
about Thornton Lacey; and not being able to catch
Edmund's ear, was detailing it to his fair neighbour
with a look of considerable earnestness.  His scheme
was to rent the house himself the following winter,
that he might have a home of his own in that neighbourhood;
and it was not merely for the use of it in the hunting-season
(as he was then telling her), though \emph{that} consideration
had certainly some weight, feeling as he did that,
in spite of all Dr.\ Grant's very great kindness, it was
impossible for him and his horses to be accommodated
where they now were without material inconvenience;
but his attachment to that neighbourhood did not depend
upon one amusement or one season of the year:  he had set
his heart upon having a something there that he could
come to at any time, a little homestall at his command,
where all the holidays of his year might be spent, and he
might find himself continuing, improving, and \emph{perfecting}
that friendship and intimacy with the Mansfield Park
family which was increasing in value to him every day.
Sir Thomas heard and was not offended.  There was no want
of respect in the young man's address; and Fanny's reception
of it was so proper and modest, so calm and uninviting,
that he had nothing to censure in her.  She said little,
assented only here and there, and betrayed no inclination
either of appropriating any part of the compliment to herself,
or of strengthening his views in favour of Northamptonshire.
Finding by whom he was observed, Henry Crawford addressed
himself on the same subject to Sir Thomas, in a more
everyday tone, but still with feeling.

``I want to be your neighbour, Sir Thomas, as you have,
perhaps, heard me telling Miss Price.  May I hope
for your acquiescence, and for your not influencing
your son against such a tenant?''

Sir Thomas, politely bowing, replied, ``It is the only way,
sir, in which I could \emph{not} wish you established as a
permanent neighbour; but I hope, and believe, that Edmund will
occupy his own house at Thornton Lacey.  Edmund, am I saying too
much?''

Edmund, on this appeal, had first to hear what was going on;
but, on understanding the question, was at no loss for an answer.

``Certainly, sir, I have no idea but of residence.
But, Crawford, though I refuse you as a tenant,
come to me as a friend.  Consider the house as half
your own every winter, and we will add to the stables
on your own improved plan, and with all the improvements
of your improved plan that may occur to you this spring.''

``We shall be the losers,'' continued Sir Thomas.
``His going, though only eight miles, will be an unwelcome
contraction of our family circle; but I should have been
deeply mortified if any son of mine could reconcile
himself to doing less.  It is perfectly natural that you
should not have thought much on the subject, Mr.\ Crawford.
But a parish has wants and claims which can be known
only by a clergyman constantly resident, and which no
proxy can be capable of satisfying to the same extent.
Edmund might, in the common phrase, do the duty of Thornton,
that is, he might read prayers and preach, without giving
up Mansfield Park:  he might ride over every Sunday, to a
house nominally inhabited, and go through divine service;
he might be the clergyman of Thornton Lacey every seventh day,
for three or four hours, if that would content him.
But it will not.  He knows that human nature needs more
lessons than a weekly sermon can convey; and that if he
does not live among his parishioners, and prove himself,
by constant attention, their well-wisher and friend, he does
very little either for their good or his own.''

Mr.\ Crawford bowed his acquiescence.

``I repeat again,'' added Sir Thomas, ``that Thornton Lacey
is the only house in the neighbourhood in which I should
\emph{not} be happy to wait on Mr.\ Crawford as occupier.''

Mr.\ Crawford bowed his thanks.

``Sir Thomas,'' said Edmund, ``undoubtedly understands
the duty of a parish priest.  We must hope his son
may prove that \emph{he} knows it too.''

Whatever effect Sir Thomas's little harangue might really
produce on Mr.\ Crawford, it raised some awkward sensations
in two of the others, two of his most attentive listeners---%
Miss Crawford and Fanny.  One of whom, having never before
understood that Thornton was so soon and so completely
to be his home, was pondering with downcast eyes on what it
would be \emph{not} to see Edmund every day; and the other,
startled from the agreeable fancies she had been previously
indulging on the strength of her brother's description,
no longer able, in the picture she had been forming of a
future Thornton, to shut out the church, sink the clergyman,
and see only the respectable, elegant, modernised,
and occasional residence of a man of independent fortune,
was considering Sir Thomas, with decided ill-will,
as the destroyer of all this, and suffering the more
from that involuntary forbearance which his character
and manner commanded, and from not daring to relieve
herself by a single attempt at throwing ridicule on his cause.

All the agreeable of \emph{her} speculation was over for that hour.
It was time to have done with cards, if sermons prevailed;
and she was glad to find it necessary to come to a conclusion,
and be able to refresh her spirits by a change of place
and neighbour.

The chief of the party were now collected irregularly
round the fire, and waiting the final break-up. William
and Fanny were the most detached.  They remained
together at the otherwise deserted card-table, talking
very comfortably, and not thinking of the rest, till some
of the rest began to think of them.  Henry Crawford's
chair was the first to be given a direction towards them,
and he sat silently observing them for a few minutes;
himself, in the meanwhile, observed by Sir Thomas,
who was standing in chat with Dr.\ Grant.

``This is the assembly night,'' said William.  ``If I were
at Portsmouth I should be at it, perhaps.''

``But you do not wish yourself at Portsmouth, William?''

``No, Fanny, that I do not.  I shall have enough of Portsmouth
and of dancing too, when I cannot have you.  And I do not
know that there would be any good in going to the assembly,
for I might not get a partner.  The Portsmouth girls turn
up their noses at anybody who has not a commission.
One might as well be nothing as a midshipman.
One \emph{is} nothing, indeed.  You remember the Gregorys;
they are grown up amazing fine girls, but they will hardly
speak to \emph{me}, because Lucy is courted by a lieutenant.''

``Oh! shame, shame!  But never mind it, William'' (her own
cheeks in a glow of indignation as she spoke). ``It is not
worth minding.  It is no reflection on \emph{you}; it is no
more than what the greatest admirals have all experienced,
more or less, in their time.  You must think of that,
you must try to make up your mind to it as one of the
hardships which fall to every sailor's share, like bad
weather and hard living, only with this advantage,
that there will be an end to it, that there will come
a time when you will have nothing of that sort to endure.
When you are a lieutenant! only think, William, when you
are a lieutenant, how little you will care for any nonsense
of this kind.''

``I begin to think I shall never be a lieutenant, Fanny.
Everybody gets made but me.''

``Oh! my dear William, do not talk so; do not be so desponding.
My uncle says nothing, but I am sure he will do everything
in his power to get you made.  He knows, as well as you do,
of what consequence it is.''

She was checked by the sight of her uncle much nearer
to them than she had any suspicion of, and each found
it necessary to talk of something else.

``Are you fond of dancing, Fanny?''

``Yes, very; only I am soon tired.''

``I should like to go to a ball with you and see
you dance.  Have you never any balls at Northampton?
I should like to see you dance, and I'd dance with you
if you \emph{would}, for nobody would know who I was here,
and I should like to be your partner once more.
We used to jump about together many a time, did not we?
when the hand-organ was in the street?  I am a pretty
good dancer in my way, but I dare say you are a better.''
And turning to his uncle, who was now close to them,
``Is not Fanny a very good dancer, sir?''

Fanny, in dismay at such an unprecedented question,
did not know which way to look, or how to be prepared
for the answer.  Some very grave reproof, or at least
the coldest expression of indifference, must be coming
to distress her brother, and sink her to the ground.
But, on the contrary, it was no worse than, ``I am sorry
to say that I am unable to answer your question.
I have never seen Fanny dance since she was a little girl;
but I trust we shall both think she acquits herself like
a gentlewoman when we do see her, which, perhaps, we may
have an opportunity of doing ere long.''

``I have had the pleasure of seeing your sister dance,
Mr.\ Price,'' said Henry Crawford, leaning forward,
``and will engage to answer every inquiry which you can
make on the subject, to your entire satisfaction.
But I believe'' (seeing Fanny looked distressed) ``it must
be at some other time.  There is \emph{one} person in company
who does not like to have Miss Price spoken of.''

True enough, he had once seen Fanny dance; and it was
equally true that he would now have answered for her gliding
about with quiet, light elegance, and in admirable time;
but, in fact, he could not for the life of him recall
what her dancing had been, and rather took it for granted
that she had been present than remembered anything about her.

He passed, however, for an admirer of her dancing;
and Sir Thomas, by no means displeased, prolonged the
conversation on dancing in general, and was so well
engaged in describing the balls of Antigua, and listening
to what his nephew could relate of the different modes
of dancing which had fallen within his observation,
that he had not heard his carriage announced, and was first
called to the knowledge of it by the bustle of Mrs.\ Norris.

``Come, Fanny, Fanny, what are you about?  We are going.
Do not you see your aunt is going?  Quick, quick!  I cannot
bear to keep good old Wilcox waiting.  You should always
remember the coachman and horses.  My dear Sir Thomas,
we have settled it that the carriage should come back for you,
and Edmund and William.''

Sir Thomas could not dissent, as it had been his
own arrangement, previously communicated to his wife
and sister; but \emph{that} seemed forgotten by Mrs.\ Norris,
who must fancy that she settled it all herself.

Fanny's last feeling in the visit was disappointment:
for the shawl which Edmund was quietly taking from the
servant to bring and put round her shoulders was seized
by Mr.\ Crawford's quicker hand, and she was obliged to be
indebted to his more prominent attention.



\chapter{Chapter 26}

\gintro{William's} desire of seeing Fanny dance made more than a
momentary impression on his uncle.  The hope of an opportunity,
which Sir Thomas had then given, was not given to be thought
of no more.  He remained steadily inclined to gratify
so amiable a feeling; to gratify anybody else who might
wish to see Fanny dance, and to give pleasure to the young
people in general; and having thought the matter over,
and taken his resolution in quiet independence,
the result of it appeared the next morning at breakfast,
when, after recalling and commending what his nephew
had said, he added, ``I do not like, William, that you
should leave Northamptonshire without this indulgence.
It would give me pleasure to see you both dance.
You spoke of the balls at Northampton.  Your cousins have
occasionally attended them; but they would not altogether
suit us now.  The fatigue would be too much for your aunt.
I believe we must not think of a Northampton ball.
A dance at home would be more eligible; and if---''

``Ah, my dear Sir Thomas!'' interrupted Mrs.\ Norris, ``I knew
what was coming.  I knew what you were going to say.  If dear
Julia were at home, or dearest Mrs.\ Rushworth at Sotherton,
to afford a reason, an occasion for such a thing, you would
be tempted to give the young people a dance at Mansfield.
I know you would.  If \emph{they} were at home to grace
the ball, a ball you would have this very Christmas.
Thank your uncle, William, thank your uncle!''

``My daughters,'' replied Sir Thomas, gravely interposing,
``have their pleasures at Brighton, and I hope are very happy;
but the dance which I think of giving at Mansfield
will be for their cousins.  Could we be all assembled,
our satisfaction would undoubtedly be more complete,
but the absence of some is not to debar the others
of amusement.''

Mrs.\ Norris had not another word to say.  She saw decision
in his looks, and her surprise and vexation required
some minutes' silence to be settled into composure.
A ball at such a time!  His daughters absent and herself
not consulted!  There was comfort, however, soon at hand.
\emph{She} must be the doer of everything:  Lady Bertram
would of course be spared all thought and exertion,
and it would all fall upon \emph{her}.  She should have to do
the honours of the evening; and this reflection quickly
restored so much of her good-humour as enabled her to join
in with the others, before their happiness and thanks were
all expressed.

Edmund, William, and Fanny did, in their different ways,
look and speak as much grateful pleasure in the promised
ball as Sir Thomas could desire.  Edmund's feelings
were for the other two.  His father had never conferred
a favour or shewn a kindness more to his satisfaction.

Lady Bertram was perfectly quiescent and contented,
and had no objections to make.  Sir Thomas engaged
for its giving her very little trouble; and she assured
him ``that she was not at all afraid of the trouble;
indeed, she could not imagine there would be any.''

Mrs.\ Norris was ready with her suggestions as to the rooms he
would think fittest to be used, but found it all prearranged;
and when she would have conjectured and hinted about
the day, it appeared that the day was settled too.
Sir Thomas had been amusing himself with shaping a very
complete outline of the business; and as soon as she
would listen quietly, could read his list of the families
to be invited, from whom he calculated, with all necessary
allowance for the shortness of the notice, to collect
young people enough to form twelve or fourteen couple:
and could detail the considerations which had induced
him to fix on the 22nd as the most eligible day.
William was required to be at Portsmouth on the 24th;
the 22nd would therefore be the last day of his visit;
but where the days were so few it would be unwise to fix
on any earlier.  Mrs.\ Norris was obliged to be satisfied
with thinking just the same, and with having been on the
point of proposing the 22nd herself, as by far the best day
for the purpose.

The ball was now a settled thing, and before the evening
a proclaimed thing to all whom it concerned.  Invitations were
sent with despatch, and many a young lady went to bed that
night with her head full of happy cares as well as Fanny.
To her the cares were sometimes almost beyond the happiness;
for young and inexperienced, with small means of choice
and no confidence in her own taste, the ``how she
should be dressed'' was a point of painful solicitude;
and the almost solitary ornament in her possession,
a very pretty amber cross which William had brought
her from Sicily, was the greatest distress of all,
for she had nothing but a bit of ribbon to fasten it to;
and though she had worn it in that manner once, would it
be allowable at such a time in the midst of all the rich
ornaments which she supposed all the other young ladies
would appear in?  And yet not to wear it!  William had
wanted to buy her a gold chain too, but the purchase had
been beyond his means, and therefore not to wear the cross
might be mortifying him.  These were anxious considerations;
enough to sober her spirits even under the prospect
of a ball given principally for her gratification.

The preparations meanwhile went on, and Lady Bertram continued
to sit on her sofa without any inconvenience from them.
She had some extra visits from the housekeeper, and her
maid was rather hurried in making up a new dress for her:
Sir Thomas gave orders, and Mrs.\ Norris ran about;
but all this gave \emph{her} no trouble, and as she had foreseen,
``there was, in fact, no trouble in the business.''

Edmund was at this time particularly full of cares:
his mind being deeply occupied in the consideration of two
important events now at hand, which were to fix his fate
in life---ordination and matrimony---events of such a serious
character as to make the ball, which would be very quickly
followed by one of them, appear of less moment in his
eyes than in those of any other person in the house.
On the 23rd he was going to a friend near Peterborough,
in the same situation as himself, and they were to
receive ordination in the course of the Christmas week.
Half his destiny would then be determined, but the other
half might not be so very smoothly wooed.  His duties would
be established, but the wife who was to share, and animate,
and reward those duties, might yet be unattainable.
He knew his own mind, but he was not always perfectly assured
of knowing Miss Crawford's. There were points on which they
did not quite agree; there were moments in which she did
not seem propitious; and though trusting altogether to
her affection, so far as to be resolved---almost resolved---%
on bringing it to a decision within a very short time,
as soon as the variety of business before him were arranged,
and he knew what he had to offer her, he had many
anxious feelings, many doubting hours as to the result.
His conviction of her regard for him was sometimes very strong;
he could look back on a long course of encouragement,
and she was as perfect in disinterested attachment as
in everything else.  But at other times doubt and alarm
intermingled with his hopes; and when he thought of her
acknowledged disinclination for privacy and retirement,
her decided preference of a London life, what could he expect
but a determined rejection? unless it were an acceptance
even more to be deprecated, demanding such sacrifices
of situation and employment on his side as conscience
must forbid.

The issue of all depended on one question.  Did she
love him well enough to forego what had used to be
essential points?  Did she love him well enough to make
them no longer essential?  And this question, which he
was continually repeating to himself, though oftenest
answered with a ``Yes,'' had sometimes its ``No.''

Miss Crawford was soon to leave Mansfield, and on this
circumstance the ``no'' and the ``yes'' had been very recently
in alternation.  He had seen her eyes sparkle as she spoke
of the dear friend's letter, which claimed a long visit from
her in London, and of the kindness of Henry, in engaging
to remain where he was till January, that he might convey
her thither; he had heard her speak of the pleasure of such
a journey with an animation which had ``no'' in every tone.
But this had occurred on the first day of its being settled,
within the first hour of the burst of such enjoyment,
when nothing but the friends she was to visit was before her.
He had since heard her express herself differently,
with other feelings, more chequered feelings:  he had heard
her tell Mrs.\ Grant that she should leave her with regret;
that she began to believe neither the friends nor
the pleasures she was going to were worth those she
left behind; and that though she felt she must go,
and knew she should enjoy herself when once away, she was
already looking forward to being at Mansfield again.
Was there not a ``yes'' in all this?

With such matters to ponder over, and arrange, and re-arrange,
Edmund could not, on his own account, think very much
of the evening which the rest of the family were looking
forward to with a more equal degree of strong interest.
Independent of his two cousins' enjoyment in it,
the evening was to him of no higher value than any
other appointed meeting of the two families might be.
In every meeting there was a hope of receiving farther
confirmation of Miss Crawford's attachment; but the whirl
of a ballroom, perhaps, was not particularly favourable
to the excitement or expression of serious feelings.
To engage her early for the two first dances was all the
command of individual happiness which he felt in his power,
and the only preparation for the ball which he could
enter into, in spite of all that was passing around him
on the subject, from morning till night.

Thursday was the day of the ball; and on Wednesday
morning Fanny, still unable to satisfy herself as to what
she ought to wear, determined to seek the counsel of the
more enlightened, and apply to Mrs.\ Grant and her sister,
whose acknowledged taste would certainly bear her blameless;
and as Edmund and William were gone to Northampton,
and she had reason to think Mr.\ Crawford likewise out,
she walked down to the Parsonage without much fear of wanting
an opportunity for private discussion; and the privacy of
such a discussion was a most important part of it to Fanny,
being more than half-ashamed of her own solicitude.

She met Miss Crawford within a few yards of the Parsonage,
just setting out to call on her, and as it seemed to her
that her friend, though obliged to insist on turning back,
was unwilling to lose her walk, she explained her business
at once, and observed, that if she would be so kind
as to give her opinion, it might be all talked over as
well without doors as within.  Miss Crawford appeared
gratified by the application, and after a moment's thought,
urged Fanny's returning with her in a much more cordial
manner than before, and proposed their going up into
her room, where they might have a comfortable coze,
without disturbing Dr.\ and Mrs.\ Grant, who were together
in the drawing-room. It was just the plan to suit Fanny;
and with a great deal of gratitude on her side for such ready
and kind attention, they proceeded indoors, and upstairs,
and were soon deep in the interesting subject.  Miss Crawford,
pleased with the appeal, gave her all her best judgment
and taste, made everything easy by her suggestions,
and tried to make everything agreeable by her encouragement.
The dress being settled in all its grander parts---%
``But what shall you have by way of necklace?'' said Miss
Crawford.  ``Shall not you wear your brother's cross?''
And as she spoke she was undoing a small parcel,
which Fanny had observed in her hand when they met.
Fanny acknowledged her wishes and doubts on this point:
she did not know how either to wear the cross, or to
refrain from wearing it.  She was answered by having
a small trinket-box placed before her, and being requested
to chuse from among several gold chains and necklaces.
Such had been the parcel with which Miss Crawford
was provided, and such the object of her intended visit:
and in the kindest manner she now urged Fanny's taking one
for the cross and to keep for her sake, saying everything
she could think of to obviate the scruples which were
making Fanny start back at first with a look of horror at
the proposal.

``You see what a collection I have,'' said she; ``more by half
than I ever use or think of.  I do not offer them as new.
I offer nothing but an old necklace.  You must forgive
the liberty, and oblige me.''

Fanny still resisted, and from her heart.  The gift was
too valuable.  But Miss Crawford persevered, and argued
the case with so much affectionate earnestness through
all the heads of William and the cross, and the ball,
and herself, as to be finally successful.  Fanny found
herself obliged to yield, that she might not be accused
of pride or indifference, or some other littleness;
and having with modest reluctance given her consent,
proceeded to make the selection.  She looked and looked,
longing to know which might be least valuable; and was
determined in her choice at last, by fancying there was
one necklace more frequently placed before her eyes than
the rest.  It was of gold, prettily worked; and though Fanny
would have preferred a longer and a plainer chain as more
adapted for her purpose, she hoped, in fixing on this,
to be chusing what Miss Crawford least wished to keep.
Miss Crawford smiled her perfect approbation; and hastened
to complete the gift by putting the necklace round her,
and making her see how well it looked.  Fanny had not a
word to say against its becomingness, and, excepting what
remained of her scruples, was exceedingly pleased with an
acquisition so very apropos.  She would rather, perhaps,
have been obliged to some other person.  But this was
an unworthy feeling.  Miss Crawford had anticipated her
wants with a kindness which proved her a real friend.
``When I wear this necklace I shall always think of you,''
said she, ``and feel how very kind you were.''

``You must think of somebody else too, when you wear
that necklace,'' replied Miss Crawford.  ``You must think
of Henry, for it was his choice in the first place.
He gave it to me, and with the necklace I make over
to you all the duty of remembering the original giver.
It is to be a family remembrancer.  The sister is not to be
in your mind without bringing the brother too.''

Fanny, in great astonishment and confusion, would have
returned the present instantly.  To take what had
been the gift of another person, of a brother too,
impossible! it must not be! and with an eagerness and
embarrassment quite diverting to her companion, she laid
down the necklace again on its cotton, and seemed resolved
either to take another or none at all.  Miss Crawford
thought she had never seen a prettier consciousness.
``My dear child,'' said she, laughing, ``what are you afraid of?
Do you think Henry will claim the necklace as mine,
and fancy you did not come honestly by it? or are you
imagining he would be too much flattered by seeing
round your lovely throat an ornament which his money
purchased three years ago, before he knew there was such
a throat in the world? or perhaps''---looking archly---%
``you suspect a confederacy between us, and that what
I am now doing is with his knowledge and at his desire?''

With the deepest blushes Fanny protested against such
a thought.

``Well, then,'' replied Miss Crawford more seriously,
but without at all believing her, ``to convince me that you
suspect no trick, and are as unsuspicious of compliment
as I have always found you, take the necklace and say
no more about it.  Its being a gift of my brother's need
not make the smallest difference in your accepting it,
as I assure you it makes none in my willingness to part
with it.  He is always giving me something or other.
I have such innumerable presents from him that it is quite
impossible for me to value or for him to remember half.
And as for this necklace, I do not suppose I have worn it
six times:  it is very pretty, but I never think of it;
and though you would be most heartily welcome to any
other in my trinket-box, you have happened to fix on
the very one which, if I have a choice, I would rather
part with and see in your possession than any other.
Say no more against it, I entreat you.  Such a trifle is
not worth half so many words.''

Fanny dared not make any farther opposition; and with
renewed but less happy thanks accepted the necklace again,
for there was an expression in Miss Crawford's eyes
which she could not be satisfied with.

It was impossible for her to be insensible of Mr.\ Crawford's
change of manners.  She had long seen it.  He evidently
tried to please her:  he was gallant, he was attentive,
he was something like what he had been to her cousins:
he wanted, she supposed, to cheat her of her tranquillity
as he had cheated them; and whether he might not have some
concern in this necklace---she could not be convinced that
he had not, for Miss Crawford, complaisant as a sister,
was careless as a woman and a friend.

Reflecting and doubting, and feeling that the possession
of what she had so much wished for did not bring much
satisfaction, she now walked home again, with a change rather
than a diminution of cares since her treading that path before.



\chapter{Chapter 27}

\gintro{On reaching home} Fanny went immediately upstairs to
deposit this unexpected acquisition, this doubtful good
of a necklace, in some favourite box in the East room,
which held all her smaller treasures; but on opening
the door, what was her surprise to find her cousin Edmund
there writing at the table!  Such a sight having never
occurred before, was almost as wonderful as it was welcome.

``Fanny,'' said he directly, leaving his seat and his pen,
and meeting her with something in his hand, ``I beg
your pardon for being here.  I came to look for you,
and after waiting a little while in hope of your coming in,
was making use of your inkstand to explain my errand.
You will find the beginning of a note to yourself;
but I can now speak my business, which is merely to beg
your acceptance of this little trifle---a chain for
William's cross.  You ought to have had it a week ago,
but there has been a delay from my brother's not
being in town by several days so soon as I expected;
and I have only just now received it at Northampton.
I hope you will like the chain itself, Fanny.  I endeavoured
to consult the simplicity of your taste; but, at any rate,
I know you will be kind to my intentions, and consider it,
as it really is, a token of the love of one of your
oldest friends.''

And so saying, he was hurrying away, before Fanny,
overpowered by a thousand feelings of pain and pleasure,
could attempt to speak; but quickened by one sovereign wish,
she then called out, ``Oh! cousin, stop a moment,
pray stop!''

He turned back.

``I cannot attempt to thank you,'' she continued, in a
very agitated manner; ``thanks are out of the question.
I feel much more than I can possibly express.
Your goodness in thinking of me in such a way is beyond---%
''

``If that is all you have to say, Fanny'' smiling and turning
away again.

``No, no, it is not.  I want to consult you.''

Almost unconsciously she had now undone the parcel he
had just put into her hand, and seeing before her,
in all the niceness of jewellers' packing, a plain
gold chain, perfectly simple and neat, she could not help
bursting forth again, ``Oh, this is beautiful indeed!
This is the very thing, precisely what I wished for!
This is the only ornament I have ever had a desire to possess.
It will exactly suit my cross.  They must and shall be
worn together.  It comes, too, in such an acceptable moment.
Oh, cousin, you do not know how acceptable it is.''

``My dear Fanny, you feel these things a great deal too much.
I am most happy that you like the chain, and that it
should be here in time for to-morrow; but your thanks are
far beyond the occasion.  Believe me, I have no pleasure
in the world superior to that of contributing to yours.
No, I can safely say, I have no pleasure so complete,
so unalloyed.  It is without a drawback.''

Upon such expressions of affection Fanny could have
lived an hour without saying another word; but Edmund,
after waiting a moment, obliged her to bring down her
mind from its heavenly flight by saying, ``But what is it
that you want to consult me about?''

It was about the necklace, which she was now most earnestly
longing to return, and hoped to obtain his approbation
of her doing.  She gave the history of her recent visit,
and now her raptures might well be over; for Edmund was so
struck with the circumstance, so delighted with what Miss
Crawford had done, so gratified by such a coincidence
of conduct between them, that Fanny could not but admit
the superior power of one pleasure over his own mind,
though it might have its drawback.  It was some time
before she could get his attention to her plan, or any
answer to her demand of his opinion:  he was in a reverie
of fond reflection, uttering only now and then a few
half-sentences of praise; but when he did awake and understand,
he was very decided in opposing what she wished.

``Return the necklace!  No, my dear Fanny, upon no account.
It would be mortifying her severely.  There can hardly
be a more unpleasant sensation than the having anything
returned on our hands which we have given with a reasonable
hope of its contributing to the comfort of a friend.
Why should she lose a pleasure which she has shewn herself
so deserving of?''

``If it had been given to me in the first instance,''
said Fanny, ``I should not have thought of returning it;
but being her brother's present, is not it fair to suppose
that she would rather not part with it, when it is
not wanted?''

``She must not suppose it not wanted, not acceptable,
at least:  and its having been originally her brother's
gift makes no difference; for as she was not prevented
from offering, nor you from taking it on that account,
it ought not to prevent you from keeping it.  No doubt it
is handsomer than mine, and fitter for a ballroom.''

``No, it is not handsomer, not at all handsomer
in its way, and, for my purpose, not half so fit.
The chain will agree with William's cross beyond
all comparison better than the necklace.''

``For one night, Fanny, for only one night, if it \emph{be}
a sacrifice; I am sure you will, upon consideration,
make that sacrifice rather than give pain to one who has been
so studious of your comfort.  Miss Crawford's attentions
to you have been---not more than you were justly entitled to---%
I am the last person to think that \emph{could} \emph{be},
but they have been invariable; and to be returning them
with what must have something the \emph{air} of ingratitude,
though I know it could never have the \emph{meaning}, is not
in your nature, I am sure.  Wear the necklace, as you
are engaged to do, to-morrow evening, and let the chain,
which was not ordered with any reference to the ball,
be kept for commoner occasions.  This is my advice.
I would not have the shadow of a coolness between the two whose
intimacy I have been observing with the greatest pleasure,
and in whose characters there is so much general resemblance
in true generosity and natural delicacy as to make the few
slight differences, resulting principally from situation,
no reasonable hindrance to a perfect friendship.  I would
not have the shadow of a coolness arise,'' he repeated,
his voice sinking a little, ``between the two dearest objects
I have on earth.''

He was gone as he spoke; and Fanny remained to tranquillise
herself as she could.  She was one of his two dearest---%
that must support her.  But the other:  the first!
She had never heard him speak so openly before, and though
it told her no more than what she had long perceived,
it was a stab, for it told of his own convictions and views.
They were decided.  He would marry Miss Crawford.
It was a stab, in spite of every long-standing expectation;
and she was obliged to repeat again and again, that she
was one of his two dearest, before the words gave
her any sensation.  Could she believe Miss Crawford to
deserve him, it would be---oh, how different would it be---%
how far more tolerable!  But he was deceived in her:
he gave her merits which she had not; her faults were
what they had ever been, but he saw them no longer.
Till she had shed many tears over this deception,
Fanny could not subdue her agitation; and the dejection
which followed could only be relieved by the influence of
fervent prayers for his happiness.

It was her intention, as she felt it to be her duty,
to try to overcome all that was excessive, all that
bordered on selfishness, in her affection for Edmund.
To call or to fancy it a loss, a disappointment, would be
a presumption for which she had not words strong enough to
satisfy her own humility.  To think of him as Miss Crawford
might be justified in thinking, would in her be insanity.
To her he could be nothing under any circumstances;
nothing dearer than a friend.  Why did such an idea occur
to her even enough to be reprobated and forbidden?  It ought
not to have touched on the confines of her imagination.
She would endeavour to be rational, and to deserve
the right of judging of Miss Crawford's character,
and the privilege of true solicitude for him by a sound
intellect and an honest heart.

She had all the heroism of principle, and was determined
to do her duty; but having also many of the feelings of youth
and nature, let her not be much wondered at, if, after making
all these good resolutions on the side of self-government,
she seized the scrap of paper on which Edmund had begun
writing to her, as a treasure beyond all her hopes,
and reading with the tenderest emotion these words,
``My very dear Fanny, you must do me the favour to accept''
locked it up with the chain, as the dearest part of the gift.
It was the only thing approaching to a letter which she
had ever received from him; she might never receive another;
it was impossible that she ever should receive another
so perfectly gratifying in the occasion and the style.
Two lines more prized had never fallen from the pen
of the most distinguished author---never more completely
blessed the researches of the fondest biographer.
The enthusiasm of a woman's love is even beyond
the biographer's. To her, the handwriting itself,
independent of anything it may convey, is a blessedness.
Never were such characters cut by any other human being
as Edmund's commonest handwriting gave!  This specimen,
written in haste as it was, had not a fault; and there
was a felicity in the flow of the first four words,
in the arrangement of ``My very dear Fanny,'' which she
could have looked at for ever.

Having regulated her thoughts and comforted her feelings
by this happy mixture of reason and weakness, she was able
in due time to go down and resume her usual employments
near her aunt Bertram, and pay her the usual observances
without any apparent want of spirits.

Thursday, predestined to hope and enjoyment, came; and opened
with more kindness to Fanny than such self-willed,
unmanageable days often volunteer, for soon after breakfast
a very friendly note was brought from Mr.\ Crawford
to William, stating that as he found himself obliged
to go to London on the morrow for a few days, he could
not help trying to procure a companion; and therefore
hoped that if William could make up his mind to leave
Mansfield half a day earlier than had been proposed,
he would accept a place in his carriage.  Mr.\ Crawford meant
to be in town by his uncle's accustomary late dinner-hour,
and William was invited to dine with him at the Admiral's.
The proposal was a very pleasant one to William himself,
who enjoyed the idea of travelling post with four horses,
and such a good-humoured, agreeable friend; and, in likening
it to going up with despatches, was saying at once everything
in favour of its happiness and dignity which his imagination
could suggest; and Fanny, from a different motive,
was exceedingly pleased; for the original plan was that
William should go up by the mail from Northampton the
following night, which would not have allowed him an hour's
rest before he must have got into a Portsmouth coach;
and though this offer of Mr.\ Crawford's would rob her
of many hours of his company, she was too happy in having
William spared from the fatigue of such a journey,
to think of anything else.  Sir Thomas approved of it
for another reason.  His nephew's introduction to Admiral
Crawford might be of service.  The Admiral, he believed,
had interest.  Upon the whole, it was a very joyous note.
Fanny's spirits lived on it half the morning, deriving
some accession of pleasure from its writer being himself to go
away.

As for the ball, so near at hand, she had too many
agitations and fears to have half the enjoyment in
anticipation which she ought to have had, or must have
been supposed to have by the many young ladies looking
forward to the same event in situations more at ease,
but under circumstances of less novelty, less interest,
less peculiar gratification, than would be attributed
to her.  Miss Price, known only by name to half the
people invited, was now to make her first appearance,
and must be regarded as the queen of the evening.
Who could be happier than Miss Price?  But Miss Price
had not been brought up to the trade of \emph{coming} \emph{out};
and had she known in what light this ball was, in general,
considered respecting her, it would very much have lessened
her comfort by increasing the fears she already had of doing
wrong and being looked at.  To dance without much observation
or any extraordinary fatigue, to have strength and partners
for about half the evening, to dance a little with Edmund,
and not a great deal with Mr.\ Crawford, to see William
enjoy himself, and be able to keep away from her aunt Norris,
was the height of her ambition, and seemed to comprehend
her greatest possibility of happiness.  As these were
the best of her hopes, they could not always prevail;
and in the course of a long morning, spent principally
with her two aunts, she was often under the influence
of much less sanguine views.  William, determined to
make this last day a day of thorough enjoyment,
was out snipe-shooting; Edmund, she had too much reason
to suppose, was at the Parsonage; and left alone to bear
the worrying of Mrs.\ Norris, who was cross because the
housekeeper would have her own way with the supper,
and whom \emph{she} could not avoid though the housekeeper might,
Fanny was worn down at last to think everything an evil
belonging to the ball, and when sent off with a parting worry
to dress, moved as languidly towards her own room, and felt
as incapable of happiness as if she had been allowed no share in
it.

As she walked slowly upstairs she thought of yesterday;
it had been about the same hour that she had returned
from the Parsonage, and found Edmund in the East room.
``Suppose I were to find him there again to-day!'' said she
to herself, in a fond indulgence of fancy.

``Fanny,'' said a voice at that moment near her.
Starting and looking up, she saw, across the lobby she
had just reached, Edmund himself, standing at the head
of a different staircase.  He came towards her.  ``You look
tired and fagged, Fanny.  You have been walking too far.''

``No, I have not been out at all.''

``Then you have had fatigues within doors, which are worse.
You had better have gone out.''

Fanny, not liking to complain, found it easiest to make
no answer; and though he looked at her with his usual kindness,
she believed he had soon ceased to think of her countenance.
He did not appear in spirits:  something unconnected with
her was probably amiss.  They proceeded upstairs together,
their rooms being on the same floor above.

``I come from Dr.\ Grant's,'' said Edmund presently.
``You may guess my errand there, Fanny.''  And he looked
so conscious, that Fanny could think but of one errand,
which turned her too sick for speech.  ``I wished to
engage Miss Crawford for the two first dances,'' was the
explanation that followed, and brought Fanny to life again,
enabling her, as she found she was expected to speak,
to utter something like an inquiry as to the result.

``Yes,'' he answered, ``she is engaged to me; but'' (with a smile
that did not sit easy) ``she says it is to be the last time
that she ever will dance with me.  She is not serious.
I think, I hope, I am sure she is not serious; but I would
rather not hear it.  She never has danced with a clergyman,
she says, and she never \emph{will}.  For my own sake, I could
wish there had been no ball just at---I mean not this
very week, this very day; to-morrow I leave home.''

Fanny struggled for speech, and said, ``I am very sorry
that anything has occurred to distress you.  This ought
to be a day of pleasure.  My uncle meant it so.''

``Oh yes, yes! and it will be a day of pleasure.
It will all end right.  I am only vexed for a moment.
In fact, it is not that I consider the ball as ill-timed;
what does it signify?  But, Fanny,'' stopping her,
by taking her hand, and speaking low and seriously,
``you know what all this means.  You see how it is;
and could tell me, perhaps better than I could tell you,
how and why I am vexed.  Let me talk to you a little.
You are a kind, kind listener.  I have been pained
by her manner this morning, and cannot get the better
of it.  I know her disposition to be as sweet and
faultless as your own, but the influence of her former
companions makes her seem---gives to her conversation,
to her professed opinions, sometimes a tinge of wrong.
She does not \emph{think} evil, but she speaks it, speaks it
in playfulness; and though I know it to be playfulness,
it grieves me to the soul.''

``The effect of education,'' said Fanny gently.

Edmund could not but agree to it.  ``Yes, that uncle and aunt!
They have injured the finest mind; for sometimes,
Fanny, I own to you, it does appear more than manner:
it appears as if the mind itself was tainted.''

Fanny imagined this to be an appeal to her judgment,
and therefore, after a moment's consideration, said, ``If you
only want me as a listener, cousin, I will be as useful
as I can; but I am not qualified for an adviser.
Do not ask advice of \emph{me}.  I am not competent.''

``You are right, Fanny, to protest against such an office,
but you need not be afraid.  It is a subject on which I
should never ask advice; it is the sort of subject on
which it had better never be asked; and few, I imagine,
do ask it, but when they want to be influenced against
their conscience.  I only want to talk to you.''

``One thing more.  Excuse the liberty; but take care
\emph{how} you talk to me.  Do not tell me anything now,
which hereafter you may be sorry for.  The time may come---''

The colour rushed into her cheeks as she spoke.

``Dearest Fanny!'' cried Edmund, pressing her hand to
his lips with almost as much warmth as if it had been
Miss Crawford's, ``you are all considerate thought!
But it is unnecessary here.  The time will never come.
No such time as you allude to will ever come.  I begin to
think it most improbable:  the chances grow less and less;
and even if it should, there will be nothing to be
remembered by either you or me that we need be afraid of,
for I can never be ashamed of my own scruples; and if they
are removed, it must be by changes that will only raise
her character the more by the recollection of the faults
she once had.  You are the only being upon earth to whom
I should say what I have said; but you have always known
my opinion of her; you can bear me witness, Fanny, that I
have never been blinded.  How many a time have we
talked over her little errors!  You need not fear me;
I have almost given up every serious idea of her;
but I must be a blockhead indeed, if, whatever befell me,
I could think of your kindness and sympathy without the
sincerest gratitude.''

He had said enough to shake the experience of eighteen.
He had said enough to give Fanny some happier feelings
than she had lately known, and with a brighter look,
she answered, ``Yes, cousin, I am convinced that \emph{you}
would be incapable of anything else, though perhaps some
might not.  I cannot be afraid of hearing anything you
wish to say.  Do not check yourself.  Tell me whatever
you like.''

They were now on the second floor, and the appearance
of a housemaid prevented any farther conversation.
For Fanny's present comfort it was concluded, perhaps,
at the happiest moment:  had he been able to talk another
five minutes, there is no saying that he might not have talked
away all Miss Crawford's faults and his own despondence.
But as it was, they parted with looks on his side of
grateful affection, and with some very precious sensations
on hers.  She had felt nothing like it for hours.
Since the first joy from Mr.\ Crawford's note to William had
worn away, she had been in a state absolutely the reverse;
there had been no comfort around, no hope within her.
Now everything was smiling.  William's good fortune
returned again upon her mind, and seemed of greater
value than at first.  The ball, too---such an evening
of pleasure before her!  It was now a real animation;
and she began to dress for it with much of the happy
flutter which belongs to a ball.  All went well:
she did not dislike her own looks; and when she came
to the necklaces again, her good fortune seemed complete,
for upon trial the one given her by Miss Crawford would
by no means go through the ring of the cross.  She had,
to oblige Edmund, resolved to wear it; but it was too
large for the purpose.  His, therefore, must be worn;
and having, with delightful feelings, joined the chain
and the cross---those memorials of the two most beloved
of her heart, those dearest tokens so formed for each
other by everything real and imaginary---and put them
round her neck, and seen and felt how full of William
and Edmund they were, she was able, without an effort,
to resolve on wearing Miss Crawford's necklace too.
She acknowledged it to be right.  Miss Crawford had a claim;
and when it was no longer to encroach on, to interfere
with the stronger claims, the truer kindness of another,
she could do her justice even with pleasure to herself.
The necklace really looked very well; and Fanny left her
room at last, comfortably satisfied with herself and all
about her.

Her aunt Bertram had recollected her on this occasion with
an unusual degree of wakefulness.  It had really occurred
to her, unprompted, that Fanny, preparing for a ball,
might be glad of better help than the upper housemaid's,
and when dressed herself, she actually sent her own maid
to assist her; too late, of course, to be of any use.
Mrs.\ Chapman had just reached the attic floor, when Miss
Price came out of her room completely dressed, and only
civilities were necessary; but Fanny felt her aunt's
attention almost as much as Lady Bertram or Mrs.\ Chapman
could do themselves.



\chapter{Chapter 28}

\gintro{Her uncle} and both her aunts were in the drawing-room
when Fanny went down.  To the former she was an interesting
object, and he saw with pleasure the general elegance
of her appearance, and her being in remarkably good looks.
The neatness and propriety of her dress was all that
he would allow himself to commend in her presence,
but upon her leaving the room again soon afterwards,
he spoke of her beauty with very decided praise.

``Yes,'' said Lady Bertram, ``she looks very well.
I sent Chapman to her.''

``Look well!  Oh, yes!'' cried Mrs.\ Norris, ``she has
good reason to look well with all her advantages:
brought up in this family as she has been, with all
the benefit of her cousins' manners before her.
Only think, my dear Sir Thomas, what extraordinary
advantages you and I have been the means of giving her.
The very gown you have been taking notice of is your own
generous present to her when dear Mrs.\ Rushworth married.
What would she have been if we had not taken her by
the hand?''

Sir Thomas said no more; but when they sat down to table
the eyes of the two young men assured him that the subject
might be gently touched again, when the ladies withdrew,
with more success.  Fanny saw that she was approved;
and the consciousness of looking well made her look
still better.  From a variety of causes she was happy,
and she was soon made still happier; for in following her
aunts out of the room, Edmund, who was holding open the door,
said, as she passed him, ``You must dance with me, Fanny;
you must keep two dances for me; any two that you like,
except the first.''  She had nothing more to wish for.
She had hardly ever been in a state so nearly approaching
high spirits in her life.  Her cousins' former gaiety
on the day of a ball was no longer surprising to her;
she felt it to be indeed very charming, and was actually
practising her steps about the drawing-room as long as she
could be safe from the notice of her aunt Norris, who was
entirely taken up at first in fresh arranging and injuring
the noble fire which the butler had prepared.

Half an hour followed that would have been at least languid
under any other circumstances, but Fanny's happiness
still prevailed.  It was but to think of her conversation
with Edmund, and what was the restlessness of Mrs.\ Norris?
What were the yawns of Lady Bertram?

The gentlemen joined them; and soon after began the sweet
expectation of a carriage, when a general spirit of ease
and enjoyment seemed diffused, and they all stood about
and talked and laughed, and every moment had its pleasure
and its hope.  Fanny felt that there must be a struggle
in Edmund's cheerfulness, but it was delightful to see
the effort so successfully made.

When the carriages were really heard, when the guests began
really to assemble, her own gaiety of heart was much subdued:
the sight of so many strangers threw her back into herself;
and besides the gravity and formality of the first great circle,
which the manners of neither Sir Thomas nor Lady Bertram
were of a kind to do away, she found herself occasionally
called on to endure something worse.  She was introduced
here and there by her uncle, and forced to be spoken to,
and to curtsey, and speak again.  This was a hard duty,
and she was never summoned to it without looking at William,
as he walked about at his ease in the background of the scene,
and longing to be with him.

The entrance of the Grants and Crawfords was a favourable epoch.
The stiffness of the meeting soon gave way before their
popular manners and more diffused intimacies:  little groups
were formed, and everybody grew comfortable.  Fanny felt
the advantage; and, drawing back from the toils of civility,
would have been again most happy, could she have kept
her eyes from wandering between Edmund and Mary Crawford.
\emph{She} looked all loveliness---and what might not be
the end of it?  Her own musings were brought to an end
on perceiving Mr.\ Crawford before her, and her thoughts
were put into another channel by his engaging her almost
instantly for the first two dances.  Her happiness on this
occasion was very much \emph{a} \emph{la} \emph{mortal}, finely chequered.
To be secure of a partner at first was a most essential good---%
for the moment of beginning was now growing seriously near;
and she so little understood her own claims as to think
that if Mr.\ Crawford had not asked her, she must have been
the last to be sought after, and should have received
a partner only through a series of inquiry, and bustle,
and interference, which would have been terrible; but at
the same time there was a pointedness in his manner of asking
her which she did not like, and she saw his eye glancing
for a moment at her necklace, with a smile---she thought
there was a smile---which made her blush and feel wretched.
And though there was no second glance to disturb her,
though his object seemed then to be only quietly agreeable,
she could not get the better of her embarrassment,
heightened as it was by the idea of his perceiving it,
and had no composure till he turned away to some one else.
Then she could gradually rise up to the genuine satisfaction
of having a partner, a voluntary partner, secured against
the dancing began.

When the company were moving into the ballroom, she found
herself for the first time near Miss Crawford, whose eyes
and smiles were immediately and more unequivocally directed
as her brother's had been, and who was beginning to speak
on the subject, when Fanny, anxious to get the story over,
hastened to give the explanation of the second necklace:
the real chain.  Miss Crawford listened; and all her intended
compliments and insinuations to Fanny were forgotten:
she felt only one thing; and her eyes, bright as they
had been before, shewing they could yet be brighter,
she exclaimed with eager pleasure, ``Did he?  Did Edmund?
That was like himself.  No other man would have thought of it.
I honour him beyond expression.''  And she looked around
as if longing to tell him so.  He was not near, he was
attending a party of ladies out of the room; and Mrs.\ Grant
coming up to the two girls, and taking an arm of each,
they followed with the rest.

Fanny's heart sunk, but there was no leisure for
thinking long even of Miss Crawford's feelings.
They were in the ballroom, the violins were playing,
and her mind was in a flutter that forbade its fixing on
anything serious.  She must watch the general arrangements,
and see how everything was done.

In a few minutes Sir Thomas came to her, and asked if
she were engaged; and the ``Yes, sir; to Mr.\ Crawford,''
was exactly what he had intended to hear.  Mr.\ Crawford
was not far off; Sir Thomas brought him to her,
saying something which discovered to Fanny, that \emph{she}
was to lead the way and open the ball; an idea that had
never occurred to her before.  Whenever she had thought
of the minutiae of the evening, it had been as a matter
of course that Edmund would begin with Miss Crawford;
and the impression was so strong, that though \emph{her} \emph{uncle}
spoke the contrary, she could not help an exclamation
of surprise, a hint of her unfitness, an entreaty even to
be excused.  To be urging her opinion against Sir Thomas's
was a proof of the extremity of the case; but such was her
horror at the first suggestion, that she could actually
look him in the face and say that she hoped it might be
settled otherwise; in vain, however:  Sir Thomas smiled,
tried to encourage her, and then looked too serious,
and said too decidedly, ``It must be so, my dear,'' for her
to hazard another word; and she found herself the next
moment conducted by Mr.\ Crawford to the top of the room,
and standing there to be joined by the rest of the dancers,
couple after couple, as they were formed.

She could hardly believe it.  To be placed above so many
elegant young women!  The distinction was too great.
It was treating her like her cousins!  And her thoughts
flew to those absent cousins with most unfeigned and truly
tender regret, that they were not at home to take their
own place in the room, and have their share of a pleasure
which would have been so very delightful to them.
So often as she had heard them wish for a ball at home
as the greatest of all felicities!  And to have them away
when it was given---and for \emph{her} to be opening the ball---%
and with Mr.\ Crawford too!  She hoped they would not envy
her that distinction \emph{now}; but when she looked back
to the state of things in the autumn, to what they had all
been to each other when once dancing in that house before,
the present arrangement was almost more than she could
understand herself.

The ball began.  It was rather honour than happiness
to Fanny, for the first dance at least:  her partner was
in excellent spirits, and tried to impart them to her;
but she was a great deal too much frightened to have
any enjoyment till she could suppose herself no longer
looked at.  Young, pretty, and gentle, however, she had
no awkwardnesses that were not as good as graces,
and there were few persons present that were not disposed
to praise her.  She was attractive, she was modest,
she was Sir Thomas's niece, and she was soon said
to be admired by Mr.\ Crawford.  It was enough to give
her general favour.  Sir Thomas himself was watching
her progress down the dance with much complacency;
he was proud of his niece; and without attributing
all her personal beauty, as Mrs.\ Norris seemed to do,
to her transplantation to Mansfield, he was pleased
with himself for having supplied everything else:
education and manners she owed to him.

Miss Crawford saw much of Sir Thomas's thoughts as he stood,
and having, in spite of all his wrongs towards her,
a general prevailing desire of recommending herself to him,
took an opportunity of stepping aside to say something
agreeable of Fanny.  Her praise was warm, and he received
it as she could wish, joining in it as far as discretion,
and politeness, and slowness of speech would allow,
and certainly appearing to greater advantage on the subject
than his lady did soon afterwards, when Mary, perceiving her
on a sofa very near, turned round before she began to dance,
to compliment her on Miss Price's looks.

``Yes, she does look very well,'' was Lady Bertram's placid reply.
``Chapman helped her to dress.  I sent Chapman to her.''
Not but that she was really pleased to have Fanny admired;
but she was so much more struck with her own kindness
in sending Chapman to her, that she could not get it out
of her head.

Miss Crawford knew Mrs.\ Norris too well to think of
gratifying \emph{her} by commendation of Fanny; to her, it was
as the occasion offered---``Ah! ma'am, how much we want dear
Mrs.\ Rushworth and Julia to-night!'' and Mrs.\ Norris paid
her with as many smiles and courteous words as she had
time for, amid so much occupation as she found for herself
in making up card-tables, giving hints to Sir Thomas,
and trying to move all the chaperons to a better part of the room.

Miss Crawford blundered most towards Fanny herself in her
intentions to please.  She meant to be giving her little
heart a happy flutter, and filling her with sensations
of delightful self-consequence; and, misinterpreting Fanny's
blushes, still thought she must be doing so when she
went to her after the two first dances, and said, with a
significant look, ``Perhaps \emph{you} can tell me why my brother
goes to town to-morrow? He says he has business there,
but will not tell me what.  The first time he ever denied
me his confidence!  But this is what we all come to.
All are supplanted sooner or later.  Now, I must apply
to you for information.  Pray, what is Henry going for?''

Fanny protested her ignorance as steadily as her
embarrassment allowed.

``Well, then,'' replied Miss Crawford, laughing, ``I must
suppose it to be purely for the pleasure of conveying
your brother, and of talking of you by the way.''

Fanny was confused, but it was the confusion of discontent;
while Miss Crawford wondered she did not smile, and thought
her over-anxious, or thought her odd, or thought her anything
rather than insensible of pleasure in Henry's attentions.
Fanny had a good deal of enjoyment in the course of the evening;
but Henry's attentions had very little to do with it.
She would much rather \emph{not} have been asked by him again
so very soon, and she wished she had not been obliged
to suspect that his previous inquiries of Mrs.\ Norris,
about the supper hour, were all for the sake of securing her
at that part of the evening.  But it was not to be avoided:
he made her feel that she was the object of all; though she
could not say that it was unpleasantly done, that there
was indelicacy or ostentation in his manner; and sometimes,
when he talked of William, he was really not unagreeable,
and shewed even a warmth of heart which did him credit.
But still his attentions made no part of her satisfaction.
She was happy whenever she looked at William, and saw how
perfectly he was enjoying himself, in every five minutes
that she could walk about with him and hear his account
of his partners; she was happy in knowing herself admired;
and she was happy in having the two dances with Edmund still
to look forward to, during the greatest part of the evening,
her hand being so eagerly sought after that her indefinite
engagement with \emph{him} was in continual perspective.
She was happy even when they did take place; but not from
any flow of spirits on his side, or any such expressions
of tender gallantry as had blessed the morning.
His mind was fagged, and her happiness sprung from
being the friend with whom it could find repose.
``I am worn out with civility,'' said he.  ``I have been
talking incessantly all night, and with nothing to say.
But with \emph{you}, Fanny, there may be peace.  You will not
want to be talked to.  Let us have the luxury of silence.''
Fanny would hardly even speak her agreement.  A weariness,
arising probably, in great measure, from the same feelings
which he had acknowledged in the morning, was peculiarly
to be respected, and they went down their two dances together
with such sober tranquillity as might satisfy any looker-on
that Sir Thomas had been bringing up no wife for his
younger son.

The evening had afforded Edmund little pleasure.  Miss Crawford
had been in gay spirits when they first danced together,
but it was not her gaiety that could do him good:
it rather sank than raised his comfort; and afterwards,
for he found himself still impelled to seek her again,
she had absolutely pained him by her manner of speaking of the
profession to which he was now on the point of belonging.
They had talked, and they had been silent; he had reasoned,
she had ridiculed; and they had parted at last with
mutual vexation.  Fanny, not able to refrain entirely from
observing them, had seen enough to be tolerably satisfied.
It was barbarous to be happy when Edmund was suffering.
Yet some happiness must and would arise from the very
conviction that he did suffer.

When her two dances with him were over, her inclination
and strength for more were pretty well at an end;
and Sir Thomas, having seen her walk rather than dance
down the shortening set, breathless, and with her hand at
her side, gave his orders for her sitting down entirely.
From that time Mr.\ Crawford sat down likewise.

``Poor Fanny!'' cried William, coming for a moment to visit her,
and working away his partner's fan as if for life, ``how soon
she is knocked up!  Why, the sport is but just begun.
I hope we shall keep it up these two hours.  How can you
be tired so soon?''

``So soon! my good friend,'' said Sir Thomas, producing his
watch with all necessary caution; ``it is three o'clock,
and your sister is not used to these sort of hours.''

``Well, then, Fanny, you shall not get up to-morrow before
I go.  Sleep as long as you can, and never mind me.''

``Oh!  William.''

``What!  Did she think of being up before you set off?''

``Oh! yes, sir,'' cried Fanny, rising eagerly from her seat
to be nearer her uncle; ``I must get up and breakfast with him.
It will be the last time, you know; the last morning.''

``You had better not.  He is to have breakfasted and be
gone by half-past nine.  Mr.\ Crawford, I think you call
for him at half-past nine?''

Fanny was too urgent, however, and had too many tears in her
eyes for denial; and it ended in a gracious ``Well, well!''
which was permission.

``Yes, half-past nine,'' said Crawford to William as the
latter was leaving them, ``and I shall be punctual,
for there will be no kind sister to get up for \emph{me}.''
And in a lower tone to Fanny, ``I shall have only a desolate
house to hurry from.  Your brother will find my ideas
of time and his own very different to-morrow.''

After a short consideration, Sir Thomas asked Crawford
to join the early breakfast party in that house
instead of eating alone:  he should himself be of it;
and the readiness with which his invitation was accepted
convinced him that the suspicions whence, he must confess
to himself, this very ball had in great measure sprung,
were well founded.  Mr.\ Crawford was in love with Fanny.
He had a pleasing anticipation of what would be.  His niece,
meanwhile, did not thank him for what he had just done.
She had hoped to have William all to herself the last morning.
It would have been an unspeakable indulgence.  But though
her wishes were overthrown, there was no spirit of murmuring
within her.  On the contrary, she was so totally unused
to have her pleasure consulted, or to have anything take
place at all in the way she could desire, that she was more
disposed to wonder and rejoice in having carried her point
so far, than to repine at the counteraction which followed.

Shortly afterward, Sir Thomas was again interfering
a little with her inclination, by advising her to go
immediately to bed.  ``Advise'' was his word, but it
was the advice of absolute power, and she had only
to rise, and, with Mr.\ Crawford's very cordial adieus,
pass quietly away; stopping at the entrance-door, like
the Lady of Branxholm Hall, ``one moment and no more,''
to view the happy scene, and take a last look at the five
or six determined couple who were still hard at work;
and then, creeping slowly up the principal staircase,
pursued by the ceaseless country-dance, feverish with hopes
and fears, soup and negus, sore-footed and fatigued,
restless and agitated, yet feeling, in spite of everything,
that a ball was indeed delightful.

In thus sending her away, Sir Thomas perhaps might not
be thinking merely of her health.  It might occur to him
that Mr.\ Crawford had been sitting by her long enough,
or he might mean to recommend her as a wife by shewing
her persuadableness.



\chapter{Chapter 29}

\gintro{The ball} was over, and the breakfast was soon over too;
the last kiss was given, and William was gone.
Mr.\ Crawford had, as he foretold, been very punctual,
and short and pleasant had been the meal.

After seeing William to the last moment, Fanny walked
back to the breakfast-room with a very saddened heart
to grieve over the melancholy change; and there her uncle
kindly left her to cry in peace, conceiving, perhaps,
that the deserted chair of each young man might exercise
her tender enthusiasm, and that the remaining cold pork
bones and mustard in William's plate might but divide
her feelings with the broken egg-shells in Mr.\ Crawford's.
She sat and cried \emph{con} \emph{amore} as her uncle intended,
but it was \emph{con} \emph{amore} fraternal and no other.
William was gone, and she now felt as if she had wasted
half his visit in idle cares and selfish solicitudes
unconnected with him.

Fanny's disposition was such that she could never even think
of her aunt Norris in the meagreness and cheerlessness
of her own small house, without reproaching herself
for some little want of attention to her when they had
been last together; much less could her feelings acquit
her of having done and said and thought everything
by William that was due to him for a whole fortnight.

It was a heavy, melancholy day.  Soon after the second
breakfast, Edmund bade them good-bye for a week, and mounted
his horse for Peterborough, and then all were gone.
Nothing remained of last night but remembrances, which she
had nobody to share in.  She talked to her aunt Bertram---%
she must talk to somebody of the ball; but her aunt had seen
so little of what had passed, and had so little curiosity,
that it was heavy work.  Lady Bertram was not certain of
anybody's dress or anybody's place at supper but her own.
``She could not recollect what it was that she had heard
about one of the Miss Maddoxes, or what it was that Lady
Prescott had noticed in Fanny:  she was not sure whether
Colonel Harrison had been talking of Mr.\ Crawford or of
William when he said he was the finest young man in the room---%
somebody had whispered something to her; she had forgot
to ask Sir Thomas what it could be.''  And these were her
longest speeches and clearest communications:  the rest
was only a languid ``Yes, yes; very well; did you? did he?
I did not see \emph{that}; I should not know one from the other.''
This was very bad.  It was only better than Mrs.\ Norris's
sharp answers would have been; but she being gone home
with all the supernumerary jellies to nurse a sick maid,
there was peace and good-humour in their little party,
though it could not boast much beside.

The evening was heavy like the day.  ``I cannot think
what is the matter with me,'' said Lady Bertram,
when the tea-things were removed.  ``I feel quite stupid.
It must be sitting up so late last night.  Fanny, you must
do something to keep me awake.  I cannot work.
Fetch the cards; I feel so very stupid.''

The cards were brought, and Fanny played at cribbage
with her aunt till bedtime; and as Sir Thomas was reading
to himself, no sounds were heard in the room for the next
two hours beyond the reckonings of the game---``And \emph{that}
makes thirty-one; four in hand and eight in crib.
You are to deal, ma'am; shall I deal for you?''  Fanny thought
and thought again of the difference which twenty-four hours
had made in that room, and all that part of the house.
Last night it had been hope and smiles, bustle and motion,
noise and brilliancy, in the drawing-room, and out of
the drawing-room, and everywhere.  Now it was languor,
and all but solitude.

A good night's rest improved her spirits.  She could think
of William the next day more cheerfully; and as the morning
afforded her an opportunity of talking over Thursday night
with Mrs.\ Grant and Miss Crawford, in a very handsome style,
with all the heightenings of imagination, and all the
laughs of playfulness which are so essential to the shade
of a departed ball, she could afterwards bring her mind
without much effort into its everyday state, and easily
conform to the tranquillity of the present quiet week.

They were indeed a smaller party than she had ever
known there for a whole day together, and \emph{he} was gone
on whom the comfort and cheerfulness of every family
meeting and every meal chiefly depended.  But this must
be learned to be endured.  He would soon be always gone;
and she was thankful that she could now sit in the same room
with her uncle, hear his voice, receive his questions,
and even answer them, without such wretched feelings
as she had formerly known.

``We miss our two young men,'' was Sir Thomas's observation
on both the first and second day, as they formed their
very reduced circle after dinner; and in consideration
of Fanny's swimming eyes, nothing more was said
on the first day than to drink their good health;
but on the second it led to something farther.
William was kindly commended and his promotion hoped for.
``And there is no reason to suppose,'' added Sir Thomas,
``but that his visits to us may now be tolerably frequent.
As to Edmund, we must learn to do without him.
This will be the last winter of his belonging to us,
as he has done.''

``Yes,'' said Lady Bertram, ``but I wish he was not going away.
They are all going away, I think.  I wish they would stay
at home.''

This wish was levelled principally at Julia, who had
just applied for permission to go to town with Maria;
and as Sir Thomas thought it best for each daughter that the
permission should be granted, Lady Bertram, though in her own
good-nature she would not have prevented it, was lamenting
the change it made in the prospect of Julia's return,
which would otherwise have taken place about this time.
A great deal of good sense followed on Sir Thomas's side,
tending to reconcile his wife to the arrangement.
Everything that a considerate parent \emph{ought} to feel was
advanced for her use; and everything that an affectionate
mother \emph{must} feel in promoting her children's enjoyment
was attributed to her nature.  Lady Bertram agreed to it
all with a calm ``Yes''; and at the end of a quarter of
an hour's silent consideration spontaneously observed,
``Sir Thomas, I have been thinking---and I am very glad we
took Fanny as we did, for now the others are away we feel
the good of it.''

Sir Thomas immediately improved this compliment by adding,
``Very true.  We shew Fanny what a good girl we think
her by praising her to her face, she is now a very
valuable companion.  If we have been kind to \emph{her},
she is now quite as necessary to \emph{us}.''

``Yes,'' said Lady Bertram presently; ``and it is a comfort
to think that we shall always have \emph{her}.''

Sir Thomas paused, half smiled, glanced at his niece,
and then gravely replied, ``She will never leave us, I hope,
till invited to some other home that may reasonably promise
her greater happiness than she knows here.''

``And \emph{that} is not very likely to be, Sir Thomas.
Who should invite her?  Maria might be very glad to see her
at Sotherton now and then, but she would not think of asking
her to live there; and I am sure she is better off here;
and besides, I cannot do without her.''

The week which passed so quietly and peaceably at the
great house in Mansfield had a very different character at
the Parsonage.  To the young lady, at least, in each family,
it brought very different feelings.  What was tranquillity
and comfort to Fanny was tediousness and vexation to Mary.
Something arose from difference of disposition and habit:
one so easily satisfied, the other so unused to endure;
but still more might be imputed to difference
of circumstances.  In some points of interest they
were exactly opposed to each other.  To Fanny's mind,
Edmund's absence was really, in its cause and its tendency,
a relief.  To Mary it was every way painful.  She felt
the want of his society every day, almost every hour,
and was too much in want of it to derive anything but
irritation from considering the object for which he went.
He could not have devised anything more likely to raise
his consequence than this week's absence, occurring as
it did at the very time of her brother's going away,
of William Price's going too, and completing the sort
of general break-up of a party which had been so animated.
She felt it keenly.  They were now a miserable trio,
confined within doors by a series of rain and snow,
with nothing to do and no variety to hope for.  Angry as
she was with Edmund for adhering to his own notions,
and acting on them in defiance of her (and she had been
so angry that they had hardly parted friends at the ball),
she could not help thinking of him continually when absent,
dwelling on his merit and affection, and longing again
for the almost daily meetings they lately had.  His absence
was unnecessarily long.  He should not have planned such
an absence---he should not have left home for a week,
when her own departure from Mansfield was so near.
Then she began to blame herself.  She wished she had not
spoken so warmly in their last conversation.  She was afraid
she had used some strong, some contemptuous expressions
in speaking of the clergy, and that should not have been.
It was ill-bred; it was wrong.  She wished such words unsaid
with all her heart.

Her vexation did not end with the week.  All this was bad,
but she had still more to feel when Friday came round
again and brought no Edmund; when Saturday came and still
no Edmund; and when, through the slight communication
with the other family which Sunday produced, she learned
that he had actually written home to defer his return,
having promised to remain some days longer with his friend.

If she had felt impatience and regret before---if she had
been sorry for what she said, and feared its too strong
effect on him---she now felt and feared it all tenfold more.
She had, moreover, to contend with one disagreeable emotion
entirely new to her---jealousy.  His friend Mr.\ Owen had sisters;
he might find them attractive.  But, at any rate, his staying
away at a time when, according to all preceding plans,
she was to remove to London, meant something that she could
not bear.  Had Henry returned, as he talked of doing,
at the end of three or four days, she should now have
been leaving Mansfield.  It became absolutely necessary
for her to get to Fanny and try to learn something more.
She could not live any longer in such solitary wretchedness;
and she made her way to the Park, through difficulties
of walking which she had deemed unconquerable a week before,
for the chance of hearing a little in addition, for the
sake of at least hearing his name.

The first half-hour was lost, for Fanny and Lady Bertram
were together, and unless she had Fanny to herself she could
hope for nothing.  But at last Lady Bertram left the room,
and then almost immediately Miss Crawford thus began,
with a voice as well regulated as she could---``And how do
\emph{you} like your cousin Edmund's staying away so long?
Being the only young person at home, I consider \emph{you}
as the greatest sufferer.  You must miss him.  Does his
staying longer surprise you?''

``I do not know,'' said Fanny hesitatingly.  ``Yes; I had
not particularly expected it.''

``Perhaps he will always stay longer than he talks of.
It is the general way all young men do.''

``He did not, the only time he went to see Mr.\ Owen before.''

``He finds the house more agreeable \emph{now}. He is a very---%
a very pleasing young man himself, and I cannot help
being rather concerned at not seeing him again before I
go to London, as will now undoubtedly be the case.
I am looking for Henry every day, and as soon as he
comes there will be nothing to detain me at Mansfield.
I should like to have seen him once more, I confess.
But you must give my compliments to him.  Yes; I think it must
be compliments.  Is not there a something wanted, Miss Price,
in our language---a something between compliments and---%
and love---to suit the sort of friendly acquaintance we have
had together?  So many months' acquaintance!  But compliments
may be sufficient here.  Was his letter a long one?
Does he give you much account of what he is doing?
Is it Christmas gaieties that he is staying for?''

``I only heard a part of the letter; it was to my uncle;
but I believe it was very short; indeed I am sure it was
but a few lines.  All that I heard was that his friend
had pressed him to stay longer, and that he had agreed
to do so.  A \emph{few} days longer, or \emph{some} days longer;
I am not quite sure which.''

``Oh! if he wrote to his father; but I thought it might
have been to Lady Bertram or you.  But if he wrote to
his father, no wonder he was concise.  Who could write
chat to Sir Thomas?  If he had written to you, there would
have been more particulars.  You would have heard of
balls and parties.  He would have sent you a description
of everything and everybody.  How many Miss Owens are there?''

``Three grown up.''

``Are they musical?''

``I do not at all know.  I never heard.''

``That is the first question, you know,'' said Miss Crawford,
trying to appear gay and unconcerned, ``which every
woman who plays herself is sure to ask about another.
But it is very foolish to ask questions about any
young ladies---about any three sisters just grown up;
for one knows, without being told, exactly what they are:
all very accomplished and pleasing, and one very pretty.
There is a beauty in every family; it is a regular thing.
Two play on the pianoforte, and one on the harp;
and all sing, or would sing if they were taught,
or sing all the better for not being taught; or something
like it.''

``I know nothing of the Miss Owens,'' said Fanny calmly.

``You know nothing and you care less, as people say.
Never did tone express indifference plainer.  Indeed, how can
one care for those one has never seen?  Well, when your
cousin comes back, he will find Mansfield very quiet;
all the noisy ones gone, your brother and mine and myself
I do not like the idea of leaving Mrs.\ Grant now the time
draws near.  She does not like my going.''

Fanny felt obliged to speak.  ``You cannot doubt your being
missed by many,'' said she.  ``You will be very much missed.''

Miss Crawford turned her eye on her, as if wanting to hear
or see more, and then laughingly said, ``Oh yes! missed
as every noisy evil is missed when it is taken away;
that is, there is a great difference felt.  But I am
not fishing; don't compliment me.  If I \emph{am} missed,
it will appear.  I may be discovered by those who want
to see me.  I shall not be in any doubtful, or distant,
or unapproachable region.''

Now Fanny could not bring herself to speak, and Miss
Crawford was disappointed; for she had hoped to hear
some pleasant assurance of her power from one who she
thought must know, and her spirits were clouded again.

``The Miss Owens,'' said she, soon afterwards; ``suppose you
were to have one of the Miss Owens settled at Thornton Lacey;
how should you like it?  Stranger things have happened.
I dare say they are trying for it.  And they are quite
in the light, for it would be a very pretty establishment
for them.  I do not at all wonder or blame them.  It is
everybody's duty to do as well for themselves as they can.
Sir Thomas Bertram's son is somebody; and now he is in their
own line.  Their father is a clergyman, and their brother
is a clergyman, and they are all clergymen together.
He is their lawful property; he fairly belongs to them.
You don't speak, Fanny; Miss Price, you don't speak.
But honestly now, do not you rather expect it than otherwise?''

``No,'' said Fanny stoutly, ``I do not expect it at all.''

``Not at all!'' cried Miss Crawford with alacrity.
``I wonder at that.  But I dare say you know exactly---%
I always imagine you are---perhaps you do not think him
likely to marry at all---or not at present.''

``No, I do not,'' said Fanny softly, hoping she did not err
either in the belief or the acknowledgment of it.

Her companion looked at her keenly; and gathering greater
spirit from the blush soon produced from such a look,
only said, ``He is best off as he is,'' and turned the subject.



\chapter{Chapter 30}

\gintro{Miss Crawford's} uneasiness was much lightened by
this conversation, and she walked home again in spirits
which might have defied almost another week of the same
small party in the same bad weather, had they been put
to the proof; but as that very evening brought her brother
down from London again in quite, or more than quite,
his usual cheerfulness, she had nothing farther to try
her own.  His still refusing to tell her what he had gone
for was but the promotion of gaiety; a day before it
might have irritated, but now it was a pleasant joke---%
suspected only of concealing something planned as a pleasant
surprise to herself.  And the next day \emph{did} bring a
surprise to her.  Henry had said he should just go and ask
the Bertrams how they did, and be back in ten minutes,
but he was gone above an hour; and when his sister,
who had been waiting for him to walk with her in the garden,
met him at last most impatiently in the sweep, and cried out,
``My dear Henry, where can you have been all this time?''
he had only to say that he had been sitting with Lady
Bertram and Fanny.

``Sitting with them an hour and a half!'' exclaimed Mary.

But this was only the beginning of her surprise.

``Yes, Mary,'' said he, drawing her arm within his,
and walking along the sweep as if not knowing where he was:
``I could not get away sooner; Fanny looked so lovely!
I am quite determined, Mary.  My mind is entirely made up.
Will it astonish you?  No: you must be aware that I am quite
determined to marry Fanny Price.''

The surprise was now complete; for, in spite of whatever
his consciousness might suggest, a suspicion of his having
any such views had never entered his sister's imagination;
and she looked so truly the astonishment she felt, that he
was obliged to repeat what he had said, and more fully
and more solemnly.  The conviction of his determination
once admitted, it was not unwelcome.  There was even
pleasure with the surprise.  Mary was in a state of mind
to rejoice in a connexion with the Bertram family,
and to be not displeased with her brother's marrying
a little beneath him.

``Yes, Mary,'' was Henry's concluding assurance.  ``I am
fairly caught.  You know with what idle designs I began;
but this is the end of them.  I have, I flatter myself,
made no inconsiderable progress in her affections;
but my own are entirely fixed.''

``Lucky, lucky girl!'' cried Mary, as soon as she could speak;
``what a match for her!  My dearest Henry, this must
be my \emph{first} feeling; but my \emph{second}, which you shall
have as sincerely, is, that I approve your choice from
my soul, and foresee your happiness as heartily as I
wish and desire it.  You will have a sweet little wife;
all gratitude and devotion.  Exactly what you deserve.
What an amazing match for her!  Mrs.\ Norris often talks
of her luck; what will she say now?  The delight of all
the family, indeed!  And she has some \emph{true} friends in it!
How \emph{they} will rejoice!  But tell me all about it!
Talk to me for ever.  When did you begin to think seriously
about her?''

Nothing could be more impossible than to answer such
a question, though nothing could be more agreeable than
to have it asked.  ``How the pleasing plague had stolen
on him'' he could not say; and before he had expressed
the same sentiment with a little variation of words
three times over, his sister eagerly interrupted him with,
``Ah, my dear Henry, and this is what took you to London!
This was your business!  You chose to consult the Admiral
before you made up your mind.''

But this he stoutly denied.  He knew his uncle too well
to consult him on any matrimonial scheme.  The Admiral
hated marriage, and thought it never pardonable in a young
man of independent fortune.

``When Fanny is known to him,'' continued Henry, ``he will doat
on her.  She is exactly the woman to do away every prejudice
of such a man as the Admiral, for she he would describe,
if indeed he has now delicacy of language enough to embody
his own ideas.  But till it is absolutely settled---%
settled beyond all interference, he shall know nothing
of the matter.  No, Mary, you are quite mistaken.
You have not discovered my business yet.''

``Well, well, I am satisfied.  I know now to whom
it must relate, and am in no hurry for the rest.
Fanny Price! wonderful, quite wonderful!  That Mansfield
should have done so much for---that \emph{you} should have
found your fate in Mansfield!  But you are quite right;
you could not have chosen better.  There is not a better
girl in the world, and you do not want for fortune;
and as to her connexions, they are more than good.
The Bertrams are undoubtedly some of the first people
in this country.  She is niece to Sir Thomas Bertram;
that will be enough for the world.  But go on, go on.
Tell me more.  What are your plans?  Does she know her
own happiness?''

``No.''

``What are you waiting for?''

``For---for very little more than opportunity.  Mary, she is
not like her cousins; but I think I shall not ask in vain.''

``Oh no! you cannot.  Were you even less pleasing---%
supposing her not to love you already (of which,
however, I can have little doubt)---you would be safe.
The gentleness and gratitude of her disposition would
secure her all your own immediately.  From my soul I do
not think she would marry you \emph{without} love; that is,
if there is a girl in the world capable of being uninfluenced
by ambition, I can suppose it her; but ask her to love you,
and she will never have the heart to refuse.''

As soon as her eagerness could rest in silence,
he was as happy to tell as she could be to listen;
and a conversation followed almost as deeply interesting
to her as to himself, though he had in fact nothing
to relate but his own sensations, nothing to dwell on
but Fanny's charms.  Fanny's beauty of face and figure,
Fanny's graces of manner and goodness of heart, were the
exhaustless theme.  The gentleness, modesty, and sweetness
of her character were warmly expatiated on; that sweetness
which makes so essential a part of every woman's worth
in the judgment of man, that though he sometimes loves
where it is not, he can never believe it absent.
Her temper he had good reason to depend on and to praise.
He had often seen it tried.  Was there one of the family,
excepting Edmund, who had not in some way or other
continually exercised her patience and forbearance?
Her affections were evidently strong.  To see her with
her brother!  What could more delightfully prove that
the warmth of her heart was equal to its gentleness?
What could be more encouraging to a man who had her love
in view?  Then, her understanding was beyond every suspicion,
quick and clear; and her manners were the mirror of
her own modest and elegant mind.  Nor was this all.
Henry Crawford had too much sense not to feel the worth of good
principles in a wife, though he was too little accustomed
to serious reflection to know them by their proper name;
but when he talked of her having such a steadiness
and regularity of conduct, such a high notion of honour,
and such an observance of decorum as might warrant any man
in the fullest dependence on her faith and integrity,
he expressed what was inspired by the knowledge of her
being well principled and religious.

``I could so wholly and absolutely confide in her,'' said he;
``and \emph{that} is what I want.''

Well might his sister, believing as she really did that his
opinion of Fanny Price was scarcely beyond her merits,
rejoice in her prospects.

``The more I think of it,'' she cried, ``the more am I convinced
that you are doing quite right; and though I should never have
selected Fanny Price as the girl most likely to attach you,
I am now persuaded she is the very one to make you happy.
Your wicked project upon her peace turns out a clever
thought indeed.  You will both find your good in it.''

``It was bad, very bad in me against such a creature;
but I did not know her then; and she shall have no reason
to lament the hour that first put it into my head.
I will make her very happy, Mary; happier than she has ever
yet been herself, or ever seen anybody else.  I will not
take her from Northamptonshire.  I shall let Everingham,
and rent a place in this neighbourhood; perhaps Stanwix Lodge.
I shall let a seven years' lease of Everingham.
I am sure of an excellent tenant at half a word.
I could name three people now, who would give me my own
terms and thank me.''

``Ha!'' cried Mary; ``settle in Northamptonshire!
That is pleasant!  Then we shall be all together.''

When she had spoken it, she recollected herself,
and wished it unsaid; but there was no need of confusion;
for her brother saw her only as the supposed inmate
of Mansfield parsonage, and replied but to invite her
in the kindest manner to his own house, and to claim
the best right in her.

``You must give us more than half your time,'' said he.
``I cannot admit Mrs.\ Grant to have an equal claim with
Fanny and myself, for we shall both have a right in you.
Fanny will be so truly your sister!''

Mary had only to be grateful and give general assurances;
but she was now very fully purposed to be the guest of
neither brother nor sister many months longer.

``You will divide your year between London and Northamptonshire?''

``Yes.''

``That's right; and in London, of course, a house of
your own:  no longer with the Admiral.  My dearest Henry,
the advantage to you of getting away from the Admiral
before your manners are hurt by the contagion of his,
before you have contracted any of his foolish opinions,
or learned to sit over your dinner as if it were the best
blessing of life!  \emph{You} are not sensible of the gain,
for your regard for him has blinded you; but, in my estimation,
your marrying early may be the saving of you.  To have seen
you grow like the Admiral in word or deed, look or gesture,
would have broken my heart.''

``Well, well, we do not think quite alike here.
The Admiral has his faults, but he is a very good man,
and has been more than a father to me.  Few fathers would
have let me have my own way half so much.  You must
not prejudice Fanny against him.  I must have them love
one another.''

Mary refrained from saying what she felt, that there could
not be two persons in existence whose characters and manners
were less accordant:  time would discover it to him;
but she could not help \emph{this} reflection on the Admiral.
``Henry, I think so highly of Fanny Price, that if I could
suppose the next Mrs.\ Crawford would have half the reason
which my poor ill-used aunt had to abhor the very name,
I would prevent the marriage, if possible; but I know you:
I know that a wife you \emph{loved} would be the happiest
of women, and that even when you ceased to love, she would
yet find in you the liberality and good-breeding of
a gentleman.''

The impossibility of not doing everything in the world to
make Fanny Price happy, or of ceasing to love Fanny Price,
was of course the groundwork of his eloquent answer.

``Had you seen her this morning, Mary,'' he continued,
``attending with such ineffable sweetness and patience to
all the demands of her aunt's stupidity, working with her,
and for her, her colour beautifully heightened as she
leant over the work, then returning to her seat to finish
a note which she was previously engaged in writing
for that stupid woman's service, and all this with such
unpretending gentleness, so much as if it were a matter
of course that she was not to have a moment at her
own command, her hair arranged as neatly as it always is,
and one little curl falling forward as she wrote, which she
now and then shook back, and in the midst of all this,
still speaking at intervals to \emph{me}, or listening,
and as if she liked to listen, to what I said.
Had you seen her so, Mary, you would not have implied
the possibility of her power over my heart ever ceasing.''

``My dearest Henry,'' cried Mary, stopping short, and smiling
in his face, ``how glad I am to see you so much in love!
It quite delights me.  But what will Mrs.\ Rushworth and
Julia say?''

``I care neither what they say nor what they feel.
They will now see what sort of woman it is that can attach me,
that can attach a man of sense.  I wish the discovery
may do them any good.  And they will now see their cousin
treated as she ought to be, and I wish they may be heartily
ashamed of their own abominable neglect and unkindness.
They will be angry,'' he added, after a moment's silence,
and in a cooler tone; ``Mrs.\ Rushworth will be very angry.
It will be a bitter pill to her; that is, like other
bitter pills, it will have two moments' ill flavour, and then
be swallowed and forgotten; for I am not such a coxcomb
as to suppose her feelings more lasting than other women's,
though \emph{I} was the object of them.  Yes, Mary, my Fanny
will feel a difference indeed:  a daily, hourly difference,
in the behaviour of every being who approaches her;
and it will be the completion of my happiness to know
that I am the doer of it, that I am the person to give
the consequence so justly her due.  Now she is dependent,
helpless, friendless, neglected, forgotten.''

``Nay, Henry, not by all; not forgotten by all; not friendless
or forgotten.  Her cousin Edmund never forgets her.''

``Edmund!  True, I believe he is, generally speaking,
kind to her, and so is Sir Thomas in his way; but it is
the way of a rich, superior, long-worded, arbitrary uncle.
What can Sir Thomas and Edmund together do, what do they
\emph{do} for her happiness, comfort, honour, and dignity in
the world, to what I \emph{shall} do?''



\chapter{Chapter 31}

\gintro{Henry Crawford} was at Mansfield Park again the next morning,
and at an earlier hour than common visiting warrants.
The two ladies were together in the breakfast-room, and,
fortunately for him, Lady Bertram was on the very point
of quitting it as he entered.  She was almost at the door,
and not chusing by any means to take so much trouble in vain,
she still went on, after a civil reception, a short sentence
about being waited for, and a ``Let Sir Thomas know''
to the servant.

Henry, overjoyed to have her go, bowed and watched her off,
and without losing another moment, turned instantly to Fanny,
and, taking out some letters, said, with a most animated look,
``I must acknowledge myself infinitely obliged to any creature
who gives me such an opportunity of seeing you alone:
I have been wishing it more than you can have any idea.
Knowing as I do what your feelings as a sister are, I could
hardly have borne that any one in the house should share
with you in the first knowledge of the news I now bring.
He is made.  Your brother is a lieutenant.  I have
the infinite satisfaction of congratulating you on your
brother's promotion.  Here are the letters which announce it,
this moment come to hand.  You will, perhaps, like to see them.''

Fanny could not speak, but he did not want her to speak.
To see the expression of her eyes, the change
of her complexion, the progress of her feelings,
their doubt, confusion, and felicity, was enough.
She took the letters as he gave them.  The first was
from the Admiral to inform his nephew, in a few words,
of his having succeeded in the object he had undertaken,
the promotion of young Price, and enclosing two more,
one from the Secretary of the First Lord to a friend,
whom the Admiral had set to work in the business,
the other from that friend to himself, by which it
appeared that his lordship had the very great happiness
of attending to the recommendation of Sir Charles;
that Sir Charles was much delighted in having such an
opportunity of proving his regard for Admiral Crawford,
and that the circumstance of Mr.\ William Price's commission
as Second Lieutenant of H.M. Sloop Thrush being made
out was spreading general joy through a wide circle
of great people.

While her hand was trembling under these letters,
her eye running from one to the other, and her heart
swelling with emotion, Crawford thus continued,
with unfeigned eagerness, to express his interest in the event---%

``I will not talk of my own happiness,'' said he, ``great as
it is, for I think only of yours.  Compared with you,
who has a right to be happy?  I have almost grudged myself
my own prior knowledge of what you ought to have known
before all the world.  I have not lost a moment, however.
The post was late this morning, but there has not been
since a moment's delay.  How impatient, how anxious,
how wild I have been on the subject, I will not attempt
to describe; how severely mortified, how cruelly disappointed,
in not having it finished while I was in London!
I was kept there from day to day in the hope of it,
for nothing less dear to me than such an object would
have detained me half the time from Mansfield.
But though my uncle entered into my wishes with all the
warmth I could desire, and exerted himself immediately,
there were difficulties from the absence of one friend,
and the engagements of another, which at last I could no longer
bear to stay the end of, and knowing in what good hands I
left the cause, I came away on Monday, trusting that many
posts would not pass before I should be followed by such
very letters as these.  My uncle, who is the very best man
in the world, has exerted himself, as I knew he would,
after seeing your brother.  He was delighted with him.
I would not allow myself yesterday to say how delighted,
or to repeat half that the Admiral said in his praise.
I deferred it all till his praise should be proved
the praise of a friend, as this day \emph{does} prove it.
\emph{Now} I may say that even I could not require William
Price to excite a greater interest, or be followed
by warmer wishes and higher commendation, than were most
voluntarily bestowed by my uncle after the evening they had
passed together.''

``Has this been all \emph{your} doing, then?'' cried Fanny.
``Good heaven! how very, very kind!  Have you really---%
was it by \emph{your} desire?  I beg your pardon, but I
am bewildered.  Did Admiral Crawford apply?  How was it?
I am stupefied.''

Henry was most happy to make it more intelligible,
by beginning at an earlier stage, and explaining very
particularly what he had done.  His last journey to London
had been undertaken with no other view than that of
introducing her brother in Hill Street, and prevailing
on the Admiral to exert whatever interest he might
have for getting him on.  This had been his business.
He had communicated it to no creature:  he had not
breathed a syllable of it even to Mary; while uncertain
of the issue, he could not have borne any participation
of his feelings, but this had been his business; and he
spoke with such a glow of what his solicitude had been,
and used such strong expressions, was so abounding
in the \emph{deepest} \emph{interest}, in \emph{twofold} \emph{motives},
in \emph{views} \emph{and} \emph{wishes} \emph{more} \emph{than} \emph{could} \emph{be} \emph{told},
that Fanny could not have remained insensible of his drift,
had she been able to attend; but her heart was so full
and her senses still so astonished, that she could listen
but imperfectly even to what he told her of William,
and saying only when he paused, ``How kind! how very kind!
Oh, Mr.\ Crawford, we are infinitely obliged to you!
Dearest, dearest William!''  She jumped up and moved in haste
towards the door, crying out, ``I will go to my uncle.
My uncle ought to know it as soon as possible.''  But this
could not be suffered.  The opportunity was too fair,
and his feelings too impatient.  He was after her immediately.
``She must not go, she must allow him five minutes longer,''
and he took her hand and led her back to her seat,
and was in the middle of his farther explanation,
before she had suspected for what she was detained.
When she did understand it, however, and found herself
expected to believe that she had created sensations which
his heart had never known before, and that everything
he had done for William was to be placed to the account
of his excessive and unequalled attachment to her,
she was exceedingly distressed, and for some moments
unable to speak.  She considered it all as nonsense,
as mere trifling and gallantry, which meant only to deceive
for the hour; she could not but feel that it was treating
her improperly and unworthily, and in such a way as she
had not deserved; but it was like himself, and entirely
of a piece with what she had seen before; and she would
not allow herself to shew half the displeasure she felt,
because he had been conferring an obligation, which no
want of delicacy on his part could make a trifle to her.
While her heart was still bounding with joy and gratitude
on William's behalf, she could not be severely resentful
of anything that injured only herself; and after having
twice drawn back her hand, and twice attempted in vain
to turn away from him, she got up, and said only,
with much agitation, ``Don't, Mr.\ Crawford, pray don't! I
beg you would not.  This is a sort of talking which is very
unpleasant to me.  I must go away.  I cannot bear it.''
But he was still talking on, describing his affection,
soliciting a return, and, finally, in words so plain
as to bear but one meaning even to her, offering himself,
hand, fortune, everything, to her acceptance.  It was so;
he had said it.  Her astonishment and confusion increased;
and though still not knowing how to suppose him serious,
she could hardly stand.  He pressed for an answer.

``No, no, no!'' she cried, hiding her face.  ``This is all nonsense.
Do not distress me.  I can hear no more of this.
Your kindness to William makes me more obliged to you
than words can express; but I do not want, I cannot bear,
I must not listen to such---No, no, don't think of me.
But you are \emph{not} thinking of me.  I know it is all nothing.''

She had burst away from him, and at that moment Sir Thomas
was heard speaking to a servant in his way towards the room
they were in.  It was no time for farther assurances
or entreaty, though to part with her at a moment when her
modesty alone seemed, to his sanguine and preassured mind,
to stand in the way of the happiness he sought, was a
cruel necessity.  She rushed out at an opposite door
from the one her uncle was approaching, and was walking
up and down the East room ill the utmost confusion
of contrary feeling, before Sir Thomas's politeness
or apologies were over, or he had reached the beginning
of the joyful intelligence which his visitor came to communicate.

She was feeling, thinking, trembling about everything;
agitated, happy, miserable, infinitely obliged,
absolutely angry.  It was all beyond belief!
He was inexcusable, incomprehensible!  But such were
his habits that he could do nothing without a mixture
of evil.  He had previously made her the happiest
of human beings, and now he had insulted---she knew
not what to say, how to class, or how to regard it.
She would not have him be serious, and yet what could
excuse the use of such words and offers, if they meant but to
trifle?

But William was a lieutenant.  \emph{That} was a fact beyond
a doubt, and without an alloy.  She would think of it
for ever and forget all the rest.  Mr.\ Crawford would
certainly never address her so again:  he must have
seen how unwelcome it was to her; and in that case,
how gratefully she could esteem him for his friendship
to William!

She would not stir farther from the East room than
the head of the great staircase, till she had satisfied
herself of Mr.\ Crawford's having left the house;
but when convinced of his being gone, she was eager to go
down and be with her uncle, and have all the happiness
of his joy as well as her own, and all the benefit
of his information or his conjectures as to what would
now be William's destination.  Sir Thomas was as joyful
as she could desire, and very kind and communicative;
and she had so comfortable a talk with him about William
as to make her feel as if nothing had occurred to vex her,
till she found, towards the close, that Mr.\ Crawford
was engaged to return and dine there that very day.
This was a most unwelcome hearing, for though he might
think nothing of what had passed, it would be quite
distressing to her to see him again so soon.

She tried to get the better of it; tried very hard,
as the dinner hour approached, to feel and appear as usual;
but it was quite impossible for her not to look most shy
and uncomfortable when their visitor entered the room.
She could not have supposed it in the power of any concurrence
of circumstances to give her so many painful sensations on
the first day of hearing of William's promotion.

Mr.\ Crawford was not only in the room---he was soon close
to her.  He had a note to deliver from his sister.
Fanny could not look at him, but there was no consciousness
of past folly in his voice.  She opened her note immediately,
glad to have anything to do, and happy, as she read it,
to feel that the fidgetings of her aunt Norris, who was
also to dine there, screened her a little from view.

``My dear Fanny,---for so I may now always call you,
to the infinite relief of a tongue that has been stumbling
at \emph{Miss} \emph{Price} for at least the last six weeks---%
I cannot let my brother go without sending you a few lines
of general congratulation, and giving my most joyful consent
and approval.  Go on, my dear Fanny, and without fear;
there can be no difficulties worth naming.  I chuse to
suppose that the assurance of my consent will be something;
so you may smile upon him with your sweetest smiles
this afternoon, and send him back to me even happier
than he goes.---Yours affectionately, M. C.''

These were not expressions to do Fanny any good;
for though she read in too much haste and confusion
to form the clearest judgment of Miss Crawford's meaning,
it was evident that she meant to compliment her on her
brother's attachment, and even to \emph{appear} to believe
it serious.  She did not know what to do, or what to think.
There was wretchedness in the idea of its being serious;
there was perplexity and agitation every way.
She was distressed whenever Mr.\ Crawford spoke to her,
and he spoke to her much too often; and she was afraid
there was a something in his voice and manner in addressing
her very different from what they were when he talked
to the others.  Her comfort in that day's dinner
was quite destroyed:  she could hardly eat anything;
and when Sir Thomas good-humouredly observed that joy had
taken away her appetite, she was ready to sink with shame,
from the dread of Mr.\ Crawford's interpretation;
for though nothing could have tempted her to turn her eyes
to the right hand, where he sat, she felt that \emph{his}
were immediately directed towards her.

She was more silent than ever.  She would hardly join
even when William was the subject, for his commission
came all from the right hand too, and there was pain
in the connexion.

She thought Lady Bertram sat longer than ever, and began
to be in despair of ever getting away; but at last they
were in the drawing-room, and she was able to think
as she would, while her aunts finished the subject
of William's appointment in their own style.

Mrs.\ Norris seemed as much delighted with the saving
it would be to Sir Thomas as with any part of it.
``\emph{Now} William would be able to keep himself, which would
make a vast difference to his uncle, for it was unknown
how much he had cost his uncle; and, indeed, it would make
some difference in \emph{her} presents too.  She was very glad
that she had given William what she did at parting,
very glad, indeed, that it had been in her power,
without material inconvenience, just at that time to give
him something rather considerable; that is, for \emph{her},
with \emph{her} limited means, for now it would all be useful
in helping to fit up his cabin.  She knew he must be at
some expense, that he would have many things to buy,
though to be sure his father and mother would be able
to put him in the way of getting everything very cheap;
but she was very glad she had contributed her mite
towards it.''

``I am glad you gave him something considerable,''
said Lady Bertram, with most unsuspicious calmness,
``for \emph{I} gave him only 10.''

``Indeed!'' cried Mrs.\ Norris, reddening.  ``Upon my word,
he must have gone off with his pockets well lined,
and at no expense for his journey to London either!''

``Sir Thomas told me 10 would be enough.''

Mrs.\ Norris, being not at all inclined to question
its sufficiency, began to take the matter in another point.

``It is amazing,'' said she, ``how much young people cost
their friends, what with bringing them up and putting them
out in the world!  They little think how much it comes to,
or what their parents, or their uncles and aunts, pay for
them in the course of the year.  Now, here are my sister
Price's children; take them all together, I dare say nobody
would believe what a sum they cost Sir Thomas every year,
to say nothing of what \emph{I} do for them.''

``Very true, sister, as you say.  But, poor things!
they cannot help it; and you know it makes very little
difference to Sir Thomas.  Fanny, William must not forget
my shawl if he goes to the East Indies; and I shall give
him a commission for anything else that is worth having.
I wish he may go to the East Indies, that I may have my shawl.
I think I will have two shawls, Fanny.''

Fanny, meanwhile, speaking only when she could not help it,
was very earnestly trying to understand what Mr.\ and Miss
Crawford were at.  There was everything in the world
\emph{against} their being serious but his words and manner.
Everything natural, probable, reasonable, was against it;
all their habits and ways of thinking, and all
her own demerits.  How could \emph{she} have excited
serious attachment in a man who had seen so many,
and been admired by so many, and flirted with so many,
infinitely her superiors; who seemed so little open
to serious impressions, even where pains had been taken
to please him; who thought so slightly, so carelessly,
so unfeelingly on all such points; who was everything
to everybody, and seemed to find no one essential to him?
And farther, how could it be supposed that his sister,
with all her high and worldly notions of matrimony,
would be forwarding anything of a serious nature in such
a quarter?  Nothing could be more unnatural in either.
Fanny was ashamed of her own doubts.  Everything might
be possible rather than serious attachment, or serious
approbation of it toward her.  She had quite convinced herself
of this before Sir Thomas and Mr.\ Crawford joined them.
The difficulty was in maintaining the conviction quite
so absolutely after Mr.\ Crawford was in the room;
for once or twice a look seemed forced on her which she
did not know how to class among the common meaning;
in any other man, at least, she would have said
that it meant something very earnest, very pointed.
But she still tried to believe it no more than what he
might often have expressed towards her cousins and fifty
other women.

She thought he was wishing to speak to her unheard
by the rest.  She fancied he was trying for it the
whole evening at intervals, whenever Sir Thomas was
out of the room, or at all engaged with Mrs.\ Norris,
and she carefully refused him every opportunity.

At last---it seemed an at last to Fanny's nervousness,
though not remarkably late---he began to talk of going away;
but the comfort of the sound was impaired by his turning
to her the next moment, and saying, ``Have you nothing to send
to Mary?  No answer to her note?  She will be disappointed
if she receives nothing from you.  Pray write to her,
if it be only a line.''

``Oh yes! certainly,'' cried Fanny, rising in haste,
the haste of embarrassment and of wanting to get away---%
``I will write directly.''

She went accordingly to the table, where she was in the
habit of writing for her aunt, and prepared her materials
without knowing what in the world to say.  She had read
Miss Crawford's note only once, and how to reply to
anything so imperfectly understood was most distressing.
Quite unpractised in such sort of note-writing, had
there been time for scruples and fears as to style she
would have felt them in abundance:  but something must
be instantly written; and with only one decided feeling,
that of wishing not to appear to think anything really intended,
she wrote thus, in great trembling both of spirits and hand---%

``I am very much obliged to you, my dear Miss Crawford,
for your kind congratulations, as far as they relate to my
dearest William.  The rest of your note I know means nothing;
but I am so unequal to anything of the sort, that I hope
you will excuse my begging you to take no farther notice.
I have seen too much of Mr.\ Crawford not to understand
his manners; if he understood me as well, he would,
I dare say, behave differently.  I do not know what I write,
but it would be a great favour of you never to mention
the subject again.  With thanks for the honour of your note,
I remain, dear Miss Crawford, etc., etc.''

The conclusion was scarcely intelligible from increasing
fright, for she found that Mr.\ Crawford, under pretence
of receiving the note, was coming towards her.

``You cannot think I mean to hurry you,'' said he,
in an undervoice, perceiving the amazing trepidation
with which she made up the note, ``you cannot think
I have any such object.  Do not hurry yourself, I entreat.''

``Oh!  I thank you; I have quite done, just done; it will
be ready in a moment; I am very much obliged to you;
if you will be so good as to give \emph{that} to Miss Crawford.''

The note was held out, and must be taken; and as she
instantly and with averted eyes walked towards the fireplace,
where sat the others, he had nothing to do but to go
in good earnest.

Fanny thought she had never known a day of greater agitation,
both of pain and pleasure; but happily the pleasure
was not of a sort to die with the day; for every day
would restore the knowledge of William's advancement,
whereas the pain, she hoped, would return no more.
She had no doubt that her note must appear excessively
ill-written, that the language would disgrace a child,
for her distress had allowed no arrangement; but at least
it would assure them both of her being neither imposed
on nor gratified by Mr.\ Crawford's attentions.



\chapter{Chapter 32}

\gintro{Fanny} had by no means forgotten Mr.\ Crawford when she
awoke the next morning; but she remembered the purport
of her note, and was not less sanguine as to its effect
than she had been the night before.  If Mr.\ Crawford would
but go away!  That was what she most earnestly desired:
go and take his sister with him, as he was to do,
and as he returned to Mansfield on purpose to do.
And why it was not done already she could not devise,
for Miss Crawford certainly wanted no delay.  Fanny had hoped,
in the course of his yesterday's visit, to hear the day named;
but he had only spoken of their journey as what would take
place ere long.

Having so satisfactorily settled the conviction her note
would convey, she could not but be astonished to see
Mr.\ Crawford, as she accidentally did, coming up to the
house again, and at an hour as early as the day before.
His coming might have nothing to do with her, but she
must avoid seeing him if possible; and being then
on her way upstairs, she resolved there to remain,
during the whole of his visit, unless actually sent for;
and as Mrs.\ Norris was still in the house, there seemed
little danger of her being wanted.

She sat some time in a good deal of agitation, listening,
trembling, and fearing to be sent for every moment;
but as no footsteps approached the East room, she grew
gradually composed, could sit down, and be able to
employ herself, and able to hope that Mr.\ Crawford had come
and would go without her being obliged to know anything of the
matter.

Nearly half an hour had passed, and she was growing
very comfortable, when suddenly the sound of a step
in regular approach was heard; a heavy step, an unusual
step in that part of the house:  it was her uncle's;
she knew it as well as his voice; she had trembled at it
as often, and began to tremble again, at the idea of his
coming up to speak to her, whatever might be the subject.
It was indeed Sir Thomas who opened the door and asked
if she were there, and if he might come in.  The terror
of his former occasional visits to that room seemed
all renewed, and she felt as if he were going to examine
her again in French and English.

She was all attention, however, in placing a chair for him,
and trying to appear honoured; and, in her agitation,
had quite overlooked the deficiencies of her apartment, till he,
stopping short as he entered, said, with much surprise,
``Why have you no fire to-day?''

There was snow on the ground, and she was sitting in a shawl.
She hesitated.

``I am not cold, sir:  I never sit here long at this time
of year.''

``But you have a fire in general?''

``No, sir.''

``How comes this about?  Here must be some mistake.
I understood that you had the use of this room by way
of making you perfectly comfortable.  In your bedchamber
I know you \emph{cannot} have a fire.  Here is some great
misapprehension which must be rectified.  It is highly
unfit for you to sit, be it only half an hour a day,
without a fire.  You are not strong.  You are chilly.
Your aunt cannot be aware of this.''

Fanny would rather have been silent; but being obliged
to speak, she could not forbear, in justice to the aunt
she loved best, from saying something in which the words
``my aunt Norris'' were distinguishable.

``I understand,'' cried her uncle, recollecting himself,
and not wanting to hear more:  ``I understand.  Your aunt
Norris has always been an advocate, and very judiciously,
for young people's being brought up without unnecessary
indulgences; but there should be moderation in everything.
She is also very hardy herself, which of course will
influence her in her opinion of the wants of others.
And on another account, too, I can perfectly comprehend.
I know what her sentiments have always been.
The principle was good in itself, but it may have been,
and I believe \emph{has} \emph{been}, carried too far in your case.
I am aware that there has been sometimes, in some points,
a misplaced distinction; but I think too well of you, Fanny,
to suppose you will ever harbour resentment on that account.
You have an understanding which will prevent you from
receiving things only in part, and judging partially
by the event.  You will take in the whole of the past,
you will consider times, persons, and probabilities,
and you will feel that \emph{they} were not least your
friends who were educating and preparing you for that
mediocrity of condition which \emph{seemed} to be your lot.
Though their caution may prove eventually unnecessary,
it was kindly meant; and of this you may be assured,
that every advantage of affluence will be doubled by the little
privations and restrictions that may have been imposed.
I am sure you will not disappoint my opinion of you,
by failing at any time to treat your aunt Norris
with the respect and attention that are due to her.
But enough of this.  Sit down, my dear.  I must speak
to you for a few minutes, but I will not detain
you long.''

Fanny obeyed, with eyes cast down and colour rising.
After a moment's pause, Sir Thomas, trying to suppress
a smile, went on.

``You are not aware, perhaps, that I have had a visitor
this morning.  I had not been long in my own room,
after breakfast, when Mr.\ Crawford was shewn in.
His errand you may probably conjecture.''

Fanny's colour grew deeper and deeper; and her uncle,
perceiving that she was embarrassed to a degree that
made either speaking or looking up quite impossible,
turned away his own eyes, and without any farther pause
proceeded in his account of Mr.\ Crawford's visit.

Mr.\ Crawford's business had been to declare himself
the lover of Fanny, make decided proposals for her,
and entreat the sanction of the uncle, who seemed to stand
in the place of her parents; and he had done it all so well,
so openly, so liberally, so properly, that Sir Thomas,
feeling, moreover, his own replies, and his own remarks
to have been very much to the purpose, was exceedingly
happy to give the particulars of their conversation;
and little aware of what was passing in his niece's mind,
conceived that by such details he must be gratifying her
far more than himself.  He talked, therefore, for several
minutes without Fanny's daring to interrupt him.
She had hardly even attained the wish to do it.  Her mind
was in too much confusion.  She had changed her position;
and, with her eyes fixed intently on one of the windows,
was listening to her uncle in the utmost perturbation
and dismay.  For a moment he ceased, but she had barely
become conscious of it, when, rising from his chair, he said,
``And now, Fanny, having performed one part of my commission,
and shewn you everything placed on a basis the most assured
and satisfactory, I may execute the remainder by prevailing
on you to accompany me downstairs, where, though I cannot
but presume on having been no unacceptable companion myself,
I must submit to your finding one still better worth
listening to.  Mr.\ Crawford, as you have perhaps foreseen,
is yet in the house.  He is in my room, and hoping to see
you there.''

There was a look, a start, an exclamation on hearing this,
which astonished Sir Thomas; but what was his increase of
astonishment on hearing her exclaim---``Oh! no, sir, I cannot,
indeed I cannot go down to him.  Mr.\ Crawford ought to know---%
he must know that:  I told him enough yesterday to convince him;
he spoke to me on this subject yesterday, and I told him
without disguise that it was very disagreeable to me,
and quite out of my power to return his good opinion.''

``I do not catch your meaning,'' said Sir Thomas, sitting
down again.  ``Out of your power to return his good opinion?
What is all this?  I know he spoke to you yesterday,
and (as far as I understand) received as much encouragement
to proceed as a well-judging young woman could permit
herself to give.  I was very much pleased with what I
collected to have been your behaviour on the occasion;
it shewed a discretion highly to be commended.  But now,
when he has made his overtures so properly, and honourably---%
what are your scruples \emph{now}?''

``You are mistaken, sir,'' cried Fanny, forced by the anxiety
of the moment even to tell her uncle that he was wrong;
``you are quite mistaken.  How could Mr.\ Crawford say
such a thing?  I gave him no encouragement yesterday.
On the contrary, I told him, I cannot recollect my exact words,
but I am sure I told him that I would not listen to him,
that it was very unpleasant to me in every respect, and that
I begged him never to talk to me in that manner again.
I am sure I said as much as that and more; and I should
have said still more, if I had been quite certain of his
meaning anything seriously; but I did not like to be,
I could not bear to be, imputing more than might be intended.
I thought it might all pass for nothing with \emph{him}.''

She could say no more; her breath was almost gone.

``Am I to understand,'' said Sir Thomas, after a few moments'
silence, ``that you mean to \emph{refuse} Mr.\ Crawford?''

``Yes, sir.''

``Refuse him?''

``Yes, sir.''

``Refuse Mr.\ Crawford!  Upon what plea?  For what reason?''

``I---I cannot like him, sir, well enough to marry him.''

``This is very strange!'' said Sir Thomas, in a voice of
calm displeasure.  ``There is something in this which my
comprehension does not reach.  Here is a young man wishing
to pay his addresses to you, with everything to recommend him:
not merely situation in life, fortune, and character,
but with more than common agreeableness, with address
and conversation pleasing to everybody.  And he is not an
acquaintance of to-day; you have now known him some time.
His sister, moreover, is your intimate friend, and he has
been doing \emph{that} for your brother, which I should suppose
would have been almost sufficient recommendation to you,
had there been no other.  It is very uncertain when my
interest might have got William on.  He has done it already.''

``Yes,'' said Fanny, in a faint voice, and looking down
with fresh shame; and she did feel almost ashamed
of herself, after such a picture as her uncle had drawn,
for not liking Mr.\ Crawford.

``You must have been aware,'' continued Sir Thomas presently,
``you must have been some time aware of a particularity
in Mr.\ Crawford's manners to you.  This cannot have taken
you by surprise.  You must have observed his attentions;
and though you always received them very properly (I have
no accusation to make on that head), I never perceived them
to be unpleasant to you.  I am half inclined to think,
Fanny, that you do not quite know your own feelings.''

``Oh yes, sir! indeed I do.  His attentions were always---%
what I did not like.''

Sir Thomas looked at her with deeper surprise.
``This is beyond me,'' said he.  ``This requires explanation.
Young as you are, and having seen scarcely any one,
it is hardly possible that your affections---''

He paused and eyed her fixedly.  He saw her lips
formed into a \emph{no}, though the sound was inarticulate,
but her face was like scarlet.  That, however, in so
modest a girl, might be very compatible with innocence;
and chusing at least to appear satisfied, he quickly added,
``No, no, I know \emph{that} is quite out of the question;
quite impossible.  Well, there is nothing more to be said.''

And for a few minutes he did say nothing.  He was deep
in thought.  His niece was deep in thought likewise, trying to
harden and prepare herself against farther questioning.
She would rather die than own the truth; and she hoped,
by a little reflection, to fortify herself beyond
betraying it.

``Independently of the interest which Mr.\ Crawford's \emph{choice}
seemed to justify'' said Sir Thomas, beginning again,
and very composedly, ``his wishing to marry at all so
early is recommendatory to me.  I am an advocate for
early marriages, where there are means in proportion,
and would have every young man, with a sufficient income,
settle as soon after four-and-twenty as he can.  This is
so much my opinion, that I am sorry to think how little
likely my own eldest son, your cousin, Mr.\ Bertram,
is to marry early; but at present, as far as I can judge,
matrimony makes no part of his plans or thoughts.
I wish he were more likely to fix.''  Here was a glance
at Fanny.  ``Edmund, I consider, from his dispositions
and habits, as much more likely to marry early than
his brother.  \emph{He}, indeed, I have lately thought,
has seen the woman he could love, which, I am convinced,
my eldest son has not.  Am I right?  Do you agree with me,
my dear?''

``Yes, sir.''

It was gently, but it was calmly said, and Sir Thomas was
easy on the score of the cousins.  But the removal of his
alarm did his niece no service:  as her unaccountableness
was confirmed his displeasure increased; and getting up
and walking about the room with a frown, which Fanny could
picture to herself, though she dared not lift up her eyes,
he shortly afterwards, and in a voice of authority, said,
``Have you any reason, child, to think ill of Mr.\ Crawford's
temper?''

``No, sir.''

She longed to add, ``But of his principles I have''; but her
heart sunk under the appalling prospect of discussion,
explanation, and probably non-conviction. Her ill opinion
of him was founded chiefly on observations, which,
for her cousins' sake, she could scarcely dare mention
to their father.  Maria and Julia, and especially Maria,
were so closely implicated in Mr.\ Crawford's misconduct,
that she could not give his character, such as she
believed it, without betraying them.  She had hoped that,
to a man like her uncle, so discerning, so honourable,
so good, the simple acknowledgment of settled \emph{dislike}
on her side would have been sufficient.  To her infinite
grief she found it was not.

Sir Thomas came towards the table where she sat
in trembling wretchedness, and with a good deal of
cold sternness, said, ``It is of no use, I perceive,
to talk to you.  We had better put an end to this
most mortifying conference.  Mr.\ Crawford must not be
kept longer waiting.  I will, therefore, only add,
as thinking it my duty to mark my opinion of your conduct,
that you have disappointed every expectation I had formed,
and proved yourself of a character the very reverse
of what I had supposed.  For I \emph{had}, Fanny, as I think
my behaviour must have shewn, formed a very favourable
opinion of you from the period of my return to England.
I had thought you peculiarly free from wilfulness of temper,
self-conceit, and every tendency to that independence
of spirit which prevails so much in modern days,
even in young women, and which in young women is offensive
and disgusting beyond all common offence.  But you
have now shewn me that you can be wilful and perverse;
that you can and will decide for yourself, without any
consideration or deference for those who have surely some
right to guide you, without even asking their advice.
You have shewn yourself very, very different from anything
that I had imagined.  The advantage or disadvantage of
your family, of your parents, your brothers and sisters,
never seems to have had a moment's share in your thoughts
on this occasion.  How \emph{they} might be benefited,
how \emph{they} must rejoice in such an establishment for you,
is nothing to \emph{you}.  You think only of yourself,
and because you do not feel for Mr.\ Crawford exactly what a
young heated fancy imagines to be necessary for happiness,
you resolve to refuse him at once, without wishing
even for a little time to consider of it, a little more
time for cool consideration, and for really examining
your own inclinations; and are, in a wild fit of folly,
throwing away from you such an opportunity of being
settled in life, eligibly, honourably, nobly settled,
as will, probably, never occur to you again.  Here is a
young man of sense, of character, of temper, of manners,
and of fortune, exceedingly attached to you, and seeking
your hand in the most handsome and disinterested way;
and let me tell you, Fanny, that you may live eighteen years
longer in the world without being addressed by a man of half
Mr.\ Crawford's estate, or a tenth part of his merits.
Gladly would I have bestowed either of my own daughters
on him.  Maria is nobly married; but had Mr.\ Crawford
sought Julia's hand, I should have given it to him with
superior and more heartfelt satisfaction than I gave
Maria's to Mr.\ Rushworth.''  After half a moment's pause:
``And I should have been very much surprised had either
of my daughters, on receiving a proposal of marriage at any
time which might carry with it only \emph{half} the eligibility
of \emph{this}, immediately and peremptorily, and without paying
my opinion or my regard the compliment of any consultation,
put a decided negative on it.  I should have been much
surprised and much hurt by such a proceeding.  I should
have thought it a gross violation of duty and respect.
\emph{You} are not to be judged by the same rule.  You do not
owe me the duty of a child.  But, Fanny, if your heart
can acquit you of \emph{ingratitude}---''

He ceased.  Fanny was by this time crying so bitterly that,
angry as he was, he would not press that article farther.
Her heart was almost broke by such a picture of what
she appeared to him; by such accusations, so heavy,
so multiplied, so rising in dreadful gradation!
Self-willed, obstinate, selfish, and ungrateful.
He thought her all this.  She had deceived his expectations;
she had lost his good opinion.  What was to become
of her?

``I am very sorry,'' said she inarticulately, through her tears,
``I am very sorry indeed.''

``Sorry! yes, I hope you are sorry; and you will probably
have reason to be long sorry for this day's transactions.''

``If it were possible for me to do otherwise'' said she,
with another strong effort; ``but I am so perfectly
convinced that I could never make him happy, and that I
should be miserable myself.''

Another burst of tears; but in spite of that burst,
and in spite of that great black word \emph{miserable},
which served to introduce it, Sir Thomas began to think
a little relenting, a little change of inclination,
might have something to do with it; and to augur favourably
from the personal entreaty of the young man himself.
He knew her to be very timid, and exceedingly nervous;
and thought it not improbable that her mind might be
in such a state as a little time, a little pressing,
a little patience, and a little impatience, a judicious
mixture of all on the lover's side, might work their
usual effect on.  If the gentleman would but persevere,
if he had but love enough to persevere, Sir Thomas began
to have hopes; and these reflections having passed across
his mind and cheered it, ``Well,'' said he, in a tone
of becoming gravity, but of less anger, ``well, child,
dry up your tears.  There is no use in these tears;
they can do no good.  You must now come downstairs with me.
Mr.\ Crawford has been kept waiting too long already.
You must give him your own answer:  we cannot expect him
to be satisfied with less; and you only can explain to him
the grounds of that misconception of your sentiments, which,
unfortunately for himself, he certainly has imbibed.  I am
totally unequal to it.''

But Fanny shewed such reluctance, such misery, at the
idea of going down to him, that Sir Thomas, after a
little consideration, judged it better to indulge her.
His hopes from both gentleman and lady suffered a small
depression in consequence; but when he looked at his niece,
and saw the state of feature and complexion which her
crying had brought her into, he thought there might
be as much lost as gained by an immediate interview.
With a few words, therefore, of no particular meaning,
he walked off by himself, leaving his poor niece to sit
and cry over what had passed, with very wretched feelings.

Her mind was all disorder.  The past, present, future,
everything was terrible.  But her uncle's anger gave
her the severest pain of all.  Selfish and ungrateful!
to have appeared so to him!  She was miserable for ever.
She had no one to take her part, to counsel, or speak
for her.  Her only friend was absent.  He might have
softened his father; but all, perhaps all, would think
her selfish and ungrateful.  She might have to endure
the reproach again and again; she might hear it, or see it,
or know it to exist for ever in every connexion about her.
She could not but feel some resentment against Mr.\ Crawford;
yet, if he really loved her, and were unhappy too!
It was all wretchedness together.

In about a quarter of an hour her uncle returned;
she was almost ready to faint at the sight of him.
He spoke calmly, however, without austerity, without reproach,
and she revived a little.  There was comfort, too,
in his words, as well as his manner, for he began with,
``Mr.\ Crawford is gone:  he has just left me.  I need not
repeat what has passed.  I do not want to add to anything
you may now be feeling, by an account of what he has felt.
Suffice it, that he has behaved in the most gentlemanlike
and generous manner, and has confirmed me in a most
favourable opinion of his understanding, heart, and temper.
Upon my representation of what you were suffering,
he immediately, and with the greatest delicacy,
ceased to urge to see you for the present.''

Here Fanny, who had looked up, looked down again.  ``Of course,''
continued her uncle, ``it cannot be supposed but that he should
request to speak with you alone, be it only for five minutes;
a request too natural, a claim too just to be denied.
But there is no time fixed; perhaps to-morrow, or whenever
your spirits are composed enough.  For the present you
have only to tranquillise yourself.  Check these tears;
they do but exhaust you.  If, as I am willing to suppose,
you wish to shew me any observance, you will not give
way to these emotions, but endeavour to reason yourself
into a stronger frame of mind.  I advise you to go out:
the air will do you good; go out for an hour on the gravel;
you will have the shrubbery to yourself, and will be the
better for air and exercise.  And, Fanny'' (turning back
again for a moment), ``I shall make no mention below of
what has passed; I shall not even tell your aunt Bertram.
There is no occasion for spreading the disappointment;
say nothing about it yourself.''

This was an order to be most joyfully obeyed; this was
an act of kindness which Fanny felt at her heart.
To be spared from her aunt Norris's interminable
reproaches! he left her in a glow of gratitude.
Anything might be bearable rather than such reproaches.
Even to see Mr.\ Crawford would be less overpowering.

She walked out directly, as her uncle recommended,
and followed his advice throughout, as far as she could;
did check her tears; did earnestly try to compose her spirits
and strengthen her mind.  She wished to prove to him that she
did desire his comfort, and sought to regain his favour;
and he had given her another strong motive for exertion,
in keeping the whole affair from the knowledge of her aunts.
Not to excite suspicion by her look or manner was now
an object worth attaining; and she felt equal to almost
anything that might save her from her aunt Norris.

She was struck, quite struck, when, on returning from her
walk and going into the East room again, the first thing
which caught her eye was a fire lighted and burning.
A fire! it seemed too much; just at that time to be giving
her such an indulgence was exciting even painful gratitude.
She wondered that Sir Thomas could have leisure to think
of such a trifle again; but she soon found, from the voluntary
information of the housemaid, who came in to attend it,
that so it was to be every day.  Sir Thomas had given
orders for it.

``I must be a brute, indeed, if I can be really ungrateful!''
said she, in soliloquy.  ``Heaven defend me from
being ungrateful!''

She saw nothing more of her uncle, nor of her aunt Norris,
till they met at dinner.  Her uncle's behaviour to her
was then as nearly as possible what it had been before;
she was sure he did not mean there should be any change,
and that it was only her own conscience that could fancy any;
but her aunt was soon quarrelling with her; and when she
found how much and how unpleasantly her having only walked
out without her aunt's knowledge could be dwelt on,
she felt all the reason she had to bless the kindness
which saved her from the same spirit of reproach,
exerted on a more momentous subject.

``If I had known you were going out, I should have got you
just to go as far as my house with some orders for Nanny,''
said she, ``which I have since, to my very great inconvenience,
been obliged to go and carry myself.  I could very ill
spare the time, and you might have saved me the trouble,
if you would only have been so good as to let us know you
were going out.  It would have made no difference to you,
I suppose, whether you had walked in the shrubbery or gone
to my house.''

``I recommended the shrubbery to Fanny as the driest place,''
said Sir Thomas.

``Oh!'' said Mrs.\ Norris, with a moment's check,
``that was very kind of you, Sir Thomas; but you do not
know how dry the path is to my house.  Fanny would have
had quite as good a walk there, I assure you, with the
advantage of being of some use, and obliging her aunt:
it is all her fault.  If she would but have let us know
she was going out but there is a something about Fanny,
I have often observed it before---she likes to go her
own way to work; she does not like to be dictated to;
she takes her own independent walk whenever she can;
she certainly has a little spirit of secrecy, and independence,
and nonsense, about her, which I would advise her to get
the better of.''

As a general reflection on Fanny, Sir Thomas thought
nothing could be more unjust, though he had been so lately
expressing the same sentiments himself, and he tried to turn
the conversation:  tried repeatedly before he could succeed;
for Mrs.\ Norris had not discernment enough to perceive,
either now, or at any other time, to what degree he
thought well of his niece, or how very far he was from
wishing to have his own children's merits set off by
the depreciation of hers.  She was talking \emph{at} Fanny,
and resenting this private walk half through the dinner.

It was over, however, at last; and the evening set in with
more composure to Fanny, and more cheerfulness of spirits
than she could have hoped for after so stormy a morning;
but she trusted, in the first place, that she had done right:
that her judgment had not misled her.  For the purity
of her intentions she could answer; and she was willing
to hope, secondly, that her uncle's displeasure was abating,
and would abate farther as he considered the matter with
more impartiality, and felt, as a good man must feel,
how wretched, and how unpardonable, how hopeless,
and how wicked it was to marry without affection.

When the meeting with which she was threatened for the
morrow was past, she could not but flatter herself that
the subject would be finally concluded, and Mr.\ Crawford
once gone from Mansfield, that everything would soon
be as if no such subject had existed.  She would not,
could not believe, that Mr.\ Crawford's affection for her
could distress him long; his mind was not of that sort.
London would soon bring its cure.  In London he would
soon learn to wonder at his infatuation, and be thankful
for the right reason in her which had saved him from its
evil consequences.

While Fanny's mind was engaged in these sort of hopes,
her uncle was, soon after tea, called out of the room;
an occurrence too common to strike her, and she thought nothing
of it till the butler reappeared ten minutes afterwards,
and advancing decidedly towards herself, said, ``Sir Thomas
wishes to speak with you, ma'am, in his own room.''
Then it occurred to her what might be going on; a suspicion
rushed over her mind which drove the colour from her cheeks;
but instantly rising, she was preparing to obey, when Mrs.\ Norris
called out, ``Stay, stay, Fanny! what are you about? where
are you going? don't be in such a hurry.  Depend upon it,
it is not you who are wanted; depend upon it, it is me''
(looking at the butler); ``but you are so very eager to put
yourself forward.  What should Sir Thomas want you for?
It is me, Baddeley, you mean; I am coming this moment.
You mean me, Baddeley, I am sure; Sir Thomas wants me,
not Miss Price.''

But Baddeley was stout.  ``No, ma'am, it is Miss Price;
I am certain of its being Miss Price.''  And there was
a half-smile with the words, which meant, ``I do not think
you would answer the purpose at all.''

Mrs.\ Norris, much discontented, was obliged to compose
herself to work again; and Fanny, walking off in
agitating consciousness, found herself, as she anticipated,
in another minute alone with Mr.\ Crawford.



\chapter{Chapter 33}

\gintro{The conference} was neither so short nor so conclusive
as the lady had designed.  The gentleman was not
so easily satisfied.  He had all the disposition to
persevere that Sir Thomas could wish him.  He had vanity,
which strongly inclined him in the first place to think
she did love him, though she might not know it herself;
and which, secondly, when constrained at last to admit
that she did know her own present feelings, convinced him
that he should be able in time to make those feelings
what he wished.

He was in love, very much in love; and it was a love which,
operating on an active, sanguine spirit, of more warmth
than delicacy, made her affection appear of greater
consequence because it was withheld, and determined him
to have the glory, as well as the felicity, of forcing
her to love him.

He would not despair:  he would not desist.  He had every
well-grounded reason for solid attachment; he knew her
to have all the worth that could justify the warmest
hopes of lasting happiness with her; her conduct at this
very time, by speaking the disinterestedness and delicacy
of her character (qualities which he believed most rare
indeed), was of a sort to heighten all his wishes,
and confirm all his resolutions.  He knew not that he had a
pre-engaged heart to attack.  Of \emph{that} he had no suspicion.
He considered her rather as one who had never thought
on the subject enough to be in danger; who had been
guarded by youth, a youth of mind as lovely as of person;
whose modesty had prevented her from understanding
his attentions, and who was still overpowered by the
suddenness of addresses so wholly unexpected, and the novelty
of a situation which her fancy had never taken into account.

Must it not follow of course, that, when he was understood,
he should succeed?  He believed it fully.  Love such as his,
in a man like himself, must with perseverance secure a return,
and at no great distance; and he had so much delight in
the idea of obliging her to love him in a very short time,
that her not loving him now was scarcely regretted.
A little difficulty to be overcome was no evil to
Henry Crawford.  He rather derived spirits from it.
He had been apt to gain hearts too easily.  His situation
was new and animating.

To Fanny, however, who had known too much opposition all her
life to find any charm in it, all this was unintelligible.
She found that he did mean to persevere; but how he could,
after such language from her as she felt herself obliged
to use, was not to be understood.  She told him that she
did not love him, could not love him, was sure she never
should love him; that such a change was quite impossible;
that the subject was most painful to her; that she must
entreat him never to mention it again, to allow her to leave
him at once, and let it be considered as concluded for ever.
And when farther pressed, had added, that in her opinion
their dispositions were so totally dissimilar as to make
mutual affection incompatible; and that they were unfitted
for each other by nature, education, and habit.  All this
she had said, and with the earnestness of sincerity;
yet this was not enough, for he immediately denied there
being anything uncongenial in their characters, or anything
unfriendly in their situations; and positively declared,
that he would still love, and still hope!

Fanny knew her own meaning, but was no judge of her own manner.
Her manner was incurably gentle; and she was not aware
how much it concealed the sternness of her purpose.
Her diffidence, gratitude, and softness made every expression
of indifference seem almost an effort of self-denial;
seem, at least, to be giving nearly as much pain to herself
as to him.  Mr.\ Crawford was no longer the Mr.\ Crawford who,
as the clandestine, insidious, treacherous admirer of
Maria Bertram, had been her abhorrence, whom she had hated
to see or to speak to, in whom she could believe no good
quality to exist, and whose power, even of being agreeable,
she had barely acknowledged.  He was now the Mr.\ Crawford
who was addressing herself with ardent, disinterested love;
whose feelings were apparently become all that was
honourable and upright, whose views of happiness were all
fixed on a marriage of attachment; who was pouring out
his sense of her merits, describing and describing again
his affection, proving as far as words could prove it,
and in the language, tone, and spirit of a man of talent too,
that he sought her for her gentleness and her goodness;
and to complete the whole, he was now the Mr.\ Crawford
who had procured William's promotion!

Here was a change, and here were claims which could
not but operate!  She might have disdained him in all
the dignity of angry virtue, in the grounds of Sotherton,
or the theatre at Mansfield Park; but he approached
her now with rights that demanded different treatment.
She must be courteous, and she must be compassionate.
She must have a sensation of being honoured, and whether
thinking of herself or her brother, she must have a strong
feeling of gratitude.  The effect of the whole was a
manner so pitying and agitated, and words intermingled
with her refusal so expressive of obligation and concern,
that to a temper of vanity and hope like Crawford's,
the truth, or at least the strength of her indifference,
might well be questionable; and he was not so irrational
as Fanny considered him, in the professions of persevering,
assiduous, and not desponding attachment which closed
the interview.

It was with reluctance that he suffered her to go; but there
was no look of despair in parting to belie his words,
or give her hopes of his being less unreasonable than he
professed himself.

Now she was angry.  Some resentment did arise at a
perseverance so selfish and ungenerous.  Here was again
a want of delicacy and regard for others which had formerly
so struck and disgusted her.  Here was again a something
of the same Mr.\ Crawford whom she had so reprobated before.
How evidently was there a gross want of feeling and humanity
where his own pleasure was concerned; and alas! how always
known no principle to supply as a duty what the heart
was deficient in!  Had her own affections been as free
as perhaps they ought to have been, he never could have engaged
them.

So thought Fanny, in good truth and sober sadness,
as she sat musing over that too great indulgence and luxury
of a fire upstairs:  wondering at the past and present;
wondering at what was yet to come, and in a nervous
agitation which made nothing clear to her but the persuasion
of her being never under any circumstances able to love
Mr.\ Crawford, and the felicity of having a fire to sit
over and think of it.

Sir Thomas was obliged, or obliged himself, to wait till
the morrow for a knowledge of what had passed between
the young people.  He then saw Mr.\ Crawford, and received
his account.  The first feeling was disappointment:
he had hoped better things; he had thought that an hour's
entreaty from a young man like Crawford could not have worked
so little change on a gentle-tempered girl like Fanny;
but there was speedy comfort in the determined views
and sanguine perseverance of the lover; and when seeing
such confidence of success in the principal, Sir Thomas
was soon able to depend on it himself.

Nothing was omitted, on his side, of civility, compliment,
or kindness, that might assist the plan.  Mr.\ Crawford's
steadiness was honoured, and Fanny was praised, and the
connexion was still the most desirable in the world.
At Mansfield Park Mr.\ Crawford would always be welcome;
he had only to consult his own judgment and feelings as
to the frequency of his visits, at present or in future.
In all his niece's family and friends, there could be
but one opinion, one wish on the subject; the influence
of all who loved her must incline one way.

Everything was said that could encourage, every encouragement
received with grateful joy, and the gentlemen parted
the best of friends.

Satisfied that the cause was now on a footing the most
proper and hopeful, Sir Thomas resolved to abstain
from all farther importunity with his niece, and to
shew no open interference.  Upon her disposition he
believed kindness might be the best way of working.
Entreaty should be from one quarter only.  The forbearance
of her family on a point, respecting which she could
be in no doubt of their wishes, might be their surest
means of forwarding it.  Accordingly, on this principle,
Sir Thomas took the first opportunity of saying to her,
with a mild gravity, intended to be overcoming,
``Well, Fanny, I have seen Mr.\ Crawford again, and learn
from him exactly how matters stand between you.  He is
a most extraordinary young man, and whatever be the event,
you must feel that you have created an attachment of no
common character; though, young as you are, and little
acquainted with the transient, varying, unsteady nature
of love, as it generally exists, you cannot be struck
as I am with all that is wonderful in a perseverance
of this sort against discouragement.  With him it is
entirely a matter of feeling:  he claims no merit in it;
perhaps is entitled to none.  Yet, having chosen so well,
his constancy has a respectable stamp.  Had his choice
been less unexceptionable, I should have condemned
his persevering.''

``Indeed, sir,'' said Fanny, ``I am very sorry that Mr.\ Crawford
should continue to know that it is paying me a very
great compliment, and I feel most undeservedly honoured;
but I am so perfectly convinced, and I have told him so,
that it never will be in my power---''

``My dear,'' interrupted Sir Thomas, ``there is no
occasion for this.  Your feelings are as well known
to me as my wishes and regrets must be to you.
There is nothing more to be said or done.  From this
hour the subject is never to be revived between us.
You will have nothing to fear, or to be agitated about.
You cannot suppose me capable of trying to persuade you
to marry against your inclinations.  Your happiness
and advantage are all that I have in view, and nothing is
required of you but to bear with Mr.\ Crawford's endeavours
to convince you that they may not be incompatible with his.
He proceeds at his own risk.  You are on safe ground.
I have engaged for your seeing him whenever he calls,
as you might have done had nothing of this sort occurred.
You will see him with the rest of us, in the same manner,
and, as much as you can, dismissing the recollection of
everything unpleasant.  He leaves Northamptonshire so soon,
that even this slight sacrifice cannot be often demanded.
The future must be very uncertain.  And now, my dear Fanny,
this subject is closed between us.''

The promised departure was all that Fanny could think
of with much satisfaction.  Her uncle's kind expressions,
however, and forbearing manner, were sensibly felt;
and when she considered how much of the truth was unknown
to him, she believed she had no right to wonder at the line
of conduct he pursued.  He, who had married a daughter
to Mr.\ Rushworth:  romantic delicacy was certainly not
to be expected from him.  She must do her duty, and trust
that time might make her duty easier than it now was.

She could not, though only eighteen, suppose Mr.\ Crawford's
attachment would hold out for ever; she could not
but imagine that steady, unceasing discouragement from
herself would put an end to it in time.  How much time
she might, in her own fancy, allot for its dominion,
is another concern.  It would not be fair to inquire
into a young lady's exact estimate of her own perfections.

In spite of his intended silence, Sir Thomas found himself
once more obliged to mention the subject to his niece,
to prepare her briefly for its being imparted to her aunts;
a measure which he would still have avoided, if possible,
but which became necessary from the totally opposite
feelings of Mr.\ Crawford as to any secrecy of proceeding.
He had no idea of concealment.  It was all known at
the Parsonage, where he loved to talk over the future
with both his sisters, and it would be rather gratifying
to him to have enlightened witnesses of the progress
of his success.  When Sir Thomas understood this, he felt
the necessity of making his own wife and sister-in-law
acquainted with the business without delay; though,
on Fanny's account, he almost dreaded the effect of the
communication to Mrs.\ Norris as much as Fanny herself.
He deprecated her mistaken but well-meaning zeal.
Sir Thomas, indeed, was, by this time, not very far from
classing Mrs.\ Norris as one of those well-meaning people
who are always doing mistaken and very disagreeable things.

Mrs.\ Norris, however, relieved him.  He pressed
for the strictest forbearance and silence towards
their niece; she not only promised, but did observe it.
She only looked her increased ill-will. Angry she was:
bitterly angry; but she was more angry with Fanny for
having received such an offer than for refusing it.
It was an injury and affront to Julia, who ought to have
been Mr.\ Crawford's choice; and, independently of that,
she disliked Fanny, because she had neglected her;
and she would have grudged such an elevation to one whom
she had been always trying to depress.

Sir Thomas gave her more credit for discretion on the
occasion than she deserved; and Fanny could have blessed
her for allowing her only to see her displeasure,
and not to hear it.

Lady Bertram took it differently.  She had been a beauty,
and a prosperous beauty, all her life; and beauty
and wealth were all that excited her respect.  To know
Fanny to be sought in marriage by a man of fortune,
raised her, therefore, very much in her opinion.
By convincing her that Fanny \emph{was} very pretty, which she
had been doubting about before, and that she would be
advantageously married, it made her feel a sort of credit
in calling her niece.

``Well, Fanny,'' said she, as soon as they were alone
together afterwards, and she really had known something
like impatience to be alone with her, and her countenance,
as she spoke, had extraordinary animation; ``Well, Fanny,
I have had a very agreeable surprise this morning.  I must
just speak of it \emph{once}, I told Sir Thomas I must \emph{once},
and then I shall have done.  I give you joy, my dear niece.''
And looking at her complacently, she added, ``Humph, we
certainly are a handsome family!''

Fanny coloured, and doubted at first what to say;
when, hoping to assail her on her vulnerable side,
she presently answered---%

``My dear aunt, \emph{you} cannot wish me to do differently from
what I have done, I am sure.  \emph{You} cannot wish me to marry;
for you would miss me, should not you?  Yes, I am sure
you would miss me too much for that.''

``No, my dear, I should not think of missing you,
when such an offer as this comes in your way.
I could do very well without you, if you were married
to a man of such good estate as Mr.\ Crawford.  And you
must be aware, Fanny, that it is every young woman's
duty to accept such a very unexceptionable offer as this.''

This was almost the only rule of conduct, the only piece
of advice, which Fanny had ever received from her aunt
in the course of eight years and a half.  It silenced her.
She felt how unprofitable contention would be.
If her aunt's feelings were against her, nothing could
be hoped from attacking her understanding.  Lady Bertram
was quite talkative.

``I will tell you what, Fanny,'' said she, ``I am sure he
fell in love with you at the ball; I am sure the mischief
was done that evening.  You did look remarkably well.
Everybody said so.  Sir Thomas said so.  And you know
you had Chapman to help you to dress.  I am very glad
I sent Chapman to you.  I shall tell Sir Thomas that I
am sure it was done that evening.''  And still pursuing
the same cheerful thoughts, she soon afterwards added,
``And will tell you what, Fanny, which is more than I did
for Maria:  the next time Pug has a litter you shall have
a puppy.''



\chapter{Chapter 34}

\gintro{Edmund} had great things to hear on his return.  Many surprises
were awaiting him.  The first that occurred was not least
in interest:  the appearance of Henry Crawford and his sister
walking together through the village as he rode into it.
He had concluded---he had meant them to be far distant.
His absence had been extended beyond a fortnight purposely
to avoid Miss Crawford.  He was returning to Mansfield
with spirits ready to feed on melancholy remembrances,
and tender associations, when her own fair self was
before him, leaning on her brother's arm, and he found
himself receiving a welcome, unquestionably friendly,
from the woman whom, two moments before, he had been
thinking of as seventy miles off, and as farther,
much farther, from him in inclination than any distance
could express.

Her reception of him was of a sort which he could not
have hoped for, had he expected to see her.  Coming as he
did from such a purport fulfilled as had taken him away,
he would have expected anything rather than a look
of satisfaction, and words of simple, pleasant meaning.
It was enough to set his heart in a glow, and to bring him
home in the properest state for feeling the full value
of the other joyful surprises at hand.

William's promotion, with all its particulars, he was soon
master of; and with such a secret provision of comfort
within his own breast to help the joy, he found in it
a source of most gratifying sensation and unvarying
cheerfulness all dinner-time.

After dinner, when he and his father were alone,
he had Fanny's history; and then all the great events
of the last fortnight, and the present situation
of matters at Mansfield were known to him.

Fanny suspected what was going on.  They sat so much
longer than usual in the dining-parlour, that she was sure
they must be talking of her; and when tea at last brought
them away, and she was to be seen by Edmund again, she felt
dreadfully guilty.  He came to her, sat down by her,
took her hand, and pressed it kindly; and at that moment
she thought that, but for the occupation and the scene
which the tea-things afforded, she must have betrayed
her emotion in some unpardonable excess.

He was not intending, however, by such action,
to be conveying to her that unqualified approbation
and encouragement which her hopes drew from it.
It was designed only to express his participation in all
that interested her, and to tell her that he had been
hearing what quickened every feeling of affection.  He was,
in fact, entirely on his father's side of the question.
His surprise was not so great as his father's at her
refusing Crawford, because, so far from supposing
her to consider him with anything like a preference,
he had always believed it to be rather the reverse,
and could imagine her to be taken perfectly unprepared,
but Sir Thomas could not regard the connexion as more
desirable than he did.  It had every recommendation to him;
and while honouring her for what she had done under the
influence of her present indifference, honouring her in
rather stronger terms than Sir Thomas could quite echo,
he was most earnest in hoping, and sanguine in believing,
that it would be a match at last, and that, united by
mutual affection, it would appear that their dispositions
were as exactly fitted to make them blessed in each other,
as he was now beginning seriously to consider them.
Crawford had been too precipitate.  He had not given her
time to attach herself.  He had begun at the wrong end.
With such powers as his, however, and such a disposition
as hers, Edmund trusted that everything would work
out a happy conclusion.  Meanwhile, he saw enough
of Fanny's embarrassment to make him scrupulously guard
against exciting it a second time, by any word, or look,
or movement.

Crawford called the next day, and on the score of Edmund's
return, Sir Thomas felt himself more than licensed to ask
him to stay dinner; it was really a necessary compliment.
He staid of course, and Edmund had then ample opportunity
for observing how he sped with Fanny, and what degree
of immediate encouragement for him might be extracted from
her manners; and it was so little, so very, very little---%
every chance, every possibility of it, resting upon her
embarrassment only; if there was not hope in her confusion,
there was hope in nothing else---that he was almost ready
to wonder at his friend's perseverance.  Fanny was worth
it all; he held her to be worth every effort of patience,
every exertion of mind, but he did not think he could have
gone on himself with any woman breathing, without something
more to warm his courage than his eyes could discern in hers.
He was very willing to hope that Crawford saw clearer,
and this was the most comfortable conclusion for his
friend that he could come to from all that he observed
to pass before, and at, and after dinner.

In the evening a few circumstances occurred which he thought
more promising.  When he and Crawford walked into the
drawing-room, his mother and Fanny were sitting as intently
and silently at work as if there were nothing else to care for.
Edmund could not help noticing their apparently deep tranquillity.

``We have not been so silent all the time,'' replied his mother.
``Fanny has been reading to me, and only put the book
down upon hearing you coming.''  And sure enough there
was a book on the table which had the air of being
very recently closed:  a volume of Shakespeare.
``She often reads to me out of those books; and she
was in the middle of a very fine speech of that man's---%
what's his name, Fanny?---when we heard your footsteps.''

Crawford took the volume.  ``Let me have the pleasure
of finishing that speech to your ladyship,'' said he.
``I shall find it immediately.''  And by carefully giving
way to the inclination of the leaves, he did find it,
or within a page or two, quite near enough to satisfy
Lady Bertram, who assured him, as soon as he mentioned the
name of Cardinal Wolsey, that he had got the very speech.
Not a look or an offer of help had Fanny given; not a syllable
for or against.  All her attention was for her work.
She seemed determined to be interested by nothing else.
But taste was too strong in her.  She could not abstract
her mind five minutes:  she was forced to listen; his reading
was capital, and her pleasure in good reading extreme.
To \emph{good} reading, however, she had been long used:
her uncle read well, her cousins all, Edmund very well,
but in Mr.\ Crawford's reading there was a variety of
excellence beyond what she had ever met with.  The King,
the Queen, Buckingham, Wolsey, Cromwell, all were given
in turn; for with the happiest knack, the happiest
power of jumping and guessing, he could always alight
at will on the best scene, or the best speeches of each;
and whether it were dignity, or pride, or tenderness,
or remorse, or whatever were to be expressed, he could
do it with equal beauty.  It was truly dramatic.
His acting had first taught Fanny what pleasure a play
might give, and his reading brought all his acting before
her again; nay, perhaps with greater enjoyment, for it
came unexpectedly, and with no such drawback as she had
been used to suffer in seeing him on the stage with Miss
Bertram.

Edmund watched the progress of her attention, and was
amused and gratified by seeing how she gradually slackened
in the needlework, which at the beginning seemed to
occupy her totally:  how it fell from her hand while
she sat motionless over it, and at last, how the eyes
which had appeared so studiously to avoid him throughout
the day were turned and fixed on Crawford---fixed on him
for minutes, fixed on him, in short, till the attraction
drew Crawford's upon her, and the book was closed,
and the charm was broken.  Then she was shrinking again
into herself, and blushing and working as hard as ever;
but it had been enough to give Edmund encouragement
for his friend, and as he cordially thanked him,
he hoped to be expressing Fanny's secret feelings too.

``That play must be a favourite with you,'' said he;
``you read as if you knew it well.''

``It will be a favourite, I believe, from this hour,''
replied Crawford; ``but I do not think I have had a volume
of Shakespeare in my hand before since I was fifteen.
I once saw Henry the Eighth acted, or I have heard
of it from somebody who did, I am not certain which.
But Shakespeare one gets acquainted with without knowing how.
It is a part of an Englishman's constitution.  His thoughts
and beauties are so spread abroad that one touches
them everywhere; one is intimate with him by instinct.
No man of any brain can open at a good part of one
of his plays without falling into the flow of his
meaning immediately.''

``No doubt one is familiar with Shakespeare in a degree,''
said Edmund, ``from one's earliest years.  His celebrated
passages are quoted by everybody; they are in half
the books we open, and we all talk Shakespeare,
use his similes, and describe with his descriptions;
but this is totally distinct from giving his sense as you
gave it.  To know him in bits and scraps is common enough;
to know him pretty thoroughly is, perhaps, not uncommon;
but to read him well aloud is no everyday talent.''

``Sir, you do me honour,'' was Crawford's answer, with a bow
of mock gravity.

Both gentlemen had a glance at Fanny, to see if a word
of accordant praise could be extorted from her; yet both
feeling that it could not be.  Her praise had been given
in her attention; \emph{that} must content them.

Lady Bertram's admiration was expressed, and strongly too.
``It was really like being at a play,'' said she.  ``I wish
Sir Thomas had been here.''

Crawford was excessively pleased.  If Lady Bertram,
with all her incompetency and languor, could feel this,
the inference of what her niece, alive and enlightened
as she was, must feel, was elevating.

``You have a great turn for acting, I am sure, Mr.\ Crawford,''
said her ladyship soon afterwards; ``and I will tell you what,
I think you will have a theatre, some time or other,
at your house in Norfolk.  I mean when you are settled there.
I do indeed.  I think you will fit up a theatre at your
house in Norfolk.''

``Do you, ma'am?'' cried he, with quickness.  ``No, no,
that will never be.  Your ladyship is quite mistaken.
No theatre at Everingham!  Oh no!''  And he looked at Fanny
with an expressive smile, which evidently meant, ``That lady
will never allow a theatre at Everingham.''

Edmund saw it all, and saw Fanny so determined \emph{not} to see it,
as to make it clear that the voice was enough to convey
the full meaning of the protestation; and such a quick
consciousness of compliment, such a ready comprehension
of a hint, he thought, was rather favourable than not.

The subject of reading aloud was farther discussed.
The two young men were the only talkers, but they,
standing by the fire, talked over the too common neglect
of the qualification, the total inattention to it, in the
ordinary school-system for boys, the consequently natural,
yet in some instances almost unnatural, degree of ignorance
and uncouthness of men, of sensible and well-informed men,
when suddenly called to the necessity of reading aloud,
which had fallen within their notice, giving instances
of blunders, and failures with their secondary causes,
the want of management of the voice, of proper modulation
and emphasis, of foresight and judgment, all proceeding
from the first cause:  want of early attention and habit;
and Fanny was listening again with great entertainment.

``Even in my profession,'' said Edmund, with a smile,
``how little the art of reading has been studied! how little
a clear manner, and good delivery, have been attended to!
I speak rather of the past, however, than the present.
There is now a spirit of improvement abroad; but among
those who were ordained twenty, thirty, forty years ago,
the larger number, to judge by their performance,
must have thought reading was reading, and preaching
was preaching.  It is different now.  The subject is more
justly considered.  It is felt that distinctness and energy
may have weight in recommending the most solid truths;
and besides, there is more general observation and taste,
a more critical knowledge diffused than formerly;
in every congregation there is a larger proportion
who know a little of the matter, and who can judge
and criticise.''

Edmund had already gone through the service once since
his ordination; and upon this being understood, he had
a variety of questions from Crawford as to his feelings
and success; questions, which being made, though with the
vivacity of friendly interest and quick taste, without any
touch of that spirit of banter or air of levity which Edmund
knew to be most offensive to Fanny, he had true pleasure
in satisfying; and when Crawford proceeded to ask his
opinion and give his own as to the properest manner in which
particular passages in the service should be delivered,
shewing it to be a subject on which he had thought before,
and thought with judgment, Edmund was still more and
more pleased.  This would be the way to Fanny's heart.
She was not to be won by all that gallantry and wit and
good-nature together could do; or, at least, she would
not be won by them nearly so soon, without the assistance
of sentiment and feeling, and seriousness on serious subjects.

``Our liturgy,'' observed Crawford, ``has beauties, which not
even a careless, slovenly style of reading can destroy;
but it has also redundancies and repetitions which require
good reading not to be felt.  For myself, at least, I must
confess being not always so attentive as I ought to be''
(here was a glance at Fanny); ``that nineteen times out of
twenty I am thinking how such a prayer ought to be read,
and longing to have it to read myself.  Did you speak?''
stepping eagerly to Fanny, and addressing her in a
softened voice; and upon her saying ``No,'' he added,
``Are you sure you did not speak?  I saw your lips move.
I fancied you might be going to tell me I ought to be
more attentive, and not \emph{allow} my thoughts to wander.
Are not you going to tell me so?''

``No, indeed, you know your duty too well for me to---%
even supposing---''

She stopt, felt herself getting into a puzzle, and could
not be prevailed on to add another word, not by dint
of several minutes of supplication and waiting.  He then
returned to his former station, and went on as if there
had been no such tender interruption.

``A sermon, well delivered, is more uncommon even than prayers
well read.  A sermon, good in itself, is no rare thing.
It is more difficult to speak well than to compose well;
that is, the rules and trick of composition are
oftener an object of study.  A thoroughly good sermon,
thoroughly well delivered, is a capital gratification.
I can never hear such a one without the greatest admiration
and respect, and more than half a mind to take orders
and preach myself.  There is something in the eloquence
of the pulpit, when it is really eloquence, which is entitled
to the highest praise and honour.  The preacher who can
touch and affect such an heterogeneous mass of hearers,
on subjects limited, and long worn threadbare in all
common hands; who can say anything new or striking,
anything that rouses the attention without offending the taste,
or wearing out the feelings of his hearers, is a man whom
one could not, in his public capacity, honour enough.
I should like to be such a man.''

Edmund laughed.

``I should indeed.  I never listened to a distinguished
preacher in my life without a sort of envy.  But then,
I must have a London audience.  I could not preach but
to the educated; to those who were capable of estimating
my composition.  And I do not know that I should be fond
of preaching often; now and then, perhaps once or twice
in the spring, after being anxiously expected for half
a dozen Sundays together; but not for a constancy;
it would not do for a constancy.''

Here Fanny, who could not but listen, involuntarily shook
her head, and Crawford was instantly by her side again,
entreating to know her meaning; and as Edmund perceived,
by his drawing in a chair, and sitting down close by her,
that it was to be a very thorough attack, that looks
and undertones were to be well tried, he sank as quietly
as possible into a corner, turned his back, and took up
a newspaper, very sincerely wishing that dear little
Fanny might be persuaded into explaining away that shake
of the head to the satisfaction of her ardent lover;
and as earnestly trying to bury every sound of the business
from himself in murmurs of his own, over the various
advertisements of ``A most desirable Estate in South
Wales''; ``To Parents and Guardians''; and a ``Capital
season'd Hunter.''

Fanny, meanwhile, vexed with herself for not having been
as motionless as she was speechless, and grieved to the heart
to see Edmund's arrangements, was trying by everything
in the power of her modest, gentle nature, to repulse
Mr.\ Crawford, and avoid both his looks and inquiries;
and he, unrepulsable, was persisting in both.

``What did that shake of the head mean?'' said he.  ``What was
it meant to express?  Disapprobation, I fear.  But of what?
What had I been saying to displease you?  Did you think me
speaking improperly, lightly, irreverently on the subject?
Only tell me if I was.  Only tell me if I was wrong.
I want to be set right.  Nay, nay, I entreat you;
for one moment put down your work.  What did that shake
of the head mean?''

In vain was her ``Pray, sir, don't; pray, Mr.\ Crawford,''
repeated twice over; and in vain did she try to move away.
In the same low, eager voice, and the same close neighbourhood,
he went on, reurging the same questions as before.
She grew more agitated and displeased.

``How can you, sir?  You quite astonish me; I wonder
how you can---''

``Do I astonish you?'' said he.  ``Do you wonder?  Is there
anything in my present entreaty that you do not understand?
I will explain to you instantly all that makes me urge
you in this manner, all that gives me an interest in
what you look and do, and excites my present curiosity.
I will not leave you to wonder long.''

In spite of herself, she could not help half a smile,
but she said nothing.

``You shook your head at my acknowledging that I should
not like to engage in the duties of a clergyman always
for a constancy.  Yes, that was the word.  Constancy:  I am
not afraid of the word.  I would spell it, read it,
write it with anybody.  I see nothing alarming in the word.
Did you think I ought?''

``Perhaps, sir,'' said Fanny, wearied at last into speaking---%
``perhaps, sir, I thought it was a pity you did not always
know yourself as well as you seemed to do at that moment.''

Crawford, delighted to get her to speak at any rate,
was determined to keep it up; and poor Fanny, who had
hoped to silence him by such an extremity of reproof,
found herself sadly mistaken, and that it was only a change
from one object of curiosity and one set of words to another.
He had always something to entreat the explanation of.
The opportunity was too fair.  None such had occurred
since his seeing her in her uncle's room, none such might
occur again before his leaving Mansfield.  Lady Bertram's
being just on the other side of the table was a trifle,
for she might always be considered as only half-awake, and
Edmund's advertisements were still of the first utility.

``Well,'' said Crawford, after a course of rapid questions
and reluctant answers; ``I am happier than I was, because I
now understand more clearly your opinion of me.  You think
me unsteady:  easily swayed by the whim of the moment,
easily tempted, easily put aside.  With such an opinion,
no wonder that.  But we shall see.  It is not by protestations
that I shall endeavour to convince you I am wronged;
it is not by telling you that my affections are steady.
My conduct shall speak for me; absence, distance, time shall
speak for me.  \emph{They} shall prove that, as far as you
can be deserved by anybody, I do deserve you.  You are
infinitely my superior in merit; all \emph{that} I know.
You have qualities which I had not before supposed
to exist in such a degree in any human creature.
You have some touches of the angel in you beyond what---%
not merely beyond what one sees, because one never sees
anything like it---but beyond what one fancies might be.
But still I am not frightened.  It is not by equality of
merit that you can be won.  That is out of the question.
It is he who sees and worships your merit the strongest,
who loves you most devotedly, that has the best
right to a return.  There I build my confidence.
By that right I do and will deserve you; and when once
convinced that my attachment is what I declare it,
I know you too well not to entertain the warmest hopes.
Yes, dearest, sweetest Fanny.  Nay'' (seeing her draw back
displeased), ``forgive me.  Perhaps I have as yet no right;
but by what other name can I call you?  Do you suppose
you are ever present to my imagination under any other?
No, it is `Fanny' that I think of all day, and dream
of all night.  You have given the name such reality
of sweetness, that nothing else can now be descriptive
of you.''

Fanny could hardly have kept her seat any longer,
or have refrained from at least trying to get away in
spite of all the too public opposition she foresaw to it,
had it not been for the sound of approaching relief,
the very sound which she had been long watching for,
and long thinking strangely delayed.

The solemn procession, headed by Baddeley, of tea-board, urn,
and cake-bearers, made its appearance, and delivered
her from a grievous imprisonment of body and mind.
Mr.\ Crawford was obliged to move.  She was at liberty,
she was busy, she was protected.

Edmund was not sorry to be admitted again among the
number of those who might speak and hear.  But though
the conference had seemed full long to him, and though
on looking at Fanny he saw rather a flush of vexation,
he inclined to hope that so much could not have been
said and listened to without some profit to the speaker.



\chapter{Chapter 35}

\gintro{Edmund} had determined that it belonged entirely to Fanny
to chuse whether her situation with regard to Crawford
should be mentioned between them or not; and that if she
did not lead the way, it should never be touched on by him;
but after a day or two of mutual reserve, he was induced
by his father to change his mind, and try what his influence
might do for his friend.

A day, and a very early day, was actually fixed for
the Crawfords' departure; and Sir Thomas thought it
might be as well to make one more effort for the young
man before he left Mansfield, that all his professions
and vows of unshaken attachment might have as much
hope to sustain them as possible.

Sir Thomas was most cordially anxious for the perfection
of Mr.\ Crawford's character in that point.  He wished him
to be a model of constancy; and fancied the best means
of effecting it would be by not trying him too long.

Edmund was not unwilling to be persuaded to engage
in the business; he wanted to know Fanny's feelings.
She had been used to consult him in every difficulty,
and he loved her too well to bear to be denied her
confidence now; he hoped to be of service to her, he thought
he must be of service to her; whom else had she to open
her heart to?  If she did not need counsel, she must need
the comfort of communication.  Fanny estranged from him,
silent and reserved, was an unnatural state of things;
a state which he must break through, and which he could
easily learn to think she was wanting him to break through.

``I will speak to her, sir:  I will take the first opportunity
of speaking to her alone,'' was the result of such thoughts
as these; and upon Sir Thomas's information of her
being at that very time walking alone in the shrubbery,
he instantly joined her.

``I am come to walk with you, Fanny,'' said he.  ``Shall I?''
Drawing her arm within his.  ``It is a long while since we
have had a comfortable walk together.''

She assented to it all rather by look than word.
Her spirits were low.

``But, Fanny,'' he presently added, ``in order to have a
comfortable walk, something more is necessary than merely
pacing this gravel together.  You must talk to me.
I know you have something on your mind.  I know what you
are thinking of.  You cannot suppose me uninformed.
Am I to hear of it from everybody but Fanny herself?''

Fanny, at once agitated and dejected, replied, ``If you
hear of it from everybody, cousin, there can be nothing
for me to tell.''

``Not of facts, perhaps; but of feelings, Fanny.
No one but you can tell me them.  I do not mean to
press you, however.  If it is not what you wish yourself,
I have done.  I had thought it might be a relief.''

``I am afraid we think too differently for me to find
any relief in talking of what I feel.''

``Do you suppose that we think differently?  I have no idea
of it.  I dare say that, on a comparison of our opinions,
they would be found as much alike as they have been used to be:
to the point---I consider Crawford's proposals as most
advantageous and desirable, if you could return his affection.
I consider it as most natural that all your family
should wish you could return it; but that, as you cannot,
you have done exactly as you ought in refusing him.
Can there be any disagreement between us here?''

``Oh no!  But I thought you blamed me.  I thought you
were against me.  This is such a comfort!''

``This comfort you might have had sooner, Fanny, had you
sought it.  But how could you possibly suppose me against you?
How could you imagine me an advocate for marriage without love?
Were I even careless in general on such matters, how could
you imagine me so where your happiness was at stake?''

``My uncle thought me wrong, and I knew he had been talking
to you.''

``As far as you have gone, Fanny, I think you perfectly right.
I may be sorry, I may be surprised---though hardly \emph{that},
for you had not had time to attach yourself---but I think
you perfectly right.  Can it admit of a question?
It is disgraceful to us if it does.  You did not love him;
nothing could have justified your accepting him.''

Fanny had not felt so comfortable for days and days.

``So far your conduct has been faultless, and they were quite
mistaken who wished you to do otherwise.  But the matter
does not end here.  Crawford's is no common attachment;
he perseveres, with the hope of creating that regard
which had not been created before.  This, we know,
must be a work of time.  But'' (with an affectionate smile)
``let him succeed at last, Fanny, let him succeed at last.
You have proved yourself upright and disinterested,
prove yourself grateful and tender-hearted; and then you
will be the perfect model of a woman which I have always
believed you born for.''

``Oh! never, never, never! he never will succeed with me.''
And she spoke with a warmth which quite astonished Edmund,
and which she blushed at the recollection of herself,
when she saw his look, and heard him reply, ``Never!  Fanny!---%
so very determined and positive!  This is not like yourself,
your rational self.''

``I mean,'' she cried, sorrowfully correcting herself,
``that I \emph{think} I never shall, as far as the future can
be answered for; I think I never shall return his regard.''

``I must hope better things.  I am aware, more aware
than Crawford can be, that the man who means to make
you love him (you having due notice of his intentions)
must have very uphill work, for there are all your early
attachments and habits in battle array; and before he
can get your heart for his own use he has to unfasten it
from all the holds upon things animate and inanimate,
which so many years' growth have confirmed, and which are
considerably tightened for the moment by the very idea
of separation.  I know that the apprehension of being
forced to quit Mansfield will for a time be arming you
against him.  I wish he had not been obliged to tell you
what he was trying for.  I wish he had known you as well as
I do, Fanny.  Between us, I think we should have won you.
My theoretical and his practical knowledge together could
not have failed.  He should have worked upon my plans.
I must hope, however, that time, proving him (as I firmly
believe it will) to deserve you by his steady affection,
will give him his reward.  I cannot suppose that you have
not the \emph{wish} to love him---the natural wish of gratitude.
You must have some feeling of that sort.  You must be sorry
for your own indifference.''

``We are so totally unlike,'' said Fanny, avoiding a
direct answer, ``we are so very, very different in all
our inclinations and ways, that I consider it as quite
impossible we should ever be tolerably happy together,
even if I \emph{could} like him.  There never were two people
more dissimilar.  We have not one taste in common.
We should be miserable.''

``You are mistaken, Fanny.  The dissimilarity is not so strong.
You are quite enough alike.  You \emph{have} tastes in common.
You have moral and literary tastes in common.  You have
both warm hearts and benevolent feelings; and, Fanny,
who that heard him read, and saw you listen to Shakespeare
the other night, will think you unfitted as companions?
You forget yourself:  there is a decided difference
in your tempers, I allow.  He is lively, you are serious;
but so much the better:  his spirits will support yours.
It is your disposition to be easily dejected and to fancy
difficulties greater than they are.  His cheerfulness
will counteract this.  He sees difficulties nowhere:
and his pleasantness and gaiety will be a constant support
to you.  Your being so far unlike, Fanny, does not in
the smallest degree make against the probability of your
happiness together:  do not imagine it.  I am myself
convinced that it is rather a favourable circumstance.
I am perfectly persuaded that the tempers had better be unlike:
I mean unlike in the flow of the spirits, in the manners,
in the inclination for much or little company, in the
propensity to talk or to be silent, to be grave or to be gay.
Some opposition here is, I am thoroughly convinced,
friendly to matrimonial happiness.  I exclude extremes,
of course; and a very close resemblance in all those
points would be the likeliest way to produce an extreme.
A counteraction, gentle and continual, is the best safeguard
of manners and conduct.''

Full well could Fanny guess where his thoughts were now:
Miss Crawford's power was all returning.  He had been
speaking of her cheerfully from the hour of his coming home.
His avoiding her was quite at an end.  He had dined at the
Parsonage only the preceding day.

After leaving him to his happier thoughts for some minutes,
Fanny, feeling it due to herself, returned to Mr.\ Crawford,
and said, ``It is not merely in \emph{temper} that I consider
him as totally unsuited to myself; though, in \emph{that}
respect, I think the difference between us too great,
infinitely too great:  his spirits often oppress me;
but there is something in him which I object to still more.
I must say, cousin, that I cannot approve his character.
I have not thought well of him from the time of the play.
I then saw him behaving, as it appeared to me, so very
improperly and unfeelingly---I may speak of it now because
it is all over---so improperly by poor Mr.\ Rushworth,
not seeming to care how he exposed or hurt him,
and paying attentions to my cousin Maria, which---in short,
at the time of the play, I received an impression which
will never be got over.''

``My dear Fanny,'' replied Edmund, scarcely hearing her
to the end, ``let us not, any of us, be judged by what we
appeared at that period of general folly.  The time of the
play is a time which I hate to recollect.  Maria was wrong,
Crawford was wrong, we were all wrong together; but none
so wrong as myself.  Compared with me, all the rest
were blameless.  I was playing the fool with my eyes open.''

``As a bystander,'' said Fanny, ``perhaps I saw more than
you did; and I do think that Mr.\ Rushworth was sometimes
very jealous.''

``Very possibly.  No wonder.  Nothing could be more improper
than the whole business.  I am shocked whenever I think
that Maria could be capable of it; but, if she could
undertake the part, we must not be surprised at the rest.''

``Before the play, I am much mistaken if \emph{Julia} did
not think he was paying her attentions.''

``Julia!  I have heard before from some one of his being
in love with Julia; but I could never see anything of it.
And, Fanny, though I hope I do justice to my sisters'
good qualities, I think it very possible that they might,
one or both, be more desirous of being admired by Crawford,
and might shew that desire rather more unguardedly than was
perfectly prudent.  I can remember that they were evidently
fond of his society; and with such encouragement, a man
like Crawford, lively, and it may be, a little unthinking,
might be led on to---there could be nothing very striking,
because it is clear that he had no pretensions:  his heart
was reserved for you.  And I must say, that its being
for you has raised him inconceivably in my opinion.
It does him the highest honour; it shews his proper estimation
of the blessing of domestic happiness and pure attachment.
It proves him unspoilt by his uncle.  It proves him, in short,
everything that I had been used to wish to believe him,
and feared he was not.''

``I am persuaded that he does not think, as he ought,
on serious subjects.''

``Say, rather, that he has not thought at all upon serious
subjects, which I believe to be a good deal the case.
How could it be otherwise, with such an education and adviser?
Under the disadvantages, indeed, which both have had,
is it not wonderful that they should be what they are?
Crawford's \emph{feelings}, I am ready to acknowledge, have hitherto
been too much his guides.  Happily, those feelings have
generally been good.  You will supply the rest; and a most
fortunate man he is to attach himself to such a creature---%
to a woman who, firm as a rock in her own principles, has a
gentleness of character so well adapted to recommend them.
He has chosen his partner, indeed, with rare felicity.
He will make you happy, Fanny; I know he will make you happy;
but you will make him everything.''

``I would not engage in such a charge,'' cried Fanny, in a
shrinking accent; ``in such an office of high responsibility!''

``As usual, believing yourself unequal to anything!
fancying everything too much for you!  Well, though I
may not be able to persuade you into different feelings,
you will be persuaded into them, I trust.
I confess myself sincerely anxious that you may.
I have no common interest in Crawford's well-doing. Next
to your happiness, Fanny, his has the first claim on me.
You are aware of my having no common interest in Crawford.''

Fanny was too well aware of it to have anything to say;
and they walked on together some fifty yards in mutual
silence and abstraction.  Edmund first began again---%

``I was very much pleased by her manner of speaking
of it yesterday, particularly pleased, because I had not
depended upon her seeing everything in so just a light.
I knew she was very fond of you; but yet I was afraid
of her not estimating your worth to her brother quite
as it deserved, and of her regretting that he had not
rather fixed on some woman of distinction or fortune.
I was afraid of the bias of those worldly maxims, which she
has been too much used to hear.  But it was very different.
She spoke of you, Fanny, just as she ought.  She desires
the connexion as warmly as your uncle or myself.
We had a long talk about it.  I should not have mentioned
the subject, though very anxious to know her sentiments;
but I had not been in the room five minutes before she
began introducing it with all that openness of heart,
and sweet peculiarity of manner, that spirit and ingenuousness
which are so much a part of herself.  Mrs.\ Grant laughed
at her for her rapidity.''

``Was Mrs.\ Grant in the room, then?''

``Yes, when I reached the house I found the two sisters
together by themselves; and when once we had begun,
we had not done with you, Fanny, till Crawford and Dr.\ Grant
came in.''

``It is above a week since I saw Miss Crawford.''

``Yes, she laments it; yet owns it may have been best.
You will see her, however, before she goes.  She is very
angry with you, Fanny; you must be prepared for that.
She calls herself very angry, but you can imagine her anger.
It is the regret and disappointment of a sister,
who thinks her brother has a right to everything he may
wish for, at the first moment.  She is hurt, as you would
be for William; but she loves and esteems you with all
her heart.''

``I knew she would be very angry with me.''

``My dearest Fanny,'' cried Edmund, pressing her arm closer
to him, ``do not let the idea of her anger distress you.
It is anger to be talked of rather than felt.  Her heart
is made for love and kindness, not for resentment.
I wish you could have overheard her tribute of praise;
I wish you could have seen her countenance, when she said
that you \emph{should} be Henry's wife.  And I observed that she
always spoke of you as `Fanny,' which she was never used to do;
and it had a sound of most sisterly cordiality.''

``And Mrs.\ Grant, did she say---did she speak; was she
there all the time?''

``Yes, she was agreeing exactly with her sister.  The surprise
of your refusal, Fanny, seems to have been unbounded.
That you could refuse such a man as Henry Crawford seems
more than they can understand.  I said what I could for you;
but in good truth, as they stated the case---you must
prove yourself to be in your senses as soon as you can
by a different conduct; nothing else will satisfy them.
But this is teasing you.  I have done.  Do not turn away
from me.''

``I \emph{should} have thought,'' said Fanny, after a pause
of recollection and exertion, ``that every woman must
have felt the possibility of a man's not being approved,
not being loved by some one of her sex at least, let him
be ever so generally agreeable.  Let him have all the
perfections in the world, I think it ought not to be set
down as certain that a man must be acceptable to every
woman he may happen to like himself.  But, even supposing
it is so, allowing Mr.\ Crawford to have all the claims
which his sisters think he has, how was I to be prepared
to meet him with any feeling answerable to his own?
He took me wholly by surprise.  I had not an idea that
his behaviour to me before had any meaning; and surely I
was not to be teaching myself to like him only because
he was taking what seemed very idle notice of me.
In my situation, it would have been the extreme of vanity
to be forming expectations on Mr.\ Crawford.  I am sure
his sisters, rating him as they do, must have thought it so,
supposing he had meant nothing.  How, then, was I to be---%
to be in love with him the moment he said he was with me?
How was I to have an attachment at his service, as soon
as it was asked for?  His sisters should consider me
as well as him.  The higher his deserts, the more improper
for me ever to have thought of him.  And, and---we think
very differently of the nature of women, if they can imagine
a woman so very soon capable of returning an affection
as this seems to imply.''

``My dear, dear Fanny, now I have the truth.  I know this
to be the truth; and most worthy of you are such feelings.
I had attributed them to you before.  I thought I could
understand you.  You have now given exactly the explanation
which I ventured to make for you to your friend and Mrs.\ Grant,
and they were both better satisfied, though your warm-hearted
friend was still run away with a little by the enthusiasm
of her fondness for Henry.  I told them that you were
of all human creatures the one over whom habit had most
power and novelty least; and that the very circumstance
of the novelty of Crawford's addresses was against him.
Their being so new and so recent was all in their disfavour;
that you could tolerate nothing that you were not used to;
and a great deal more to the same purpose, to give them
a knowledge of your character.  Miss Crawford made us
laugh by her plans of encouragement for her brother.
She meant to urge him to persevere in the hope of being
loved in time, and of having his addresses most kindly
received at the end of about ten years' happy marriage.''

Fanny could with difficulty give the smile that was
here asked for.  Her feelings were all in revolt.
She feared she had been doing wrong:  saying too much,
overacting the caution which she had been fancying necessary;
in guarding against one evil, laying herself open
to another; and to have Miss Crawford's liveliness
repeated to her at such a moment, and on such a subject,
was a bitter aggravation.

Edmund saw weariness and distress in her face,
and immediately resolved to forbear all farther discussion;
and not even to mention the name of Crawford again,
except as it might be connected with what \emph{must} be agreeable
to her.  On this principle, he soon afterwards observed---%
``They go on Monday.  You are sure, therefore, of seeing
your friend either to-morrow or Sunday.  They really go
on Monday; and I was within a trifle of being persuaded
to stay at Lessingby till that very day!  I had almost
promised it.  What a difference it might have made!
Those five or six days more at Lessingby might have been
felt all my life.''

``You were near staying there?''

``Very.  I was most kindly pressed, and had nearly consented.
Had I received any letter from Mansfield, to tell me how you
were all going on, I believe I should certainly have staid;
but I knew nothing that had happened here for a fortnight,
and felt that I had been away long enough.''

``You spent your time pleasantly there?''

``Yes; that is, it was the fault of my own mind if I did not.
They were all very pleasant.  I doubt their finding me so.
I took uneasiness with me, and there was no getting rid
of it till I was in Mansfield again.''

``The Miss Owens---you liked them, did not you?''

``Yes, very well.  Pleasant, good-humoured, unaffected girls.
But I am spoilt, Fanny, for common female society.
Good-humoured, unaffected girls will not do for a man
who has been used to sensible women.  They are two distinct
orders of being.  You and Miss Crawford have made me
too nice.''

Still, however, Fanny was oppressed and wearied;
he saw it in her looks, it could not be talked away;
and attempting it no more, he led her directly, with the
kind authority of a privileged guardian, into the house.



\chapter{Chapter 36}

\gintro{Edmund} now believed himself perfectly acquainted with all
that Fanny could tell, or could leave to be conjectured
of her sentiments, and he was satisfied.  It had been,
as he before presumed, too hasty a measure on Crawford's side,
and time must be given to make the idea first familiar,
and then agreeable to her.  She must be used to the
consideration of his being in love with her, and then
a return of affection might not be very distant.

He gave this opinion as the result of the conversation
to his father; and recommended there being nothing more said
to her:  no farther attempts to influence or persuade;
but that everything should be left to Crawford's assiduities,
and the natural workings of her own mind.

Sir Thomas promised that it should be so.  Edmund's account
of Fanny's disposition he could believe to be just;
he supposed she had all those feelings, but he must consider
it as very unfortunate that she \emph{had}; for, less willing
than his son to trust to the future, he could not help
fearing that if such very long allowances of time and habit
were necessary for her, she might not have persuaded
herself into receiving his addresses properly before
the young man's inclination for paying them were over.
There was nothing to be done, however, but to submit
quietly and hope the best.

The promised visit from ``her friend,'' as Edmund called
Miss Crawford, was a formidable threat to Fanny,
and she lived in continual terror of it.  As a sister,
so partial and so angry, and so little scrupulous of what
she said, and in another light so triumphant and secure,
she was in every way an object of painful alarm.
Her displeasure, her penetration, and her happiness were
all fearful to encounter; and the dependence of having
others present when they met was Fanny's only support
in looking forward to it.  She absented herself as little
as possible from Lady Bertram, kept away from the East room,
and took no solitary walk in the shrubbery, in her caution
to avoid any sudden attack.

She succeeded.  She was safe in the breakfast-room, with her aunt,
when Miss Crawford did come; and the first misery over,
and Miss Crawford looking and speaking with much less
particularity of expression than she had anticipated,
Fanny began to hope there would be nothing worse
to be endured than a half-hour of moderate agitation.
But here she hoped too much; Miss Crawford was not the slave
of opportunity.  She was determined to see Fanny alone,
and therefore said to her tolerably soon, in a low voice,
``I must speak to you for a few minutes somewhere'';
words that Fanny felt all over her, in all her pulses
and all her nerves.  Denial was impossible.  Her habits
of ready submission, on the contrary, made her almost
instantly rise and lead the way out of the room.
She did it with wretched feelings, but it was inevitable.

They were no sooner in the hall than all restraint
of countenance was over on Miss Crawford's side.
She immediately shook her head at Fanny with arch,
yet affectionate reproach, and taking her hand,
seemed hardly able to help beginning directly.
She said nothing, however, but, ``Sad, sad girl!
I do not know when I shall have done scolding you,''
and had discretion enough to reserve the rest till they
might be secure of having four walls to themselves.
Fanny naturally turned upstairs, and took her guest to the
apartment which was now always fit for comfortable use;
opening the door, however, with a most aching heart,
and feeling that she had a more distressing scene before
her than ever that spot had yet witnessed.  But the evil
ready to burst on her was at least delayed by the sudden
change in Miss Crawford's ideas; by the strong effect
on her mind which the finding herself in the East room
again produced.

``Ha!'' she cried, with instant animation, ``am I here again?
The East room!  Once only was I in this room before'';
and after stopping to look about her, and seemingly
to retrace all that had then passed, she added, ``Once
only before.  Do you remember it?  I came to rehearse.
Your cousin came too; and we had a rehearsal.  You were
our audience and prompter.  A delightful rehearsal.
I shall never forget it.  Here we were, just in this
part of the room:  here was your cousin, here was I,
here were the chairs.  Oh! why will such things ever
pass away?''

Happily for her companion, she wanted no answer.
Her mind was entirely self-engrossed. She was in a reverie
of sweet remembrances.

``The scene we were rehearsing was so very remarkable!
The subject of it so very---very---what shall I say?
He was to be describing and recommending matrimony to me.
I think I see him now, trying to be as demure and composed
as Anhalt ought, through the two long speeches.
`When two sympathetic hearts meet in the marriage state,
matrimony may be called a happy life.'  I suppose no time
can ever wear out the impression I have of his looks
and voice as he said those words.  It was curious,
very curious, that we should have such a scene to play!
If I had the power of recalling any one week of my existence,
it should be that week---that acting week.  Say what
you would, Fanny, it should be \emph{that}; for I never knew
such exquisite happiness in any other.  His sturdy spirit
to bend as it did!  Oh! it was sweet beyond expression.
But alas, that very evening destroyed it all.  That very
evening brought your most unwelcome uncle.  Poor Sir Thomas,
who was glad to see you?  Yet, Fanny, do not imagine I would
now speak disrespectfully of Sir Thomas, though I certainly
did hate him for many a week.  No, I do him justice now.
He is just what the head of such a family should be.
Nay, in sober sadness, I believe I now love you all.''
And having said so, with a degree of tenderness and
consciousness which Fanny had never seen in her before,
and now thought only too becoming, she turned away
for a moment to recover herself.  ``I have had a little
fit since I came into this room, as you may perceive,''
said she presently, with a playful smile, ``but it is
over now; so let us sit down and be comfortable; for as to
scolding you, Fanny, which I came fully intending to do,
I have not the heart for it when it comes to the point.''
And embracing her very affectionately, ``Good, gentle Fanny!
when I think of this being the last time of seeing you for I
do not know how long, I feel it quite impossible to do anything
but love you.''

Fanny was affected.  She had not foreseen anything of this,
and her feelings could seldom withstand the melancholy
influence of the word ``last.''  She cried as if she
had loved Miss Crawford more than she possibly could;
and Miss Crawford, yet farther softened by the sight
of such emotion, hung about her with fondness, and said,
``I hate to leave you.  I shall see no one half so amiable
where I am going.  Who says we shall not be sisters?
I know we shall.  I feel that we are born to be connected;
and those tears convince me that you feel it too,
dear Fanny.''

Fanny roused herself, and replying only in part, said,
``But you are only going from one set of friends to another.
You are going to a very particular friend.''

``Yes, very true.  Mrs.\ Fraser has been my intimate friend
for years.  But I have not the least inclination to go
near her.  I can think only of the friends I am leaving:
my excellent sister, yourself, and the Bertrams in general.
You have all so much more \emph{heart} among you than one
finds in the world at large.  You all give me a feeling
of being able to trust and confide in you, which in common
intercourse one knows nothing of.  I wish I had settled
with Mrs.\ Fraser not to go to her till after Easter, a much
better time for the visit, but now I cannot put her off.
And when I have done with her I must go to her sister,
Lady Stornaway, because \emph{she} was rather my most particular
friend of the two, but I have not cared much for \emph{her}
these three years.''

After this speech the two girls sat many minutes silent,
each thoughtful:  Fanny meditating on the different sorts
of friendship in the world, Mary on something of less
philosophic tendency.  \emph{She} first spoke again.

``How perfectly I remember my resolving to look for
you upstairs, and setting off to find my way to the
East room, without having an idea whereabouts it was!
How well I remember what I was thinking of as I came along,
and my looking in and seeing you here sitting at this
table at work; and then your cousin's astonishment,
when he opened the door, at seeing me here!  To be sure,
your uncle's returning that very evening!  There never
was anything quite like it.''

Another short fit of abstraction followed, when,
shaking it off, she thus attacked her companion.

``Why, Fanny, you are absolutely in a reverie.
Thinking, I hope, of one who is always thinking of you.
Oh! that I could transport you for a short time into
our circle in town, that you might understand how your
power over Henry is thought of there!  Oh! the envyings
and heartburnings of dozens and dozens; the wonder,
the incredulity that will be felt at hearing what you
have done!  For as to secrecy, Henry is quite the hero
of an old romance, and glories in his chains.  You should
come to London to know how to estimate your conquest.
If you were to see how he is courted, and how I am courted
for his sake!  Now, I am well aware that I shall not be
half so welcome to Mrs.\ Fraser in consequence of his
situation with you.  When she comes to know the truth
she will, very likely, wish me in Northamptonshire again;
for there is a daughter of Mr.\ Fraser, by a first wife,
whom she is wild to get married, and wants Henry to take.
Oh! she has been trying for him to such a degree.
Innocent and quiet as you sit here, you cannot have an
idea of the \emph{sensation} that you will be occasioning,
of the curiosity there will be to see you, of the endless
questions I shall have to answer!  Poor Margaret Fraser
will be at me for ever about your eyes and your teeth,
and how you do your hair, and who makes your shoes.
I wish Margaret were married, for my poor friend's sake,
for I look upon the Frasers to be about as unhappy as most
other married people.  And yet it was a most desirable
match for Janet at the time.  We were all delighted.
She could not do otherwise than accept him, for he was rich,
and she had nothing; but he turns out ill-tempered
and \emph{exigeant}, and wants a young woman, a beautiful young
woman of five-and-twenty, to be as steady as himself.
And my friend does not manage him well; she does not seem
to know how to make the best of it.  There is a spirit
of irritation which, to say nothing worse, is certainly
very ill-bred. In their house I shall call to mind the
conjugal manners of Mansfield Parsonage with respect.
Even Dr.\ Grant does shew a thorough confidence in my sister,
and a certain consideration for her judgment, which makes
one feel there \emph{is} attachment; but of that I shall
see nothing with the Frasers.  I shall be at Mansfield
for ever, Fanny.  My own sister as a wife, Sir Thomas
Bertram as a husband, are my standards of perfection.
Poor Janet has been sadly taken in, and yet there was
nothing improper on her side:  she did not run into the
match inconsiderately; there was no want of foresight.
She took three days to consider of his proposals,
and during those three days asked the advice of everybody
connected with her whose opinion was worth having,
and especially applied to my late dear aunt, whose
knowledge of the world made her judgment very generally
and deservedly looked up to by all the young people
of her acquaintance, and she was decidedly in favour
of Mr.\ Fraser.  This seems as if nothing were a security
for matrimonial comfort.  I have not so much to say
for my friend Flora, who jilted a very nice young man
in the Blues for the sake of that horrid Lord Stornaway,
who has about as much sense, Fanny, as Mr.\ Rushworth,
but much worse-looking, and with a blackguard character.
I \emph{had} my doubts at the time about her being right,
for he has not even the air of a gentleman, and now I am
sure she was wrong.  By the bye, Flora Ross was dying
for Henry the first winter she came out.  But were I
to attempt to tell you of all the women whom I have
known to be in love with him, I should never have done.
It is you, only you, insensible Fanny, who can think
of him with anything like indifference.  But are you
so insensible as you profess yourself?  No, no, I see you
are not.''

There was, indeed, so deep a blush over Fanny's face
at that moment as might warrant strong suspicion
in a predisposed mind.

``Excellent creature!  I will not tease you.  Everything shall
take its course.  But, dear Fanny, you must allow that you
were not so absolutely unprepared to have the question asked
as your cousin fancies.  It is not possible but that you
must have had some thoughts on the subject, some surmises
as to what might be.  You must have seen that he was
trying to please you by every attention in his power.
Was not he devoted to you at the ball?  And then before
the ball, the necklace!  Oh! you received it just as it
was meant.  You were as conscious as heart could desire.
I remember it perfectly.''

``Do you mean, then, that your brother knew of the
necklace beforehand?  Oh!  Miss Crawford, \emph{that} was not fair.''

``Knew of it!  It was his own doing entirely, his own thought.
I am ashamed to say that it had never entered my head,
but I was delighted to act on his proposal for both
your sakes.''

``I will not say,'' replied Fanny, ``that I was not half
afraid at the time of its being so, for there was something
in your look that frightened me, but not at first;
I was as unsuspicious of it at first---indeed, indeed I was.
It is as true as that I sit here.  And had I had an idea of it,
nothing should have induced me to accept the necklace.
As to your brother's behaviour, certainly I was sensible of
a particularity:  I had been sensible of it some little time,
perhaps two or three weeks; but then I considered it as
meaning nothing:  I put it down as simply being his way,
and was as far from supposing as from wishing him to have
any serious thoughts of me.  I had not, Miss Crawford,
been an inattentive observer of what was passing between him
and some part of this family in the summer and autumn.
I was quiet, but I was not blind.  I could not but see
that Mr.\ Crawford allowed himself in gallantries which did
mean nothing.''

``Ah!  I cannot deny it.  He has now and then been a sad flirt,
and cared very little for the havoc he might be making in
young ladies' affections.  I have often scolded him for it,
but it is his only fault; and there is this to be said,
that very few young ladies have any affections worth
caring for.  And then, Fanny, the glory of fixing one
who has been shot at by so many; of having it in one's
power to pay off the debts of one's sex!  Oh!  I am sure
it is not in woman's nature to refuse such a triumph.''

Fanny shook her head.  ``I cannot think well of a man
who sports with any woman's feelings; and there may often
be a great deal more suffered than a stander-by can judge of.''

``I do not defend him.  I leave him entirely to your mercy,
and when he has got you at Everingham, I do not care how much
you lecture him.  But this I will say, that his fault,
the liking to make girls a little in love with him, is not
half so dangerous to a wife's happiness as a tendency to fall
in love himself, which he has never been addicted to.
And I do seriously and truly believe that he is attached
to you in a way that he never was to any woman before;
that he loves you with all his heart, and will love you
as nearly for ever as possible.  If any man ever loved
a woman for ever, I think Henry will do as much for you.''

Fanny could not avoid a faint smile, but had nothing
to say.

``I cannot imagine Henry ever to have been happier,''
continued Mary presently, ``than when he had succeeded
in getting your brother's commission.''

She had made a sure push at Fanny's feelings here.

``Oh! yes.  How very, very kind of him.''

``I know he must have exerted himself very much, for I know
the parties he had to move.  The Admiral hates trouble,
and scorns asking favours; and there are so many
young men's claims to be attended to in the same way,
that a friendship and energy, not very determined,
is easily put by.  What a happy creature William must be!
I wish we could see him.''

Poor Fanny's mind was thrown into the most distressing
of all its varieties.  The recollection of what had
been done for William was always the most powerful
disturber of every decision against Mr.\ Crawford;
and she sat thinking deeply of it till Mary, who had been
first watching her complacently, and then musing on
something else, suddenly called her attention by saying:
``I should like to sit talking with you here all day,
but we must not forget the ladies below, and so good-bye,
my dear, my amiable, my excellent Fanny, for though we
shall nominally part in the breakfast-parlour, I must
take leave of you here.  And I do take leave, longing for
a happy reunion, and trusting that when we meet again,
it will be under circumstances which may open our hearts
to each other without any remnant or shadow of reserve.''

A very, very kind embrace, and some agitation of manner,
accompanied these words.

``I shall see your cousin in town soon:  he talks of
being there tolerably soon; and Sir Thomas, I dare say,
in the course of the spring; and your eldest cousin,
and the Rushworths, and Julia, I am sure of meeting again
and again, and all but you.  I have two favours to ask,
Fanny:  one is your correspondence.  You must write to me.
And the other, that you will often call on Mrs.\ Grant,
and make her amends for my being gone.''

The first, at least, of these favours Fanny would rather
not have been asked; but it was impossible for her to refuse
the correspondence; it was impossible for her even not to
accede to it more readily than her own judgment authorised.
There was no resisting so much apparent affection.
Her disposition was peculiarly calculated to value a fond
treatment, and from having hitherto known so little of it,
she was the more overcome by Miss Crawford's. Besides,
there was gratitude towards her, for having made their
\emph{tete-a-tete} so much less painful than her fears had predicted.

It was over, and she had escaped without reproaches
and without detection.  Her secret was still her own;
and while that was the case, she thought she could resign
herself to almost everything.

In the evening there was another parting.  Henry Crawford
came and sat some time with them; and her spirits not being
previously in the strongest state, her heart was softened
for a while towards him, because he really seemed to feel.
Quite unlike his usual self, he scarcely said anything.
He was evidently oppressed, and Fanny must grieve for him,
though hoping she might never see him again till he were the
husband of some other woman.

When it came to the moment of parting, he would take her hand,
he would not be denied it; he said nothing, however,
or nothing that she heard, and when he had left the room,
she was better pleased that such a token of friendship
had passed.

On the morrow the Crawfords were gone.



\chapter{Chapter 37}

\gintro{Mr.\ Crawford} gone, Sir Thomas's next object was that he
should be missed; and he entertained great hope that his
niece would find a blank in the loss of those attentions
which at the time she had felt, or fancied, an evil.
She had tasted of consequence in its most flattering form;
and he did hope that the loss of it, the sinking again
into nothing, would awaken very wholesome regrets
in her mind.  He watched her with this idea; but he
could hardly tell with what success.  He hardly knew
whether there were any difference in her spirits or not.
She was always so gentle and retiring that her emotions
were beyond his discrimination.  He did not understand her:
he felt that he did not; and therefore applied to Edmund
to tell him how she stood affected on the present occasion,
and whether she were more or less happy than she
had been.

Edmund did not discern any symptoms of regret, and thought
his father a little unreasonable in supposing the first
three or four days could produce any.

What chiefly surprised Edmund was, that Crawford's sister,
the friend and companion who had been so much to her,
should not be more visibly regretted.  He wondered that Fanny
spoke so seldom of \emph{her}, and had so little voluntarily
to say of her concern at this separation.

Alas! it was this sister, this friend and companion,
who was now the chief bane of Fanny's comfort.  If she
could have believed Mary's future fate as unconnected
with Mansfield as she was determined the brother's
should be, if she could have hoped her return thither
to be as distant as she was much inclined to think his,
she would have been light of heart indeed; but the more
she recollected and observed, the more deeply was she
convinced that everything was now in a fairer train
for Miss Crawford's marrying Edmund than it had ever
been before.  On his side the inclination was stronger,
on hers less equivocal.  His objections, the scruples of
his integrity, seemed all done away, nobody could tell how;
and the doubts and hesitations of her ambition were
equally got over---and equally without apparent reason.
It could only be imputed to increasing attachment.
His good and her bad feelings yielded to love, and such
love must unite them.  He was to go to town as soon as
some business relative to Thornton Lacey were completed---%
perhaps within a fortnight; he talked of going,
he loved to talk of it; and when once with her again,
Fanny could not doubt the rest.  Her acceptance must
be as certain as his offer; and yet there were bad
feelings still remaining which made the prospect of it
most sorrowful to her, independently, she believed,
independently of self.

In their very last conversation, Miss Crawford, in spite
of some amiable sensations, and much personal kindness,
had still been Miss Crawford; still shewn a mind led astray
and bewildered, and without any suspicion of being so;
darkened, yet fancying itself light.  She might love,
but she did not deserve Edmund by any other sentiment.
Fanny believed there was scarcely a second feeling
in common between them; and she may be forgiven by older
sages for looking on the chance of Miss Crawford's future
improvement as nearly desperate, for thinking that if Edmund's
influence in this season of love had already done so little
in clearing her judgment, and regulating her notions,
his worth would be finally wasted on her even in years
of matrimony.

Experience might have hoped more for any young people
so circumstanced, and impartiality would not have denied
to Miss Crawford's nature that participation of the general
nature of women which would lead her to adopt the opinions
of the man she loved and respected as her own.  But as such
were Fanny's persuasions, she suffered very much from them,
and could never speak of Miss Crawford without pain.

Sir Thomas, meanwhile, went on with his own hopes and
his own observations, still feeling a right, by all his
knowledge of human nature, to expect to see the effect
of the loss of power and consequence on his niece's spirits,
and the past attentions of the lover producing a craving
for their return; and he was soon afterwards able to account
for his not yet completely and indubitably seeing all this,
by the prospect of another visitor, whose approach he
could allow to be quite enough to support the spirits
he was watching.  William had obtained a ten days'
leave of absence, to be given to Northamptonshire,
and was coming, the happiest of lieutenants, because the
latest made, to shew his happiness and describe his uniform.

He came; and he would have been delighted to shew his uniform
there too, had not cruel custom prohibited its appearance
except on duty.  So the uniform remained at Portsmouth,
and Edmund conjectured that before Fanny had any chance
of seeing it, all its own freshness and all the freshness
of its wearer's feelings must be worn away.  It would be sunk
into a badge of disgrace; for what can be more unbecoming,
or more worthless, than the uniform of a lieutenant,
who has been a lieutenant a year or two, and sees
others made commanders before him?  So reasoned Edmund,
till his father made him the confidant of a scheme which
placed Fanny's chance of seeing the second lieutenant
of H.M.S. Thrush in all his glory in another light.

This scheme was that she should accompany her brother
back to Portsmouth, and spend a little time with her
own family.  It had occurred to Sir Thomas, in one of his
dignified musings, as a right and desirable measure;
but before he absolutely made up his mind, he consulted
his son.  Edmund considered it every way, and saw nothing
but what was right.  The thing was good in itself,
and could not be done at a better time; and he had no doubt
of it being highly agreeable to Fanny.  This was enough
to determine Sir Thomas; and a decisive ``then so it shall be''
closed that stage of the business; Sir Thomas retiring
from it with some feelings of satisfaction, and views
of good over and above what he had communicated to his son;
for his prime motive in sending her away had very little
to do with the propriety of her seeing her parents again,
and nothing at all with any idea of making her happy.
He certainly wished her to go willingly, but he as certainly
wished her to be heartily sick of home before her visit ended;
and that a little abstinence from the elegancies and luxuries
of Mansfield Park would bring her mind into a sober state,
and incline her to a juster estimate of the value
of that home of greater permanence, and equal comfort,
of which she had the offer.

It was a medicinal project upon his niece's understanding,
which he must consider as at present diseased.
A residence of eight or nine years in the abode of wealth
and plenty had a little disordered her powers of comparing
and judging.  Her father's house would, in all probability,
teach her the value of a good income; and he trusted that
she would be the wiser and happier woman, all her life,
for the experiment he had devised.

Had Fanny been at all addicted to raptures, she must have
had a strong attack of them when she first understood
what was intended, when her uncle first made her the offer
of visiting the parents, and brothers, and sisters,
from whom she had been divided almost half her life;
of returning for a couple of months to the scenes of
her infancy, with William for the protector and companion
of her journey, and the certainty of continuing to see
William to the last hour of his remaining on land.
Had she ever given way to bursts of delight, it must have
been then, for she was delighted, but her happiness was
of a quiet, deep, heart-swelling sort; and though never
a great talker, she was always more inclined to silence
when feeling most strongly.  At the moment she could
only thank and accept.  Afterwards, when familiarised
with the visions of enjoyment so suddenly opened, she could
speak more largely to William and Edmund of what she felt;
but still there were emotions of tenderness that could
not be clothed in words.  The remembrance of all her
earliest pleasures, and of what she had suffered in being
torn from them, came over her with renewed strength,
and it seemed as if to be at home again would heal
every pain that had since grown out of the separation.
To be in the centre of such a circle, loved by so many,
and more loved by all than she had ever been before;
to feel affection without fear or restraint; to feel
herself the equal of those who surrounded her; to be at
peace from all mention of the Crawfords, safe from every
look which could be fancied a reproach on their account.
This was a prospect to be dwelt on with a fondness that could
be but half acknowledged.

Edmund, too---to be two months from \emph{him} (and perhaps
she might be allowed to make her absence three)
must do her good.  At a distance, unassailed by his looks
or his kindness, and safe from the perpetual irritation
of knowing his heart, and striving to avoid his confidence,
she should be able to reason herself into a properer state;
she should be able to think of him as in London,
and arranging everything there, without wretchedness.
What might have been hard to bear at Mansfield was to become
a slight evil at Portsmouth.

The only drawback was the doubt of her aunt Bertram's being
comfortable without her.  She was of use to no one else;
but \emph{there} she might be missed to a degree that she did
not like to think of; and that part of the arrangement
was, indeed, the hardest for Sir Thomas to accomplish,
and what only \emph{he} could have accomplished at all.

But he was master at Mansfield Park.  When he had really
resolved on any measure, he could always carry it through;
and now by dint of long talking on the subject,
explaining and dwelling on the duty of Fanny's sometimes
seeing her family, he did induce his wife to let her go;
obtaining it rather from submission, however, than conviction,
for Lady Bertram was convinced of very little more than
that Sir Thomas thought Fanny ought to go, and therefore
that she must.  In the calmness of her own dressing-room,
in the impartial flow of her own meditations, unbiassed by
his bewildering statements, she could not acknowledge any
necessity for Fanny's ever going near a father and mother
who had done without her so long, while she was so useful
to herself And as to the not missing her, which under
Mrs.\ Norris's discussion was the point attempted to be proved,
she set herself very steadily against admitting any such thing.

Sir Thomas had appealed to her reason, conscience, and dignity.
He called it a sacrifice, and demanded it of her goodness
and self-command as such.  But Mrs.\ Norris wanted to persuade
her that Fanny could be very well spared---\emph{she} being
ready to give up all her own time to her as requested---%
and, in short, could not really be wanted or missed.

``That may be, sister,'' was all Lady Bertram's reply.
``I dare say you are very right; but I am sure I shall miss
her very much.''

The next step was to communicate with Portsmouth.  Fanny wrote
to offer herself; and her mother's answer, though short,
was so kind---a few simple lines expressed so natural and
motherly a joy in the prospect of seeing her child again,
as to confirm all the daughter's views of happiness in
being with her---convincing her that she should now find
a warm and affectionate friend in the ``mama'' who had
certainly shewn no remarkable fondness for her formerly;
but this she could easily suppose to have been her own
fault or her own fancy.  She had probably alienated love
by the helplessness and fretfulness of a fearful temper,
or been unreasonable in wanting a larger share than
any one among so many could deserve.  Now, when she
knew better how to be useful, and how to forbear,
and when her mother could be no longer occupied by the
incessant demands of a house full of little children,
there would be leisure and inclination for every comfort,
and they should soon be what mother and daughter ought
to be to each other.

William was almost as happy in the plan as his sister.
It would be the greatest pleasure to him to have her there
to the last moment before he sailed, and perhaps find
her there still when he came in from his first cruise.
And besides, he wanted her so very much to see the Thrush
before she went out of harbour---the Thrush was certainly
the finest sloop in the service---and there were several
improvements in the dockyard, too, which he quite longed to
shew her.

He did not scruple to add that her being at home
for a while would be a great advantage to everybody.

``I do not know how it is,'' said he; ``but we seem to want
some of your nice ways and orderliness at my father's. The
house is always in confusion.  You will set things going
in a better way, I am sure.  You will tell my mother how it
all ought to be, and you will be so useful to Susan, and you
will teach Betsey, and make the boys love and mind you.
How right and comfortable it will all be!''

By the time Mrs.\ Price's answer arrived, there remained
but a very few days more to be spent at Mansfield;
and for part of one of those days the young travellers
were in a good deal of alarm on the subject of their
journey, for when the mode of it came to be talked of,
and Mrs.\ Norris found that all her anxiety to save her
brother-in-law's money was vain, and that in spite of her
wishes and hints for a less expensive conveyance of Fanny,
they were to travel post; when she saw Sir Thomas actually
give William notes for the purpose, she was struck with
the idea of there being room for a third in the carriage,
and suddenly seized with a strong inclination to go
with them, to go and see her poor dear sister Price.
She proclaimed her thoughts.  She must say that she
had more than half a mind to go with the young people;
it would be such an indulgence to her; she had not seen
her poor dear sister Price for more than twenty years;
and it would be a help to the young people in their journey
to have her older head to manage for them; and she could
not help thinking her poor dear sister Price would feel it
very unkind of her not to come by such an opportunity.

William and Fanny were horror-struck at the idea.

All the comfort of their comfortable journey would
be destroyed at once.  With woeful countenances they
looked at each other.  Their suspense lasted an hour
or two.  No one interfered to encourage or dissuade.
Mrs.\ Norris was left to settle the matter by herself;
and it ended, to the infinite joy of her nephew and niece,
in the recollection that she could not possibly be spared
from Mansfield Park at present; that she was a great deal
too necessary to Sir Thomas and Lady Bertram for her to be
able to answer it to herself to leave them even for a week,
and therefore must certainly sacrifice every other pleasure
to that of being useful to them.

It had, in fact, occurred to her, that though taken
to Portsmouth for nothing, it would be hardly possible
for her to avoid paying her own expenses back again.
So her poor dear sister Price was left to all the
disappointment of her missing such an opportunity,
and another twenty years' absence, perhaps, begun.

Edmund's plans were affected by this Portsmouth journey,
this absence of Fanny's. He too had a sacrifice to make
to Mansfield Park as well as his aunt.  He had intended,
about this time, to be going to London; but he could
not leave his father and mother just when everybody else
of most importance to their comfort was leaving them;
and with an effort, felt but not boasted of, he delayed
for a week or two longer a journey which he was looking
forward to with the hope of its fixing his happiness
for ever.

He told Fanny of it.  She knew so much already,
that she must know everything.  It made the substance
of one other confidential discourse about Miss Crawford;
and Fanny was the more affected from feeling it to be
the last time in which Miss Crawford's name would ever
be mentioned between them with any remains of liberty.
Once afterwards she was alluded to by him.  Lady Bertram had
been telling her niece in the evening to write to her soon
and often, and promising to be a good correspondent herself;
and Edmund, at a convenient moment, then added in a whisper,
``And \emph{I} shall write to you, Fanny, when I have anything
worth writing about, anything to say that I think you
will like to hear, and that you will not hear so soon
from any other quarter.''  Had she doubted his meaning
while she listened, the glow in his face, when she looked
up at him, would have been decisive.

For this letter she must try to arm herself.  That a
letter from Edmund should be a subject of terror!
She began to feel that she had not yet gone through all
the changes of opinion and sentiment which the progress
of time and variation of circumstances occasion in this
world of changes.  The vicissitudes of the human mind
had not yet been exhausted by her.

Poor Fanny! though going as she did willingly and eagerly,
the last evening at Mansfield Park must still
be wretchedness.  Her heart was completely sad at parting.
She had tears for every room in the house, much more
for every beloved inhabitant.  She clung to her aunt,
because she would miss her; she kissed the hand of her
uncle with struggling sobs, because she had displeased him;
and as for Edmund, she could neither speak, nor look,
nor think, when the last moment came with \emph{him}; and it
was not till it was over that she knew he was giving
her the affectionate farewell of a brother.

All this passed overnight, for the journey was to
begin very early in the morning; and when the small,
diminished party met at breakfast, William and Fanny
were talked of as already advanced one stage.



\chapter{Chapter 38}

\gintro{The novelty} of travelling, and the happiness of being
with William, soon produced their natural effect on
Fanny's spirits, when Mansfield Park was fairly left behind;
and by the time their first stage was ended, and they
were to quit Sir Thomas's carriage, she was able to take
leave of the old coachman, and send back proper messages,
with cheerful looks.

Of pleasant talk between the brother and sister there
was no end.  Everything supplied an amusement to the high
glee of William's mind, and he was full of frolic and
joke in the intervals of their higher-toned subjects,
all of which ended, if they did not begin, in praise
of the Thrush, conjectures how she would be employed,
schemes for an action with some superior force,
which (supposing the first lieutenant out of the way,
and William was not very merciful to the first lieutenant)
was to give himself the next step as soon as possible,
or speculations upon prize-money, which was to be generously
distributed at home, with only the reservation of enough
to make the little cottage comfortable, in which he and Fanny
were to pass all their middle and later life together.

Fanny's immediate concerns, as far as they involved
Mr.\ Crawford, made no part of their conversation.
William knew what had passed, and from his heart lamented
that his sister's feelings should be so cold towards a man
whom he must consider as the first of human characters;
but he was of an age to be all for love, and therefore
unable to blame; and knowing her wish on the subject,
he would not distress her by the slightest allusion.

She had reason to suppose herself not yet forgotten by
Mr.\ Crawford.  She had heard repeatedly from his sister within
the three weeks which had passed since their leaving Mansfield,
and in each letter there had been a few lines from himself,
warm and determined like his speeches.  It was a correspondence
which Fanny found quite as unpleasant as she had feared.
Miss Crawford's style of writing, lively and affectionate,
was itself an evil, independent of what she was thus
forced into reading from the brother's pen, for Edmund
would never rest till she had read the chief of the letter
to him; and then she had to listen to his admiration
of her language, and the warmth of her attachments.
There had, in fact, been so much of message, of allusion,
of recollection, so much of Mansfield in every letter,
that Fanny could not but suppose it meant for him to hear;
and to find herself forced into a purpose of that kind,
compelled into a correspondence which was bringing her
the addresses of the man she did not love, and obliging
her to administer to the adverse passion of the man she did,
was cruelly mortifying.  Here, too, her present removal
promised advantage.  When no longer under the same roof
with Edmund, she trusted that Miss Crawford would have no
motive for writing strong enough to overcome the trouble,
and that at Portsmouth their correspondence would dwindle
into nothing.

With such thoughts as these, among ten hundred others,
Fanny proceeded in her journey safely and cheerfully,
and as expeditiously as could rationally be hoped
in the dirty month of February.  They entered Oxford,
but she could take only a hasty glimpse of Edmund's
college as they passed along, and made no stop anywhere
till they reached Newbury, where a comfortable meal,
uniting dinner and supper, wound up the enjoyments and
fatigues of the day.

The next morning saw them off again at an early hour;
and with no events, and no delays, they regularly advanced,
and were in the environs of Portsmouth while there was yet
daylight for Fanny to look around her, and wonder at the
new buildings.  They passed the drawbridge, and entered
the town; and the light was only beginning to fail as,
guided by William's powerful voice, they were rattled
into a narrow street, leading from the High Street,
and drawn up before the door of a small house now inhabited
by Mr.\ Price.

Fanny was all agitation and flutter; all hope and apprehension.
The moment they stopped, a trollopy-looking maidservant,
seemingly in waiting for them at the door, stepped forward,
and more intent on telling the news than giving them any help,
immediately began with, ``The Thrush is gone out of harbour,
please sir, and one of the officers has been here to---''
She was interrupted by a fine tall boy of eleven years old,
who, rushing out of the house, pushed the maid aside,
and while William was opening the chaise-door himself,
called out, ``You are just in time.  We have been looking
for you this half-hour. The Thrush went out of harbour
this morning.  I saw her.  It was a beautiful sight.
And they think she will have her orders in a day or two.
And Mr.\ Campbell was here at four o'clock to ask for you:
he has got one of the Thrush's boats, and is going off
to her at six, and hoped you would be here in time to go
with him.''

A stare or two at Fanny, as William helped her out of
the carriage, was all the voluntary notice which this
brother bestowed; but he made no objection to her
kissing him, though still entirely engaged in detailing
farther particulars of the Thrush's going out of harbour,
in which he had a strong right of interest, being to
commence his career of seamanship in her at this very time.

Another moment and Fanny was in the narrow entrance-passage
of the house, and in her mother's arms, who met her
there with looks of true kindness, and with features
which Fanny loved the more, because they brought her aunt
Bertram's before her, and there were her two sisters:
Susan, a well-grown fine girl of fourteen, and Betsey,
the youngest of the family, about five---both glad to see
her in their way, though with no advantage of manner
in receiving her.  But manner Fanny did not want.
Would they but love her, she should be satisfied.

She was then taken into a parlour, so small that her
first conviction was of its being only a passage-room
to something better, and she stood for a moment expecting
to be invited on; but when she saw there was no other door,
and that there were signs of habitation before her,
she called back her thoughts, reproved herself, and grieved
lest they should have been suspected.  Her mother,
however, could not stay long enough to suspect anything.
She was gone again to the street-door, to welcome William.
``Oh! my dear William, how glad I am to see you.
But have you heard about the Thrush?  She is gone out of
harbour already; three days before we had any thought of it;
and I do not know what I am to do about Sam's things,
they will never be ready in time; for she may have her orders
to-morrow, perhaps.  It takes me quite unawares.  And now
you must be off for Spithead too.  Campbell has been here,
quite in a worry about you; and now what shall we do?
I thought to have had such a comfortable evening with you,
and here everything comes upon me at once.''

Her son answered cheerfully, telling her that everything
was always for the best; and making light of his own
inconvenience in being obliged to hurry away so soon.

``To be sure, I had much rather she had stayed in harbour,
that I might have sat a few hours with you in comfort;
but as there is a boat ashore, I had better go off at once,
and there is no help for it.  Whereabouts does the Thrush
lay at Spithead?  Near the Canopus?  But no matter;
here's Fanny in the parlour, and why should we stay in
the passage?  Come, mother, you have hardly looked at your
own dear Fanny yet.''

In they both came, and Mrs.\ Price having kindly kissed
her daughter again, and commented a little on her growth,
began with very natural solicitude to feel for their
fatigues and wants as travellers.

``Poor dears! how tired you must both be! and now,
what will you have?  I began to think you would never come.
Betsey and I have been watching for you this half-hour.
And when did you get anything to eat?  And what would you
like to have now?  I could not tell whether you would be
for some meat, or only a dish of tea, after your journey,
or else I would have got something ready.  And now I
am afraid Campbell will be here before there is time
to dress a steak, and we have no butcher at hand.
It is very inconvenient to have no butcher in the street.
We were better off in our last house.  Perhaps you would
like some tea as soon as it can be got.''

They both declared they should prefer it to anything.
``Then, Betsey, my dear, run into the kitchen and see if Rebecca
has put the water on; and tell her to bring in the tea-things
as soon as she can.  I wish we could get the bell mended;
but Betsey is a very handy little messenger.''

Betsey went with alacrity, proud to shew her abilities
before her fine new sister.

``Dear me!'' continued the anxious mother, ``what a sad
fire we have got, and I dare say you are both starved
with cold.  Draw your chair nearer, my dear.  I cannot
think what Rebecca has been about.  I am sure I told her
to bring some coals half an hour ago.  Susan, you should
have taken care of the fire.''

``I was upstairs, mama, moving my things,'' said Susan,
in a fearless, self-defending tone, which startled Fanny.
``You know you had but just settled that my sister Fanny
and I should have the other room; and I could not get
Rebecca to give me any help.''

Farther discussion was prevented by various bustles:
first, the driver came to be paid; then there was a squabble
between Sam and Rebecca about the manner of carrying up
his sister's trunk, which he would manage all his own way;
and lastly, in walked Mr.\ Price himself, his own loud
voice preceding him, as with something of the oath kind
he kicked away his son's port-manteau and his daughter's
bandbox in the passage, and called out for a candle;
no candle was brought, however, and he walked into the room.

Fanny with doubting feelings had risen to meet him,
but sank down again on finding herself undistinguished
in the dusk, and unthought of.  With a friendly shake
of his son's hand, and an eager voice, he instantly began---%
``Ha! welcome back, my boy.  Glad to see you.  Have you heard
the news?  The Thrush went out of harbour this morning.
Sharp is the word, you see!  By G---, you are just in time!
The doctor has been here inquiring for you:  he has got
one of the boats, and is to be off for Spithead by six,
so you had better go with him.  I have been to Turner's
about your mess; it is all in a way to be done.
I should not wonder if you had your orders to-morrow:
but you cannot sail with this wind, if you are to cruise
to the westward; and Captain Walsh thinks you will certainly
have a cruise to the westward, with the Elephant.
By G---, I wish you may!  But old Scholey was saying,
just now, that he thought you would be sent first to
the Texel.  Well, well, we are ready, whatever happens.
But by G---, you lost a fine sight by not being here
in the morning to see the Thrush go out of harbour!
I would not have been out of the way for a thousand pounds.
Old Scholey ran in at breakfast-time, to say she had
slipped her moorings and was coming out, I jumped up,
and made but two steps to the platform.  If ever there
was a perfect beauty afloat, she is one; and there she lays
at Spithead, and anybody in England would take her for an
eight-and-twenty. I was upon the platform two hours this
afternoon looking at her.  She lays close to the Endymion,
between her and the Cleopatra, just to the eastward of the
sheer hulk.''

``Ha!'' cried William, ``\emph{that's} just where I should have
put her myself.  It's the best berth at Spithead.
But here is my sister, sir; here is Fanny,'' turning and
leading her forward; ``it is so dark you do not see her.''

With an acknowledgment that he had quite forgot her,
Mr.\ Price now received his daughter; and having given
her a cordial hug, and observed that she was grown into
a woman, and he supposed would be wanting a husband soon,
seemed very much inclined to forget her again.
Fanny shrunk back to her seat, with feelings sadly
pained by his language and his smell of spirits;
and he talked on only to his son, and only of the Thrush,
though William, warmly interested as he was in that subject,
more than once tried to make his father think of Fanny,
and her long absence and long journey.

After sitting some time longer, a candle was obtained;
but as there was still no appearance of tea, nor, from
Betsey's reports from the kitchen, much hope of any under
a considerable period, William determined to go and change
his dress, and make the necessary preparations for his removal
on board directly, that he might have his tea in comfort
afterwards.

As he left the room, two rosy-faced boys, ragged and dirty,
about eight and nine years old, rushed into it just released
from school, and coming eagerly to see their sister,
and tell that the Thrush was gone out of harbour;
Tom and Charles.  Charles had been born since Fanny's
going away, but Tom she had often helped to nurse,
and now felt a particular pleasure in seeing again.
Both were kissed very tenderly, but Tom she wanted
to keep by her, to try to trace the features of the baby
she had loved, and talked to, of his infant preference
of herself.  Tom, however, had no mind for such treatment:
he came home not to stand and be talked to, but to run about
and make a noise; and both boys had soon burst from her,
and slammed the parlour-door till her temples ached.

She had now seen all that were at home; there remained
only two brothers between herself and Susan,
one of whom was a clerk in a public office in London,
and the other midshipman on board an Indiaman.
But though she had \emph{seen} all the members of the family,
she had not yet \emph{heard} all the noise they could make.
Another quarter of an hour brought her a great deal more.
William was soon calling out from the landing-place
of the second story for his mother and for Rebecca.
He was in distress for something that he had left there,
and did not find again.  A key was mislaid, Betsey accused
of having got at his new hat, and some slight, but essential
alteration of his uniform waistcoat, which he had been
promised to have done for him, entirely neglected.

Mrs.\ Price, Rebecca, and Betsey all went up to defend themselves,
all talking together, but Rebecca loudest, and the job
was to be done as well as it could in a great hurry;
William trying in vain to send Betsey down again, or keep
her from being troublesome where she was; the whole of which,
as almost every door in the house was open, could be plainly
distinguished in the parlour, except when drowned at intervals
by the superior noise of Sam, Tom, and Charles chasing
each other up and down stairs, and tumbling about and hallooing.

Fanny was almost stunned.  The smallness of the house
and thinness of the walls brought everything so close
to her, that, added to the fatigue of her journey, and all
her recent agitation, she hardly knew how to bear it.
\emph{Within} the room all was tranquil enough, for Susan having
disappeared with the others, there were soon only her father
and herself remaining; and he, taking out a newspaper,
the accustomary loan of a neighbour, applied himself to
studying it, without seeming to recollect her existence.
The solitary candle was held between himself and the paper,
without any reference to her possible convenience;
but she had nothing to do, and was glad to have the light
screened from her aching head, as she sat in bewildered,
broken, sorrowful contemplation.

She was at home.  But, alas! it was not such a home,
she had not such a welcome, as---she checked herself;
she was unreasonable.  What right had she to be of importance
to her family?  She could have none, so long lost sight of!
William's concerns must be dearest, they always had been,
and he had every right.  Yet to have so little said
or asked about herself, to have scarcely an inquiry made
after Mansfield!  It did pain her to have Mansfield forgotten;
the friends who had done so much---the dear, dear friends!
But here, one subject swallowed up all the rest.
Perhaps it must be so.  The destination of the Thrush
must be now preeminently interesting.  A day or two
might shew the difference.  \emph{She} only was to blame.
Yet she thought it would not have been so at Mansfield.
No, in her uncle's house there would have been a
consideration of times and seasons, a regulation of subject,
a propriety, an attention towards everybody which there
was not here.

The only interruption which thoughts like these received
for nearly half an hour was from a sudden burst of her
father's, not at all calculated to compose them.  At a more
than ordinary pitch of thumping and hallooing in the passage,
he exclaimed, ``Devil take those young dogs!  How they are
singing out!  Ay, Sam's voice louder than all the rest!
That boy is fit for a boatswain.  Holla, you there!
Sam, stop your confounded pipe, or I shall be after you.''

This threat was so palpably disregarded, that though
within five minutes afterwards the three boys all burst
into the room together and sat down, Fanny could not
consider it as a proof of anything more than their being
for the time thoroughly fagged, which their hot faces
and panting breaths seemed to prove, especially as they
were still kicking each other's shins, and hallooing
out at sudden starts immediately under their father's eye.

The next opening of the door brought something more welcome:
it was for the tea-things, which she had begun almost
to despair of seeing that evening.  Susan and an
attendant girl, whose inferior appearance informed Fanny,
to her great surprise, that she had previously seen the
upper servant, brought in everything necessary for the meal;
Susan looking, as she put the kettle on the fire and glanced
at her sister, as if divided between the agreeable triumph
of shewing her activity and usefulness, and the dread
of being thought to demean herself by such an office.
``She had been into the kitchen,'' she said, ``to hurry Sally
and help make the toast, and spread the bread and butter,
or she did not know when they should have got tea,
and she was sure her sister must want something after
her journey.''

Fanny was very thankful.  She could not but own that she
should be very glad of a little tea, and Susan immediately
set about making it, as if pleased to have the employment
all to herself; and with only a little unnecessary bustle,
and some few injudicious attempts at keeping her brothers
in better order than she could, acquitted herself very well.
Fanny's spirit was as much refreshed as her body; her head
and heart were soon the better for such well-timed kindness.
Susan had an open, sensible countenance; she was like William,
and Fanny hoped to find her like him in disposition
and goodwill towards herself.

In this more placid state of things William reentered,
followed not far behind by his mother and Betsey.
He, complete in his lieutenant's uniform, looking and
moving all the taller, firmer, and more graceful for it,
and with the happiest smile over his face, walked up directly
to Fanny, who, rising from her seat, looked at him for a
moment in speechless admiration, and then threw her arms
round his neck to sob out her various emotions of pain and
pleasure.

Anxious not to appear unhappy, she soon recovered herself;
and wiping away her tears, was able to notice and admire
all the striking parts of his dress; listening with reviving
spirits to his cheerful hopes of being on shore some part
of every day before they sailed, and even of getting
her to Spithead to see the sloop.

The next bustle brought in Mr.\ Campbell, the surgeon
of the Thrush, a very well-behaved young man, who came
to call for his friend, and for whom there was with some
contrivance found a chair, and with some hasty washing of
the young tea-maker's, a cup and saucer; and after another
quarter of an hour of earnest talk between the gentlemen,
noise rising upon noise, and bustle upon bustle, men and
boys at last all in motion together, the moment came
for setting off; everything was ready, William took leave,
and all of them were gone; for the three boys, in spite
of their mother's entreaty, determined to see their brother
and Mr.\ Campbell to the sally-port; and Mr.\ Price walked
off at the same time to carry back his neighbour's newspaper.

Something like tranquillity might now be hoped for;
and accordingly, when Rebecca had been prevailed on
to carry away the tea-things, and Mrs.\ Price had walked
about the room some time looking for a shirt-sleeve, which
Betsey at last hunted out from a drawer in the kitchen,
the small party of females were pretty well composed,
and the mother having lamented again over the impossibility
of getting Sam ready in time, was at leisure to think
of her eldest daughter and the friends she had come from.

A few inquiries began:  but one of the earliest---``How did
sister Bertram manage about her servants?''  ``Was she
as much plagued as herself to get tolerable servants?''---%
soon led her mind away from Northamptonshire, and fixed it
on her own domestic grievances, and the shocking character
of all the Portsmouth servants, of whom she believed her
own two were the very worst, engrossed her completely.
The Bertrams were all forgotten in detailing the faults
of Rebecca, against whom Susan had also much to depose,
and little Betsey a great deal more, and who did seem
so thoroughly without a single recommendation, that Fanny
could not help modestly presuming that her mother meant
to part with her when her year was up.

``Her year!'' cried Mrs.\ Price; ``I am sure I hope I
shall be rid of her before she has staid a year,
for that will not be up till November.  Servants are come
to such a pass, my dear, in Portsmouth, that it is quite
a miracle if one keeps them more than half a year.
I have no hope of ever being settled; and if I was to
part with Rebecca, I should only get something worse.
And yet I do not think I am a very difficult mistress
to please; and I am sure the place is easy enough,
for there is always a girl under her, and I often do half
the work myself.''

Fanny was silent; but not from being convinced that there
might not be a remedy found for some of these evils.
As she now sat looking at Betsey, she could not but think
particularly of another sister, a very pretty little girl,
whom she had left there not much younger when she went
into Northamptonshire, who had died a few years afterwards.
There had been something remarkably amiable about her.
Fanny in those early days had preferred her to Susan;
and when the news of her death had at last reached Mansfield,
had for a short time been quite afflicted.  The sight
of Betsey brought the image of little Mary back again,
but she would not have pained her mother by alluding to her
for the world.  While considering her with these ideas,
Betsey, at a small distance, was holding out something to
catch her eyes, meaning to screen it at the same time from
Susan's.

``What have you got there, my love?'' said Fanny;
``come and shew it to me.''

It was a silver knife.  Up jumped Susan, claiming it
as her own, and trying to get it away; but the child ran
to her mother's protection, and Susan could only reproach,
which she did very warmly, and evidently hoping to
interest Fanny on her side.  ``It was very hard that she
was not to have her \emph{own} knife; it was her own knife;
little sister Mary had left it to her upon her deathbed,
and she ought to have had it to keep herself long ago.
But mama kept it from her, and was always letting Betsey
get hold of it; and the end of it would be that Betsey
would spoil it, and get it for her own, though mama
had \emph{promised} her that Betsey should not have it in her
own hands.''

Fanny was quite shocked.  Every feeling of duty,
honour, and tenderness was wounded by her sister's
speech and her mother's reply.

``Now, Susan,'' cried Mrs.\ Price, in a complaining voice,
``now, how can you be so cross?  You are always quarrelling
about that knife.  I wish you would not be so quarrelsome.
Poor little Betsey; how cross Susan is to you!  But you
should not have taken it out, my dear, when I sent you
to the drawer.  You know I told you not to touch it,
because Susan is so cross about it.  I must hide it
another time, Betsey.  Poor Mary little thought it would
be such a bone of contention when she gave it me to keep,
only two hours before she died.  Poor little soul! she could
but just speak to be heard, and she said so prettily, `Let sister
Susan have my knife, mama, when I am dead and buried.'
Poor little dear! she was so fond of it, Fanny, that she
would have it lay by her in bed, all through her illness.
It was the gift of her good godmother, old Mrs.\ Admiral
Maxwell, only six weeks before she was taken for death.
Poor little sweet creature!  Well, she was taken away
from evil to come.  My own Betsey'' (fondling her),
``\emph{you} have not the luck of such a good godmother.
Aunt Norris lives too far off to think of such little
people as you.''

Fanny had indeed nothing to convey from aunt Norris,
but a message to say she hoped that her god-daughter
was a good girl, and learnt her book.  There had been
at one moment a slight murmur in the drawing-room
at Mansfield Park about sending her a prayer-book;
but no second sound had been heard of such a purpose.
Mrs.\ Norris, however, had gone home and taken down two
old prayer-books of her husband with that idea; but,
upon examination, the ardour of generosity went off.
One was found to have too small a print for a child's eyes,
and the other to be too cumbersome for her to carry about.

Fanny, fatigued and fatigued again, was thankful to accept
the first invitation of going to bed; and before Betsey
had finished her cry at being allowed to sit up only one
hour extraordinary in honour of sister, she was off,
leaving all below in confusion and noise again; the boys
begging for toasted cheese, her father calling out for his
rum and water, and Rebecca never where she ought to be.

There was nothing to raise her spirits in the confined
and scantily furnished chamber that she was to share
with Susan.  The smallness of the rooms above and below,
indeed, and the narrowness of the passage and staircase,
struck her beyond her imagination.  She soon learned to think
with respect of her own little attic at Mansfield Park,
in \emph{that} house reckoned too small for anybody's comfort.



\chapter{Chapter 39}

\gintro{Could Sir Thomas} have seen all his niece's feelings,
when she wrote her first letter to her aunt, he would
not have despaired; for though a good night's rest,
a pleasant morning, the hope of soon seeing William again,
and the comparatively quiet state of the house, from Tom
and Charles being gone to school, Sam on some project of
his own, and her father on his usual lounges, enabled her
to express herself cheerfully on the subject of home,
there were still, to her own perfect consciousness,
many drawbacks suppressed.  Could he have seen only half
that she felt before the end of a week, he would have
thought Mr.\ Crawford sure of her, and been delighted with
his own sagacity.

Before the week ended, it was all disappointment.
In the first place, William was gone.  The Thrush
had had her orders, the wind had changed, and he was
sailed within four days from their reaching Portsmouth;
and during those days she had seen him only twice,
in a short and hurried way, when he had come ashore
on duty.  There had been no free conversation, no walk
on the ramparts, no visit to the dockyard, no acquaintance
with the Thrush, nothing of all that they had planned
and depended on.  Everything in that quarter failed her,
except William's affection.  His last thought on leaving
home was for her.  He stepped back again to the door
to say, ``Take care of Fanny, mother.  She is tender,
and not used to rough it like the rest of us.  I charge you,
take care of Fanny.''

William was gone:  and the home he had left her in was,
Fanny could not conceal it from herself, in almost every
respect the very reverse of what she could have wished.
It was the abode of noise, disorder, and impropriety.
Nobody was in their right place, nothing was done as it ought
to be.  She could not respect her parents as she had hoped.
On her father, her confidence had not been sanguine, but he
was more negligent of his family, his habits were worse,
and his manners coarser, than she had been prepared for.
He did not want abilities but he had no curiosity,
and no information beyond his profession; he read only
the newspaper and the navy-list; he talked only of
the dockyard, the harbour, Spithead, and the Motherbank;
he swore and he drank, he was dirty and gross.
She had never been able to recall anything approaching
to tenderness in his former treatment of herself.
There had remained only a general impression of roughness
and loudness; and now he scarcely ever noticed her,
but to make her the object of a coarse joke.

Her disappointment in her mother was greater:
\emph{there} she had hoped much, and found almost nothing.
Every flattering scheme of being of consequence to her
soon fell to the ground.  Mrs.\ Price was not unkind;
but, instead of gaining on her affection and confidence,
and becoming more and more dear, her daughter never met
with greater kindness from her than on the first day of
her arrival.  The instinct of nature was soon satisfied,
and Mrs.\ Price's attachment had no other source.
Her heart and her time were already quite full;
she had neither leisure nor affection to bestow on Fanny.
Her daughters never had been much to her.  She was fond
of her sons, especially of William, but Betsey was the first
of her girls whom she had ever much regarded.  To her she
was most injudiciously indulgent.  William was her pride;
Betsey her darling; and John, Richard, Sam, Tom, and Charles
occupied all the rest of her maternal solicitude, alternately
her worries and her comforts.  These shared her heart:
her time was given chiefly to her house and her servants.
Her days were spent in a kind of slow bustle; all was busy
without getting on, always behindhand and lamenting it,
without altering her ways; wishing to be an economist,
without contrivance or regularity; dissatisfied with
her servants, without skill to make them better,
and whether helping, or reprimanding, or indulging them,
without any power of engaging their respect.

Of her two sisters, Mrs.\ Price very much more resembled Lady
Bertram than Mrs.\ Norris.  She was a manager by necessity,
without any of Mrs.\ Norris's inclination for it, or any
of her activity.  Her disposition was naturally easy
and indolent, like Lady Bertram's; and a situation of similar
affluence and do-nothingness would have been much more
suited to her capacity than the exertions and self-denials
of the one which her imprudent marriage had placed her in.
She might have made just as good a woman of consequence
as Lady Bertram, but Mrs.\ Norris would have been a more
respectable mother of nine children on a small income.

Much of all this Fanny could not but be sensible of.
She might scruple to make use of the words, but she
must and did feel that her mother was a partial,
ill-judging parent, a dawdle, a slattern, who neither taught
nor restrained her children, whose house was the scene
of mismanagement and discomfort from beginning to end,
and who had no talent, no conversation, no affection
towards herself; no curiosity to know her better,
no desire of her friendship, and no inclination for her
company that could lessen her sense of such feelings.

Fanny was very anxious to be useful, and not to appear above
her home, or in any way disqualified or disinclined, by her
foreign education, from contributing her help to its comforts,
and therefore set about working for Sam immediately;
and by working early and late, with perseverance and
great despatch, did so much that the boy was shipped
off at last, with more than half his linen ready.
She had great pleasure in feeling her usefulness, but could
not conceive how they would have managed without her.

Sam, loud and overbearing as he was, she rather regretted
when he went, for he was clever and intelligent, and glad
to be employed in any errand in the town; and though
spurning the remonstrances of Susan, given as they were,
though very reasonable in themselves, with ill-timed
and powerless warmth, was beginning to be influenced
by Fanny's services and gentle persuasions; and she found
that the best of the three younger ones was gone in him:
Tom and Charles being at least as many years as they were
his juniors distant from that age of feeling and reason,
which might suggest the expediency of making friends,
and of endeavouring to be less disagreeable.  Their sister
soon despaired of making the smallest impression on \emph{them};
they were quite untameable by any means of address which she
had spirits or time to attempt.  Every afternoon brought
a return of their riotous games all over the house; and she
very early learned to sigh at the approach of Saturday's
constant half-holiday.

Betsey, too, a spoiled child, trained up to think the
alphabet her greatest enemy, left to be with the servants
at her pleasure, and then encouraged to report any evil
of them, she was almost as ready to despair of being
able to love or assist; and of Susan's temper she had
many doubts.  Her continual disagreements with her mother,
her rash squabbles with Tom and Charles, and petulance
with Betsey, were at least so distressing to Fanny that,
though admitting they were by no means without provocation,
she feared the disposition that could push them to such
length must be far from amiable, and from affording
any repose to herself.

Such was the home which was to put Mansfield out of
her head, and teach her to think of her cousin Edmund with
moderated feelings.  On the contrary, she could think of
nothing but Mansfield, its beloved inmates, its happy ways.
Everything where she now was in full contrast to it.
The elegance, propriety, regularity, harmony, and perhaps,
above all, the peace and tranquillity of Mansfield,
were brought to her remembrance every hour of the day,
by the prevalence of everything opposite to them \emph{here}.

The living in incessant noise was, to a frame and temper
delicate and nervous like Fanny's, an evil which no
superadded elegance or harmony could have entirely
atoned for.  It was the greatest misery of all.
At Mansfield, no sounds of contention, no raised voice,
no abrupt bursts, no tread of violence, was ever heard;
all proceeded in a regular course of cheerful orderliness;
everybody had their due importance; everybody's feelings
were consulted.  If tenderness could be ever supposed wanting,
good sense and good breeding supplied its place; and as to
the little irritations sometimes introduced by aunt Norris,
they were short, they were trifling, they were as a drop
of water to the ocean, compared with the ceaseless
tumult of her present abode.  Here everybody was noisy,
every voice was loud (excepting, perhaps, her mother's,
which resembled the soft monotony of Lady Bertram's,
only worn into fretfulness). Whatever was wanted was
hallooed for, and the servants hallooed out their excuses
from the kitchen.  The doors were in constant banging,
the stairs were never at rest, nothing was done without
a clatter, nobody sat still, and nobody could command
attention when they spoke.

In a review of the two houses, as they appeared to her
before the end of a week, Fanny was tempted to apply
to them Dr.\ Johnson's celebrated judgment as to matrimony
and celibacy, and say, that though Mansfield Park might
have some pains, Portsmouth could have no pleasures.



\chapter{Chapter 40}

\gintro{Fanny} was right enough in not expecting to hear from Miss
Crawford now at the rapid rate in which their correspondence
had begun; Mary's next letter was after a decidedly longer
interval than the last, but she was not right in supposing
that such an interval would be felt a great relief
to herself.  Here was another strange revolution of mind!
She was really glad to receive the letter when it did come.
In her present exile from good society, and distance from
everything that had been wont to interest her, a letter
from one belonging to the set where her heart lived,
written with affection, and some degree of elegance,
was thoroughly acceptable.  The usual plea of increasing
engagements was made in excuse for not having
written to her earlier; ``And now that I have begun,''
she continued, ``my letter will not be worth your reading,
for there will be no little offering of love at the end,
no three or four lines \emph{passionnees} from the most
devoted H. C. in the world, for Henry is in Norfolk;
business called him to Everingham ten days ago,
or perhaps he only pretended to call, for the sake of being
travelling at the same time that you were.  But there
he is, and, by the bye, his absence may sufficiently account
for any remissness of his sister's in writing, for there
has been no `Well, Mary, when do you write to Fanny?
Is not it time for you to write to Fanny?' to spur me on.
At last, after various attempts at meeting, I have seen
your cousins, `dear Julia and dearest Mrs.\ Rushworth';
they found me at home yesterday, and we were glad to
see each other again.  We \emph{seemed} \emph{very} glad to see
each other, and I do really think we were a little.
We had a vast deal to say.  Shall I tell you how
Mrs.\ Rushworth looked when your name was mentioned?
I did not use to think her wanting in self-possession,
but she had not quite enough for the demands of yesterday.
Upon the whole, Julia was in the best looks of the two,
at least after you were spoken of.  There was no
recovering the complexion from the moment that I spoke
of `Fanny,' and spoke of her as a sister should.
But Mrs.\ Rushworth's day of good looks will come;
we have cards for her first party on the 28th.  Then she
will be in beauty, for she will open one of the best
houses in Wimpole Street.  I was in it two years ago,
when it was Lady Lascelle's, and prefer it to almost
any I know in London, and certainly she will then feel,
to use a vulgar phrase, that she has got her pennyworth
for her penny.  Henry could not have afforded her such
a house.  I hope she will recollect it, and be satisfied,
as well as she may, with moving the queen of a palace,
though the king may appear best in the background;
and as I have no desire to tease her, I shall never \emph{force}
your name upon her again.  She will grow sober by degrees.
From all that I hear and guess, Baron Wildenheim's
attentions to Julia continue, but I do not know that he
has any serious encouragement.  She ought to do better.
A poor honourable is no catch, and I cannot imagine any
liking in the case, for take away his rants, and the poor
baron has nothing.  What a difference a vowel makes!
If his rents were but equal to his rants!  Your cousin
Edmund moves slowly; detained, perchance, by parish duties.
There may be some old woman at Thornton Lacey to be converted.
I am unwilling to fancy myself neglected for a \emph{young} one.
Adieu! my dear sweet Fanny, this is a long letter from London:
write me a pretty one in reply to gladden Henry's eyes,
when he comes back, and send me an account of all the dashing
young captains whom you disdain for his sake.''

There was great food for meditation in this letter,
and chiefly for unpleasant meditation; and yet, with all
the uneasiness it supplied, it connected her with the absent,
it told her of people and things about whom she had never
felt so much curiosity as now, and she would have been
glad to have been sure of such a letter every week.
Her correspondence with her aunt Bertram was her only
concern of higher interest.

As for any society in Portsmouth, that could at all make
amends for deficiencies at home, there were none within
the circle of her father's and mother's acquaintance
to afford her the smallest satisfaction:  she saw nobody
in whose favour she could wish to overcome her own
shyness and reserve.  The men appeared to her all coarse,
the women all pert, everybody underbred; and she gave
as little contentment as she received from introductions
either to old or new acquaintance.  The young ladies who
approached her at first with some respect, in consideration
of her coming from a baronet's family, were soon offended
by what they termed ``airs''; for, as she neither played
on the pianoforte nor wore fine pelisses, they could,
on farther observation, admit no right of superiority.

The first solid consolation which Fanny received for
the evils of home, the first which her judgment could
entirely approve, and which gave any promise of durability,
was in a better knowledge of Susan, and a hope of being
of service to her.  Susan had always behaved pleasantly
to herself, but the determined character of her general
manners had astonished and alarmed her, and it was at least
a fortnight before she began to understand a disposition
so totally different from her own.  Susan saw that much
was wrong at home, and wanted to set it right.  That a girl
of fourteen, acting only on her own unassisted reason,
should err in the method of reform, was not wonderful;
and Fanny soon became more disposed to admire the natural
light of the mind which could so early distinguish justly,
than to censure severely the faults of conduct to which it led.
Susan was only acting on the same truths, and pursuing
the same system, which her own judgment acknowledged,
but which her more supine and yielding temper would
have shrunk from asserting.  Susan tried to be useful,
where \emph{she} could only have gone away and cried; and that
Susan was useful she could perceive; that things, bad as
they were, would have been worse but for such interposition,
and that both her mother and Betsey were restrained from
some excesses of very offensive indulgence and vulgarity.

In every argument with her mother, Susan had in point
of reason the advantage, and never was there any maternal
tenderness to buy her off.  The blind fondness which was
for ever producing evil around her she had never known.
There was no gratitude for affection past or present
to make her better bear with its excesses to the others.

All this became gradually evident, and gradually placed
Susan before her sister as an object of mingled compassion
and respect.  That her manner was wrong, however, at times
very wrong, her measures often ill-chosen and ill-timed,
and her looks and language very often indefensible,
Fanny could not cease to feel; but she began to hope they
might be rectified.  Susan, she found, looked up to her
and wished for her good opinion; and new as anything like an
office of authority was to Fanny, new as it was to imagine
herself capable of guiding or informing any one, she did
resolve to give occasional hints to Susan, and endeavour
to exercise for her advantage the juster notions of what was
due to everybody, and what would be wisest for herself,
which her own more favoured education had fixed in her.

Her influence, or at least the consciousness and use of it,
originated in an act of kindness by Susan, which, after many
hesitations of delicacy, she at last worked herself up to.
It had very early occurred to her that a small sum
of money might, perhaps, restore peace for ever on the
sore subject of the silver knife, canvassed as it now
was continually, and the riches which she was in possession
of herself, her uncle having given her 10 at parting,
made her as able as she was willing to be generous.
But she was so wholly unused to confer favours,
except on the very poor, so unpractised in removing evils,
or bestowing kindnesses among her equals, and so fearful
of appearing to elevate herself as a great lady at home,
that it took some time to determine that it would not be
unbecoming in her to make such a present.  It was made,
however, at last:  a silver knife was bought for Betsey,
and accepted with great delight, its newness giving it
every advantage over the other that could be desired;
Susan was established in the full possession of her own,
Betsey handsomely declaring that now she had got one so much
prettier herself, she should never want \emph{that} again; and no
reproach seemed conveyed to the equally satisfied mother,
which Fanny had almost feared to be impossible.  The deed
thoroughly answered:  a source of domestic altercation
was entirely done away, and it was the means of opening
Susan's heart to her, and giving her something more to love
and be interested in.  Susan shewed that she had delicacy:
pleased as she was to be mistress of property which she
had been struggling for at least two years, she yet
feared that her sister's judgment had been against her,
and that a reproof was designed her for having so struggled
as to make the purchase necessary for the tranquillity of
the house.

Her temper was open.  She acknowledged her fears,
blamed herself for having contended so warmly;
and from that hour Fanny, understanding the worth of her
disposition and perceiving how fully she was inclined
to seek her good opinion and refer to her judgment,
began to feel again the blessing of affection, and to
entertain the hope of being useful to a mind so much in
need of help, and so much deserving it.  She gave advice,
advice too sound to be resisted by a good understanding,
and given so mildly and considerately as not to irritate
an imperfect temper, and she had the happiness of observing
its good effects not unfrequently.  More was not expected
by one who, while seeing all the obligation and expediency
of submission and forbearance, saw also with sympathetic
acuteness of feeling all that must be hourly grating
to a girl like Susan.  Her greatest wonder on the subject
soon became---not that Susan should have been provoked into
disrespect and impatience against her better knowledge---%
but that so much better knowledge, so many good notions
should have been hers at all; and that, brought up in the
midst of negligence and error, she should have formed
such proper opinions of what ought to be; she, who had
had no cousin Edmund to direct her thoughts or fix her principles.

The intimacy thus begun between them was a material
advantage to each.  By sitting together upstairs,
they avoided a great deal of the disturbance of the house;
Fanny had peace, and Susan learned to think it no
misfortune to be quietly employed.  They sat without
a fire; but that was a privation familiar even to Fanny,
and she suffered the less because reminded by it of
the East room.  It was the only point of resemblance.
In space, light, furniture, and prospect, there was nothing
alike in the two apartments; and she often heaved a sigh
at the remembrance of all her books and boxes, and various
comforts there.  By degrees the girls came to spend the
chief of the morning upstairs, at first only in working
and talking, but after a few days, the remembrance of the
said books grew so potent and stimulative that Fanny found
it impossible not to try for books again.  There were none
in her father's house; but wealth is luxurious and daring,
and some of hers found its way to a circulating library.
She became a subscriber; amazed at being anything \emph{in}
\emph{propria} \emph{persona}, amazed at her own doings in every way,
to be a renter, a chuser of books!  And to be having any
one's improvement in view in her choice!  But so it was.
Susan had read nothing, and Fanny longed to give her
a share in her own first pleasures, and inspire a taste
for the biography and poetry which she delighted in herself.

In this occupation she hoped, moreover, to bury some
of the recollections of Mansfield, which were too apt
to seize her mind if her fingers only were busy;
and, especially at this time, hoped it might be useful
in diverting her thoughts from pursuing Edmund to London,
whither, on the authority of her aunt's last letter,
she knew he was gone.  She had no doubt of what would ensue.
The promised notification was hanging over her head.
The postman's knock within the neighbourhood was beginning
to bring its daily terrors, and if reading could banish
the idea for even half an hour, it was something gained.



\chapter{Chapter 41}

\gintro{A week} was gone since Edmund might be supposed
in town, and Fanny had heard nothing of him.
There were three different conclusions to be drawn from
his silence, between which her mind was in fluctuation;
each of them at times being held the most probable.
Either his going had been again delayed, or he had yet
procured no opportunity of seeing Miss Crawford alone,
or he was too happy for letter-writing!

One morning, about this time, Fanny having now been nearly
four weeks from Mansfield, a point which she never failed
to think over and calculate every day, as she and Susan
were preparing to remove, as usual, upstairs, they were
stopped by the knock of a visitor, whom they felt they could
not avoid, from Rebecca's alertness in going to the door,
a duty which always interested her beyond any other.

It was a gentleman's voice; it was a voice that Fanny
was just turning pale about, when Mr.\ Crawford walked
into the room.

Good sense, like hers, will always act when really
called upon; and she found that she had been able to name
him to her mother, and recall her remembrance of the name,
as that of ``William's friend,'' though she could not
previously have believed herself capable of uttering a
syllable at such a moment.  The consciousness of his being
known there only as William's friend was some support.
Having introduced him, however, and being all reseated,
the terrors that occurred of what this visit might lead
to were overpowering, and she fancied herself on the point
of fainting away.

While trying to keep herself alive, their visitor, who had
at first approached her with as animated a countenance
as ever, was wisely and kindly keeping his eyes away,
and giving her time to recover, while he devoted himself
entirely to her mother, addressing her, and attending to
her with the utmost politeness and propriety, at the same
time with a degree of friendliness, of interest at least,
which was making his manner perfect.

Mrs.\ Price's manners were also at their best.  Warmed by
the sight of such a friend to her son, and regulated
by the wish of appearing to advantage before him, she was
overflowing with gratitude---artless, maternal gratitude---%
which could not be unpleasing.  Mr.\ Price was out,
which she regretted very much.  Fanny was just recovered
enough to feel that \emph{she} could not regret it; for to her
many other sources of uneasiness was added the severe
one of shame for the home in which he found her.
She might scold herself for the weakness, but there was
no scolding it away.  She was ashamed, and she would have
been yet more ashamed of her father than of all the rest.

They talked of William, a subject on which Mrs.\ Price
could never tire; and Mr.\ Crawford was as warm in his
commendation as even her heart could wish.  She felt
that she had never seen so agreeable a man in her life;
and was only astonished to find that, so great and so
agreeable as he was, he should be come down to Portsmouth
neither on a visit to the port-admiral, nor the commissioner,
nor yet with the intention of going over to the island,
nor of seeing the dockyard.  Nothing of all that she
had been used to think of as the proof of importance,
or the employment of wealth, had brought him to Portsmouth.
He had reached it late the night before, was come for a
day or two, was staying at the Crown, had accidentally
met with a navy officer or two of his acquaintance since
his arrival, but had no object of that kind in coming.

By the time he had given all this information, it was not
unreasonable to suppose that Fanny might be looked at
and spoken to; and she was tolerably able to bear his eye,
and hear that he had spent half an hour with his sister
the evening before his leaving London; that she had sent
her best and kindest love, but had had no time for writing;
that he thought himself lucky in seeing Mary for even half
an hour, having spent scarcely twenty-four hours in London,
after his return from Norfolk, before he set off again;
that her cousin Edmund was in town, had been in town,
he understood, a few days; that he had not seen him himself,
but that he was well, had left them all well at Mansfield,
and was to dine, as yesterday, with the Frasers.

Fanny listened collectedly, even to the last-mentioned
circumstance; nay, it seemed a relief to her worn
mind to be at any certainty; and the words, ``then by
this time it is all settled,'' passed internally,
without more evidence of emotion than a faint blush.

After talking a little more about Mansfield, a subject
in which her interest was most apparent, Crawford began
to hint at the expediency of an early walk.  ``It was a
lovely morning, and at that season of the year a fine morning
so often turned off, that it was wisest for everybody not
to delay their exercise''; and such hints producing nothing,
he soon proceeded to a positive recommendation to Mrs.\ Price
and her daughters to take their walk without loss of time.
Now they came to an understanding.  Mrs.\ Price, it appeared,
scarcely ever stirred out of doors, except of a Sunday;
she owned she could seldom, with her large family,
find time for a walk.  ``Would she not, then, persuade her
daughters to take advantage of such weather, and allow
him the pleasure of attending them?''  Mrs.\ Price was
greatly obliged and very complying.  ``Her daughters
were very much confined; Portsmouth was a sad place;
they did not often get out; and she knew they had some
errands in the town, which they would be very glad to do.''
And the consequence was, that Fanny, strange as it was---%
strange, awkward, and distressing---found herself and Susan,
within ten minutes, walking towards the High Street
with Mr.\ Crawford.

It was soon pain upon pain, confusion upon confusion;
for they were hardly in the High Street before they met
her father, whose appearance was not the better from its
being Saturday.  He stopt; and, ungentlemanlike as he looked,
Fanny was obliged to introduce him to Mr.\ Crawford.
She could not have a doubt of the manner in which
Mr.\ Crawford must be struck.  He must be ashamed
and disgusted altogether.  He must soon give her up,
and cease to have the smallest inclination for the match;
and yet, though she had been so much wanting his affection
to be cured, this was a sort of cure that would be almost
as bad as the complaint; and I believe there is scarcely
a young lady in the United Kingdoms who would not rather
put up with the misfortune of being sought by a clever,
agreeable man, than have him driven away by the vulgarity
of her nearest relations.

Mr.\ Crawford probably could not regard his future
father-in-law with any idea of taking him for a model
in dress; but (as Fanny instantly, and to her great
relief, discerned) her father was a very different man,
a very different Mr.\ Price in his behaviour to this most
highly respected stranger, from what he was in his own
family at home.  His manners now, though not polished,
were more than passable:  they were grateful, animated, manly;
his expressions were those of an attached father,
and a sensible man; his loud tones did very well in the
open air, and there was not a single oath to be heard.
Such was his instinctive compliment to the good manners
of Mr.\ Crawford; and, be the consequence what it might,
Fanny's immediate feelings were infinitely soothed.

The conclusion of the two gentlemen's civilities was an offer
of Mr.\ Price's to take Mr.\ Crawford into the dockyard,
which Mr.\ Crawford, desirous of accepting as a favour
what was intended as such, though he had seen the dockyard
again and again, and hoping to be so much the longer
with Fanny, was very gratefully disposed to avail himself of,
if the Miss Prices were not afraid of the fatigue;
and as it was somehow or other ascertained, or inferred,
or at least acted upon, that they were not at all afraid,
to the dockyard they were all to go; and but for
Mr.\ Crawford, Mr.\ Price would have turned thither directly,
without the smallest consideration for his daughters'
errands in the High Street.  He took care, however, that they
should be allowed to go to the shops they came out expressly
to visit; and it did not delay them long, for Fanny could
so little bear to excite impatience, or be waited for,
that before the gentlemen, as they stood at the door,
could do more than begin upon the last naval regulations,
or settle the number of three-deckers now in commission,
their companions were ready to proceed.

They were then to set forward for the dockyard at once,
and the walk would have been conducted---according to
Mr.\ Crawford's opinion---in a singular manner,
had Mr.\ Price been allowed the entire regulation of it,
as the two girls, he found, would have been left
to follow, and keep up with them or not, as they could,
while they walked on together at their own hasty pace.
He was able to introduce some improvement occasionally,
though by no means to the extent he wished; he absolutely
would not walk away from them; and at any crossing
or any crowd, when Mr.\ Price was only calling out,
``Come, girls; come, Fan; come, Sue, take care of yourselves;
keep a sharp lookout!'' he would give them his particular
attendance.

Once fairly in the dockyard, he began to reckon upon
some happy intercourse with Fanny, as they were very soon
joined by a brother lounger of Mr.\ Price's, who was come
to take his daily survey of how things went on, and who
must prove a far more worthy companion than himself;
and after a time the two officers seemed very well satisfied
going about together, and discussing matters of equal
and never-failing interest, while the young people sat down
upon some timbers in the yard, or found a seat on board
a vessel in the stocks which they all went to look at.
Fanny was most conveniently in want of rest.  Crawford could
not have wished her more fatigued or more ready to sit down;
but he could have wished her sister away.  A quick-looking
girl of Susan's age was the very worst third in the world:
totally different from Lady Bertram, all eyes and ears;
and there was no introducing the main point before her.
He must content himself with being only generally agreeable,
and letting Susan have her share of entertainment,
with the indulgence, now and then, of a look or hint
for the better-informed and conscious Fanny.  Norfolk was
what he had mostly to talk of:  there he had been some time,
and everything there was rising in importance from his
present schemes.  Such a man could come from no place,
no society, without importing something to amuse;
his journeys and his acquaintance were all of use,
and Susan was entertained in a way quite new to her.
For Fanny, somewhat more was related than the accidental
agreeableness of the parties he had been in.
For her approbation, the particular reason of his going into
Norfolk at all, at this unusual time of year, was given.
It had been real business, relative to the renewal of a
lease in which the welfare of a large and---he believed---%
industrious family was at stake.  He had suspected his
agent of some underhand dealing; of meaning to bias him
against the deserving; and he had determined to go himself,
and thoroughly investigate the merits of the case.
He had gone, had done even more good than he had foreseen,
had been useful to more than his first plan had comprehended,
and was now able to congratulate himself upon it, and to
feel that in performing a duty, he had secured agreeable
recollections for his own mind.  He had introduced himself
to some tenants whom he had never seen before; he had begun
making acquaintance with cottages whose very existence,
though on his own estate, had been hitherto unknown to him.
This was aimed, and well aimed, at Fanny.  It was pleasing
to hear him speak so properly; here he had been acting
as he ought to do.  To be the friend of the poor and
the oppressed!  Nothing could be more grateful to her;
and she was on the point of giving him an approving look,
when it was all frightened off by his adding a something
too pointed of his hoping soon to have an assistant,
a friend, a guide in every plan of utility or charity
for Everingham:  a somebody that would make Everingham
and all about it a dearer object than it had ever been
yet.

She turned away, and wished he would not say such things.
She was willing to allow he might have more good
qualities than she had been wont to suppose.  She began
to feel the possibility of his turning out well at last;
but he was and must ever be completely unsuited to her,
and ought not to think of her.

He perceived that enough had been said of Everingham,
and that it would be as well to talk of something else,
and turned to Mansfield.  He could not have chosen better;
that was a topic to bring back her attention and her looks
almost instantly.  It was a real indulgence to her to hear
or to speak of Mansfield.  Now so long divided from
everybody who knew the place, she felt it quite the voice
of a friend when he mentioned it, and led the way to her
fond exclamations in praise of its beauties and comforts,
and by his honourable tribute to its inhabitants allowed
her to gratify her own heart in the warmest eulogium,
in speaking of her uncle as all that was clever and good,
and her aunt as having the sweetest of all sweet tempers.

He had a great attachment to Mansfield himself; he said so;
he looked forward with the hope of spending much, very much,
of his time there; always there, or in the neighbourhood.
He particularly built upon a very happy summer and
autumn there this year; he felt that it would be so:
he depended upon it; a summer and autumn infinitely superior
to the last.  As animated, as diversified, as social,
but with circumstances of superiority undescribable.

``Mansfield, Sotherton, Thornton Lacey,'' he continued;
``what a society will be comprised in those houses!
And at Michaelmas, perhaps, a fourth may be added:
some small hunting-box in the vicinity of everything so dear;
for as to any partnership in Thornton Lacey, as Edmund
Bertram once good-humouredly proposed, I hope I foresee
two objections:  two fair, excellent, irresistible objections
to that plan.''

Fanny was doubly silenced here; though when the moment
was passed, could regret that she had not forced herself into
the acknowledged comprehension of one half of his meaning,
and encouraged him to say something more of his sister
and Edmund.  It was a subject which she must learn to speak of,
and the weakness that shrunk from it would soon be quite
unpardonable.

When Mr.\ Price and his friend had seen all that they wished,
or had time for, the others were ready to return;
and in the course of their walk back, Mr.\ Crawford contrived
a minute's privacy for telling Fanny that his only
business in Portsmouth was to see her; that he was come
down for a couple of days on her account, and hers only,
and because he could not endure a longer total separation.
She was sorry, really sorry; and yet in spite of this and the
two or three other things which she wished he had not said,
she thought him altogether improved since she had seen him;
he was much more gentle, obliging, and attentive to other
people's feelings than he had ever been at Mansfield;
she had never seen him so agreeable---so \emph{near} being agreeable;
his behaviour to her father could not offend, and there
was something particularly kind and proper in the notice
he took of Susan.  He was decidedly improved.  She wished
the next day over, she wished he had come only for one day;
but it was not so very bad as she would have expected:
the pleasure of talking of Mansfield was so very great!

Before they parted, she had to thank him for another pleasure,
and one of no trivial kind.  Her father asked him to do
them the honour of taking his mutton with them, and Fanny
had time for only one thrill of horror, before he declared
himself prevented by a prior engagement.  He was engaged
to dinner already both for that day and the next; he had met
with some acquaintance at the Crown who would not be denied;
he should have the honour, however, of waiting on them
again on the morrow, etc., and so they parted---Fanny in
a state of actual felicity from escaping so horrible an evil!

To have had him join their family dinner-party, and see
all their deficiencies, would have been dreadful!
Rebecca's cookery and Rebecca's waiting, and Betsey's
eating at table without restraint, and pulling everything
about as she chose, were what Fanny herself was not yet
enough inured to for her often to make a tolerable meal.
\emph{She} was nice only from natural delicacy, but \emph{he} had been
brought up in a school of luxury and epicurism.



\chapter{Chapter 42}

\gintro{The Prices} were just setting off for church the next day
when Mr.\ Crawford appeared again.  He came, not to stop,
but to join them; he was asked to go with them to the
Garrison chapel, which was exactly what he had intended,
and they all walked thither together.

The family were now seen to advantage.  Nature had given
them no inconsiderable share of beauty, and every Sunday
dressed them in their cleanest skins and best attire.
Sunday always brought this comfort to Fanny, and on this
Sunday she felt it more than ever.  Her poor mother now
did not look so very unworthy of being Lady Bertram's
sister as she was but too apt to look.  It often grieved
her to the heart to think of the contrast between them;
to think that where nature had made so little difference,
circumstances should have made so much, and that her mother,
as handsome as Lady Bertram, and some years her junior,
should have an appearance so much more worn and faded,
so comfortless, so slatternly, so shabby.  But Sunday
made her a very creditable and tolerably cheerful-looking
Mrs.\ Price, coming abroad with a fine family of children,
feeling a little respite of her weekly cares, and only
discomposed if she saw her boys run into danger, or Rebecca
pass by with a flower in her hat.

In chapel they were obliged to divide, but Mr.\ Crawford
took care not to be divided from the female branch;
and after chapel he still continued with them, and made
one in the family party on the ramparts.

Mrs.\ Price took her weekly walk on the ramparts every
fine Sunday throughout the year, always going directly
after morning service and staying till dinner-time. It
was her public place:  there she met her acquaintance,
heard a little news, talked over the badness of the
Portsmouth servants, and wound up her spirits for the six
days ensuing.

Thither they now went; Mr.\ Crawford most happy to consider
the Miss Prices as his peculiar charge; and before they
had been there long, somehow or other, there was no
saying how, Fanny could not have believed it, but he
was walking between them with an arm of each under his,
and she did not know how to prevent or put an end to it.
It made her uncomfortable for a time, but yet there
were enjoyments in the day and in the view which would
be felt.

The day was uncommonly lovely.  It was really March;
but it was April in its mild air, brisk soft wind,
and bright sun, occasionally clouded for a minute;
and everything looked so beautiful under the influence
of such a sky, the effects of the shadows pursuing each
other on the ships at Spithead and the island beyond,
with the ever-varying hues of the sea, now at high water,
dancing in its glee and dashing against the ramparts with
so fine a sound, produced altogether such a combination
of charms for Fanny, as made her gradually almost careless
of the circumstances under which she felt them.  Nay, had she
been without his arm, she would soon have known that she
needed it, for she wanted strength for a two hours'
saunter of this kind, coming, as it generally did,
upon a week's previous inactivity.  Fanny was beginning
to feel the effect of being debarred from her usual
regular exercise; she had lost ground as to health
since her being in Portsmouth; and but for Mr.\ Crawford
and the beauty of the weather would soon have been knocked
up now.

The loveliness of the day, and of the view, he felt
like herself.  They often stopt with the same sentiment
and taste, leaning against the wall, some minutes,
to look and admire; and considering he was not Edmund,
Fanny could not but allow that he was sufficiently open
to the charms of nature, and very well able to express
his admiration.  She had a few tender reveries now and then,
which he could sometimes take advantage of to look in her
face without detection; and the result of these looks was,
that though as bewitching as ever, her face was less
blooming than it ought to be.  She \emph{said} she was
very well, and did not like to be supposed otherwise;
but take it all in all, he was convinced that her present
residence could not be comfortable, and therefore could
not be salutary for her, and he was growing anxious for
her being again at Mansfield, where her own happiness,
and his in seeing her, must be so much greater.

``You have been here a month, I think?'' said he.

``No; not quite a month.  It is only four weeks to-morrow
since I left Mansfield.''

``You are a most accurate and honest reckoner.  I should
call that a month.''

``I did not arrive here till Tuesday evening.''

``And it is to be a two months' visit, is not?''

``Yes.  My uncle talked of two months.  I suppose it
will not be less.''

``And how are you to be conveyed back again?  Who comes
for you?''

``I do not know.  I have heard nothing about it yet
from my aunt.  Perhaps I may be to stay longer.
It may not be convenient for me to be fetched exactly
at the two months' end.''

After a moment's reflection, Mr.\ Crawford replied,
``I know Mansfield, I know its way, I know its faults
towards \emph{you}.  I know the danger of your being so
far forgotten, as to have your comforts give way to the
imaginary convenience of any single being in the family.
I am aware that you may be left here week after week,
if Sir Thomas cannot settle everything for coming himself,
or sending your aunt's maid for you, without involving
the slightest alteration of the arrangements which he
may have laid down for the next quarter of a year.
This will not do.  Two months is an ample allowance;
I should think six weeks quite enough.  I am considering
your sister's health,'' said he, addressing himself to Susan,
``which I think the confinement of Portsmouth unfavourable to.
She requires constant air and exercise.  When you know her
as well as I do, I am sure you will agree that she does,
and that she ought never to be long banished from the free air
and liberty of the country.  If, therefore'' (turning again
to Fanny), ``you find yourself growing unwell, and any
difficulties arise about your returning to Mansfield,
without waiting for the two months to be ended,
\emph{that} must not be regarded as of any consequence,
if you feel yourself at all less strong or comfortable
than usual, and will only let my sister know it, give her
only the slightest hint, she and I will immediately
come down, and take you back to Mansfield.  You know
the ease and the pleasure with which this would be done.
You know all that would be felt on the occasion.''

Fanny thanked him, but tried to laugh it off.

``I am perfectly serious,'' he replied, ``as you perfectly know.
And I hope you will not be cruelly concealing any
tendency to indisposition.  Indeed, you shall \emph{not};
it shall not be in your power; for so long only as you
positively say, in every letter to Mary, `I am well,'
and I know you cannot speak or write a falsehood, so long
only shall you be considered as well.''

Fanny thanked him again, but was affected and distressed
to a degree that made it impossible for her to say much,
or even to be certain of what she ought to say.
This was towards the close of their walk.  He attended
them to the last, and left them only at the door of their
own house, when he knew them to be going to dinner,
and therefore pretended to be waited for elsewhere.

``I wish you were not so tired,'' said he, still detaining
Fanny after all the others were in the house---``I wish I
left you in stronger health.  Is there anything I can
do for you in town?  I have half an idea of going into
Norfolk again soon.  I am not satisfied about Maddison.
I am sure he still means to impose on me if possible,
and get a cousin of his own into a certain mill, which I
design for somebody else.  I must come to an understanding
with him.  I must make him know that I will not be
tricked on the south side of Everingham, any more than on
the north:  that I will be master of my own property.
I was not explicit enough with him before.  The mischief
such a man does on an estate, both as to the credit of his
employer and the welfare of the poor, is inconceivable.
I have a great mind to go back into Norfolk directly,
and put everything at once on such a footing as cannot
be afterwards swerved from.  Maddison is a clever fellow;
I do not wish to displace him, provided he does not try
to displace \emph{me}; but it would be simple to be duped
by a man who has no right of creditor to dupe me,
and worse than simple to let him give me a hard-hearted,
griping fellow for a tenant, instead of an honest man,
to whom I have given half a promise already.  Would it not
be worse than simple?  Shall I go?  Do you advise it?''

``I advise!  You know very well what is right.''

``Yes.  When you give me your opinion, I always know
what is right.  Your judgment is my rule of right.''

``Oh, no! do not say so.  We have all a better guide
in ourselves, if we would attend to it, than any other person
can be.  Good-bye; I wish you a pleasant journey to-morrow.''

``Is there nothing I can do for you in town?''

``Nothing; I am much obliged to you.''

``Have you no message for anybody?''

``My love to your sister, if you please; and when you see
my cousin, my cousin Edmund, I wish you would be so good
as to say that I suppose I shall soon hear from him.''

``Certainly; and if he is lazy or negligent, I will write
his excuses myself.''

He could say no more, for Fanny would be no longer detained.
He pressed her hand, looked at her, and was gone.
\emph{He} went to while away the next three hours as he could,
with his other acquaintance, till the best dinner that
a capital inn afforded was ready for their enjoyment,
and \emph{she} turned in to her more simple one immediately.

Their general fare bore a very different character;
and could he have suspected how many privations, besides that
of exercise, she endured in her father's house, he would
have wondered that her looks were not much more affected
than he found them.  She was so little equal to Rebecca's
puddings and Rebecca's hashes, brought to table, as they
all were, with such accompaniments of half-cleaned plates,
and not half-cleaned knives and forks, that she was very
often constrained to defer her heartiest meal till she could
send her brothers in the evening for biscuits and buns.
After being nursed up at Mansfield, it was too late in the
day to be hardened at Portsmouth; and though Sir Thomas,
had he known all, might have thought his niece in the
most promising way of being starved, both mind and body,
into a much juster value for Mr.\ Crawford's good company
and good fortune, he would probably have feared to push
his experiment farther, lest she might die under the cure.

Fanny was out of spirits all the rest of the day.
Though tolerably secure of not seeing Mr.\ Crawford again,
she could not help being low.  It was parting with somebody
of the nature of a friend; and though, in one light,
glad to have him gone, it seemed as if she was now
deserted by everybody; it was a sort of renewed separation
from Mansfield; and she could not think of his returning
to town, and being frequently with Mary and Edmund,
without feelings so near akin to envy as made her hate
herself for having them.

Her dejection had no abatement from anything passing
around her; a friend or two of her father's, as always
happened if he was not with them, spent the long,
long evening there; and from six o'clock till half-past nine,
there was little intermission of noise or grog.  She was
very low.  The wonderful improvement which she still
fancied in Mr.\ Crawford was the nearest to administering
comfort of anything within the current of her thoughts.
Not considering in how different a circle she had been
just seeing him, nor how much might be owing to contrast,
she was quite persuaded of his being astonishingly
more gentle and regardful of others than formerly.
And, if in little things, must it not be so in great?
So anxious for her health and comfort, so very feeling
as he now expressed himself, and really seemed, might not
it be fairly supposed that he would not much longer
persevere in a suit so distressing to her?



\chapter{Chapter 43}

\gintro{It was presumed} that Mr.\ Crawford was travelling back,
to London, on the morrow, for nothing more was seen
of him at Mr.\ Price's; and two days afterwards, it was
a fact ascertained to Fanny by the following letter from
his sister, opened and read by her, on another account,
with the most anxious curiosity:---%

``I have to inform you, my dearest Fanny, that Henry
has been down to Portsmouth to see you; that he had a
delightful walk with you to the dockyard last Saturday,
and one still more to be dwelt on the next day,
on the ramparts; when the balmy air, the sparkling sea,
and your sweet looks and conversation were altogether
in the most delicious harmony, and afforded sensations
which are to raise ecstasy even in retrospect.  This, as well
as I understand, is to be the substance of my information.
He makes me write, but I do not know what else is to
be communicated, except this said visit to Portsmouth,
and these two said walks, and his introduction to
your family, especially to a fair sister of yours, a fine
girl of fifteen, who was of the party on the ramparts,
taking her first lesson, I presume, in love.  I have
not time for writing much, but it would be out of place
if I had, for this is to be a mere letter of business,
penned for the purpose of conveying necessary information,
which could not be delayed without risk of evil.  My dear,
dear Fanny, if I had you here, how I would talk to you!
You should listen to me till you were tired, and advise
me till you were still tired more; but it is impossible
to put a hundredth part of my great mind on paper,
so I will abstain altogether, and leave you to guess what
you like.  I have no news for you.  You have politics,
of course; and it would be too bad to plague you with
the names of people and parties that fill up my time.
I ought to have sent you an account of your cousin's
first party, but I was lazy, and now it is too long ago;
suffice it, that everything was just as it ought to be,
in a style that any of her connexions must have been
gratified to witness, and that her own dress and manners did
her the greatest credit.  My friend, Mrs.\ Fraser, is mad
for such a house, and it would not make \emph{me} miserable.
I go to Lady Stornaway after Easter; she seems in high spirits,
and very happy.  I fancy Lord S. is very good-humoured
and pleasant in his own family, and I do not think him so
very ill-looking as I did---at least, one sees many worse.
He will not do by the side of your cousin Edmund.
Of the last-mentioned hero, what shall I say?  If I
avoided his name entirely, it would look suspicious.
I will say, then, that we have seen him two or three times,
and that my friends here are very much struck with his
gentlemanlike appearance.  Mrs.\ Fraser (no bad judge)
declares she knows but three men in town who have so good
a person, height, and air; and I must confess, when he dined
here the other day, there were none to compare with him,
and we were a party of sixteen.  Luckily there is no
distinction of dress nowadays to tell tales, but---but---%
but Yours affectionately.''

``I had almost forgot (it was Edmund's fault:  he gets into
my head more than does me good) one very material thing I
had to say from Henry and myself---I mean about our taking
you back into Northamptonshire.  My dear little creature,
do not stay at Portsmouth to lose your pretty looks.
Those vile sea-breezes are the ruin of beauty and health.
My poor aunt always felt affected if within ten miles
of the sea, which the Admiral of course never believed,
but I know it was so.  I am at your service and Henry's,
at an hour's notice.  I should like the scheme, and we would
make a little circuit, and shew you Everingham in our way,
and perhaps you would not mind passing through London,
and seeing the inside of St. George's, Hanover Square.
Only keep your cousin Edmund from me at such a time:
I should not like to be tempted.  What a long letter!
one word more.  Henry, I find, has some idea of going
into Norfolk again upon some business that \emph{you} approve;
but this cannot possibly be permitted before the middle
of next week; that is, he cannot anyhow be spared till
after the 14th, for \emph{we} have a party that evening.
The value of a man like Henry, on such an occasion,
is what you can have no conception of; so you must take it
upon my word to be inestimable.  He will see the Rushworths,
which own I am not sorry for---having a little curiosity,
and so I think has he---though he will not acknowledge
it.''

This was a letter to be run through eagerly, to be
read deliberately, to supply matter for much reflection,
and to leave everything in greater suspense than ever.
The only certainty to be drawn from it was, that nothing
decisive had yet taken place.  Edmund had not yet spoken.
How Miss Crawford really felt, how she meant to act,
or might act without or against her meaning; whether his
importance to her were quite what it had been before
the last separation; whether, if lessened, it were likely
to lessen more, or to recover itself, were subjects
for endless conjecture, and to be thought of on that day
and many days to come, without producing any conclusion.
The idea that returned the oftenest was that Miss Crawford,
after proving herself cooled and staggered by a return
to London habits, would yet prove herself in the end
too much attached to him to give him up.  She would
try to be more ambitious than her heart would allow.
She would hesitate, she would tease, she would condition,
she would require a great deal, but she would finally
accept.

This was Fanny's most frequent expectation.  A house
in town---that, she thought, must be impossible.
Yet there was no saying what Miss Crawford might not ask.
The prospect for her cousin grew worse and worse.
The woman who could speak of him, and speak only of
his appearance!  What an unworthy attachment!  To be
deriving support from the commendations of Mrs.\ Fraser!
\emph{She} who had known him intimately half a year!
Fanny was ashamed of her.  Those parts of the letter which
related only to Mr.\ Crawford and herself, touched her,
in comparison, slightly.  Whether Mr.\ Crawford went
into Norfolk before or after the 14th was certainly
no concern of hers, though, everything considered,
she thought he \emph{would} go without delay.  That Miss
Crawford should endeavour to secure a meeting between him
and Mrs.\ Rushworth, was all in her worst line of conduct,
and grossly unkind and ill-judged; but she hoped \emph{he}
would not be actuated by any such degrading curiosity.
He acknowledged no such inducement, and his sister
ought to have given him credit for better feelings than
her own.

She was yet more impatient for another letter from
town after receiving this than she had been before;
and for a few days was so unsettled by it altogether,
by what had come, and what might come, that her usual
readings and conversation with Susan were much suspended.
She could not command her attention as she wished.
If Mr.\ Crawford remembered her message to her cousin,
she thought it very likely, most likely, that he would write
to her at all events; it would be most consistent with his
usual kindness; and till she got rid of this idea, till it
gradually wore off, by no letters appearing in the course
of three or four days more, she was in a most restless,
anxious state.

At length, a something like composure succeeded.
Suspense must be submitted to, and must not be allowed
to wear her out, and make her useless.  Time did something,
her own exertions something more, and she resumed her
attentions to Susan, and again awakened the same interest
in them.

Susan was growing very fond of her, and though without
any of the early delight in books which had been
so strong in Fanny, with a disposition much less
inclined to sedentary pursuits, or to information for
information's sake, she had so strong a desire of not
\emph{appearing} ignorant, as, with a good clear understanding,
made her a most attentive, profitable, thankful pupil.
Fanny was her oracle.  Fanny's explanations and remarks
were a most important addition to every essay, or every
chapter of history.  What Fanny told her of former times
dwelt more on her mind than the pages of Goldsmith; and she
paid her sister the compliment of preferring her style
to that of any printed author.  The early habit of reading was
wanting.

Their conversations, however, were not always on subjects
so high as history or morals.  Others had their hour;
and of lesser matters, none returned so often,
or remained so long between them, as Mansfield Park,
a description of the people, the manners, the amusements,
the ways of Mansfield Park.  Susan, who had an innate taste
for the genteel and well-appointed, was eager to hear,
and Fanny could not but indulge herself in dwelling on
so beloved a theme.  She hoped it was not wrong; though,
after a time, Susan's very great admiration of everything
said or done in her uncle's house, and earnest longing
to go into Northamptonshire, seemed almost to blame
her for exciting feelings which could not be gratified.

Poor Susan was very little better fitted for home
than her elder sister; and as Fanny grew thoroughly
to understand this, she began to feel that when her
own release from Portsmouth came, her happiness would
have a material drawback in leaving Susan behind.
That a girl so capable of being made everything good should
be left in such hands, distressed her more and more.
Were \emph{she} likely to have a home to invite her to,
what a blessing it would be!  And had it been possible
for her to return Mr.\ Crawford's regard, the probability
of his being very far from objecting to such a measure would
have been the greatest increase of all her own comforts.
She thought he was really good-tempered, and could fancy
his entering into a plan of that sort most pleasantly.



\chapter{Chapter 44}

\gintro{Seven weeks} of the two months were very nearly gone,
when the one letter, the letter from Edmund, so long expected,
was put into Fanny's hands.  As she opened, and saw
its length, she prepared herself for a minute detail
of happiness and a profusion of love and praise towards
the fortunate creature who was now mistress of his fate.
These were the contents---%

``My Dear Fanny,---Excuse me that I have not written before.
Crawford told me that you were wishing to hear from me,
but I found it impossible to write from London,
and persuaded myself that you would understand my silence.
Could I have sent a few happy lines, they should not
have been wanting, but nothing of that nature was ever
in my power.  I am returned to Mansfield in a less assured
state that when I left it.  My hopes are much weaker.
You are probably aware of this already.  So very fond of you
as Miss Crawford is, it is most natural that she should tell
you enough of her own feelings to furnish a tolerable guess
at mine.  I will not be prevented, however, from making my
own communication.  Our confidences in you need not clash.
I ask no questions.  There is something soothing in the
idea that we have the same friend, and that whatever
unhappy differences of opinion may exist between us,
we are united in our love of you.  It will be a comfort
to me to tell you how things now are, and what are my
present plans, if plans I can be said to have.  I have been
returned since Saturday.  I was three weeks in London,
and saw her (for London) very often.  I had every attention
from the Frasers that could be reasonably expected.
I dare say I was not reasonable in carrying with me
hopes of an intercourse at all like that of Mansfield.
It was her manner, however, rather than any unfrequency
of meeting.  Had she been different when I did see her,
I should have made no complaint, but from the very first
she was altered:  my first reception was so unlike
what I had hoped, that I had almost resolved on leaving
London again directly.  I need not particularise.
You know the weak side of her character, and may imagine
the sentiments and expressions which were torturing me.
She was in high spirits, and surrounded by those who
were giving all the support of their own bad sense
to her too lively mind.  I do not like Mrs.\ Fraser.
She is a cold-hearted, vain woman, who has married entirely
from convenience, and though evidently unhappy in her marriage,
places her disappointment not to faults of judgment,
or temper, or disproportion of age, but to her being,
after all, less affluent than many of her acquaintance,
especially than her sister, Lady Stornaway, and is the
determined supporter of everything mercenary and ambitious,
provided it be only mercenary and ambitious enough.  I look
upon her intimacy with those two sisters as the greatest
misfortune of her life and mine.  They have been leading
her astray for years.  Could she be detached from them!---%
and sometimes I do not despair of it, for the affection
appears to me principally on their side.  They are very
fond of her; but I am sure she does not love them as she
loves you.  When I think of her great attachment to you,
indeed, and the whole of her judicious, upright conduct
as a sister, she appears a very different creature,
capable of everything noble, and I am ready to blame
myself for a too harsh construction of a playful manner.
I cannot give her up, Fanny.  She is the only woman
in the world whom I could ever think of as a wife.
If I did not believe that she had some regard for me,
of course I should not say this, but I do believe it.
I am convinced that she is not without a decided preference.
I have no jealousy of any individual.  It is the influence
of the fashionable world altogether that I am jealous of.
It is the habits of wealth that I fear.  Her ideas are
not higher than her own fortune may warrant, but they
are beyond what our incomes united could authorise.
There is comfort, however, even here.  I could better
bear to lose her because not rich enough, than because
of my profession.  That would only prove her affection
not equal to sacrifices, which, in fact, I am scarcely
justified in asking; and, if I am refused, that, I think,
will be the honest motive.  Her prejudices, I trust,
are not so strong as they were.  You have my thoughts
exactly as they arise, my dear Fanny; perhaps they are
sometimes contradictory, but it will not be a less faithful
picture of my mind.  Having once begun, it is a pleasure
to me to tell you all I feel.  I cannot give her up.
Connected as we already are, and, I hope, are to be,
to give up Mary Crawford would be to give up the society
of some of those most dear to me; to banish myself from
the very houses and friends whom, under any other distress,
I should turn to for consolation.  The loss of Mary I must
consider as comprehending the loss of Crawford and of Fanny.
Were it a decided thing, an actual refusal, I hope I
should know how to bear it, and how to endeavour to weaken
her hold on my heart, and in the course of a few years---%
but I am writing nonsense.  Were I refused, I must bear it;
and till I am, I can never cease to try for her.
This is the truth.  The only question is \emph{how}?  What may
be the likeliest means?  I have sometimes thought of going
to London again after Easter, and sometimes resolved on
doing nothing till she returns to Mansfield.  Even now,
she speaks with pleasure of being in Mansfield in June;
but June is at a great distance, and I believe I shall
write to her.  I have nearly determined on explaining
myself by letter.  To be at an early certainty is a
material object.  My present state is miserably irksome.
Considering everything, I think a letter will be decidedly
the best method of explanation.  I shall be able to write
much that I could not say, and shall be giving her time
for reflection before she resolves on her answer,
and I am less afraid of the result of reflection
than of an immediate hasty impulse; I think I am.
My greatest danger would lie in her consulting Mrs.\ Fraser,
and I at a distance unable to help my own cause.
A letter exposes to all the evil of consultation,
and where the mind is anything short of perfect decision,
an adviser may, in an unlucky moment, lead it to do what it
may afterwards regret.  I must think this matter over
a little.  This long letter, full of my own concerns alone,
will be enough to tire even the friendship of a Fanny.
The last time I saw Crawford was at Mrs.\ Fraser's party.
I am more and more satisfied with all that I see and hear
of him.  There is not a shadow of wavering.  He thoroughly
knows his own mind, and acts up to his resolutions:
an inestimable quality.  I could not see him and my eldest
sister in the same room without recollecting what you
once told me, and I acknowledge that they did not meet
as friends.  There was marked coolness on her side.
They scarcely spoke.  I saw him draw back surprised,
and I was sorry that Mrs.\ Rushworth should resent any
former supposed slight to Miss Bertram.  You will wish
to hear my opinion of Maria's degree of comfort as a wife.
There is no appearance of unhappiness.  I hope they get
on pretty well together.  I dined twice in Wimpole Street,
and might have been there oftener, but it is mortifying
to be with Rushworth as a brother.  Julia seems to enjoy
London exceedingly.  I had little enjoyment there,
but have less here.  We are not a lively party.  You are
very much wanted.  I miss you more than I can express.
My mother desires her best love, and hopes to hear
from you soon.  She talks of you almost every hour,
and I am sorry to find how many weeks more she is likely
to be without you.  My father means to fetch you himself,
but it will not be till after Easter, when he has
business in town.  You are happy at Portsmouth, I hope,
but this must not be a yearly visit.  I want you at home,
that I may have your opinion about Thornton Lacey.
I have little heart for extensive improvements till
I know that it will ever have a mistress.  I think I
shall certainly write.  It is quite settled that the
Grants go to Bath; they leave Mansfield on Monday.
I am glad of it.  I am not comfortable enough to be fit
for anybody; but your aunt seems to feel out of luck
that such an article of Mansfield news should fall
to my pen instead of hers.---Yours ever, my dearest
Fanny.''

``I never will, no, I certainly never will wish for a
letter again,'' was Fanny's secret declaration as she
finished this.  ``What do they bring but disappointment
and sorrow?  Not till after Easter!  How shall I bear it?
And my poor aunt talking of me every hour!''

Fanny checked the tendency of these thoughts as well as
she could, but she was within half a minute of starting
the idea that Sir Thomas was quite unkind, both to her aunt
and to herself.  As for the main subject of the letter,
there was nothing in that to soothe irritation.  She was
almost vexed into displeasure and anger against Edmund.
``There is no good in this delay,'' said she.  ``Why is not
it settled?  He is blinded, and nothing will open his eyes;
nothing can, after having had truths before him so long
in vain.  He will marry her, and be poor and miserable.
God grant that her influence do not make him cease
to be respectable!''  She looked over the letter again.
``\,`So very fond of me!' 'tis nonsense all.  She loves
nobody but herself and her brother.  Her friends leading
her astray for years!  She is quite as likely to have led
\emph{them} astray.  They have all, perhaps, been corrupting
one another; but if they are so much fonder of her than
she is of them, she is the less likely to have been hurt,
except by their flattery.  `The only woman in the world
whom he could ever think of as a wife.'  I firmly
believe it.  It is an attachment to govern his whole life.
Accepted or refused, his heart is wedded to her for ever.
`The loss of Mary I must consider as comprehending the loss
of Crawford and Fanny.'  Edmund, you do not know me.
The families would never be connected if you did not
connect them!  Oh! write, write.  Finish it at once.
Let there be an end of this suspense.  Fix, commit,
condemn yourself.''

Such sensations, however, were too near akin to
resentment to be long guiding Fanny's soliloquies.
She was soon more softened and sorrowful.  His warm regard,
his kind expressions, his confidential treatment,
touched her strongly.  He was only too good to everybody.
It was a letter, in short, which she would not but have had
for the world, and which could never be valued enough.
This was the end of it.

Everybody at all addicted to letter-writing, without
having much to say, which will include a large proportion
of the female world at least, must feel with Lady Bertram
that she was out of luck in having such a capital piece of
Mansfield news as the certainty of the Grants going to Bath,
occur at a time when she could make no advantage of it,
and will admit that it must have been very mortifying
to her to see it fall to the share of her thankless son,
and treated as concisely as possible at the end of a
long letter, instead of having it to spread over the largest
part of a page of her own.  For though Lady Bertram rather
shone in the epistolary line, having early in her marriage,
from the want of other employment, and the circumstance
of Sir Thomas's being in Parliament, got into the way
of making and keeping correspondents, and formed for
herself a very creditable, common-place, amplifying style,
so that a very little matter was enough for her:  she could
not do entirely without any; she must have something
to write about, even to her niece; and being so soon
to lose all the benefit of Dr.\ Grant's gouty symptoms
and Mrs.\ Grant's morning calls, it was very hard upon her
to be deprived of one of the last epistolary uses she could put
them to.

There was a rich amends, however, preparing for her.
Lady Bertram's hour of good luck came.  Within a few days
from the receipt of Edmund's letter, Fanny had one from
her aunt, beginning thus---%

``My Dear Fanny,---I take up my pen to communicate some
very alarming intelligence, which I make no doubt will
give you much concern''.

This was a great deal better than to have to take up the pen
to acquaint her with all the particulars of the Grants'
intended journey, for the present intelligence was of a
nature to promise occupation for the pen for many days
to come, being no less than the dangerous illness of her
eldest son, of which they had received notice by express
a few hours before.

Tom had gone from London with a party of young men
to Newmarket, where a neglected fall and a good deal
of drinking had brought on a fever; and when the party
broke up, being unable to move, had been left by himself
at the house of one of these young men to the comforts of
sickness and solitude, and the attendance only of servants.
Instead of being soon well enough to follow his friends,
as he had then hoped, his disorder increased considerably,
and it was not long before he thought so ill of himself
as to be as ready as his physician to have a letter
despatched to Mansfield.

``This distressing intelligence, as you may suppose,''
observed her ladyship, after giving the substance of it,
``has agitated us exceedingly, and we cannot prevent
ourselves from being greatly alarmed and apprehensive
for the poor invalid, whose state Sir Thomas fears may
be very critical; and Edmund kindly proposes attending
his brother immediately, but I am happy to add that Sir
Thomas will not leave me on this distressing occasion,
as it would be too trying for me.  We shall greatly miss
Edmund in our small circle, but I trust and hope he
will find the poor invalid in a less alarming state than
might be apprehended, and that he will be able to bring
him to Mansfield shortly, which Sir Thomas proposes
should be done, and thinks best on every account, and I
flatter myself the poor sufferer will soon be able to bear
the removal without material inconvenience or injury.
As I have little doubt of your feeling for us, my dear Fanny,
under these distressing circumstances, I will write again
very soon.''

Fanny's feelings on the occasion were indeed considerably
more warm and genuine than her aunt's style of writing.
She felt truly for them all.  Tom dangerously ill,
Edmund gone to attend him, and the sadly small party
remaining at Mansfield, were cares to shut out every
other care, or almost every other.  She could just find
selfishness enough to wonder whether Edmund \emph{had} written
to Miss Crawford before this summons came, but no sentiment
dwelt long with her that was not purely affectionate and
disinterestedly anxious.  Her aunt did not neglect her:
she wrote again and again; they were receiving frequent
accounts from Edmund, and these accounts were as regularly
transmitted to Fanny, in the same diffuse style,
and the same medley of trusts, hopes, and fears,
all following and producing each other at haphazard.
It was a sort of playing at being frightened.
The sufferings which Lady Bertram did not see had little
power over her fancy; and she wrote very comfortably
about agitation, and anxiety, and poor invalids, till Tom
was actually conveyed to Mansfield, and her own eyes had
beheld his altered appearance.  Then a letter which she
had been previously preparing for Fanny was finished
in a different style, in the language of real feeling
and alarm; then she wrote as she might have spoken.
``He is just come, my dear Fanny, and is taken upstairs;
and I am so shocked to see him, that I do not know
what to do.  I am sure he has been very ill.  Poor Tom!
I am quite grieved for him, and very much frightened,
and so is Sir Thomas; and how glad I should be if you
were here to comfort me.  But Sir Thomas hopes he
will be better to-morrow, and says we must consider
his journey.''

The real solicitude now awakened in the maternal bosom
was not soon over.  Tom's extreme impatience to be
removed to Mansfield, and experience those comforts
of home and family which had been little thought of in
uninterrupted health, had probably induced his being
conveyed thither too early, as a return of fever came on,
and for a week he was in a more alarming state than ever.
They were all very seriously frightened.  Lady Bertram
wrote her daily terrors to her niece, who might now be said
to live upon letters, and pass all her time between suffering
from that of to-day and looking forward to to-morrow's.
Without any particular affection for her eldest cousin,
her tenderness of heart made her feel that she could
not spare him, and the purity of her principles added yet
a keener solicitude, when she considered how little useful,
how little self-denying his life had (apparently) been.

Susan was her only companion and listener on this, as on
more common occasions.  Susan was always ready to hear and
to sympathise.  Nobody else could be interested in so remote
an evil as illness in a family above an hundred miles off;
not even Mrs.\ Price, beyond a brief question or two,
if she saw her daughter with a letter in her hand,
and now and then the quiet observation of, ``My poor
sister Bertram must be in a great deal of trouble.''

So long divided and so differently situated, the ties
of blood were little more than nothing.  An attachment,
originally as tranquil as their tempers, was now become
a mere name.  Mrs.\ Price did quite as much for Lady
Bertram as Lady Bertram would have done for Mrs.\ Price.
Three or four Prices might have been swept away,
any or all except Fanny and William, and Lady Bertram
would have thought little about it; or perhaps might have
caught from Mrs.\ Norris's lips the cant of its being
a very happy thing and a great blessing to their poor
dear sister Price to have them so well provided for.



\chapter{Chapter 45}

\gintro{At about the week's end} from his return to Mansfield,
Tom's immediate danger was over, and he was so far
pronounced safe as to make his mother perfectly easy;
for being now used to the sight of him in his suffering,
helpless state, and hearing only the best, and never thinking
beyond what she heard, with no disposition for alarm
and no aptitude at a hint, Lady Bertram was the happiest
subject in the world for a little medical imposition.
The fever was subdued; the fever had been his complaint;
of course he would soon be well again.  Lady Bertram could
think nothing less, and Fanny shared her aunt's security,
till she received a few lines from Edmund, written purposely
to give her a clearer idea of his brother's situation,
and acquaint her with the apprehensions which he and his
father had imbibed from the physician with respect to some
strong hectic symptoms, which seemed to seize the frame
on the departure of the fever.  They judged it best
that Lady Bertram should not be harassed by alarms which,
it was to be hoped, would prove unfounded; but there was
no reason why Fanny should not know the truth.  They were
apprehensive for his lungs.

A very few lines from Edmund shewed her the patient
and the sickroom in a juster and stronger light than
all Lady Bertram's sheets of paper could do.  There was
hardly any one in the house who might not have described,
from personal observation, better than herself;
not one who was not more useful at times to her son.
She could do nothing but glide in quietly and look at him;
but when able to talk or be talked to, or read to,
Edmund was the companion he preferred.  His aunt worried
him by her cares, and Sir Thomas knew not how to bring down
his conversation or his voice to the level of irritation
and feebleness.  Edmund was all in all.  Fanny would
certainly believe him so at least, and must find that her
estimation of him was higher than ever when he appeared
as the attendant, supporter, cheerer of a suffering brother.
There was not only the debility of recent illness to assist:
there was also, as she now learnt, nerves much affected,
spirits much depressed to calm and raise, and her own
imagination added that there must be a mind to be
properly guided.

The family were not consumptive, and she was more inclined
to hope than fear for her cousin, except when she thought
of Miss Crawford; but Miss Crawford gave her the idea
of being the child of good luck, and to her selfishness
and vanity it would be good luck to have Edmund the only son.

Even in the sick chamber the fortunate Mary was
not forgotten.  Edmund's letter had this postscript.
``On the subject of my last, I had actually begun a letter
when called away by Tom's illness, but I have now changed
my mind, and fear to trust the influence of friends.
When Tom is better, I shall go.''

Such was the state of Mansfield, and so it continued,
with scarcely any change, till Easter.  A line occasionally
added by Edmund to his mother's letter was enough for
Fanny's information.  Tom's amendment was alarmingly slow.

Easter came particularly late this year, as Fanny had most
sorrowfully considered, on first learning that she had
no chance of leaving Portsmouth till after it.  It came,
and she had yet heard nothing of her return---nothing even
of the going to London, which was to precede her return.
Her aunt often expressed a wish for her, but there was
no notice, no message from the uncle on whom all depended.
She supposed he could not yet leave his son, but it was a cruel,
a terrible delay to her.  The end of April was coming on;
it would soon be almost three months, instead of two,
that she had been absent from them all, and that her days
had been passing in a state of penance, which she loved
them too well to hope they would thoroughly understand;
and who could yet say when there might be leisure to think
of or fetch her?

Her eagerness, her impatience, her longings to be with them,
were such as to bring a line or two of Cowper's Tirocinium
for ever before her.  ``With what intense desire she wants
her home,'' was continually on her tongue, as the truest
description of a yearning which she could not suppose
any schoolboy's bosom to feel more keenly.

When she had been coming to Portsmouth, she had loved to call
it her home, had been fond of saying that she was going home;
the word had been very dear to her, and so it still was,
but it must be applied to Mansfield.  \emph{That} was now
the home.  Portsmouth was Portsmouth; Mansfield was home.
They had been long so arranged in the indulgence of her
secret meditations, and nothing was more consolatory
to her than to find her aunt using the same language:
``I cannot but say I much regret your being from home
at this distressing time, so very trying to my spirits.
I trust and hope, and sincerely wish you may never be absent
from home so long again,'' were most delightful sentences
to her.  Still, however, it was her private regale.
Delicacy to her parents made her careful not to betray
such a preference of her uncle's house.  It was always:
``When I go back into Northamptonshire, or when I return
to Mansfield, I shall do so and so.''  For a great
while it was so, but at last the longing grew stronger,
it overthrew caution, and she found herself talking of what
she should do when she went home before she was aware.
She reproached herself, coloured, and looked fearfully towards
her father and mother.  She need not have been uneasy.
There was no sign of displeasure, or even of hearing her.
They were perfectly free from any jealousy of Mansfield.
She was as welcome to wish herself there as to be there.

It was sad to Fanny to lose all the pleasures of spring.
She had not known before what pleasures she \emph{had} to lose
in passing March and April in a town.  She had not known
before how much the beginnings and progress of vegetation
had delighted her.  What animation, both of body and mind,
she had derived from watching the advance of that season
which cannot, in spite of its capriciousness, be unlovely,
and seeing its increasing beauties from the earliest
flowers in the warmest divisions of her aunt's garden,
to the opening of leaves of her uncle's plantations,
and the glory of his woods.  To be losing such pleasures
was no trifle; to be losing them, because she was in
the midst of closeness and noise, to have confinement,
bad air, bad smells, substituted for liberty,
freshness, fragrance, and verdure, was infinitely worse:
but even these incitements to regret were feeble,
compared with what arose from the conviction of being
missed by her best friends, and the longing to be useful
to those who were wanting her!

Could she have been at home, she might have been of service
to every creature in the house.  She felt that she must
have been of use to all.  To all she must have saved some
trouble of head or hand; and were it only in supporting
the spirits of her aunt Bertram, keeping her from the evil
of solitude, or the still greater evil of a restless,
officious companion, too apt to be heightening danger
in order to enhance her own importance, her being there
would have been a general good.  She loved to fancy how she
could have read to her aunt, how she could have talked
to her, and tried at once to make her feel the blessing
of what was, and prepare her mind for what might be;
and how many walks up and down stairs she might have
saved her, and how many messages she might have carried.

It astonished her that Tom's sisters could be satisfied
with remaining in London at such a time, through an
illness which had now, under different degrees of danger,
lasted several weeks.  \emph{They} might return to Mansfield
when they chose; travelling could be no difficulty to \emph{them},
and she could not comprehend how both could still keep away.
If Mrs.\ Rushworth could imagine any interfering obligations,
Julia was certainly able to quit London whenever she chose.
It appeared from one of her aunt's letters that Julia
had offered to return if wanted, but this was all.
It was evident that she would rather remain where she was.

Fanny was disposed to think the influence of London
very much at war with all respectable attachments.
She saw the proof of it in Miss Crawford, as well as in
her cousins; \emph{her} attachment to Edmund had been respectable,
the most respectable part of her character; her friendship
for herself had at least been blameless.  Where was
either sentiment now?  It was so long since Fanny had had
any letter from her, that she had some reason to think
lightly of the friendship which had been so dwelt on.
It was weeks since she had heard anything of Miss Crawford
or of her other connexions in town, except through Mansfield,
and she was beginning to suppose that she might never
know whether Mr.\ Crawford had gone into Norfolk again
or not till they met, and might never hear from his
sister any more this spring, when the following letter
was received to revive old and create some new sensations---%

``Forgive me, my dear Fanny, as soon as you can, for my
long silence, and behave as if you could forgive me directly.
This is my modest request and expectation, for you are so good,
that I depend upon being treated better than I deserve,
and I write now to beg an immediate answer.  I want to know
the state of things at Mansfield Park, and you, no doubt,
are perfectly able to give it.  One should be a brute not
to feel for the distress they are in; and from what I hear,
poor Mr.\ Bertram has a bad chance of ultimate recovery.
I thought little of his illness at first.  I looked
upon him as the sort of person to be made a fuss with,
and to make a fuss himself in any trifling disorder,
and was chiefly concerned for those who had to nurse him;
but now it is confidently asserted that he is really
in a decline, that the symptoms are most alarming,
and that part of the family, at least, are aware of it.
If it be so, I am sure you must be included in that part,
that discerning part, and therefore entreat you to let
me know how far I have been rightly informed.  I need
not say how rejoiced I shall be to hear there has been
any mistake, but the report is so prevalent that I confess
I cannot help trembling.  To have such a fine young man
cut off in the flower of his days is most melancholy.
Poor Sir Thomas will feel it dreadfully.  I really am quite
agitated on the subject.  Fanny, Fanny, I see you smile
and look cunning, but, upon my honour, I never bribed
a physician in my life.  Poor young man!  If he is to die,
there will be \emph{two} poor young men less in the world;
and with a fearless face and bold voice would I say to any one,
that wealth and consequence could fall into no hands
more deserving of them.  It was a foolish precipitation
last Christmas, but the evil of a few days may be blotted
out in part.  Varnish and gilding hide many stains.
It will be but the loss of the Esquire after his name.
With real affection, Fanny, like mine, more might be overlooked.
Write to me by return of post, judge of my anxiety,
and do not trifle with it.  Tell me the real truth,
as you have it from the fountainhead.  And now, do not
trouble yourself to be ashamed of either my feelings or
your own.  Believe me, they are not only natural, they are
philanthropic and virtuous.  I put it to your conscience,
whether `Sir Edmund' would not do more good with all
the Bertram property than any other possible `Sir.'
Had the Grants been at home I would not have troubled you,
but you are now the only one I can apply to for the truth,
his sisters not being within my reach.  Mrs.\ R. has
been spending the Easter with the Aylmers at Twickenham
(as to be sure you know), and is not yet returned;
and Julia is with the cousins who live near Bedford Square,
but I forget their name and street.  Could I immediately
apply to either, however, I should still prefer you,
because it strikes me that they have all along been so
unwilling to have their own amusements cut up, as to shut
their eyes to the truth.  I suppose Mrs.\ R.'s Easter
holidays will not last much longer; no doubt they are
thorough holidays to her.  The Aylmers are pleasant people;
and her husband away, she can have nothing but enjoyment.
I give her credit for promoting his going dutifully down
to Bath, to fetch his mother; but how will she and the
dowager agree in one house?  Henry is not at hand, so I
have nothing to say from him.  Do not you think Edmund would
have been in town again long ago, but for this illness?---%
Yours ever, Mary.''

``I had actually begun folding my letter when Henry walked in,
but he brings no intelligence to prevent my sending it.
Mrs.\ R. knows a decline is apprehended; he saw her this morning:
she returns to Wimpole Street to-day; the old lady is come.
Now do not make yourself uneasy with any queer fancies
because he has been spending a few days at Richmond.
He does it every spring.  Be assured he cares for nobody
but you.  At this very moment he is wild to see you,
and occupied only in contriving the means for doing so,
and for making his pleasure conduce to yours.  In proof,
he repeats, and more eagerly, what he said at Portsmouth
about our conveying you home, and I join him in it with all
my soul.  Dear Fanny, write directly, and tell us to come.
It will do us all good.  He and I can go to the Parsonage,
you know, and be no trouble to our friends at Mansfield Park.
It would really be gratifying to see them all again, and a
little addition of society might be of infinite use to them;
and as to yourself, you must feel yourself to be so wanted there,
that you cannot in conscience---conscientious as you are---%
keep away, when you have the means of returning.
I have not time or patience to give half Henry's messages;
be satisfied that the spirit of each and every one is
unalterable affection.''

Fanny's disgust at the greater part of this letter,
with her extreme reluctance to bring the writer of it
and her cousin Edmund together, would have made her (as
she felt) incapable of judging impartially whether
the concluding offer might be accepted or not.
To herself, individually, it was most tempting.  To be
finding herself, perhaps within three days, transported
to Mansfield, was an image of the greatest felicity,
but it would have been a material drawback to be owing
such felicity to persons in whose feelings and conduct,
at the present moment, she saw so much to condemn:
the sister's feelings, the brother's conduct,
\emph{her} cold-hearted ambition, \emph{his} thoughtless vanity.
To have him still the acquaintance, the flirt perhaps,
of Mrs.\ Rushworth!  She was mortified.  She had thought
better of him.  Happily, however, she was not left to weigh
and decide between opposite inclinations and doubtful
notions of right; there was no occasion to determine
whether she ought to keep Edmund and Mary asunder or not.
She had a rule to apply to, which settled everything.
Her awe of her uncle, and her dread of taking a liberty
with him, made it instantly plain to her what she
had to do.  She must absolutely decline the proposal.
If he wanted, he would send for her; and even to offer
an early return was a presumption which hardly anything
would have seemed to justify.  She thanked Miss Crawford,
but gave a decided negative.  ``Her uncle, she understood,
meant to fetch her; and as her cousin's illness had continued
so many weeks without her being thought at all necessary,
she must suppose her return would be unwelcome at present,
and that she should be felt an encumbrance.''

Her representation of her cousin's state at this time
was exactly according to her own belief of it, and such
as she supposed would convey to the sanguine mind of her
correspondent the hope of everything she was wishing for.
Edmund would be forgiven for being a clergyman, it seemed,
under certain conditions of wealth; and this, she suspected,
was all the conquest of prejudice which he was so ready
to congratulate himself upon.  She had only learnt to think
nothing of consequence but money.



\chapter{Chapter 46}

\gintro{As Fanny} could not doubt that her answer was conveying
a real disappointment, she was rather in expectation,
from her knowledge of Miss Crawford's temper, of being
urged again; and though no second letter arrived for the
space of a week, she had still the same feeling when it
did come.

On receiving it, she could instantly decide on its
containing little writing, and was persuaded of its
having the air of a letter of haste and business.
Its object was unquestionable; and two moments were
enough to start the probability of its being merely
to give her notice that they should be in Portsmouth
that very day, and to throw her into all the agitation
of doubting what she ought to do in such a case.
If two moments, however, can surround with difficulties,
a third can disperse them; and before she had opened
the letter, the possibility of Mr.\ and Miss Crawford's
having applied to her uncle and obtained his permission
was giving her ease.  This was the letter---%

``A most scandalous, ill-natured rumour has just reached me,
and I write, dear Fanny, to warn you against giving the
least credit to it, should it spread into the country.
Depend upon it, there is some mistake, and that a day or two
will clear it up; at any rate, that Henry is blameless,
and in spite of a moment's \emph{etourderie}, thinks of
nobody but you.  Say not a word of it; hear nothing,
surmise nothing, whisper nothing till I write again.
I am sure it will be all hushed up, and nothing proved
but Rushworth's folly.  If they are gone, I would lay
my life they are only gone to Mansfield Park, and Julia
with them.  But why would not you let us come for you?
I wish you may not repent it.---Yours, etc.''

Fanny stood aghast.  As no scandalous, ill-natured rumour
had reached her, it was impossible for her to understand
much of this strange letter.  She could only perceive
that it must relate to Wimpole Street and Mr.\ Crawford,
and only conjecture that something very imprudent had just
occurred in that quarter to draw the notice of the world,
and to excite her jealousy, in Miss Crawford's apprehension,
if she heard it.  Miss Crawford need not be alarmed
for her.  She was only sorry for the parties concerned
and for Mansfield, if the report should spread so far;
but she hoped it might not.  If the Rushworths were gone
themselves to Mansfield, as was to be inferred from
what Miss Crawford said, it was not likely that anything
unpleasant should have preceded them, or at least should
make any impression.

As to Mr.\ Crawford, she hoped it might give him a knowledge
of his own disposition, convince him that he was not capable
of being steadily attached to any one woman in the world,
and shame him from persisting any longer in addressing herself.

It was very strange!  She had begun to think he really
loved her, and to fancy his affection for her something
more than common; and his sister still said that he cared
for nobody else.  Yet there must have been some marked
display of attentions to her cousin, there must have
been some strong indiscretion, since her correspondent
was not of a sort to regard a slight one.

Very uncomfortable she was, and must continue, till she
heard from Miss Crawford again.  It was impossible to
banish the letter from her thoughts, and she could not
relieve herself by speaking of it to any human being.
Miss Crawford need not have urged secrecy with so much warmth;
she might have trusted to her sense of what was due
to her cousin.

The next day came and brought no second letter.
Fanny was disappointed.  She could still think of little
else all the morning; but, when her father came back
in the afternoon with the daily newspaper as usual,
she was so far from expecting any elucidation through such
a channel that the subject was for a moment out of her head.

She was deep in other musing.  The remembrance of her first
evening in that room, of her father and his newspaper,
came across her.  No candle was now wanted.
The sun was yet an hour and half above the horizon.
She felt that she had, indeed, been three months there;
and the sun's rays falling strongly into the parlour,
instead of cheering, made her still more melancholy,
for sunshine appeared to her a totally different thing
in a town and in the country.  Here, its power was only
a glare:  a stifling, sickly glare, serving but to bring
forward stains and dirt that might otherwise have slept.
There was neither health nor gaiety in sunshine in a town.
She sat in a blaze of oppressive heat, in a cloud of
moving dust, and her eyes could only wander from the walls,
marked by her father's head, to the table cut and notched
by her brothers, where stood the tea-board never
thoroughly cleaned, the cups and saucers wiped in streaks,
the milk a mixture of motes floating in thin blue,
and the bread and butter growing every minute more
greasy than even Rebecca's hands had first produced it.
Her father read his newspaper, and her mother lamented
over the ragged carpet as usual, while the tea was
in preparation, and wished Rebecca would mend it;
and Fanny was first roused by his calling out to her,
after humphing and considering over a particular paragraph:
``What's the name of your great cousins in town, Fan?''

A moment's recollection enabled her to say, ``Rushworth, sir.''

``And don't they live in Wimpole Street?''

``Yes, sir.''

``Then, there's the devil to pay among them, that's all!
There'' (holding out the paper to her); ``much good may such
fine relations do you.  I don't know what Sir Thomas may
think of such matters; he may be too much of the courtier
and fine gentleman to like his daughter the less.  But,
by G---! if she belonged to \emph{me}, I'd give her the rope's end
as long as I could stand over her.  A little flogging for
man and woman too would be the best way of preventing such things.''

Fanny read to herself that ``it was with infinite concern
the newspaper had to announce to the world a matrimonial
\emph{fracas} in the family of Mr.\ R. of Wimpole Street;
the beautiful Mrs.\ R., whose name had not long been
enrolled in the lists of Hymen, and who had promised
to become so brilliant a leader in the fashionable world,
having quitted her husband's roof in company with the
well-known and captivating Mr.\ C., the intimate friend
and associate of Mr.\ R., and it was not known even
to the editor of the newspaper whither they were gone.''

``It is a mistake, sir,'' said Fanny instantly; ``it must be
a mistake, it cannot be true; it must mean some other people.''

She spoke from the instinctive wish of delaying shame;
she spoke with a resolution which sprung from despair,
for she spoke what she did not, could not believe herself.
It had been the shock of conviction as she read.  The truth
rushed on her; and how she could have spoken at all, how she
could even have breathed, was afterwards matter of wonder
to herself.

Mr.\ Price cared too little about the report to make her
much answer.  ``It might be all a lie,'' he acknowledged;
``but so many fine ladies were going to the devil nowadays
that way, that there was no answering for anybody.''

``Indeed, I hope it is not true,'' said Mrs.\ Price plaintively;
``it would be so very shocking!  If I have spoken once
to Rebecca about that carpet, I am sure I have spoke at
least a dozen times; have not I, Betsey?  And it would
not be ten minutes' work.''

The horror of a mind like Fanny's, as it received the
conviction of such guilt, and began to take in some part
of the misery that must ensue, can hardly be described.
At first, it was a sort of stupefaction; but every moment
was quickening her perception of the horrible evil.
She could not doubt, she dared not indulge a hope,
of the paragraph being false.  Miss Crawford's letter,
which she had read so often as to make every line her own,
was in frightful conformity with it.  Her eager defence
of her brother, her hope of its being \emph{hushed} \emph{up},
her evident agitation, were all of a piece with something
very bad; and if there was a woman of character in existence,
who could treat as a trifle this sin of the first magnitude,
who would try to gloss it over, and desire to have it
unpunished, she could believe Miss Crawford to be the woman!
Now she could see her own mistake as to \emph{who} were gone,
or \emph{said} to be gone.  It was not Mr.\ and Mrs.\ Rushworth;
it was Mrs.\ Rushworth and Mr.\ Crawford.

Fanny seemed to herself never to have been shocked before.
There was no possibility of rest.  The evening passed
without a pause of misery, the night was totally sleepless.
She passed only from feelings of sickness to shudderings
of horror; and from hot fits of fever to cold.  The event
was so shocking, that there were moments even when her
heart revolted from it as impossible:  when she thought
it could not be.  A woman married only six months ago;
a man professing himself devoted, even \emph{engaged} to another;
that other her near relation; the whole family,
both families connected as they were by tie upon tie;
all friends, all intimate together!  It was too horrible
a confusion of guilt, too gross a complication of evil,
for human nature, not in a state of utter barbarism,
to be capable of! yet her judgment told her it was so.
\emph{His} unsettled affections, wavering with his vanity,
\emph{Maria's} decided attachment, and no sufficient principle
on either side, gave it possibility:  Miss Crawford's
letter stampt it a fact.

What would be the consequence?  Whom would it not injure?
Whose views might it not affect?  Whose peace would it
not cut up for ever?  Miss Crawford, herself, Edmund;
but it was dangerous, perhaps, to tread such ground.
She confined herself, or tried to confine herself, to the simple,
indubitable family misery which must envelop all, if it were
indeed a matter of certified guilt and public exposure.
The mother's sufferings, the father's; there she paused.
Julia's, Tom's, Edmund's; there a yet longer pause.
They were the two on whom it would fall most horribly.
Sir Thomas's parental solicitude and high sense of honour
and decorum, Edmund's upright principles, unsuspicious temper,
and genuine strength of feeling, made her think it
scarcely possible for them to support life and reason
under such disgrace; and it appeared to her that, as far
as this world alone was concerned, the greatest blessing
to every one of kindred with Mrs.\ Rushworth would be
instant annihilation.

Nothing happened the next day, or the next, to weaken
her terrors.  Two posts came in, and brought no refutation,
public or private.  There was no second letter to explain
away the first from Miss Crawford; there was no intelligence
from Mansfield, though it was now full time for her
to hear again from her aunt.  This was an evil omen.
She had, indeed, scarcely the shadow of a hope to soothe
her mind, and was reduced to so low and wan and trembling
a condition, as no mother, not unkind, except Mrs.\ Price
could have overlooked, when the third day did bring the
sickening knock, and a letter was again put into her hands.
It bore the London postmark, and came from Edmund.

``Dear Fanny,---You know our present wretchedness.
May God support you under your share!  We have been here
two days, but there is nothing to be done.  They cannot
be traced.  You may not have heard of the last blow---%
Julia's elopement; she is gone to Scotland with Yates.
She left London a few hours before we entered it.
At any other time this would have been felt dreadfully.
Now it seems nothing; yet it is an heavy aggravation.
My father is not overpowered.  More cannot be hoped.
He is still able to think and act; and I write,
by his desire, to propose your returning home.
He is anxious to get you there for my mother's sake.
I shall be at Portsmouth the morning after you receive this,
and hope to find you ready to set off for Mansfield.
My father wishes you to invite Susan to go with you for a
few months.  Settle it as you like; say what is proper;
I am sure you will feel such an instance of his
kindness at such a moment!  Do justice to his meaning,
however I may confuse it.  You may imagine something
of my present state.  There is no end of the evil let
loose upon us.  You will see me early by the mail.---%
Yours, etc.''

Never had Fanny more wanted a cordial.  Never had she felt
such a one as this letter contained.  To-morrow! to leave
Portsmouth to-morrow! She was, she felt she was, in the
greatest danger of being exquisitely happy, while so many
were miserable.  The evil which brought such good to her!
She dreaded lest she should learn to be insensible of it.
To be going so soon, sent for so kindly, sent for as
a comfort, and with leave to take Susan, was altogether
such a combination of blessings as set her heart in
a glow, and for a time seemed to distance every pain,
and make her incapable of suitably sharing the distress
even of those whose distress she thought of most.
Julia's elopement could affect her comparatively but little;
she was amazed and shocked; but it could not occupy her,
could not dwell on her mind.  She was obliged to call
herself to think of it, and acknowledge it to be terrible
and grievous, or it was escaping her, in the midst of all
the agitating pressing joyful cares attending this summons
to herself.

There is nothing like employment, active indispensable employment,
for relieving sorrow.  Employment, even melancholy,
may dispel melancholy, and her occupations were hopeful.
She had so much to do, that not even the horrible
story of Mrs.\ Rushworth---now fixed to the last point
of certainty could affect her as it had done before.
She had not time to be miserable.  Within twenty-four
hours she was hoping to be gone; her father and mother
must be spoken to, Susan prepared, everything got ready.
Business followed business; the day was hardly long enough.
The happiness she was imparting, too, happiness very little
alloyed by the black communication which must briefly
precede it---the joyful consent of her father and mother
to Susan's going with her---the general satisfaction with
which the going of both seemed regarded, and the ecstasy
of Susan herself, was all serving to support her spirits.

The affliction of the Bertrams was little felt in the family.
Mrs.\ Price talked of her poor sister for a few minutes,
but how to find anything to hold Susan's clothes,
because Rebecca took away all the boxes and spoilt them,
was much more in her thoughts:  and as for Susan,
now unexpectedly gratified in the first wish of her heart,
and knowing nothing personally of those who had sinned,
or of those who were sorrowing---if she could help rejoicing
from beginning to end, it was as much as ought to be expected
from human virtue at fourteen.

As nothing was really left for the decision of Mrs.\ Price,
or the good offices of Rebecca, everything was rationally
and duly accomplished, and the girls were ready for
the morrow.  The advantage of much sleep to prepare
them for their journey was impossible.  The cousin
who was travelling towards them could hardly have less
than visited their agitated spirits---one all happiness,
the other all varying and indescribable perturbation.

By eight in the morning Edmund was in the house.  The girls
heard his entrance from above, and Fanny went down.
The idea of immediately seeing him, with the knowledge
of what he must be suffering, brought back all her own
first feelings.  He so near her, and in misery.  She was
ready to sink as she entered the parlour.  He was alone,
and met her instantly; and she found herself pressed
to his heart with only these words, just articulate,
``My Fanny, my only sister; my only comfort now!''
She could say nothing; nor for some minutes could he
say more.

He turned away to recover himself, and when he spoke again,
though his voice still faltered, his manner shewed
the wish of self-command, and the resolution of avoiding
any farther allusion.  ``Have you breakfasted?  When shall
you be ready?  Does Susan go?'' were questions following
each other rapidly.  His great object was to be off
as soon as possible.  When Mansfield was considered,
time was precious; and the state of his own mind made
him find relief only in motion.  It was settled that he
should order the carriage to the door in half an hour.
Fanny answered for their having breakfasted and being quite
ready in half an hour.  He had already ate, and declined
staying for their meal.  He would walk round the ramparts,
and join them with the carriage.  He was gone again;
glad to get away even from Fanny.

He looked very ill; evidently suffering under
violent emotions, which he was determined to suppress.
She knew it must be so, but it was terrible to her.

The carriage came; and he entered the house again at
the same moment, just in time to spend a few minutes with
the family, and be a witness---but that he saw nothing---%
of the tranquil manner in which the daughters were
parted with, and just in time to prevent their sitting
down to the breakfast-table, which, by dint of much
unusual activity, was quite and completely ready as
the carriage drove from the door.  Fanny's last meal
in her father's house was in character with her first:
she was dismissed from it as hospitably as she had been welcomed.

How her heart swelled with joy and gratitude as she
passed the barriers of Portsmouth, and how Susan's face
wore its broadest smiles, may be easily conceived.
Sitting forwards, however, and screened by her bonnet,
those smiles were unseen.

The journey was likely to be a silent one.  Edmund's deep
sighs often reached Fanny.  Had he been alone with her,
his heart must have opened in spite of every resolution;
but Susan's presence drove him quite into himself, and his
attempts to talk on indifferent subjects could never be
long supported.

Fanny watched him with never-failing solicitude,
and sometimes catching his eye, revived an affectionate smile,
which comforted her; but the first day's journey passed
without her hearing a word from him on the subjects
that were weighing him down.  The next morning produced
a little more.  Just before their setting out from Oxford,
while Susan was stationed at a window, in eager observation
of the departure of a large family from the inn,
the other two were standing by the fire; and Edmund,
particularly struck by the alteration in Fanny's looks,
and from his ignorance of the daily evils of her
father's house, attributing an undue share of the change,
attributing \emph{all} to the recent event, took her hand,
and said in a low, but very expressive tone, ``No wonder---%
you must feel it---you must suffer.  How a man who had
once loved, could desert you!  But \emph{yours}---your regard
was new compared with\gdash{}Fanny, think of \emph{me}!''

The first division of their journey occupied a long day,
and brought them, almost knocked up, to Oxford;
but the second was over at a much earlier hour.
They were in the environs of Mansfield long before
the usual dinner-time, and as they approached the
beloved place, the hearts of both sisters sank a little.
Fanny began to dread the meeting with her aunts and Tom,
under so dreadful a humiliation; and Susan to feel with
some anxiety, that all her best manners, all her lately
acquired knowledge of what was practised here, was on
the point of being called into action.  Visions of good
and ill breeding, of old vulgarisms and new gentilities,
were before her; and she was meditating much upon
silver forks, napkins, and finger-glasses. Fanny had
been everywhere awake to the difference of the country
since February; but when they entered the Park her
perceptions and her pleasures were of the keenest sort.
It was three months, full three months, since her
quitting it, and the change was from winter to summer.
Her eye fell everywhere on lawns and plantations of the
freshest green; and the trees, though not fully clothed,
were in that delightful state when farther beauty is known
to be at hand, and when, while much is actually given
to the sight, more yet remains for the imagination.
Her enjoyment, however, was for herself alone.  Edmund could
not share it.  She looked at him, but he was leaning back,
sunk in a deeper gloom than ever, and with eyes closed,
as if the view of cheerfulness oppressed him, and the
lovely scenes of home must be shut out.

It made her melancholy again; and the knowledge of what must
be enduring there, invested even the house, modern, airy,
and well situated as it was, with a melancholy aspect.

By one of the suffering party within they were expected
with such impatience as she had never known before.
Fanny had scarcely passed the solemn-looking servants,
when Lady Bertram came from the drawing-room to meet her;
came with no indolent step; and falling on her neck, said,
``Dear Fanny! now I shall be comfortable.''



\chapter{Chapter 47}

\gintro{It had been} a miserable party, each of the three believing
themselves most miserable.  Mrs.\ Norris, however, as most
attached to Maria, was really the greatest sufferer.
Maria was her first favourite, the dearest of all;
the match had been her own contriving, as she had been
wont with such pride of heart to feel and say, and this
conclusion of it almost overpowered her.

She was an altered creature, quieted, stupefied, indifferent to
everything that passed.  The being left with her sister
and nephew, and all the house under her care, had been
an advantage entirely thrown away; she had been unable
to direct or dictate, or even fancy herself useful.
When really touched by affliction, her active powers
had been all benumbed; and neither Lady Bertram nor Tom
had received from her the smallest support or attempt
at support.  She had done no more for them than they
had done for each other.  They had been all solitary,
helpless, and forlorn alike; and now the arrival of the
others only established her superiority in wretchedness.
Her companions were relieved, but there was no good
for \emph{her}.  Edmund was almost as welcome to his brother
as Fanny to her aunt; but Mrs.\ Norris, instead of having
comfort from either, was but the more irritated by the
sight of the person whom, in the blindness of her anger,
she could have charged as the daemon of the piece.
Had Fanny accepted Mr.\ Crawford this could not have happened.

Susan too was a grievance.  She had not spirits to notice
her in more than a few repulsive looks, but she felt
her as a spy, and an intruder, and an indigent niece,
and everything most odious.  By her other aunt, Susan was
received with quiet kindness.  Lady Bertram could not
give her much time, or many words, but she felt her,
as Fanny's sister, to have a claim at Mansfield,
and was ready to kiss and like her; and Susan was more
than satisfied, for she came perfectly aware that nothing
but ill-humour was to be expected from aunt Norris;
and was so provided with happiness, so strong in that
best of blessings, an escape from many certain evils,
that she could have stood against a great deal more
indifference than she met with from the others.

She was now left a good deal to herself, to get acquainted
with the house and grounds as she could, and spent her
days very happily in so doing, while those who might
otherwise have attended to her were shut up, or wholly
occupied each with the person quite dependent on them,
at this time, for everything like comfort; Edmund trying
to bury his own feelings in exertions for the relief
of his brother's, and Fanny devoted to her aunt Bertram,
returning to every former office with more than former zeal,
and thinking she could never do enough for one who seemed
so much to want her.

To talk over the dreadful business with Fanny, talk and lament,
was all Lady Bertram's consolation.  To be listened to and
borne with, and hear the voice of kindness and sympathy
in return, was everything that could be done for her.
To be otherwise comforted was out of the question.
The case admitted of no comfort.  Lady Bertram did not
think deeply, but, guided by Sir Thomas, she thought
justly on all important points; and she saw, therefore,
in all its enormity, what had happened, and neither
endeavoured herself, nor required Fanny to advise her,
to think little of guilt and infamy.

Her affections were not acute, nor was her mind tenacious.
After a time, Fanny found it not impossible to direct
her thoughts to other subjects, and revive some interest
in the usual occupations; but whenever Lady Bertram \emph{was}
fixed on the event, she could see it only in one light,
as comprehending the loss of a daughter, and a disgrace
never to be wiped off.

Fanny learnt from her all the particulars which had
yet transpired.  Her aunt was no very methodical narrator,
but with the help of some letters to and from Sir Thomas,
and what she already knew herself, and could reasonably
combine, she was soon able to understand quite as much
as she wished of the circumstances attending the story.

Mrs.\ Rushworth had gone, for the Easter holidays,
to Twickenham, with a family whom she had just grown
intimate with:  a family of lively, agreeable manners,
and probably of morals and discretion to suit, for to \emph{their}
house Mr.\ Crawford had constant access at all times.
His having been in the same neighbourhood Fanny already knew.
Mr.\ Rushworth had been gone at this time to Bath, to pass
a few days with his mother, and bring her back to town,
and Maria was with these friends without any restraint,
without even Julia; for Julia had removed from Wimpole Street
two or three weeks before, on a visit to some relations
of Sir Thomas; a removal which her father and mother were
now disposed to attribute to some view of convenience
on Mr.\ Yates's account.  Very soon after the Rushworths'
return to Wimpole Street, Sir Thomas had received a
letter from an old and most particular friend in London,
who hearing and witnessing a good deal to alarm him
in that quarter, wrote to recommend Sir Thomas's coming
to London himself, and using his influence with his
daughter to put an end to the intimacy which was already
exposing her to unpleasant remarks, and evidently making
Mr.\ Rushworth uneasy.

Sir Thomas was preparing to act upon this letter, without
communicating its contents to any creature at Mansfield,
when it was followed by another, sent express from the
same friend, to break to him the almost desperate situation
in which affairs then stood with the young people.
Mrs.\ Rushworth had left her husband's house:  Mr.\ Rushworth
had been in great anger and distress to \emph{him} (Mr.\ Harding)
for his advice; Mr.\ Harding feared there had been \emph{at}
\emph{least} very flagrant indiscretion.  The maidservant
of Mrs.\ Rushworth, senior, threatened alarmingly.  He was
doing all in his power to quiet everything, with the hope
of Mrs.\ Rushworth's return, but was so much counteracted
in Wimpole Street by the influence of Mr.\ Rushworth's mother,
that the worst consequences might be apprehended.

This dreadful communication could not be kept from the rest
of the family.  Sir Thomas set off, Edmund would go with him,
and the others had been left in a state of wretchedness,
inferior only to what followed the receipt of the next
letters from London.  Everything was by that time public
beyond a hope.  The servant of Mrs.\ Rushworth, the mother,
had exposure in her power, and supported by her mistress,
was not to be silenced.  The two ladies, even in the short
time they had been together, had disagreed; and the bitterness
of the elder against her daughter-in-law might perhaps arise
almost as much from the personal disrespect with which
she had herself been treated as from sensibility for her son.

However that might be, she was unmanageable.  But had she
been less obstinate, or of less weight with her son,
who was always guided by the last speaker, by the person
who could get hold of and shut him up, the case would
still have been hopeless, for Mrs.\ Rushworth did not
appear again, and there was every reason to conclude
her to be concealed somewhere with Mr.\ Crawford,
who had quitted his uncle's house, as for a journey,
on the very day of her absenting herself.

Sir Thomas, however, remained yet a little longer in town,
in the hope of discovering and snatching her from farther vice,
though all was lost on the side of character.

\emph{His} present state Fanny could hardly bear to think of.
There was but one of his children who was not at this time
a source of misery to him.  Tom's complaints had been
greatly heightened by the shock of his sister's conduct,
and his recovery so much thrown back by it, that even
Lady Bertram had been struck by the difference, and all
her alarms were regularly sent off to her husband;
and Julia's elopement, the additional blow which had met
him on his arrival in London, though its force had been
deadened at the moment, must, she knew, be sorely felt.
She saw that it was.  His letters expressed how much he
deplored it.  Under any circumstances it would have been
an unwelcome alliance; but to have it so clandestinely
formed, and such a period chosen for its completion,
placed Julia's feelings in a most unfavourable light,
and severely aggravated the folly of her choice.
He called it a bad thing, done in the worst manner,
and at the worst time; and though Julia was yet as more
pardonable than Maria as folly than vice, he could not
but regard the step she had taken as opening the worst
probabilities of a conclusion hereafter like her sister's.
Such was his opinion of the set into which she had
thrown herself.

Fanny felt for him most acutely.  He could have no comfort
but in Edmund.  Every other child must be racking his heart.
His displeasure against herself she trusted, reasoning
differently from Mrs.\ Norris, would now be done away.
\emph{She} should be justified.  Mr.\ Crawford would have
fully acquitted her conduct in refusing him; but this,
though most material to herself, would be poor consolation
to Sir Thomas.  Her uncle's displeasure was terrible to her;
but what could her justification or her gratitude and
attachment do for him?  His stay must be on Edmund alone.

She was mistaken, however, in supposing that Edmund gave
his father no present pain.  It was of a much less poignant
nature than what the others excited; but Sir Thomas
was considering his happiness as very deeply involved
in the offence of his sister and friend; cut off by it,
as he must be, from the woman whom he had been pursuing
with undoubted attachment and strong probability of success;
and who, in everything but this despicable brother,
would have been so eligible a connexion.  He was aware
of what Edmund must be suffering on his own behalf,
in addition to all the rest, when they were in town:
he had seen or conjectured his feelings; and, having reason
to think that one interview with Miss Crawford had taken place,
from which Edmund derived only increased distress, had been
as anxious on that account as on others to get him out of town,
and had engaged him in taking Fanny home to her aunt,
with a view to his relief and benefit, no less than theirs.
Fanny was not in the secret of her uncle's feelings,
Sir Thomas not in the secret of Miss Crawford's character.
Had he been privy to her conversation with his son, he would
not have wished her to belong to him, though her twenty
thousand pounds had been forty.

That Edmund must be for ever divided from Miss Crawford did
not admit of a doubt with Fanny; and yet, till she knew
that he felt the same, her own conviction was insufficient.
She thought he did, but she wanted to be assured of it.
If he would now speak to her with the unreserve which
had sometimes been too much for her before, it would
be most consoling; but \emph{that} she found was not to be.
She seldom saw him:  never alone.  He probably avoided
being alone with her.  What was to be inferred?  That his
judgment submitted to all his own peculiar and bitter share
of this family affliction, but that it was too keenly
felt to be a subject of the slightest communication.
This must be his state.  He yielded, but it was with
agonies which did not admit of speech.  Long, long would
it be ere Miss Crawford's name passed his lips again,
or she could hope for a renewal of such confidential
intercourse as had been.

It \emph{was} long.  They reached Mansfield on Thursday,
and it was not till Sunday evening that Edmund began
to talk to her on the subject.  Sitting with her on
Sunday evening---a wet Sunday evening---the very time of
all others when, if a friend is at hand, the heart must
be opened, and everything told; no one else in the room,
except his mother, who, after hearing an affecting sermon,
had cried herself to sleep, it was impossible not to speak;
and so, with the usual beginnings, hardly to be traced
as to what came first, and the usual declaration that
if she would listen to him for a few minutes, he should
be very brief, and certainly never tax her kindness
in the same way again; she need not fear a repetition;
it would be a subject prohibited entirely:  he entered
upon the luxury of relating circumstances and sensations
of the first interest to himself, to one of whose
affectionate sympathy he was quite convinced.

How Fanny listened, with what curiosity and concern,
what pain and what delight, how the agitation of his
voice was watched, and how carefully her own eyes were
fixed on any object but himself, may be imagined.
The opening was alarming.  He had seen Miss Crawford.
He had been invited to see her.  He had received a note
from Lady Stornaway to beg him to call; and regarding
it as what was meant to be the last, last interview
of friendship, and investing her with all the feelings
of shame and wretchedness which Crawford's sister ought
to have known, he had gone to her in such a state of mind,
so softened, so devoted, as made it for a few moments
impossible to Fanny's fears that it should be the last.
But as he proceeded in his story, these fears were over.
She had met him, he said, with a serious---certainly a serious---%
even an agitated air; but before he had been able
to speak one intelligible sentence, she had introduced
the subject in a manner which he owned had shocked him.
``\,`I heard you were in town,' said she; `I wanted to see you.
Let us talk over this sad business.  What can equal the folly
of our two relations?'  I could not answer, but I believe
my looks spoke.  She felt reproved.  Sometimes how quick
to feel!  With a graver look and voice she then added,
`I do not mean to defend Henry at your sister's expense.'
So she began, but how she went on, Fanny, is not fit,
is hardly fit to be repeated to you.  I cannot recall
all her words.  I would not dwell upon them if I could.
Their substance was great anger at the \emph{folly} of each.
She reprobated her brother's folly in being drawn on
by a woman whom he had never cared for, to do what must
lose him the woman he adored; but still more the folly of
poor Maria, in sacrificing such a situation, plunging into
such difficulties, under the idea of being really loved
by a man who had long ago made his indifference clear.
Guess what I must have felt.  To hear the woman whom---%
no harsher name than folly given!  So voluntarily,
so freely, so coolly to canvass it!  No reluctance,
no horror, no feminine, shall I say, no modest loathings?
This is what the world does.  For where, Fanny, shall we
find a woman whom nature had so richly endowed?  Spoilt,
spoilt!''

After a little reflection, he went on with a sort
of desperate calmness.  ``I will tell you everything,
and then have done for ever.  She saw it only as folly,
and that folly stamped only by exposure.  The want of
common discretion, of caution:  his going down to Richmond
for the whole time of her being at Twickenham; her putting
herself in the power of a servant; it was the detection,
in short---oh, Fanny! it was the detection, not the offence,
which she reprobated.  It was the imprudence which had
brought things to extremity, and obliged her brother
to give up every dearer plan in order to fly with her.''

He stopt.  ``And what,'' said Fanny (believing herself
required to speak), ``what could you say?''

``Nothing, nothing to be understood.  I was like a man stunned.
She went on, began to talk of you; yes, then she began
to talk of you, regretting, as well she might, the loss
of such a---.  There she spoke very rationally.  But she
has always done justice to you.  `He has thrown away,'
said she, `such a woman as he will never see again.
She would have fixed him; she would have made him happy
for ever.'  My dearest Fanny, I am giving you, I hope,
more pleasure than pain by this retrospect of what might
have been---but what never can be now.  You do not wish me
to be silent?  If you do, give me but a look, a word, and I
have done.''

No look or word was given.

``Thank God,'' said he.  ``We were all disposed to wonder,
but it seems to have been the merciful appointment
of Providence that the heart which knew no guile
should not suffer.  She spoke of you with high praise
and warm affection; yet, even here, there was alloy,
a dash of evil; for in the midst of it she could exclaim,
`Why would not she have him?  It is all her fault.
Simple girl!  I shall never forgive her.  Had she accepted
him as she ought, they might now have been on the point
of marriage, and Henry would have been too happy and too
busy to want any other object.  He would have taken
no pains to be on terms with Mrs.\ Rushworth again.
It would have all ended in a regular standing flirtation,
in yearly meetings at Sotherton and Everingham.'  Could you
have believed it possible?  But the charm is broken.
My eyes are opened.''

``Cruel!'' said Fanny, ``quite cruel.  At such a moment to
give way to gaiety, to speak with lightness, and to you!
Absolute cruelty.''

``Cruelty, do you call it?  We differ there.  No, hers is
not a cruel nature.  I do not consider her as meaning
to wound my feelings.  The evil lies yet deeper:
in her total ignorance, unsuspiciousness of there being
such feelings; in a perversion of mind which made it
natural to her to treat the subject as she did.  She was
speaking only as she had been used to hear others speak,
as she imagined everybody else would speak.  Hers are
not faults of temper.  She would not voluntarily give
unnecessary pain to any one, and though I may deceive myself,
I cannot but think that for me, for my feelings, she would---%
Hers are faults of principle, Fanny; of blunted delicacy
and a corrupted, vitiated mind.  Perhaps it is best for me,
since it leaves me so little to regret.  Not so, however.
Gladly would I submit to all the increased pain of
losing her, rather than have to think of her as I do.
I told her so.''

``Did you?''

``Yes; when I left her I told her so.''

``How long were you together?''

``Five-and-twenty minutes.  Well, she went on to say that
what remained now to be done was to bring about a marriage
between them.  She spoke of it, Fanny, with a steadier
voice than I can.''  He was obliged to pause more than once
as he continued.  ``\,`We must persuade Henry to marry her,'
said she; `and what with honour, and the certainty of having
shut himself out for ever from Fanny, I do not despair
of it.  Fanny he must give up.  I do not think that even
\emph{he} could now hope to succeed with one of her stamp,
and therefore I hope we may find no insuperable difficulty.
My influence, which is not small shall all go that way;
and when once married, and properly supported by her
own family, people of respectability as they are, she may
recover her footing in society to a certain degree.
In some circles, we know, she would never be admitted,
but with good dinners, and large parties, there will
always be those who will be glad of her acquaintance;
and there is, undoubtedly, more liberality and candour
on those points than formerly.  What I advise is,
that your father be quiet.  Do not let him injure his own
cause by interference.  Persuade him to let things take
their course.  If by any officious exertions of his,
she is induced to leave Henry's protection, there will be
much less chance of his marrying her than if she remain
with him.  I know how he is likely to be influenced.
Let Sir Thomas trust to his honour and compassion, and it
may all end well; but if he get his daughter away, it will
be destroying the chief hold.'\,''

After repeating this, Edmund was so much affected that Fanny,
watching him with silent, but most tender concern,
was almost sorry that the subject had been entered
on at all.  It was long before he could speak again.
At last, ``Now, Fanny,'' said he, ``I shall soon have done.
I have told you the substance of all that she said.
As soon as I could speak, I replied that I had not
supposed it possible, coming in such a state of mind
into that house as I had done, that anything could
occur to make me suffer more, but that she had been
inflicting deeper wounds in almost every sentence.
That though I had, in the course of our acquaintance,
been often sensible of some difference in our opinions,
on points, too, of some moment, it had not entered my
imagination to conceive the difference could be such as she
had now proved it.  That the manner in which she treated
the dreadful crime committed by her brother and my sister
(with whom lay the greater seduction I pretended not to say),
but the manner in which she spoke of the crime itself,
giving it every reproach but the right; considering its ill
consequences only as they were to be braved or overborne
by a defiance of decency and impudence in wrong; and last
of all, and above all, recommending to us a compliance,
a compromise, an acquiescence in the continuance of the sin,
on the chance of a marriage which, thinking as I now thought
of her brother, should rather be prevented than sought;
all this together most grievously convinced me that I had
never understood her before, and that, as far as related
to mind, it had been the creature of my own imagination,
not Miss Crawford, that I had been too apt to dwell on
for many months past.  That, perhaps, it was best for me;
I had less to regret in sacrificing a friendship, feelings,
hopes which must, at any rate, have been torn from me now.
And yet, that I must and would confess that, could I
have restored her to what she had appeared to me before,
I would infinitely prefer any increase of the pain
of parting, for the sake of carrying with me the right of
tenderness and esteem.  This is what I said, the purport
of it; but, as you may imagine, not spoken so collectedly
or methodically as I have repeated it to you.  She was
astonished, exceedingly astonished---more than astonished.
I saw her change countenance.  She turned extremely red.
I imagined I saw a mixture of many feelings:  a great,
though short struggle; half a wish of yielding to truths,
half a sense of shame, but habit, habit carried it.
She would have laughed if she could.  It was a sort of laugh,
as she answered, `A pretty good lecture, upon my word.
Was it part of your last sermon?  At this rate you will
soon reform everybody at Mansfield and Thornton Lacey;
and when I hear of you next, it may be as a celebrated preacher
in some great society of Methodists, or as a missionary
into foreign parts.'  She tried to speak carelessly,
but she was not so careless as she wanted to appear.
I only said in reply, that from my heart I wished her well,
and earnestly hoped that she might soon learn to think
more justly, and not owe the most valuable knowledge we
could any of us acquire, the knowledge of ourselves and of
our duty, to the lessons of affliction, and immediately
left the room.  I had gone a few steps, Fanny, when I
heard the door open behind me.  `Mr.\ Bertram,' said she.
I looked back.  `Mr.\ Bertram,' said she, with a smile;
but it was a smile ill-suited to the conversation that
had passed, a saucy playful smile, seeming to invite
in order to subdue me; at least it appeared so to me.
I resisted; it was the impulse of the moment to resist,
and still walked on.  I have since, sometimes, for a moment,
regretted that I did not go back, but I know I was right,
and such has been the end of our acquaintance.  And what
an acquaintance has it been!  How have I been deceived!
Equally in brother and sister deceived!  I thank you for
your patience, Fanny.  This has been the greatest relief,
and now we will have done.''

And such was Fanny's dependence on his words, that for five
minutes she thought they \emph{had} done.  Then, however, it all
came on again, or something very like it, and nothing
less than Lady Bertram's rousing thoroughly up could
really close such a conversation.  Till that happened,
they continued to talk of Miss Crawford alone, and how she
had attached him, and how delightful nature had made her,
and how excellent she would have been, had she fallen into
good hands earlier.  Fanny, now at liberty to speak openly,
felt more than justified in adding to his knowledge
of her real character, by some hint of what share his
brother's state of health might be supposed to have in
her wish for a complete reconciliation.  This was not an
agreeable intimation.  Nature resisted it for a while.
It would have been a vast deal pleasanter to have had
her more disinterested in her attachment; but his vanity
was not of a strength to fight long against reason.
He submitted to believe that Tom's illness had influenced her,
only reserving for himself this consoling thought,
that considering the many counteractions of opposing habits,
she had certainly been \emph{more} attached to him than could
have been expected, and for his sake been more near
doing right.  Fanny thought exactly the same; and they were
also quite agreed in their opinion of the lasting effect,
the indelible impression, which such a disappointment
must make on his mind.  Time would undoubtedly abate
somewhat of his sufferings, but still it was a sort
of thing which he never could get entirely the better of;
and as to his ever meeting with any other woman who could---%
it was too impossible to be named but with indignation.
Fanny's friendship was all that he had to cling to.



\chapter{Chapter 48}

\gintro{Let other pens dwell} on guilt and misery.  I quit such odious
subjects as soon as I can, impatient to restore everybody,
not greatly in fault themselves, to tolerable comfort,
and to have done with all the rest.

My Fanny, indeed, at this very time, I have the satisfaction
of knowing, must have been happy in spite of everything.
She must have been a happy creature in spite of all that she felt,
or thought she felt, for the distress of those around her.
She had sources of delight that must force their way.
She was returned to Mansfield Park, she was useful,
she was beloved; she was safe from Mr.\ Crawford;
and when Sir Thomas came back she had every proof that
could be given in his then melancholy state of spirits,
of his perfect approbation and increased regard;
and happy as all this must make her, she would still have
been happy without any of it, for Edmund was no longer
the dupe of Miss Crawford.

It is true that Edmund was very far from happy himself.
He was suffering from disappointment and regret,
grieving over what was, and wishing for what could never be.
She knew it was so, and was sorry; but it was with a
sorrow so founded on satisfaction, so tending to ease,
and so much in harmony with every dearest sensation,
that there are few who might not have been glad to exchange
their greatest gaiety for it.

Sir Thomas, poor Sir Thomas, a parent, and conscious of errors
in his own conduct as a parent, was the longest to suffer.
He felt that he ought not to have allowed the marriage;
that his daughter's sentiments had been sufficiently known
to him to render him culpable in authorising it; that in so
doing he had sacrificed the right to the expedient, and been
governed by motives of selfishness and worldly wisdom.
These were reflections that required some time to soften;
but time will do almost everything; and though little
comfort arose on Mrs.\ Rushworth's side for the misery she
had occasioned, comfort was to be found greater than he had
supposed in his other children.  Julia's match became a less
desperate business than he had considered it at first.
She was humble, and wishing to be forgiven; and Mr.\ Yates,
desirous of being really received into the family, was disposed
to look up to him and be guided.  He was not very solid;
but there was a hope of his becoming less trifling,
of his being at least tolerably domestic and quiet;
and at any rate, there was comfort in finding his estate
rather more, and his debts much less, than he had feared,
and in being consulted and treated as the friend best
worth attending to.  There was comfort also in Tom,
who gradually regained his health, without regaining the
thoughtlessness and selfishness of his previous habits.
He was the better for ever for his illness.  He had suffered,
and he had learned to think:  two advantages that he had
never known before; and the self-reproach arising from
the deplorable event in Wimpole Street, to which he felt
himself accessory by all the dangerous intimacy of his
unjustifiable theatre, made an impression on his mind which,
at the age of six-and-twenty, with no want of sense
or good companions, was durable in its happy effects.
He became what he ought to be:  useful to his father,
steady and quiet, and not living merely for himself.

Here was comfort indeed! and quite as soon as Sir
Thomas could place dependence on such sources of good,
Edmund was contributing to his father's ease by improvement
in the only point in which he had given him pain before---%
improvement in his spirits.  After wandering about and
sitting under trees with Fanny all the summer evenings,
he had so well talked his mind into submission as to be
very tolerably cheerful again.

These were the circumstances and the hopes which gradually
brought their alleviation to Sir Thomas, deadening his sense
of what was lost, and in part reconciling him to himself;
though the anguish arising from the conviction of his
own errors in the education of his daughters was never
to be entirely done away.

Too late he became aware how unfavourable to the character
of any young people must be the totally opposite treatment
which Maria and Julia had been always experiencing at home,
where the excessive indulgence and flattery of their aunt
had been continually contrasted with his own severity.
He saw how ill he had judged, in expecting to counteract
what was wrong in Mrs.\ Norris by its reverse in himself;
clearly saw that he had but increased the evil by teaching
them to repress their spirits in his presence so as to make
their real disposition unknown to him, and sending them
for all their indulgences to a person who had been able
to attach them only by the blindness of her affection,
and the excess of her praise.

Here had been grievous mismanagement; but, bad as it was,
he gradually grew to feel that it had not been the most
direful mistake in his plan of education.  Something must
have been wanting \emph{within}, or time would have worn
away much of its ill effect.  He feared that principle,
active principle, had been wanting; that they had never
been properly taught to govern their inclinations and
tempers by that sense of duty which can alone suffice.
They had been instructed theoretically in their religion,
but never required to bring it into daily practice.
To be distinguished for elegance and accomplishments,
the authorised object of their youth, could have had no
useful influence that way, no moral effect on the mind.
He had meant them to be good, but his cares had been directed
to the understanding and manners, not the disposition;
and of the necessity of self-denial and humility,
he feared they had never heard from any lips that could
profit them.

Bitterly did he deplore a deficiency which now he
could scarcely comprehend to have been possible.
Wretchedly did he feel, that with all the cost and care
of an anxious and expensive education, he had brought up
his daughters without their understanding their first duties,
or his being acquainted with their character and temper.

The high spirit and strong passions of Mrs.\ Rushworth,
especially, were made known to him only in their sad result.
She was not to be prevailed on to leave Mr.\ Crawford.
She hoped to marry him, and they continued together
till she was obliged to be convinced that such hope
was vain, and till the disappointment and wretchedness
arising from the conviction rendered her temper so bad,
and her feelings for him so like hatred, as to make them
for a while each other's punishment, and then induce
a voluntary separation.

She had lived with him to be reproached as the ruin
of all his happiness in Fanny, and carried away no better
consolation in leaving him than that she \emph{had} divided them.
What can exceed the misery of such a mind in such a situation?

Mr.\ Rushworth had no difficulty in procuring a divorce;
and so ended a marriage contracted under such circumstances
as to make any better end the effect of good luck not to
be reckoned on.  She had despised him, and loved another;
and he had been very much aware that it was so.
The indignities of stupidity, and the disappointments
of selfish passion, can excite little pity.  His punishment
followed his conduct, as did a deeper punishment the
deeper guilt of his wife.  \emph{He} was released from the
engagement to be mortified and unhappy, till some other
pretty girl could attract him into matrimony again,
and he might set forward on a second, and, it is to
be hoped, more prosperous trial of the state:  if duped,
to be duped at least with good humour and good luck;
while she must withdraw with infinitely stronger feelings
to a retirement and reproach which could allow no second
spring of hope or character.

Where she could be placed became a subject of most
melancholy and momentous consultation.  Mrs.\ Norris,
whose attachment seemed to augment with the demerits
of her niece, would have had her received at home and
countenanced by them all.  Sir Thomas would not hear of it;
and Mrs.\ Norris's anger against Fanny was so much the greater,
from considering \emph{her} residence there as the motive.
She persisted in placing his scruples to \emph{her} account,
though Sir Thomas very solemnly assured her that,
had there been no young woman in question, had there
been no young person of either sex belonging to him,
to be endangered by the society or hurt by the character
of Mrs.\ Rushworth, he would never have offered so great an
insult to the neighbourhood as to expect it to notice her.
As a daughter, he hoped a penitent one, she should be
protected by him, and secured in every comfort, and supported
by every encouragement to do right, which their relative
situations admitted; but farther than \emph{that} he could not go.
Maria had destroyed her own character, and he would not,
by a vain attempt to restore what never could be restored,
by affording his sanction to vice, or in seeking to lessen
its disgrace, be anywise accessory to introducing such
misery in another man's family as he had known himself.

It ended in Mrs.\ Norris's resolving to quit Mansfield
and devote herself to her unfortunate Maria, and in an
establishment being formed for them in another country,
remote and private, where, shut up together with little society,
on one side no affection, on the other no judgment,
it may be reasonably supposed that their tempers became
their mutual punishment.

Mrs.\ Norris's removal from Mansfield was the great supplementary
comfort of Sir Thomas's life.  His opinion of her had
been sinking from the day of his return from Antigua:
in every transaction together from that period, in their
daily intercourse, in business, or in chat, she had been
regularly losing ground in his esteem, and convincing
him that either time had done her much disservice,
or that he had considerably over-rated her sense,
and wonderfully borne with her manners before.  He had
felt her as an hourly evil, which was so much the worse,
as there seemed no chance of its ceasing but with life;
she seemed a part of himself that must be borne for ever.
To be relieved from her, therefore, was so great a
felicity that, had she not left bitter remembrances
behind her, there might have been danger of his learning
almost to approve the evil which produced such a good.

She was regretted by no one at Mansfield.  She had never
been able to attach even those she loved best; and since
Mrs.\ Rushworth's elopement, her temper had been in a state
of such irritation as to make her everywhere tormenting.
Not even Fanny had tears for aunt Norris, not even when
she was gone for ever.

That Julia escaped better than Maria was owing, in some measure,
to a favourable difference of disposition and circumstance,
but in a greater to her having been less the darling
of that very aunt, less flattered and less spoilt.
Her beauty and acquirements had held but a second place.
She had been always used to think herself a little inferior
to Maria.  Her temper was naturally the easiest of the two;
her feelings, though quick, were more controllable,
and education had not given her so very hurtful a degree
of self-consequence.

She had submitted the best to the disappointment
in Henry Crawford.  After the first bitterness of the
conviction of being slighted was over, she had been
tolerably soon in a fair way of not thinking of him again;
and when the acquaintance was renewed in town,
and Mr.\ Rushworth's house became Crawford's object,
she had had the merit of withdrawing herself from it,
and of chusing that time to pay a visit to her other friends,
in order to secure herself from being again too much attracted.
This had been her motive in going to her cousin's.
Mr.\ Yates's convenience had had nothing to do with it.
She had been allowing his attentions some time,
but with very little idea of ever accepting him;
and had not her sister's conduct burst forth as it did,
and her increased dread of her father and of home,
on that event, imagining its certain consequence to herself
would be greater severity and restraint, made her hastily
resolve on avoiding such immediate horrors at all risks,
it is probable that Mr.\ Yates would never have succeeded.
She had not eloped with any worse feelings than those
of selfish alarm.  It had appeared to her the only
thing to be done.  Maria's guilt had induced Julia's folly.

Henry Crawford, ruined by early independence and bad
domestic example, indulged in the freaks of a cold-blooded
vanity a little too long.  Once it had, by an opening
undesigned and unmerited, led him into the way of happiness.
Could he have been satisfied with the conquest of one
amiable woman's affections, could he have found sufficient
exultation in overcoming the reluctance, in working himself
into the esteem and tenderness of Fanny Price, there would
have been every probability of success and felicity for him.
His affection had already done something.  Her influence
over him had already given him some influence over her.
Would he have deserved more, there can be no doubt
that more would have been obtained, especially when
that marriage had taken place, which would have given
him the assistance of her conscience in subduing her
first inclination, and brought them very often together.
Would he have persevered, and uprightly, Fanny must have
been his reward, and a reward very voluntarily bestowed,
within a reasonable period from Edmund's marrying Mary.

Had he done as he intended, and as he knew he ought,
by going down to Everingham after his return from Portsmouth,
he might have been deciding his own happy destiny.
But he was pressed to stay for Mrs.\ Fraser's party;
his staying was made of flattering consequence, and he
was to meet Mrs.\ Rushworth there.  Curiosity and vanity
were both engaged, and the temptation of immediate pleasure
was too strong for a mind unused to make any sacrifice
to right:  he resolved to defer his Norfolk journey,
resolved that writing should answer the purpose of it,
or that its purpose was unimportant, and staid.  He saw
Mrs.\ Rushworth, was received by her with a coldness which
ought to have been repulsive, and have established apparent
indifference between them for ever; but he was mortified,
he could not bear to be thrown off by the woman whose
smiles had been so wholly at his command:  he must exert
himself to subdue so proud a display of resentment; it was
anger on Fanny's account; he must get the better of it,
and make Mrs.\ Rushworth Maria Bertram again in her treatment
of himself.

In this spirit he began the attack, and by animated
perseverance had soon re-established the sort of familiar
intercourse, of gallantry, of flirtation, which bounded
his views; but in triumphing over the discretion which,
though beginning in anger, might have saved them both,
he had put himself in the power of feelings on her side
more strong than he had supposed.  She loved him;
there was no withdrawing attentions avowedly dear to her.
He was entangled by his own vanity, with as little
excuse of love as possible, and without the smallest
inconstancy of mind towards her cousin.  To keep Fanny
and the Bertrams from a knowledge of what was passing
became his first object.  Secrecy could not have been
more desirable for Mrs.\ Rushworth's credit than he
felt it for his own.  When he returned from Richmond,
he would have been glad to see Mrs.\ Rushworth no more.
All that followed was the result of her imprudence;
and he went off with her at last, because he could
not help it, regretting Fanny even at the moment,
but regretting her infinitely more when all the bustle of
the intrigue was over, and a very few months had taught him,
by the force of contrast, to place a yet higher value
on the sweetness of her temper, the purity of her mind,
and the excellence of her principles.

That punishment, the public punishment of disgrace,
should in a just measure attend \emph{his} share of the offence is,
we know, not one of the barriers which society gives
to virtue.  In this world the penalty is less equal than
could be wished; but without presuming to look forward
to a juster appointment hereafter, we may fairly consider
a man of sense, like Henry Crawford, to be providing
for himself no small portion of vexation and regret:
vexation that must rise sometimes to self-reproach, and
regret to wretchedness, in having so requited hospitality,
so injured family peace, so forfeited his best, most estimable,
and endeared acquaintance, and so lost the woman whom
he had rationally as well as passionately loved.

After what had passed to wound and alienate the two families,
the continuance of the Bertrams and Grants in such
close neighbourhood would have been most distressing;
but the absence of the latter, for some months purposely
lengthened, ended very fortunately in the necessity,
or at least the practicability, of a permanent removal.
Dr.\ Grant, through an interest on which he had almost
ceased to form hopes, succeeded to a stall in Westminster,
which, as affording an occasion for leaving Mansfield,
an excuse for residence in London, and an increase of
income to answer the expenses of the change, was highly
acceptable to those who went and those who staid.

Mrs.\ Grant, with a temper to love and be loved, must have
gone with some regret from the scenes and people she
had been used to; but the same happiness of disposition
must in any place, and any society, secure her a great
deal to enjoy, and she had again a home to offer Mary;
and Mary had had enough of her own friends, enough of vanity,
ambition, love, and disappointment in the course of the
last half-year, to be in need of the true kindness of her
sister's heart, and the rational tranquillity of her ways.
They lived together; and when Dr.\ Grant had brought
on apoplexy and death, by three great institutionary
dinners in one week, they still lived together; for Mary,
though perfectly resolved against ever attaching herself
to a younger brother again, was long in finding among
the dashing representatives, or idle heir-apparents,
who were at the command of her beauty, and her 20,000,
any one who could satisfy the better taste she had acquired
at Mansfield, whose character and manners could authorise
a hope of the domestic happiness she had there learned
to estimate, or put Edmund Bertram sufficiently out of her head.

Edmund had greatly the advantage of her in this respect.
He had not to wait and wish with vacant affections for an
object worthy to succeed her in them.  Scarcely had he
done regretting Mary Crawford, and observing to Fanny
how impossible it was that he should ever meet with such
another woman, before it began to strike him whether
a very different kind of woman might not do just as well,
or a great deal better:  whether Fanny herself were not
growing as dear, as important to him in all her smiles
and all her ways, as Mary Crawford had ever been;
and whether it might not be a possible, an hopeful
undertaking to persuade her that her warm and sisterly
regard for him would be foundation enough for wedded love.

I purposely abstain from dates on this occasion,
that every one may be at liberty to fix their own,
aware that the cure of unconquerable passions, and the
transfer of unchanging attachments, must vary much as
to time in different people.  I only entreat everybody
to believe that exactly at the time when it was quite
natural that it should be so, and not a week earlier,
Edmund did cease to care about Miss Crawford, and became
as anxious to marry Fanny as Fanny herself could desire.

With such a regard for her, indeed, as his had long been,
a regard founded on the most endearing claims of innocence
and helplessness, and completed by every recommendation
of growing worth, what could be more natural than
the change?  Loving, guiding, protecting her, as he
had been doing ever since her being ten years old,
her mind in so great a degree formed by his care,
and her comfort depending on his kindness, an object to him
of such close and peculiar interest, dearer by all his
own importance with her than any one else at Mansfield,
what was there now to add, but that he should learn
to prefer soft light eyes to sparkling dark ones.
And being always with her, and always talking confidentially,
and his feelings exactly in that favourable state
which a recent disappointment gives, those soft light
eyes could not be very long in obtaining the pre-eminence.

Having once set out, and felt that he had done so on
this road to happiness, there was nothing on the side
of prudence to stop him or make his progress slow;
no doubts of her deserving, no fears of opposition of taste,
no need of drawing new hopes of happiness from dissimilarity
of temper.  Her mind, disposition, opinions, and habits
wanted no half-concealment, no self-deception on the present,
no reliance on future improvement.  Even in the midst
of his late infatuation, he had acknowledged Fanny's
mental superiority.  What must be his sense of it now,
therefore?  She was of course only too good for him;
but as nobody minds having what is too good for them,
he was very steadily earnest in the pursuit of the blessing,
and it was not possible that encouragement from her should
be long wanting.  Timid, anxious, doubting as she was,
it was still impossible that such tenderness as hers
should not, at times, hold out the strongest hope of success,
though it remained for a later period to tell him the whole
delightful and astonishing truth.  His happiness in knowing
himself to have been so long the beloved of such a heart,
must have been great enough to warrant any strength of
language in which he could clothe it to her or to himself;
it must have been a delightful happiness.  But there
was happiness elsewhere which no description can reach.
Let no one presume to give the feelings of a young woman
on receiving the assurance of that affection of which
she has scarcely allowed herself to entertain a hope.

Their own inclinations ascertained, there were no
difficulties behind, no drawback of poverty or parent.
It was a match which Sir Thomas's wishes had even forestalled.
Sick of ambitious and mercenary connexions, prizing more
and more the sterling good of principle and temper,
and chiefly anxious to bind by the strongest securities
all that remained to him of domestic felicity, he had
pondered with genuine satisfaction on the more than
possibility of the two young friends finding their natural
consolation in each other for all that had occurred
of disappointment to either; and the joyful consent
which met Edmund's application, the high sense of having
realised a great acquisition in the promise of Fanny
for a daughter, formed just such a contrast with his
early opinion on the subject when the poor little girl's
coming had been first agitated, as time is for ever
producing between the plans and decisions of mortals,
for their own instruction, and their neighbours' entertainment.

Fanny was indeed the daughter that he wanted.  His charitable
kindness had been rearing a prime comfort for himself.
His liberality had a rich repayment, and the general
goodness of his intentions by her deserved it.  He might
have made her childhood happier; but it had been an error
of judgment only which had given him the appearance
of harshness, and deprived him of her early love;
and now, on really knowing each other, their mutual
attachment became very strong.  After settling her at
Thornton Lacey with every kind attention to her comfort,
the object of almost every day was to see her there,
or to get her away from it.

Selfishly dear as she had long been to Lady Bertram,
she could not be parted with willingly by \emph{her}.
No happiness of son or niece could make her wish
the marriage.  But it was possible to part with her,
because Susan remained to supply her place.
Susan became the stationary niece, delighted to be so;
and equally well adapted for it by a readiness of mind,
and an inclination for usefulness, as Fanny had been
by sweetness of temper, and strong feelings of gratitude.
Susan could never be spared.  First as a comfort to Fanny,
then as an auxiliary, and last as her substitute,
she was established at Mansfield, with every appearance
of equal permanency.  Her more fearless disposition
and happier nerves made everything easy to her there.
With quickness in understanding the tempers of those she
had to deal with, and no natural timidity to restrain
any consequent wishes, she was soon welcome and useful
to all; and after Fanny's removal succeeded so naturally
to her influence over the hourly comfort of her aunt,
as gradually to become, perhaps, the most beloved of the two.
In \emph{her} usefulness, in Fanny's excellence, in William's
continued good conduct and rising fame, and in the general
well-doing and success of the other members of the family,
all assisting to advance each other, and doing credit
to his countenance and aid, Sir Thomas saw repeated,
and for ever repeated, reason to rejoice in what he had
done for them all, and acknowledge the advantages of early
hardship and discipline, and the consciousness of being born
to struggle and endure.

With so much true merit and true love, and no want of
fortune and friends, the happiness of the married cousins
must appear as secure as earthly happiness can be.
Equally formed for domestic life, and attached to
country pleasures, their home was the home of affection
and comfort; and to complete the picture of good,
the acquisition of Mansfield living, by the death of
Dr.\ Grant, occurred just after they had been married long
enough to begin to want an increase of income, and feel
their distance from the paternal abode an inconvenience.

On that event they removed to Mansfield; and the Parsonage
there, which, under each of its two former owners, Fanny had
never been able to approach but with some painful sensation
of restraint or alarm, soon grew as dear to her heart,
and as thoroughly perfect in her eyes, as everything else
within the view and patronage of Mansfield Park had long been.

\end{document}

% <THE END>
%
%
% End of the Project Gutenberg text of Mansfield Park
