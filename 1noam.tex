% Project Gutenberg Etext North America, V. 1, by Anthony Trollope
% #3 in our series by Anthony Trollope
% 
% 
% Copyright laws are changing all over the world, be sure to check
% the copyright laws for your country before posting these files!!
% 
% Please take a look at the important information in this header.
% We encourage you to keep this file on your own disk, keeping an
% electronic path open for the next readers.  Do not remove this.
% 
% 
% **Welcome To The World of Free Plain Vanilla Electronic Texts**
% 
% **Etexts Readable By Both Humans and By Computers, Since 1971**
% 
% *These Etexts Prepared By Hundreds of Volunteers and Donations*
% 
% Information on contacting Project Gutenberg to get Etexts, and
% further information is included below.  We need your donations.
% 
% 
% North America
% 
% Volume 1
% 
% by Anthony Trollope
% 
% August, 1999  [Etext #1865]
% 
% 
% Project Gutenberg Etext North America, V. 1, by Anthony Trollope
% *******This file should be named 1noam.txt or 1noam10.zip*******
% 
% Corrected EDITIONS of our etexts get a new NUMBER, 1noam11.txt.
% VERSIONS based on separate sources get new LETTER, 1noam10a.txt.
% 
% 
% This etext was prepared by Donald Lainson, charlie@idirect.com.
% 
% 
% We are now trying to release all our books one month in advance
% of the official release dates, for time for better editing.
% 
% Please note:  neither this list nor its contents are final till
% midnight of the last day of the month of any such announcement.
% The official release date of all Project Gutenberg Etexts is at
% Midnight, Central Time, of the last day of the stated month.  A
% preliminary version may often be posted for suggestion, comment
% and editing by those who wish to do so.  To be sure you have an
% up to date first edition [xxxxx10x.xxx] please check file sizes
% in the first week of the next month.  Since our ftp program has
% a bug in it that scrambles the date [tried to fix and failed] a
% look at the file size will have to do, but we will try to see a
% new copy has at least one byte more or less.
% 
% 
% Information about Project Gutenberg (one page)
% 
% We produce about two million dollars for each hour we work.  The
% fifty hours is one conservative estimate for how long it we take
% to get any etext selected, entered, proofread, edited, copyright
% searched and analyzed, the copyright letters written, etc.  This
% projected audience is one hundred million readers.  If our value
% per text is nominally estimated at one dollar then we produce $2
% million dollars per hour this year as we release thirty-two text
% files per month, or 384 more Etexts in 1997 for a total of 1000+
% If these reach just 10% of the computerized population, then the
% total should reach over 100 billion Etexts given away.
% 
% The Goal of Project Gutenberg is to Give Away One Trillion Etext
% Files by the December 31, 2001.  [10,000 x 100,000,000=Trillion]
% This is ten thousand titles each to one hundred million readers,
% which is only 10% of the present number of computer users.  2001
% should have at least twice as many computer users as that, so it
% will require us reaching less than 5% of the users in 2001.
% 
% 
% We need your donations more than ever!
% 
% 
% All donations should be made to "Project Gutenberg/CMU": and are
% tax deductible to the extent allowable by law.  (CMU = Carnegie-
% Mellon University).
% 
% For these and other matters, please mail to:
% 
% Project Gutenberg
% P. O. Box  2782
% Champaign, IL 61825
% 
% When all other email fails try our Executive Director:
% Michael S. Hart <hart@pobox.com>
% 
% We would prefer to send you this information by email
% (Internet, Bitnet, Compuserve, ATTMAIL or MCImail).
% 
% ******
% If you have an FTP program (or emulator), please
% FTP directly to the Project Gutenberg archives:
% [Mac users, do NOT point and click. . .type]
% 
% ftp uiarchive.cso.uiuc.edu
% login:  anonymous
% password:  your@login
% cd etext/etext90 through /etext96
% or cd etext/articles [get suggest gut for more information]
% dir [to see files]
% get or mget [to get files. . .set bin for zip files]
% GET INDEX?00.GUT
% for a list of books
% and
% GET NEW GUT for general information
% and
% MGET GUT* for newsletters.
% 
% **Information prepared by the Project Gutenberg legal advisor**
% (Three Pages)
% 
% 
% ***START**THE SMALL PRINT!**FOR PUBLIC DOMAIN ETEXTS**START***
% Why is this "Small Print!" statement here?  You know: lawyers.
% They tell us you might sue us if there is something wrong with
% your copy of this etext, even if you got it for free from
% someone other than us, and even if what's wrong is not our
% fault.  So, among other things, this "Small Print!" statement
% disclaims most of our liability to you.  It also tells you how
% you can distribute copies of this etext if you want to.
% 
% *BEFORE!* YOU USE OR READ THIS ETEXT
% By using or reading any part of this PROJECT GUTENBERG-tm
% etext, you indicate that you understand, agree to and accept
% this "Small Print!" statement.  If you do not, you can receive
% a refund of the money (if any) you paid for this etext by
% sending a request within 30 days of receiving it to the person
% you got it from.  If you received this etext on a physical
% medium (such as a disk), you must return it with your request.
% 
% ABOUT PROJECT GUTENBERG-TM ETEXTS
% This PROJECT GUTENBERG-tm etext, like most PROJECT GUTENBERG-
% tm etexts, is a "public domain" work distributed by Professor
% Michael S. Hart through the Project Gutenberg Association at
% Carnegie-Mellon University (the "Project").  Among other
% things, this means that no one owns a United States copyright
% on or for this work, so the Project (and you!) can copy and
% distribute it in the United States without permission and
% without paying copyright royalties.  Special rules, set forth
% below, apply if you wish to copy and distribute this etext
% under the Project's "PROJECT GUTENBERG" trademark.
% 
% To create these etexts, the Project expends considerable
% efforts to identify, transcribe and proofread public domain
% works.  Despite these efforts, the Project's etexts and any
% medium they may be on may contain "Defects".  Among other
% things, Defects may take the form of incomplete, inaccurate or
% corrupt data, transcription errors, a copyright or other
% intellectual property infringement, a defective or damaged
% disk or other etext medium, a computer virus, or computer
% codes that damage or cannot be read by your equipment.
% 
% LIMITED WARRANTY; DISCLAIMER OF DAMAGES
% But for the "Right of Replacement or Refund" described below,
% [1] the Project (and any other party you may receive this
% etext from as a PROJECT GUTENBERG-tm etext) disclaims all
% liability to you for damages, costs and expenses, including
% legal fees, and [2] YOU HAVE NO REMEDIES FOR NEGLIGENCE OR
% UNDER STRICT LIABILITY, OR FOR BREACH OF WARRANTY OR CONTRACT,
% INCLUDING BUT NOT LIMITED TO INDIRECT, CONSEQUENTIAL, PUNITIVE
% OR INCIDENTAL DAMAGES, EVEN IF YOU GIVE NOTICE OF THE
% POSSIBILITY OF SUCH DAMAGES.
% 
% If you discover a Defect in this etext within 90 days of
% receiving it, you can receive a refund of the money (if any)
% you paid for it by sending an explanatory note within that
% time to the person you received it from.  If you received it
% on a physical medium, you must return it with your note, and
% such person may choose to alternatively give you a replacement
% copy.  If you received it electronically, such person may
% choose to alternatively give you a second opportunity to
% receive it electronically.
% 
% THIS ETEXT IS OTHERWISE PROVIDED TO YOU "AS-IS".  NO OTHER
% WARRANTIES OF ANY KIND, EXPRESS OR IMPLIED, ARE MADE TO YOU AS
% TO THE ETEXT OR ANY MEDIUM IT MAY BE ON, INCLUDING BUT NOT
% LIMITED TO WARRANTIES OF MERCHANTABILITY OR FITNESS FOR A
% PARTICULAR PURPOSE.
% 
% Some states do not allow disclaimers of implied warranties or
% the exclusion or limitation of consequential damages, so the
% above disclaimers and exclusions may not apply to you, and you
% may have other legal rights.
% 
% INDEMNITY
% You will indemnify and hold the Project, its directors,
% officers, members and agents harmless from all liability, cost
% and expense, including legal fees, that arise directly or
% indirectly from any of the following that you do or cause:
% [1] distribution of this etext, [2] alteration, modification,
% or addition to the etext, or [3] any Defect.
% 
% DISTRIBUTION UNDER "PROJECT GUTENBERG-tm"
% You may distribute copies of this etext electronically, or by
% disk, book or any other medium if you either delete this
% "Small Print!" and all other references to Project Gutenberg,
% or:
% 
% [1]  Only give exact copies of it.  Among other things, this
%      requires that you do not remove, alter or modify the
%      etext or this "small print!" statement.  You may however,
%      if you wish, distribute this etext in machine readable
%      binary, compressed, mark-up, or proprietary form,
%      including any form resulting from conversion by word pro-
%      cessing or hypertext software, but only so long as
%      *EITHER*:
% 
%      [*]  The etext, when displayed, is clearly readable, and
%           does *not* contain characters other than those
%           intended by the author of the work, although tilde
%           (~), asterisk (*) and underline (_) characters may
%           be used to convey punctuation intended by the
%           author, and additional characters may be used to
%           indicate hypertext links; OR
% 
%      [*]  The etext may be readily converted by the reader at
%           no expense into plain ASCII, EBCDIC or equivalent
%           form by the program that displays the etext (as is
%           the case, for instance, with most word processors);
%           OR
% 
%      [*]  You provide, or agree to also provide on request at
%           no additional cost, fee or expense, a copy of the
%           etext in its original plain ASCII form (or in EBCDIC
%           or other equivalent proprietary form).
% 
% [2]  Honor the etext refund and replacement provisions of this
%      "Small Print!" statement.
% 
% [3]  Pay a trademark license fee to the Project of 20% of the
%      net profits you derive calculated using the method you
%      already use to calculate your applicable taxes.  If you
%      don't derive profits, no royalty is due.  Royalties are
%      payable to "Project Gutenberg Association/Carnegie-Mellon
%      University" within the 60 days following each
%      date you prepare (or were legally required to prepare)
%      your annual (or equivalent periodic) tax return.
% 
% WHAT IF YOU *WANT* TO SEND MONEY EVEN IF YOU DON'T HAVE TO?
% The Project gratefully accepts contributions in money, time,
% scanning machines, OCR software, public domain etexts, royalty
% free copyright licenses, and every other sort of contribution
% you can think of.  Money should be paid to "Project Gutenberg
% Association / Carnegie-Mellon University".
% 
% *END*THE SMALL PRINT! FOR PUBLIC DOMAIN ETEXTS*Ver.04.29.93*END*
% 
% 
% 
% 
% 
% This etext was prepared by Donald Lainson, charlie@idirect.com.

%
% converted to LaTeX by Peter Monta <pmonta@pmonta.com>
% July 2002
%

\input gutenberg-toc.tex

\begin{document}





\gtitle{North America}
% by
\gauthor{Anthony Trollope}


% VOLUME I.
\part{Volume I}




% CONTENTS OF VOL. I.
% 
% 
% CHAPTER I.
% 
% INTRODUCTION
% 
% CHAPTER II.
% 
% Newport---Rhode Island
% 
% CHAPTER III.
% 
% Maine, New Hampshire, and Vermont
% 
% CHAPTER IV.
% 
% Lower Canada
% 
% CHAPTER V.
% 
% Upper Canada
% 
% CHAPTER VI.
% 
% The Connection of the Canadas with Great Britain
% 
% CHAPTER VII.
% 
% Niagara
% 
% CHAPTER VIII.
% 
% North and West
% 
% CHAPTER IX.
% 
% From Niagara to the Mississippi
% 
% CHAPTER X.
% 
% The Upper Mississippi
% 
% CHAPTER XI.
% 
% Ceres Americana
% 
% CHAPTER XII.
% 
% Buffalo to New York
% 
% CHAPTER XIII.
% 
% An Apology for the War
% 
% CHAPTER XIV.
% 
% New York
% 
% CHAPTER XV.
% 
% The Constitution of the State of New York
% 
% CHAPTER XVI.
% 
% Boston
% 
% CHAPTER XVII.
% 
% Cambridge and Lowell
% 
% CHAPTER XVIII.
% 
% The Rights of Women
% 
% CHAPTER XIX.
% 
% Education
% 
% CHAPTER XX.
% 
% From Boston to Washington
% 
% 
% 
% 
% NORTH AMERICA.



\chapter{Introduction}


It has been the ambition of my literary life to write a book about
the United States, and I had made up my mind to visit the country
with this object before the intestine troubles of the United States
government had commenced.  I have not allowed the division among
the States and the breaking out of civil war to interfere with my
intention; but I should not purposely have chosen this period
either for my book or for my visit.  I say so much, in order that
it may not be supposed that it is my special purpose to write an
account of the struggle as far as it has yet been carried.  My wish
is to describe, as well as I can, the present social and political
state of the country.  This I should have attempted, with more
personal satisfaction in the work, had there been no disruption
between the North and South; but I have not allowed that disruption
to deter me from an object which, if it were delayed, might
probably never be carried out.  I am therefore forced to take the
subject in its present condition, and being so forced I must write
of the war, of the causes which have led to it, and of its probable
termination.  But I wish it to be understood that it was not my
selected task to do so, and is not now my primary object.

Thirty years ago my mother wrote a book about the Americans, to
which I believe I may allude as a well-known and successful work
without being guilty of any undue family conceit.  That was
essentially a woman's book.  She saw with a woman's keen eye, and
described with a woman's light but graphic pen, the social defects
and absurdities which our near relatives had adopted into their
domestic life.  All that she told was worth the telling, and the
telling, if done successfully, was sure to produce a good result.
I am satisfied that it did so.  But she did not regard it as a part
of her work to dilate on the nature and operation of those
political arrangements which had produced the social absurdities
which she saw, or to explain that though such absurdities were the
natural result of those arrangements in their newness, the defects
would certainly pass away, while the political arrangements, if
good, would remain.  Such a work is fitter for a man than for a
woman, I am very far from thinking that it is a task which I can
perform with satisfaction either to myself or to others.  It is a
work which some man will do who has earned a right by education,
study, and success to rank himself among the political sages of his
age.  But I may perhaps be able to add something to the familiarity
of Englishmen with Americans.  The writings which have been most
popular in England on the subject of the United States have
hitherto dealt chiefly with social details; and though in most
cases true and useful, have created laughter on one side of the
Atlantic, and soreness on the other.  if I could do anything to
mitigate the soreness, if I could in any small degree add to the
good feeling which should exist between two nations which ought to
love each other so well, and which do hang upon each other so
constantly, I should think that I had cause to be proud of my work.

But it is very hard to write about any country a book that does not
represent the country described in a more or less ridiculous point
of view.  It is hard at least to do so in such a book as I must
write.  A de Tocqueville may do it.  It may be done by any
philosophico-political or politico-statistical, or statistico-
scientific writer; but it can hardly be done by a man who professes
to use a light pen, and to manufacture his article for the use of
general readers.  Such a writer may tell all that he sees of the
beautiful; but he must also tell, if not all that he sees of the
ludicrous, at any rate the most piquant part of it.  How to do this
without being offensive is the problem which a man with such a task
before him has to solve.  His first duty is owed to his readers,
and consists mainly in this: that he shall tell the truth, and
shall so tell that truth that what he has written may be readable.
But a second duty is due to those of whom he writes; and he does
not perform that duty well if he gives offense to those as to whom,
on the summing up of the whole evidence for and against them in his
own mind, he intends to give a favorable verdict.  There are of
course those against whom a writer does not intend to give a
favorable verdict; people and places whom he desires to describe,
on the peril of his own judgment, as bad, ill educated, ugly, and
odious.  In such cases his course is straightforward enough.  His
judgment may be in great peril, but his volume or chapter will be
easily written.  Ridicule and censure run glibly from the pen, and
form themselves into sharp paragraphs which are pleasant to the
reader.  Whereas eulogy is commonly dull, and too frequently sounds
as though it were false.  There is much difficulty in expressing a
verdict which is intended to be favorable; but which, though
favorable, shall not be falsely eulogistic; and though true, not
offensive.

Who has ever traveled in foreign countries without meeting
excellent stories against the citizens of such countries?  And how
few can travel without hearing such stories against themselves!  It
is impossible for me to avoid telling of a very excellent gentleman
whom I met before I had been in the United States a week, and who
asked me whether lords in England ever spoke to men who were not
lords.  Nor can I omit the opening address of another gentleman to
my wife.  ``You like our institutions, ma'am?''  ``Yes, indeed,'' said
my wife, not with all that eagerness of assent which the occasion
perhaps required.  ``Ah,'' said he, ``I never yet met the down-trodden
subject of a despot who did not hug his chains.''  The first
gentleman was certainly somewhat ignorant of our customs, and the
second was rather abrupt in his condemnation of the political
principles of a person whom he only first saw at that moment.  It
comes to me in the way of my trade to repeat such incidents; but I
can tell stories which are quite as good against Englishmen.  As,
for instance, when I was tapped on the back in one of the galleries
of Florence by a countryman of mine, and asked to show him where
stood the medical Venus.  Nor is anything that one can say of the
inconveniences attendant upon travel in the United States to be
beaten by what foreigners might truly say of us.  I shall never
forget the look of a Frenchman whom I found on a wet afternoon in
the best inn of a provincial town in the west of England.  He was
seated on a horsehair-covered chair in the middle of a small,
dingy, ill-furnished private sitting-room.  No eloquence of mine
could make intelligible to a Frenchman or an American the utter
desolation of such an apartment.  The world as then seen by that
Frenchman offered him solace of no description.  The air without
was heavy, dull, and thick.  The street beyond the window was dark
and narrow.  The room contained mahogany chairs covered with horse-
hair, a mahogany table, rickety in its legs, and a mahogany
sideboard ornamented with inverted glasses and old cruet-stands.
The Frenchman had come to the house for shelter and food, and had
been asked whether he was commercial.  Whereupon he shook his head.
``Did he want a sitting-room?''  Yes, he did.  ``He was a leetle tired
and vanted to seet.''  Whereupon he was presumed to have ordered a
private room, and was shown up to the Eden I have described.  I
found him there at death's door.  Nothing that I can say with
reference to the social habits of the Americans can tell more
against them than the story of that Frenchman's fate tells against
those of our country.

From which remarks I would wish to be understood as deprecating
offense from my American friends, if in the course of my book
should be found aught which may seem to argue against the
excellence of their institutions and the grace of their social
life.  Of this at any rate I can assure them, in sober earnestness,
that I admire what they have done in the world and for the world
with a true and hearty admiration; and that whether or no all their
institutions be at present excellent, and their social life all
graceful, my wishes are that they should be so, and my convictions
are that that improvement will come for which there may perhaps
even yet be some little room.

And now touching this war which had broken out between the North
and South before I left England.  I would wish to explain what my
feelings were; or rather what I believe the general feelings of
England to have been before I found myself among the people by whom
it was being waged.  It is very difficult for the people of any one
nation to realize the political relations of another, and to chew
the cud and digest the bearings of those external politics.  But it
is unjust in the one to decide upon the political aspirations and
doings of that other without such understanding.  Constantly as the
name of France is in our mouths, comparatively few Englishmen
understand the way in which France is governed; that is, how far
absolute despotism prevails, and how far the power of the one ruler
is tempered, or, as it may be, hampered by the voices and influence
of others.  And as regards England, how seldom is it that in common
society a foreigner is met who comprehends the nature of her
political arrangements!  To a Frenchman---I do not of course include
great men who have made the subject a study,---but to the ordinary
intelligent Frenchman the thing is altogether incomprehensible.
Language, it may be said, has much to do with that.  But an
American speaks English; and how often is an American met who has
combined in his mind the idea of a monarch, so called, with that of
a republic, properly so named---a combination of ideas which I take
to be necessary to the understanding of English politics!  The
gentleman who scorned my wife for hugging her chains had certainly
not done so, and yet he conceived that he had studied the subject.
The matter is one most difficult of comprehension.  How many
Englishmen have failed to understand accurately their own
constitution, or the true bearing of their own politics!  But when
this knowledge has been attained, it has generally been filtered
into the mind slowly, and has come from the unconscious study of
many years.  An Englishman handles a newspaper for a quarter of an
hour daily, and daily exchanges some few words in politics with
those around him, till drop by drop the pleasant springs of his
liberty creep into his mind and water his heart; and thus, earlier
or later in life, according to the nature of his intelligence, he
understands why it is that he is at all points a free man.  But if
this be so of our own politics; if it be so rare a thing to find a
foreigner who understands them in all their niceties, why is it
that we are so confident in our remarks on all the niceties of
those of other nations?

I hope that I may not be misunderstood as saying that we should not
discuss foreign politics in our press, our parliament, our public
meetings, or our private houses.  No man could be mad enough to
preach such a doctrine.  As regards our parliament, that is
probably the best British school of foreign politics, seeing that
the subject is not there often taken up by men who are absolutely
ignorant, and that mistakes when made are subject to a correction
which is both rough and ready.  The press, though very liable to
error, labors hard at its vocation in teaching foreign politics,
and spares no expense in letting in daylight.  If the light let in
be sometimes moonshine, excuse may easily be made.  Where so much
is attempted, there must necessarily be some failure.  But even the
moonshine does good if it be not offensive moonshine.  What I would
deprecate is, that aptness at reproach which we assume; the
readiness with scorn, the quiet words of insult, the instant
judgment and condemnation with which we are so inclined to visit,
not the great outward acts, but the smaller inward politics of our
neighbors.

And do others spare us? will be the instant reply of all who may
read this.  In my counter reply I make bold to place myself and my
country on very high ground, and to say that we, the older and
therefore more experienced people as regards the United States, and
the better governed as regards France, and the stronger as regards
all the world beyond, should not throw mud again even though mud be
thrown at us.  I yield the path to a small chimney-sweeper as
readily as to a lady; and forbear from an interchange of courtesies
with a Billingsgate heroine, even though at heart I may have a
proud consciousness that I should not altogether go to the wall in
such an encounter.

I left England in August last---August, 1861.  At that time, and for
some months previous, I think that the general English feeling on
the American question was as follows: ``This wide-spread nationality
of the United States, with its enormous territorial possessions and
increasing population, has fallen asunder, torn to pieces by the
weight of its own discordant parts---as a congregation when its size
has become unwieldy will separate, and reform itself into two
wholesome wholes.  It is well that this should be so, for the
people are not homogeneous, as a people should be who are called to
live together as one nation.  They have attempted to combine free-
soil sentiments with the practice of slavery, and to make these two
antagonists live together in peace and unity under the same roof;
but, as we have long expected, they have failed.  Now has come the
period for separation; and if the people would only see this, and
act in accordance with the circumstances which Providence and the
inevitable hand of the world's Ruler has prepared for them, all
would be well.  But they will not do this.  They will go to war
with each other.  The South will make her demands for secession
with an arrogance and instant pressure which exasperates the North;
and the North, forgetting that an equable temper in such matters is
the most powerful of all weapons, will not recognize the strength
of its own position.  It allows itself to be exasperated, and goes
to war for that which if regained would only be injurious to it.
Thus millions on millions sterling will be spent.  A heavy debt
will be incurred; and the North, which divided from the South might
take its place among the greatest of nations, will throw itself
back for half a century, and perhaps injure the splendor of its
ultimate prospects.  If only they would be wise, throw down their
arms, and agree to part!  But they will not.''

This was I think the general opinion when I left England.  It would
not, however, be necessary to go back many months to reach the time
when Englishmen were saying how impossible it was that so great a
national power should ignore its own greatness and destroy its own
power by an internecine separation.  But in August last all that
had gone by, and we in England had realized the probability of
actual secession.

To these feelings on the subject maybe added another, which was
natural enough though perhaps not noble.  ``These western cocks have
crowed loudly,'' we said; ``too loudly for the comfort of those who
live after all at no such great distance from them.  It is well
that their combs should be clipped.  Cocks who crow so very loudly
are a nuisance.  It might have gone so far that the clipping would
become a work necessarily to be done from without.  But it is ten
times better for all parties that it should be done from within;
and as the cocks are now clipping their own combs, in God's name
let them do it, and the whole world will be the quieter.''  That, I
say, was not a very noble idea; but it was natural enough, and
certainly has done somewhat in mitigating that grief which the
horrors of civil war and the want of cotton have caused to us in
England.

Such certainly had been my belief as to the country.  I speak here
of my opinion as to the ultimate success of secession and the folly
of the war, repudiating any concurrence of my own in the ignoble
but natural sentiment alluded to in the last paragraph.  I
certainly did think that the Northern States, if wise, would have
let the Southern States go.  I had blamed Buchanan as a traitor for
allowing the germ of secession to make any growth; and as I thought
him a traitor then, so do I think him a traitor now.  But I had
also blamed Lincoln, or rather the government of which Mr.\ Lincoln
in this matter is no more than the exponent, for his efforts to
avoid that which is inevitable.  In this I think that I---or as I
believe I may say we, we Englishmen---were wrong.  I do not see how
the North, treated as it was and had been, could have submitted to
secession without resistance.  We all remember what Shakspeare says
of the great armies which were led out to fight for a piece of
ground not large enough to cover the bodies of those who would be
slain in the battle; but I do not remember that Shakspeare says
that the battle was on this account necessarily unreasonable.  It
is the old point of honor which, till it had been made absurd by
certain changes of circumstances, was always grand and usually
beneficent.  These changes of circumstances have altered the manner
in which appeal may be made, but have not altered the point of
honor.  Had the Southern States sought to obtain secession by
constitutional means, they might or might not have been successful;
but if successful, there would have been no war.  I do not mean to
brand all the Southern States with treason, nor do I intend to say
that, having secession at heart, they could have obtained it by
constitutional means.  But I do intend to say that, acting as they
did, demanding secession not constitutionally, but in opposition to
the constitution, taking upon themselves the right of breaking up a
nationality of which they formed only a part, and doing that
without consent of the other part, opposition from the North and
war was an inevitable consequence.

It is, I think, only necessary to look back to the Revolution by
which the United States separated themselves from England to see
this.  There is hardly to be met, here and there, an Englishman who
now regrets the loss of the revolted American colonies; who now
thinks that civilization was retarded and the world injured by that
revolt; who now conceives that England should have expended more
treasure and more lives in the hope of retaining those colonies.
It is agreed that the revolt was a good thing; that those who were
then rebels became patriots by success, and that they deserved well
of all coming ages of mankind.  But not the less absolutely
necessary was it that England should endeavor to hold her own.  She
was as the mother bird when the young bird will fly alone.  She
suffered those pangs which Nature calls upon mothers to endure.

As was the necessity of British opposition to American
independence, so was the necessity of Northern opposition to
Southern secession.  I do not say that in other respects the two
cases were parallel.  The States separated from us because they
would not endure taxation without representation---in other words,
because they were old enough and big enough to go alone.  The South
is seceding from the North because the two are not homogeneous.
They have different instincts, different appetites, different
morals, and a different culture.  It is well for one man to say
that slavery has caused the separation, and for another to say that
slavery has not caused it.  Each in so saying speaks the truth.
Slavery has caused it, seeing that slavery is the great point on
which the two have agreed to differ.  But slavery has not caused
it, seeing that other points of difference are to be found in every
circumstance and feature of the two people.  The North and the
South must ever be dissimilar.  In the North labor will always be
honorable, and because honorable, successful.  In the South labor
has ever been servile---at least in some sense---and therefore
dishonorable; and because dishonorable, has not, to itself, been
successful.  In the South, I say, labor ever has been dishonorable;
and I am driven to confess that I have not hitherto seen a sign of
any change in the Creator's fiat on this matter.  That labor will
be honorable all the world over as years advance and the millennium
draws nigh, I for one never doubt.

So much for English opinion about America in August last.  And now
I will venture to say a word or two as to American feeling
respecting this English opinion at that period.  It will of course
be remembered by all my readers that, at the beginning of the war,
Lord Russell, who was then in the lower house, declared, as Foreign
Secretary of State, that England would regard the North and South
as belligerents, and would remain neutral as to both of them.  This
declaration gave violent offense to the North, and has been taken
as indicating British sympathy with the cause of the seceders.  I
am not going to explain---indeed, it would be necessary that I
should first understand---the laws of nations with regard to
blockaded ports, privateering, ships and men and goods contraband
of war, and all those semi-nautical, semi-military rules and axioms
which it is necessary that all attorneys-general and such like
should, at the present moment, have at their fingers' end.  But it
must be evident to the most ignorant in those matters, among which
large crowd I certainly include myself, that it was essentially
necessary that Lord John Russell should at that time declare openly
what England intended to do.  It was essential that our seamen
should know where they would be protected and where not, and that
the course to be taken by England should be defined.  Reticence in
the matter was not within the power of the British government.  It
behooved the Foreign Secretary of State to declare openly that
England intended to side either with one party or with the other,
or else to remain neutral between them.

I had heard this matter discussed by Americans before I left
England, and I have of course heard it discussed very frequently in
America.  There can be no doubt that the front of the offense given
by England to the Northern States was this declaration of Lord John
Russell's.  But it has been always made evident to me that the sin
did not consist in the fact of England's neutrality---in the fact of
her regarding the two parties as belligerents---but in the open
declaration made to the world by a Secretary of State that she did
intend so to regard them.  If another proof were wanting, this
would afford another proof of the immense weight attached in
America to all the proceedings and to all the feelings of England
on this matter.  The very anger of the North is a compliment paid
by the North to England.  But not the less is that anger
unreasonable.  To those in America who understand our constitution,
it must be evident that our government cannot take official
measures without a public avowal of such measures.  France can do
so.  Russia can do so.  The government of the United States can do
so, and could do so even before this rupture.  But the government
of England cannot do so.  All men connected with the government in
England have felt themselves from time to time more or less
hampered by the necessity of publicity.  Our statesmen have been
forced to fight their battles with the plan of their tactics open
before their adversaries.  But we in England are inclined to
believe that the general result is good, and that battles so fought
and so won will be fought with the honestest blows and won with the
surest results.  Reticence in this matter was not possible; and
Lord John Russell, in making the open avowal which gave such
offense to the Northern States, only did that which, as a servant
of England, England required him to do.

``What would you in England have thought,'' a gentleman of much
weight in Boston said to me, ``if, when you were in trouble in
India, we had openly declared that we regarded your opponents there
are as belligerents on equal terms with yourselves?''  I was forced
to say that, as far as I could see, there was no analogy between
the two cases.  In India an army had mutinied, and that an army
composed of a subdued, if not a servile race.  The analogy would
have been fairer had it referred to any sympathy shown by us to
insurgent negroes.  But, nevertheless, had the army which mutinied
in India been in possession of ports and sea-board; had they held
in their hands vast commercial cities and great agricultural
districts; had they owned ships and been masters of a wide-spread
trade, America could have done nothing better toward us than have
remained neutral in such a conflict and have regarded the parties
as belligerents.  The only question is whether she would have done
so well by us.  ``But,'' said my friend, in answer to all this, ``we
should not have proclaimed to the world that we regarded you and
them as standing on an equal footing.''  There again appeared the
true gist of the offense.  A word from England such as that spoken
by Lord John Russell was of such weight to the South that the North
could not endure to have it spoken.  I did not say to that
gentleman, but here I may say that, had such circumstances arisen
as those conjectured, and had America spoken such a word, England
would not have felt herself called upon to resent it.

But the fairer analogy lies between Ireland and the Southern
States.  The monster meetings and O'Connell's triumphs are not so
long gone by but that many of us can remember the first demand for
secession made by Ireland, and the line which was then taken by
American sympathies.  It is not too much to say that America then
believed that Ireland would secure secession, and that the great
trust of the Irish repealers was in the moral aid which she did and
would receive from America.  ``But our government proclaimed no
sympathy with Ireland,'' said my friend.  No.  The American
government is not called on to make such proclamations, nor had
Ireland ever taken upon herself the nature and labors of a
belligerent.

That this anger on the part of the North is unreasonable, I cannot
doubt.  That it is unfortunate, grievous, and very bitter, I am
quite sure.  But I do not think that it is in any degree
surprising.  I am inclined to think that, did I belong to Boston as
I do belong to London, I should share in the feeling, and rave as
loudly as all men there have raved against the coldness of England.
When men have on hand such a job of work as the North has now
undertaken, they are always guided by their feelings rather than
their reason.  What two men ever had a quarrel in which each did
not think that all the world, if just, would espouse his own side
of the dispute?  The North feels that it has been more than loyal
to the South, and that the South has taken advantage of that over-
loyalty to betray the North.  ``We have worked for them, and fought
for them, and paid for them,'' says the North.  ``By our labor we
have raised their indolence to a par with our energy.  While we
have worked like men, we have allowed them to talk and bluster.  We
have warmed them in our bosom, and now they turn against us and
sting us.  The world sees that this is so.  England, above all,
must see it, and, seeing it, should speak out her true opinion.''
The North is hot with such thoughts as these; and one cannot wonder
that she should be angry with her friend when her friend, with an
expression of certain easy good wishes, bids her fight out her own
battles.  The North has been unreasonable with England; but I
believe that every reader of this page would have been as
unreasonable had that reader been born in Massachusetts.

Mr.\ and Mrs.\ Jones are the dearly-beloved friends of my family.  My
wife and I have lived with Mrs.\ Jones on terms of intimacy which
have been quite endearing.  Jones has had the run of my house with
perfect freedom; and in Mrs.\ Jones's drawing-room I have always had
my own arm-chair, and have been regaled with large breakfast-cups
of tea, quite as though I were at home.  But of a sudden Jones and
his wife have fallen out, and there is for awhile in Jones Hall a
cat-and-dog life that may end---in one hardly dare to surmise what
calamity.  Mrs.\ Jones begs that I will interfere with her husband,
and Jones entreats the good offices of my wife in moderating the
hot temper of his own.  But we know better than that.  If we
interfere, the chances are that my dear friends will make it up and
turn upon us.  I grieve beyond measure in a general way at the
temporary break up of the Jones-Hall happiness.  I express general
wishes that it may be temporary.  But as for saying which is right
or which is wrong---as to expressing special sympathy on either side
in such a quarrel---it is out of the question.  ``My dear Jones, you
must excuse me.  Any news in the city to-day?  Sugars have fallen;
how are teas?''  Of course Jones thinks that I'm a brute; but what
can I do?

I have been somewhat surprised to find the trouble that has been
taken by American orators, statesmen, and logicians to prove that
this secession on the part of the South has been revolutionary---%
that is to say, that it has been undertaken and carried on not in
compliance with the Constitution of the United States, but in
defiance of it.  This has been done over and over again by some of
the greatest men of the North, and has been done most successfully.
But what then?  Of course the movement has been revolutionary and
anti-constitutional.  Nobody, no single Southerner, can really
believe that the Constitution of the United States as framed in
1787, or altered since, intended to give to the separate States the
power of seceding as they pleased.  It is surely useless going
through long arguments to prove this, seeing that it is absolutely
proved by the absence of any clause giving such license to the
separate States.  Such license would have been destructive to the
very idea of a great nationality.  Where would New England have
been, as a part of the United States, if New York, which stretches
from the Atlantic to the borders of Canada, had been endowed with
the power of cutting off the six Northern States from the rest of
the Union?  No one will for a moment doubt that the movement was
revolutionary, and yet infinite pains are taken to prove a fact
that is patent to every one.

It is revolutionary; but what then?  Have the Northern States of
the American Union taken upon themselves, in 1861, to proclaim
their opinion that revolution is a sin?  Are they going back to the
divine right of any sovereignty?  Are they going to tell the world
that a nation or a people is bound to remain in any political
status because that status is the recognized form of government
under which such a people have lived?  Is this to be the doctrine
of United States citizens---of all people?  And is this the doctrine
preached now, of all times, when the King of Naples and the Italian
dukes have just been dismissed from their thrones with such
enchanting nonchalance because their people have not chosen to keep
them?  Of course the movement is revolutionary; and why not?  It is
agreed now among all men and all nations that any people may change
its form of government to any other, if it wills to do so---and if
it can do so.

There are two other points on which these Northern statesmen and
logicians also insist, and these two other points are at any rate
better worth an argument than that which touches the question of
revolution.  It being settled that secession on the part of the
Southerners is revolution, it is argued, firstly, that no occasion
for revolution had been given by the North to the South; and,
secondly, that the South has been dishonest in its revolutionary
tactics.  Men certainly should not raise a revolution for nothing;
and it may certainly be declared that whatever men do they should
do honestly.

But in that matter of the cause and ground for revolution, it is so
very easy for either party to put in a plea that shall be
satisfactory to itself!  Mr.\ and Mrs.\ Jones each had a separate
story.  Mr.\ Jones was sure that the right lay with him; but Mrs.\ %
Jones was no less sure.  No doubt the North had done much for the
South; had earned money for it; had fed it; and had, moreover, in a
great measure fostered all its bad habits.  It had not only been
generous to the South, but over-indulgent.  But also it had
continually irritated the South by meddling with that which the
Southerners believed to be a question absolutely private to
themselves.  The matter was illustrated to me by a New Hampshire
man who was conversant with black bears.  At the hotels in the New
Hampshire mountains it is customary to find black bears chained to
poles.  These bears are caught among the hills, and are thus
imprisoned for the amusement of the hotel guests.  ``Them
Southerners,'' said my friend, ``are jist as one as that 'ere bear.
We feeds him and gives him a house, and his belly is ollers full.
But then, jist becase he's a black bear, we're ollers a poking him
with sticks, and a' course the beast is a kinder riled.  He wants
to be back to the mountains.  He wouldn't have his belly filled,
but he'd have his own way.  It's jist so with them Southerners.''

It is of no use proving to any man or to any nation that they have
got all they should want, if they have not got all that they do
want.  If a servant desires to go, it is of no avail to show him
that he has all he can desire in his present place.  The
Northerners say that they have given no offense to the Southerners,
and that therefore the South is wrong to raise a revolution.  The
very fact that the North is the North, is an offence to the South.
As long as Mr.\ and Mrs.\ Jones were one in heart and one in feeling,
having the same hopes and the same joys, it was well that they
should remain together.  But when it is proved that they cannot so
live without tearing out each other's eyes, Sir Cresswell
Cresswell, the revolutionary institution of domestic life,
interferes and separates them.  This is the age of such
separations.  I do not wonder that the North should use its logic
to show that it has received cause of offense but given none; but I
do think that such logic is thrown away.  The matter is not one for
argument.  The South has thought that it can do better without the
North than with it; and if it has the power to separate itself, it
must be conceded that it has the right.

And then as to that question of honesty.  Whatever men do they
certainly should do honestly.  Speaking broadly, one may say that
the rule applies to nations as strongly as to individuals, and
should be observed in politics as accurately as in other matters.
We must, however, confess that men who are scrupulous in their
private dealings do too constantly drop those scruples when they
handle public affairs, and especially when they handle them at
stirring moments of great national changes.  The name of Napoleon
III. stands fair now before Europe, and yet he filched the French
empire with a falsehood.  The union of England and Ireland is a
successful fact, but nevertheless it can hardly be said that it was
honestly achieved.  I heartily believe that the whole of Texas is
improved in every sense by having been taken from Mexico and added
to the Southern States, but I much doubt whether that annexation
was accomplished with absolute honesty.  We all reverence the name
of Cavour, but Cavour did not consent to abandon Nice to France
with clean hands.  When men have political ends to gain they regard
their opponents as adversaries, and then that old rule of war is
brought to bear, deceit or valor---either may be used against a foe.
Would it were not so!  The rascally rule---rascally in reference to
all political contests---is becoming less universal than it was.
But it still exists with sufficient force to be urged as an excuse;
and while it does exist it seems almost needless to show that a
certain amount of fraud has been used by a certain party in a
revolution.  If the South be ultimately successful, the fraud of
which it may have been guilty will be condoned by the world.

The Southern or Democratic party of the United States had, as all
men know, been in power for many years.  Either Southern Presidents
had been elected, or Northern Presidents with Southern politics.
The South for many years had had the disposition of military
matters, and the power of distributing military appliances of all
descriptions.  It is now alleged by the North that a conspiracy had
long been hatching in the South with the view of giving to the
Southern States the power of secession whenever they might think
fit to secede; and it is further alleged that President after
President, for years back, has unduly sent the military treasure of
the nation away from the North down to the South, in order that the
South might be prepared when the day should come.  That a President
with Southern instincts should unduly favor the South, that he
should strengthen the South, and feel that arms and ammunition were
stored there with better effect than they could be stored in the
North, is very probable.  We all understand what is the bias of a
man's mind, and how strong that bias may become when the man is not
especially scrupulous.  But I do not believe that any President
previous to Buchanan sent military materials to the South with the
self-acknowledged purpose of using them against the Union.  That
Buchanan did so, or knowingly allowed this to be done, I do
believe, and I think that Buchanan was a traitor to the country
whose servant he was and whose pay he received.

And now, having said so much in the way of introduction, I will
begin my journey.



\chapter{Newport---Rhode Island}


We---the we consisting of my wife and myself---left Liverpool for
Boston on the 24th August, 1861, in the Arabia, one of Cunard's
North American mail packets.  We had determined that my wife should
return alone at the beginning of winter, when I intended to go to a
part of the country in which, under the existing circumstances of
the war, a lady might not feel herself altogether comfortable.  I
proposed staying in America over the winter, and returning in the
spring; and this programme I have carried out with sufficient
exactness.

The Arabia touched at Halifax; and as the touch extended from 11 A.M.
to 6 P.M. we had an opportunity of seeing a good deal of that
colony; not quite sufficient to justify me at this critical age in
writing a chapter of travels in Nova Scotia, but enough perhaps to
warrant a paragraph.  It chanced that a cousin of mine was then in
command of the troops there, so that we saw the fort with all the
honors.  A dinner on shore was, I think, a greater treat to us even
than this.  We also inspected sundry specimens of the gold which is
now being found for the first time in Nova Scotia, as to the glory
and probable profits of which the Nova Scotians seemed to be fully
alive.  But still, I think the dinner on shore took rank with us as
the most memorable and meritorious of all that we did and saw at
Halifax.  At seven o'clock on the morning but one after that we
were landed at Boston.

At Boston I found friends ready to receive us with open arms,
though they were friends we had never known before.  I own that I
felt myself burdened with much nervous anxiety at my first
introduction to men and women in Boston.  I knew what the feeling
there was with reference to England, and I knew also how impossible
it is for an Englishman to hold his tongue and submit to dispraise
of England.  As for going among a people whose whole minds were
filled with affairs of the war, and saying nothing about the war, I
knew that no resolution to such an effect could be carried out.  If
one could not trust one's self to speak, one should have stayed at
home in England.  I will here state that I always did speak out
openly what I thought and felt, and that though I encountered very
strong---sometimes almost fierce---opposition, I never was subjected
to anything that was personally disagreeable to me.

In September we did not stay above a week in Boston, having been
fairly driven out of it by the musquitoes.  I had been told that I
should find nobody in Boston whom I cared to see, as everybody was
habitually out of town during the heat of the latter summer and
early autumn; but this was not so.  The war and attendant turmoils
of war had made the season of vacation shorter than usual, and most
of those for whom I asked were back at their posts.  I know no
place at which an Englishman may drop down suddenly among a
pleasanter circle of acquaintance, or find himself with a more
clever set of men, than he can do at Boston.  I confess that in
this respect I think that but few towns are at present more
fortunately circumstanced than the capital of the Bay State, as
Massachusetts is called, and that very few towns make a better use
of their advantages.  Boston has a right to be proud of what it has
done for the world of letters.  It is proud; but I have not found
that its pride was carried too far.

Boston is not in itself a fine city, but it is a very pleasant
city.  They say that the harbor is very grand and very beautiful.
It certainly is not so fine as that of Portland, in a nautical
point of view, and as certainly it is not as beautiful.  It is the
entrance from the sea into Boston of which people say so much; but
I did not think it quite worthy of all I had heard.  In such
matters, however, much depends on the peculiar light in which
scenery is seen.  An evening light is generally the best for all
landscapes; and I did not see the entrance to Boston harbor by an
evening light.  It was not the beauty of the harbor of which I
thought the most, but of the tea which had been sunk there, and of
all that came of that successful speculation.  Few towns now
standing have a right to be more proud of their antecedents than
Boston.

But as I have said, it is not specially interesting to the eye;
what new town, or even what simply adult town, can be so?  There is
an Atheneum, and a State Hall, and a fashionable street,---Beacon
Street, very like Piccadilly as it runs along the Green Park,---and
there is the Green Park opposite to this Piccadilly, called Boston
Common.  Beacon Street and Boston Common are very pleasant.
Excellent houses there are, and large churches, and enormous
hotels; but of such things as these a man can write nothing that is
worth the reading.  The traveler who desires to tell his experience
of North America must write of people rather than of things.

As I have said, I found myself instantly involved in discussions on
American politics and the bearing of England upon those politics.
``What do you think, you in England---what do you believe will be the
upshot of this war?''  That was the question always asked in those
or other words.  ``Secession, certainly,'' I always said, but not
speaking quite with that abruptness.  ``And you believe, then, that
the South will beat the North?''  I explained that I personally had
never so thought, and that I did not believe that to be the general
idea.  Men's opinions in England, however, were too divided to
enable me to say that there was any prevailing conviction on the
matter.  My own impression was, and is, that the North will, in a
military point of view, have the best of the contest---will beat the
South; but that the Northerners will not prevent secession, let
their success be what it may.  Should the North prevail after a two
years' conflict, the North will not admit the South to an equal
participation of good things with themselves, even though each
separate rebellious State should return suppliant, like a prodigal
son, kneeling on the floor of Congress, each with a separate rope
of humiliation round its neck.  Such was my idea as expressed then,
and I do not know that I have since had much cause to change it.

``We will never give it up,'' one gentleman said to me---and, indeed,
many have said the same---``till the whole territory is again united
from the Bay to the Gulf.  It is impossible that we should allow of
two nationalities within those limits.''  ``And do you think it
possible,'' I asked, ``that you should receive back into your bosom
this people which you now hate with so deep a hatred, and receive
them again into your arms as brothers on equal terms?  Is it in
accordance with experience that a conquered people should be so
treated, and that, too, a people whose every habit of life is at
variance with the habits of their presumed conquerors?  When you
have flogged them into a return of fraternal affection, are they to
keep their slaves or are they to abolish them?''  ``No,'' said my
friend, ``it may not be practicable to put those rebellious States
at once on an equality with ourselves.  For a time they will
probably be treated as the Territories are now treated.''  (The
Territories are vast outlying districts belonging to the Union, but
not as yet endowed with State governments or a participation in the
United States Congress.)  ``For a time they must, perhaps, lose
their full privileges; but the Union will be anxious to readmit
them at the earliest possible period.''  ``And as to the slaves?'' I
asked again.  ``Let them emigrate to Liberia---back to their own
country.''  I could not say that I thought much of the solution of
the difficulty.  It would, I suggested, overtask even the energy of
America to send out an emigration of four million souls, to provide
for their wants in a new and uncultivated country, and to provide,
after that, for the terrible gap made in the labor market of the
Southern States.  ``The Israelites went back from bondage,'' said my
friend.  But a way was opened for them by a miracle across the sea,
and food was sent to them from heaven, and they had among them a
Moses for a leader, and a Joshua to fight their battles.  I could
not but express my fear that the days of such immigrations were
over.  This plan of sending back the negroes to Africa did not
reach me only from one or from two mouths, and it was suggested by
men whose opinions respecting their country have weight at home and
are entitled to weight abroad.  I mention this merely to show how
insurmountable would be the difficulty of preventing secession, let
which side win that may.

``We will never abandon the right to the mouth of the Mississippi.''
That, in all such arguments, is a strong point with men of the
Northern States---perhaps the point to which they all return with
the greatest firmness.  It is that on which Mr.\ Everett insists in
the last paragraph of the oration which he made in New York on the
4th of July, 1861.  ``The Missouri and the Mississippi Rivers,'' he
says, ``with their hundred tributaries, give to the great central
basin of our continent its character and destiny.  The outlet of
this system lies between the States of Tennessee and Missouri, of
Mississippi and Arkansas, and through the State of Louisiana.  The
ancient province so called, the proudest monument of the mighty
monarch whose name it bears, passed from the jurisdiction of France
to that of Spain in 1763.  Spain coveted it---not that she might
fill it with prosperous colonies and rising States, but that it
might stretch as a broad waste barrier, infested with warlike
tribes, between the Anglo-American power and the silver mines of
Mexico.  With the independence of the United States the fear of a
still more dangerous neighbor grew upon Spain; and, in the insane
expectation of checking the progress of the Union westward, she
threatened, and at times attempted, to close the mouth of the
Mississippi on the rapidly-increasing trade of the West.  The bare
suggestion of such a policy roused the population upon the banks of
the Ohio, then inconsiderable, as one man.  Their confidence in
Washington scarcely restrained them from rushing to the seizure of
New Orleans, when the treaty of San Lorenzo El Real, in 1795,
stipulated for them a precarious right of navigating the noble
river to the sea, with a right of deposit at New Orleans.  This
subject was for years the turning-point of the politics of the
West; and it was perfectly well understood that, sooner or later,
she would be content with nothing less than the sovereign control
of the mighty stream from its head-spring to its outlet in the
Gulf.  \emph{And that is as true now as it was then.}''

This is well put.  It describes with force the desires, ambition,
and necessities of a great nation, and it tells with historical
truth the story of the success of that nation.  It was a great
thing done when the purchase of the whole of Louisiana was
completed by the United States---that cession by France, however,
having been made at the instance of Napoleon, and not in
consequence of any demand made by the States.  The district then
called Louisiana included the present State of that name and the
States of Missouri and Arkansas---included also the right to
possess, if not the absolute possession of all that enormous
expanse of country running from thence back to the Pacific: a huge
amount of territory, of which the most fertile portion is watered
by the Mississippi and its vast tributaries.  That river and those
tributaries are navigable through the whole center of the American
continent up to Wisconsin and Minnesota.  To the United States the
navigation of the Mississippi was, we may say, indispensable; and
to the States, when no longer united, the navigation will be
equally indispensable.  But the days are gone when any country such
as Spain was can interfere to stop the highways of the world with
the all but avowed intention of arresting the progress of
civilization.  It may be that the North and the South can never
again be friends as the component parts of one nation.  Such, I
take it, is the belief of all politicians in Europe, and of many of
those who live across the water.  But as separate nations they may
yet live together in amity, and share between them the great water-
ways which God has given them for their enrichment.  The Rhine is
free to Prussia and to Holland.  The Danube is not closed against
Austria.  It will be said that the Danube has in fact been closed
against Austria, in spite of treaties to the contrary.  But the
faults of bad and weak governments are made known as cautions to
the world, and not as facts to copy.  The free use of the waters of
a common river between two nations is an affair for treaty; and it
has not yet come to that that treaties must necessarily be null and
void through the falseness of politicians.

``And what will England do for cotton?  Is it not the fact that Lord
John Russell, with his professed neutrality, intends to express
sympathy with the South---intends to pave the way for the advent of
Southern cotton?''  ``You ought to love us,'' so say men in Boston,
``because we have been with you in heart and spirit for long, long
years.  But your trade has eaten into your souls, and you love
American cotton better than American loyalty and American
fellowship.''  This I found to be unfair, and in what politest
language I could use I said so.  I had not any special knowledge of
the minds of English statesmen on this matter; but I knew as well
as Americans could do what our statesmen had said and done
respecting it.  That cotton, if it came from the South, would be
made very welcome in Liverpool, of course I knew.  If private
enterprise could bring it, it might be brought.  But the very
declaration made by Lord John Russell was the surest pledge that
England, as a nation, would not interfere even to supply her own
wants.  It may easily be imagined what eager words all this would
bring about; but I never found that eager words led to feelings
which were personally hostile.

All the world has heard of Newport, in Rhode Island, as being the
Brighton, and Tenby, and Scarborough of New England.  And the glory
of Newport is by no means confined to New England, but is shared by
New York and Washington, and in ordinary years by the extreme
South.  It is the habit of Americans to go to some watering-place
every summer---that is, to some place either of sea water or of
inland waters.  This is done much in England, more in Ireland than
in England, but I think more in the States than even in Ireland.
But of all such summer haunts, Newport is supposed to be in many
ways the most captivating.  In the first place, it is certainly the
most fashionable, and, in the next place, it is said to be the most
beautiful.  We decided on going to Newport---led thither by the
latter reputation rather than the former.  As we were still in the
early part of September, we expected to find the place full, but in
this we were disappointed---disappointed, I say, rather than
gratified, although a crowded house at such a place is certainly a
nuisance.  But a house which is prepared to make up six hundred
beds, and which is called on to make up only twenty-five, becomes,
after awhile, somewhat melancholy.  The natural depression of the
landlord communicates itself to his servants, and from the servants
it descends to the twenty-five guests, who wander about the long
passages and deserted balconies like the ghosts of those of the
summer visitors, who cannot rest quietly in their graves at home.

In England we know nothing of hotels prepared for six hundred
visitors, all of whom are expected to live in common.  Domestic
architects would be frightened at the dimensions which are needed,
and at the number of apartments which are required to be clustered
under one roof.  We went to the Ocean Hotel at Newport, and
fancied, as we first entered the hall under a veranda as high as
the house, and made our way into the passage, that we had been
taken to a well-arranged barrack.  ``Have you rooms?'' I asked, as a
man always does ask on first reaching his inn.  ``Rooms enough,'' the
clerk said; ``we have only fifty here.''  But that fifty dwindled
down to twenty-five during the next day or two.

We were a melancholy set, the ladies appearing to be afflicted in
this way worse than the gentlemen, on account of their enforced
abstinence from tobacco.  What can twelve ladies do scattered about
a drawing-room, so called, intended for the accommodation of two
hundred?  The drawing-room at the Ocean Hotel, Newport, is not as
big as Westminster Hall, but would, I should think, make a very
good House of Commons for the British nation.  Fancy the feelings
of a lady when she walks into such a room, intending to spend her
evening there, and finds six or seven other ladies located on
various sofas at terrible distances, all strangers to her.  She has
come to Newport probably to enjoy herself; and as, in accordance
with the customs of the place, she has dined at two, she has
nothing before her for the evening but the society of that huge,
furnished cavern.  Her husband, if she have one, or her father, or
her lover, has probably entered the room with her.  But a man has
never the courage to endure such a position long.  He sidles out
with some muttered excuse, and seeks solace with a cigar.  The
lady, after half an hour of contemplation, creeps silently near
some companion in the desert, and suggests in a whisper that
Newport does not seem to be very full at present.

We stayed there for a week, and were very melancholy; but in our
melancholy we still talked of the war.  Americans are said to be
given to bragging, and it is a sin of which I cannot altogether
acquit them.  But I have constantly been surprised at hearing the
Northern men speak of their own military achievements with anything
but self-praise.  ``We've been whipped, sir; and we shall be whipped
again before we've done; uncommon well whipped we shall be.''  ``We
began cowardly, and were afraid to send our own regiments through
one of our own cities.''  This alluded to a demand that had been
made on the Government that troops going to Washington should not
be sent through Baltimore, because of the strong feeling for
rebellion which was known to exist in that city.  President Lincoln
complied with this request, thinking it well to avoid a collision
between the mob and the soldiers.  ``We began cowardly, and now
we're going on cowardly, and darn't attack them.  Well; when we've
been whipped often enough, then we shall learn the trade.''  Now all
this---and I heard much of such a nature---could not be called
boasting.  But yet with it all there was a substratum of
confidence.  I have heard Northern gentlemen complaining of the
President, complaining of all his ministers, one after another,
complaining of the contractors who were robbing the army, of the
commanders who did not know how to command the army, and of the
army itself, which did not know how to obey; but I do not remember
that I have discussed the matter with any Northerner who would
admit a doubt as to ultimate success.

We were certainly rather melancholy at Newport, and the empty house
may perhaps have given its tone to the discussions on the war.  I
confess that I could not stand the drawing-room---the ladies'
drawing-room, as such like rooms are always called at the hotels---%
and that I basely deserted my wife.  I could not stand it either
here or elsewhere, and it seemed to me that other husbands---ay, and
even lovers---were as hard pressed as myself.  I protest that there
is no spot on the earth's surface so dear to me as my own drawing-
room, or rather my wife's drawing-room, at home; that I am not a
man given hugely to clubs, but one rather rejoicing in the rustle
of petticoats.  I like to have women in the same room with me.  But
at these hotels I found myself driven away---propelled as it were by
some unknown force---to absent myself from the feminine haunts.
Anything was more palatable than them, even ``liquoring up'' at a
nasty bar, or smoking in a comfortless reading-room among a deluge
of American newspapers.  And I protest also---hoping as I do so that
I may say much in this book to prove the truth of such
protestation---that this comes from no fault of the American women.
They are as lovely as our own women.  Taken generally, they are
better instructed, though perhaps not better educated.  They are
seldom troubled with mauvaise honte; I do not say it in irony, but
begging that the words may be taken at their proper meaning.  They
can always talk, and very often can talk well.  But when assembled
together in these vast, cavernous, would-be luxurious, but in truth
horribly comfortless hotel drawing-rooms, they are unapproachable.
I have seen lovers, whom I have known to be lovers, unable to
remain five minutes in the same cavern with their beloved ones.

And then the music!  There is always a piano in a hotel drawing-
room, on which, of course, some one of the forlorn ladies is
generally employed.  I do not suppose that these pianos are in
fact, as a rule, louder and harsher, more violent and less musical,
than other instruments of the kind.  They seem to be so, but that,
I take it, arises from the exceptional mental depression of those
who have to listen to them.  Then the ladies, or probably some one
lady, will sing, and as she hears her own voice ring and echo
through the lofty corners and round the empty walls, she is
surprised at her own force, and with increased efforts sings louder
and still louder.  She is tempted to fancy that she is suddenly
gifted with some power of vocal melody unknown to her before, and,
filled with the glory of her own performance, shouts till the whole
house rings.  At such moments she at least is happy, if no one else
is so.  Looking at the general sadness of her position, who can
grudge her such happiness?

And then the children---babies, I should say if I were speaking of
English bairns of their age; but seeing that they are Americans, I
hardly dare to call them children.  The actual age of these
perfectly-civilized and highly-educated beings may be from three to
four.  One will often see five or six such seated at the long
dinner-table of the hotel, breakfasting and dining with their
elders, and going through the ceremony with all the gravity, and
more than all the decorum, of their grandfathers.  When I was three
years old I had not yet, as I imagine, been promoted beyond a
silver spoon of my own wherewith to eat my bread and milk in the
nursery; and I feel assured that I was under the immediate care of
a nursemaid, as I gobbled up my minced mutton mixed with potatoes
and gravy.  But at hotel life in the States the adult infant lisps
to the waiter for everything at table, handles his fish with
epicurean delicacy, is choice in his selection of pickles, very
particular that his beef-steak at breakfast shall be hot, and is
instant in his demand for fresh ice in his water.  But perhaps his,
or in this case her, retreat from the room when the meal is over,
is the chef-d'oeuvre of the whole performance.  The little,
precocious, full-blown beauty of four signifies that she has
completed her meal---or is ``through'' her dinner, as she would
express it---by carefully extricating herself from the napkin which
has been tucked around her.  Then the waiter, ever attentive to her
movements, draws back the chair on which she is seated, and the
young lady glides to the floor.  A little girl in Old England would
scramble down, but little girls in New England never scramble.  Her
father and mother, who are no more than her chief ministers, walk
before her out of the saloon, and then she---swims after them.  But
swimming is not the proper word.  Fishes, in making their way
through the water, assist, or rather impede, their motion with no
dorsal wriggle.  No animal taught to move directly by its Creator
adopts a gait so useless, and at the same time so graceless.  Many
women, having received their lessons in walking from a less
eligible instructor, do move in this way, and such women this
unfortunate little lady has been instructed to copy.  The peculiar
step to which I allude is to be seen often on the boulevards in
Paris.  It is to be seen more often in second-rate French towns,
and among fourth-rate French women.  Of all signs in women
betokening vulgarity, bad taste, and aptitude to bad morals, it is
the surest.  And this is the gait of going which American mothers---%
some American mothers I should say---love to teach their daughters!
As a comedy at a hotel it is very delightful, but in private life I
should object to it.

To me Newport could never be a place charming by reason of its own
charms.  That it is a very pleasant place when it is full of people
and the people are in spirits and happy, I do not doubt.  But then
the visitors would bring, as far as I am concerned, the
pleasantness with them.  The coast is not fine.  To those who know
the best portions of the coast of Wales or Cornwall---or better
still, the western coast of Ireland, of Clare and Kerry for
instance---it would not be in any way remarkable.  It is by no means
equal to Dieppe or Biarritz, and not to be talked of in the same
breath with Spezzia.  The hotels, too, are all built away from the
sea; so that one cannot sit and watch the play of the waves from
one's windows.  Nor are there pleasant rambling paths down among
the rocks, and from one short strand to another.  There is
excellent bathing for those who like bathing on shelving sand.  I
don't.  The spot is about half a mile from the hotels, and to this
the bathers are carried in omnibuses.  Till one o'clock ladies
bathe, which operation, however, does not at all militate against
the bathing of men, but rather necessitates it as regards those men
who have ladies with them.  For here ladies and gentlemen bathe in
decorous dresses, and are very polite to each other.  I must say
that I think the ladies have the best of it.  My idea of sea
bathing, for my own gratification, is not compatible with a full
suit of clothing.  I own that my tastes are vulgar, and perhaps
indecent; but I love to jump into the deep, clear sea from off a
rock, and I love to be hampered by no outward impediments as I do
so.  For ordinary bathers, for all ladies, and for men less savage
in their instincts than I am, the bathing at Newport is very good.

The private houses---villa residences as they would be termed by an
auctioneer in England---are excellent.  Many of them are, in fact,
large mansions, and are surrounded with grounds which, as the
shrubs grow up, will be very beautiful.  Some have large, well-kept
lawns, stretching down to the rocks, and these, to my taste, give
the charm to Newport.  They extend about two miles along the coast.
Should my lot have made me a citizen of the United States, I should
have had no objection to become the possessor of one of these
``villa residences;'' but I do not think that I should have ``gone in''
for hotel life at Newport.

We hired saddle-horses, and rode out nearly the length of the
island.  It was all very well, but there was little in it
remarkable either as regards cultivation or scenery.  We found
nothing that it would be possible either to describe or remember.
The Americans of the United States have had time to build and
populate vast cities, but they have not yet had time to surround
themselves with pretty scenery.  Outlying grand scenery is given by
nature; but the prettiness of home scenery is a work of art.  It
comes from the thorough draining of land, from the planting and
subsequent thinning of trees, from the controlling of waters, and
constant use of minute patches of broken land.  In another hundred
years or so, Rhode Island may be, perhaps, as pretty as the Isle of
Wight.  The horses which we got were not good.  They were unhandy
and badly mouthed, and that which my wife rode was altogether
ignorant of the art of walking.  We hired them from an Englishman
who had established himself at New York as a riding-master for
ladies, and who had come to Newport for the season on the same
business.  He complained to me with much bitterness of the saddle-
horses which came in his way---of course thinking that it was the
special business of a country to produce saddle-horses, as I think
it the special business of a country to produce pens, ink, and
paper of good quality.  According to him, riding has not yet become
an American art, and hence the awkwardness of American horses.
``Lord bless you, sir! they don't give an animal a chance of a
mouth.''  In this he alluded only, I presume, to saddle-horses.  I
know nothing of the trotting horses, but I should imagine that a
fine mouth must be an essential requisite for a trotting match in
harness.  As regards riding at Newport, we were not tempted to
repeat the experiment.  The number of carriages which we saw there---%
remembering as I did that the place was comparatively empty---and
their general smartness, surprised me very much.  It seemed that
every lady, with a house of her own, had also her own carriage.
These carriages were always open, and the law of the land
imperatively demands that the occupants shall cover their knees
with a worked worsted apron of brilliant colors.  These aprons at
first I confess seemed tawdry; but the eye soon becomes used to
bright colors, in carriage aprons as well as in architecture, and I
soon learned to like them.

Rhode Island, as the State is usually called, is the smallest State
in the Union.  I may perhaps best show its disparity to other
States by saying that New York extends about two hundred and fifty
miles from north to south, and the same distance from east to west;
whereas the State called Rhode Island is about forty miles long by
twenty broad, independently of certain small islands.  It would, in
fact, not form a considerable addition if added on to many of the
other States.  Nevertheless, it has all the same powers of self-
government as are possessed by such nationalities as the States of
New York and Pennsylvania, and sends two Senators to the Senate at
Washington, as do those enormous States.  Small as the State is,
Rhode Island itself forms but a small portion of it.  The
authorized and proper name of the State is Providence Plantation
and Rhode Island.  Roger Williams was the first founder of the
colony, and he established himself on the mainland at a spot which
he called Providence.  Here now stands the City of Providence, the
chief town of the State; and a thriving, comfortable town it seems
to be, full of banks, fed by railways and steamers, and going ahead
quite as quickly as Roger Williams could in his fondest hopes have
desired.

Rhode Island, as I have said, has all the attributes of government
in common with her stouter and more famous sisters.  She has a
governor, and an upper house and a lower house of legislature; and
she is somewhat fantastic in the use of these constitutional
powers, for she calls on them to sit now in one town and now in
another.  Providence is the capital of the State; but the Rhode
Island parliament sits sometimes at Providence and sometimes at
Newport.  At stated times also it has to collect itself at Bristol,
and at other stated times at Kingston, and at others at East
Greenwich.  Of all legislative assemblies it is the most
peripatetic.  Universal suffrage does not absolutely prevail in
this State, a certain property qualification being necessary to
confer a right to vote even for the State representatives.  I
should think it would be well for all parties if the whole State
could be swallowed up by Massachusetts or by Connecticut, either of
which lie conveniently for the feat; but I presume that any
suggestion of such a nature would be regarded as treason by the men
of Providence Plantation.

We returned back to Boston by Attleborough, a town at which, in
ordinary times, the whole population is supported by the jewelers'
trade.  It is a place with a specialty, upon which specialty it has
thriven well and become a town.  But the specialty is one ill
adapted for times of war and we were assured that the trade was for
the present at an end.  What man could now-a-days buy jewels, or
even what woman, seeing that everything would be required for the
war?  I do not say that such abstinence from luxury has been
begotten altogether by a feeling of patriotism.  The direct taxes
which all Americans will now be called on to pay, have had and will
have much to do with such abstinence.  In the mean time the poor
jewelers of Attleborough have gone altogether to the wall.



\chapter{Maine, New Hampshire, and Vermont}


Perhaps I ought to assume that all the world in England knows that
that portion of the United States called New England consists of
the six States of Maine, New Hampshire, Vermont, Massachusetts,
Connecticut, and Rhode Island.  This is especially the land of
Yankees, and none can properly be called Yankees but those who
belong to New England.  I have named the States as nearly as may be
in order from the north downward.  Of Rhode Island, the smallest
State in the Union, I have already said what little I have to say.
Of these six States Boston may be called the capital.  Not that it
is so in any civil or political sense; it is simply the capital of
Massachusetts.  But as it is the Athens of the Western world; as it
was the cradle of American freedom; as everybody of course knows
that into Boston harbor was thrown the tea which George III. would
tax, and that at Boston, on account of that and similar taxes,
sprang up the new revolution; and as it has grown in wealth, and
fame, and size beyond other towns in New England, it may be allowed
to us to regard it as the capital of these six Northern States,
without guilt of lese majeste toward the other five.  To me, I
confess this Northern division of our once-unruly colonies is, and
always has been, the dearest.  I am no Puritan myself, and fancy
that, had I lived in the days of the Puritans, I should have been
anti-Puritan to the full extent of my capabilities.  But I should
have been so through ignorance and prejudice, and actuated by that
love of existing rights and wrongs which men call loyalty.  If the
Canadas were to rebel now, I should be for putting down the
Canadians with a strong hand; but not the less have I an idea that
it will become the Canadas to rebel and assert their independence
at some future period, unless it be conceded to them without such
rebellion.  Who, on looking back, can now refuse to admire the
political aspirations of the English Puritans, or decline to
acknowledge the beauty and fitness of what they did?  It was by
them that these States of New England were colonized.  They came
hither, stating themselves to be pilgrims, and as such they first
placed their feet on that hallowed rock at Plymouth, on the shore
of Massachusetts.  They came here driven by no thirst of conquest,
by no greed for gold, dreaming of no Western empire such as Cortez
had achieved and Raleigh had meditated.  They desired to earn their
bread in the sweat of their brow, worshiping God according to their
own lights, living in harmony under their own laws, and feeling
that no master could claim a right to put a heel upon their necks.
And be it remembered that here in England, in those days, earthly
masters were still apt to put their heels on the necks of men.  The
Star Chamber was gone, but Jeffreys had not yet reigned.  What
earthly aspirations were ever higher than these, or more manly?
And what earthly efforts ever led to grander results?

We determined to go to Portland, in Maine, from thence to the White
Mountains in New Hampshire---the American Alps, as they love to call
them---and then on to Quebec, and up through the two Canadas to
Niagara; and this route we followed.  From Boston to Portland we
traveled by railroad---the carriages on which are in America always
called cars.  And here I beg, once for all, to enter my protest
loudly against the manner in which these conveyances are conducted.
The one grand fault---there are other smaller faults---but the one
grand fault is that they admit but one class.  Two reasons for this
are given.  The first is that the finances of the companies will
not admit of a divided accommodation; and the second is that the
republican nature of the people will not brook a superior or
aristocratic classification of traveling.  As regards the first, I
do not in the least believe in it.  If a more expensive manner of
railway traveling will pay in England, it would surely do so here.
Were a better class of carriages organized, as large a portion of
the population would use them in the United States as in any
country in Europe.  And it seems to be evident that in arranging
that there shall be only one rate of traveling, the price is
enhanced on poor travelers exactly in proportion as it is made
cheap to those who are not poor.  For the poorer classes, traveling
in America is by no means cheap, the average rate being, as far as
I can judge, fully three halfpence a mile.  It is manifest that
dearer rates for one class would allow of cheaper rates for the
other; and that in this manner general traveling would be
encouraged and increased.

But I do not believe that the question of expenditure has had
anything to do with it.  I conceive it to be true that the railways
are afraid to put themselves at variance with the general feeling
of the people.  If so, the railways may be right.  But then, on the
other band, the general feeling of the people must in such case be
wrong.  Such a feeling argues a total mistake as to the nature of
that liberty and equality for the security of which the people are
so anxious, and that mistake the very one which has made shipwreck
so many attempts at freedom in other countries.  It argues that
confusion between social and political equality which has led
astray multitudes who have longed for liberty fervently, but who
have not thought of it carefully.  If a first-class railway
carriage should be held as offensive, so should a first-class
house, or a first-class horse, or a first-class dinner.  But first-
class houses, first-class horses, and first-class dinners are very
rife in America.  Of course it may be said that the expenditure
shown in these last-named objects is private expenditure, and
cannot be controlled; and that railway traveling is of a public
nature, and can be made subject to public opinion.  But the fault
is in that public opinion which desires to control matters of this
nature.  Such an arrangement partakes of all the vice of a
sumptuary law, and sumptuary laws are in their very essence
mistakes.  It is well that a man should always have all for which
he is willing to pay.  If he desires and obtains more than is good
for him, the punishment, and thus also the preventive, will come
from other sources.

It will be said that the American cars are good enough for all
purposes.  The seats are not very hard, and the room for sitting is
sufficient.  Nevertheless I deny that they are good enough for all
purposes.  They are very long, and to enter them and find a place
often requires a struggle and almost a fight.  There is rarely any
person to tell a stranger which car he should enter.  One never
meets an uncivil or unruly man, but the women of the lower ranks
are not courteous.  American ladies love to lie at ease in their
carriages, as thoroughly as do our women in Hyde Park; and to those
who are used to such luxury, traveling by railroad in their own
country must be grievous.  I would not wish to be thought a
Sybarite myself, or to be held as complaining because I have been
compelled to give up my seat to women with babies and bandboxes who
have accepted the courtesy with very scanty grace.  I have borne
worse things than these, and have roughed it much in my days, from
want of means and other reasons.  Nor am I yet so old but what I
can rough it still.  Nevertheless I like to see things as well done
as is practicable, and railway traveling in the States is not well
done.  I feel bound to say as much as this, and now I have said it,
once for all.

Few cities, or localities for cities, have fairer natural
advantages than Portland and I am bound to say that the people of
Portland have done much in turning them to account.  This town is
not the capital of the State in a political point of view.
Augusta, which is farther to the north, on the Kennebec River, is
the seat of the State government for Maine.  It is very generally
the case that the States do not hold their legislatures and carry
on their government at their chief towns.  Augusta and not Portland
is the capital of Maine.  Of the State of New York, Albany is the
capital, and not the city which bears the State's name.  And of
Pennsylvania, Harrisburg and not Philadelphia is the capital.  I
think the idea has been that old-fashioned notions were bad in that
they were old fashioned; and that a new people, bound by no
prejudices, might certainly make improvement by choosing for
themselves new ways.  If so, the American politicians have not been
the first in the world who have thought that any change must be a
change for the better.  The assigned reason is the centrical
position of the selected political capitals; but I have generally
found the real commercial capital to be easier of access than the
smaller town in which the two legislative houses are obliged to
collect themselves.

What must be the natural excellence of the harbor of Portland, will
be understood when it is borne in mind that the Great Eastern can
enter it at all times, and that it can lay along the wharves at any
hour of the tide.  The wharves which have been prepared for her---%
and of which I will say a word further by-and-by---are joined to,
and in fact, are a portion of, the station of the Grand Trunk
Railway, which runs from Portland up to Canada.  So that passengers
landing at Portland out of a vessel so large even as the Great
Eastern can walk at once on shore, and goods can be passed on to
the railway without any of the cost of removal.  I will not say
that there is no other harbor in the world that would allow of
this, but I do not know any other that would do so.

From Portland a line of railway, called as a whole by the name of
the Canada Grand Trunk Line, runs across the State of Maine,
through the northern parts of New Hampshire and Vermont, to
Montreal, a branch striking from Richmond, a little within the
limits of Canada, to Quebec, and down the St. Lawrence to Riviere
du Loup.  The main line is continued from Montreal, through Upper
Canada to Toronto, and from thence to Detroit in the State of
Michigan.  The total distance thus traversed is, in a direct line,
about 900 miles.  From Detroit there is railway communications
through the immense Northwestern States of Michigan, Wisconsin, and
Illinois, than which perhaps the surface of the globe affords no
finer districts for purposes of agriculture.  The produce of the
two Canadas must be poured forth to the Eastern world, and the men
of the Eastern world must throng into these lands by means of this
railroad, and, as at present arranged, through the harbor of
Portland.  At present the line has been opened, and they who have
opened are sorely suffering in pocket for what they have done.  The
question of the railway is rather one applying to Canada than to
the State of Maine, and I will therefore leave it for the present.

But the Great Eastern has never been to Portland, and as far as I
know has no intention of going there.  She was, I believe, built
with that object.  At any rate, it was proclaimed during her
building that such was her destiny, and the Portlanders believed it
with a perfect faith.  They went to work and built wharves
expressly for her; two wharves prepared to fit her two gangways, or
ways of exit and entrance.  They built a huge hotel to receive her
passengers.  They prepared for her advent with a full conviction
that a millennium of trade was about to be wafted to their happy
port.  ``Sir, the town has expended two hundred thousand dollars in
expectation of that ship, and that ship has deceived us.''  So was
the matter spoken of to me by an intelligent Portlander.  I
explained to that intelligent gentleman that two hundred thousand
dollars would go a very little way toward making up the loss which
the ill-fortuned vessel had occasioned on the other side of the
water.  He did not in words express gratification at this
information, but he looked it.  The matter was as it were a
partnership without deed of contract between the Portlanders and
the shareholders of the vessel, and the Portlanders, though they
also have suffered their losses, have not had the worst of it.

But there are still good days in store for the town.  Though the
Great Eastern has not gone there, other ships from Europe, more
profitable if less in size, must eventually find their way thither.
At present the Canada line of packets runs to Portland only during
those months in which it is shut out from the St. Lawrence and
Quebec by ice.  But the St. Lawrence and Quebec cannot offer the
advantages which Portland enjoys, and that big hotel and those new
wharves will not have been built in vain.

I have said that a good time is coming, but I would by no means
wish to signify that the present times in Portland are bad.  So far
from it that I doubt whether I ever saw a town with more evident
signs of prosperity.  It has about it every mark of ample means,
and no mark of poverty.  It contains about 27,000 people, and for
that population covers a very large space of ground.  The streets
are broad and well built, the main streets not running in those
absolutely straight parallels which are so common in American
towns, and are so distressing to English eyes and English feelings.
All these, except the streets devoted exclusively to business, are
shaded on both sides by trees, generally, if I remember rightly, by
the beautiful American elm, whose drooping boughs have all the
grace of the willow without its fantastic melancholy.  What the
poorer streets of Portland may be like, I cannot say.  I saw no
poor street.  But in no town of 30,000 inhabitants did I ever see
so many houses which must require an expenditure of from six to
eight hundred a year to maintain them.

The place, too, is beautifully situated.  It is on a long
promontory, which takes the shape of a peninsula, for the neck
which joins it to the main-land is not above half a mile across.
But though the town thus stands out into the sea, it is not exposed
and bleak.  The harbor, again, is surrounded by land, or so guarded
and locked by islands as to form a series of salt-water lakes
running round the town.  Of those islands there are, of course,
three hundred and sixty-five.  Travelers who write their travels
are constantly called upon to record that number, so that it may
now be considered as a superlative in local phraseology, signifying
a very great many indeed.  The town stands between two hills, the
suburbs or outskirts running up on to each of them.  The one
looking out toward the sea is called Mountjoy, though the obstinate
Americans will write it Munjoy on their maps.  From thence the view
out to the harbor and beyond the harbor to the islands is, I may
not say unequaled, or I shall be guilty of running into
superlatives myself, but it is in its way equal to anything I have
seen.  Perhaps it is more like Cork harbor, as seen from certain
heights over Passage, than anything else I can remember; but
Portland harbor, though equally landlocked, is larger; and then
from Portland harbor there is, as it were, a river outlet running
through delicious islands, most unalluring to the navigator, but
delicious to the eyes of an uncommercial traveler.  There are in
all four outlets to the sea, one of which appears to have been made
expressly for the Great Eastern.  Then there is the hill looking
inward.  If it has a name, I forget it.  The view from this hill is
also over the water on each side, and, though not so extensive, is
perhaps as pleasing as the other.

The ways of the people seemed to be quiet, smooth, orderly, and
republican.  There is nothing to drink in Portland, of course; for,
thanks to Mr.\ Neal Dow, the Father Matthew of the State of Maine,
the Maine liquor law is still in force in that State.  There is
nothing to drink, I should say, in such orderly houses as that I
selected.  ``People do drink some in the town, they say,'' said my
hostess to me, ``and liquor is to be got.  But I never venture to
sell any.  An ill-natured person might turn on me; and where should
I be then?''  I did not press her, and she was good enough to put a
bottle of porter at my right hand at dinner, for which I observed
she made no charge.  ``But they advertise beer in the shop windows,''
I said to a man who was driving me---``Scotch ale and bitter beer.  A
man can get drunk on them.''  ``Waal, yes.  If he goes to work hard,
and drinks a bucketful,'' said the driver, ``perhaps he may.''  From
which and other things I gathered that the men of Maine drank
pottle deep before Mr.\ Neal Dow brought his exertions to a
successful termination.

The Maine liquor law still stands in Maine, and is the law of the
land throughout New England; but it is not actually put in force in
the other States.  By this law no man may retail wine, spirits, or,
in truth, beer, except with a special license, which is given only
to those who are presumed to sell them as medicines.  A man may
have what he likes in his own cellar for his own use---such, at
least, is the actual working of the law---but may not obtain it at
hotels and public houses.  This law, like all sumptuary laws, must
fail.  And it is fast failing even in Maine.  But it did appear to
me, from such information as I could collect, that the passing of
it had done much to hinder and repress a habit of hard drinking
which was becoming terribly common, not only in the towns of Maine,
but among the farmers and hired laborers in the country.

But, if the men and women of Portland may not drink, they may eat;
and it is a place, I should say, in which good living on that side
of the question is very rife.  It has an air of supreme plenty, as
though the agonies of an empty stomach were never known there.  The
faces of the people tell of three regular meals of meat a day, and
of digestive powers in proportion.  O happy Portlanders, if they
only knew their own good fortune!  They get up early, and go to bed
early.  The women are comely and sturdy, able to take care of
themselves, without any fal-lal of chivalry, and the men are
sedate, obliging, and industrious.  I saw the young girls in the
streets coming home from their tea parties at nine o'clock, many of
them alone, and all with some basket in their hands, which
betokened an evening not passed absolutely in idleness.  No fear
there of unruly questions on the way, or of insolence from the ill-
conducted of the other sex.  All was, or seemed to be, orderly,
sleek, and unobtrusive.  Probably, of all modes of life that are
allotted to man by his Creator, life such as this is the most
happy.  One hint, however, for improvement, I must give even to
Portland: It would be well if they could make their streets of some
material harder than sand.

I must not leave the town without desiring those who may visit it
to mount the observatory.  They will from thence get the best view
of the harbor and of the surrounding land; and, if they chance to
do so under the reign of the present keeper of the signals, they
will find a man there able and willing to tell them everything
needful about the State of Maine in general and the harbor in
particular.  He will come out in his shirt sleeves, and, like a
true American, will not at first be very smooth in his courtesy;
but he will wax brighter in conversation, and, if not stroked the
wrong way, will turn out to be an uncommonly pleasant fellow.  Such
I believe to be the case with most of them.

From Portland we made our way up to the White Mountains, which lay
on our route to Canada.  Now, I would ask any of my readers who are
candid enough to expose their own ignorance whether they ever
heard, or at any rate whether they know anything, of the White
Mountains?  As regards myself, I confess that the name had reached
my ears; that I had an indefinite idea that they formed an
intermediate stage between the Rocky Mountains and the Alleghanies;
and that they were inhabited either by Mormons, Indians, or simply
by black bears.  That there was a district in New England
containing mountain scenery superior to much that is yearly crowded
by tourists in Europe, that this is to be reached with ease by
railways and stagecoaches, and that it is dotted with huge hotels
almost as thickly as they lie in Switzerland, I had no idea.  Much
of this scenery, I say, is superior to the famed and classic lands
of Europe.  I know nothing, for instance, on the Rhine equal to the
view from Mount Willard down the mountain pass called the Notch.

Let the visitor of these regions be as late in the year as he can,
taking care that he is not so late as to find the hotels closed.
October, no doubt, is the most beautiful month among these
mountains; but, according to the present arrangement of matters
here, the hotels are shut up by the end of September.  With us,
August, September, and October are the holiday months; whereas our
rebel children across the Atlantic love to disport themselves in
July and August.  The great beauty of the autumn, or fall, is in
the brilliant hues which are then taken by the foliage.  The
autumnal tints are fine with us.  They are lovely and bright
wherever foliage and vegetation form a part of the beauty of
scenery.  But in no other land do they approach the brilliancy of
the fall in America.  The bright rose color, the rich bronze which
is almost purple in its richness, and the glorious golden yellows
must be seen to be understood.  By me, at any rate, they cannot be
described.  They begin to show themselves in September; and perhaps
I might name the latter half of that month as the best time for
visiting the White Mountains.

I am not going to write a guide book, feeling sure that Mr.\ Murray
will do New England and Canada, including Niagara, and the Hudson
River, with a peep into Boston and New York, before many more
seasons have passed by.  But I cannot forbear to tell my countrymen
that any enterprising individual, with a hundred pounds to spend on
his holiday---a hundred and twenty would make him more comfortable
in regard to wine, washing, and other luxuries---and an absence of
two months from his labors, may see as much and do as much here for
the money as he can see or do elsewhere.  In some respects he may
do more; for he will learn more of American nature in such a
journey than he can ever learn of the nature of Frenchmen or
Americans by such an excursion among them.  Some three weeks of the
time, or perhaps a day or two over, he must be at sea, and that
portion of his trip will cost him fifty pounds, presuming that he
chooses to go in the most comfortable and costly way; but his time
on board ship will not be lost.  He will learn to know much of
Americans there, and will perhaps form acquaintances of which he
will not altogether lose sight for many a year.  He will land at
Boston, and, staying a day or two there, will visit Cambridge,
Lowell, and Bunker Hill, and, if he be that way given, will
remember that here live, and occasionally are to be seen alive, men
such as Longfellow, Emerson, Hawthorne, and a host of others, whose
names and fames have made Boston the throne of Western literature.
He will then, if he take my advice and follow my track, go by
Portland up into the White Mountains.  At Gorham, a station on the
Grand Trunk Line, he will find a hotel as good as any of its kind,
and from thence he will take a light wagon, so called in these
countries.  And here let me presume that the traveler is not alone:
he has his wife or friend, or perhaps a pair of sisters, and in his
wagon he will go up through primeval forests to the Glen House.
When there, he will ascend Mount Washington on a pony.  That is de
rigueur, and I do not therefore dare to recommend him to omit the
ascent.  I did not gain much myself by my labor.  He will not stay
at the Glen House, but will go on to---Jackson's I think they call
the next hotel, at which he will sleep.  From thence he will take
his wagon on through the Notch to the Crawford house, sleeping
there again; and when here, let him, of all things, remember to go
up Mount Willard.  It is but a walk of two hours up and down, if so
much.  When reaching the top, he will be startled to find that he
looks down into the ravine without an inch of foreground.  He will
come out suddenly on a ledge of rock, from whence, as it seems, he
might leap down at once into the valley below.  Then, going on from
the Crawford House, he will be driven through the woods of Cherry
Mount, passing, I fear without toll of custom, the house of my
excellent friend Mr.\ Plaistead, who keeps a hotel at Jefferson.
``Sir,'' said Mr.\ Plaistead, ``I have everything here that a man ought
to want: air, sir, that aint to be got better nowhere; trout,
chickens, beef, mutton, milk---and all for a dollar a day!  A-top of
that hill, sir, there's a view that aint to be beaten this side of
the Atlantic, or I believe the other.  And an echo, sir!---we've an
echo that comes back to us six times, sir; floating on the light
wind, and wafted about from rock to rock, till you would think the
angels were talking to you.  If I could raise that echo, sir, every
day at command, I'd give a thousand dollars for it.  It would be
worth all the money to a house like this.''  And he waved his hand
about from hill to hill, pointing out in graceful curves the lines
which the sounds would take.  Had destiny not called on Mr.\ %
Plaistead to keep an American hotel, he might have been a poet.

My traveler, however, unless time were plenty with him, would pass
Mr.\ Plaistead, merely lighting a friendly cigar, or perhaps
breaking the Maine liquor law if the weather be warm, and would
return to Gorham on the railway.  All this mountain district is in
New Hampshire; and, presuming him to be capable of going about the
world with his mouth, ears, and eyes open, he would learn much of
the way in which men are settling themselves in this still
sparsely-populated country.  Here young farmers go into the woods
as they are doing far down West in the Territories, and buying some
hundred acres at perhaps six shillings an acre, fell and burn the
trees, and build their huts, and take the first steps, as far as
man's work is concerned, toward accomplishing the will of the
Creator in those regions.  For such pioneers of civilization there
is still ample room even in the long-settled States of New
Hampshire and Vermont.

But to return to my traveler, whom, having brought so far, I must
send on.  Let him go on from Gorham to Quebec and the heights of
Abraham, stopping at Sherbrooke that he might visit from thence the
Lake of Memphra Magog.  As to the manner of traveling over this
ground I shall say a little in the next chapter, when I come to the
progress of myself and my wife.  From Quebec he will go up the St.
Lawrence to Montreal.  He will visit Ottawa, the new capital, and
Toronto.  He will cross the lake to Niagara, resting probably at
the Clifton House on the Canada side.  He will then pass on to
Albany, taking the Trenton Falls on his way.  From Albany he will
go down the Hudson to West Point.  He cannot stop at the Catskill
Mountains, for the hotel will be closed.  And then he will take the
river boat, and in a few hours will find himself at New York.  If
he desires to go into American city society, he will find New York
agreeable; but in that case he must exceed his two months.  If he
do not so desire, a short sojourn at New York will show him all
that there is to be seen and all that there is not to be seen in
that great city.  That the Cunard line of steamers will bring him
safely back to Liverpool in about eleven days, I need not tell to
any Englishman, or, as I believe, to any American.  So much, in the
spirit of a guide, I vouchsafe to all who are willing to take my
counsel---thereby anticipating Murray, and leaving these few pages
as a legacy to him or to his collaborateurs.

I cannot say that I like the hotels in those parts, or, indeed, the
mode of life at American hotels in general.  In order that I may
not unjustly defame them, I will commence these observations by
declaring that they are cheap to those who choose to practice the
economy which they encourage, that the viands are profuse in
quantity and wholesome in quality, that the attendance is quick and
unsparing, and that travelers are never annoyed by that grasping,
greedy hunger and thirst after francs and shillings which disgrace,
in Europe, many English and many continental inns.  All this is, as
must be admitted, great praise; and yet I do not like the American
hotels.

One is in a free country, and has come from a country in which one
has been brought up to hug one's chains---so at least the English
traveler is constantly assured---and yet in an American inn one can
never do as one likes.  A terrific gong sounds early in the
morning, breaking one's sweet slumbers; and then a second gong,
sounding some thirty minutes later, makes you understand that you
must proceed to breakfast whether you be dressed or no.  You
certainly can go on with your toilet, and obtain your meal after
half an hour's delay.  Nobody actually scolds you for so doing, but
the breakfast is, as they say in this country, ``through.''  You sit
down alone, and the attendant stands immediately over you.
Probably there are two so standing.  They fill your cup the instant
it is empty.  They tender you fresh food before that which has
disappeared from your plate has been swallowed.  They begrudge you
no amount that you can eat or drink; but they begrudge you a single
moment that you sit there neither eating nor drinking.  This is
your fate if you're too late; and therefore, as a rule, you are not
late.  In that case, you form one of a long row of eaters who
proceed through their work with a solid energy that is past all
praise.  It is wrong to say that Americans will not talk at their
meals.  I never met but few who would not talk to me, at any rate
till I got to the far West; but I have rarely found that they would
address me first.  Then the dinner comes early---at least it always
does so in New England---and the ceremony is much of the same kind.
You came there to eat, and the food is pressed upon you ad nauseam.
But, as far as one can see, there is no drinking.  In these days, I
am quite aware that drinking has become improper, even in England.
We are apt, at home, to speak of wine as a thing tabooed, wondering
how our fathers lived and swilled.  I believe that, as a fact, we
drink as much as they did; but, nevertheless, that is our theory.
I confess, however, that I like wine.  It is very wicked, but it
seems to me that my dinner goes down better with a glass of sherry
than without it.  As a rule, I always did get it at hotels in
America.  But I had no comfort with it.  Sherry they do not
understand at all.  Of course I am only speaking of hotels.  Their
claret they get exclusively from Mr.\ Gladstone, and, looking at the
quality, have a right to quarrel even with Mr.\ Gladstone's price.
But it is not the quality of the wine that I hereby intend to
subject to ignominy so much as the want of any opportunity for
drinking it.  After dinner, if all that I hear be true, the
gentlemen occasionally drop into the hotel bar and ``liquor up.''  Or
rather this is not done specially after dinner, but, without
prejudice to the hour, at any time that may be found desirable.  I
also have ``liquored up,'' but I cannot say that I enjoy the process.
I do not intend hereby to accuse Americans of drinking much; but I
maintain that what they do drink, they drink in the most
uncomfortable manner that the imagination can devise.

The greatest luxury at an English inn is one's tea, one's fire, and
one's book.  Such an arrangement is not practicable at an American
hotel.  Tea, like breakfast, is a great meal, at which meat should
be eaten, generally with the addition of much jelly, jam, and sweet
preserve; but no person delays over his teacup.  I love to have my
teacup emptied and filled with gradual pauses, so that time for
oblivion may accrue, and no exact record be taken.  No such meal is
known at American hotels.  It is possible to hire a separate room,
and have one's meals served in it; but in doing so a man runs
counter to all the institutions of the country, and a woman does so
equally.  A stranger does not wish to be viewed askance by all
around him; and the rule which holds that men at Rome should do as
Romans do, if true anywhere, is true in America.  Therefore I say
that in an American inn one can never do as one pleases.

In what I have here said I do not intend to speak of hotels in the
largest cities, such as Boston or New York.  At them meals are
served in the public room separately, and pretty nearly at any or
at all hours of the day; but at them also the attendant stands over
the unfortunate eater and drives him.  The guest feels that he is
controlled by laws adapted to the usages of the Medes and Persians.
He is not the master on the occasion, but the slave---a slave well
treated, and fattened up to the full endurance of humanity, but yet
a slave.

From Gorham we went on to Island Pond, a station on the same Canada
Trunk Railway, on a Saturday evening, and were forced by the
circumstances of the line to pass a melancholy Sunday at the place.
The cars do not run on Sundays, and run but once a day on other
days over the whole line, so that, in fact, the impediment to
traveling spreads over two days.  Island Pond is a lake with an
island in it; and the place which has taken the name is a small
village, about ten years old, standing in the midst of uncut
forests, and has been created by the railway.  In ten years more
there will no doubt be a spreading town at Island Pond; the forests
will recede; and men, rushing out from the crowded cities, will
find here food, and space, and wealth.  For myself, I never remain
long in such a spot without feeling thankful that it has not been
my mission to be a pioneer of civilization.

The farther that I got away from Boston the less strong did I find
the feeling of anger against England.  There, as I have said
before, there was a bitter animosity against the mother country in
that she had shown no open sympathy with the North.  In Maine and
New Hampshire I did not find this to be the case to any violent
degree.  Men spoke of the war as openly as they did at Boston, and,
in speaking to me, generally connected England with the subject.
But they did so simply to ask questions as to England's policy.
What will she do for cotton when her operatives are really pressed?
Will she break the blockade?  Will she insist on a right to trade
with Charleston and new Orleans?  I always answered that she would
insist on no such right, if that right were denied to others and
the denial enforced.  England, I took upon myself to say, would not
break a veritable blockade, let her be driven to what shifts she
might in providing for her operatives.  ``Ah! that's what we fear,''
a very stanch patriot said to me, if words may be taken as a proof
of stauchness.  ``If England allies herself with the Southerners,
all our trouble is for nothing.''  It was impossible not to feel
that all that was said was complimentary to England.  It is her
sympathy that the Northern men desire, to her co-operation that
they would willingly trust, on her honesty that they would choose
to depend.  It is the same feeling whether it shows itself in anger
or in curiosity.  An American, whether he be embarked in politics,
in literature, or in commerce, desires English admiration, English
appreciation of his energy, and English encouragement.  The anger
of Boston is but a sign of its affectionate friendliness.  What
feeling is so hot as that of a friend when his dearest friend
refuses to share his quarrel or to sympathize in his wrongs!  To my
thinking, the men of Boston are wrong and unreasonable in their
anger; but were I a man of Boston, I should be as wrong and as
unreasonable as any of them.  All that, however, will come right.
I will not believe it possible that there should in very truth be a
quarrel between England and the Northern States.

In the guidance of those who are not quite au fait at the details
of American government, I will here in a few words describe the
outlines of State government as it is arranged in New Hampshire.
The States, in this respect, are not all alike, the modes of
election of their officers, and periods of service, being
different.  Even the franchise is different in different States.
Universal suffrage is not the rule throughout the United States,
though it is, I believe, very generally thought in England that
such is the fact.  I need hardly say that the laws in the different
States may be as various as the different legislatures may choose
to make them.

In New Hampshire universal suffrage does prevail, which means that
any man may vote who lives in the State, supports himself, and
assists to support the poor by means of poor rates.  A governor of
the State is elected for one year only; but it is customary, or at
any rate not uncustomary, to re-elect him for a second year.  His
salary is a thousand dollars a year, or two hundred pounds.  It
must be presumed, therefore, that glory, and not money, is his
object.  To him is appended a Council, by whose opinions he must in
a great degree be guided.  His functions are to the State what
those of the President are to the country; and, for the short
period of his reign, he is as it were a Prime Minister of the
State, with certain very limited regal attributes.  He, however, by
no means enjoys the regal attribute of doing no wrong.  In every
State there is an Assembly, consisting of two houses of elected
representatives---the Senate, or upper house, and the House of
Representatives so called.  In New Hampshire, this Assembly or
Parliament is styled The General Court of New Hampshire.  It sits
annually, whereas the legislature in many States sits only every
other year.  Both houses are re-elected every year.  This Assembly
passes laws with all the power vested in our Parliament, but such
laws apply of course only to the State in question.  The Governor
of the State has a veto on all bills passed by the two houses.
But, after receipt of his veto, any bill so stopped by the Governor
can be passed by a majority of two-thirds in each house.  The
General Court usually sits for about ten weeks.  There are in the
State eight judges---three supreme, who sit at Concord, the capital,
as a court of appeal both in civil and criminal matters, and then
five lesser judges, who go circuit through the State.  The salaries
of these lesser judges do not exceed from 250 pounds to 300 pounds
a year; but they are, I believe, allowed to practice as lawyers in
any counties except those in which they sit as judges---being
guided, in this respect, by the same law as that which regulates
the work of assistant barristers in Ireland.  The assistant
barristers in Ireland are attached to the counties as judges at
Quarter Sessions, but they practice, or may practice, as advocates
in all counties except that to which they are so attached.  The
judges in New Hampshire are appointed by the Governor, with the
assistance of his Council.  No judge in New Hampshire can hold his
seat after he has reached seventy years of age.

So much at the present moment with reference to the government of
New Hampshire.



\chapter{Lower Canada}


The Grand Trunk Railway runs directly from Portland to Montreal,
which latter town is, in fact, the capital of Canada, though it
never has been so exclusively, and, as it seems, never is to be so
as regards authority, government, and official name.  In such
matters, authority and government often say one thing while
commerce says another; but commerce always has the best of it and
wins the game, whatever government may decree.  Albany, in this
way, is the capital of the State of New York, as authorized by the
State government; but New York has made herself the capital of
America, and will remain so.  So also Montreal has made herself the
capital of Canada.  The Grand Trunk Railway runs from Portland to
Montreal; but there is a branch from Richmond, a township within
the limits of Canada, to Quebec; so that travelers to Quebec, as we
were, are not obliged to reach that place via Montreal.

Quebec is the present seat of Canadian government, its turn for
that honor having come round some two years ago; but it is about to
be deserted in favor of Ottawa, a town which is, in fact, still to
be built on the river of that name.  The public edifices are,
however, in a state of forwardness; and if all goes well, the
Governor, the two Councils, and the House of Representatives will
be there before two years are over, whether there be any town to
receive them or no.  Who can think of Ottawa without bidding his
brothers to row, and reminding them that the stream runs fast, that
the rapids are near and the daylight past?  I asked, as a matter of
course, whether Quebec was much disgusted at the proposed change,
and I was told that the feeling was not now very strong.  Had it
been determined to make Montreal the permanent seat of government,
Quebec and Toronto would both have been up in arms.

I must confess that, in going from the States into Canada, an
Englishman is struck by the feeling that he is going from a richer
country into one that is poorer, and from a greater country into
one that is less.  An Englishman going from a foreign land into a
land which is in one sense his own, of course finds much in the
change to gratify him.  He is able to speak as the master, instead
of speaking as the visitor.  His tongue becomes more free, and he
is able to fall back to his national habits and national
expressions.  He no longer feels that he is admitted on sufferance,
or that he must be careful to respect laws which he does not quite
understand.  This feeling was naturally strong in an Englishman in
passing from the States into Canada at the time of my visit.
English policy, at that moment, was violently abused by Americans,
and was upheld as violently in Canada.  But nevertheless, with all
this, I could not enter Canada without seeing, and hearing, and
feeling that there was less of enterprise around me there than in
the States, less of general movement, and less of commercial
success.  To say why this is so would require a long and very
difficult discussion, and one which I am not prepared to hold.  It
may be that a dependent country, let the feeling of dependence be
ever so much modified by powers of self-governance, cannot hold its
own against countries which are in all respects their own masters.
Few, I believe, would now maintain that the Northern States of
America would have risen in commerce as they have risen, had they
still remained attached to England as colonies.  If this be so,
that privilege of self-rule which they have acquired has been the
cause of their success.  It does not follow as a consequence that
the Canadas, fighting their battle alone in the world, could do as
the States have done.  Climate, or size, or geographical position
might stand in their way.  But I fear that it does follow, if not
as a logical conclusion, at least as a natural result, that they
never will do so well unless some day they shall so fight their
battle.  It may be argued that Canada has in fact the power of
self-governance; that she rules herself and makes her own laws as
England does; that the Sovereign of England has but a veto on those
laws, and stands in regard to Canada exactly as she does in regard
to England.  This is so, I believe, by the letter of the
Constitution, but is not so in reality, and cannot in truth be so
in any colony even of Great Britain.  In England the political
power of the Crown is nothing.  The Crown has no such power, and
now-a-days makes no attempt at having any.  But the political power
of the Crown as it is felt in Canada is everything.  The Crown has
no such power in England, because it must change its ministers
whenever called upon to do so by the House of Commons.  But the
Colonial Minister in Downing Street is the Crown's Prime Minister
as regards the colonies, and he is changed not as any colonial
House of Assembly may wish, but in accordance with the will of the
British Commons.  Both the houses in Canada---that, namely, of the
Representatives, or Lower Houses and of the Legislative Council, or
Upper House---are now elective, and are filled without direct
influence from the Crown.  The power of self-government is as
thoroughly developed as perhaps may be possible in a colony.  But,
after all, it is a dependent form of government, and as such may
perhaps not conduce to so thorough a development of the resources
of the country as might be achieve under a ruling power of its own,
to which the welfare of Canada itself would be the chief if not the
only object.

I beg that it may not be considered from this that I would propose
to Canada to set up for itself at once and declare itself
independent.  In the first place I do not wish to throw over
Canada; and in the next place I do not wish to throw over England.
If such a separation shall ever take place, I trust that it may be
caused, not by Canadian violence, but by British generosity.  Such
a separation, however, never can be good till Canada herself shall
wish it.  That she does not wish it yet, is certain.  If Canada
ever should wish it, and should ever press for the accomplishment
of such a wish, she must do so in connection with Nova Scotia and
New Brunswick.  If at any future time there be formed such a
separate political power, it must include the whole of British
North America.

In the mean time, I return to my assertion, that in entering Canada
from the States one clearly comes from a richer to a poorer
country.  When I have said so, I have heard no Canadian absolutely
deny it; though in refraining from denying it, they have usually
expressed a general conviction, that in settling himself for life
it is better for a man to set up his staff in Canada than in the
States.  ``I do not know that we are richer,'' a Canadian says, ``but
on the whole we are doing better and are happier.''  Now, I regard
the golden rules against the love of gold, the ``aurum irrepertum et
sic melius situm,'' and the rest of it, as very excellent when
applied to individuals.  Such teaching has not much effect,
perhaps, in inducing men to abstain from wealth; but such effect as
it may have will be good.  Men and women do, I suppose, learn to be
happier when they learn to disregard riches.  But such a doctrine
is absolutely false as regards a nation.  National wealth produces
education and progress, and through them produces plenty of food,
good morals, and all else that is good.  It produces luxury also,
and certain evils attendant on luxury.  But I think it may be
clearly shown, and that it is universally acknowledged, that
national wealth produces individual well-being.  If this be so, the
argument of my friend the Canadian is naught.

To the feeling of a refined gentleman, or of a lady whose eye loves
to rest always on the beautiful, an agricultural population that
touches its hat, eats plain victuals, and goes to church, is more
picturesque and delightful than the thronged crowd of a great city,
by which a lady and gentleman is hustled without remorse, which
never touches its hat, and perhaps also never goes to church.  And
as we are always tempted to approve of that which we like, and to
think that that which is good to us is good altogether, we---the
refined gentlemen and ladies of England I mean---are very apt to
prefer the hat touchers to those who are not hat touchers.  In
doing so we intend, and wish, and strive to be philanthropical.  We
argue to ourselves that the dear excellent lower classes receive an
immense amount of consoling happiness from that ceremony of hat
touching, and quite pity those who, unfortunately for themselves,
know nothing about it.  I would ask any such lady or gentleman
whether he or she does not feel a certain amount of commiseration
for the rudeness of the town-bred artisan who walks about with his
hands in his pockets as though he recognized a superior in no one?

But that which is good and pleasant to us is often not good and
pleasant altogether.  Every man's chief object is himself; and the
philanthropist should endeavor to regard this question, not from
his own point of view, but from that which would be taken by the
individuals for whose happiness he is anxious.  The honest, happy
rustic makes a very pretty picture; and I hope that honest rustics
are happy.  But the man who earns two shillings a day in the
country would always prefer to earn five in the town.  The man who
finds himself bound to touch his hat to the squire would be glad to
dispense with that ceremony, if circumstances would permit.  A
crowd of greasy-coated town artisans, with grimy hands and pale
faces, is not in itself delectable; but each of that crowd has
probably more of the goods of life than any rural laborer.  He
thinks more, reads more, feels more, sees more, hears more, learns
more, and lives more.  It is through great cities that the
civilization of the world has progressed, and the charms of life
been advanced.  Man in his rudest state begins in the country, and
in his most finished state may retire there.  But the battle of the
world has to be fought in the cities; and the country that shows
the greatest city population is ever the one that is going most
ahead in the world's history.

If this be so, I say that the argument of my Canadian friend was
naught.  It may be that he does not desire crowded cities, with
dirty, independent artisans; that to view small farmers, living
sparingly, but with content, on the sweat of their brows, are surer
signs of a country's prosperity than hives of men and smoking
chimneys.  He has probably all the upper classes of England with
him in so thinking, and as far as I know the upper classes of all
Europe.  But the crowds themselves, the thick masses of which are
composed those populations which we count by millions, are against
him.  Up in those regions which are watered by the great lakes---%
Lake Michigan, Lake Huron, Lake Erie, Lake Ontario---and by the St.
Lawrence, the country is divided between Canada and the States.
The cities in Canada were settled long before those in the States.
Quebec and Montreal were important cities before any of the towns
belonging to the States had been founded.  But taking the
population of three of each, including the three largest Canadian
towns, we find they are as follows: In Canada, Quebec has 60,000;
Montreal, 85,000; Toronto, 55,000.  In the States, Chicago has
120,000; Detroit, 70,000; and Buffalo, 80,000.  If the population
had been equal, it would have shown a great superiority in the
progress of those belonging to the States, because the towns of
Canada had so great a start.  But the numbers are by no means
equal, showing instead a vast preponderance in favor of the States.
There can be no stronger proof that the States are advancing faster
than Canada, and in fact doing better than Canada.

Quebec is a very picturesque town; from its natural advantages
almost as much so as any town I know.  Edinburgh, perhaps, and
Innspruck may beat it.  But Quebec has very little to recommend it
beyond the beauty of its situation.  Its public buildings and works
of art do not deserve a long narrative.  It stands at the
confluence of the St. Lawrence and St. Charles Rivers; the best
part of the town is built high upon the rock---the rock which forms
the celebrated plains of Abram; and the view from thence down to
the mountains which shut in the St. Lawrence is magnificent.  The
best point of view is, I think, from the esplanade, which is
distant some five minutes' walk from the hotels.  When that has
been seen by the light of the setting sun, and seen again, if
possible, by moonlight, the most considerable lion of Quebec may be
regarded as ``done,'' and may be ticked off from the list.

The most considerable lion, according to my taste.  Lions which
roar merely by the force of association of ideas are not to me very
valuable beasts.  To many the rock over which Wolfe climbed to the
plains of Abram, and on the summit of which he fell in the hour of
victory, gives to Quebec its chiefest charm.  But I confess to
being somewhat dull in such matters.  I can count up Wolfe, and
realize his glory, and put my hand as it were upon his monument, in
my own room at home as well as I can at Quebec.  I do not say this
boastingly or with pride, but truly acknowledging a deficiency.  I
have never cared to sit in chairs in which old kings have sat, or
to have their crowns upon my head.

Nevertheless, and as a matter of course, I went to see the rock,
and can only say, as so many have said before me, that it is very
steep.  It is not a rock which I think it would be difficult for
any ordinarily active man to climb, providing, of course, that he
was used to such work.  But Wolfe took regiments of men up there at
night, and that in face of enemies who held the summits.  One
grieves that he should have fallen there and have never tasted the
sweet cup of his own fame.  For fame is sweet, and the praise of
ones's brother men the sweetest draught which a man can drain.  But
now, and for coming ages, Wolfe's name stands higher than it
probably would have done had he lived to enjoy his reward.

But there is another very worthy lion near Quebec---the Falls,
namely, of Montmorency.  They are eight miles from the town, and
the road lies through the suburb of St. Roch, and the long,
straggling French village of Beauport.  These are in themselves
very interesting, as showing the quiet, orderly, unimpulsive manner
in which the French Canadians live.  Such is their character,
although there have been such men as Papineau, and although there
have been times in which English rule has been unpopular with the
French settlers.  As far as I could learn there is no such feeling
now.  These people are quiet, contented; and, as regards a
sufficiency of the simple staples of living, sufficiently well to
do.  They are thrifty, but they do not thrive.  They do not
advance, and push ahead, and become a bigger people from year to
year, as settlers in a new country should do.  They do not even
hold their own in comparison with those around them.  But has not
this always been the case with colonists out of France; and has it
not always been the case with Roman Catholics when they have been
forced to measure themselves against Protestants?  As to the
ultimate fate in the world of this people, one can hardly form a
speculation.  There are, as nearly as I could learn, about 800,000
of them in Lower Canada; but it seems that the wealth and
commercial enterprise of the country is passing out of their hands.
Montreal, and even Quebec, are, I think, becoming less and less
French every day; but in the villages and on the small farms the
French still remain, keeping up their language, their habits, and
their religion.  In the cities they are becoming hewers of wood and
drawers of water.  I am inclined to think that the same will
ultimately be their fate in the country.  Surely one may declare as
a fact that a Roman Catholic population can never hold its ground
against one that is Protestant.  I do not speak of numbers; for the
Roman Catholics will increase and multiply, and stick by their
religion, although their religion entails poverty and dependence,
as they have done and still do in Ireland.  But in progress and
wealth the Romanists have always gone to the wall when the two have
been made to compete together.  And yet I love their religion.
There is something beautiful, and almost divine, in the faith and
obedience of a true son of the Holy Mother.  I sometimes fancy that
I would fain be a Roman Catholic---if I could; as also I would often
wish to be still a child---if that were possible.

All this is on the way to the Falls of Montmorency.  These falls
are placed exactly at the mouth of the little river of the same
name, so that it may be said absolutely to fall into the St.
Lawrence.  The people of the country, however, declare that the
river into which the waters of the Montmorency fall is not the St.
Lawrence, but the Charles.  Without a map I do not know that I can
explain this.  The River Charles appears to, and in fact does, run
into the St. Lawrence just below Quebec.  But the waters do not
mix.  The thicker, browner stream of the lesser river still keeps
the northeastern bank till it comes to the Island of Orleans, which
lies in the river five or six miles below Quebec.  Here or
hereabouts are the Falls of the Montmorency, and then the great
river is divided for twenty-five miles by the Isle of Orleans.  It
is said that the waters of the Charles and the St. Lawrence do not
mix till they meet each other at the foot of this island.

I do not know that I am particularly happy at describing a
waterfall, and what little capacity I may have in this way I would
wish to keep for Niagara.  One thing I can say very positively
about Montmorency, and one piece of advice I can give to those who
visit the falls.  The place from which to see them is not the
horrible little wooden temple which has been built immediately over
them on that side which lies nearest to Quebec.  The stranger is
put down at a gate through which a path leads to this temple, and
at which a woman demands from him twenty-five cents for the
privilege of entrance.  Let him by all means pay the twenty-five
cents.  Why should he attempt to see the falls for nothing, seeing
that this woman has a vested interest in the showing of them?  I
declare that if I thought that I should hinder this woman from her
perquisites by what I write, I would leave it unwritten, and let my
readers pursue their course to the temple---to their manifest
injury.  But they will pay the twenty-five cents.  Then let them
cross over the bridge, eschewing the temple, and wander round on
the open field till they get the view of the falls, and the view of
Quebec also, from the other side.  It is worth the twenty-five
cents and the hire of the carriage also.  Immediately over the
falls there was a suspension bridge, of which the supporting, or
rather non-supporting, pillars are still to be seen.  But the
bridge fell down, one day, into the river; and---alas! alas!---with
the bridge fell down an old woman, and a boy, and a cart---a cart
and horse---and all found a watery grave together in the spray.  No
attempt has been made since that to renew the suspension bridge;
but the present wooden bridge has been built higher up in lieu of
it.

Strangers naturally visit Quebec in summer or autumn, seeing that a
Canada winter is a season with which a man cannot trifle; but I
imagine that the mid-winter is the best time for seeing the Falls
of Montmorency.  The water in its fall is dashed into spray, and
that spray becomes frozen, till a cone of ice is formed immediately
under the cataract, which gradually rises till the temporary
glacier reaches nearly half way to the level of the higher river.
Up this men climb---and ladies also, I am told---and then descend,
with pleasant rapidity, on sledges of wood, sometimes not without
an innocent tumble in the descent.  As we were at Quebec in
September, we did not experience the delights of this pastime.

As I was too early for the ice cone under the Montmorency Falls, so
also was I too late to visit the Saguenay River, which runs into
the St. Lawrence some hundred miles below Quebec.  I presume that
the scenery of the Saguenay is the finest in Canada.  During the
summer steamers run down the St. Lawrence and up the Saguenay, but
I was too late for them.  An offer was made to us through the
kindness of Sir Edmund Head, who was then the Governor-General, of
the use of a steam-tug belonging to a gentleman who carries on a
large commercial enterprise at Chicoutimi, far up the Saguenay; but
an acceptance of this offer would have entailed some delay at
Quebec, and, as we were anxious to get into the Northwestern States
before the winter commenced, we were obliged with great regret to
decline the journey.

I feel bound to say that a stranger, regarding Quebec merely as a
town, finds very much of which he cannot but complain.  The
footpaths through the streets are almost entirely of wood, as
indeed seems to be general throughout Canada.  Wood is, of course,
the cheapest material; and, though it may not be altogether good
for such a purpose, it would not create animadversion if it were
kept in tolerable order.  But in Quebec the paths are intolerably
bad.  They are full of holes.  The boards are rotten, and worn in
some places to dirt.  The nails have gone, and the broken planks go
up and down under the feet, and in the dark they are absolutely
dangerous.  But if the paths are bad, the road-ways are worse.  The
street through the lower town along the quays is, I think, the most
disgraceful thoroughfare I ever saw in any town.  I believe the
whole of it, or at any rate a great portion, has been paved with
wood; but the boards have been worked into mud, and the ground
under the boards has been worked into holes, till the street is
more like the bottom of a filthy ditch than a road-way through one
of the most thickly populated parts of a city.  Had Quebec in
Wolfe's time been as it is now, Wolfe would have stuck in the mud
between the river and the rock before he reached the point which he
desired to climb.  In the upper town the roads are not as bad as
they are below, but still they are very bad.  I was told that this
arose from disputes among the municipal corporations.  Everything
in Canada relating to roads, and a very great deal affecting the
internal government of the people, is done by these municipalities.
It is made a subject of great boast in Canada that the communal
authorities do carry on so large a part of the public business, and
that they do it generally so well and at so cheap a rate.  I have
nothing to say against this, and, as a whole, believe that the
boast is true.  I must protest, however, that the streets of the
greater cities---for Montreal is nearly as bad as Quebec---prove the
rule by a very sad exception.  The municipalities of which I speak
extend, I believe, to all Canada---the two provinces being divided
into counties, and the counties subdivided into townships, to
which, as a matter of course, the municipalities are attached.

From Quebec to Montreal there are two modes of travel.  There are
the steamers up the St. Lawrence, which, as all the world know, is,
or at any rate hitherto has been, the high-road of the Canadas; and
there is the Grand Trunk Railway.  Passengers choosing the latter
go toward Portland as far as Richmond, and there join the main line
of the road, passing from Richmond on to Montreal.  We learned
while at Quebec that it behooved us not to leave the colony till we
had seen the lake and mountains of Memphremagog; and, as we were
clearly neglecting our duty with regard to the Saguenay, we felt
bound to make such amends as lay in our power by deviating from our
way to the lake above named.  In order to do this we were obliged
to choose the railway, and to go back beyond Richmond to the
station at Sherbrooke.  Sherbrooke is a large village on the
confines of Canada, and, as it is on the railway, will no doubt
become a large town.  It is very prettily situated on the meeting
of two rivers; it has three or four churches, and intends to
thrive.  It possesses two newspapers, of the prosperity of which I
should be inclined to feel less assured.  The annual subscription
to such a newspaper, published twice a week, is ten shillings.  A
sale of a thousand copies is not considered bad.  Such a sale would
produce 500 pounds a year; and this would, if entirely devoted to
that purpose, give a moderate income to a gentleman qualified to
conduct a newspaper.  But the paper and printing must cost
something, and the capital invested should receive its proper
remuneration.  And then---such at least is the general idea---the
getting together of news and the framing of intelligence is a
costly operation.  I can only hope that all this is paid for by the
advertisements, for I must trust that the editors do not receive
less than the moderate sum above named.  At Sherbrooke we are still
in Lower Canada.  Indeed, as regards distance, we are when there
nearly as far removed from Upper Canada as at Quebec.  But the race
of people here is very different.  The French population had made
their way down into these townships before the English and American
war broke out, but had not done so in great numbers.  The country
was then very unapproachable, being far to the south of the St.
Lawrence, and far also from-any great line of internal
communication toward the Atlantic.  But, nevertheless, many
settlers made their way in here from the States---men who preferred
to live under British rule, and perhaps doubted the stability of
the new order of things.  They or their children have remained here
since; and, as the whole country has been opened up by the railway,
many others have flocked in.  Thus a better class of people than
the French hold possession of the larger farms, and are on the
whole doing well.  I am told that many Americans are now coming
here, driven over the borders from Maine, New Hampshire, and
Vermont by fears of the war and the weight of taxation.  I do not
think that fears of war or the paying of taxes drive many
individuals away from home.  Men who would be so influenced have
not the amount of foresight which would induce them to avoid such
evils; or, at any rate, such fears would act slowly.  Laborers,
however, will go where work is certain, where work is well paid,
and where the wages to be earned will give plenty in return.  It
may be that work will become scarce in the States, as it has done
with those poor jewelers at Attleborough of whom we spoke, and that
food will become dear.  If this be so, laborers from the States
will no doubt find their way into Canada.

From Sherbrooke we went with the mails on a pair-horse wagon to
Magog.  Cross-country mails are not interesting to the generality
of readers, but I have a professional liking for them myself.  I
have spent the best part of my life in looking after, and I hope in
improving, such mails; and I always endeavor to do a stroke of work
when I come across them.  I learned on this occasion that the
conveyance of mails with a pair of horses, in Canada, costs little
more than half what is paid for the same work in England with one
horse, and something less than what is paid in Ireland, also for
one horse.  But in Canada the average pace is only five miles an
hour.  In Ireland it is seven, and the time is accurately kept,
which does not seem to be the case in Canada.  In England the pace
is eight miles an hour.  In Canada and in Ireland these conveyances
carry passengers; but in England they are prohibited from doing so.
In Canada the vehicles are much better got up than they are in
England, and the horses too look better.  Taking Ireland as a
whole, they are more respectable in appearance there than in
England.  From all which it appears that pace is the article that
costs the highest price, and that appearance does not go for much
in the bill.  In Canada the roads are very bad in comparison with
the English or Irish roads; but, to make up for this, the price of
forage is very low.

I have said that the cross-mail conveyances in Canada did not seem
to be very closely bound as to time; but they are regulated by
clock-work in comparison with some of them in the United States.
``Are you going this morning?'' I said to a mail-driver in Vermont.
``I thought you always started in the evening.''  ``Wa'll, I guess I
do; but it rained some last night, so I jist stayed at home.''  I do
not know that I ever felt more shocked in my life, and I could
hardly keep my tongue off the man.  The mails, however, would have
paid no respect to me in Vermont, and I was obliged to walk away
crest-fallen.

We went with the mails from Sherbrooke to a village called Magog,
at the outlet of the lake, and from thence by a steamer up the
lake, to a solitary hotel called the Mountain House, which is built
at the foot of the mountain, on the shore, and which is surrounded
on every side by thick forest.  There is no road within two miles
of the house.  The lake therefore is the only highway, and that is
frozen up for four months in the year.  When frozen, however, it is
still a road, for it is passable for sledges.  I have seldom been
in a house that seemed so remote from the world, and so little
within reach of doctors, parsons, or butchers.  Bakers in this
country are not required, as all persons make their own bread.  But
in spite of its position the hotel is well kept, and on the whole
we were more comfortable there than at any other inn in Lower
Canada.  The Mountain house is but five miles from the borders of
Vermont, in which State the head of the lake lies.  The steamer
which brought us runs on to Newport, or rather from Newport to
Magog and back again.  And Newport is in Vermont.

The one thing to be done at the Mountain House is the ascent of the
mountain called the Owl's head.  The world there offers nothing
else of active enterprise to the traveler, unless fishing be
considered an active enterprise.  I am not capable of fishing,
therefore we resolved on going up the Owl's Head.  To dine in the
middle of the day is absolutely imperative at these hotels, and
thus we were driven to select either the morning or the afternoon.
Evening lights we declared were the best for all views, and
therefore we decided on the afternoon.  It is but two miles; but
then, as we were told more than once by those who had spoken to us
on the subject, those two miles are not like other miles.  ``I doubt
if the lady can do it,'' one man said to me.  I asked if ladies did
not sometimes go up.  ``Yes; young women do, at times,'' he said.
After that my wife resolved that she would see the top of the Owl's
Head, or die in the attempt, and so we started.  They never think
of sending a guide with one in these places, whereas in Europe a
traveler is not allowed to go a step without one.  When I asked for
one to show us the way up Mount Washington, I was told that there
were no idle boys about that place.  The path was indicated to us,
and off we started with high hopes.

I have been up many mountains, and have climbed some that were
perhaps somewhat dangerous in their ascent.  In climbing the Owl's
Head there is no danger.  One is closed in by thick trees the whole
way.  But I doubt if I ever went up a steeper ascent.  It was very
hard work, but we were not beaten.  We reached the top, and there
sitting down, thoroughly enjoyed our victory.  It was then half-
past five o'clock, and the sun was not yet absolutely sinking.  It
did not seem to give us any warning that we should especially
require its aid, and, as the prospect below us was very lovely, we
remained there for a quarter of an hour.  The ascent of the Owl's
Head is certainly a thing to do, and I still think, in spite of our
following misfortune, that it is a thing to do late in the
afternoon.  The view down upon the lakes and the forests around,
and on the wooded hills below, is wonderfully lovely.  I never was
on a mountain which gave me a more perfect command of all the
country round.  But as we arose to descend we saw a little cloud
coming toward us from over Newport.

The little cloud came on with speed, and we had hardly freed
ourselves from the rocks of the summit before we were surrounded by
rain.  As the rain became thicker, we were surrounded by darkness
also, or, if not by darkness, by so dim a light that it became a
task to find our path.  I still thought that the daylight had not
gone, and that as we descended, and so escaped from the cloud, we
should find light enough to guide us.  But it was not so.  The rain
soon became a matter of indifference, and so also did the mud and
briers beneath our feet.  Even the steepness of the way was almost
forgotten as we endeavored to thread our path through the forest
before it should become impossible to discern the track.  A dog had
followed us up, and though the beast would not stay with us so as
to be our guide, he returned ever and anon, and made us aware of
his presence by dashing by us.  I may confess now that I became
much frightened.  We were wet through, and a night out in the
forest would have been unpleasant to us.  At last I did utterly
lose the track, it had become quite dark, so dark that we could
hardly see each other.  We had succeeded in getting down the
steepest and worst part of the mountain, but we were still among
dense forest trees, and up to our knees in mud.  But the people at
the Mountain house were Christians, and men with lanterns were sent
hallooing after us through the dark night.  When we were thus found
we were not many yards from the path, but unfortunately on the
wrong side of a stream.  Through that we waded, and then made our
way in safety to the inn.  In spite of which misadventure I advise
all travelers in Lower Canada to go up the Owl's Head.

On the following day we crossed the lake to Georgeville, and drove
around another lake called the Massawhippi back to Sherbrooke.
This was all very well, for it showed us a part of the country
which is comparatively well tilled, and has been long settled; but
the Massawhippi itself is not worth a visit.  The route by which we
returned occupies a longer time than the other, and is more costly,
as it must be made in a hired vehicle.  The people here are quiet,
orderly, and I should say a little slow.  It is manifest that a
strong feeling against the Northern States has lately sprung up.
This is much to be deprecated, but I cannot but say that it is
natural.  It is not that the Canadians have any special secession
feelings, or that they have entered with peculiar warmth into the
questions of American politics; but they have been vexed and
acerbated by the braggadocio of the Northern States.  They
constantly hear that they are to be invaded, and translated into
citizens of the Union; that British rule is to be swept off the
continent, and that the star-spangled banner is to be waved over
them in pity.  The star-spangled banner is in fact a fine flag, and
has waved to some purpose; but those who live near it, and not
under it, fancy that they hear too much of it.  At the present
moment the loyalty of both the Canadas to Great Britain is beyond
all question.  From all that I can hear, I doubt whether this
feeling in the provinces was ever so strong, and under such
circumstances American abuse of England and American braggadocio is
more than usually distasteful.  All this abuse and all this
braggadocio come to Canada from the Northern States, and therefore
the Southern cause is at the present moment the more popular with
them.

I have said that the Canadians hereabouts are somewhat slow.  As we
were driving back to Sherbrooke it became necessary that we should
rest for an hour or so in the middle of the day, and for this
purpose we stopped at a village inn.  It was a large house, in
which there appeared to be three public sitting-rooms of ample
size, one of which was occupied as the bar.  In this there were
congregated some six or seven men, seated in arm-chairs round a
stove, and among these I placed myself.  No one spoke a word either
to me or to any one else.  No one smoked, and no one read, nor did
they even whittle sticks.  I asked a question, first of one and
then of another, and was answered with monosyllables.  So I gave up
any hope in that direction, and sat staring at the big stove in the
middle of the room, as the others did.  Presently another stranger
entered, having arrived in a wagon, as I had done.  He entered the
room and sat down, addressing no one, and addressed by no one.
After awhile, however, he spoke.  ``Will there be any chance of
dinner here?'' he said.  ``I guess there'll be dinner by-and-by,''
answered the landlord, and then there was silence for another ten
minutes, during which the stranger stared at the stove.  ``Is that
dinner any way ready?'' he asked again.  ``I guess it is,'' said the
landlord.  And then the stranger went out to see after his dinner
himself.  When we started, at the end of an hour, nobody said
anything to us.  The driver ``hitched'' on the horses, as they call
it, and we started on our way, having been charged nothing for our
accommodation.  That some profit arose from the horse provender is
to be hoped.

On the following day we reached Montreal, which, as I have said
before, is the commercial capital of the two Provinces.  This
question of the capitals is at the present moment a subject of
great interest in Canada; but, as I shall be driven to say
something on the matter when I report myself as being at Ottawa, I
will refrain now.  There are two special public affairs at the
present moment to interest a traveler in Canada.  The first I have
named, and the second is the Grand Trunk Railway.  I have already
stated what is the course of this line.  It runs from the Western
State of Michigan to Portland, on the Atlantic, in the State of
Maine, sweeping the whole length of Canada in its route.  It was
originally made by three companies.  The Atlantic and St. Lawrence
constructed it from Portland to Island Pond, on the borders of the
States.  The St. Lawrence and Atlantic took it from the
southeastern side of the river at Montreal to the same point, viz.,
Island Pond.  And the Grand Trunk Company have made it from Detroit
to Montreal, crossing the river there with a stupendous tubular
bridge, and have also made the branch connecting the main line with
Quebec and Riviere du Loup.  This latter company is now
incorporated with the St. Lawrence and Atlantic, but has only
leased the portion of the line running through the States.  This
they have done, guaranteeing the shareholders an interest of six
per cent.  There never was a grander enterprise set on foot.  I
will not say there never was one more unfortunate, for is there not
the Great Eastern, which, by the weight and constancy of its
failures, demands for itself a proud pre-eminence of misfortune?
But surely the Grand Trunk comes next to it.  I presume it to be
quite out of the question that the shareholders should get any
interest whatever on their shares for years.  The company, when I
was at Montreal, had not paid the interest due to the Atlantic and
St. Lawrence Company for the last year, and there was a doubt
whether the lease would not be broken.  No party that had advanced
money to the undertaking was able to recover what had been
advanced.  I believe that one firm in London had lent nearly a
million to the company, and is now willing to accept half the sum
so lent in quittance of the whole debt.  In 1860 the line could not
carry the freight that offered, not having or being able to obtain
the necessary rolling stock; and on all sides I heard men
discussing whether the line would be kept open for traffic.  The
government of Canada advanced to the company three millions of
money, with an understanding that neither interest nor principal
should be demanded till all other debts were paid and all
shareholders in receipt of six per cent.  interest.  But the three
millions were clogged with conditions which, though they have been
of service to the country, have been so expensive to the company
that it is hardly more solvent with it than it would have been
without it.  As it is, the whole property seems to be involved in
ruin; and yet the line is one of the grandest commercial
conceptions that was ever carried out on the face of the globe, and
in the process of a few years will do more to make bread cheap in
England than any other single enterprise that exists.

I do not know that blame is to be attached to any one.  I at least
attach no such blame.  Probably it might be easy now to show that
the road might have been made with sufficient accommodation for
ordinary purposes without some of the more costly details.  The
great tubular bridge, on which was expended 1,300,000 pounds,
might, I should think, have been dispensed with.  The Detroit end
of the line might have been left for later time.  As it stands now,
however, it is a wonderful operation carried to a successful issue
as far as the public are concerned; and one can on]y grieve that it
should be so absolute a failure to those who have placed their
money in it.  There are schemes which seem to be too big for men to
work out with any ordinary regard to profit and loss.  The Great
Eastern is one, and this is another.  The national advantage
arising from such enterprises is immense; but the wonder is that
men should be found willing to embark their money where the risk is
so great and the return even hoped for is so small.

While I was in Canada some gentlemen were there from the Lower
Provinces---Nova Scotia, that is, and New Brunswick---agitating the
subject of another great line of railway, from Quebec to Halifax.
The project is one in favor of which very much may be said.  In a
national point of view an Englishman or a Canadian cannot but
regret that there should be no winter mode of exit from, or
entrance to, Canada, except through the United States.  The St.
Lawrence is blocked up for four or five months in winter, and the
steamers which run to Quebec in the summer run to Portland during
the season of ice.  There is at present no mode of public
conveyance between the Canadas and the Lower Provinces; and an
immense district of country on the borders of Lower Canada, through
New Brunswick, and into Nova Scotia, is now absolutely closed
against civilization, which by such a railway would be opened up to
the light of day.  We all know how much the want of such a road was
felt when our troops were being forwarded to Canada during the last
winter.  It was necessary they should reach their destination
without delay; and as the river was closed, and the passing of
troops through the States was of course out of the question, that
long overland journey across Nova Scotia and New Brunswick became a
necessity.  It would certainly be a very great thing for British
interests if a direct line could be made from such a port as
Halifax, a port which is open throughout the whole year, up into
the Canadas.  If these colonies belonged to France or to any other
despotic government, the thing would be done.  But the colonies do
not belong to any despotic government.

Such a line would, in fact, be a continuance of the Grand Trunk;
and who that looks at the present state of the finances of the
Grand Trunk can think it to be on the cards that private enterprise
should come forward with more money---with more millions?  The idea
is that England will advance the money, and that the English House
of Commons will guarantee the interest, with some counter-guarantee
from the colonies that this interest shall be duly paid.  But it
would seem that, if such colonial guarantee is to go for anything,
the colonies might raise the money in the money market without the
intervention of the British House of Commons.

Montreal is an exceedingly good commercial town, and business there
is brisk.  It has now 85,000 inhabitants.  Having said that of it,
I do not know what more there is left to say.  Yes; one word there
is to say of Sir William Logan, the creator of the Geological
Museum there, and the head of all matters geological throughout the
province.  While he was explaining to me with admirable perspicuity
the result of investigations into which he had poured his whole
heart, I stood by, understanding almost nothing, but envying
everything.  That I understood almost nothing, I know he perceived.
That, ever and anon, with all his graciousness, became apparent.
But I wonder whether he perceived also that I did envy everything.
I have listened to geologists by the hour before---have had to
listen to them, desirous simply of escape.  I have listened, and
understood absolutely nothing, and have only wished myself away.
But I could have listened to Sir William Logan for the whole day,
if time allowed.  I found, even in that hour, that some ideas found
their way through to me, and I began to fancy that even I could
become a geologist at Montreal.

Over and beyond Sir William Logan, there is at Montreal for
strangers the drive round the mountain, not very exciting, and
there is the tubular bridge over the St. Lawrence.  This, it must
be understood, is not made in one tube, as is that over the Menai
Straits, but is divided into, I think, thirteen tubes.  To the eye
there appear to be twenty-five tubes; but each of the six side
tubes is supported by a pier in the middle.  A great part of the
expense of the bridge was incurred in sinking the shafts for these
piers.



\chapter{Upper Canada}


Ottawa is in Upper Canada, but crossing the suspension bridge from
Ottawa into Hull, the traveler is in Lower Canada.  It is therefore
exactly in the confines, and has been chosen as the site of the new
government capital very much for this reason.  Other reasons have
no doubt had a share in the decision.  At the time when the choice
was made Ottawa was not large enough to create the jealousy of the
more populous towns.  Though not on the main line of railway, it
was connected with it by a branch railway, and it is also connected
with the St. Lawrence by water communication.  And then it stands
nobly on a magnificent river, with high, overhanging rock, and a
natural grandeur of position which has perhaps gone far in
recommending it to those whose voice in the matter has been
potential.  Having the world of Canada from whence to choose the
site of a new town, the choosers have certainly chosen well.  It is
another question whether or no a new town should have been deemed
necessary.

Perhaps it may be well to explain the circumstances under which it
was thought expedient thus to establish a new Canadian capital.  In
1841, when Lord Sydenham was Governor-General of the provinces, the
two Canadas, separate till then, were united under one government.
At that time the people of Lower or French Canada, and the people
of Upper or English Canada, differed much more in their habits and
language than they do now.  I do not know that the English have
become in any way Gallicized, but the French have been very
materially Anglicized.  But while this has been in progress
national jealousy has been at work, and even yet that national
jealousy is not at an end.  While the two provinces were divided
there were, of course, two capitals, and two seats of government.
These were at Quebec for Lower Canada, and at Toronto for Upper
Canada, both which towns are centrically situated as regards the
respective provinces.  When the union was effected, it was deemed
expedient that there should be but one capital; and the small town
of Kingstown was selected, which is situated on the lower end of
Lake Ontario, in the upper province.  But Kingstown was found to be
inconvenient, lacking space and accommodation for those who had to
follow the government, and the Governor removed it and himself to
Montreal.  Montreal is in the lower province, but is very central
to both the provinces; and it is moreover the chief town in Canada.
This would have done very well but for an unforeseen misfortune.

It will be remembered by most readers that in 1837 took place the
Mackenzie-Papineau rebellion, of which those who were then old
enough to be politicians heard so much in England.  I am not going
back to recount the history of the period, otherwise than to say
that the English Canadians at that time, in withstanding and
combating the rebels, did considerable injury to the property of
certain French Canadians, and that, when the rebellion had blown
over and those in fault had been pardoned, a question arose whether
or no the government should make good the losses of those French
Canadians who had been injured.  The English Canadians protested
that it would be monstrous that they should be taxed to repair
damages suffered by rebels, and made necessary in the suppression
of rebellion.  The French Canadians declared that the rebellion had
been only a just assertion of their rights; that if there had been
crime on the part of those who took up arms, that crime had been
condoned, and that the damages had not fallen exclusively or even
chiefly on those who had done so.  I will give no opinion on the
merits of the question, but simply say that blood ran very hot when
it was discussed.  At last the Houses of the Provincial Parliament,
then assembled at Montreal, decreed that the losses should be made
good by the public treasury; and the English mob in Montreal, when
this decree became known, was roused to great wrath by a decision
which seemed to be condemnatory of English loyalty.  It pelted Lord
Elgin, the Governor-General, with rotten eggs, and burned down the
Parliament house.  Hence there arose, not unnaturally, a strong
feeling of anger on the part of the local government against
Montreal; and moreover there was no longer a house in which the
Parliament could be held in that town.  For these conjoint reasons
it was decided to move the seat of government again, and it was
resolved that the Governor and the Parliament should sit
alternately at Toronto in Upper Canada, and at Quebec in Lower
Canada, remaining four years at each place.  They went at first to
Toronto for two years only, having agreed that they should be there
on this occasion only for the remainder of the term of the then
Parliament.  After that they were at Quebec for four years; then at
Toronto for four; and now again are at Quebec.  But this
arrangement has been found very inconvenient.  In the first place
there is a great national expenditure incurred in moving old
records and in keeping double records, in moving the library, and,
as I have been informed, even the pictures.  The government clerks
also are called on to move as the government moves; and though an
allowance is made to them from the national purse to cover their
loss, the arrangement has nevertheless been felt by them to be a
grievance, as may be well understood.  The accommodation also for
the ministers of the government and for members of the two Houses
has been insufficient.  Hotels, lodgings, and furnished houses
could not be provided to the extent required, seeing that they
would be left nearly empty for every alternate space of four years.
Indeed, it needs but little argument to prove that the plan adopted
must have been a thoroughly uncomfortable plan, and the wonder is
that it should have been adopted.  Lower Canada had undertaken to
make all her leading citizens wretched, providing Upper Canada
would treat hers with equal severity.  This has now gone on for
some twelve years, and as the system was found to be an unendurable
nuisance, it has been at last admitted that some steps must be
taken toward selecting one capital for the country.

I should here, in justice to the Canadians, state a remark made to
me on this matter by one of the present leading politicians of the
colony.  I cannot think that the migratory scheme was good but he
defended it, asserting that it had done very much to amalgamate the
people of the two provinces; that it had brought Lower Canadians
into Upper Canada, and Upper Canadians into Lower Canada, teaching
English to those who spoke only French before, and making each
pleasantly acquainted with the other.  I have no doubt that
something---perhaps much---has been done in this way; but valuable as
the result may have been, I cannot think it worth the cost of the
means employed.  The best answer to the above argument consists in
the undoubted fact that a migratory government would never have
been established for such a reason.  It was so established because
Montreal, the central town, had given offense, and because the
jealousy of the provinces against each other would not admit of the
government being placed entirely at Quebec, or entirely at Toronto.

But it was necessary that some step should be taken; and as it was
found to be unlikely that any resolution should be reached by the
joint provinces themselves, it was loyally and wisely determined to
refer the matter to the Queen.  That Her Majesty has
constitutionally the power to call the Parliament of Canada at any
town of Canada which she may select, admits, I conceive, of no
doubt.  It is, I imagine, within her prerogative to call the
Parliament of England where she may please within that realm,
though her lieges would be somewhat startled if it were called
otherwhere than in London.  It was therefore well done to ask Her
Majesty to act as arbiter in the matter.  But there are not wanting
those in Canada who say that in referring the matter to the Queen
it was in truth referring it to those by whom very many of the
Canadians were least willing to be guided in the matter; to the
Governor-General namely, and the Colonial Secretary.  Many indeed
in Canada now declare that the decision simply placed the matter in
the hands of the Governor-General.

Be that as it may, I do not think that any unbiased traveler will
doubt that the best possible selection has been made, presuming
always, as we may presume in the discussion, that Montreal could
not be selected.  I take for granted that the rejection of Montreal
was regarded as a sine qua non in the decision.  To me it appears
grievous that this should have been so.  It is a great thing for
any country to have a large, leading, world-known city, and I think
that the government should combine with the commerce of the country
in carrying out this object.  But commerce can do a great deal more
for government than government can do for commerce.  Government has
selected Ottawa as the capital of Canada; but commerce has already
made Montreal the capital, and Montreal will be the chief city of
Canada, let government do what it may to foster the other town.
The idea of spiting a town because there has been a row in it seems
to me to be preposterous.  The row was not the work of those who
have made Montreal rich and respectable.  Montreal is more
centrical than Ottawa---nay, it is as nearly centrical as any town
can be.  It is easier to get to Montreal from Toronto than to
Ottawa; and if from Toronto, then from all that distant portion of
Upper Canada back of Toronto.  To all Lower Canada Montreal is, as
a matter of course, much easier of access than Ottawa.  But having
said so much in favor of Montreal, I will again admit that, putting
aside Montreal, the best possible selection has been made.

When Ottawa was named, no time was lost in setting to work to
prepare for the new migration.  In 1859 the Parliament was removed
to Quebec, with the understanding that it should remain there till
the new buildings should be completed.  These buildings were
absolutely commenced in April, 1860, and it was, and I believe
still is, expected that they will be completed in 1863.  I am now
writing in the winter of 1861; and, as is necessary in Canadian
winters, the works are suspended.  But unfortunately they were
suspended in the early part of October---on the first of October---%
whereas they might have been continued, as far as the season is
concerned, up to the end of November.  We reached Ottawa on the
third of October, and more than a thousand men had then been just
dismissed.  All the money in hand had been expended, and the
government---so it was said---could give no more money till
Parliament should meet again.  This was most unfortunate.  In the
first place the suspension was against the contract as made with
the contractors for the building; in the next place there was the
delay; and then, worst of all, the question again became agitated
whether the colonial legislature were really in earnest with
reference to Ottawa.  Many men of mark in the colony were still
anxious---I believe are still anxious---to put an end to the Ottawa
scheme, and think that there still exists for them a chance of
success.  And very many men who are not of mark are thus united,
and a feeling of doubt on the subject has been created.  Two
hundred and twenty-five thousand pounds have already been spent on
these buildings, and I have no doubt myself that they will be duly
completed and duly used.

We went up to the new town by boat, taking the course of the River
Ottawa.  We passed St. Ann's, but no one at St. Ann's seemed to
know anything of the brothers who were to rest there on their weary
oars.  At Maxwellstown I could hear nothing of Annie Laurie or of
her trysting-place on the braes; and the turnpike man at Tara could
tell me nothing of the site of the hall, and had never even heard
of the harp.  When I go down South, I shall expect to find that the
negro melodies have not yet reached ``Old Virginie.''  This boat
conveyance from Montreal to Ottawa is not all that could be wished
in convenience, for it is allied too closely with railway
traveling.  Those who use it leave Montreal by a railway; after
nine miles, they are changed into a steamboat.  Then they encounter
another railway, and at last reach Ottawa in a second steamboat.
But the river is seen, and a better idea of the country is obtained
than can be had solely from the railway cars.  The scenery is by no
means grand, nor is it strikingly picturesque, but it is in its way
interesting.  For a long portion of the river the old primeval
forests come down close to the water's edge, and in the fall of the
year the brilliant coloring is very lovely.  It should not be
imagined, as I think it often is imagined, that these forests are
made up of splendid trees, or that splendid trees are even common.
When timber grows on undrained ground, and when it is uncared for,
it does not seem to approach nearer to its perfection than wheat
and grass do under similar circumstances.  Seen from a little
distance, the color and effect is good; but the trees themselves
have shallow roots, and grow up tall, narrow, and shapeless.  It
necessarily is so with all timber that is not thinned in its
growth.  When fine forest trees are found, and are left standing
alone by any cultivator who may have taste enough to wish for such
adornment, they almost invariably die.  They are robbed of the
sickly shelter by which they have been surrounded; the hot sun
strikes the uncovered fibers of the roots, and the poor, solitary
invalid languishes, and at last dies.

As one ascends the river, which by its breadth forms itself into
lakes, one is shown Indian villages clustering down upon the bank.
Some years ago these Indians were rich, for the price of furs, in
which they dealt, was high; but furs have become cheaper, and the
beavers, with which they used to trade, are almost valueless.  That
a change in the fashion of hats should have assisted to polish
these poor fellows off the face of creation, must, one may suppose,
be very unintelligible to them; but nevertheless it is probably a
subject of deep speculation.  If the reading world were to take to
sermons again and eschew their novels, Messrs. Thackeray, Dickens,
and some others would look about them and inquire into the causes
of such a change with considerable acuteness.  They might not,
perhaps, hit the truth, and these Indians are much in that
predicament.  It is said that very few pure-blooded Indians are now
to be found in their villages, but I doubt whether this is not
erroneous.  The children of the Indians are now fed upon baked
bread and on cooked meat, and are brought up in houses.  They are
nursed somewhat as the children of the white men are nursed; and
these practices no doubt have done much toward altering their
appearance.  The negroes who have been bred in the States, and
whose fathers have been so bred before them, differ both in color
and form from their brothers who have been born and nurtured in
Africa.

I said in the last chapter that the City of Ottawa was still to be
built; but I must explain, lest I should draw down on my head the
wrath of the Ottawaites, that the place already contains a
population of 15,000 inhabitants.  As, however, it is being
prepared for four times that number---for eight times that number,
let us hope---and as it straggles over a vast extent of ground, it
gives one the idea of a city in an active course of preparation.
In England we know nothing about unbuilt cities.  With us four or
five blocks of streets together never assume that ugly, unfledged
appearance which belongs to the half-finished carcass of a house,
as they do so often on the other side of the Atlantic.  Ottawa is
preparing for itself broad streets and grand thoroughfares.  The
buildings already extend over a length considerably exceeding two
miles; and half a dozen hotels have been opened, which, if I were
writing a guide-book in a complimentary tone, it would be my duty
to describe as first rate.  But the half dozen first-rate hotels,
though open, as yet enjoy but a moderate amount of custom.  All
this justifies me, I think, in saying that the city has as yet to
get itself built.  The manner in which this is being done justifies
me also in saying that the Ottawaites are going about their task
with a worthy zeal.

To me I confess that the nature of the situation has great charms,
regarding it as the site for a town.  It is not on a plain; and
from the form of the rock overhanging the river, and of the hill
that falls from thence down to the water, it has been found
impracticable to lay out the place in right-angled parallelograms.
A right-angled parallelogramical city, such as are Philadelphia and
the new portion of New York, is from its very nature odious to me.
I know that much may be said in its favor---that drainage and gas-
pipes come easier to such a shape, and that ground can be better
economized.  Nevertheless, I prefer a street that is forced to
twist itself about.  I enjoy the narrowness of Temple Bar and the
misshapen curvature of Picket Street.  The disreputable dinginess
of Hollowell Street is dear to me, and I love to thread my way up
the Olympic into Covent Garden.  Fifth Avenue in New York is as
grand as paint and glass can make it; but I would not live in a
palace in Fifth Avenue if the corporation of the city would pay my
baker's and butcher's bills.

The town of Ottawa lies between two waterfalls.  The upper one, or
Rideau Fall, is formed by the confluence of a small river with the
larger one; and the lower fall---designated as lower because it is
at the foot of the hill, though it is higher up the Ottawa River---%
is called the Chaudiere, from its resemblance to a boiling kettle.
This is on the Ottawa River itself.  The Rideau Fall is divided
into two branches, thus forming an island in the middle, as is the
case at Niagara.  It is pretty enough, and worth visiting even were
it farther from the town than it is; but by those who have hunted
out many cataracts in their travels it will not be considered very
remarkable.  The Chaudiere Fall I did think very remarkable.  It is
of trifling depth, being formed by fractures in the rocky bed of
the river; but the waters have so cut the rock as to create
beautiful forms in the rush which they make in their descent.
Strangers are told to look at these falls from the suspension
bridge; and it is well that they should do so.  But, in so looking
at them, they obtain but a very small part of their effect.  On the
Ottawa side of the bridge is a brewery, which brewery is surrounded
by a huge timber-yard.  This timber yard I found to be very muddy,
and the passing and repassing through it is a work of trouble; but
nevertheless let the traveler by all means make his way through the
mud, and scramble over the timber, and cross the plank bridges
which traverse the streams of the saw-mills, and thus take himself
to the outer edge of the wood-work over the water.  If he will then
seat himself, about the hour of sunset, he will see the Chaudiere
Fall aright.

But the glory of Ottawa will be---and, indeed, already is---the set
of public buildings which is now being erected on the rock which
guards, as it were, the town from the river.  How much of the
excellence of these buildings may be due to the taste of Sir Edmund
Head, the late governor, I do not know.  That he has greatly
interested himself in the subject, is well known; and, as the style
of the different buildings is so much alike as to make one whole,
though the designs of different architects were selected and these
different architects employed, I imagine that considerable
alterations must have been made in the original drawings.  There
are three buildings, forming three sides of a quadrangle; but they
are not joined, the vacant spaces at the corner being of
considerable extent.  The fourth side of the quadrangle opens upon
one of the principal streets of the town.  The center building is
intended for the Houses of Parliament, and the two side buildings
for the government offices.  Of the first Messrs. Fuller and Jones
are the architects, and of the latter Messrs. Stent and Laver.  I
did not have the pleasure of meeting any of these gentlemen; but I
take upon myself to say that, as regards purity of art and
manliness of conception, their joint work is entitled to the very
highest praise.  How far the buildings may be well arranged for the
required purposes---how far they maybe economical in construction or
specially adapted to the severe climate of the country---I cannot
say; but I have no hesitation in risking my reputation for judgment
in giving my warmest commendation to them as regards beauty of
outline and truthful nobility of detail.

I shall not attempt to describe them, for I should interest no one
in doing so, and should certainly fail in my attempt to make any
reader understand me.  I know no modern Gothic purer of its kind or
less sullied with fictitious ornamentation.  Our own Houses of
Parliament are very fine, but it is, I believe, generally felt that
the ornamentation is too minute; and, moreover, it may be
questioned whether perpendicular Gothic is capable of the highest
nobility which architecture can achieve.  I do not pretend to say
that these Canadian public buildings will reach that highest
nobility.  They must be finished before any final judgment can be
pronounced; but I do feel very certain that that final judgment
will be greatly in their favor.  The total frontage of the
quadrangle, including the side buildings, is 1200 feet; that of the
center buildings is 475.  As I have said before, 225,000 pounds
have already been expended; and it is estimated that the total
cost, including the arrangement and decoration of the ground behind
the building and in the quadrangle, will be half a million.

The buildings front upon what will, I suppose, be the principal
street of Ottawa, and they stand upon a rock looking immediately
down upon the river.  In this way they are blessed with a site
peculiarly happy.  Indeed, I cannot at this moment remember any so
much so.  The Castle of Edinburgh stands very well; but then, like
many other castles, it stands on a summit by itself, and can only
be approached by a steep ascent.  These buildings at Ottawa, though
they look down from a grand eminence immediately on the river, are
approached from the town without any ascent.  The rock, though it
falls almost precipitously down to the water is covered with trees
and shrubs; and then the river that runs beneath is rapid, bright,
and picturesque in the irregularity of all its lines.  The view
from the back of the library, up to the Chaudiere Falls and to the
saw-mills by which they are surrounded, is very lovely.  So that I
will say again that I know no site for such a set of buildings so
happy as regards both beauty and grandeur.  It is intended that the
library, of which the walls were only ten feet above the ground
when I was there, shall be an octagonal building, in shape and
outward character like the chapter house of a cathedral.  This
structure will, I presume, be surrounded by gravel walks and green
sward.  Of the library there is a large model showing all the
details of the architecture; and if that model be ultimately
followed, this building alone will be worthy of a visit from
English tourists.  To me it was very wonderful to find such an
edifice in the course of erection on the banks of a wild river
almost at the back of Canada.  But if ever I visit Canada again, it
will be to see those buildings when completed.

And now, like all friendly critics, having bestowed my modicum of
praise, I must proceed to find fault.  I cannot bring myself to
administer my sugar-plum without adding to it some bitter morsel by
way of antidote.  The building to the left of the quadrangle as it
is entered is deficient in length, and on that account appears mean
to the eye.  The two side buildings are brought up close to the
street, so that each has a frontage immediately on the street.
Such being the case, they should be of equal length, or nearly so.
Had the center of one fronted the center of the other, a difference
of length might have been allowed; but in this case the side front
of the smaller one would not have reached the street.  As it is,
the space between the main building and the smaller wing is
disproportionably large, and the very distance at which it stands
will, I fear, give to it that appearance of meanness of which I
have spoken.  The clerk of the works, who explained to me with much
courtesy the plan of the buildings, stated that the design of this
wing was capable of elongation, and had been expressly prepared
with that object.  If this be so, I trust that the defect will be
remedied.

The great trade of Canada is lumbering; and lumbering consists in
cutting down pine-trees up in the far distant forests, in hewing or
sawing them into shape for market, and getting them down the rivers
to Quebec, from whence they are exported to Europe, and chiefly to
England.  Timber in Canada is called lumber; those engaged in the
trade are called lumberers, and the business itself is called
lumbering.  After a lapse of time it must no doubt become
monotonous to those engaged in it, and the name is not engaging;
but there is much about it that is very picturesque.  A saw-mill
worked by water power is almost always a pretty object; and stacks
of new-cut timber are pleasant to the smell, and group themselves
not amiss on the water's edge.  If I had the time, and were a year
or two younger, I should love well to go up lumbering into the
woods.  The men for this purpose are hired in the fall of the year,
and are sent up hundreds of miles away to the pine forests in
strong gangs.  Everything is there found for them.  They make log
huts for their shelter, and food of the best and the strongest is
taken up for their diet.  But no strong drink of any kind is
allowed, nor is any within reach of the men.  There are no publics,
no shebeen houses, no grog-shops.  Sobriety is an enforced virtue;
and so much is this considered by the masters, and understood by
the men, that very little contraband work is done in the way of
taking up spirits to these settlements.  It may be said that the
work up in the forests is done with the assistance of no stronger
drink than tea; and it is very hard work.  There cannot be much
work that is harder; and it is done amid the snows and forests of a
Canadian winter.  A convict in Bermuda cannot get through his daily
eight hours of light labor without an allowance of rum; but a
Canadian lumberer can manage to do his daily task on tea without
milk.  These men, however, are by no means teetotalers.  When they
come back to the towns they break out, and reward themselves for
their long-enforced moderation.  The wages I found to be very
various, running from thirteen or fourteen dollars a month to
twenty-eight or thirty, according to the nature of the work.  The
men who cut down the trees receive more than those who hew them
when down, and these again more than the under class who make the
roads and clear the ground.  These money wages, however, are in
addition to their diet.  The operation requiring the most skill is
that of marking the trees for the axe.  The largest only are worth
cutting, and form and soundness must also be considered.

But if I were about to visit a party of lumberers in the forest, I
should not be disposed to pass a whole winter with them.  Even of a
very good thing one may have too much, I would go up in the spring,
when the rafts are being formed in the small tributary streams, and
I would come down upon one of them, shooting the rapids of the
rivers as soon as the first freshets had left the way open.  A
freshet in the rivers is the rush of waters occasioned by melting
snow and ice.  The first freshets take down the winter waters of
the nearer lakes and rivers.  Then the streams become for a time
navigable, and the rafts go down.  After that comes the second
freshet, occasioned by the melting of far-off snow and ice up in
the great northern lakes, which are little known.  These rafts are
of immense construction, such as those which we have seen on the
Rhone and Rhine, and often contain timber to the value of two,
three, and four thousand pounds.  At the rapids the large rafts
are, as it were, unyoked, and divided into small portions, which go
down separately.  The excitement and motion of such transit must, I
should say, be very joyous.  I was told that the Prince of Wales
desired to go down a rapid on a raft, but that the men in charge
would not undertake to say that there was no possible danger;
whereupon those who accompanied the prince requested his Royal
Highness to forbear.  I fear that, in these careful days, crowned
heads and their heirs must often find themselves in the position of
Sancho at the banquet.  The sailor prince, who came after his
brother, was allowed to go down a rapid, and got, as I was told,
rather a rough bump as he did so.

Ottawa is a great place for these timber rafts.  Indeed, it may, I
think, be called the headquarters of timber for the world.  Nearly
all the best pine-wood comes down the Ottawa and its tributaries.
The other rivers by which timber is brought down to the St.
Lawrence are chiefly the St. Maurice, the Madawaska, and the
Saguenay; but the Ottawa and its tributaries water 75,000 square
miles, whereas the other three rivers, with their tributaries,
water only 53,000.  The timber from the Ottawa and St. Maurice
finds its way down the St. Lawrence to Quebec, where, however, it
loses the whole of its picturesque character.  The Saguenay and the
Madawaska fall into the St. Lawrence below Quebec.

From Ottawa we went by rail to Prescott, which is surely one of the
most wretched little places to be found in any country.
Immediately opposite to it, on the other side of the St. Lawrence,
is the thriving town of Ogdensburg.  But Ogdensburg is in the
United States.  Had we been able to learn at Ottawa any facts as to
the hours of the river steamers and railways, we might have saved
time and have avoided Prescott; but this was out of the question.
Had I asked the exact hour at which I might reach Calcutta by the
quickest route, an accurate reply would not have been more out of
the question.  I was much struck, at Prescott---and, indeed, all
through Canada, though more in the upper than in the lower
province---by the sturdy roughness, some would call it insolence, of
those of the lower classes of the people with whom I was brought
into contact.  If the words ``lower classes'' give offense to any
reader, I beg to apologize---to apologize, and to assert that I am
one of the last of men to apply such a term in a sense of reproach
to those who earn their bread by the labor of their hands.  But it
is hard to find terms which will be understood; and that term,
whether it give offense or no, will be understood.  Of course such
a complaint as that I now make is very common as made against the
States.  (Men in the States, with horned hands and fustian coats,
are very often most unnecessarily insolent in asserting their
independence.  What I now mean to say is that precisely the same
fault is to be found in Canada.  I know well what the men mean when
they offend in this manner.  And when I think on the subject with
deliberation at my own desk, I can not only excuse, but almost
approve them.  But when one personally encounters this corduroy
braggadocio; when the man to whose services one is entitled answers
one with determined insolence; when one is bidden to follow ``that
young lady,'' meaning the chambermaid, or desired, with a toss of
the head, to wait for the ``gentleman who is coming,'' meaning the
boots, the heart is sickened, and the English traveler pines for
the civility---for the servility, if my American friends choose to
call it so---of a well-ordered servant.  But the whole scene is
easily construed, and turned into English.  A man is asked by a
stranger some question about his employment, and he replies in a
tone which seems to imply anger, insolence, and a dishonest
intention to evade the service for which he is paid.  Or, if there
be no question of service or payment, the man's manner will be the
same, and the stranger feels that he is slapped in the face and
insulted.  The translation of it is this: The man questioned, who
is aware that as regards coat, hat, boots, and outward cleanliness
he is below him by whom he is questioned, unconsciously feels
himself called upon to assert his political equality.  It is his
shibboleth that he is politically equal to the best, that he is
independent, and that his labor, though it earn him but a dollar a
day by porterage, places him as a citizen on an equal rank with the
most wealthy fellow-man that may employ or accost him.  But, being
so inferior in that coat, hat, and boots matter, he is forced to
assert his equality by some effort.  As he improves in externals,
he will diminish the roughness of his claim.  As long as the man
makes his claim with any roughness, so long does he acknowledge
within himself some feeling of external inferiority.  When that has
gone---when the American has polished himself up by education and
general well-being to a feeling of external equality with
gentlemen, he shows, I think, no more of that outward braggadocio
of independence than a Frenchman.

But the blow at the moment of the stroke is very galling.  I
confess that I have occasionally all but broken down beneath it.
But when it is thought of afterward it admits of full excuse.  No
effort that a man can make is better than a true effort at
independence.  But this insolence is a false effort, it will be
said.  It should rather be called a false accompaniment to a life-
long true effort.  The man probably is not dishonest, does not
desire to shirk any service which is due from him, is not even
inclined to insolence.  Accept his first declaration of equality
for that which it is intended to represent, and the man afterward
will be found obliging and communicative.  If occasion offer he
will sit down in the room with you, and will talk with you on any
subject that he may choose; but having once ascertained that you
show no resentment for this assertion of equality, he will do
pretty nearly all that is asked.  He will at any rate do as much in
that way as an Englishman.  I say thus much on this subject now
especially, because I was quite as much struck by the feeling in
Canada as I was within the States.

From Prescott we went on by the Grand Trunk Railway to Toronto, and
stayed there for a few days.  Toronto is the capital of the
province of Upper Canada, and I presume will in some degree remain
so, in spite of Ottawa and its pretensions.  That is, the law
courts will still be held there.  I do not know that it will enjoy
any other supremacy unless it be that of trade and population.
Some few years ago Toronto was advancing with rapid strides, and
was bidding fair to rival Quebec, or even perhaps Montreal.
Hamilton also, another town of Upper Canada, was going ahead in the
true American style; but then reverses came in trade, and the towns
were checked for awhile.  Toronto, with a neighboring suburb which
is a part of it, as Southwark is of London, contains now over
50,000 inhabitants.  The streets are all parallelogramical, and
there is not a single curvature to rest the eye.  It is built down
close upon Lake Ontario; and as it is also on the Grand Trunk
Railway, it has all the aid which facility of traffic can give it.

The two sights of Toronto are the Osgoode Hall and the University.
The Osgoode Hall is to Upper Canada what the Four Courts are to
Ireland.  The law courts are all held there.  Exteriorly, little
can be said for Osgoode Hall, whereas the exterior of the Four
Courts in Dublin is very fine; but as an interior, the temple of
Themis at Toronto beats hollow that which the goddess owns in
Dublin.  In Dublin the courts themselves are shabby, and the space
under the dome is not so fine as the exterior seems to promise that
it should be.  In Toronto the courts themselves are, I think, the
most commodious that I ever saw, and the passages, vestibules, and
hall are very handsome.  In Upper Canada the common-law judges and
those in chancery are divided as they are in England; but it is, as
I was told, the opinion of Canadian lawyers that the work may be
thrown together.  Appeal is allowed in criminal cases; but as far
as I could learn such power of appeal is held to be both
troublesome and useless.  In Lower Canada the old French laws are
still administered.

But the University is the glory of Toronto.  This is a Gothic
building, and will take rank after, but next to, the buildings at
Ottawa.  It will be the second piece of noble architecture in
Canada, and as far as I know on the American continent.  It is, I
believe, intended to be purely Norman, though I doubt whether the
received types of Norman architecture have not been departed from
in many of the windows.  Be this as it may, the college is a manly,
noble structure, free from false decoration, and infinitely
creditable to those who projected it.  I was informed by the head
of the college that it has been open only two years; and here also
I fancy that the colony has been much indebted to the taste of the
late Governor, Sir Edmund Head.

Toronto as a city is not generally attractive to a traveler.  The
country around it is flat; and, though it stands on a lake, that
lake has no attributes of beauty.  Large inland seas, such as are
these great Northern lakes of America, never have such attributes.
Picturesque mountains rise from narrow valleys, such as form the
beds of lakes in Switzerland, Scotland, and Northern Italy; but
from such broad waters as those of Lake Ontario, Lake Erie, and
Lake Michigan, the shores shelve very gradually, and have none of
the materials of lovely scenery.

The streets in Toronto are framed with wood, or rather planked, as
are those of Montreal and Quebec; but they are kept in better
order.  I should say that the planks are first used at Toronto,
then sent down by the lake to Montreal, and when all but rotted out
there, are again floated off by the St. Lawrence to be used in the
thoroughfares of the old French capital.  But if the streets of
Toronto are better than those of the other towns, the roads around
it are worse.  I had the honor of meeting two distinguished members
of the Provincial Parliament at dinner some few miles out of town,
and, returning back a short while after they had left our host's
house, was glad to be of use in picking them up from a ditch into
which their carriage had been upset.  To me it appeared all but
miraculous that any carriage should make its way over that road
without such misadventure.  I may perhaps be allowed to hope that
the discomfiture of these worthy legislators may lead to some
improvement in the thoroughfare.

I had on a previous occasion gone down the St. Lawrence, through
the Thousand Isles and over the Rapids, in one of those large
summer steamboats which ply upon the lake and river.  I cannot say
that I was much struck by the scenery, and therefore did not
encroach upon my time by making the journey again.  Such an opinion
will be regarded as heresy by many who think much of the Thousand
Islands.  I do not believe that they would be expressly noted by
any traveler who was not expressly bidden to admire them.

From Toronto we went across to Niagara, re-entering the States at
Lewiston, in New York.



\chapter{The Connection of the Canadas with Great Britain}


When the American war began troops were sent out to Canada, and
when I was in the provinces more troops were then expected.  The
matter was much talked of, as a matter of course, in Canada, and it
had been discussed in England before I left.  I had seen much said
about it in the English papers since, and it also had become the
subject of very hot question among the politicians of the Northern
States.  The measure had at that time given more umbrage to the
North than anything else done or said by England from the beginning
of the war up to that time, except the declaration made by Lord
John Russell in the House of Commons as to the neutrality to be
preserved by England between the two belligerents.  The argument
used by the Northern States was this: if France collects men and
material of war in the neighborhood of England, England considers
herself injured, calls for an explanation, and talks of invasion.
Therefore, as England is now collecting men and material of war in
our neighborhood, we will consider ourselves injured.  It does not
suit us to ask for an explanation, because it is not our habit to
interfere with other nations.  We will not pretend to say that we
think we are to be invaded.  But as we clearly are injured, we will
express our anger at that injury, and when the opportunity shall
come will take advantage of having that new grievance.

As we all know, a very large increase of force was sent when we
were still in doubt as to the termination of the Trent affair, and
imagined that war was imminent.  But the sending of that large
force did not anger the Americans as the first dispatch of troops
to Canada had angered them.  Things had so turned out that measures
of military precaution were acknowledged by them to be necessary.
I cannot, however, but think that Mr.\ Seward might have spared that
offer to send British troops across Maine, and so also have all his
countrymen thought by whom I have heard the matter discussed.

As to any attempt at invasion of Canada by the Americans, or idea
of punishing the alleged injuries suffered by the States from Great
Britain by the annexation of those provinces, I do not believe that
any sane-minded citizens of the States believe in the possibility
of such retaliation.  Some years since the Americans thought that
Canada might shine in the Union firmament as a new star; but that
delusion is, I think, over.  Such annexation, if ever made, must
have been made not only against the arms of England, but must also
have been made in accordance with the wishes of the people so
annexed.  It was then believed that the Canadians were not averse
to such a change, and there may possibly have then been among them
the remnant of such a wish.  There is certainly no such desire now,
not even a remnant of such a desire; and the truth on this matter
is, I think, generally acknowledged.  The feeling in Canada is one
of strong aversion to the United States government and of
predilection for self-government under the English Crown.  A
faineant governor and the prestige of British power is now the
political aspiration of the Canadians in general; and I think that
this is understood in the States.  Moreover, the States have a job
of work on hand which, as they themselves are well aware, is taxing
all their energies.  Such being the case, I do not think that
England needs to fear any invasion of Canada authorized by the
States government.

This feeling of a grievance on the part of the States was a
manifest absurdity.  The new reinforcement of the garrisons in
Canada did not, when I was in Canada, amount, as I believe, to more
than 2000 men.  But had it amounted to 20,000, the States would
have had no just ground for complaint.  Of all nationalities that
in modern days have risen to power, they, above all others, have
shown that they would do what they liked with their own,
indifferent to foreign counsels and deaf to foreign remonstrance.
``Do you go your way, and let us go ours.  We will trouble you with
no question, nor do you trouble us.''  Such has been their national
policy, and it has obtained for them great respect.  They have
resisted the temptation of putting their fingers into the caldron
of foreign policy; and foreign politicians, acknowledging their
reserve in this respect, have not been offended at the bristles
with which their Noli me tangere has been proclaimed.  Their
intelligence has been appreciated, and their conduct has been
respected.  But if this has been their line of policy, they must be
entirely out of court in raising any question as to the position of
British troops on British soil.

``It shows us that you doubt us,'' an American says, with an air of
injured honor---or did say, before that Trent affair.  ``And it is
done to express sympathy with the South.  The Southerners
understand it, and we understand it also.  We know where your
hearts are---nay, your very souls.  They are among the slave-
begotten cotton bales of the rebel South.''  Then comes the whole of
the long argument in which it seems so easy to an Englishman to
prove that England, in the whole of this sad matter, has been true
and loyal to her friend.  She could not interfere when the husband
and wife would quarrel.  She could only grieve, and wish that
things might come right and smooth for both parties.  But the
argument, though so easy, is never effectual.

It seems to me foolish in an American to quarrel with England for
sending soldiers to Canada; but I cannot say that I thought it was
well done to send them at the beginning of the war.  The English
government did not, I presume, take this step with reference to any
possible invasion of Canada by the government of the States.  We
are fortifying Portsmouth, and Portland, and Plymouth, because we
would fain be safe against the French army acting under a French
Emperor.  But we sent 2000 troops to Canada, if I understand the
matter rightly, to guard our provinces against the filibustering
energies of a mass of unemployed American soldiers, when those
soldiers should come to be disbanded.  When this war shall be over---%
a war during which not much, if any, under a million of American
citizens will have been under arms---it will not be easy for all who
survive to return to their old homes and old occupations.  Nor does
a disbanded soldier always make a good husbandman, notwithstanding
the great examples of Cincinnatus and Bird-o'-freedom Sawin.  It
may be that a considerable amount of filibustering energy will be
afloat, and that the then government of those who neighbor us in
Canada will have other matters in hand more important to them than
the controlling of these unruly spirits.  That, as I take it, was
the evil against which we of Great Britain and of Canada desired to
guard ourselves.

But I doubt whether 2000 or 10,000 British soldiers would be any
effective guard against such inroads, and I doubt more strongly
whether any such external guarding will be necessary.  If the
Canadians were prepared to fraternize with filibusters from the
States, neither three nor ten thousand soldiers would avail against
such a feeling over a frontier stretching from the State of Maine
to the shores of Lake Huron and Lake Erie.  If such a feeling did
exist---if the Canadians wished the change---in God's name let them
go.  It is for their sakes, and not for our own, that we would have
them bound to us.  But the Canadians are averse to such a change
with a degree of feeling that amounts to national intensity.  Their
sympathies are with the Southern States, not because they care for
cotton, not because they are anti-abolitionists, not because they
admire the hearty pluck of those who are endeavoring to work out
for themselves a new revolution.  They sympathize with the South
from strong dislike to the aggression, the braggadocio, and the
insolence they have felt upon their own borders.  They dislike Mr.\ %
Seward's weak and vulgar joke with the Duke of Newcastle.  They
dislike Mr.\ Everett's flattering hints to his countrymen as to the
one nation that is to occupy the whole continent.  They dislike the
Monroe doctrine.  They wonder at the meekness with which England
has endured the vauntings of the Northern States, and are endued
with no such meekness of their own.  They would, I believe, be well
prepared to meet and give an account of any filibusters who might
visit them; and I am not sure that it is wisely done on our part to
show any intention of taking the work out of their hands.

But I am led to this opinion in no degree by a feeling that Great
Britain ought to grudge the cost of the soldiers.  If Canada will
be safer with them, in Heaven's name let her have them.  It has
been argued in many places, not only with regard to Canada, but as
to all our self-governed colonies, that military service should not
be given at British expense and with British men to any colony
which has its own representative government and which levies its
own taxes.  ``While Great Britain absolutely held the reins of
government, and did as it pleased with the affairs of its
dependencies,'' such politicians say, ``it was just and right that
she should pay the bill.  As long as her government of a colony was
paternal, so long was it right that the mother country should put
herself in the place of a father, and enjoy a father's undoubted
prerogative of putting his hand into his breeches pocket to provide
for all the wants of his child.  But when the adult son set up for
himself in business---having received education from the parent, and
having had his apprentice fees duly paid---then that son should
settle his own bills, and look no longer to the paternal pocket.''
Such is the law of the world all over, from little birds, whose
young fly away when fledged, upward to men and nations.  Let the
father work for the child while he is a child; but when the child
has become a man, let him lean no longer on his father's staff.

The argument is, I think, very good; but it proves not that we are
relieved from the necessity of assisting our colonies with payments
made out of British taxes, but that we are still bound to give such
assistance, and that we shall continue to be so bound as long as we
allow these colonies to adhere to us or as they allow us to adhere
to them.  In fact, the young bird is not yet fully fledged.  That
illustration of the father and the child is a just one, but in
order to make it just it should be followed throughout.  When the
son is in fact established on his own bottom, then the father
expects that he will live without assistance.  But when the son
does so live, he is freed from all paternal control.  The father,
while he expects to be obeyed, continues to fill the paternal
office of paymaster---of paymaster, at any rate, to some extent.
And so, I think, it must be with our colonies.  The Canadas at
present are not independent, and have not political power of their
own apart from the political power of Great Britain.  England has
declared herself neutral as regards the Northern and Southern
States, and by that neutrality the Canadas are bound; and yet the
Canadas were not consulted in the matter.  Should England go to war
with France, Canada must close her ports against French vessels.
If England chooses to send her troops to Canadian barracks, Canada
cannot refuse to accept them.  If England should send to Canada an
unpopular governor, Canada has no power to reject his services.  As
long as Canada is a colony so called, she cannot be independent,
and should not be expected to walk alone.  It is exactly the same
with the colonies of Australia, with New Zealand, with the Cape of
Good Hope, and with Jamaica.  While England enjoys the prestige of
her colonies, while she boasts that such large and now populous
territories are her dependencies, she must and should be content to
pay some portion of the bill.  Surely it is absurd on our part to
quarrel with Caffre warfare, with New Zealand fighting, and the
rest of it.  Such complaints remind one of an ancient pater
familias who insists on having his children and his grandchildren
under the old paternal roof, and then grumbles because the
butcher's bill is high.  Those who will keep large households and
bountiful tables should not be afraid of facing the butcher's bill
or unhappy at the tonnage of the coal.  It is a grand thing, that
power of keeping a large table; but it ceases to be grand when the
items heaped upon it cause inward groans and outward moodiness.

Why should the colonies remain true to us as children are true to
their parents, if we grudge them the assistance which is due to a
child?  They raise their own taxes, it is said, and administer
them.  True; and it is well that the growing son should do
something for himself.  While the father does all for him, the
son's labor belongs to the father.  Then comes a middle state in
which the son does much for himself, but not all.  In that middle
state now stand our prosperous colonies.  Then comes the time when
the son shall stand alone by his own strength; and to that period
of manly, self-respected strength let us all hope that those
colonies are advancing.  It is very hard for a mother country to
know when such a time has come; and hard also for the child-colony
to recognize justly the period of its own maturity.  Whether or no
such severance may ever take place without a quarrel, without
weakness on one side and pride on the other, is a problem in the
world's history yet to be solved.  The most successful child that
ever yet has gone off from a successful parent, and taken its own
path into the world, is without doubt the nation of the United
States.  Their present troubles are the result and the proofs of
their success.  The people that were too great to be dependent on
any nation have now spread till they are themselves too great for a
single nationality.  No one now thinks that that daughter should
have remained longer subject to her mother.  But the severance was
not made in amity, and the shrill notes of the old family quarrel
are still sometimes heard across the waters.

From all this the question arises whether that problem may ever be
solved with reference to the Canadas.  That it will never be their
destiny to join themselves to the States of the Union, I feel fully
convinced.  In the first place it is becoming evident from the
present circumstances of the Union, if it had never been made
evident by history before, that different people with different
habits, living at long distances from each other, cannot well be
brought together on equal terms under one government.  That noble
ambition of the Americans that all the continent north of the
isthmus should be united under one flag, has already been thrown
from its saddle.  The North and South are virtually separated, and
the day will come in which the West also will secede.  As
population increases and trades arise peculiar to those different
climates, the interests of the people will differ, and a new
secession will take place beneficial alike to both parties.  If
this be so, if even there be any tendency this way, it affords the
strongest argument against the probability of any future annexation
of the Canadas.  And then, in the second place, the feeling of
Canada is not American, but British.  If ever she be separated from
Great Britain, she will be separated as the States were separated.
She will desire to stand alone, and to enter herself as one among
the nations of the earth.

She will desire to stand alone; alone, that is without dependence
either on England or on the States.  But she is so circumstanced
geographically that she can never stand alone without amalgamation
with our other North American provinces.  She has an outlet to the
sea at the Gulf of St. Lawrence, but it is only a summer outlet.
Her winter outlet is by railway through the States, and no other
winter outlet is possible for her except through the sister
provinces.  Before Canada can be nationally great, the line of
railway which now runs for some hundred miles below Quebec to
Riviere du Loup must be continued on through New Brunswick and Nova
Scotia to the port of Halifax.

When I was in Canada I heard the question discussed of a federal
government between the provinces of the two Canadas, New Brunswick,
and Nova Scotia.  To these were added, or not added, according to
the opinion of those who spoke, the smaller outlying colonies of
Newfoundland and Prince Edward's Island.  If a scheme for such a
government were projected in Downing Street, all would no doubt be
included, and a clean sweep would be made without difficulty.  But
the project as made in the colonies appears in different guises, as
it comes either from Canada or from one of the other provinces.
The Canadian idea would be that the two Canadas should form two
States of such a confederation, and the other provinces a third
State.  But this slight participation in power would hardly suit
the views of New Brunswick and Nova Scotia.  In speaking of such a
federal government as this, I shall of course be understood as
meaning a confederation acting in connection with a British
governor, and dependent upon Great Britain as far as the different
colonies are now dependent.

I cannot but think that such a confederation might be formed with
great advantage to all the colonies and to Great Britain.  At
present the Canadas are in effect almost more distant from Nova
Scotia and New Brunswick than they are from England.  The
intercourse between them is very slight---so slight that it may
almost be said that there is no intercourse.  A few men of science
or of political importance may from time to time make their way
from one colony into the other, but even this is not common.
Beyond that they seldom see each other.  Though New Brunswick
borders both with Lower Canada and with Nova Scotia, thus making
one whole of the three colonies, there is neither railroad nor
stage conveyance running from one to the other.  And yet their
interests should be similar.  From geographical position their
modes of life must be alike, and a close conjunction between them
is essentially necessary to give British North America any
political importance in the world.  There can be no such
conjunction, no amalgamation of interests, until a railway shall
have been made joining the Canada Grand Trunk Line with the two
outlying colonies.  Upper Canada can feed all England with wheat,
and could do so without any aid of railway through the States, if a
railway were made from Quebec to Halifax.  But then comes the
question of the cost.  The Canada Grand Trunk is at the present
moment at the lowest ebb of commercial misfortune, and with such a
fact patent to the world, what company will come forward with funds
for making four or five hundred miles of railway, through a
district of which one-half is not yet prepared for population?  It
would be, I imagine, out of the question that such a speculation
should for many years give any fair commercial interest on the
money to be expended.  But nevertheless to the colonies---that is,
to the enormous regions of British North America---such a railroad
would be invaluable.  Under such circumstances it is for the Home
Government and the colonies between them to see how such a measure
may be carried out.  As a national expenditure, to be defrayed in
the course of years by the territories interested, the sum of money
required would be very small.

But how would this affect England?  And how would England be
affected by a union of the British North American colonies under
one federal government?  Before this question can be answered, he
who prepares to answer it must consider what interest England has
in her colonies, and for what purpose she holds them.  Does she
hold them for profit, or for glory, or for power; or does she hold
them in order that she may carry out the duty which has devolved
upon her of extending civilization, freedom, and well-being through
the new uprising nations of the world?  Does she hold them, in
fact, for her own benefit, or does she hold them for theirs?  I
know nothing of the ethics of the Colonial Office, and not much
perhaps of those of the House of Commons; but looking at what Great
Britain has hitherto done in the way of colonization, I cannot but
think that the national ambition looks to the welfare of the
colonists, and not to home aggrandizement.  That the two may run
together is most probable.  Indeed, there can be no glory to a
people so great or so readily recognized by mankind at large as
that of spreading civilization from east to west and from north to
south.  But the one object should be the prosperity of the
colonists, and not profit, nor glory, nor even power, to the parent
country.

There is no virtue of which more has been said and sung than
patriotism, and none which, when pure and true, has led to finer
results.  Dulce et decorum est pro patria mori.  To live for one's
country also is a very beautiful and proper thing.  But if we
examine closely much patriotism, that is so called, we shall find
it going hand in hand with a good deal that is selfish, and with
not a little that is devilish.  It was some fine fury of patriotic
feeling which enabled the national poet to put into the mouth of
every Englishman that horrible prayer with regard to our enemies
which we sing when we wish to do honor to our sovereign.  It did
not seem to him that it might be well to pray that their hearts
should be softened, and our own hearts softened also.  National
success was all that a patriotic poet could desire, and therefore
in our national hymn have we gone on imploring the Lord to arise
and scatter our enemies; to confound their politics, whether they
be good or ill; and to expose their knavish tricks---such knavish
tricks being taken for granted.  And then, with a steady
confidence, we used to declare how certain we were that we should
achieve all that was desirable, not exactly by trusting to our
prayer to heaven, but by relying almost exclusively on George the
Third or George the Fourth.  Now I have always thought that that
was rather a poor patriotism.  Luckily for us, our national conduct
has not squared itself with our national anthem.  Any patriotism
must be poor which desires glory, or even profit, for a few at the
expense of the many, even though the few be brothers and the many
aliens.  As a rule, patriotism is a virtue only because man's
aptitude for good is so finite that he cannot see and comprehend a
wider humanity.  He can hardly bring himself to understand that
salvation should be extended to Jew and Gentile alike.  The word
philanthropy has become odious, and I would fain not use it; but
the thing itself is as much higher than patriotism as heaven is
above the earth.

A wish that British North America should ever be severed from
England, or that the Australian colonies should ever be so severed,
will by many Englishmen be deemed unpatriotic.  But I think that
such severance is to be wished if it be the case that the colonies
standing alone would become more prosperous than they are under
British rule.  We have before us an example in the United States of
the prosperity which has attended such a rupture of old ties.  I
will not now contest the point with those who say that the present
moment of an American civil war is ill chosen for vaunting that
prosperity.  There stand the cities which the people have built,
and their power is attested by the world-wide importance of their
present contest.  And if the States have so risen since they left
their parent's apron-string, why should not British North America
rise as high?  That the time has as yet come for such rising I do
not think; but that it will soon come I do most heartily hope.  The
making of the railway of which I have spoken, and the amalgamation
of the provinces would greatly tend to such an event.  If
therefore, England desires to keep these colonies in a state of
dependency; if it be more essential to her to maintain her own
power with regard to them than to increase their influence; if her
main object be to keep the colonies and not to improve the
colonies, then I should say that an amalgamation of the Canadas
with Nova Scotia and New Brunswick should not be regarded with
favor by statesmen in Downing Street.  But if, as I would fain
hope, and do partly believe, such ideas of national power as these
are now out of vogue with British statesmen, then I think that such
an amalgamation should receive all the support which Downing Street
can give it.

The United States severed themselves from Great Britain with a
great struggle, and after heart-burnings and bloodshed.  Whether
Great Britain will ever allow any colony of hers to depart from out
of her nest, to secede and start for herself, without any struggle
or heart-burnings, with all furtherance for such purpose which an
old and powerful country can give to a new nationality then first
taking its own place in the world's arena, is a problem yet to be
solved.  There is, I think, no more beautiful sight than that of a
mother, still in all the glory of womanhood, preparing the wedding
trousseau for her daughter.  The child hitherto has been obedient
and submissive.  She has been one of a household in which she has
held no command.  She has sat at table as a child, fitting herself
in all things to the behests of others.  But the day of her power
and her glory, and also of her cares and solicitude, is at hand.
She is to go forth, and do as she best may in the world under that
teaching which her old home has given her.  The hour of separation
has come; and the mother, smiling through her tears, sends her
forth decked with a bounteous hand, and furnished with full stores,
so that all may be well with her as she enters on her new duties.
So is it that England should send forth her daughters.  They should
not escape from her arms with shrill screams and bleeding wounds,
with ill-omened words which live so long, though the speakers of
them lie cold in their graves.

But this sending forth of a child-nation to take its own political
status in the world has never yet been done by Great Britain.  I
cannot remember that such has ever been done by any great power
with reference to its dependency; by any power that was powerful
enough to keep such dependency within its grasp.  But a man
thinking on these matters cannot but hope that a time will come
when such amicable severance may be effected.  Great Britain cannot
think that through all coming ages she is to be the mistress of the
vast continent of Australia, lying on the other side of the globe's
surface; that she is to be the mistress of all South Africa, as
civilization shall extend northward; that the enormous territories
of British North America are to be subject forever to a veto from
Downing Street.  If the history of past empires does not teach her
that this may not be so, at least the history of the United States
might so teach her.  ``But we have learned a lesson from those
United States,'' the patriot will argue who dares to hope that the
glory and extent of the British empire may remain unimpaired in
saecula saeculorum.  ``Since that day we have given political rights
to our colonies, and have satisfied the political longings of their
inhabitants.  We do not tax their tea and stamps, but leave it to
them to tax themselves as they may please.''  True.  But in
political aspirations the giving of an inch has ever created the
desire for an ell.  If the Australian colonies even now, with their
scanty population and still young civilization, chafe against
imperial interference, will they submit to it when they feel within
their veins all the full blood of political manhood?  What is the
cry even of the Canadians---of the Canadians who are thoroughly
loyal to England?  Send us a faineant governor, a King Log, who
will not presume to interfere with us; a governor who will spend
his money and live like a gentleman, and care little or nothing for
politics.  That is the Canadian beau ideal of a governor.  They are
to govern themselves; and he who comes to them from England is to
sit among them as the silent representative of England's
protection.  If that be true---and I do not think that any who know
the Canadas will deny it---must it not be presumed that they will
soon also desire a faineant minister in Downing Street?  Of course
they will so desire.  Men do not become milder in their aspirations
for political power the more that political power is extended to
them.  Nor would it be well that they should be so humble in their
desires.  Nations devoid of political power have never risen high
in the world's esteem.  Even when they have been commercially
successful, commerce has not brought to them the greatness which it
has always given when joined with a strong political existence.
The Greeks are commercially rich and active; but ``Greece'' and
``Greek'' are bywords now for all that is mean.  Cuba is a colony,
and putting aside the cities of the States, the Havana is the
richest town on the other side of the Atlantic, and commercially
the greatest; but the political villainy of Cuba, her daily
importation of slaves, her breaches of treaty, and the bribery of
her all but royal governor, are known to all men.  But Canada is
not dishonest; Canada is no byword for anything evil; Canada eats
her own bread in the sweat of her brow, and fears a bad word from
no man.  True.  But why does New York, with its suburbs boast a
million of inhabitants, while Montreal has 85,000?  Why has that
babe in years, Chicago, 120,000, while Toronto has not half the
number?  I do not say that Montreal and Toronto should have gone
ahead abreast with New York and Chicago.  In such races one must be
first, and one last.  But I do say that the Canadian towns will
have no equal chance till they are actuated by that feeling of
political independence which has created the growth of the towns in
the United States.

I do not think that the time has yet come in which Great Britain
should desire the Canadians to start for themselves.  There is the
making of that railroad to be effected, and something done toward
the union of those provinces.  Canada could no more stand alone
without New Brunswick and Nova Scotia, than could those latter
colonies without Canada.  But I think it would be well to be
prepared for such a coming day; and that it would at any rate be
well to bring home to ourselves and realize the idea of such
secession on the part of our colonies, when the time shall have
come at which such secession may be carried out with profit and
security to them.  Great Britain, should she ever send forth her
child alone into the world, must of course guarantee her security.
Such guarantees are given by treaties; and, in the wording of them,
it is presumed that such treaties will last forever.  It will be
argued that in starting British North America as a political power
on its own bottom, we should bind ourself to all the expense of its
defense, while we should give up all right to any interference in
its concerns; and that, from a state of things so unprofitable as
this, there would be no prospect of a deliverance.  But such
treaties, let them be worded how they will, do not last forever.
For a time, no doubt, Great Britain would be so hampered---if indeed
she would feel herself hampered by extending her name and prestige
to a country bound to her by ties such as those which would then
exist between her and this new nation.  Such treaties are not
everlasting, nor can they be made to last even for ages.  Those who
word them seem to think that powers and dynasties will never pass
away.  But they do pass away, and the balance of power will not
keep itself fixed forever on the same pivot.  The time may come---%
that it may not come soon we will all desire---but the time may come
when the name and prestige of what we call British North America
will be as serviceable to Great Britain as those of Great Britain
are now serviceable to her colonies.

But what shall be the new form of government for the new kingdom?
That is a speculation very interesting to a politician, though one
which to follow out at great length in these early days would be
rather premature.  That it should be a kingdom---that the political
arrangement should be one of which a crowned hereditary king should
form part---nineteen out of every twenty Englishmen would desire;
and, as I fancy, so would also nineteen out of every twenty
Canadians.  A king for the United States, when they first
established themselves, was impossible.  A total rupture from the
Old World and all its habits was necessary for them.  The name of a
king, or monarch, or sovereign had become horrible to their ears.
Even to this day they have not learned the difference between
arbitrary power retained in the hand of one man, such as that now
held by the Emperor over the French, and such hereditary headship
in the State as that which belongs to the Crown in Great Britain.
And this was necessary, seeing that their division from us was
effected by strife, and carried out with war and bitter
animosities.  In those days also there was a remnant, though but a
small remnant, of the power of tyranny left within the scope of the
British Crown.  That small remnant has been removed; and to me it
seems that no form of existing government, no form of government
that ever did exist, gives or has given so large a measure of
individual freedom to all who live under it as a constitutional
monarchy in which the Crown is divested of direct political power.

I will venture then to suggest a king for this new nation; and,
seeing that we are rich in princes, there need be no difficulty in
the selection.  Would it not be beautiful to see a new nation
established under such auspices, and to establish a people to whom
their independence had been given, to whom it had been freely
surrendered as soon as they were capable of holding the position
assigned to them!



\chapter{Niagara}


Of all the sights on this earth of ours which tourists travel to
see---at least of all those which I have seen---I am inclined to give
the palm to the Falls of Niagara.  In the catalogue of such sights
I intend to include all buildings, pictures, statues, and wonders
of art made by men's hands, and also all beauties of nature
prepared by the Creator for the delight of his creatures.  This is
a long word; but, as far as my taste and judgment go, it is
justified.  I know no other one thing so beautiful, so glorious,
and so powerful.  I would not by this be understood as saying that
a traveler wishing to do the best with his time should first of all
places seek Niagara.  In visiting Florence he may learn almost all
that modern art can teach.  At Rome he will be brought to
understand the cold hearts, correct eyes, and cruel ambition of the
old Latin race.  In Switzerland he will surround himself with a
flood of grandeur and loveliness, and fill himself, if he be
capable of such filling, with a flood of romance.  The tropics will
unfold to him all that vegetation in its greatest richness can
produce.  In Paris he will find the supreme of polish, the ne plus
ultra of varnish according to the world's capability of varnishing.
And in London he will find the supreme of power, the ne plus ultra
of work according to the world's capability of working.  Any one of
such journeys may be more valuable to a man---nay, any one such
journey must be more valuable to a man---than a visit to Niagara.
At Niagara there is that fall of waters alone.  But that fall is
more graceful than Giotto's tower, more noble than the Apollo.  The
peaks of the Alps are not so astounding in their solitude.  The
valleys of the Blue Mountains in Jamaica are less green.  The
finished glaze of life in Paris is less invariable; and the full
tide of trade round the Bank of England is not so inexorably
powerful.

I came across an artist at Niagara who was attempting to draw the
spray of the waters.  ``You have a difficult subject,'' said I.  ``All
subjects are difficult,'' he replied, ``to a man who desires to do
well.''  ``But yours, I fear is impossible,'' I said.  ``You have no
right to say so till I have finished my picture,'' he replied.  I
acknowledged the justice of his rebuke, regretted that I could not
remain till the completion of his work should enable me to revoke
my words, and passed on.  Then I began to reflect whether I did not
intend to try a task as difficult in describing the falls, and
whether I felt any of that proud self-confidence which kept him
happy at any rate while his task was in hand.  I will not say that
it is as difficult to describe aright that rush of waters as it is
to paint it well.  But I doubt whether it is not quite as difficult
to write a description that shall interest the reader as it is to
paint a picture of them that shall be pleasant to the beholder.  My
friend the artist was at any rate not afraid to make the attempt,
and I also will try my hand.

That the waters of Lake Erie have come down in their courses from
the broad basins of Lake Michigan, Lake Superior, and Lake Huron;
that these waters fall into Lake Ontario by the short and rapid
river of Niagara; and that the falls of Niagara are made by a
sudden break in the level of this rapid river, is probably known to
all who will read this book.  All the waters of these huge northern
inland seas run over that breach in the rocky bottom of the stream;
and thence it comes that the flow is unceasing in its grandeur, and
that no eye can perceive a difference in the weight, or sound, or
violence of the fall whether it be visited in the drought of
autumn, amid the storms of winter, or after the melting of the
upper worlds of ice in the days of the early summer.  How many
cataracts does the habitual tourist visit at which the waters fail
him!  But at Niagara the waters never fail.  There it thunders over
its ledge in a volume that never ceases and is never diminished---as
it has done from times previous to the life of man, and as it will
do till tens of thousands of years shall see the rocky bed of the
river worn away back to the upper lake.

This stream divides Canada from the States---the western or
farthermost bank belonging to the British Crown, and the eastern or
nearer bank being in the State of New York.  In visiting Niagara,
it always becomes a question on which side the visitor shall take
up his quarters.  On the Canada side there is no town; but there is
a large hotel beautifully placed immediately opposite to the falls
and this is generally thought to be the best locality for tourists.
In the State of New York is the town called Niagara Falls; and here
there are two large hotels, which, as to their immediate site, are
not so well placed as that in Canada.  I first visited Niagara some
three years since.  I stayed then at the Clifton House, on the
Canada side, and have since sworn by that position.  But the
Clifton House was closed for the season when I was last there, and
on that account we went to the Cataract House, in the town on the
other side.  I now think that I should set up my staff on the
American side, if I went again.  My advice on the subject to any
party starting for Niagara would depend upon their habits or on
their nationality.  I would send Americans to the Canadian side,
because they dislike walking; but English people I would locate on
the American side, seeing that they are generally accustomed to the
frequent use of their own legs.  The two sides are not very easily
approached one from the other.  Immediately below the falls there
is a ferry, which may be traversed at the expense of a shilling;
but the labor of getting up and down from the ferry is
considerable, and the passage becomes wearisome.  There is also a
bridge; but it is two miles down the river, making a walk or drive
of four miles necessary, and the toll for passing is four
shillings, or a dollar, in a carriage, and one shilling on foot.
As the greater variety of prospect can be had on the American side,
as the island between the two falls is approachable from the
American side and not from the Canadian, and as it is in this
island that visitors will best love to linger, and learn to measure
in their minds the vast triumph of waters before them, I recommend
such of my readers as can trust a little---it need be but a little---%
to their own legs to select their hotel at Niagara Falls town.

It has been said that it matters much from what point the falls are
first seen, but to this I demur.  It matters, I think, very little,
or not at all.  Let the visitor first see it all, and learn the
whereabouts of every point, so as to understand his own position
and that of the waters; and then, having done that in the way of
business, let him proceed to enjoyment.  I doubt whether it be not
the best to do this with all sight-seeing.  I am quite sure that it
is the way in which acquaintance may be best and most pleasantly
made with a new picture.

The falls, as I have said, are made by a sudden breach in the level
of the river.  All cataracts are, I presume, made by such breaches;
but generally the waters do not fall precipitously as they do at
Niagara, and never elsewhere, as far as the world yet knows, has a
breach so sudden been made in a river carrying in its channel such
or any approach to such a body of water.  Up above the falls for
more than a mile the waters leap and burst over rapids, as though
conscious of the destiny that awaits them.  Here the river is very
broad and comparatively shallow; but from shore to shore it frets
itself into little torrents, and begins to assume the majesty of
its power.  Looking at it even here, in the expanse which forms
itself over the greater fall, one feels sure that no strongest
swimmer could have a chance of saving himself if fate had cast him
in even among those petty whirlpools.  The waters though so broken
in their descent, are deliciously green.  This color, as seen early
in the morning or just as the sun has set, is so bright as to give
to the place one of its chiefest charms.

This will be best seen from the farther end of the island---Goat
Island as it is called---which, as the reader will understand,
divides the river immediately above the falls.  Indeed, the island
is a part of that precipitously-broken ledge over which the river
tumbles, and no doubt in process of time will be worn away and
covered with water.  The time, however, will be very long.  In the
mean while, it is perhaps a mile round, and is covered thickly with
timber.  At the upper end of the island the waters are divided,
and, coming down in two courses each over its own rapids, form two
separate falls.  The bridge by which the island is entered is a
hundred yards or more above the smaller fall.  The waters here have
been turned by the island, and make their leap into the body of the
river below at a right angle with it---about two hundred yards below
the greater fall.  Taken alone, this smaller cataract would, I
imagine, be the heaviest fall of water known; but taken in
conjunction with the other, it is terribly shorn of its majesty.
The waters here are not green as they are at the larger cataract;
and, though the ledge has been hollowed and bowed by them so as to
form a curve, that curve does not deepen itself into a vast abyss
as it does at the horseshoe up above.  This smaller fall is again
divided; and the visitor, passing down a flight of steps and over a
frail wooden bridge, finds himself on a smaller island in the midst
of it.

But we will go at once on to the glory, and the thunder, and the
majesty, and the wrath of that upper hell of waters.  We are still,
let the reader remember, on Goat Island---still in the States---and
on what is called the American side of the main body of the river.
Advancing beyond the path leading down to the lesser fall, we come
to that point of the island at which the waters of the main river
begin to descend.  From hence across to the Canadian side the
cataract continues itself in one unabated line.  But the line is
very far from being direct or straight.  After stretching for some
little way from the shore to a point in the river which is reached
by a wooden bridge at the end of which stands a tower upon the
rock,---after stretching to this, the line of the ledge bends inward
against the flood---in, and in, and in---till one is led to think
that the depth of that horseshoe is immeasurable.  It has been cut
with no stinting hand.  A monstrous cantle has been worn back out
of the center of the rock, so that the fury of the waters
converges; and the spectator, as he gazes into the hollow with
wishful eyes, fancies that he can hardly trace out the center of
the abyss.

Go down to the end of that wooden bridge, seat yourself on the
rail, and there sit till all the outer world is lost to you.  There
is no grander spot about Niagara than this.  The waters are
absolutely around you.  If you have that power of eye-contrio which
is so necessary to the full enjoyment of scenery, you will see
nothing but the water.  You will certainly hear nothing else; and
the sound, I beg you to remember, is not an ear-cracking, agonizing
crash and clang of noises, but is melodious and soft withal, though
loud as thunder.  It fills your ears, and, as it were, envelops
them, but at the same time you can speak to your neighbor without
an effort.  But at this place, and in these moments, the less of
speaking, I should say, the better.  There is no grander spot than
this.  Here, seated on the rail of the bridge, you will not see the
whole depth of the fall.  In looking at the grandest works of
nature, and of art too, I fancy it is never well to see all.  There
should be something left to the imagination, and much should be
half concealed in mystery.  The greatest charm of a mountain range
is the wild feeling that there must be strange, unknown, desolate
worlds in those far-off valleys beyond.  And so here, at Niagara,
that converging rush of waters may fall down, down at once into a
hell of rivers, for what the eye can see.  It is glorious to watch
them in their first curve over the rocks.  They come green as a
bank of emeralds, but with a fitful, flying color, as though
conscious that in one moment more they would be dashed into spray
and rise into air, pale as driven snow.  The vapor rises high into
the air, and is gathered there, visible always as a permanent white
cloud over the cataract; but the bulk of the spray which fills the
lower hollow of that horseshoe is like a tumult of snow.  This you
will not fully see from your seat on the rail.  The head of it
rises ever and anon out of that caldron below, but the caldron
itself will be invisible.  It is ever so far down---far as your own
imagination can sink it.  But your eyes will rest full upon the
curve of the waters.  The shape you will be looking at is that of a
horseshoe, but of a horseshoe miraculously deep from toe to heel;
and this depth becomes greater as you sit there.  That which at
first was only great and beautiful becomes gigantic and sublime,
till the mind is at loss to find an epithet for its own use.  To
realize Niagara, you must sit there till you see nothing else than
that which you have come to see.  You will hear nothing else, and
think of nothing else.  At length you will be at one with the
tumbling river before you.  You will find yourself among the waters
as though you belonged to them.  The cool, liquid green will run
through your veins, and the voice of the cataract will be the
expression of your own heart.  You will fall as the bright waters
fall, rushing down into your new world with no hesitation and with
no dismay; and you will rise again as the spray rises, bright,
beautiful, and pure.  Then you will flow away in your course to the
uncompassed, distant, and eternal ocean.

When this state has been reached and has passed away, you may get
off your rail and mount the tower.  I do not quite approve of that
tower, seeing that it has about it a gingerbread air, and reminds
one of those well-arranged scenes of romance in which one is told
that on the left you turn to the lady's bower, price sixpence; and
on the right ascend to the knight's bed, price sixpence more, with
a view of the hermit's tomb thrown in.  But nevertheless the tower
is worth mounting, and no money is charged for the use of it.  It
is not very high, and there is a balcony at the top on which some
half dozen persons may stand at ease.  Here the mystery is lost,
but the whole fall is seen.  It is not even at this spot brought so
fully before your eye, made to show itself in so complete and
entire a shape, as it will do when you come to stand near to it on
the opposite or Canadian shore.  But I think that it shows itself
more beautifully.  And the form of the cataract is such that here,
on Goat Island, on the American side, no spray will reach you,
although you are absolutely over the waters.  But on the Canadian
side, the road as it approaches the fall is wet and rotten with
spray, and you, as you stand close upon the edge, will be wet also.
The rainbows as they are seen through the rising cloud---for the
sun's rays as seen through these waters show themselves in a bow,
as they do when seen through rain---are pretty enough, and are
greatly loved.  For myself, I do not care for this prettiness at
Niagara.  It is there, but I forget it, and do not mind how soon it
is forgotten.

But we are still on the tower; and here I must declare that though
I forgive the tower, I cannot forgive the horrid obelisk which has
latterly been built opposite to it, on the Canadian side, up above
the fall; built apparently---for I did not go to it---with some
camera-obscura intention for which the projector deserves to be put
in Coventry by all good Christian men and women.  At such a place
as Niagara tasteless buildings, run up in wrong places with a view
to money making, are perhaps necessary evils.  It may be that they
are not evils at all; that they give more pleasure than pain,
seeing that they tend to the enjoyment of the multitude.  But there
are edifices of this description which cry aloud to the gods by the
force of their own ugliness and malposition.  As to such, it may be
said that there should somewhere exist a power capable of crushing
them in their birth.  This new obelisk, or picture-building at
Niagara, is one of such.

And now we will cross the water, and with this object will return
by the bridge out of Goat Island, on the main land of the American
side.  But as we do so, let me say that one of the great charms of
Niagara consists in this: that over and above that one great object
of wonder and beauty, there is so much little loveliness---%
loveliness especially of water I mean.  There are little rivulets
running here and there over little falls, with pendent boughs above
them, and stones shining under their shallow depths.  As the
visitor stands and looks through the trees, the rapids glitter
before him, and then hide themselves behind islands.  They glitter
and sparkle in far distances under the bright foliage, till the
remembrance is lost, and one knows not which way they run.  And
then the river below, with its whirlpool,---but we shall come to
that by-and-by, and to the mad voyage which was made down the
rapids by that mad captain who ran the gantlet of the waters at the
risk of his own life, with fifty to one against him, in order that
he might save another man's property from the sheriff.

The readiest way across to Canada is by the ferry; and on the
American side this is very pleasantly done.  You go into a little
house, pay twenty cents, take a seat on a wooden car of wonderful
shape, and on the touch of a spring find yourself traveling down an
inclined plane of terrible declivity, and at a very fast rate.  You
catch a glance of the river below you, and recognize the fact that
if the rope by which you are held should break, you would go down
at a very fast rate indeed, and find your final resting-place in
the river.  As I have gone down some dozen times, and have come to
no such grief, I will not presume that you will be less lucky.
Below there is a boat generally ready.  If it be not there, the
place is not chosen amiss for a rest of ten minutes, for the lesser
fall is close at hand, and the larger one is in full view.  Looking
at the rapidity of the river, you will think that the passage must
be dangerous and difficult.  But no accidents ever happen, and the
lad who takes you over seems to do it with sufficient ease.  The
walk up the hill on the other side is another thing.  It is very
steep, and for those who have not good locomotive power of their
own, will be found to be disagreeable.  In the full season,
however, carriages are generally waiting there.  In so short a
distance I have always been ashamed to trust to other legs than my
own, but I have observed that Americans are always dragged up.  I
have seen single young men of from eighteen to twenty-five, from
whose outward appearance no story of idle, luxurious life can be
read, carried about alone in carriages over distances which would
be counted as nothing by any healthy English lady of fifty.  None
but the old invalids should require the assistance of carriages in
seeing Niagara, but the trade in carriages is to all appearance the
most brisk trade there.

Having mounted the hill on the Canada side, you will walk on toward
the falls.  As I have said before, you will from this side look
directly into the full circle of the upper cataract, while you will
have before you, at your left hand, the whole expanse of the lesser
fall.  For those who desire to see all at a glance, who wish to
comprise the whole with their eyes, and to leave nothing to be
guessed, nothing to be surmised, this no doubt is the best point of
view.

You will be covered with spray as you walk up to the ledge of
rocks, but I do not think that the spray will hurt you.  If a man
gets wet through going to his daily work, cold, catarrh, cough, and
all their attendant evils, may be expected; but these maladies
usually spare the tourist.  Change of air, plenty of air,
excellence of air, and increased exercise, make these things
powerless.  I should therefore bid you disregard the spray.  If,
however, you are yourself of a different opinion, you may hire a
suit of oil-cloth clothes for, I believe, a quarter of a dollar.
They are nasty of course, and have this further disadvantage, that
you become much more wet having them on than you would be without
them.

Here, on this side, you walk on to the very edge of the cataract,
and, if your tread be steady and your legs firm, you dip your foot
into the water exactly at the spot where the thin outside margin of
the current reaches the rocky edge and jumps to join the mass of
the fall.  The bed of white foam beneath is certainly seen better
here than elsewhere, and the green curve of the water is as bright
here as when seen from the wooden rail across.  But nevertheless I
say again that that wooden rail is the one point from whence
Niagara may be best seen aright.

Close to the cataract, exactly at the spot from whence in former
days the Table Rock used to project from the land over the boiling
caldron below, there is now a shaft, down which you will descend to
the level of the river, and pass between the rock and the torrent.
This Table Rock broke away from the cliff and fell, as up the whole
course of the river the seceding rocks have split and fallen from
time to time through countless years, and will continue to do till
the bed of the upper lake is reached.  You will descend this shaft,
taking to yourself or not taking to yourself a suit of oil-clothes
as you may think best.  I have gone with and without the suit, and
again recommend that they be left behind.  I am inclined to think
that the ordinary payment should be made for their use, as
otherwise it will appear to those whose trade it is to prepare them
that you are injuring them in their vested rights.

Some three years since I visited Niagara on my way back to England
from Bermuda, and in a volume of travels which I then published I
endeavored to explain the impression made upon me by this passage
between the rock and the waterfall.  An author should not quote
himself; but as I feel myself bound, in writing a chapter specially
about Niagara, to give some account of this strange position, I
will venture to repeat my own words.

In the spot to which I allude the visitor stands on a broad, safe
path, made of shingles, between the rock over which the water
rushes and the rushing water.  He will go in so far that the spray,
rising back from the bed of the torrent, does not incommode him.
With this exception, the farther he can go in the better; but
circumstances will clearly show him the spot to which he should
advance.  Unless the water be driven in by a very strong wind, five
yards make the difference between a comparatively dry coat and an
absolutely wet one.  And then let him stand with his back to the
entrance, thus hiding the last glimmer of the expiring day.  So
standing, he will look up among the falling waters, or down into
the deep, misty pit, from which they re-ascend in almost as
palpable a bulk.  The rock will be at his right hand, high and
hard, and dark and straight, like the wall of some huge cavern,
such as children enter in their dreams.  For the first five minutes
he will be looking but at the waters of a cataract---at the waters,
indeed, of such a cataract as we know no other, and at their
interior curves which elsewhere we cannot see.  But by-and-by all
this will change.  He will no longer be on a shingly path beneath a
waterfall; but that feeling of a cavern wall will grow upon him, of
a cavern deep, below roaring seas, in which the waves are there,
though they do not enter in upon him; or rather, not the waves, but
the very bowels of the ocean.  He will feel as though the floods
surrounded him, coming and going with their wild sounds, and he
will hardly recognize that though among them he is not in them.
And they, as they fall with a continual roar, not hurting the ear,
but musical withal, will seem to move as the vast ocean waters may
perhaps move in their internal currents.  He will lose the sense of
one continued descent, and think that they are passing round him in
their appointed courses.  The broken spray that rises from the
depths below, rises so strongly, so palpably, so rapidly that the
motion in every direction will seem equal.  And, as he looks on,
strange colors will show themselves through the mist; the shades of
gray will become green or blue, with ever and anon a flash of
white; and then, when some gust of wind blows in with greater
violence, the sea-girt cavern will become all dark and black.  Oh,
my friend, let there be no one there to speak to thee then; no, not
even a brother.  As you stand there speak only to the waters.

Two miles below the falls the river is crossed by a suspension
bridge of marvelous construction.  It affords two thoroughfares,
one above the other.  The lower road is for carriages and horses,
and the upper one bears a railway belonging to the Great Western
Canada Line.  The view from hence, both up and down the river, is
very beautiful, for the bridge is built immediately over the first
of a series of rapids.  One mile below the bridge these rapids end
in a broad basin called the whirlpool, and, issuing out of this,
the current turns to the right through a narrow channel overhung by
cliffs and trees, and then makes its way down to Lake Ontario with
comparative tranquillity.

But I will beg you to take notice of those rapids from the bridge,
and to ask yourself what chance of life would remain to any ship,
craft, or boat required by destiny to undergo navigation beneath
the bridge and down into that whirlpool.  Heretofore all men would
have said that no chance of life could remain to so ill-starred a
bark.  The navigation, however, has been effected.  But men used to
the river still say that the chances would be fifty to one against
any vessel which should attempt to repeat the experiment.

The story of that wondrous voyage was as follows:  A small steamer,
called the Maid of the Mist, was built upon the river, between the
falls and the rapids, and was used for taking adventurous tourists
up amid the spray as near to the cataract as was possible.  ``The
Maid of the Mist plied in this way for a year or two, and was, I
believe, much patronized during the season.  But in the early part
of last summer an evil time had come.  Either the Maid got into
debt, or her owner had embarked in other and less profitable
speculations.  At any rate, he became subject to the law, and
tidings reached him that the sheriff would seize the Maid.  On most
occasions the sheriff is bound to keep such intentions secret,
seeing that property is movable, and that an insolvent debtor will
not always await the officers of justice.  But with the poor Maid
there was no need of such secrecy.  There was but a mile or so of
water on which she could ply, and she was forbidden by the nature
of her properties to make any way upon land, The sheriff's prey,
therefore, was easy, and the poor Maid was doomed.

In any country in the world but America such would have been the
case; but an American would steam down Phlegethon to save his
property from the sheriff---he would steam down Phlegethon, or get
some one else to do it for him.  Whether or no, in this case, the
captain of the boat was the proprietor, or whether, as I was told,
he was paid for the job, I do not know.  But he determined to run
the rapids, and he procured two others to accompany him in the
risk.  He got up his steam, and took the Maid up amid the spray
according to his custom.  Then, suddenly turning on his course, he,
with one of his companions, fixed himself at the wheel, while the
other remained at his engine.  I wish I could look into the mind of
that man, and understand what his thoughts were at that moment---%
what were his thoughts and what his beliefs.  As to one of the men,
I was told that he was carried down not knowing what he was about
to do but I am inclined to believe that all the three were joined
together in the attempt.

I was told by a man who saw the boat pass under the bridge that she
made one long leap down, as she came thither; that her funnel was
at once knocked flat on the deck by the force of the blow; that the
waters covered her from stem to stern; and that then she rose
again, and skimmed into the whirlpool a mile below.  When there she
rode with comparative ease upon the waters, and took the sharp turn
round into the river below without a struggle.  The feat was done,
and the Maid was rescued from the sheriff.  It is said that she was
sold below at the mouth of the river, and carried from thence over
Lake Ontario, and down the St. Lawrence to Quebec.



\chapter{North and West}


From Niagara we determined to proceed Northwest---as far to the
Northwest as we could go with any reasonable hope of finding
American citizens in a state of political civilization, and perhaps
guided also in some measure by our hopes as to hotel accommodation.
Looking to these two matters, we resolved to get across to the
Mississippi, and to go up that river as far as the town of St. Paul
and the Falls of St. Anthony, which are some twelve miles above the
town; then to descend the river as far as the States of Iowa on the
west and Illinois on the east; and to return eastward through
Chicago and the large cities on the southern shores of Lake Erie,
from whence we would go across to Albany, the capital of New York
state, and down the Hudson to New York, the capital of the Western
World.  For such a journey, in which scenery was one great object,
we were rather late, as we did not leave Niagara till the 10th of
October; but though the winters are extremely cold through all this
portion of the American continent---fifteen, twenty, and even
twenty-five degrees below zero being an ordinary state of the
atmosphere in latitudes equal to those of Florence, Nice, and
Turin---nevertheless the autumns are mild, the noonday being always
warm, and the colors of the foliage are then in all their glory.  I
was also very anxious to ascertain, if it might be in my power to
do so, with what spirit or true feeling as to the matter the work
of recruiting for the now enormous army of the States was going on
in those remote regions.  That men should be on fire in Boston and
New York, in Philadelphia and along the borders of secession, I
could understand.  I could understand also that they should be on
fire throughout the cotton, sugar, and rice plantations of the
South.  But I could hardly understand that this political fervor
should have communicated itself to the far-off farmers who had
thinly spread themselves over the enormous wheat-growing districts
of the Northwest.  St. Paul, the capital of Minnesota, is nine
hundred miles directly north of St. Louis, the most northern point
to which slavery extends in the Western States of the Union; and
the farming lands of Minnesota stretch away again for some hundreds
of miles north and west of St. Paul.  Could it be that those scanty
and far-off pioneers of agriculture---those frontier farmers, who
are nearly one-half German and nearly the other half Irish, would
desert their clearings and ruin their chances of progress in the
world for distant wars of which the causes must, as I thought, be
to them unintelligible?  I had been told that distance had but lent
enchantment to the view, and that the war was even more popular in
the remote and newly-settled States than in those which have been
longer known as great political bodies.  So I resolved that I would
go and see.

It may be as well to explain here that that great political Union
hitherto called the United States of America may be more properly
divided into three than into two distinct interests, In England we
have long heard of North and South as pitted against each other,
and we have always understood that the Southern politicians, or
Democrats, have prevailed over the Northern politicians, or
Republicans, because they were assisted in their views by Northern
men of mark who have held Southern principles---that is, by Northern
men who have been willing to obtain political power by joining
themselves to the Southern party.  That, as far as I can
understand, has been the general idea in England, and in a broad
way it has been true, But as years have advanced, and as the States
have extended themselves westward, a third large party has been
formed, which sometimes rejoices to call itself The Great West; and
though, at the present time, the West and the North are joined
together against the South, the interests of the North and West are
not, I think, more closely interwoven than are those of the West
and South; and when the final settlement of this question shall be
made, there will doubtless be great difficulty in satisfying the
different aspirations and feelings of two great free-soil
populations.  The North, I think, will ultimately perceive that it
will gain much by the secession of the South; but it will be very
difficult to make the West believe that secession will suit its
views.

I will attempt, in a rough way, to divide the States, as they seem
to divide themselves, into these three parties.  As to the majority
of them, there is no difficulty in locating them; but this cannot
be done with absolute certainty as to some few that lie on the
borders.

New England consists of six States, of which all of course belong
to the North.  They are Maine, New Hampshire, Vermont,
Massachusetts, Rhode Island, and Connecticut---the six States which
should be most dear to England, and in which the political success
of the United States as a nation is to my eyes the most apparent.
But even in them there was till quite of late a strong section so
opposed to the Republican party as to give a material aid to the
South.  This, I think, was particularly so in New Hampshire, from
whence President Pierce came.  He had been one of the Senators from
New Hampshire; and yet to him, as President, is affixed the
disgrace---whether truly affixed or not I do not say---of having
first used his power in secretly organizing those arrangements
which led to secession and assisted at its birth.  In Massachusetts
itself, also, there was a strong Democratic party, of which
Massachusetts now seems to be somewhat ashamed.  Then, to make up
the North, must be added the two great States of New York and
Pennsylvania and the small State of New Jersey.  The West will not
agree even to this absolutely, seeing that they claim all territory
west of the Alleghanies, and that a portion of Pennsylvania and
some part also of New York lie westward of that range; but, in
endeavoring to make these divisions ordinarily intelligible, I may
say that the North consists of the nine States above named.  But
the North will also claim Maryland and Delaware, and the eastern
half of Virginia.  The North will claim them, though they are
attached to the South by joint participation in the great social
institution of slavery---for Maryland, Delaware, and Virginia are
slave States---and I think that the North will ultimately make good
its claim.  Maryland and Delaware lie, as it were, behind the
capital, and Eastern Virginia is close upon the capital.  And these
regions are not tropical in their climate or influences.  They are
and have been slave States, but will probably rid themselves of
that taint, and become a portion of the free North.

The Southern or slave States, properly so called, are easily
defined.  They are Texas, Louisiana, Arkansas, Mississippi,
Alabama, Florida, Georgia, South Carolina, and North Carolina.  The
South will also claim Tennessee, Kentucky, Missouri, Virginia,
Delaware, and Maryland, and will endeavor to prove its right to the
claim by the fact of the social institution being the law of the
land in those States.  Of Delaware, Maryland, and Eastern Virginia,
I have already spoken.  Western Virginia is, I think, so little
tainted with slavery that, as she stands even at present, she
properly belongs to the West.  As I now write, the struggle is
going on in Kentucky and Missouri.  In Missouri the slave
population is barely more than a tenth of the whole, while in South
Carolina and Mississippi it is more than half.  And, therefore, I
venture to count Missouri among the Western States, although
slavery is still the law of the land within its borders.  It is
surrounded on three sides by free States of the West, and its soil,
let us hope, must become free.  Kentucky I must leave as doubtful,
though I am inclined to believe that slavery will be abolished
there also.  Kentucky, at any rate, will never throw in its lot
with the Southern States.  As to Tennessee, it seceded heart and
soul, and I fear that it must be accounted as Southern, although
the Northern army has now, in May, 1862, possessed itself of the
greater part of the State.

To the great West remains an enormous territory, of which, however,
the population is as yet but scanty; though perhaps no portion of
the world has increased so fast in population as have these Western
States.  The list is as follows: Ohio, Indiana, Illinois, Michigan,
Wisconsin, Minnesota, Iowa, Kansas to which I would add Missouri,
and probably the Western half of Virginia.  We have then to account
for the two already admitted States on the Pacific, California and
Oregon, and also for the unadmitted Territories, Dacotah, Nebraska,
Washington, Utah, New Mexico, Colorado, and Nevada.  I should be
refining too much for my present very general purpose, if I were to
attempt to marshal these huge but thinly-populated regions in
either rank.  Of California and Oregon it may probably be said that
it is their ambition to form themselves into a separate division---a
division which may be called the farther West.

I know that all statistical statements are tedious, and I believe
that but few readers believe them.  I will, however, venture to
give the populations of these States in the order I have named
them, seeing that power in America depends almost entirely on
population.  The census of 1860 gave the following results:---%


In the North:

{\small
\begin{verbatim}
Maine               619,000
New Hampshire       326,872
Vermont             325,827
Massachusetts     1,231,494
Rhode Island        174,621
Connecticut         460,670
New York          3,851,563
Pennsylvania      2,916,018
New Jersey          676,034
                 ----------
Total            10,582,099
\end{verbatim}}

In the South, the population of which must be divided into free and
slave:


{\small
\begin{verbatim}
                      Free.      Slave.      Total.

Texas               415,999     184,956     600,955
Louisiana           354,245     312,186     666,431
Arkansas            331,710     109,065     440,775
Mississippi         407,051     479,607     886,658
Alabama             520,444     435,473     955,917
Florida              81,885      63,809     145,694
Georgia             615,366     467,461   1,082,827
South Carolina      308,186     407,185     715,371
North Carolina      679,965     328,377   1,008,342
Tennessee           859,578     287,112   1,146,690
                  ---------   ---------   ---------
Total             4,574,429   3,075,231   7,649,660
\end{verbatim}}


in the doubtful States:


{\small
\begin{verbatim}
                      Free.      Slave.      Total.

Maryland            646,183      85,382     731,565
Delaware            110,548       1,805     112,353
Virginia          1,097,373     495,826   1,593,199
Kentucky            920,077     225,490   1,145,567
                  ---------     -------   ---------
Total             2,774,181     808,503   3,582,684
\end{verbatim}}


In the West:

{\small
\begin{verbatim}
Ohio              2,377,917
Indiana           1,350,802
Illinois          1,691,238
Michigan            754,291
Wisconsin           763,485
Minnesota           172,796
Iowa                682,002
Kansas              143,645
Missouri          1,204,214*
                  ---------
Total             9,140,390
\end{verbatim}}


* Of which number, in Missouri, 115,619 are slaves.


To these must be added, to make up the population of the United
States as it stood in 1860,---%

{\small
\begin{verbatim}
The separate District of Columbia, in which is
  included Washington, the seat of the Federal
  Government         75,321
California          384,770
Oregon               52,566
The Territories of---
  Dacotah             4,839
  Nebraska           28,892
  Washington         11,624
  Utah               49,000
  New Mexico         98,024
  Colorado           34,197
  Nevada              6,857
                    -------
Total               741,090
\end{verbatim}}


And thus the total population may be given as follows:---%

{\small
\begin{verbatim}
North                             10,582,099
South                              7,649,660
Doubtful                           3,582,684
West                               9,140,390
Outlying States and Territories      741,090
                                  ----------
Total                             31,695,923
\end{verbatim}}


Each of the three interests would consider itself wronged by the
division above made, but the South would probably be the loudest in
asserting its grievance.  The South claims all the slave States,
and would point to secession in Virginia to justify such claim, and
would point also to Maryland and Baltimore, declaring that
secession would be as strong there as at New Orleans, if secession
were practicable.  Maryland and Baltimore lie behind Washington,
and are under the heels of the Northern troops, so that secession
is not practicable; but the South would say that they have seceded
in heart.  In this the South would have some show of reason for its
assertion; but nevertheless I shall best convey a true idea of the
position of these States by classing them as doubtful.  When
secession shall have been accomplished---if ever it be accomplished---%
it will hardly be possible that they should adhere to the South.

It will be seen by the foregoing tables that the population of the
West is nearly equal to that of the North, and that therefore
Western power is almost as great as Northern.  It is almost as
great already, and as population in the West increases faster than
it does in the North, the two will soon be equalized.  They are
already sufficiently on a par to enable them to fight on equal
terms, and they will be prepared for fighting---political fighting,
if no other---as soon as they have established their supremacy over
a common enemy.

While I am on the subject of population I should explain---though
the point is not one which concerns the present argument---that the
numbers given, as they regard the South, include both the whites
and the blacks, the free men and the slaves.  The political power
of the South is of course in the hands of the white race only, and
the total white population should therefore be taken as the number
indicating the Southern power.  The political power of the South,
however, as contrasted with that of the North, has, since the
commencement of the Union, been much increased by the slave
population.  The slaves have been taken into account in determining
the number of representatives which should be sent to Congress by
each State.  That number depends on the population but it was
decided in 1787 that in counting up the number of representatives
to which each State should be held to be entitled, five slaves
should represent three white men.  A Southern population,
therefore, of five thousand free men and five thousand slaves would
claim as many representatives as a Northern population of eight
thousand free men, although the voting would be confined to the
free population.  This has ever since been the law of the United
States.

The Western power is nearly equal to that of the North, and this
fact, somewhat exaggerated in terms, is a frequent boast in the
mouths of Western men.  ``We ran Fremont for President,'' they say,
``and had it not been for Northern men with Southern principles, we
should have put him in the White House instead of the traitor
Buchanan.  If that had been done there would have been no
secession.''  How things might have gone had Fremont been elected in
lieu of Buchanan, I will not pretend to say; but the nature of the
argument shows the difference that exists between Northern and
Western feeling.  At the time that I was in the West, General
Fremont was the great topic of public interest.  Every newspaper
was discussing his conduct, his ability as a soldier, his energy,
and his fate.  At that time General McClellan was in command at
Washington on the Potomac, it being understood that he held his
power directly under the President, free from the exercise of
control on the part of the veteran General Scott, though at that
time General Scott had not actually resigned his position as head
of the army.  And General Fremont, who some five years before had
been ``run'' for President by the Western States, held another
command of nearly equal independence in Missouri.  He had been put
over General Lyon in the Western command, and directly after this
General Lyon had fallen in battle at Springfield, in the first
action in which the opposing armies were engaged in the West.
General Fremont at once proceeded to carry matters with a very high
hand, On the 30th of August, 1861, he issued a proclamation by
which he declared martial law at St. Louis, the city at which he
held his headquarters, and indeed throughout the State of Missouri
generally.  In this proclamation he declared his intention of
exercising a severity beyond that ever threatened, as I believe, in
modern warfare.  He defines the region presumed to be held by his
army of occupation, drawing his lines across the State, and then
declares ``that all persons who shall be taken with arms in their
hands within those lines shall be tried by court-martial, and if
found guilty will be shot.''  He then goes on to say that he will
confiscate all the property of persons in the State who shall have
taken up arms against the Union, or shall have taken part with the
enemies of the Union, and that he will make free all slaves
belonging to such persons.  This proclamation was not approved at
Washington, and was modified by the order of the President.  It was
understood also that he issued orders for military expenditure
which were not recognized at Washington, and men began to
understand that the army in the West was gradually assuming that
irresponsible military position which, in disturbed countries and
in times of civil war, has so frequently resulted in a military
dictatorship.  Then there arose a clamor for the removal of General
Fremont.  A semi-official account of his proceedings, which had
reached Washington from an officer under his command, was made
public, and also the correspondence which took place on the subject
between the President and General Fremont's wife.  The officer in
question was thereupon placed under arrest, but immediately
released by orders from Washington.  He then made official
complaint of his general, sending forward a list of charges, in
which Fremont was accused of rashness, incompetency, want of
fidelity of the interests of the government, and disobedience to
orders from headquarters.  After awhile the Secretary of War
himself proceeded from Washington to the quarters of General
Fremont at St. Louis, and remained there for a day or two making,
or pretending to make, inquiry into the matter.  But when he
returned he left the General still in command.  During the whole
month of October the papers were occupied in declaring in the
morning that General Fremont had been recalled from his command,
and in the evening that he was to remain.  In the mean time they
who befriended his cause, and this included the whole West, were
hoping from day to day that he would settle the matter for himself
and silence his accusers, by some great military success.  General
Price held the command opposed to him, and men said that Fremont
would sweep General Price and his army down the valley of the
Mississippi into the sea.  But General Price would not be so swept,
and it began to appear that a guerrilla warfare would prevail; that
General Price, if driven southward, would reappear behind the backs
of his pursuers, and that General Fremont would not accomplish all
that was expected of him with that rapidity for which his friends
had given him credit.  So the newspapers still went on waging the
war, and every morning General Fremont was recalled, and every
evening they who had recalled him were shown up as having known
nothing of the matter.

``Never mind; he is a pioneer man, and will do a'most anything he
puts his hand to,'' his friends in the West still said.  ``He
understands the frontier.''  Understanding the frontier is a great
thing in Western America, across which the vanguard of civilization
continues to march on in advance from year to year.  ``And it's he
that is bound to sweep slavery from off the face of this continent.
He's the man, and he's about the only man.''  I am not qualified to
write the life of General Fremont, and can at present only make
this slight reference to the details of his romantic career.  That
it has been full of romance, and that the man himself is endued
with a singular energy, and a high, romantic idea of what may be
done by power and will, there is no doubt.  Five times he has
crossed the Continent of North America from Missouri to Oregon and
California, enduring great hardships in the service of advancing
civilization and knowledge.  That he has considerable talent,
immense energy, and strong self-confidence, I believe.  He is a
frontier man---one of those who care nothing for danger, and who
would dare anything with the hope of accomplishing a great career.
But I have never heard that he has shown any practical knowledge of
high military matters.  It may be doubted whether a man of this
stamp is well fitted to hold the command of a nation's army for
great national purposes.  May it not even be presumed that a man of
this class is of all men the least fitted for such a work?  The
officer required should be a man with two specialties---a specialty
for military tactics and a specialty for national duty.  The army
in the West was far removed from headquarters in Washington, and it
was peculiarly desirable that the general commanding it should be
one possessing a strong idea of obedience to the control of his own
government.  Those frontier capabilities---that self-dependent
energy for which his friends gave Fremont, and probably justly gave
him, such unlimited credit---are exactly the qualities which are
most dangerous in such a position.

I have endeavored to explain the circumstances of the Western
command in Missouri as they existed at the time when I was in the
Northwestern States, in order that the double action of the North
and West may be understood.  I, of course, was not in the secret of
any official persons; but I could not but feel sure that the
government in Washington would have been glad to have removed
Fremont at once from the command, had they not feared that by so
doing they would have created a schism, as it were, in their own
camp, and have done much to break up the integrity or oneness of
Northern loyalty.  The Western people almost to a man desired
abolition.  The States there were sending out their tens of
thousands of young men into the army with a prodigality as to their
only source of wealth which they hardly recognized themselves,
because this to them was a fight against slavery.  The Western
population has been increased to a wonderful degree by a German
infusion---so much so that the Western towns appear to have been
peopled with Germans.  I found regiments of volunteers consisting
wholly of Germans.  And the Germans are all abolitionists.  To all
the men of the West the name of Fremont is dear.  He is their hero
and their Hercules.  He is to cleanse the stables of the Southern
king, and turn the waters of emancipation through the foul stalls
of slavery.  And therefore, though the Cabinet in Washington would
have been glad for many reasons to have removed Fremont in October
last, it was at first scared from committing itself to so strong a
measure.  At last, however, the charges made against him were too
fully substantiated to allow of their being set on one side; and
early in November, 1861, he was superseded.  I shall be obliged to
allude again to General Fremont's career as I go on with my
narrative.

At this time the North was looking for a victory on the Potomac;
but they were no longer looking for it with that impatience which
in the summer had led to the disgrace at Bull's Run.  They had
recognized the fact that their troops must be equipped, drilled,
and instructed; and they had also recognized the perhaps greater
fact that their enemies were neither weak, cowardly, nor badly
officered.  I have always thought that the tone and manner with
which the North bore the defeat at Bull's Run was creditable to it.
It was never denied, never explained away, never set down as
trifling.  ``We have been whipped,'' was what all Northerners said;
``we've got an almighty whipping, and here we are.''  I have heard
many Englishmen complain of this---saying that the matter was taken
almost as a joke, that no disgrace was felt, and that the licking
was owned by a people who ought never to have allowed that they had
been licked.  To all this, however, I demur.  Their only chance of
speedy success consisted in their seeing and recognizing the truth.
Had they confessed the whipping, and then sat down with their hands
in their pockets---had they done as second-rate boys at school will
do, declare that they had been licked, and then feel that all the
trouble is over---they would indeed have been open to reproach.  The
old mother across the water would in such case have disowned her
son.  But they did the very reverse of this.  ``I have been
whipped,'' Jonathan said, and he immediately went into training
under a new system for another fight.

And so all through September and October the great armies on the
Potomac rested comparatively in quiet---the Northern forces drawing
to themselves immense levies.  The general confidence in McClellan
was then very great; and the cautious measures by which he
endeavored to bring his vast untrained body of men under discipline
were such as did at that time recommend themselves to most military
critics.  Early in September the Northern party obtained a
considerable advantage by taking the fort at Cape Hatteras, in
North Carolina, situated on one of those long banks which lie along
the shores of the Southern States; but, toward the end of October,
they experienced a considerable reverse in an attack which was made
on the secessionists by General Stone, and in which Colonel Baker
was killed.  Colonel Baker had been Senator for Oregon, and was
well known as an orator.  Taking all things together, however,
nothing material had been done up to the end of October; and at
that time Northern men were waiting---not perhaps impatiently,
considering the great hopes and perhaps great fears which filled
their hearts, but with eager expectation---for some event of which
they might talk with pride.

The man to whom they had trusted all their hopes was young for so
great a command.  I think that, at this time, (October, 1861,)
General McClellan was not yet thirty-five.  He had served, early in
life, in the Mexican war, having come originally from Pennsylvania,
and having been educated at the military college at West Point.
During our war with Russia he was sent to the Crimea by his own
government, in conjunction with two other officers of the United
States army, that they might learn all that was to be learned there
as to military tactics, and report especially as to the manner in
which fortifications were made and attacked.  I have been informed
that a very able report was sent in by them to the government on
their return, and that this was drawn up by McClellan.  But in
America a man is not only a soldier, or always a soldier, nor is he
always a clergyman if once a clergyman: he takes a spell at
anything suitable that may be going.  And in this way McClellan
was, for some years, engaged on the Central Illinois Railway, and
was for a considerable time the head manager of that concern.  We
all know with what suddenness he rose to the highest command in the
army immediately after the defeat at Bull's Run.

I have endeavored to describe what were the feelings of the West in
the autumn of 1861 with regard to the war.  The excitement and
eagerness there were very great, and they were perhaps as great in
the North.  But in the North the matter seemed to me to be regarded
from a different point of view.  As a rule, the men of the North
are not abolitionists.  It is quite certain that they were not so
before secession began.  They hate slavery as we in England hate
it; but they are aware, as also are we, that the disposition of
four million of black men and women forms a question which cannot
be solved by the chivalry of any modern Orlando.  The property
invested in these four million slaves forms the entire wealth of
the South.  If they could be wafted by a philanthropic breeze back
to the shores of Africa---a breeze of which the philanthropy would
certainly not be appreciated by those so wafted---the South would be
a wilderness.  The subject is one as full of difficulty as any with
which the politicians of these days are tormented.  The Northerners
fully appreciate this, and, as a rule, are not abolitionists in the
Western sense of the word.  To them the war is recommended by
precisely those feelings which animated us when we fought for our
colonies---when we strove to put down American independence.
Secession is rebellion against the government, and is all the more
bitter to the North because that rebellion broke out at the first
moment of Northern ascendency.  ``We submitted,'' the North says, ``to
Southern Presidents, and Southern statesmen, and Southern councils,
because we obeyed the vote of the people.  But as to you---the voice
of the people is nothing in your estimation!  At the first moment
in which the popular vote places at Washington a President with
Northern feelings, you rebel.  We submitted in your days; and, by
Heaven! you shall submit in ours.  We submitted loyally, through
love of the law and the Constitution.  You have disregarded the law
and thrown over the Constitution.  But you shall be made to submit,
as a child is made to submit to its governor.''

It must also be remembered that on commercial questions the North
and the West are divided.  The Morrill tariff is as odious to the
West as it is to the South.  The South and West are both
agricultural productive regions, desirous of sending cotton and
corn to foreign countries, and of receiving back foreign
manufactures on the best terms.  But the North is a manufacturing
country---a poor manufacturing country as regards excellence of
manufacture---and therefore the more anxious to foster its own
growth by protective laws.  The Morrill tariff is very injurious to
the West, and is odious there.  I might add that its folly has
already been so far recognized even in the North as to make it very
generally odious there also.

So much I have said endeavoring to make it understood how far the
North and West were united in feeling against the South in the
autumn of 1861, and how far there existed between them a diversity
of interests.



\chapter{From Niagara to the Mississippi}


From Niagara we went by the Canada Great Western Railway to
Detroit, the big city of Michigan.  It is an American institution
that the States should have a commercial capital---or what I call
their big city---as well as a political capital, which may, as a
rule, be called the State's central city.  The object in choosing
the political capital is average nearness of approach from the
various confines of the State but commerce submits to no such
Procrustean laws in selecting her capitals and consequently she has
placed Detroit on the borders of Michigan, on the shore of the neck
of water which joins Lake Huron to Lake Erie, through which all the
trade must flow which comes down from Lakes Michigan, Superior, and
Huron on its way to the Eastern States and to Europe.  We had
thought of going from Buffalo across Lake Erie to Detroit; but we
found that the better class of steamers had been taken off the
waters for the winter.  And we also found that navigation among
these lakes is a mistake whenever the necessary journey can be
taken by railway.  Their waters are by no means smooth, and then
there is nothing to be seen.  I do not know whether others may have
a feeling, almost instinctive, that lake navigation must be
pleasant---that lakes must of necessity be beautiful.  I have such a
feeling, but not now so strongly as formerly.  Such an idea should
be kept for use in Europe, and never brought over to America with
other traveling gear.  The lakes in America are cold, cumbrous,
uncouth, and uninteresting---intended by nature for the conveyance
of cereal produce, but not for the comfort of traveling men and
women.  So we gave up our plan of traversing the lake, and, passing
back into Canada by the suspension bridge at Niagara, we reached
the Detroit River at Windsor by the Great Western line, and passed
thence by the ferry into the City of Detroit.

In making this journey at night we introduced ourselves to the
thoroughly American institution of sleeping-cars---that is, of cars
in which beds are made up for travelers.  The traveler may have a
whole bed, or half a bed, or no bed at all, as he pleases, paying a
dollar or half a dollar extra should he choose the partial or full
fruition of a couch.  I confess I have always taken a delight in
seeing these beds made up, and consider that the operations of the
change are generally as well executed as the manoeuvres of any
pantomime at Drury Lane.  The work is usually done by negroes or
colored men, and the domestic negroes of America are always light-
handed and adroit.  The nature of an American car is no doubt known
to all men.  It looks as far removed from all bed-room
accommodation as the baker's barrow does from the steam engine into
which it is to be converted by Harlequin's wand.  But the negro
goes to work much more quietly than the Harlequin; and for every
four seats in the railway car he builds up four beds almost as
quickly as the hero of the pantomime goes through his performance.
The great glory of the Americans is in their wondrous contrivances---%
in their patent remedies for the usually troublous operations of
life.  In their huge hotels all the bell ropes of each house ring
on one bell only; but a patent indicator discloses a number, and
the whereabouts of the ringer is shown.  One fire heats every room,
passage, hall, and cupboard, and does it so effectually that the
inhabitants are all but stifled.  Soda-water bottles open
themselves without any trouble of wire or strings.  Men and women
go up and down stairs without motive power of their own.  Hot and
cold water are laid on to all the chambers; though it sometimes
happens that the water from both taps is boiling, and that, when
once turned on, it cannot be turned off again by any human energy.
Everything is done by a new and wonderful patent contrivance; and
of all their wonderful contrivances, that of their railroad beds is
by no means the least.  For every four seats the negro builds up
four beds---that is, four half beds, or accommodation for four
persons.  Two are supposed to be below, on the level of the
ordinary four seats, and two up above on shelves which are let down
from the roof.  Mattresses slip out from one nook and pillows from
another.  Blankets are added, and the bed is ready.  Any over-
particular individual---an islander, for instance, who hugs his
chains---will generally prefer to pay the dollar for the double
accommodation.  Looking at the bed in the light of a bed---taking,
as it were, an abstract view of it---or comparing it with some other
bed or beds with which the occupant may have acquaintance, I cannot
say that it is in all respects perfect.  But distances are long in
America; and he who declines to travel by night will lose very much
time.  He who does so travel will find the railway bed a great
relief.  I must confess that the feeling of dirt, on the following
morning, is rather oppressive.

From Windsor, on the Canada side, we passed over to Detroit, in the
State of Michigan, by a steam ferry.  But ferries in England and
ferries in America are very different.  Here, on this Detroit
ferry, some hundred of passengers, who were going forward from the
other side without delay, at once sat down to breakfast.  I may as
well explain the way in which disposition is made of one's luggage
as one takes these long journeys.  The traveler, when he starts,
has his baggage checked.  He abandons his trunk---generally a box,
studded with nails, as long as a coffin and as high as a linen
chest---and, in return for this, he receives an iron ticket with a
number on it.  As he approaches the end of his first installment of
travel and while the engine is still working its hardest, a man
comes up to him, bearing with him, suspended on a circular bar, an
infinite variety of other checks.  The traveler confides to this
man his wishes, and, if he be going farther without delay,
surrenders his check and receives a counter-check in return.  Then,
while the train is still in motion, the new destiny of the trunk is
imparted to it.  But another man, with another set of checks, also
comes the way, walking leisurely through the train as he performs
his work.  This is the minister of the hotel-omnibus institution.
His business is with those who do not travel beyond the next
terminus.  To him, if such be your intention, you make your
confidence, giving up your tallies, and taking other tallies by way
of receipt; and your luggage is afterward found by you in the hall
of your hotel.  There is undoubtedly very much of comfort in this;
and the mind of the traveler is lost in amazement as he thinks of
the futile efforts with which he would struggle to regain his
luggage were there no such arrangement.  Enormous piles of boxes
are disclosed on the platform at all the larger stations, the
numbers of which are roared forth with quick voice by some two or
three railway denizens at once.  A modest English voyager, with six
or seven small packages, would stand no chance of getting anything
if he were left to his own devices.  As it is, I am bound to say
that the thing is well done.  I have had my desk with all my money
in it lost for a day, and my black leather bag was on one occasion
sent back over the line.  They, however, were recovered; and, on
the whole, I feel grateful to the check system of the American
railways.  And then, too, one never hears of extra luggage.  Of
weight they are quite regardless.  On two or three occasions an
overwrought official has muttered between his teeth that ten
packages were a great many, and that some of those ``light fixings''
might have been made up into one.  And when I came to understand
that the number of every check was entered in a book, and re-
entered at every change, I did whisper to my wife that she ought to
do without a bonnet box.  The ten, however, went on, and were
always duly protected.  I must add, however, that articles
requiring tender treatment will sometimes reappear a little the
worse from the hardships of their journey.

I have not much to say of Detroit---not much, that is, beyond what I
have to say of all the North.  It is a large, well-built, half-
finished city lying on a convenient waterway, and spreading itself
out with promises of a wide and still wider prosperity.  It has
about it perhaps as little of intrinsic interest as any of those
large Western towns which I visited.  It is not so pleasant as
Milwaukee, nor so picturesque as St. Paul, nor so grand as Chicago,
nor so civilized as Cleveland, nor so busy as Buffalo.  Indeed,
Detroit is neither pleasant nor picturesque at all.  I will not say
that it is uncivilized; but it has a harsh, crude, unprepossessing
appearance.  It has some 70,000 inhabitants, and good accommodation
for shipping.  It was doing an enormous business before the war
began, and, when these troublous times are over, will no doubt
again go ahead.  I do not, however, think it well to recommend any
Englishman to make a special visit to Detroit who may be wholly
uncommercial in his views, and travel in search of that which is
either beautiful or interesting.

From Detroit we continued our course westward across the State of
Michigan, through a country that was absolutely wild till the
railway pierced it, Very much of it is still absolutely wild.  For
miles upon miles the road passes the untouched forest, showing that
even in Michigan the great work of civilization has hardly more
than been commenced.  One thinks of the all but countless
population which is, before long, to be fed from these regions---of
the cities which will grow here, and of the amount of government
which in due time will be required---one can hardly fail to feel
that the division of the United States into separate nationalities
is merely a part of the ordained work of creation as arranged for
the well-being of mankind.  The States already boast of thirty
millions of inhabitants---not of unnoticed and unnoticeable beings
requiring little, knowing little, and doing little, such as are the
Eastern hordes, which may be counted by tens of millions, but of
men and women who talk loudly and are ambitious, who eat beef, who
read and write, and understand the dignity of manhood.  But these
thirty millions are as nothing to the crowds which will grow sleek,
and talk loudly, and become aggressive on these wheat and meat
producing levels.  The country is as yet but touched by the
pioneering hand of population.  In the old countries, agriculture,
following on the heels of pastoral, patriarchal life, preceded the
birth of cities.  But in this young world the cities have come
first.  The new Jasons, blessed with the experience of the Old-
World adventurers, have gone forth in search of their golden
fleeces, armed with all that the science and skill of the East had
as yet produced, and, in settling up their new Colchis, have begun
by the erection of first class hotels and the fabrication of
railroads.  Let the Old World bid them God speed in their work.
Only it would be well if they could be brought to acknowledge from
whence they have learned all that they know.

Our route lay right across the State to a place called Grand Haven,
on Lake Michigan, from whence we were to take boat for Milwaukee, a
town in Wisconsin, on the opposite or western shore of the lake.
Michigan is sometimes called the Peninsular State, from the fact
that the main part of its territory is surrounded by Lakes Michigan
and Huron, by the little Lake St. Clair and by Lake Erie.  It juts
out to the northward from the main land of Indiana and Ohio, and is
circumnavigable on the east, north, and west.  These particulars,
however, refer to a part of the State only; for a portion of it
lies on the other side of Lake Michigan, between that and Lake
Superior.  I doubt whether any large inland territory in the world
is blessed with such facilities of water carriage.

On arriving at Grand Haven we found that there had been a storm on
the lake, and that the passengers from the trains of the preceding
day were still remaining there, waiting to be carried over to
Milwaukee.  The water however---or the sea, as they all call it---was
still very high, and the captain declared his intention of
remaining there that night; whereupon all our fellow-travelers
huddled themselves into the great lake steamboat, and proceeded to
carry on life there as though they were quite at home.  The men
took themselves to the bar-room, and smoked cigars and talked about
the war with their feet upon the counter; and the women got
themselves into rocking-chairs in the saloon, and sat there
listless and silent, but not more listless and silent than they
usually are in the big drawing-rooms of the big hotels.  There was
supper there precisely at six o'clock---beef-steaks, and tea, and
apple jam, and hot cakes, and light fixings, to all which luxuries
an American deems himself entitled, let him have to seek his meal
where he may.  And I was soon informed, with considerable energy,
that let the boat be kept there as long as it might by stress of
weather, the beef-steaks and apple jam, light fixings and heavy
fixings, must be supplied at the cost of the owners of the ship.
``Your first supper you pay for,'' my informant told me, ``because you
eat that on your own account.  What you consume after that comes of
their doing, because they don't start; and if it's three meals a
day for a week, it's their look out.''  It occurred to me that,
under such circumstances, a captain would be very apt to sail
either in foul weather or in fair.

It was a bright moonlight night---moonlight such as we rarely have
in England---and I started off by myself for a walk, that I might
see of what nature were the environs of Grand Haven.  A more
melancholy place I never beheld.  The town of Grand Haven itself is
placed on the opposite side of a creek, and was to be reached by a
ferry.  On our side, to which the railway came and from which the
boat was to sail, there was nothing to be seen but sand hills,
which stretched away for miles along the shore of the lake.  There
were great sand mountains and sand valleys, on the surface of which
were scattered the debris of dead trees, scattered logs white with
age, and boughs half buried beneath the sand.  Grand Haven itself
is but a poor place, not having succeeded in catching much of the
commerce which comes across the lake from Wisconsin, and which
takes itself on Eastward by the railway.  Altogether, it is a
dreary place, such as might break a man's heart should he find that
inexorable fate required him there to pitch his tent.

On my return I went down into the bar-room of the steamer, put my
feet upon the counter, lit my cigar, and struck into the debate
then proceeding on the subject of the war.  I was getting West, and
General Fremont was the hero of the hour.  ``He's a frontier man,
and that's what we want.  I guess he'll about go through.  Yes,
sir.''  ``As for relieving General Fre-mont,'' (with the accent always
strongly on the ``mont,'') ``I guess you may as well talk of relieving
the whole West.  They won't meddle with Fre-mont.  They are
beginning to know in Washington what stuff he's made of.''  ``Why,
sir, there are 50,000 men in these States who will follow Fre-mont,
who would not stir a foot after any other man.''  From which, and
the like of it in many other places, I began to understand how
difficult was the task which the statesmen in Washington had in
hand.

I received no pecuniary advantage whatever from that law as to the
steamboat meals which my new friend had revealed to me.  For my one
supper of course I paid, looking forward to any amount of
subsequent gratuitous provisions.  But in the course of the night
the ship sailed, and we found ourselves at Milwaukee in time for
breakfast on the following morning.

Milwaukee is a pleasant town, a very pleasant town, containing
45,000 inhabitants.  How many of my readers can boast that they
know anything of Milwaukee, or even have heard of it?  To me its
name was unknown until I saw it on huge railway placards stuck up
in the smoking-rooms and lounging halls of all American hotels.  It
is the big town of Wisconsin, whereas Madison is the capital.  It
stands immediately on the western shore of Lake Michigan, and is
very pleasant.  Why it should be so, and why Detroit should be the
contrary, I can hardly tell; only I think that the same verdict
would be given by any English tourist.  It must be always borne in
mind that 10,000 or 40,000 inhabitants in an American town, and
especially in any new Western town, is a number which means much
more than would be implied by any similar number as to an old town
in Europe.  Such a population in America consumes double the amount
of beef which it would in England, wears double the amount of
clothes, and demands double as much of the comforts of life.  If a
census could be taken of the watches, it would be found, I take it,
that the American population possessed among them nearly double as
many as would the English; and I fear also that it would be found
that many more of the Americans were readers and writers by habit.
In any large town in England it is probable that a higher
excellence of education would be found than in Milwaukee, and also
a style of life into which more of refinement and more of luxury
had found its way.  But the general level of these things, of
material and intellectual well-being---of beef, that is, and book
learning---is no doubt infinitely higher in a new American than in
an old European town.  Such an animal as a beggar is as much
unknown as a mastodon.  Men out of work and in want are almost
unknown.  I do not say that there are none of the hardships of
life---and to them I will come by-and-by---but want is not known as a
hardship in these towns, nor is that dense ignorance in which so
large a proportion of our town populations is still steeped.  And
then the town of 40,000 inhabitants is spread over a surface which
would suffice in England for a city of four times the size.  Our
towns in England---and the towns, indeed, of Europe generally---have
been built as they have been wanted.  No aspiring ambition as to
hundreds of thousands of people warmed the bosoms of their first
founders.  Two or three dozen men required habitations in the same
locality, and clustered them together closely.  Many such have
failed and died out of the world's notice.  Others have thriven,
and houses have been packed on to houses, till London and
Manchester, Dublin and Glasgow have been produced.  Poor men have
built, or have had built for them, wretched lanes, and rich men
have erected grand palaces.  From the nature of their beginnings
such has, of necessity, been the manner of their creation.  But in
America, and especially in Western America, there has been no such
necessity and there is no such result.  The founders of cities have
had the experience of the world before them.  They have known of
sanitary laws as they began.  That sewerage, and water, and gas,
and good air would be needed for a thriving community has been to
them as much a matter of fact as are the well-understood
combinations between timber and nails, and bricks and mortar.  They
have known that water carriage is almost a necessity for commercial
success, and have chosen their sites accordingly.  Broad streets
cost as little, while land by the foot is not as yet of value to be
regarded, as those which are narrow; and therefore the sites of
towns have been prepared with noble avenues and imposing streets.
A city at its commencement is laid out with an intention that it
shall be populous.  The houses are not all built at once, but there
are the places allocated for them.  The streets are not made, but
there are the spaces.  Many an abortive attempt at municipal
greatness has so been made and then all but abandoned.  There are
wretched villages, with huge, straggling parallel ways, which will
never grow into towns.  They are the failures---failures in which
the pioneers of civilization, frontier men as they call themselves,
have lost their tens of thousands of dollars.  But when the success
comes, when the happy hit has been made, and the ways of commerce
have been truly foreseen with a cunning eye, then a great and
prosperous city springs up, ready made as it were, from the earth.
Such a town is Milwaukee, now containing 45,000 inhabitants, but
with room apparently for double that number; with room for four
times that number, were men packed as closely there as they are
with us.

In the principal business streets of all these towns one sees vast
buildings.  They are usually called blocks, and are often so
denominated in large letters on their front, as Portland Block,
Devereux Block, Buel's Block.  Such a block may face to two, three,
or even four streets, and, as I presume, has generally been a
matter of one special speculation.  It may be divided into separate
houses, or kept for a single purpose, such as that of a hotel, or
grouped into shops below, and into various sets of chambers above.
I have had occasion in various towns to mount the stairs within
these blocks, and have generally found some portion of them vacant---%
have sometimes found the greater portion of them vacant.  Men
build on an enormous scale, three times, ten times as much as is
wanted.  The only measure of size is an increase on what men have
built before.  Monroe P. Jones, the speculator, is very probably
ruined, and then begins the world again nothing daunted.  But
Jones's block remains, and gives to the city in its aggregate a
certain amount of wealth.  Or the block becomes at once of service
and finds tenants.  In which case Jones probably sells it, and
immediately builds two others twice as big.  That Monroe P. Jones
will encounter ruin is almost a matter of course; but then he is
none the worse for being ruined.  It hardly makes him unhappy.  He
is greedy of dollars with a terrible covetousness; but he is greedy
in order that he may speculate more widely.  He would sooner have
built Jones's tenth block, with a prospect of completing a
twentieth, than settle himself down at rest for life as the owner
of a Chatsworth or a Woburn.  As for his children, he has no desire
of leaving them money.  Let the girls marry.  And for the boys---for
them it will be good to begin as he begun.  If they cannot build
blocks for themselves, let them earn their bread in the blocks of
other men.  So Monroe P. Jones, with his million of dollars
accomplished, advances on to a new frontier, goes to work again on
a new city, and loses it all.  As an individual I differ very much
from Monroe P. Jones.  The first block accomplished, with an
adequate rent accruing to me as the builder, I fancy that I should
never try a second.  But Jones is undoubtedly the man for the West.
It is that love of money to come, joined to a strong disregard for
money made, which constitutes the vigorous frontier mind, the true
pioneering organization.  Monroe P. Jones would be a great man to
all posterity if only he had a poet to sing of his valor.

It may be imagined how large in proportion to its inhabitants will
be a town which spreads itself in this way.  There are great houses
left untenanted, and great gaps left unfilled.  But if the place be
successful, if it promise success, it will be seen at once that
there is life all through it.  Omnibuses, or street cars working on
rails, run hither and thither.  The shops that have been opened are
well filled.  The great hotels are thronged.  The quays are crowded
with vessels, and a general feeling of progress pervades the place.
It is easy to perceive whether or no an American town is going
ahead.  The days of my visit to Milwaukee were days of civil war
and national trouble, but in spite of civil war and national
trouble Milwaukee looked healthy.

I have said that there was but little poverty---little to be seen of
real want in these thriving towns---but that they who labored in
them had nevertheless their own hardships.  This is so.  I would
not have any man believe that he can take himself to the Western
States of America---to those States of which I am now speaking---%
Michigan, Wisconsin, Minnesota, Iowa, or Illinois, and there by
industry escape the ills to which flesh is heir.  The laboring
Irish in these towns eat meat seven days a week, but I have met
many a laboring Irishman among them who has wished himself back in
his old cabin.  Industry is a good thing, and there is no bread so
sweet as that which is eaten in the sweat of a man's brow; but
labor carried to excess wearies the mind as well as body, and the
sweat that is ever running makes the bread bitter.  There is, I
think, no task-master over free labor so exacting as an American.
He knows nothing of hours, and seems to have that idea of a man
which a lady always has of a horse.  He thinks that he will go
forever.  I wish those masons in London who strike for nine hours'
work with ten hours' pay could be driven to the labor market of
Western America for a spell.  And moreover, which astonished me, I
have seen men driven and hurried, as it were forced forward at
their work, in a manner which, to an English workman, would be
intolerable.  This surprised me much, as it was at variance with
our---or perhaps I should say with my---preconceived ideas as to
American freedom.  I had fancied that an American citizen would not
submit to be driven; that the spirit of the country, if not the
spirit of the individual, would have made it impossible.  I thought
that the shoe would have pinched quite on the other foot.  But I
found that such driving did exist, and American masters in the West
with whom I had an opportunity of discussing the subject all
admitted it.  ``Those men'll never half move unless they're driven,''
a foreman said to me once as we stood together over some twenty men
who were at their work.  ``They kinder look for it, and don't well
know how to get along when they miss it.''  It was not his business
at this moment to drive---nor was he driving.  He was standing at
some little distance from the scene with me, and speculating on the
sight before him.  I thought the men were working at their best;
but their movements did not satisfy his practiced eye, and he saw
at a glance that there was no one immediately over them.

But there is worse even than this.  Wages in these regions are what
we should call high.  An agricultural laborer will earn perhaps
fifteen dollars a month and his board, and a town laborer will earn
a dollar a day.  A dollar may be taken as representing four
shillings, though it is in fact more.  Food in these parts is much
cheaper than in England, and therefore the wages must be considered
as very good.  In making, however, a just calculation it must be
borne in mind that clothing is dearer than in England, and that
much more of it is necessary.  The wages nevertheless are high, and
will enable the laborer to save money, if only he can get them
paid.  The complaint that wages are held back, and not even
ultimately paid, is very common.  There is no fixed rule for
satisfying all such claims once a week, and thus debts to laborers
are contracted, and when contracted are ignored.  With us there is
a feeling that it is pitiful, mean almost beyond expression, to
wrong a laborer of his hire.  We have men who go in debt to
tradesmen perhaps without a thought of paying them; but when we
speak of such a one who has descended into the lowest mire of
insolvency, we say that he has not paid his washerwoman.  Out there
in the West the washerwoman is as fair game as the tailor, the
domestic servant as the wine merchant.  If a man be honest he will
not willingly take either goods or labor without payment; and it
may be hard to prove that he who takes the latter is more dishonest
than he who takes the former; but with us there is a prejudice in
favor of one's washerwoman by which the Western mind is not
weakened.  ``They certainly have to be smart to get it,'' a gentleman
said to me whom I had taxed on the subject.  ``You see, on the
frontier a man is bound to be smart.  If he aint smart, he'd better
go back East, perhaps as far as Europe; he'll do there.''  I had got
my answer, and my friend had turned the question; but the fact was
admitted by him, as it had been by many others.

Why this should be so is a question to answer which thoroughly
would require a volume in itself.  As to the driving, why should
men submit to it, seeing that labor is abundant, and that in all
newly-settled countries the laborer is the true hero of the age?
In answer to this is to be alleged the fact that hired labor is
chiefly done by fresh comers, by Irish and Germans, who have not as
yet among them any combination sufficient to protect them from such
usage.  The men over them are new as masters, masters who are rough
themselves, who themselves have been roughly driven, and who have
not learned to be gracious to those below them.  It is a part of
their contract that very hard work shall be exacted, and the
driving resolves itself into this: that the master, looking after
his own interest, is constantly accusing his laborer of a breach of
his part of the contract.  The men no doubt do become used to it,
and slacken probably in their endeavors when the tongue of the
master or foreman is not heard.  But as to that matter of non-
payment of wages, the men must live; and here, as elsewhere, the
master who omits to pay once will hardly find laborers in future.
The matter would remedy itself elsewhere, and does it not do so
here?  This of course is so, and it is not to be understood that
labor as a rule is defrauded of its hire.  But the relation of the
master and the man admit of such fraud here much more frequently
than in England.  In England the laborer who did not get his wages
on the Saturday, could not go on for the next week.  To him, under
such circumstances, the world would be coming to an end.  But in
the Western States the laborer does not live so completely from
hand to mouth.  He is rarely paid by the week, is accustomed to
give some credit, and, till hard pressed by bad circumstances,
generally has something by him.  They do save money, and are thus
fattened up to a state which admits of victimization.  I cannot owe
money to the little village cobbler who mends my shoes, because he
demands and receives his payment when his job is done.  But to my
friend in Regent Street I extend my custom on a different system;
and when I make my start for continental life I have with him a
matter of unsettled business to a considerable extent.  The
American laborer is in the condition of the Regent Street
bootmaker, excepting in this respect, that he gives his credit
under compulsion.  ``But does not the law set him right?  Is there
no law against debtors?''  The laws against debtors are plain enough
as they are written down, but seem to be anything but plain when
called into action.  They are perfectly understood, and operations
are carried on with the express purpose of evading them.  If you
proceed against a man, you find that his property is in the hands
of some one else.  You work in fact for Jones, who lives in the
street next to you; but when you quarrel with Jones about your
wages, you find that according to law you have been working for
Smith, in another State.  In all countries such dodges are probably
practicable.  But men will or will not have recourse to such dodges
according to the light in which they are regarded by the community.
In the Western States such dodges do not appear to be regarded as
disgraceful.  ``It behoves a frontier man to be smart, sir.''

Honesty is the best policy.  That is a doctrine which has been
widely preached, and which has recommended itself to many minds as
being one of absolute truth.  It is not very ennobling in its
sentiment, seeing that it advocates a special virtue, not on the
ground that that virtue is in itself a thing beautiful, but on
account of the immediate reward which will be its consequence.
Smith is enjoined not to cheat Jones, because he will, in the long
run, make more money by dealing with Jones on the square.  This is
not teaching of the highest order; but it is teaching well adapted
to human circumstances, and has obtained for itself a wide credit.
One is driven, however, to doubt whether even this teaching is not
too high for the frontier man.  Is it possible that a frontier man
should be scrupulous and at the same time successful?  Hitherto
those who have allowed scruples to stand in their way have not
succeeded; and they who have succeeded and made for themselves
great names, who have been the pioneers of civilization, have not
allowed ideas of exact honesty to stand in their way.  From General
Jason down to General Fremont there have been men of great
aspirations but of slight scruples.  They have been ambitious of
power and desirous of progress, but somewhat regardless how power
and progress shall be attained.  Clive and Warren Hastings were
great frontier men, but we cannot imagine that they had ever
realized the doctrine that honesty is the best policy.  Cortez, and
even Columbus, the prince of frontier men, are in the same
category.  The names of such heroes is legion; but with none of
them has absolute honesty been a favorite virtue.  ``It behoves a
frontier man to be smart, sir.''  Such, in that or other language,
has been the prevailing idea.  Such is the prevailing idea.  And
one feels driven to ask one's self whether such must not be the
prevailing idea with those who leave the world and its rules behind
them, and go forth with the resolve that the world and its rules
shall follow them.

Of filibustering, annexation, and polishing savages off the face of
creation there has been a great deal, and who can deny that
humanity has been the gainer?  It seems to those who look widely
back over history, that all such works have been carried on in
obedience to God's laws.  When Jacob by Rebecca's aid cheated his
elder brother, he was very smart; but we cannot but suppose that a
better race was by this smartness put in possession of the
patriarchal scepter.  Esau was polished off, and readers of
Scripture wonder why heaven, with its thunder, did not open over
the heads of Rebecca and her son.  But Jacob, with all his fraud,
was the chosen one.  Perhaps the day may come when scrupulous
honesty may be the best policy, even on the frontier.  I can only
say that hitherto that day seems to be as distant as ever.  I do
not pretend to solve the problem, but simply record my opinion that
under circumstances as they still exist I should not willingly
select a frontier life for my children.

I have said that all great frontier men have been unscrupulous.
There is, however, an exception in history which may perhaps serve
to prove the rule.  The Puritans who colonized New England were
frontier men, and were, I think, in general scrupulously honest.
They had their faults.  They were stern, austere men, tyrannical at
the backbone when power came in their way, as are all pioneers,
hard upon vices for which they who made the laws had themselves no
minds; but they were not dishonest.

At Milwaukee I went up to see the Wisconsin volunteers, who were
then encamped on open ground in the close vicinity of the town.  Of
Wisconsin I had heard before---and have heard the same opinion
repeated since---that it was more backward in its volunteering than
its neighbor States in the West.  Wisconsin has 760,000
inhabitants, and its tenth thousand of volunteers was not then made
up; whereas Indiana, with less than double its number, had already
sent out thirty-six thousand.  Iowa, with a hundred thousand less
of inhabitants, had then made up fifteen thousand.  But neverthless
to me it seemed that Wisconsin was quite alive to its presumed duty
in that respect.  Wisconsin, with its three-quarters of a million
of people, is as large as England.  Every acre of it may be made
productive, but as yet it is not half cleared.  Of such a country
its young men are its heart's blood.  Ten thousand men, fit to bear
arms, carried away from such a land to the horrors of civil war, is
a sight as full of sadness as any on which the eye can rest.  Ah
me, when will they return, and with what altered hopes!  It is, I
fear, easier to turn the sickle into the sword than to recast the
sword back again into the sickle!

We found a completed regiment at Wisconsin consisting entirely of
Germans.  A thousand Germans had been collected in that State and
brought together in one regiment, and I was informed by an officer
on the ground that there are many Germans in sundry other of the
Wisconsin regiments.  It may be well to mention here that the
number of Germans through all these Western States is very great.
Their number and well-being were to me astonishing.  That they form
a great portion of the population of New York, making the German
quarter of that city the third largest German town in the world, I
have long known; but I had no previous idea of their expansion
westward.  In Detroit nearly every third shop bore a German name,
and the same remark was to be made at Milwaukee; and on all hands I
heard praises of their morals, of their thrift, and of their new
patriotism.  I was continually told how far they exceeded the Irish
settlers.  To me in all parts of the world an Irishman is dear.
When handled tenderly he becomes a creature most lovable.  But with
all my judgment in the Irishman's favor, and with my prejudices
leaning the same way, I feel myself bound to state what I heard and
what I saw as to the Germans.

But this regiment of Germans, and another not completed regiment,
called from the State generally, were as yet without arms,
accouterments, or clothing.  There was the raw material of the
regiment, but there was nothing else.  Winter was coming on---winter
in which the mercury is commonly twenty degrees below zero---and the
men were in tents with no provision against the cold.  These tents
held each two men, and were just large enough for two to lie.  The
canvas of which they were made seemed to me to be thin, but was, I
think, always double.  At this camp there was a house in which the
men took their meals, but I visited other camps in which there was
no such accommodation.  I saw the German regiment called to its
supper by tuck of drum, and the men marched in gallantly, armed
each with a knife and spoon.  I managed to make my way in at the
door after them, and can testify to the excellence of the
provisions of which their supper consisted.  A poor diet never
enters into any combination of circumstances contemplated by an
American.  Let him be where he will, animal food is with him the
first necessary of life, and he is always provided accordingly.  As
to those Wisconsin men whom I saw, it was probable that they might
be marched off, down South to Washington, or to the doubtful
glories of the Western campaign under Fremont, before the winter
commenced.  The same might have been said of any special regiment.
But taking the whole mass of men who were collected under canvas at
the end of the autumn of 1861, and who were so collected without
arms or military clothing, and without protection from the weather,
it did seem that the task taken in hand by the Commissariat of the
Northern army was one not devoid of difficulty.

The view from Milwaukee over Lake Michigan is very pleasing.  One
looks upon a vast expanse of water to which the eye finds no
bounds, and therefore there are none of the common attributes of
lake beauty; but the color of the lake is bright, and within a walk
of the city the traveler comes to the bluffs or low round-topped
hills, from which we can look down upon the shores.  These bluffs
form the beauty of Wisconsin and Minnesota, and relieve the eye
after the flat level of Michigan.  Round Detroit there is no rising
ground, and therefore, perhaps, it is that Detroit is
uninteresting.

I have said that those who are called on to labor in these States
have their own hardships, and I have endeavored to explain what are
the sufferings to which the town laborer is subject.  To escape
from this is the laborer's great ambition, and his mode of doing so
consists almost universally in the purchase of land.  He saves up
money in order that he may buy a section of an allotment, and thus
become his own master.  All his savings are made with a view to
this independence.  Seated on his own land he will have to work
probably harder than ever, but he will work for himself.  No task-
master can then stand over him and wound his pride with harsh
words.  He will be his own master; will eat the food which he
himself has grown, and live in the cabin which his own hands have
built.  This is the object of his life; and to secure this position
he is content to work late and early and to undergo the indignities
of previous servitude.  The government price for land is about five
shillings an acre---one dollar and a quarter---and the settler may
get it for this price if he be contented to take it not only
untouched as regards clearing, but also far removed from any
completed road.  The traffic in these lands has been the great
speculating business of Western men.  Five or six years ago, when
the rage for such purchases was at its height, land was becoming a
scarce article in the market.  Individuals or companies bought it
up with the object of reselling it at a profit; and many, no doubt,
did make money.  Railway companies were, in fact, companies
combined for the purchase of land.  They purchased land, looking to
increase the value of it fivefold by the opening of a railroad.  It
may easily be understood that a railway, which could not be in
itself remunerative, might in this way become a lucrative
speculation.  No settler could dare to place himself absolutely at
a distance from any thoroughfare.  At first the margins of nature's
highways, the navigable rivers and lakes, were cleared.  But as the
railway system grew and expanded itself, it became manifest that
lands might be rendered quickly available which were not so
circumstanced by nature.  A company which had purchased an enormous
territory from the United States government at five shillings an
acre might well repay itself all the cost of a railway through that
territory, even though the receipts of the railway should do no
more than maintain the current expenses.  It is in this way that
the thousands of miles of American railroads have been opened; and
here again must be seen the immense advantages which the States as
a new country have enjoyed.  With us the purchase of valuable land
for railways, together with the legal expenses which those
compulsory purchases entailed, have been so great that with all our
traffic railways are not remunerative.  But in the States the
railways have created the value of the land.  The States have been
able to begin at the right end, and to arrange that the districts
which are benefited shall themselves pay for the benefit they
receive.

The government price of land is 125 cents, or about five shillings
an acre; and even this need not be paid at once if the settler
purchase directly from the government.  He must begin by making
certain improvements on the selected land---clearing and cultivating
some small portion, building a hut, and probably sinking a well.
When this has been done---when he has thus given a pledge of his
intentions by depositing on the land the value of a certain amount
of labor, he cannot be removed.  He cannot be removed for a term of
years, and then if he pays the price of the land it becomes his own
with an indefeasible title.  Many such settlements are made on the
purchase of warrants for land.  Soldiers returning from the Mexican
wars were donated with warrants for land---the amount being 160
acres, or the quarter of a section.  The localities of such lands
were not specified, but the privilege granted was that of occupying
any quarter-section not hitherto tenanted.  It will, of course, be
understood that lands favorably situated would be tenanted.  Those
contiguous to railways were of course so occupied, seeing that the
lines were not made till the lands were in the hands of the
companies.  It may therefore be understood of what nature would be
the traffic in these warrants.  The owner of a single warrant might
find it of no value to him.  To go back utterly into the woods,
away from river or road, and there to commence with 160 acres of
forest, or even of prairie, would be a hopeless task even to an
American settler.  Some mode of transport for his produce must be
found before his produce would be of value---before, indeed, he
could find the means of living.  But a company buying up a large
aggregate of such warrants would possess the means of making such
allotments valuable and of reselling them at greatly increased
prices.

The primary settler, therefore---who, however, will not usually have
been the primary owner---goes to work upon his land amid all the
wildness of nature.  He levels and burns the first trees, and
raises his first crop of corn amid stumps still standing four or
five feet above the soil; but he does not do so till some mode of
conveyance has been found for him.  So much I have said hoping to
explain the mode in which the frontier speculator paves the way for
the frontier agriculturist.  But the permanent farmer very
generally comes on the land as the third owner.  The first settler
is a rough fellow, and seems to be so wedded to his rough life that
he leaves his land after his first wild work is done, and goes
again farther off to some untouched allotment.  He finds that he
can sell his improvements at a profitable rate and takes the price.
He is a preparer of farms rather than a farmer.  He has no love for
the soil which his hand has first turned.  He regards it merely as
an investment; and when things about him are beginning to wear an
aspect of comfort, when his property has become valuable, he sells
it, packs up his wife and little ones, and goes again into the
woods.  The Western American has no love for his own soil or his
own house.  The matter with him is simply one of dollars.  To keep
a farm which he could sell at an advantage from any feeling of
affection---from what we should call an association of ideas---would
be to him as ridiculous as the keeping of a family pig would be in
an English farmer's establishment.  The pig is a part of the
farmer's stock in trade, and must go the way of all pigs.  And so
is it with house and land in the life of the frontier man in the
Western States.

But yet this man has his romance, his high poetic feeling, and
above all his manly dignity.  Visit him, and you will find him
without coat or waistcoat, unshorn, in ragged blue trowsers and old
flannel shirt, too often bearing on his lantern jaws the signs of
ague and sickness; but he will stand upright before you and speak
to you with all the ease of a lettered gentleman in his own
library.  All the odious incivility of the republican servant has
been banished.  He is his own master, standing on his own
threshold, and finds no need to assert his equality by rudeness.
He is delighted to see you, and bids you sit down on his battered
bench without dreaming of any such apology as an English cottier
offers to a Lady Bountiful when she calls.  He has worked out his
independence, and shows it in every easy movement of his body.  He
tells you of it unconsciously in every tone of his voice.  You will
always find in his cabin some newspaper, some book, some token of
advance in education.  When he questions you about the old country
he astonishes you by the extent of his knowledge.  I defy you not
to feel that he is superior to the race from whence he has sprung
in England or in Ireland.  To me I confess that the manliness of
such a man is very charming.  He is dirty, and, perhaps, squalid.
His children are sick and he is without comforts.  His wife is
pale, and you think you see shortness of life written in the faces
of all the family.  But over and above it all there is an
independence which sits gracefully on their shoulders, and teaches
you at the first glance that the man has a right to assume himself
to be your equal.  It is for this position that the laborer works,
bearing hard words and the indignity of tyranny; suffering also too
often the dishonest ill usage which his superior power enables the
master to inflict.

``I have lived very rough,'' I heard a poor woman say, whose husband
had ill used and deserted her.  ``I have known what it is to be
hungry and cold, and to work hard till my bones have ached.  I only
wish that I might have the same chance again.  If I could have ten
acres cleared two miles away from any living being, I could be
happy with my children.  I find a kind of comfort when I am at work
from daybreak to sundown, and know that it is all my own.''  I
believe that life in the backwoods has an allurement to those who
have been used to it that dwellers in cities can hardly comprehend.

From Milwaukee we went across Wisconsin, and reached the
Mississippi at La Crosse.  From hence, according to agreement, we
were to start by steamer at once up the river.  But we were delayed
again, as had happened to us before on Lake Michigan at Grand
Haven.



\chapter{The Upper Mississippi}


It had been promised to us that we should start from La Crosse by
the river steamer immediately on our arrival there; but, on
reaching La Crosse, we found that the vessel destined to take us up
the river had not yet come down.  She was bringing a regiment from
Minnesota, and, under such circumstances, some pardon might be
extended to irregularities.  This plea was made by one of the boat
clerks in a very humble tone, and was fully accepted by us.  The
wonder was that, at such a period, all means of public conveyance
were not put absolutely out of gear.  One might surmise that when
regiments were constantly being moved for the purposes of civil
war---when the whole North had but the one object of collecting
together a sufficient number of men to crush the South---ordinary
traveling for ordinary purposes would be difficult, slow, and
subject to sudden stoppages.  Such, however, was not the case
either in the Northern or Western States.  The trains ran much as
usual, and those connected with the boats and railways were just as
anxious as ever to secure passengers.  The boat clerk at La Crosse
apologized amply for the delay; and we sat ourselves down with
patience to await the arrival of the second Minnesota Regiment on
its way to Washington.

During the four hours that we were kept waiting we were harbored on
board a small steamer; and at about eleven the terribly harsh
whistle that is made by the Mississippi boats informed us that the
regiment was arriving.  It came up to the quay in two steamers---750
being brought in that which was to take us back, and 250 in a
smaller one.  The moon was very bright, and great flaming torches
were lit on the vessel's side, so that all the operations of the
men were visible.  The two steamers had run close up, thrusting us
away from the quay in their passage, but doing it so gently that we
did not even feel the motion.  These large boats---and their size
may be understood from the fact that one of them had just brought
down 750 men---are moved so easily and so gently that they come
gliding in among each other without hesitation and without pause.
On English waters we do not willingly run ships against each other;
and when we do so unwillingly, they bump and crush and crash upon
each other, and timbers fly while men are swearing.  But here there
was neither crashing nor swearing; and the boats noiselessly
pressed against each other as though they were cased in muslin and
crinoline.

I got out upon the quay and stood close by the plank, watching each
man as he left the vessel and walked across toward the railway.
Those whom I had previously seen in tents were not equipped; but
these men were in uniform, and each bore his musket.  Taking them
altogether, they were as fine a set of men as I ever saw collected.
No man could doubt, on seeing them, that they bore on their
countenances the signs of higher breeding and better education than
would be seen in a thousand men enlisted in England.  I do not mean
to argue from this that Americans are better than English.  I do
not mean to argue here that they are even better educated.  My
assertion goes to show that the men generally were taken from a
higher level in the community than that which fills our own ranks.
It was a matter of regret to me, here and on many subsequent
occasions, to see men bound for three years to serve as common
soldiers who were so manifestly fitted for a better and more useful
life.  To me it is always a source of sorrow to see a man enlisted.
I feel that the individual recruit is doing badly with himself---%
carrying himself, and the strength and intelligence which belong to
him, to a bad market.  I know that there must be soldiers; but as
to every separate soldier I regret that he should be one of them.
And the higher is the class from which such soldiers are drawn, the
greater the intelligence of the men so to be employed, the deeper
with me is that feeling of regret.  But this strikes one much less
in an old country than in a country that is new.  In the old
countries population is thick and food sometimes scarce.  Men can
be spared; and any employment may be serviceable, even though that
employment be in itself so unproductive as that of fighting battles
or preparing for them.  But in the Western States of America every
arm that can guide a plow is of incalculable value.  Minnesota was
admitted as a State about three years before this time, and its
whole population is not much above 150,000.  Of this number perhaps
40,000 may be working men.  And now this infant State, with its
huge territory and scanty population, is called upon to send its
heart's blood out to the war.

And it has sent its heart's best blood.  Forth they came---fine,
stalwart, well-grown fellows---looking, to my eye, as though they
had as yet but faintly recognized the necessary severity of
military discipline.  To them hitherto the war had seemed to be an
arena on which each might do something for his country which that
country would recognize.  To themselves as yet---and to me also---%
they were a band of heroes, to be reduced by the compressing power
of military discipline to the lower level, but more necessary
position, of a regiment of soldiers.  Ah, me! how terrible to them
has been the breaking up of that delusion!  When a poor yokel in
England is enlisted with a shilling and a promise of unlimited beer
and glory, one pities, and, if possible, would save him.  But with
him the mode of life to which he goes may not be much inferior to
that he leaves.  It may be that for him soldiering is the best
trade possible in his circumstances.  It may keep him from the hen-
roosts, and perhaps from his neighbors' pantries; and discipline
may be good for him.  Population is thick with us; and there are
many whom it may be well to collect and make available under the
strictest surveillance.  But of these men whom I saw entering on
their career upon the banks of the Mississippi, many were fathers
of families, many were owners of lands, many were educated men
capable of high aspirations---all were serviceable members of their
State.  There were probably there not three or four of whom it
would be well that the State should be rid.  As soldiers, fit or
capable of being made fit for the duties they had undertaken, I
could find but one fault with them.  Their average age was too
high.  There were men among them with grizzled beards, and many who
had counted thirty, thirty-five, and forty years.  They had, I
believe, devoted themselves with a true spirit of patriotism.  No
doubt each had some ulterior hope as to himself, as has every
mortal patriot.  Regulus, when he returned hopeless to Carthage,
trusted that some Horace would tell his story.  Each of these men
from Minnesota looked probably forward to his reward; but the
reward desired was of a high class.

The first great misery to be endured by these regiments will be the
military lesson of obedience which they must learn before they can
be of any service.  It always seemed to me, when I came near them,
that they had not as yet recognized the necessary austerity of an
officer's duty.  Their idea of a captain was the stage idea of a
leader of dramatic banditti---a man to be followed and obeyed as a
leader, but to be obeyed with that free and easy obedience which is
accorded to the reigning chief of the forty thieves.  ``Waal,
captain,'' I have heard a private say to his officer, as he sat on
one seat in a railway car, with his feet upon the back of another.
And the captain has looked as though he did not like it.  The
captain did not like it; but the poor private was being fast
carried to that destiny which he would like still less.  From the
first I have had faith in the Northern army; but from the first I
have felt that the suffering to be endured by these free and
independent volunteers would be very great.  A man, to be available
as a private soldier, must be compressed and belted in till he be a
machine.

As soon as the men had left the vessel we walked over the side of
it and took possession.  ``I am afraid your cabin won't be ready for
a quarter of an hour,'' said the clerk.  ``Such a body of men as that
will leave some dirt after them.''  I assured him, of course, that
our expectations under such circumstances were very limited, and
that I was fully aware that the boat and the boat's company were
taken up with matters of greater moment than the carriage of
ordinary passengers.  But to this he demurred altogether.  ``The
regiments were very little to them, but occasioned much trouble.
Everything, however, should be square in fifteen minutes.''  At the
expiration of the time named the key of our state-room was given to
us, and we found the appurtenances as clean as though no soldier
had ever put his foot upon the vessel.

From La Crosse to St. Paul the distance up the river is something
over 200 miles; and from St. Paul down to Dubuque in Iowa, to which
we went on our return, the distance is 450 miles.  We were,
therefore, for a considerable time on board these boats---more so
than such a journey may generally make necessary, as we were
delayed at first by the soldiers, and afterward by accidents, such
as the breaking of a paddle-wheel, and other causes, to which
navigation on the Upper Mississippi seems to be liable.  On the
whole, we slept on board four nights, and lived on board as many
days.  I cannot say that the life was comfortable, though I do not
know that it could be made more so by any care on the part of the
boat owners.  My first complaint would be against the great heat of
the cabins.  The Americans, as a rule, live in an atmosphere which
is almost unbearable by an Englishman.  To this cause, I am
convinced, is to be attributed their thin faces, their pale skins,
their unenergetic temperament---unenergetic as regards physical
motion---and their early old age.  The winters are long and cold in
America, and mechanical ingenuity is far extended.  These two facts
together have created a system of stoves, hot-air pipes, steam
chambers, and heating apparatus so extensive that, from autumn till
the end of spring, all inhabited rooms are filled with the
atmosphere of a hot oven.  An Englishman fancies that he is to be
baked, and for awhile finds it almost impossible to exist in the
air prepared for him.  How the heat is engendered on board the
river steamers I do not know, but it is engendered to so great a
degree that the sitting-cabins are unendurable.  The patient is
therefore driven out at all hours into the outside balconies of the
boat, or on to the top roof---for it is a roof rather than a deck---%
and there, as he passes through the air at the rate of twenty miles
an hour, finds himself chilled to the very bones.  That is my first
complaint.  But as the boats are made for Americans, and as
Americans like hot air, I do not put it forward with any idea that
a change ought to be effected.  My second complaint is equally
unreasonable, and is quite as incapable of a remedy as the first.
Nine-tenths of the travelers carry children with them.  They are
not tourists engaged on pleasure excursions, but men and women
intent on the business of life.  They are moving up and down
looking for fortune and in search of new homes.  Of course they
carry with them all their household goods.  Do not let any critic
say that I grudge these young travelers their right to locomotion.
Neither their right to locomotion is grudged by me, nor any of
those privileges which are accorded in America to the rising
generation.  The habits of their country and the choice of their
parents give to them full dominion over all hours and over all
places, and it would ill become a foreigner to make such habits and
such choice a ground of serious complaint.  But, nevertheless, the
uncontrolled energies of twenty children round one's legs do not
convey comfort or happiness, when the passing events are producing
noise and storm rather than peace and sunshine.  I must protest
that American babies are an unhappy race.  They eat and drink just
as they please; they are never punished; they are never banished,
snubbed, and kept in the background as children are kept with us,
and yet they are wretched and uncomfortable.  My heart has bled for
them as I have heard them squalling by the hour together in agonies
of discontent and dyspepsia.  Can it be, I wonder, that children
are happier when they are made to obey orders, and are sent to bed
at six o'clock, than when allowed to regulate their own conduct;
that bread and milk are more favorable to laughter and soft,
childish ways than beef-steaks and pickles three times a day; that
an occasional whipping, even, will conduce to rosy cheeks?  It is
an idea which I should never dare to broach to an American mother;
but I must confess that, after my travels on the Western Continent,
my opinions have a tendency in that direction.  Beef-steaks and
pickles certainly produce smart little men and women.  Let that be
taken for granted.  But rosy laughter and winning, childish ways
are, I fancy, the produce of bread and milk.  But there was a third
reason why traveling on these boats was not so pleasant as I had
expected.  I could not get my fellow-travelers to talk to me.  It
must be understood that our fellow-travelers were not generally of
that class which we Englishmen, in our pride, designate as
gentlemen and ladies.  They were people, as I have said, in search
of new homes and new fortunes.  But I protest that as such they
would have been, in those parts, much more agreeable as companions
to me than any gentlemen or any ladies, if only they would have
talked to me.  I do not accuse them of any incivility.  If
addressed, they answered me.  If application was made by me for any
special information, trouble was taken to give it me.  But I found
no aptitude, no wish for conversation---nay, even a disinclination
to converse.  In the Western States I do not think that I was ever
addressed first by an American sitting next to me at table.
Indeed, I never held any conversation at a public table in the
West.  I have sat in the same room with men for hours, and have not
had a word spoken to me.  I have done my very best to break through
this ice, and have always failed.  A Western American man is not a
talking man.  He will sit for hours over a stove, with a cigar in
his mouth and his hat over his eyes, chewing the cud of reflection.
A dozen will sit together in the same way, and there shall not be a
dozen words spoken between them in an hour.  With the women one's
chance of conversation is still worse.  It seemed as though the
cares of the world had been too much for them, and that all talking
excepting as to business---demands, for instance, on the servants
for pickles for their children---had gone by the board.  They were
generally hard, dry, and melancholy.  I am speaking, of course, of
aged females---from five and twenty, perhaps, to thirty---who had
long since given up the amusements and levities of life.  I very
soon abandoned any attempt at drawing a word from these ancient
mothers of families; but not the less did I ponder in my mind over
the circumstances of their lives.  Had things gone with them so
sadly---was the struggle for independence so hard---that all the
softness of existence had been trodden out of them?  In the cities,
too, it was much the same.  It seemed to me that a future mother of
a family, in those parts, had left all laughter behind her when she
put out her finger for the wedding ring.

For these reasons I must say that life on board these steamboats
was not as pleasant as I had hoped to find it; but for our
discomfort in this respect we found great atonement in the scenery
through which we passed.  I protest that of all the river scenery
that I know that of the Upper Mississippi is by far the finest and
the most continued.  One thinks, of course, of the Rhine; but,
according to my idea of beauty, the Rhine is nothing to the Upper
Mississippi.  For miles upon miles---for hundreds of miles---the
course of the river runs through low hills, which are there called
bluffs.  These bluffs rise in every imaginable form, looking
sometimes like large, straggling, unwieldy castles, and then
throwing themselves into sloping lawns which stretch back away from
the river till the eye is lost in their twists and turnings.
Landscape beauty, as I take it, consists mainly in four attributes---%
in water; in broken land; in scattered timber, timber scattered as
opposed to continuous forest timber; and in the accident of color.
In all these particulars the banks of the Upper Mississippi can
hardly be beaten.  There are no high mountains; but high mountains
themselves are grand rather than beautiful.  There are no high
mountains; but there is a succession of hills, which group
themselves forever without monotony.  It is, perhaps, the ever-
variegated forms of these bluffs which chiefly constitute the
wonderful loveliness of this river.  The idea constantly occurs
that some point on every hillside would form the most charming site
ever yet chosen for a noble residence.  I have passed up and down
rivers clothed to the edge with continuous forest.  This at first
is grand enough, but the eye and feeling soon become weary.  Here
the trees are scattered so that the eye passes through them, and
ever and again a long lawn sweeps back into the country and up the
steep side of a hill, making the traveler long to stay there and
linger through the oaks, and climb the bluffs, and lay about on the
bold but easy summits.  The boat, however, steams quickly up
against the current, and the happy valleys are left behind one
quickly after another.  The river is very various in its breadth,
and is constantly divided by islands.  It is never so broad that
the beauty of the banks is lost in the distance or injured by it.
It is rapid, but has not the beautifully bright color of some
European rivers---of the Rhine, for instance, and the Rhone.  But
what is wanting in the color of the water is more than compensated
by the wonderful hues and luster of the shores.  We visited the
river in October, and I must presume that they who seek it solely
for the sake of scenery should go there in that month.  It was not
only that the foliage of the trees was bright with every imaginable
color, but that the grass was bronzed and that the rocks were
golden.  And this beauty did not last only for awhile, and then
cease.  On the Rhine there are lovely spots and special morsels of
scenery with which the traveler becomes duly enraptured.  But on
the Upper Mississippi there are no special morsels.  The position
of the sun in the heavens will, as it always does, make much
difference in the degree of beauty.  The hour before and the half
hour after sunset are always the loveliest for such scenes.  But of
the shores themselves one may declare that they are lovely
throughout those four hundred miles which run immediately south
from St. Paul.

About half way between La Crosse and St. Paul we came upon Lake
Pepin, and continued our course up the lake for perhaps fifty or
sixty miles.  This expanse of water is narrow for a lake, and, by
those who know the lower courses of great rivers, would hardly be
dignified by that name.  But, nevertheless, the breadth here
lessens the beauty.  There are the same bluffs, the same scattered
woodlands, and the same colors.  But they are either at a distance,
or else they are to be seen on one side only.  The more that I see
of the beauty of scenery, and the more I consider its elements, the
stronger becomes my conviction that size has but little to do with
it, and rather detracts from it than adds to it.  Distance gives
one of its greatest charms, but it does so by concealing rather
than displaying an expanse of surface.  The beauty of distance
arises from the romance, the feeling of mystery which it creates.
It is like the beauty of woman, which allures the more the more
that it is vailed.  But open, uncovered land and water, mountains
which simply rise to great heights, with long, unbroken slopes,
wide expanses of lake, and forests which are monotonous in their
continued thickness, are never lovely to me.  A landscape should
always be partly vailed, and display only half its charms.

To my taste the finest stretch of the river was that immediately
above Lake Pepin; but then, at this point, we had all the glory of
the setting sun.  It was like fairy-land, so bright were the golden
hues, so fantastic were the shapes of the hills, so broken and
twisted the course of the waters!  But the noisy steamer went
groaning up the narrow passages with almost unabated speed, and
left the fairy land behind all too quickly.  Then the bell would
ring for tea, and the children with the beef-steaks, the pickled
onions, and the light fixings would all come over again.  The care-
laden mothers would tuck the bibs under the chins of their tyrant
children, and some embryo senator of four years old would listen
with concentrated attention while the negro servant recapitulated
to him the delicacies of the supper-table, in order that he might
make his choice with due consideration.  ``Beef-steak,'' the embryo
four-year old senator would lisp, ``and stewed potato, and buttered
toast, and corn-cake, and coffee,---and---and---and---mother, mind you
get me the pickles.''

St. Paul enjoys the double privilege of being the commercial and
political capital of Minnesota.  The same is the case with Boston,
in Massachusetts, but I do not remember another instance in which
it is so.  It is built on the eastern bank of the Mississippi,
though the bulk of the State lies to the west of the river.  It is
noticeable as the spot up to which the river is navigable.
Immediately above St. Paul there are narrow rapids up which no boat
can pass.  North of this continuous navigation does not go; but
from St. Paul down to New Orleans and the Gulf of Mexico it is
uninterrupted.  The distance to St. Louis in Missouri, a town built
below the confluence of the three rivers, Mississippi, Missouri,
and Illinois, is 900 miles and then the navigable waters down to
the Gulf wash a southern country of still greater extent.  No river
on the face of the globe forms a highway for the produce of so wide
an extent of agricultural land.  The Mississippi, with its
tributaries, carried to market, before the war, the produce of
Wisconsin, Minnesota, Iowa, Illinois, Indiana, Ohio, Kentucky,
Tennessee, Missouri, Kansas, Arkansas, Mississippi, and Louisiana.
This country is larger than England, Ireland, Scotland, Holland,
Belgium, France, Germany, and Spain together, and is undoubtedly
composed of much more fertile land.  The States named comprise the
great center valley of the continent, and are the farming lands and
garden grounds of the Western World.  He who has not seen corn on
the ground in Illinois or Minnesota, does not know to what extent
the fertility of land may go, or how great may be the weight of
cereal crops.  And for all this the Mississippi was the high-road
to market.  When the crop of 1861 was garnered, this high-road was
stopped by the war.  What suffering this entailed on the South I
will not here stop to say, but on the West the effect was terrible.
Corn was in such plenty---Indian-corn, that is, or maize---that it
was not worth the farmer's while to prepare it for market.  When I
was in Illinois, the second quality of Indian-corn, when shelled,
was not worth more than from eight to ten cents a bushel.  But the
shelling and preparation is laborious, and in some instances it was
found better to burn it for fuel than to sell it.  Respecting the
export of corn from the West, I must say a further word or two in
the next chapter; but it seemed to be indispensable that I should
point out here how great to the United States is the need of the
Mississippi.  Nor is it for corn and wheat only that its waters are
needed.  Timber, lead, iron, coal, pork---all find, or should find,
their exit to the world at large by this road.  There are towns on
it, and on its tributaries, already holding more than one hundred
and fifty thousand inhabitants.  The number of Cincinnati exceeds
that, as also does the number of St. Louis.  Under these
circumstances it is not wonderful that the States should wish to
keep in their own hands the navigation of this river.

It is not wonderful.  But it will not, I think, be admitted by the
politicians of the world that the navigation of the Mississippi
need be closed against the West, even though the Southern States
should succeed in raising themselves to the power and dignity of a
separate nationality.  If the waters of the Danube be not open to
Austria, it is through the fault of Austria.  That the subject will
be one of trouble, no man can doubt; and of course it would be well
for the North to avoid that, or any other trouble.  In the mean
time the importance of this right of way must be admitted; and it
must be admitted, also, that whatever may be the ultimate resolve
of the North, it will be very difficult to reconcile the West to a
divided dominion of the Mississippi.

St. Paul contains about 14,000 inhabitants, and, like all other
American towns, is spread over a surface of ground adapted to the
accommodation of a very extended population.  As it is belted on
one side by the river, and on the other by the bluffs which
accompany the course of the river, the site is pretty, and almost
romantic.  Here also we found a great hotel, a huge, square
building, such as we in England might perhaps place near to a
railway terminus in such a city as Glasgow or Manchester, but on
which no living Englishman would expend his money in a town even
five times as big again as St. Paul.  Everything was sufficiently
good, and much more than sufficiently plentiful.  The whole thing
went on exactly as hotels do down in Massachusetts or the State of
New York.  Look at the map and see where St. Paul is.  Its distance
from all known civilization---all civilization that has succeeded in
obtaining acquaintance with the world at large---is very great.
Even American travelers do not go up there in great numbers,
excepting those who intend to settle there.  A stray sportsman or
two, American or English, as the case may be, makes his way into
Minnesota for the sake of shooting, and pushes on up through St.
Paul to the Red River.  Some few adventurous spirits visit the
Indian settlements, and pass over into the unsettled regions of
Dacotah and Washington Territory.  But there is no throng of
traveling.  Nevertheless, a hotel has been built there capable of
holding three hundred guests, and other hotels exist in the
neighborhood, one of which is even larger than that at St. Paul.
Who can come to them, and create even a hope that such an
enterprise may be remunerative?  In America it is seldom more than
hope, for one always hears that such enterprises fail.

When I was there the war was in hand, and it was hardly to be
expected that any hotel should succeed.  The landlord told me that
he held it at the present time for a very low rent, and that he
could just manage to keep it open without loss.  The war which
hindered people from traveling, and in that way injured the
innkeepers, also hindered people from housekeeping, and reduced
them to the necessity of boarding out, by which the innkeepers were
of course benefited.  At St. Paul I found that the majority of the
guests were inhabitants of the town, boarding at the hotel, and
thus dispensing with the cares of a separate establishment.  I do
not know what was charged for such accommodation at St. Paul, but I
have come across large houses at which a single man could get all
that he required for a dollar a day.  Now Americans are great
consumers, especially at hotels, and all that a man requires
includes three hot meals, with a choice from about two dozen dishes
at each.

From St. Paul there are two waterfalls to be seen, which we, of
course, visited.  We crossed the river at Fort Snelling, a rickety,
ill-conditioned building standing at the confluence of the
Minnesota and Mississippi Rivers, built there to repress the
Indians.  It is, I take it, very necessary, especially at the
present moment, as the Indians seem to require repressing.  They
have learned that the attention of the Federal government has been
called to the war, and have become bold in consequence.  When I was
at St. Paul I heard of a party of Englishmen who had been robbed of
everything they possessed, and was informed that the farmers in the
distant parts of the State were by no means secure.  The Indians
are more to be pitied than the farmers.  They are turning against
enemies who will neither forgive nor forget any injuries done.
When the war is over they will be improved, and polished, and
annexed, till no Indian will hold an acre of land in Minnesota.  At
present Fort Snelling is the nucleus of a recruiting camp.  On the
point between the bluffs of the two rivers there is a plain,
immediately in front of the fort, and there we saw the newly-joined
Minnesota recruits going through their first military exercises.
They were in detachments of twenties, and were rude enough at their
goose step.  The matter which struck me most in looking at them was
the difference of condition which I observed in the men.  There
were the country lads, fresh from the farms, such as we see
following the recruiting sergeant through English towns; but there
were also men in black coats and black trowsers, with thin boots,
and trimmed beards---beards which had been trimmed till very lately;
and some of them with beards which showed that they were no longer
young.  It was inexpressibly melancholy to see such men as these
twisting and turning about at the corporal's word, each handling
some stick in his hand in lieu of weapon.  Of course, they were
more awkward than the boys, even though they were twice more
assiduous in their efforts.  Of course, they were sad and wretched.
I saw men there that were very wretched---all but heart-broken, if
one might judge from their faces.  They should not have been there
handling sticks, and moving their unaccustomed legs in cramped
paces.  They were as razors, for which no better purpose could be
found than the cutting of blocks.  When such attempts are made the
block is not cut, but the razor is spoiled.  Most unfit for the
commencement of a soldier's life were some that I saw there, but I
do not doubt that they had been attracted to the work by the one
idea of doing something for their country in its trouble.

From Fort Snelling we went on to the Falls of Minnehaha.
Minnehaha, laughing water.  Such, I believe, is the interpretation.
The name in this case is more imposing than the fall.  It is a
pretty little cascade, and might do for a picnic in fine weather,
but it is not a waterfall of which a man can make much when found
so far away from home.  Going on from Minnehaha we came to
Minneapolis, at which place there is a fine suspension bridge
across the river, just above the falls of St. Anthony and leading
to the town of that name.  Till I got there I could hardly believe
that in these days there should be a living village called
Minneapolis by living men.  I presume I should describe it as a
town, for it has a municipality, and a post-office, and, of course,
a large hotel.  The interest of the place, however, is in the saw-
mills.  On the opposite side of the water, at St. Anthony, is
another very large hotel---and also a smaller one.  The smaller one
may be about the size of the first-class hotels at Cheltenham or
Leamington.  They were both closed, and there seemed to be but
little prospect that either would be opened till the war should be
over.  The saw-mills, however, were at full work, and to my eyes
were extremely picturesque.  I had been told that the beauty of the
falls had been destroyed by the mills.  Indeed, all who had spoken
to me about St. Anthony had said so.  But I did not agree with
them.  Here, as at Ottawa, the charm in fact consists, not in an
uninterrupted shoot of water, but in a succession of rapids over a
bed of broken rocks.  Among these rocks logs of loose timber are
caught, which have escaped from their proper courses, and here they
lie, heaped up in some places, and constructing themselves into
bridges in others, till the freshets of the spring carry them off.
The timber is generally brought down in logs to St. Anthony, is
sawn there, and then sent down the Mississippi in large rafts.
These rafts on other rivers are, I think, generally made of unsawn
timber.  Such logs as have escaped in the manner above described
are recognized on their passage down the river by their marks, and
are made up separately, the original owners receiving the value---or
not receiving it as the case may be.  ``There is quite a trade going
on with the loose lumber,'' my informant told me.  And from his tone
I was led to suppose that he regarded the trade as sufficiently
lucrative, if not peculiarly honest.

There is very much in the mode of life adopted by the settlers in
these regions which creates admiration.  The people are all
intelligent.  They are energetic and speculative, conceiving grand
ideas, and carrying them out almost with the rapidity of magic.  A
suspension bridge half a mile long is erected, while in England we
should be fastening together a few planks for a foot passage.
Progress, mental as well as material, is the demand of the people
generally.  Everybody understands everything, and everybody intends
sooner or later to do everything.  All this is very grand; but then
there is a terrible drawback.  One hears on every side of
intelligence, but one hears also on every side of dishonesty.  Talk
to whom you will, of whom you will, and you will hear some tale of
successful or unsuccessful swindling.  It seems to be the
recognized rule of commerce in the far West that men shall go into
the world's markets prepared to cheat and to be cheated.  It may be
said that as long as this is acknowledged and understood on all
sides, no harm will be done.  It is equally fair for all.  When I
was a child there used to be certain games at which it was agreed
in beginning either that there should be cheating or that there
should not.  It may be said that out there in the Western States,
men agree to play the cheating game; and that the cheating game has
more of interest in it than the other.  Unfortunately, however,
they who agree to play this game on a large scale do not keep
outsiders altogether out of the playground.  Indeed, outsiders
become very welcome to them; and then it is not pleasant to hear
the tone in which such outsiders speak of the peculiarities of the
sport to which they have been introduced.  When a beginner in trade
finds himself furnished with a barrel of wooden nutmegs, the joke
is not so good to him as to the experienced merchant who supplies
him.  This dealing in wooden nutmegs, this selling of things which
do not exist, and buying of goods for which no price is ever to be
given, is an institution which is much honored in the West.  We
call it swindling---and so do they.  But it seemed to me that in the
Western States the word hardly seemed to leave the same impress on
the mind that it does elsewhere.

On our return down the river we passed La Crosse, at which we had
embarked, and went down as far as Dubuque in Iowa.  On our way down
we came to grief and broke one of our paddle-wheels to pieces.  We
had no special accident.  We struck against nothing above or below
water.  But the wheel went to pieces, and we laid to on the river
side for the greater part of a day while the necessary repairs were
being made.  Delay in traveling is usually an annoyance, because it
causes the unsettlement of a settled purpose.  But the loss of the
day did us no harm, and our accident had happened at a very pretty
spot.  I climbed up to the top of the nearest bluff, and walked
back till I came to the open country, and also went up and down the
river banks, visiting the cabins of two settlers who live there by
supplying wood to the river steamers.  One of these was close to
the spot at which we were lying; and yet though most of our
passengers came on shore, I was the only one who spoke to the
inmates of the cabin.  These people must live there almost in
desolation from one year's end to another.  Once in a fortnight or
so they go up to a market town in their small boats, but beyond
that they can have little intercourse with their fellow-creatures.
Nevertheless none of these dwellers by the river side came out to
speak to the men and women who were lounging about from eleven in
the morning till four in the afternoon; nor did one of the
passengers, except myself, knock at the door or enter the cabin, or
exchange a word with those who lived there.

I spoke to the master of the house, whom I met outside, and he at
once asked me to come in and sit down.  I found his father there
and his mother, his wife, his brother, and two young children.  The
wife, who was cooking, was a very pretty, pale young woman, who,
however, could have circulated round her stove more conveniently
had her crinoline been of less dimensions.  She bade me welcome
very prettily, and went on with her cooking, talking the while, as
though she were in the habit of entertaining guests in that way
daily.  The old woman sat in a corner knitting---as old women always
do.  The old man lounged with a grandchild on his knee, and the
master of the house threw himself on the floor while the other
child crawled over him.  There was no stiffness or uneasiness in
their manners, nor was there anything approaching to that
republican roughness which so often operates upon a poor, well-
intending Englishman like a slap on the cheek.  I sat there for
about an hour, and when I had discussed with them English politics
and the bearing of English politics upon the American war, they
told me of their own affairs.  Food was very plenty, but life was
very hard.  Take the year through, each man could not earn above
half a dollar a day by cutting wood.  This, however, they owned,
did not take up all their time.  Working on favorable wood on
favorable days they could each earn two dollars a day; but these
favorable circumstances did not come together very often.  They did
not deal with the boats themselves, and the profits were eaten up
by the middleman.  He, the middleman, had a good thing of it,
because he could cheat the captains of the boats in the measurement
of the wood.  The chopper was obliged to supply a genuine cord of
logs---true measure.  But the man who took it off in the barge to
the steamer could so pack it that fifteen true cords would make
twenty-two false cords.  ``It cuts up into a fine trade, you see,
sir,'' said the young man, as he stroked back the little girl's hair
from her forehead.  ``But the captains of course must find it out,''
said I.  This he acknowledged, but argued that the captains on this
account insisted on buying the wood so much cheaper, and that the
loss all came upon the chopper.  I tried to teach him that the
remedy lay in his own hands, and the three men listened to me quite
patiently while I explained to them how they should carry on their
own trade.  But the young father had the last word.  ``I guess we
don't get above the fifty cents a day any way.''  He knew at least
where the shoe pinched him.  He was a handsome, manly, noble-
looking fellow, tall and thin, with black hair and bright eyes.
But he had the hollow look about his jaws, and so had his wife, and
so had his brother.  They all owned to fever and ague.  They had a
touch of it most years, and sometimes pretty sharply.  ``It was a
coarse place to live in,'' the old woman said, ``but there was no one
to meddle with them, and she guessed that it suited.''  They had
books and newspapers, tidy delf, and clean glass upon their
shelves, and undoubtedly provisions in plenty.  Whether fever and
ague yearly, and cords of wood stretched from fifteen to twenty-two
are more than a set-off for these good things, I will leave every
one to decide according to his own taste.

In another cabin I found women and children only, and one of the
children was in the last stage of illness.  But nevertheless the
woman of the house seemed glad to see me, and talked cheerfully as
long as I would remain.  She inquired what had happened to the
vessel, but it had never occurred to her to go out and see.  Her
cabin was neat and well furnished, and there also I saw newspapers
and Harper's everlasting magazine.  She said it was a coarse,
desolate place for living, but that she could raise almost anything
in her garden.

I could not then understand, nor can I now understand, why none of
the numerous passengers out of the boat should have entered those
cabins except myself, and why the inmates of the cabins should not
have come out to speak to any one.  Had they been surly, morose
people, made silent by the specialties of their life, it would have
been explicable; but they were delighted to talk and to listen.
The fact, I take it, is that the people are all harsh to each
other.  They do not care to go out of their way to speak to any one
unless something is to be gained.  They say that two Englishmen
meeting in the desert would not speak unless they were introduced.
The farther I travel the less true do I find this of Englishmen,
and the more true of other people.



\chapter{Ceres Americana}


We stopped at the Julien House, Dubuque.  Dubuque is a city in
Iowa, on the western shore of the Mississippi, and as the names
both of the town and of the hotel sounded French in my ears, I
asked for an explanation.  I was then told that Julien Dubuque, a
Canadian Frenchman, had been buried on one of the bluffs of the
river within the precincts of the present town; that he had been
the first white settler in Iowa, and had been the only man who had
ever prevailed upon the Indians to work.  Among them he had become
a great ``Medicine,'' and seems for awhile to have had absolute power
over them.  He died, I think, in 1800, and was buried on one of the
hills over the river.  ``He was a bold, bad man,'' my informant told
me, ``and committed every sin under heaven.  But he made the Indians
work.''

Lead mines are the glory of Dubuque, and very large sums of money
have been made from them.  I was taken out to see one of them, and
to go down it; but we found, not altogether to my sorrow, that the
works had been stopped on account of the water.  No effort has been
made in any of these mines to subdue the water, nor has steam been
applied to the working of them.  The lodes have been so rich with
lead that the speculators have been content to take out the metal
that was easily reached, and to go off in search of fresh ground
when disturbed by water.  ``And are wages here paid pretty
punctually?'' I asked.  ``Well, a man has to be smart, you know.''
And then my friend went on to acknowledge that it would be better
for the country if smartness were not so essential.

Iowa has a population of 674,000 souls, and in October, 1861, had
already mustered eighteen regiments of one thousand men each.  Such
a population would give probably 170,000 men capable of bearing
arms, and therefore the number of soldiers sent had already
amounted to more than a decimation of the available strength of the
State.  When we were at Dubuque, nothing was talked of but the
army.  It seemed that mines, coal-pits, and corn-fields were all of
no account in comparison with the war.  How many regiments could be
squeezed out of the State, was the one question which filled all
minds; and the general desire was that such regiments should be
sent to the Western army, to swell the triumph which was still
expected for General Fremont, and to assist in sweeping slavery out
into the Gulf of Mexico.  The patriotism of the West has been quite
as keen as that of the North, and has produced results as
memorable; but it has sprung from a different source, and been
conducted and animated by a different sentiment.  National
greatness and support of the law have been the idea of the North;
national greatness and abolition of slavery have been those of the
West.  How they are to agree as to terms when between them they
have crushed the South---that is the difficulty.

At Dubuque in Iowa, I ate the best apple that I ever encountered.
I make that statement with the purpose of doing justice to the
Americans on a matter which is to them one of considerable
importance.  Americans, as rule, do not believe in English apples.
They declare that there are none, and receive accounts of
Devonshire cider with manifest incredulity.  ``But at any rate there
are no apples in England equal to ours.''  That is an assertion to
which an Englishman is called upon to give an absolute assent; and
I hereby give it.  Apples so excellent as some which were given to
us at Dubuque I have never eaten in England.  There is a great
jealousy respecting all the fruits of the earth.  ``Your peaches are
fine to look at,'' was said to me, ``but they have no flavor.''  This
was the assertion of a lady, and I made no answer.  My idea had
been that American peaches had no flavor; that French peaches had
none; that those of Italy had none; that little as there might be
of which England could boast with truth, she might at any rate
boast of her peaches without fear of contradiction.  Indeed, my
idea had been that good peaches were to be got in England only.  I
am beginning to doubt whether my belief on the matter has not been
the product of insular ignorance and idolatrous self-worship.  It
may be that a peach should be a combination of an apple and a
turnip.  ``My great objection to your country, sir,'' said another,
``is that you have got no vegetables.''  Had he told me that we had
got no sea-board, or no coals, he would not have surprised me more.
No vegetables in England!  I could not restrain myself altogether,
and replied by a confession ``that we `raised' no squash.''  Squash
is the pulp of the pumpkin, and is much used in the States, both as
a vegetable and for pies.  No vegetables in England!  Did my
surprise arise from the insular ignorance and idolatrous self-
worship of a Britisher, or was my American friend laboring under a
delusion?  Is Covent Garden well supplied with vegetables, or is it
not?  Do we cultivate our kitchen-gardens with success, or am I
under a delusion on that subject?  Do I dream, or is it true that
out of my own little patches at home I have enough, for all
domestic purposes, of peas, beans, broccoli, cauliflower, celery,
beet-root, onions, carrots, parsnips, turnips, sea-kale, asparagus,
French beans, artichokes, vegetable marrow, cucumbers, tomatoes,
endive, lettuce, as well as herbs of many kinds, cabbages
throughout the year, and potatoes?  No vegetables!  Had the
gentleman told me that England did not suit him because we had
nothing but vegetables, I should have been less surprised.

From Dubuque, on the western shore of the river, we passed over to
Dunleath, in Illinois, and went on from thence by railway to Dixon.
I was induced to visit this not very flourishing town by a desire
to see the rolling prairie of Illinois, and to learn by eyesight
something of the crops of corn or Indian maize which are produced
upon the land.  Had that gentleman told me that we knew nothing of
producing corn in England, he would have been nearer the mark; for
of corn, in the profusion in which it is grown here, we do not know
much.  Better land than the prairies of Illinois for cereal crops
the world's surface probably cannot show.  And here there has been
no necessity for the long previous labor of banishing the forest.
Enormous prairies stretch across the State, into which the plow can
be put at once.  The earth is rich with the vegetation of thousands
of years, and the farmer's return is given to him without delay.
The land bursts with its own produce, and the plenty is such that
it creates wasteful carelessness in the gathering of the crop.  It
is not worth a man's while to handle less than large quantities.
Up in Minnesota I had been grieved by the loose manner in which
wheat was treated.  I have seen bags of it upset and left upon the
ground.  The labor of collecting it was more than it was worth.
There wheat is the chief crop, and as the lands become cleared and
cultivation spreads itself, the amount coming down the Mississippi
will be increased almost to infinity.  The price of wheat in Europe
will soon depend, not upon the value of the wheat in the country
which grows it, but on the power and cheapness of the modes which
may exist for transporting it.  I have not been able to obtain the
exact prices with reference to the carriage of wheat from St. Paul
(the capital of Minnesota) to Liverpool, but I have done so as
regards Indian-corn from the State of Illinois.  The following
statement will show what proportion the value of the article at the
place of its growth bears to the cost of the carriage; and it shows
also how enormous an effect on the price of corn in England would
follow any serious decrease in the cost of carriage:---%

{\small
\begin{verbatim}
A bushel of Indian-corn at Bloomington, in Illinois,
 cost, in October, 1861                             10 cents.
Freight to Chicago                                  10   ''
Storage                                              2   ''
Freight from Chicago to Buffalo                     22   ''
Elevating, and canal freight to New York            19   ''
Transfer in New York and insurance                   3   ''
Ocean freight                                       23   ''
                                                    ---------
Cost of a bushel of Indian-corn at Liverpool        89 cents.
\end{verbatim}}


Thus corn which in Liverpool costs 3s. 10d. has been sold by the
farmer who produced it for 5d.!  It is probable that no great
reduction can be expected in the cost of ocean transit; but it will
be seen by the above figures that out of the Liverpool price of 3s.
10d., or 89 cents, considerably more than half is paid for carriage
across the United States.  All or nearly all this transit is by
water; and there can, I think, be no doubt but that a few years
will see it reduced by fifty per cent.  In October last the
Mississippi was closed, the railways had not rolling stock
sufficient for their work, the crops of the two last years had been
excessive, and there existed the necessity of sending out the corn
before the internal navigation had been closed by frost.  The
parties who had the transit in their hands put their heads
together, and were able to demand any prices that they pleased.  It
will be seen that the cost of carrying a bushel of corn from
Chicago to Buffalo, by the lakes, was within one cent of the cost
of bringing it from New York to Liverpool.  These temporary causes
for high prices of transit will cease; a more perfect system of
competition between the railways and the water transit will be
organized; and the result must necessarily be both an increase of
price to the producer and a decrease of price to the consumer.  It
certainly seems that the produce of cereal crops in the valleys of
the Mississippi and its tributaries increases at a faster rate than
population increases.  Wheat and corn are sown by the thousand
acres in a piece.  I heard of one farmer who had 10,000 acres of
corn.  Thirty years ago grain and flour were sent Westward out of
the State of New York to supply the wants of those who had
immigrated into the prairies; and now we find that it will be the
destiny of those prairies to feed the universe.  Chicago is the
main point of exportation Northwestward from Illinois, and at the
present time sends out from its granaries more cereal produce than
any other town in the world.  The bulk of this passes, in the shape
of grain or flour, from Chicago to Buffalo, which latter place is,
as it were, a gateway leading from the lakes, or big waters, to the
canals, or small waters.  I give below the amount of grain and
flour in bushels received into Buffalo for transit in the month of
October during four consecutive years:---%

{\small
\begin{verbatim}
October,  1858       4,429,055 bushels.
   ''     1859       5,523,448    ''
   ''     1860       6,500,864    ''
   ''     1861      12,483,797    ''
\end{verbatim}}


In 1860, from the opening to the close of navigation, 30,837,632
bushels of grain and flour passed through Buffalo.  In 1861, the
amount received up to the 31st of October was 51,969,142 bushels.
As the navigation would be closed during the month of November, the
above figures may be taken as representing not quite the whole
amount transported for the year.  It may be presumed the 52,000,000
of bushels, as quoted above, will swell itself to 60,000,000.  I
confess that to my own mind statistical amounts do not bring home
any enduring idea.  Fifty million bushels of corn and flour simply
seems to mean a great deal.  It is a powerful form of superlative,
and soon vanishes away, as do other superlatives in this age of
strong words.  I was at Chicago and at Buffalo in October, 1861.  I
went down to the granaries and climbed up into the elevators.  I
saw the wheat running in rivers from one vessel into another, and
from the railroad vans up into the huge bins on the top stores of
the warehouses---for these rivers of food run up hill as easily as
they do down.  I saw the corn measured by the forty-bushel measure
with as much ease as we measure an ounce of cheese and with greater
rapidity.  I ascertained that the work went on, week day and
Sunday, day and night, incessantly---rivers of wheat and rivers of
maize ever running.  I saw the men bathed in corn as they
distributed it in its flow.  I saw bins by the score laden with
wheat, in each of which bins there was space for a comfortable
residence.  I breathed the flour and drank the flour, and felt
myself to be enveloped in a world of breadstuff.  And then I
believed, understood, and brought it home to myself as a fact that
here in the corn-lands of Michigan, and amid the bluffs of
Wisconsin, and on the high table plains of Minnesota, and the
prairies of Illinois had God prepared the food for the increasing
millions of the Eastern World, as also for the coming millions of
the Western.

I do not find many minds constituted like my own, and therefore I
venture to publish the above figures.  I believe them to be true in
the main; and they will show, if credited, that the increase during
the last four years has gone on with more than fabulous rapidity.
For myself, I own that those figures would have done nothing unless
I had visited the spot myself.  A man can not, perhaps count up the
results of such a work by a quick glance of his eye, nor
communicate with precision to another the conviction which his own
short experience has made so strong within himself; but to himself
seeing is believing.  To me it was so at Chicago and at Buffalo.  I
began then to know what it was for a country to overflow with milk
and honey, to burst with its own fruits and be smothered by its own
riches.  From St. Paul down the Mississippi, by the shores of
Wisconsin and Iowa; by the ports on Lake Pepin; by La Crosse, from
which one railway runs Eastward; by Prairie du Chien, the terminus
of a second; by Dunleath, Fulton, and Rock Island, from whence
three other lines run Eastward; all through that wonderful State of
Illinois, the farmer's glory; along the ports of the Great Lakes;
through Michigan, Indiana, Ohio, and further Pennsylvania, up to
Buffalo? the great gate of the Western Ceres, the loud cry was
this: ``How shall we rid ourselves of our corn and wheat?''  The
result has been the passage of 60,000,000 bushels of breadstuffs
through that gate in one year!  Let those who are susceptible of
statistics ponder that.  For them who are not I can only give this
advice: Let them go to Buffalo next October, and look for
themselves.

In regarding the above figures, and the increase shown between the
years 1860 and 1861, it must of course be borne in mind that,
during the latter autumn, no corn or wheat was carried into the
Southern States, and that none was exported from New Orleans or the
mouth of the Mississippi.  The States of Mississippi, Alabama, and
Louisiana have for some time past received much of their supplies
from the Northwestern lands; and the cutting off of this current of
consumption has tended to swell the amount of grain which has been
forced into the narrow channel of Buffalo.  There has been no
Southern exit allowed, and the Southern appetite has been deprived
of its food.  But taking this item for all that it is worth---or
taking it, as it generally will be taken, for much more than it can
be worth---the result left will be materially the same.  The grand
markets to which the Western States look and have looked are those
of New England, New York, and Europe.  Already corn and wheat are
not the common crops of New England.  Boston, and Hartford, and
Lowell are fed from the great Western States.  The State of New
York, which, thirty years ago, was famous chiefly for its cereal
produce, is now fed from these States.  New York City would be
starved if it depended on its own State; and it will soon be as
true that England would be starved if it depended on itself.  It
was but the other day that we were talking of free trade in corn as
a thing desirable, but as yet doubtful---but the other day that Lord
Derby, who may be Prime Minister to-morrow, and Mr.\ Disraeli, who
may be Chancellor of the Exchequer to-morrow, were stoutly of
opinion that the corn laws might be and should be maintained---but
the other day that the same opinion was held with confidence by Sir
Robert Peel, who, however, when the day for the change came, was
not ashamed to become the instrument used by the people for their
repeal.  Events in these days march so quickly that they leave men
behind; and our dear old Protectionists at home will have grown
sleek upon American flour before they have realized the fact that
they are no longer fed from their own furrows.

I have given figures merely as regards the trade of Buffalo; but it
must not be presumed that Buffalo is the only outlet from the great
corn-lands of Northern America.  In the first place, no grain of
the produce of Canada finds its way to Buffalo.  Its exit is by the
St. Lawrence or by the Grand Trunk Railway as I have stated when
speaking of Canada.  And then there is the passage for large
vessels from the upper lakes---Lake Michigan, Lake Huron, and Lake
Erie---through the Welland Canal, into Lake Ontario, and out by the
St. Lawrence.  There is also the direct communication from Lake
Erie, by the New York and Erie Railway to New York.  I have more
especially alluded to the trade of Buffalo, because I have been
enabled to obtain a reliable return of the quantity of grain and
flour which passes through that town, and because Buffalo and
Chicago are the two spots which are becoming most famous in the
cereal history of the Western States.

Everybody has a map of North America.  A reference to such a map
will show the peculiar position of Chicago.  It is at the south or
head of Lake Michigan, and to it converge railways from Wisconsin,
Iowa, Illinois, and Indiana.  At Chicago is found the nearest water
carriage which can be obtained for the produce of a large portion
of these States.  From Chicago there is direct water conveyance
round through the lakes to Buffalo, at the foot of Lake Erie.  At
Milwaukee, higher up on the lake, certain lines of railway come in,
joining the lake to the Upper Mississippi, and to the wheat-lands
of Minnesota.  Thence the passage is round by Detroit, which is the
port for the produce of the greatest part of Michigan, and still it
all goes on toward Buffalo.  Then on Lake Erie there are the ports
of Toledo, Cleveland, and Erie.  At the bottom of Lake Erie there
is this city of corn, at which the grain and flour are transhipped
into the canal-boats and into the railway cars for New York; and
there is also the Welland Canal, through which large vessels pass
from the upper lakes without transhipment of their cargo.

I have said above that corn---meaning maize or Indian-corn---was to
be bought at Bloomington, in Illinois, for ten cents (or five
pence) a bushel.  I found this also to be the case at Dixon, and
also that corn of inferior quality might be bought for four pence;
but I found also that it was not worth the farmer's while to shell
it and sell it at such prices.  I was assured that farmers were
burning their Indian-corn in some places, finding it more available
to them as fuel than it was for the market.  The labor of detaching
a bushel of corn from the hulls or cobs is considerable, as is also
the task of carrying it to market.  I have known potatoes in
Ireland so cheap that they would not pay for digging and carrying
away for purposes of sale.  There was then a glut of potatoes in
Ireland; and in the same way there was, in the autumn of 1861, a
glut of corn in the Western States.  The best qualities would fetch
a price, though still a low price; but corn that was not of the
best quality was all but worthless.  It did for fuel, and was
burned.  The fact was that the produce had re-created itself
quicker than mankind had multiplied.  The ingenuity of man had not
worked quick enough for its disposal.  The earth had given forth
her increase so abundantly that the lap of created humanity could
not stretch itself to hold it.  At Dixon, in 1861, corn cost four
pence a bushel.  In Ireland, in 1848, it was sold for a penny a
pound, a pound being accounted sufficient to sustain life for a
day; and we all felt that at that price food was brought into the
country cheaper than it had ever been brought before.

Dixon is not a town of much apparent prosperity.  It is one of
those places at which great beginnings have been made, but as to
which the deities presiding over new towns have not been
propitious.  Much of it has been burned down, and more of it has
never been built up.  It had a straggling, ill-conditioned,
uncommercial aspect, very different from the look of Detroit,
Milwaukee, or St. Paul.  There was, however, a great hotel there,
as usual, and a grand bridge over the Rock River, a tributary of
the Mississippi, which runs by or through the town.  I found that
life might be maintained on very cheap terms at Dixon.  To me, as a
passing traveler, the charges at the hotel were, I take it, the
same as elsewhere.  But I learned from an inmate there that he,
with his wife and horse, were fed and cared for and attended, for
two dollars (or eight shillings and four pence) a day.  This
included a private sitting-room, coals, light, and all the wants of
life---as my informant told me---except tobacco and whisky.  Feeding
at such a house means a succession of promiscuous hot meals, as
often as the digestion of the patient can face them.  Now I do not
know any locality where a man can keep himself and his wife, with
all material comforts and the luxury of a horse and carriage, on
cheaper terms than that.  Whether or no it might be worth a man's
while to live at all at such a place as Dixon, is altogether
another question.

We went there because it is surrounded by the prairie, and out into
the prairie we had ourselves driven.  We found some difficulty in
getting away from the corn, though we had selected this spot as one
at which the open rolling prairie was specially attainable.  As
long as I could see a corn-field or a tree I was not satisfied.
Nor, indeed, was I satisfied at last.  To have been thoroughly on
the prairie, and in the prairie, I should have been a day's journey
from tilled land.  But I doubt whether that could now be done in
the State of Illinois.  I got out into various patches and brought
away specimens of corn---ears bearing sixteen rows of grain, with
forty grains in each row, each ear bearing a meal for a hungry man.

At last we did find ourselves on the prairie, amid the waving
grass, with the land rolling on before us in a succession of gentle
sweeps, never rising so as to impede the view, or apparently
changing in its general level, but yet without the monotony of
flatness.  We were on the prairie, but still I felt no
satisfaction.  It was private property, divided among holders and
pastured over by private cattle.  Salisbury Plain is as wild, and
Dartmoor almost wilder.  Deer, they told me, were to be had within
reach of Dixon, but for the buffalo one has to go much farther
afield than Illinois.  The farmer may rejoice in Illinois, but the
hunter and the trapper must cross the big rivers and pass away into
the Western Territories before he can find lands wild enough for
his purposes.  My visit to the corn-fields of Illinois was in its
way successful, but I felt, as I turned my face eastward toward
Chicago, that I had no right to boast that I had as yet made
acquaintance with a prairie.

All minds were turned to the war, at Dixon as elsewhere.  In
Illinois the men boasted that, as regards the war, they were the
leading State of the union.  But the same boast was made in
Indiana, and also in Massachusetts, and probably in half the States
of the North and West.  They, the Illinoisians, call their country
the war-nest of the West.  The population of the State is
1,700,000, and it had undertaken to furnish sixty volunteer
regiments of 1000 men each.  And let it be borne in mind that these
regiments, when furnished, are really full---absolutely containing
the thousand men when they are sent away from the parent States.
The number of souls above named will give 420,000 working men, and
if, out of these, 60,000 are sent to the war, the State, which is
almost purely agricultural, will have given more than one man in
eight.  When I was in Illinois, over forty regiments had already
been sent---forty-six, if I remember rightly---and there existed no
doubt whatever as to the remaining number.  From the next State,
Indiana, with a population of 1,350,000, giving something less than
350,000 working men, thirty-six regiments had been sent.  I fear
that I am mentioning these numbers usque ad nauseam; but I wish to
impress upon English readers the magnitude of the effort made by
the States in mustering and equipping an army within six or seven
months of the first acknowledgment that such an army would be
necessary.  The Americans have complained bitterly of the want of
English sympathy, and I think they have been weak in making that
complaint.  But I would not wish that they should hereafter have
the power of complaining of a want of English justice.  There can
be no doubt that a genuine feeling of patriotism was aroused
throughout the North and West, and that men rushed into the ranks
actuated by that feeling, men for whom war and army life, a camp
and fifteen dollars a month; would not of themselves have had any
attraction.  It came to that, that young men were ashamed not to go
into the army.  This feeling of course produced coercion, and the
movement was in that way tyrannical.  There is nothing more
tyrannical than a strong popular feeling among a democratic people.
During the period of enlistment this tyranny was very strong.  But
the existence of such a tyranny proves the passion and patriotism
of the people.  It got the better of the love of money, of the love
of children, and of the love of progress.  Wives who with their
bairns were absolutely dependent on their husbands' labors, would
wish their husbands to be at the war.  Not to conduce, in some
special way, toward the war; to have neither father there, nor
brother nor son; not to have lectured, or preached, or written for
the war; to have made no sacrifice for the war, to have had no
special and individual interest in the war, was disgraceful.  One
sees at a glance the tyranny of all this in such a country as the
States.  One can understand how quickly adverse stories would
spread themselves as to the opinion of any man who chose to remain
tranquil at such a time.  One shudders at the absolute absence of
true liberty which such a passion throughout a democratic country
must engender.  But he who has observed all this must acknowledge
that that passion did exist.  Dollars, children, progress,
education, and political rivalry all gave way to the one strong
national desire for the thrashing and crushing of those who had
rebelled against the authority of the stars and stripes.

When we were at Dixon they were getting up the Dement regiment.
The attempt at the time did not seem to be prosperous, and the few
men who had been collected had about them a forlorn, ill-
conditioned look.  But then, as I was told, Dixon had already been
decimated and redecimated by former recruiting colonels.  Colonel
Dement, from whom the regiment was to be named, and whose military
career was only now about to commence, had come late into the
field.  I did not afterward ascertain what had been his success,
but I hardly doubt that he did ultimately scrape together his
thousand men.  ``Why don't you go?'' I said to a burly Irishman who
was driving me.  ``I'm not a sound man, yer honor,'' said the
Irishman; ``I'm deficient in me liver.''  Taking the Irishmen,
however, throughout the Union, they had not been found deficient in
any of the necessaries for a career of war.  I do not think that
any men have done better than the Irish in the American army.

From Dixon we went to Chicago.  Chicago is in many respects the
most remarkable city among all the remarkable cities of the Union.
Its growth has been the fastest and its success the most assured.
Twenty-five years ago there was no Chicago, and now it contains
120,000 inhabitants.  Cincinnati, on the Ohio, and St. Louis, at
the junction of the Missouri and Mississippi, are larger towns; but
they have not grown large so quickly nor do they now promise so
excessive a development of commerce.  Chicago may be called the
metropolis of American corn---the favorite city haunt of the
American Ceres.  The goddess seats herself there amid the dust of
her full barns, and proclaims herself a goddess ruling over things
political and philosophical as well as agricultural.  Not furrows
only are in her thoughts, but free trade also and brotherly love.
And within her own bosom there is a boast that even yet she will be
stronger than Mars.  In Chicago there are great streets, and rows
of houses fit to be the residences of a new Corn-Exchange nobility.
They look out on the wide lake which is now the highway for
breadstuffs, and the merchant, as he shaves at his window, sees his
rapid ventures as they pass away, one after the other, toward the
East.

I went over one great grain store in Chicago possessed by gentlemen
of the name of Sturgess and Buckenham.  It was a world in itself,
and the dustiest of all the worlds.  It contained, when I was
there, half a million bushels of wheat---or a very great many, as I
might say in other language.  But it was not as a storehouse that
this great building was so remarkable, but as a channel or a river-
course for the flooding freshets of corn.  It is so built that both
railway vans and vessels come immediately under its claws, as I may
call the great trunks of the elevators.  Out of the railway vans
the corn and wheat is clawed up into the building, and down similar
trunks it is at once again poured out into the vessels.  I shall be
at Buffalo in a page or two, and then I will endeavor to explain
more minutely how this is done.  At Chicago the corn is bought and
does change hands; and much of it, therefore, is stored there for
some space of time, shorter or longer as the case may be.  When I
was at Chicago, the only limit to the rapidity of its transit was
set by the amount of boat accommodation.  There were not bottoms
enough to take the corn away from Chicago, nor, indeed, on the
railway was there a sufficiency of rolling stock or locomotive
power to bring it into Chicago.  As I said before, the country was
bursting with its own produce and smothered in its own fruits.

At Chicago the hotel was bigger than other hotels and grander.
There were pipes without end for cold water which ran hot, and for
hot water which would not run at all.  The post-office also was
grander and bigger than other post-offices, though the postmaster
confessed to me that that matter of the delivery of letters was one
which could not be compassed.  Just at that moment it was being
done as a private speculation; but it did not pay, and would be
discontinued.  The theater, too, was large, handsome, and
convenient; but on the night of my attendance it seemed to lack an
audience.  A good comic actor it did not lack, and I never laughed
more heartily in my life.  There was something wrong, too, just at
that time---I could not make out what---in the Constitution of
Illinois, and the present moment had been selected for voting a new
Constitution.  To us in England such a necessity would be
considered a matter of importance, but it did not seem to be much
thought of here, ``Some slight alteration probably,'' I suggested.
``No,'' said my informant, one of the judges of their courts, ``it is
to be a thorough, radical change of the whole Constitution.  They
are voting the delegates to-day.''  I went to see them vote the
delegates, but, unfortunately, got into a wrong place---by
invitation---and was turned out, not without some slight tumult.  I
trust that the new Constitution was carried through successfully.

From these little details it may, perhaps, be understood how a town
like Chicago goes on and prospers in spite of all the drawbacks
which are incident to newness.  Men in those regions do not mind
failures, and, when they have failed, instantly begin again.  They
make their plans on a large scale, and they who come after them
fill up what has been wanting at first.  Those taps of hot and cold
water will be made to run by the next owner of the hotel, if not by
the present owner.  In another ten years the letters, I do not
doubt, will all be delivered.  Long before that time the theater
will probably be full.  The new Constitution is no doubt already at
work, and, if found deficient, another will succeed to it without
any trouble to the State or any talk on the subject through the
Union.  Chicago was intended as a town of export for corn, and
therefore the corn stores have received the first attention.  When
I was there they were in perfect working order.

From Chicago we went on to Cleveland, a town in the State of Ohio,
on Lake Erie, again traveling by the sleeping-cars.  I found that
these cars were universally mentioned with great horror and disgust
by Americans of the upper class.  They always declared that they
would not travel in them on any account.  Noise and dirt were the
two objections.  They are very noisy, but to us belonged the happy
power of sleeping down noise.  I invariably slept all through the
night, and knew nothing about the noise.  They are also very dirty---%
extremely dirty---dirty so as to cause much annoyance.  But then
they are not quite so dirty as the day cars.  If dirt is to be a
bar against traveling in America, men and women must stay at home.
For myself, I don't much care for dirt, having a strong reliance on
soap and water and scrubbing-brushes.  No one regards poisons who
carries antidotes in which he has perfect faith.

Cleveland is another pleasant town---pleasant as Milwaukee and
Portland.  The streets are handsome and are shaded by grand avenues
of trees.  One of these streets is over a mile in length, and
throughout the whole of it there are trees on each side---not
little, paltry trees as are to be seen on the boulevards of Paris,
but spreading elms: the beautiful American elm, which not only
spreads, but droops also, and makes more of its foliage than any
other tree extant.  And there is a square in Cleveland, well sized,
as large as Russell Square I should say, with open paths across it,
and containing one or two handsome buildings.  I cannot but think
that all men and women in London would be great gainers if the iron
rails of the squares were thrown down and the grassy inclosures
thrown open to the public.  Of course the edges of the turf would
be worn, and the paths would not keep their exact shapes.  But the
prison look would be banished, and the somber sadness of the
squares would be relieved.

I was particularly struck by the size and comfort of the houses at
Cleveland.  All down that street of which I have spoken they do not
stand continuously together, but are detached and separate---houses
which in England would require some fifteen or eighteen hundred a
year for their maintenance.  In the States, however, men commonly
expend upon house rent a much greater proportion of their income
than they do in England.  With us it is, I believe, thought that a
man should certainly not apportion more than a seventh of his
spending income to his house rent---some say not more than a tenth.
But in many cities of the States a man is thought to live well
within bounds if he so expends a fourth.  There can be no doubt as
to Americans living in better houses than Englishmen, making the
comparison of course between men of equal incomes.  But the
Englishman has many more incidental expenses than the American.  He
spends more on wine, on entertainments, on horses, and on
amusements.  He has a more numerous establishment, and keeps up the
adjuncts and outskirts of his residence with a more finished
neatness.

These houses in Cleveland were very good, as, indeed, they are in
most Northern towns; but some of them have been erected with an
amount of bad taste that is almost incredible.  It is not uncommon
to see in front of a square brick house a wooden quasi-Greek
portico, with a pediment and Ionic columns, equally high with the
house itself.  Wooden columns with Greek capitals attached to the
doorways, and wooden pediments over the windows, are very frequent.
As a rule, these are attached to houses which, without such
ornamentation, would be simple, unpretentious, square, roomy
residences.  An Ionic or Corinthian capital stuck on to a log of
wood called a column, and then fixed promiscuously to the outside
of an ordinary house, is to my eye the vilest of architectural
pretenses.  Little turrets are better than this, or even brown
battlements made of mortar.  Except in America I do not remember to
have seen these vicious bits of white timber---timber painted white---%
plastered on to the fronts and sides of red brick houses.

Again we went on by rail to Buffalo.  I have traveled some
thousands of miles by railway in the States, taking long journeys
by night and longer journeys by day; but I do not remember that
while doing so I ever made acquaintance with an American.  To an
American lady in a railway car I should no more think of speaking
than I should to an unknown female in the next pew to me at a
London church.  It is hard to understand from whence come the laws
which govern societies in this respect; but there are different
laws in different societies, which soon obtain recognition for
themselves.  American ladies are much given to talking, and are
generally free from all mauvaise honte.  They are collected in
manner, well instructed, and resolved to have their share of the
social advantages of the world.  In this phase of life they come
out more strongly than English women.  But on a railway journey, be
it ever so long, they are never seen speaking to a stranger.
English women, however, on English railways are generally willing
to converse: they will do so if they be on a journey; but will not
open their mouths if they be simply passing backward and forward
between their homes and some neighboring town.  We soon learn the
rules on these subjects; but who make the rules?  If you cross the
Atlantic with an American lady you invariably fall in love with her
before the journey is over.  Travel with the same woman in a
railway car for twelve hours, and you will have written her down in
your own mind in quite other language than that of love.

And now for Buffalo, and the elevators.  I trust I have made it
understood that corn comes into Buffalo, not only from Chicago, of
which I have spoken specially, but from all the ports round the
lakes: Racine, Milwaukee, Grand Haven, Port Sarnia, Detroit,
Toledo, Cleveland, and many others.  At these ports the produce is
generally bought and sold; but at Buffalo it is merely passed
through a gateway.  It is taken from vessels of a size fitted for
the lakes, and placed in other vessels fitted for the canal.  This
is the Erie Canal, which connects the lakes with the Hudson River
and with New York.  The produce which passes through the Welland
Canal---the canal which connects Lake Erie and the upper lakes with
Lake Ontario and the St. Lawrence---is not transhipped, seeing that
the Welland Canal, which is less than thirty miles in length, gives
a passage to vessels of 500 tons.  As I have before said,
60,000,000 bushels of breadstuff were thus pushed through Buffalo
in the open months of the year 1861.  These open months run from
the middle of April to the middle of November; but the busy period
is that of the last two months---the time, that is, which intervenes
between the full ripening of the corn and the coming of the ice.

An elevator is as ugly a monster as has been yet produced.  In
uncouthness of form it outdoes those obsolete old brutes who used
to roam about the semi-aqueous world, and live a most uncomfortable
life with their great hungering stomachs and huge unsatisfied maws.
The elevator itself consists of a big movable trunk---movable as is
that of an elephant, but not pliable, and less graceful even than
an elephant's.  This is attached to a huge granary or barn; but in
order to give altitude within the barn for the necessary moving up
and down of this trunk---seeing that it cannot be curled gracefully
to its purposes as the elephant's is curled---there is an awkward
box erected on the roof of the barn, giving some twenty feet of
additional height, up into which the elevator can be thrust.  It
will be understood, then, that this big movable trunk, the head of
which, when it is at rest, is thrust up into the box on the roof,
is made to slant down in an oblique direction from the building to
the river; for the elevator is an amphibious institution, and
flourishes only on the banks of navigable waters.  When its head is
ensconced within its box, and the beast of prey is thus nearly
hidden within the building, the unsuspicious vessel is brought up
within reach of the creature's trunk, and down it comes, like a
musquito's proboscis, right through the deck, in at the open
aperture of the hole, and so into the very vitals and bowels of the
ship.  When there, it goes to work upon its food with a greed and
an avidity that is disgusting to a beholder of any taste or
imagination.  And now I must explain the anatomical arrangement by
which the elevator still devours and continues to devour, till the
corn within its reach has all been swallowed, masticated, and
digested.  Its long trunk, as seen slanting down from out of the
building across the wharf and into the ship, is a mere wooden pipe;
but this pipe is divided within.  It has two departments; and as
the grain-bearing troughs pass up the one on a pliable band, they
pass empty down the other.  The system, therefore, is that of an
ordinary dredging machine only that corn and not mud is taken away,
and that the buckets or troughs are hidden from sight.  Below,
within the stomach of the poor bark, three or four laborers are at
work, helping to feed the elevator.  They shovel the corn up toward
its maw, so that at every swallow he should take in all that he can
hold.  Thus the troughs, as they ascend, are kept full, and when
they reach the upper building they empty themselves into a shoot,
over which a porter stands guard, moderating the shoot by a door,
which the weight of his finger can open and close.  Through this
doorway the corn runs into a measure, and is weighed.  By measures
of forty bushels each, the tale is kept.  There stands the
apparatus, with the figures plainly marked, over against the
porter's eye; and as the sum mounts nearly up to forty bushels he
closes the door till the grains run thinly through, hardly a
handful at a time, so that the balance is exactly struck.  Then the
teller standing by marks down his figure, and the record is made.
The exact porter touches the string of another door, and the forty
bushels of corn run out at the bottom of the measure, disappear
down another shoot, slanting also toward the water, and deposit
themselves in the canal boat.  The transit of the bushels of corn
from the larger vessel to the smaller will have taken less than a
minute, and the cost of that transit will have been---a farthing.

But I have spoken of the rivers of wheat, and I must explain what
are those rivers.  In the working of the elevator, which I have
just attempted to describe, the two vessels were supposed to be
lying at the same wharf on the same side of the building, in the
same water, the smaller vessel inside the larger one.  When this is
the case the corn runs direct from the weighing measure into the
shoot that communicates with the canal boat.  But there is not room
or time for confining the work to one side of the building.  There
is water on both sides, and the corn or wheat is elevated on the
one side, and reshipped on the other.  To effect this the corn is
carried across the breadth of the building; but, nevertheless, it
is never handled or moved in its direction on trucks or carriages
requiring the use of men's muscles for its motion.  Across the
floor of the building are two gutters, or channels, and through
these, small troughs on a pliable band circulate very quickly.
They which run one way, in one channel, are laden; they which
return by the other channel are empty.  The corn pours itself into
these, and they again pour it into the shoot which commands the
other water.  And thus rivers of corn are running through these
buildings night and day.  The secret of all the motion and
arrangement consists, of course, in the elevation.  The corn is
lifted up; and when lifted up can move itself and arrange itself,
and weigh itself, and load itself.

I should have stated that all this wheat which passes through
Buffalo comes loose, in bulk.  Nothing is known of sacks or bags.
To any spectator at Buffalo this becomes immediately a matter of
course; but this should be explained, as we in England are not
accustomed to see wheat traveling in this open, unguarded, and
plebeian manner.  Wheat with us is aristocratic, and travels always
in its private carriage.

Over and beyond the elevators there is nothing specially worthy of
remark at Buffalo.  It is a fine city, like all other American
cities of its class.  The streets are broad, the ``blocks'' are high,
and cars on tramways run all day, and nearly all night as well.



\chapter{Buffalo to New York}


We had now before us only two points of interest before we should
reach New York---the Falls of Trenton, and West Point on the Hudson
River.  We were too late in the year to get up to Lake George,
which lies in the State of New York north of Albany, and is, in
fact, the southern continuation of Lake Champlain.  Lake George, I
know, is very lovely, and I would fain have seen it; but visitors
to it must have some hotel accommodation, and the hotel was closed
when we were near enough to visit it.  I was in its close
neighborhood three years since, in June; but then the hotel was not
yet opened.  A visitor to Lake George must be very exact in his
time.  July and August are the months---with, perhaps, the grace of
a week in September.

The hotel at Trenton was also closed, as I was told.  But even if
there were no hotel at Trenton, it can be visited without
difficulty.  It is within a carriage drive of Utica, and there is,
moreover, a direct railway from Utica, with a station at the
Trenton Falls.  Utica is a town on the line of railway from Buffalo
to New York via Albany, and is like all the other towns we had
visited.  There are broad streets, and avenues of trees, and large
shops, and excellent houses.  A general air of fat prosperity
pervades them all, and is strong at Utica as elsewhere.

I remember to have been told, thirty years ago, that a traveler
might go far and wide in search of the picturesque without finding
a spot more romantic in its loveliness than Trenton Falls.  The
name of the river is Canada Creek West; but as that is hardly
euphonious, the course of the water which forms the falls has been
called after the town or parish.  This course is nearly two miles
in length; and along the space of this two miles it is impossible
to say where the greatest beauty exists.  To see Trenton aright,
one must be careful not to have too much water.  A sufficiency is
no doubt desirable; and it may be that at the close of summer,
before any of the autumnal rains have fallen, there may
occasionally be an insufficiency.  But if there be too much, the
passage up the rocks along the river is impossible.  The way on
which the tourist should walk becomes the bed of the stream, and
the great charm of the place cannot be enjoyed.  That charm
consists in descending into the ravine of the river, down amid the
rocks through which it has cut its channel, and in walking up the
bed against the stream, in climbing the sides of the various falls,
and sticking close to the river till an envious block is reached
which comes sheer down into the water and prevents farther
progress.  This is nearly two miles above the steps by which the
descent is made; and not a foot of this distance but is wildly
beautiful.  When the river is very low there is a pathway even
beyond that block; but when this is the case there can hardly be
enough of water to make the fall satisfactory.

There is no one special cataract at Trenton which is in itself
either wonderful or pre-eminently beautiful.  It is the position,
form, color, and rapidity of the river which gives the charm.  It
runs through a deep ravine, at the bottom of which the water has
cut for itself a channel through the rocks, the sides of which rise
sometimes with the sharpness of the walls of a stone sarcophagus.
They are rounded, too, toward the bed as I have seen the bottom of
a sarcophagus.  Along the side of the right bank of the river there
is a passage which, when the freshets come, is altogether covered.
This passage is sometimes very narrow; but in the narrowest parts
an iron chain is affixed into the rock.  It is slippery and wet;
and it is well for ladies, when visiting the place, to be provided
with outside India-rubber shoes, which keep a hold upon the stone.
If I remember rightly, there are two actual cataracts---one not far
above the steps by which the descent is made into the channel, and
the other close under a summer-house, near to which the visitors
reascend into the wood.  But these cataracts, though by no means
despicable as cataracts, leave comparatively a slight impression.
They tumble down with sufficient violence and the usual fantastic
disposition of their forces; but simply as cataracts within a day's
journey of Niagara, they would be nothing.  Up beyond the summer-
house the passage along the river can be continued for another
mile; but it is rough, and the climbing in some places rather
difficult for ladies.  Every man, however, who has the use of his
legs should do it; for the succession of rapids, and the twistings
of the channels, and the forms of the rocks are as wild and
beautiful as the imagination can desire.  The banks of the river
are closely wooded on each side; and though this circumstance does
not at first seem to add much to the beauty, seeing that the ravine
is so deep that the absence of wood above would hardly be noticed,
still there are broken clefts ever and anon through which the
colors of the foliage show themselves, and straggling boughs and
rough roots break through the rocks here and there, and add to the
wildness and charm of the whole.

The walk back from the summer-house through the wood is very
lovely; but it would be a disappointing walk to visitors who had
been prevented by a flood in the river from coming up the channel,
for it indicates plainly how requisite it is that the river should
be seen from below and not from above.  The best view of the larger
fall itself is that seen from the wood.  And here again I would
point out that any male visitor should walk the channel of the
river up and down.  The descent is too slippery and difficult for
bipeds laden with petticoats.  We found a small hotel open at
Trenton, at which we got a comfortable dinner, and then in the
evening were driven back to Utica.

Albany is the capital of the State of New York, and our road from
Trenton to West Point lay through that town; but these political
State capitals have no interest in themselves.  The State
legislature was not sitting; and we went on, merely remarking that
the manner in which the railway cars are made to run backward and
forward through the crowded streets of the town must cause a
frequent loss of human life.  One is led to suppose that children
in Albany can hardly have a chance of coming to maturity.  Such
accidents do not become the subject of long-continued and strong
comment in the States as they do with us; but nevertheless I should
have thought that such a state of things as we saw there would have
given rise to some remark on the part of the philanthropists.  I
cannot myself say that I saw anybody killed, and therefore should
not be justified in making more than this passing remark on the
subject.

When first the Americans of the Northern States began to talk much
of their country, their claims as to fine scenery were confined to
Niagara and the Hudson River.  Of Niagara I have spoken; and all
the world has acknowledged that no claim made on that head can be
regarded as exaggerated.  As to the Hudson I am not prepared to say
so much generally, though there is one spot upon it which cannot be
beaten for sweetness.  I have been up and down the Hudson by water,
and confess that the entire river is pretty.  But there is much of
it that is not pre-eminently pretty among rivers.  As a whole, it
cannot be named with the Upper Mississippi, with the Rhine, with
the Moselle, or with the Upper Rhone.  The palisades just out of
New York are pretty, and the whole passage through the mountains
from West Point up to Catskill and Hudson is interesting.  But the
glory of the Hudson is at West Point itself; and thither on this
occasion we went direct by railway, and there we remained for two
days.  The Catskill Mountains should be seen by a detour from off
the river.  We did not visit them, because here again the hotel was
closed.  I will leave them, therefore, for the new hand book which
Mr.\ Murray will soon bring out.

Of West Point there is something to be said independently of its
scenery.  It is the Sandhurst of the States.  Here is their
military school, from which officers are drafted to their
regiments, and the tuition for military purposes is, I imagine, of
a high order.  It must of course be borne in mind that West Point,
even as at present arranged, is fitted to the wants of the old
army, and not to that of the army now required.  It can go but a
little way to supply officers for 500,000 men; but would do much
toward supplying them for 40,000.  At the time of my visit to West
Point the regular army of the Northern States had not even then
swelled itself to the latter number.

I found that there were 220 students at West Point; that about
forty graduate every year, each of whom receives a commission in
the army; that about 120 pupils are admitted every year; and that
in the course of every year about eighty either resign, or are
called upon to leave on account of some deficiency, or fail in
their final examination.  The result is simply this, that one-third
of those who enter succeeds, and that two-thirds fail.  The number
of failures seemed to me to be terribly large---so large as to give
great ground of hesitation to a parent in accepting a nomination
for the college.  I especially inquired into the particulars of
these dismissals and resignations, and was assured that the
majority of them take place in the first year of the pupilage.  It
is soon seen whether or no a lad has the mental and physical
capacities necessary for the education and future life required of
him, and care is taken that those shall be removed early as to whom
it may be determined that the necessary capacity is clearly
wanting.  If this is done---and I do not doubt it---the evil is much
mitigated.  The effect otherwise would be very injurious.  The lads
remain till they are perhaps one and twenty, and have then acquired
aptitudes for military life, but no other aptitudes.  At that age
the education cannot be commenced anew, and, moreover, at that age
the disgrace of failure is very injurious.  The period of education
used to be five years, but has now been reduced to four.  This was
done in order that a double class might be graduated in 1861 to
supply the wants of the war.  I believe it is considered that but
for such necessity as that, the fifth year of education can be ill
spared.

The discipline, to our English ideas, is very strict.  In the first
place no kind of beer, wine, or spirits is allowed at West Point.
The law upon this point may be said to be very vehement, for it
debars even the visitors at the hotel from the solace of a glass of
beer.  The hotel is within the bounds of the college, and as the
lads might become purchasers at the bar, there is no bar allowed.
Any breach of this law leads to instant expulsion; or, I should say
rather, any detection of such breach.  The officer who showed us
over the college assured me that the presence of a glass of wine in
a young man's room would secure his exclusion, even though there
should be no evidence that he had tasted it.  He was very firm as
to this; but a little bird of West Point, whose information, though
not official or probably accurate in words, seemed to me to be
worthy of reliance in general, told me that eyes were wont to wink
when such glasses of wine made themselves unnecessarily visible.
Let us fancy an English mess of young men from seventeen to twenty-
one, at which a mug of beer would be felony and a glass of wine
high treason!  But the whole management of the young with the
Americans differs much from that in vogue with us.  We do not
require so much at so early an age, either in knowledge, in morals,
or even in manliness.  In America, if a lad be under control, as at
West Point, he is called upon for an amount of labor and a degree
of conduct which would be considered quite transcendental and out
of the question in England.  But if he be not under control, if at
the age of eighteen he be living at home, or be from his
circumstances exempt from professorial power, he is a full-fledged
man, with his pipe apparatus and his bar acquaintances.

And then I was told, at West Point, how needful and yet how painful
it was that all should be removed who were in any way deficient in
credit to the establishment.  ``Our rules are very exact,'' my
informant told me; ``but the carrying out of our rules is a task not
always very easy.''  As to this also I had already heard something
from that little bird of West Point; but of course I wisely
assented to my informant, remarking that discipline in such an
establishment was essentially necessary.  The little bird had told
me that discipline at West Point had been rendered terribly
difficult by political interference.  ``A young man will be
dismissed by the unanimous voice of the board, and will be sent
away.  And then, after a week or two, he will be sent back, with an
order from Washington that another trial shall be given him.  The
lad will march back into the college with all the honors of a
victory, and will be conscious of a triumph over the superintendent
and his officers.''  ``And is that common?'' I asked.  ``Not at the
present moment,'' I was told.  ``But it was common before the war.
While Mr.\ Buchanan, and Mr.\ Pierce, and Mr.\ Polk were Presidents,
no officer or board of officers then at West Point was able to
dismiss a lad whose father was a Southerner, and who had friends
among the government.''

Not only was this true of West Point, but the same allegation is
true as to all matters of patronage throughout the United States.
During the three or four last presidencies, and I believe back to
the time of Jackson, there has been an organized system of
dishonesty in the management of all beneficial places under the
control of the government.  I doubt whether any despotic court of
Europe has been so corrupt in the distribution of places---that is,
in the selection of public officers---as has been the assemblage of
statesmen at Washington.  And this is the evil which the country is
now expiating with its blood and treasure.  It has allowed its
knaves to stand in the high places; and now it finds that knavish
works have brought about evil results.  But of this I shall be
constrained to say something further hereafter.

We went into all the schools of the college, and made ourselves
fully aware that the amount of learning imparted was far above our
comprehension.  It always occurs to me, in looking through the new
schools of the present day, that I ought to be thankful to persons
who know so much for condescending to speak to me at all in plain
English.  I said a word to the gentleman who was with me about
horses, seeing a lot of lads going to their riding lesson.  But he
was down upon me, and crushed me instantly beneath the weight of my
own ignorance.  He walked me up to the image of a horse, which he
took to pieces, bit by bit, taking off skin, muscle, flesh, nerves,
and bones, till the animal was a heap of atoms, and assured me that
the anatomy of the horse throughout was one of the necessary
studies of the place.  We afterward went to see the riding.  The
horses themselves were poor enough.  This was accounted for by the
fact that such of them as had been found fit for military service
had been taken for the use of the army.

There is a gallery in the college in which are hung sketches and
pictures by former students.  I was greatly struck with the merit
of many of these.  There were some copies from well-known works of
art of very high excellence, when the age is taken into account of
those by whom they were done.  I don't know how far the art of
drawing, as taught generally, and with no special tendency to
military instruction, may be necessary for military training; but
if it be necessary I should imagine that more is done in that
direction at West Point than at Sandhurst.  I found, however, that
much of that in the gallery, which was good, had been done by lads
who had not obtained their degree, and who had shown an aptitude
for drawing, but had not shown any aptitude for other pursuits
necessary to their intended career.

And then we were taken to the chapel, and there saw, displayed as
trophies, two of our own dear old English flags.  I have seen many
a banner hung up in token of past victory, and many a flag taken on
the field of battle mouldering by degrees into dust on some
chapel's wall---but they have not been the flags of England.  Till
this day I had never seen our own colors in any position but one of
self-assertion and independent power.  From the tone used by the
gentleman who showed them to me, I could gather that he would have
passed them by, had he not foreseen that he could not do so without
my notice.  ``I don't know that we are right to put them there,'' he
said.  ``Quite right,'' was my reply, ``as long as the world does such
things.''  In private life it is vulgar to triumph over one's
friends, and malicious to triumph over one's enemies.  We have not
got so far yet in public life, but I hope we are advancing toward
it.  In the mean time I did not begrudge the Americans our two
flags.  If we keep flags and cannons taken from our enemies, and
show them about as signs of our own prowess after those enemies
have become friends, why should not others do so as regards us?  It
clearly would not be well for the world that we should always beat
other nations and never be beaten.  I did not begrudge that chapel
our two flags.  But, nevertheless, the sight of them made me sick
in the stomach and uncomfortable.  As an Englishman I do not want
to be ascendant over any one.  But it makes me very ill when any
one tries to be ascendant over me.  I wish we could send back with
our compliments all the trophies that we hold, carriage paid, and
get back in return those two flags, and any other flag or two of
our own that may be doing similar duty about the world.  I take it
that the parcel sent away would be somewhat more bulky than that
which would reach us in return.

The discipline at West Point seemed, as I have said, to be very
severe; but it seemed also that that severity could not in all
cases be maintained.  The hours of study also were long, being
nearly continuous throughout the day.  ``English lads of that age
could not do it,'' I said; thus confessing that English lads must
have in them less power of sustained work than those of America.
``They must do it here,'' said my informant, ``or else leave us.''  And
then he took us off to one of the young gentlemen's quarters, in
order that we might see the nature of their rooms.  We found the
young gentleman fast asleep on his bed, and felt uncommonly grieved
that we should have thus intruded on him.  As the hour was one of
those allocated by my informant in the distribution of the day to
private study, I could not but take the present occupation of the
embryo warrior as an indication that the amount of labor required
might be occasionally too much even for an American youth.  ``The
heat makes one so uncommonly drowsy,'' said the young man.  I was
not the least surprised at the exclamation.  The air of the
apartment had been warmed up to such a pitch by the hot-pipe
apparatus of the building that prolonged life to me would, I should
have thought, be out of the question in such an atmosphere.  ``Do
you always have it as hot as this?'' I asked.  The young man swore
that it was so, and with considerable energy expressed his opinion
that all his health, and spirits, and vitality were being baked out
of him.  He seemed to have a strong opinion on the matter, for
which I respected him; but it had never occurred to him, and did
not then occur to him, that anything could be done to moderate that
deathly flow of hot air which came up to him from the neighboring
infernal regions.  He was pale in the face, and all the lads there
were pale.  American lads and lasses are all pale.  Men at thirty
and women at twenty-five have had all semblance of youth baked out
of them.  Infants even are not rosy, and the only shades known on
the cheeks of children are those composed of brown, yellow, and
white.  All this comes of those damnable hot-air pipes with which
every tenement in America is infested.  ``We cannot do without
them,'' they say.  ``Our cold is so intense that we must heat our
houses throughout.  Open fire-places in a few rooms would not keep
our toes and fingers from the frost.''  There is much in this.  The
assertion is no doubt true, and thereby a great difficulty is
created.  It is no doubt quite within the power of American
ingenuity to moderate the heat of these stoves, and to produce such
an atmosphere as may be most conducive to health.  In hospitals no
doubt this will be done; perhaps is done at present---though even in
hospitals I have thought the air hotter than it should be.  But
hot-air drinking is like dram-drinking.  There is the machine
within the house capable of supplying any quantity, and those who
consume it unconsciously increase their draughts, and take their
drains stronger and stronger, till a breath of fresh air is felt to
be a blast direct from Boreas.

West Point is at all points a military colony, and, as such,
belongs exclusively to the Federal government as separate from the
government of any individual State.  It is the purchased property
of the United States as a whole, and is devoted to the necessities
of a military college.  No man could take a house there, or succeed
in getting even permanent lodgings, unless he belonged to or were
employed by the establishment.  There is no intercourse by road
between West Point and other towns or villages on the river side,
and any such intercourse even by water is looked upon with jealousy
by the authorities.  The wish is that West Point should be isolated
and kept apart for military instruction to the exclusion of all
other purposes whatever---especially love-making purposes.  The
coming over from the other side of the water of young ladies by the
ferry is regarded as a great hinderance.  They will come, and then
the military students will talk to them.  We all know to what such
talking leads!  A lad when I was there had been tempted to get out
of barracks in plain clothes, in order that he might call on a
young lady at the hotel; and was in consequence obliged to abandon
his commission and retire from the Academy.  Will that young lady
ever again sleep quietly in her bed?  I should hope not.  An
opinion was expressed to me that there should be no hotel in such a
place---that there should be no ferry, no roads, no means by which
the attention of the students should be distracted---that these
military Rasselases should live in a happy military valley from
which might be excluded both strong drinks and female charms---those
two poisons from which youthful military ardor is supposed to
suffer so much.

It always seems to me that such training begins at the wrong end.
I will not say that nothing should be done to keep lads of eighteen
from strong drinks.  I will not even say that there should not be
some line of moderation with reference to feminine allurements.
But, as a rule, the restraint should come from the sense, good
feeling, and education of him who is restrained.  There is no
embargo on the beer-shops either at Harrow or at Oxford---and
certainly none upon the young ladies.  Occasional damage may accrue
from habits early depraved, or a heart too early and too easily
susceptible; but the injury so done is not, I think, equal to that
inflicted by a Draconian code of morals, which will probably be
evaded, and will certainly create a desire for its evasion.

Nevertheless, I feel assured that West Point, taken as a whole, is
an excellent military academy, and that young men have gone forth
from it, and will go forth from it, fit for officers as far as
training can make men fit.  The fault, if fault there be, is that
which is to be found in so many of the institutions of the United
States, and is one so allied to a virtue, that no foreigner has a
right to wonder that it is regarded in the light of a virtue by all
Americans.  There has been an attempt to make the place too
perfect.  In the desire to have the establishment self-sufficient
at all points, more has been attempted than human nature can
achieve.  The lad is taken to West Point, and it is presumed that
from the moment of his reception he shall expend every energy of
his mind and body in making himself a soldier.  At fifteen he is
not to be a boy, at twenty he is not to be a young man.  He is to
be a gentleman, a soldier, and an officer.  I believe that those
who leave the college for the army are gentlemen, soldiers, and
officers, and, therefore, the result is good.  But they are also
young men; and it seems that they have become so, not in accordance
with their training, but in spite of it.

But I have another complaint to make against the authorities of
West Point, which they will not be able to answer so easily as that
already preferred.  What right can they have to take the very
prettiest spot on the Hudson---the prettiest spot on the continent---%
one of the prettiest spots which Nature, with all her vagaries,
ever formed---and shut it up from all the world for purposes of war?
Would not any plain, however ugly, do for military exercises?
Cannot broadsword, goose-step, and double-quick time be instilled
into young hands and legs in any field of thirty, forty, or fifty
acres?  I wonder whether these lads appreciate the fact that they
are studying fourteen hours a day amid the sweetest river, rock,
and mountain scenery that the imagination can conceive.  Of course
it will be said, that the world at large is not excluded from West
Point, that the ferry to the place is open, and that there is even
a hotel there, closed against no man or woman who will consent to
become a teetotaller for the period of his visit.  I must admit
that this is so; but still one feels that one is only admitted as a
guest.  I want to go and live at West Point, and why should I be
prevented?  The government had a right to buy it of course, but
government should not buy up the prettiest spots on a country's
surface.  If I were an American, I should make a grievance of this;
but Americans will suffer things from their government which no
Englishmen would endure.

It is one of the peculiarities of West Point that everything there
is in good taste.  The point itself consists of a bluff of land so
formed that the River Hudson is forced to run round three sides of
it.  It is consequently a peninsula; and as the surrounding country
is mountainous on both sides of the river, it may be imagined that
the site is good.  The views both up and down the river are lovely,
and the mountains behind break themselves so as to make the
landscape perfect.  But this is not all.  At West Point there is
much of buildings, much of military arrangement in the way of
cannons, forts, and artillery yards.  All these things are so
contrived as to group themselves well into pictures.  There is no
picture of architectural grandeur; but everything stands well and
where it should stand, and the eye is not hurt at any spot.  I
regard West Point as a delightful place, and was much gratified by
the kindness I received there.

From West Point we went direct to new York.



\chapter{An Apology for the War}


I think it may be received as a fact that the Northern States,
taken together, sent a full tenth of their able-bodied men into the
ranks of the army in the course of the summer and autumn of 1861.
The South, no doubt, sent a much larger proportion; but the effect
of such a drain upon the South would not be the same, because the
slaves were left at home to perform the agricultural work of the
country.  I very much doubt whether any other nation ever made such
an effort in so short a time.  To a people who can do this it may
well be granted that they are in earnest; and I do not think it
should be lightly decided by any foreigner that they are wrong.
The strong and unanimous impulse of a great people is seldom wrong.
And let it be borne in mind that in this case both people may be
right---the people both of North and South.  Each may have been
guided by a just and noble feeling, though each was brought to its
present condition by bad government and dishonest statesmen.

There can be no doubt that, since the commencement of the war the
American feeling against England has been very bitter.  All
Americans to whom I spoke on the subject admitted that it was so.
I, as an Englishman, felt strongly the injustice of this feeling,
and lost no opportunity of showing, or endeavoring to show, that
the line of conduct pursued by England toward the States was the
only line which was compatible with her own policy and just
interests and also with the dignity of the States government.  I
heard much of the tender sympathy of Russia.  Russia sent a
flourishing general message, saying that she wished the North might
win, and ending with some good general advice proposing peace.  It
was such a message as strong nations send to those which are
weaker.  Had England ventured on such counsel, the diplomatic paper
would probably have been returned to her.  It is, I think, manifest
that an absolute and disinterested neutrality has been the only
course which could preserve England from deserved rebuke---a
neutrality on which her commercial necessity for importing cotton
or exporting her own manufactures should have no effect.  That our
government would preserve such a neutrality I have always insisted;
and I believe it has been done with a pure and strict disregard to
any selfish views on the part of Great Britain.  So far I think
England may feel that she has done well in this matter.  But I must
confess that I have not been so proud of the tone of all our people
at home as I have been of the decisions of our statesmen.  It seems
to me that some of us never tire in abusing the Americans, and
calling them names for having allowed themselves to be driven into
this civil war.  We tell them that they are fools and idiots; we
speak of their doings as though there had been some plain course by
which the war might have been avoided; and we throw it in their
teeth that they have no capability for war.  We tell them of the
debt which they are creating, and point out to them that they can
never pay it.  We laugh at their attempt to sustain loyalty, and
speak of them as a steady father of a family is wont to speak of
some unthrifty prodigal who is throwing away his estate and
hurrying from one ruinous debauchery to another.  And, alas! we too
frequently allow to escape from us some expression of that
satisfaction which one rival tradesman has in the downfall of
another.  ``Here you are with all your boasting,'' is what we say.
``You were going to whip all creation the other day; and it has come
to this!  Brag is a good dog, but Holdfast is a better.  Pray
remember that, if ever you find yourselves on your legs again.''
That little advice about the two dogs is very well, and was not
altogether inapplicable.  But this is not the time in which it
should be given.  Putting aside slight asperities, we will all own
that the people of the States have been and are our friends, and
that as friends we cannot spare them.  For one Englishman who
brings home to his own heart a feeling of cordiality for France---a
belief in the affection of our French alliance---there are ten who
do so with reference to the States.  Now, in these days of their
trouble, I think that we might have borne with them more tenderly.

And how was it possible that they should have avoided this war?  I
will not now go into the cause of it, or discuss the course which
it has taken, but will simply take up the fact of the rebellion.
The South rebelled against the North; and such being the case, was
it possible that the North should yield without a war?  It may very
likely be well that Hungary should be severed from Austria, or
Poland from Russia, or Venice from Austria.  Taking Englishmen in a
lump, they think that such separation would be well.  The subject
people do not speak the language of those that govern them or enjoy
kindred interests.  But yet when military efforts are made by those
who govern Hungary, Poland, and Venice to prevent such separation,
we do not say that Russia and Austria are fools.  We are not
surprised that they should take up arms against the rebels, but
would be very much surprised indeed if they did not do so.  We know
that nothing but weakness would prevent their doing so.  But if
Austria and Russia insist on tying to themselves a people who do
not speak their language or live in accordance with their habits,
and are not considered unreasonable in so insisting, how much more
thoroughly would they carry with them the sympathy of their
neighbors in preventing any secession by integral parts of their
own nationalities!  Would England let Ireland walk off by herself,
if she wished it?  In 1843 she did wish it.  Three-fourths of the
Irish population would have voted for such a separation; but
England would have prevented such a secession vi et armis, had
Ireland driven her to the necessity of such prevention.

I will put it to any reader of history whether, since government
commenced, it has not been regarded as the first duty of government
to prevent a separation of the territories governed; and whether,
also, it has not been regarded as a point of honor with all
nationalities to preserve uninjured each its own greatness and its
own power?  I trust that I may not be thought to argue that all
governments, or even all nationalities, should succeed in such
endeavors.  Few kings have fallen, in my day, in whose fate I have
not rejoiced---none, I take it, except that poor citizen King of the
French.  And I can rejoice that England lost her American colonies,
and shall rejoice when Spain has been deprived of Cuba.  But I hold
that citizen King of the French in small esteem, seeing that he
made no fight; and I know that England was bound to struggle when
the Boston people threw her tea into the water.  Spain keeps a
tighter hand on Cuba than we thought she would some ten years
since, and therefore she stands higher in the world's respect.

It may be well that the South should be divided from the North.  I
am inclined to think that it would be well---at any rate for the
North; but the South must have been aware that such division could
only be effected in two ways: either by agreement, in which case
the proposition must have been brought forward by the South and
discussed by the North, or by violence.  They chose the latter way,
as being the readier and the surer, as most seceding nations have
done.  O'Connell, when struggling for the secession of Ireland,
chose the other, and nothing came of it.  The South chose violence,
and prepared for it secretly and with great adroitness.  If that be
not rebellion, there never has been rebellion since history began;
and if civil war was ever justified in one portion of a nation by
turbulence in another, it has now been justified in the Northern
States of America.

What was the North to do; this foolish North, which has been so
liberally told by us that she has taken up arms for nothing, that
she is fighting for nothing, and will ruin herself for nothing?
When was she to take the first step toward peace?  Surely every
Englishman will remember that when the earliest tidings of the
coming quarrel reached us on the election of Mr.\ Lincoln, we all
declared that any division was impossible; it was a mere madness to
speak of it.  The States, which were so great in their unity, would
never consent to break up all their prestige and all their power by
a separation!  Would it have been well for the North then to say,
``If the South wish it we will certainly separate?''  After that,
when Mr.\ Lincoln assumed the power to which he had been elected,
and declared with sufficient manliness, and sufficient dignity
also, that he would make no war upon the South, but would collect
the customs and carry on the government, did we turn round and
advise him that he was wrong?  No.  The idea in England then was
that his message was, if anything, too mild.  ``If he means to be
President of the whole Union,'' England said, ``he must come out with
something stronger than that.''  Then came Mr.\ Seward's speech,
which was, in truth, weak enough.  Mr.\ Seward had ran Mr.\ Lincoln
very hard for the President's chair on the Republican interest, and
was, most unfortunately, as I think, made Secretary of State by Mr.\ %
Lincoln, or by his party.  The Secretary of State holds the highest
office in the United States government under the President.  He
cannot be compared to our Prime Minister, seeing that the President
himself exercises political power, and is responsible for its
exercise.  Mr.\ Seward's speech simply amounted to a declaration
that separation was a thing of which the Union would neither hear,
speak, nor, if possible, think.  Things looked very like it; but
no, they could never come to that!  The world was too good, and
especially the American world.  Mr.\ Seward had no specific against
secession; but let every free man strike his breast, look up to
heaven, determine to be good, and all would go right.  A great deal
had been expected from Mr.\ Seward, and when this speech came out,
we in England were a little disappointed, and nobody presumed even
then that the North would let the South go.

It will be argued by those who have gone into the details of
American politics that an acceptance of the Crittenden compromise
at this point would have saved the war.  What is or was the
Crittenden compromise I will endeavor to explain hereafter; but the
terms and meaning of that compromise can have no bearing on the
subject.  The Republican party who were in power disapproved of
that compromise, and could not model their course upon it.  The
Republican party may have been right or may have been wrong; but
surely it will not be argued that any political party elected to
power by a majority should follow the policy of a minority, lest
that minority should rebel.  I can conceive of no government more
lowly placed than one which deserts the policy of the majority
which supports it, fearing either the tongues or arms of a
minority.

As the next scene in the play, the State of South Carolina
bombarded Fort Sumter.  Was that to be the moment for a peaceable
separation?  Let us suppose that O'Connell had marched down to the
Pigeon House, at Dublin, and had taken it, in 1843, let us say,
would that have been an argument to us for allowing Ireland to set
up for herself?  Is that the way of men's minds, or of the minds of
nations?  The powers of the President were defined by law, as
agreed upon among all the States of the Union, and against that
power and against that law South Carolina raised her hand, and the
other States joined her in rebellion.  When circumstances had come
to that, it was no longer possible that the North should shun the
war.  To my thinking the rights of rebellion are holy.  Where would
the world have been, or where would the world hope to be, without
rebellion?  But let rebellion look the truth in the face, and not
blanch from its own consequences.  She has to judge her own
opportunities and to decide on her own fitness.  Success is the
test of her judgment.  But rebellion can never be successful except
by overcoming the power against which she raises herself.  She has
no right to expect bloodless triumphs; and if she be not the
stronger in the encounter which she creates, she must bear the
penalty of her rashness.  Rebellion is justified by being better
served than constituted authority, but cannot be justified
otherwise.  Now and again it may happen that rebellion's cause is
so good that constituted authority will fall to the ground at the
first glance of her sword.  This was so the other day in Naples,
when Garibaldi blew away the king's armies with a breath.  But this
is not so often.  Rebellion knows that it must fight, and the
legalized power against which rebels rise must of necessity fight
also.

I cannot see at what point the North first sinned; nor do I think
that had the North yielded, England would have honored her for her
meekness.  Had she yielded without striking a blow, she would have
been told that she had suffered the Union to drop asunder by her
supineness.  She would have been twitted with cowardice, and told
that she was no match for Southern energy.  It would then have
seemed to those who sat in judgment on her that she might have
righted everything by that one blow from which she had abstained.
But having struck that one blow, and having found that it did not
suffice, could she then withdraw, give way, and own herself beaten?
Has it been so usually with Anglo-Saxon pluck?  In such case as
that, would there have been no mention of those two dogs, Brag and
Holdfast?  The man of the Northern States knows that he has
bragged---bragged as loudly as his English forefathers.  In that
matter of bragging, the British lion and the star-spangled banner
may abstain from throwing mud at each other.  And now the Northern
man wishes to show that he can hold fast also.  Looking at all this
I cannot see that peace has been possible to the North.

As to the question of secession and rebellion being one and the
same thing, the point to me does not seem to bear an argument.  The
confederation of States had a common army, a common policy, a
common capital, a common government, and a common debt.  If one
might secede, any or all might secede, and where then would be
their property, their debt, and their servants?  A confederation
with such a license attached to it would have been simply playing
at national power.  If New York had seceded---a State which
stretches from the Atlantic to British North America---it would have
cut New England off from the rest of the Union.  Was it legally
within the power of New York to place the six States of New England
in such a position?  And why should it be assumed that so suicidal
a power of destroying a nationality should be inherent in every
portion of the nation?  The Slates are bound together by a written
compact, but that compact gives each State no such power.  Surely
such a power would have been specified had it been intended that it
should be given.  But there are axioms in politics as in
mathematics, which recommend themselves to the mind at once, and
require no argument for their proof.  Men who are not argumentative
perceive at once that they are true.  A part cannot be greater than
the whole.

I think it is plain that the remnant of the Union was bound to take
up arms against those States which had illegally torn themselves
off from her; and if so, she could only do so with such weapons as
were at her hand.  The United States army had never been numerous
or well appointed; and of such officers and equipments as it
possessed, the more valuable part was in the hands of the
Southerners.  It was clear enough that she was ill provided, and
that in going to war she was undertaking a work as to which she had
still to learn many of the rudiments.  But Englishmen should be the
last to twit her with such ignorance.  It is not yet ten years
since we were all boasting that swords and guns were useless
things, and that military expenditure might be cut down to any
minimum figure that an economizing Chancellor of the Exchequer
could name.  Since that we have extemporized two if not three
armies.  There are our volunteers at home; and the army which holds
India can hardly be considered as one with that which is to
maintain our prestige in Europe and the West.  We made some natural
blunders in the Crimea, but in making those blunders we taught
ourselves the trade.  It is the misfortune of the Northern States
that they must learn these lessons in fighting their own
countrymen.  In the course of our history we have suffered the same
calamity more than once.  The Round-heads, who beat the Cavaliers
and created English liberty, made themselves soldiers on the bodies
of their countrymen.  But England was not ruined by that civil war;
nor was she ruined by those which preceded it.  From out of these
she came forth stronger than she entered them---stronger, better,
and more fit for a great destiny in the history of nations.  The
Northern States had nearly five hundred thousand men under arms
when the winter of 1861 commenced, and for that enormous multitude
all commissariat requirements were well supplied.  Camps and
barracks sprang up through the country as though by magic.
Clothing was obtained with a rapidity that has I think, never been
equaled.  The country had not been prepared for the fabrication of
arms, and yet arms were put into the men's hands almost as quickly
as the regiments could be mustered.  The eighteen millions of the
Northern States lent themselves to the effort as one man.  Each
State gave the best it had to give.  Newspapers were as rabid
against each other as ever, but no newspaper could live which did
not support the war.  ``The South has rebelled against the law, and
the law shall be supported.''  This has been the cry and the
heartfelt feeling of all men; and it is a feeling which cannot but
inspire respect.

We have heard much of the tyranny of the present government of the
United States, and of the tyranny also of the people.  They have
both been very tyrannical.  The ``habeas corpus'' has been suspended
by the word of one man.  Arrests have been made on men who have
been hardly suspected of more than secession principles.  Arrests
have, I believe, been made in cases which have been destitute even
of any fair ground for such suspicion.  Newspapers have been
stopped for advocating views opposed to the feelings of the North,
as freely as newspapers were ever stopped in France for opposing
the Emperor.  A man has not been safe in the streets who was known
to be a secessionist.  It must be at once admitted that opinion in
the Northern States was not free when I was there.  But has opinion
ever been free anywhere on all subjects?  In the best built
strongholds of freedom, have there not always been questions on
which opinion has not been free; and must it not always be so?
When the decision of a people on any matter has become, so to say,
unanimous---when it has shown itself to be so general as to be
clearly the expression of the nation's voice as a single chorus,
that decision becomes holy, and may not be touched.  Could any
newspaper be produced in England which advocated the overthrow of
the Queen?  And why may not the passion for the Union be as strong
with the Northern States, as the passion for the Crown is strong
with us?  The Crown with us is in no danger, and therefore the
matter is at rest.  But I think we must admit that in any nation,
let it be ever so free, there may be points on which opinion must
be held under restraint.  And as to those summary arrests, and the
suspension of the ``habeas corpus,'' is there not something to be
said for the States government on that head also?  Military arrests
are very dreadful, and the soul of a nation's liberty is that
personal freedom from arbitrary interference which is signified to
the world by those two unintelligible Latin words.  A man's body
shalt not be kept in duress at any man's will, but shall be brought
up into open court, with uttermost speed, in order that the law may
say whether or no it should be kept in duress.  That I take it is
the meaning of ``habeas corpus,'' and it is easy to see that the
suspension of that privilege destroys all freedom, and places the
liberty of every individual at the mercy of him who has the power
to suspend it.  Nothing can be worse than this: and such
suspension, if extended over any long period of years, will
certainly make a nation weak, mean spirited, and poor.  But in a
period of civil war, or even of a widely-extended civil commotion,
things cannot work in their accustomed grooves.  A lady does not
willingly get out of her bedroom-window with nothing on but her
nightgown; but when her house is on fire she is very thankful for
an opportunity of doing so.  It is not long since the ``habeas
corpus'' was suspended in parts of Ireland, and absurd arrests were
made almost daily when that suspension first took effect.  It was
grievous that there should be necessity for such a step; and it is
very grievous now that such necessity should be felt in the
Northern States.  But I do not think that it becomes Englishmen to
bear hardly upon Americans generally for what has been done in that
matter.  Mr.\ Seward, in an official letter to the British Minister
at Washington---which letter, through official dishonesty, found its
way to the press---claimed for the President the right of suspending
the ``habeas corpus'' in the States whenever it might seem good to
him to do so.  If this be in accordance with the law of the land,
which I think must be doubted, the law of the land is not favorable
to freedom.  For myself, I conceive that Mr.\ Lincoln and Mr.\ Seward
have been wrong in their law, and that no such right is given to
the President by the Constitution of the United States.  This I
will attempt to prove in some subsequent chapter.  But I think it
must be felt by all who have given any thought to the Constitution
of the States, that let what may be the letter of the law, the
Presidents of the United States have had no such power.  It is
because the States have been no longer united, that Mr.\ Lincoln has
had the power, whether it be given to him by the law or no.

And then as to the debt; it seems to me very singular that we in
England should suppose that a great commercial people would be
ruined by a national debt.  As regards ourselves, I have always
looked on our national debt as the ballast in our ship.  We have a
great deal of ballast, but then the ship is very big.  The States
also are taking in ballast at a rather rapid rate; and we too took
it in quickly when we were about it.  But I cannot understand why
their ship should not carry, without shipwreck, that which our ship
has carried without damage, and, as I believe, with positive
advantage to its sailing.  The ballast, if carried honestly, will
not, I think, bring the vessel to grief.  The fear is lest the
ballast should be thrown overboard.

So much I have said wishing to plead the cause of the Northern
States before the bar of English opinion, and thinking that there
is ground for a plea in their favor.  But yet I cannot say that
their bitterness against Englishmen has been justified, or that
their tone toward England has been dignified.  Their complaint is
that they have received no sympathy from England; but it seems to
me that a great nation should not require an expression of sympathy
during its struggle.  Sympathy is for the weak rather than for the
strong.  When I hear two powerful men contending together in
argument, I do not sympathize with him who has the best of it; but
I watch the precision of his logic and acknowledge the effects of
his rhetoric.  There has been a whining weakness in the complaints
made by Americans against England, which has done more to lower
them as a people in my judgment than any other part of their
conduct during the present crisis.  When we were at war with
Russia, the feeling of the States was strongly against us.  All
their wishes were with our enemies.  When the Indian mutiny was at
its worst, the feeling of France was equally adverse to us.  The
joy expressed by the French newspapers was almost ecstatic.  But I
do not think that on either occasion we bemoaned ourselves sadly on
the want of sympathy shown by our friends.  On each occasion we
took the opinion expressed for what it was worth, and managed to
live it down.  We listened to what was said, and let it pass by.
When in each case we had been successful, there was an end of our
friends' croakings.

But in the Northern States of America the bitterness against
England has amounted almost to a passion.  The players---those
chroniclers of the time---have had no hits so sure as those which
have been aimed at Englishmen as cowards, fools, and liars.  No
paper has dared to say that England has been true in her American
policy.  The name of an Englishman has been made a by-word for
reproach.  In private intercourse private amenities have remained.
I, at any rate, may boast that such has been the case as regards
myself.  But, even in private life, I have been unable to keep down
the feeling that I have always been walking over smothered ashes.

It may be that, when the civil war in America is over, all this
will pass by, and there will be nothing left of international
bitterness but its memory.  It is sincerely to be hoped that this
may be so---that even the memory of the existing feeling may fade
away and become unreal.  I for one cannot think that two nations
situated as are the States and England should permanently quarrel
and avoid each other.  But words have been spoken which will, I
fear, long sound in men's ears, and thoughts have sprung up which
will not easily allow themselves to be extinguished.



\chapter{New York}


Speaking of New York as a traveler, I have two faults to find with
it.  In the first place, there is nothing to see; and, in the
second place, there is no mode of getting about to see anything.
Nevertheless, New York is a most interesting city.  It is the third
biggest city in the known world, for those Chinese congregations of
unwinged ants are not cities in the known world.  In no other city
is there a population so mixed and cosmopolitan in their modes of
life.  And yet in no other city that I have seen are there such
strong and ever visible characteristics of the social and political
bearings of the nation to which it belongs.  New York appears to me
as infinitely more American than Boston, Chicago, or Washington.
It has no peculiar attribute of its own, as have those three
cities---Boston in its literature and accomplished intelligence,
Chicago in its internal trade, and Washington in its Congressional
and State politics.  New York has its literary aspirations, its
commercial grandeur, and, Heaven knows, it has its politics also.
But these do not strike the visitor as being specially
characteristic of the city.  That it is pre-eminently American is
its glory or its disgrace, as men of different ways of thinking may
decide upon it.  Free institutions, general education, and the
ascendency of dollars are the words written on every paving-stone
along Fifth Avenue, down Broadway, and up Wall Street.  Every man
can vote, and values the privilege.  Every man can read, and uses
the privilege.  Every man worships the dollar, and is down before
his shrine from morning to night.

As regards voting and reading, no American will be angry with me
for saying so much of him; and no Englishman, whatever may be his
ideas as to the franchise in his own country, will conceive that I
have said aught to the dishonor of an American.  But as to that
dollar-worshiping, it will of course seem that I am abusing the New
Yorkers.  We all know what a wretchedly wicked thing money is---how
it stands between us and heaven---how it hardens our hearts and
makes vulgar our thoughts!  Dives has ever gone to the devil, while
Lazarus has been laid up in heavenly lavender.  The hand that
employs itself in compelling gold to enter the service of man has
always been stigmatized as the ravisher of things sacred.  The
world is agreed about that, and therefore the New Yorker is in a
bad way.  There are very few citizens in any town known to me which
under this dispensation are in a good way, but the New Yorker is in
about the worst way of all.  Other men, the world over, worship
regularly at the shrine with matins and vespers, nones and
complines, and whatever other daily services may be known to the
religious houses; but the New Yorker is always on his knees.

That is the amount of the charge which I bring against New York;
and now, having laid on my paint thickly, I shall proceed, like an
unskillful artist, to scrape a great deal of it off again.  New
York has been a leading commercial city in the world for not more
than fifty or sixty years.  As far as I can learn, its population
at the close of the last century did not exceed 60,000, and ten
years later it had not reached 100,000.  In 1860 it had reached
nearly 800,000 in the City of New York itself.  To this number must
be added the numbers of Brooklyn, Williamsburg, and Jersey City, in
order that a true conception may be had of the population of this
American metropolis, seeing that those places are as much a part of
New York as Southwark is of London.  By this the total will be
swelled to considerably above a million.  It will no doubt be
admitted that this growth has been very fast, and that New York may
well be proud of it.  Increase of population is, I take it, the
only trustworthy sign of a nation's success or of a city's success.
We boast that London has beaten the other cities of the world, and
think that that boast is enough to cover all the social sins for
which London has to confess her guilt.  New York, beginning with
60,000 sixty years since, has now a million souls---a million
mouths, all of which eat a sufficiency of bread, all of which speak
ore rotundo, and almost all of which can read.  And this has come
of its love of dollars.

For myself I do not believe that Dives is so black as he is painted
or that his peril is so imminent.  To reconcile such an opinion
with holy writ might place me in some difficulty were I a
clergyman.  Clergymen, in these days, are surrounded by
difficulties of this nature---finding it necessary to explain away
many old-established teachings which narrowed the Christian Church,
and to open the door wide enough to satisfy the aspirations and
natural hopes of instructed men.  The brethren of Dives are now so
many and so intelligent that they will no longer consent to be
damned without looking closely into the matter themselves.  I will
leave them to settle the matter with the Church, merely assuring
them of my sympathy in their little difficulties in any case in
which mere money causes the hitch.

To eat his bread in the sweat of his brow was man's curse in Adam's
day, but is certainly man's blessing in our day.  And what is
eating one's bread in the sweat of one's brow but making money?  I
will believe no man who tells me that he would not sooner earn two
loaves than one---and if two, then two hundred.  I will believe no
man who tells me that he would sooner earn one dollar a day than
two---and if two, then two hundred.  That is, in the very nature of
the argument, caeteris paribus.  When a man tells me that he would
prefer one honest loaf to two that are dishonest, I will, in all
possible cases, believe him.  So also a man may prefer one quiet
loaf to two that are unquiet.  But under circumstances that are the
same, and to a man who is sane, a whole loaf is better than half,
and two loaves are better than one.  The preachers have preached
well, but on this matter they have preached in vain.  Dives has
never believed that he will be damned because he is Dives.  He has
never even believed that the temptations incident to his position
have been more than a fair counterpoise, or even so much as a fair
counterpoise, to his opportunities for doing good.  All men who
work desire to prosper by their work, and they so desire by the
nature given to them from God.  Wealth and progress must go on hand
in hand together, let the accidents which occasionally divide them
for a time happen as often as they may.  The progress of the
Americans has been caused by their aptitude for money-making; and
that continual kneeling at the shrine of the coined goddess has
carried them across from New York to San Francisco.  Men who kneel
at that shrine are called on to have ready wits and quick hands,
and not a little aptitude for self-denial.  The New Yorker has been
true to his dollar because his dollar has been true to him.

But not on this account can I, nor on this account will any
Englishman, reconcile himself to the savor of dollars which
pervades the atmosphere of New York.  The ars celare artem is
wanting.  The making of money is the work of man; but he need not
take his work to bed with him, and have it ever by his side at
table, amid his family, in church, while he disports himself, as he
declares his passion to the girl of his heart, in the moments of
his softest bliss, and at the periods of his most solemn
ceremonies.  That many do so elsewhere than in New York---in London,
for instance, in Paris, among the mountains of Switzerland, and the
steppes of Russia---I do not doubt.  But there is generally a vail
thrown over the object of the worshiper's idolatry.  In New York
one's ear is constantly filled with the fanatic's voice as he
prays, one's eyes are always on the familiar altar.  The
frankincense from the temple is ever in one's nostrils.  I have
never walked down Fifth Avenue alone without thinking of money.  I
have never walked there with a companion without talking of it.  I
fancy that every man there, in order to maintain the spirit of the
place, should bear on his forehead a label stating how many dollars
he is worth, and that every label should be expected to assert a
falsehood.

I do not think that New York has been less generous in the use of
its money than other cities, or that the men of New York generally
are so.  Perhaps I might go farther and say that in no city has
more been achieved for humanity by the munificence of its richest
citizens than in New York.  Its hospitals, asylums, and
institutions for the relief of all ailments to which flesh is heir,
are very numerous, and beyond praise in the excellence of their
arrangements.  And this has been achieved in a great degree by
private liberality.  Men in America are not as a rule anxious to
leave large fortunes to their children.  The millionaire when
making his will very generally gives back a considerable portion of
the wealth which he has made to the city in which he made it.  The
rich citizen is always anxious that the poor citizen shall be
relieved.  It is a point of honor with him to raise the character
of his municipality, and to provide that the deaf and dumb, the
blind, the mad, the idiots, the old, and the incurable shall have
such alleviation in their misfortune as skill and kindness can
afford.

Nor is the New Yorker a hugger-mugger with his money.  He does not
hide up his dollars in old stockings and keep rolls of gold in
hidden pots.  He does not even invest it where it will not grow but
only produce small though sure fruit.  He builds houses, he
speculates largely, he spreads himself in trade to the extent of
his wings---and not seldom somewhat farther.  He scatters his wealth
broadcast over strange fields, trusting that it may grow with an
increase of a hundredfold, but bold to bear the loss should the
strange field prove itself barren.  His regret at losing his money
is by no means commensurate with his desire to make it.  In this
there is a living spirit which to me divests the dollar-worshiping
idolatry of something of its ugliness.  The hand when closed on the
gold is instantly reopened.  The idolator is anxious to get, but he
is anxious also to spend.  He is energetic to the last, and has no
comfort with his stock unless it breeds with Transatlantic rapidity
of procreation.

So much I say, being anxious to scrape off some of that daub of
black paint with which I have smeared the face of my New Yorker;
but not desiring to scrape it all off.  For myself, I do not love
to live amid the clink of gold, and never have ``a good time,'' as
the Americans say, when the price of shares and percentages come up
in conversation.  That state of men's minds here which I have
endeavored to explain tends, I think, to make New York
disagreeable.  A stranger there who has no great interest in
percentages soon finds himself anxious to escape.  By degrees he
perceives that he is out of his element, and had better go away.
He calls at the bank, and when he shows himself ignorant as to the
price at which his sovereigns should be done, he is conscious that
he is ridiculous.  He is like a man who goes out hunting for the
first time at forty years of age.  He feels himself to be in the
wrong place, and is anxious to get out of it.  Such was my
experience of New York, at each of the visits that I paid to it.

But yet, I say again, no other American city is so intensely
American as New York.  It is generally considered that the
inhabitants of New England, the Yankees properly so called, have
the American characteristics of physiognomy in the fullest degree.
The lantern jaws, the thin and lithe body, the dry face on which
there has been no tint of the rose since the baby's long-clothes
were first abandoned, the harsh, thick hair, the thin lips, the
intelligent eyes, the sharp voice with the nasal twang---not
altogether harsh, though sharp and nasal---all these traits are
supposed to belong especially to the Yankee.  Perhaps it was so
once, but at present they are, I think, more universally common in
New York than in any other part of the States.  Go to Wall Street,
the front of the Astor House, and the regions about Trinity Church,
and you will find them in their fullest perfection.

What circumstances of blood or food, of early habit or subsequent
education, have created for the latter-day American his present
physiognomy?  It is as completely marked, as much his own, as is
that of any race under the sun that has bred in and in for
centuries.  But the American owns a more mixed blood than any other
race known.  The chief stock is English, which is itself so mixed
that no man can trace its ramifications.  With this are mingled the
bloods of Ireland, Holland, France, Sweden, and Germany.  All this
has been done within but a few years, so that the American may be
said to have no claim to any national type of face.  Nevertheless,
no man has a type of face so clearly national as the American.  He
is acknowledged by it all over the continent of Europe, and on his
own side of the water is gratified by knowing that he is never
mistaken for his English visitor.  I think it comes from the hot-
air pipes and from dollar worship.  In the Jesuit his mode of
dealing with things divine has given a peculiar cast of
countenance; and why should not the American be similarly moulded
by his special aspirations?  As to the hot-air pipes, there can, I
think, be no doubt that to them is to be charged the murder of all
rosy cheeks throughout the States.  If the effect was to be noticed
simply in the dry faces of the men about Wall Street, I should be
very indifferent to the matter.  But the young ladies of Fifth
Avenue are in the same category.  The very pith and marrow of life
is baked out of their young bones by the hot-air chambers to which
they are accustomed.  Hot air is the great destroyer of American
beauty.

In saying that there is very little to be seen in New York I have
also said that there is no way of seeing that little.  My assertion
amounts to this; that there are no cabs.  To the reading world at
large this may not seem to be much, but let the reading world go to
New York, and it will find out how much the deficiency means.  In
London, in Paris, in Florence, in Rome, in the Havana, or at Grand
Cairo, the cab-driver or attendant does not merely drive the cab or
belabor the donkey, but he is the visitor's easiest and cheapest
guide.  In London, the Tower, Westminster Abbey, and Madame Tussaud
are found by the stranger without difficulty, and almost without a
thought, because the cab-driver knows the whereabouts and the way.
Space is moreover annihilated, and the huge distances of the
English metropolis are brought within the scope of mortal power.
But in New York there is no such institution.

In New York there are street omnibuses as we have---there are street
cars such as last year we declined to have, and there are very
excellent public carriages; but none of these give you the
accommodation of a cab, nor can all of them combined do so.  The
omnibuses, though clean and excellent, were to me very
unintelligible.  They have no conductor to them.  To know their
different lines and usages a man should have made a scientific
study of the city.  To those going up and down Broadway I became
accustomed, but in them I was never quite at my ease.  The money
has to be paid through a little hole behind the driver's back, and
should, as I learned at last, be paid immediately on entrance.  But
in getting up to do this I always stumbled about, and it would
happen that when with considerable difficulty I had settled my own
account, two or three ladies would enter, and would hand me,
without a word, some coins with which I had no life-long
familiarity, in order that I might go through the same ceremony on
their account.  The change I would usually drop into the straw, and
then there would arise trouble and unhappiness.  Before I became
aware of that law as to instant payment, bells used to be rung at
me, which made me uneasy.  I knew I was not behaving as a citizen
should behave, but could not compass the exact points of my
delinquency.  And then, when I desired to escape, the door being
strapped up tight, I would halloo vainly at the driver through the
little hole; whereas, had I known my duty, I should have rung a
bell, or pulled a strap, according to the nature of the omnibus in
question.  In a month or two all these things may possibly be
learned; but the visitor requires his facilities for locomotion at
the first moment of his entrance into the city.  I heard it
asserted by a lecturer in Boston, Mr.\ Wendell Phillips, whose name
is there a household word, that citizens of the United States
carried brains in their fingers as well as in their heads; whereas
``common people,'' by which Mr.\ Phillips intended to designate the
remnant of mankind beyond the United States, were blessed with no
such extended cerebral development.  Having once learned this fact
from Mr.\ Phillips, I understood why it was that a New York omnibus
should be so disagreeable to me, and at the same time so suitable
to the wants of the New Yorkers.

And then there are street cars---very long omnibuses---which run on
rails but are dragged by horses.  They are capable of holding forty
passengers each, and as far as my experience goes carry an average
load of sixty.  The fare of the omnibus is six cents, or three
pence.  That of the street car five cents, or two pence halfpenny.
They run along the different avenues, taking the length of the
city.  In the upper or new part of the town their course is simple
enough, but as they descend to the Bowery, Peck Slip, and Pearl
Street, nothing can be conceived more difficult or devious than
their courses.  The Broadway omnibus, on the other hand, is a
straightforward, honest vehicle in the lower part of the town,
becoming, however, dangerous and miscellaneous when it ascends to
Union Square and the vicinities of fashionable life.

The street cars are manned with conductors, and, therefore, are
free from many of the perils of the omnibus; but they have perils
of their own.  They are always quite full.  By that I mean that
every seat is crowded, that there is a double row of men and women
standing down the center, and that the driver's platform in front
is full, and also the conductor's platform behind.  That is the
normal condition of a street car in the Third Avenue.  You, as a
stranger in the middle of the car, wish to be put down at, let us
say, 89th Street.  In the map of New York now before me, the cross
streets running from east to west are numbered up northward as far
as 154th Street.  It is quite useless for you to give the number as
you enter.  Even an American conductor, with brains all over him,
and an anxious desire to accommodate, as is the case with all these
men, cannot remember.  You are left therefore in misery to
calculate the number of the street as you move along, vainly
endeavoring through the misty glass to decipher the small numbers
which after a day or two you perceive to be written on the lamp
posts.

But I soon gave up all attempts at keeping a seat in one of these
cars.  It became my practice to sit down on the outside iron rail
behind, and as the conductor generally sat in my lap I was in a
measure protected.  As for the inside of these vehicles the women
of New York were, I must confess, too much for me.  I would no
sooner place myself on a seat, than I would be called on by a mute,
unexpressive, but still impressive stare into my face, to surrender
my place.  From cowardice if not from gallantry I would always
obey; and as this led to discomfort and an irritated spirit, I
preferred nursing the conductor on the hard bar in the rear.

And here if I seem to say a word against women in America, I beg
that it may be understood that I say that word only against a
certain class; and even as to that class I admit that they are
respectable, intelligent, and, as I believe, industrious.  Their
manners, however, are to me more odious than those of any other
human beings that I ever met elsewhere.  Nor can I go on with that
which I have to say without carrying my apology further, lest,
perchance, I should be misunderstood by some American women whom I
would not only exclude from my censure, but would include in the
very warmest eulogium which words of mine could express as to those
of the female sex whom I love and admire the most.  I have known,
do know, and mean to continue to know as far as in me may lie,
American ladies as bright, as beautiful, as graceful, as sweet, as
mortal limits for brightness, beauty, grace, and sweetness will
permit.  They belong to the aristocracy of the land, by whatever
means they may have become aristocrats.  In America one does not
inquire as to their birth, their training, or their old names.  The
fact of their aristocratic power comes out in every word and look.
It is not only so with those who have traveled or with those who
are rich.  I have found female aristocrats with families and
slender means, who have as yet made no grand tour across the ocean.
These women are charming beyond expression.  It is not only their
beauty.  Had he been speaking of such, Wendell Phillips would have
been right in saying that they have brains all over them.  So much
for those who are bright and beautiful, who are graceful and sweet!
And now a word as to those who to me are neither bright nor
beautiful, and who can be to none either graceful or sweet.

It is a hard task, that of speaking ill of any woman; but it seems
to me that he who takes upon himself to praise incurs the duty of
dispraising also where dispraise is, or to him seems to be,
deserved.  The trade of a novelist is very much that of describing
the softness, sweetness, and loving dispositions of women; and this
he does, copying as best he can from nature.  But if he only sings
of that which is sweet, whereas that which is not sweet too
frequently presents itself, his song will in the end be untrue and
ridiculous.  Women are entitled to much observance from men, but
they are entitled to no observance which is incompatible with
truth.  Women, by the conventional laws of society, are allowed to
exact much from men, but they are allowed to exact nothing for
which they should not make some adequate return.  It is well that a
man should kneel in spirit before the grace and weakness of a
woman, but it is not well that he should kneel either in spirit or
body if there be neither grace nor weakness.  A man should yield
everything to a woman for a word, for a smile---to one look of
entreaty.  But if there be no look of entreaty, no word, no smile,
I do not see that he is called upon to yield much.

The happy privileges with which women are at present blessed have
come to them from the spirit of chivalry.  That spirit has taught
man to endure in order that women may be at their ease; and has
generally taught women to accept the ease bestowed on them with
grace and thankfulness.  But in America the spirit of chivalry has
sunk deeper among men than it has among women.  It must be borne in
mind that in that country material well-being and education are
more extended than with us; and that, therefore, men there have
learned to be chivalrous who with us have hardly progressed so far.
The conduct of men to women throughout the States is always
gracious.  They have learned the lesson.  But it seems to me that
the women have not advanced as far as the men have done.  They have
acquired a sufficient perception of the privileges which chivalry
gives them, but no perception of that return which chivalry demands
from them.  Women of the class to which I allude are always talking
of their rights, but seem to have a most indifferent idea of their
duties.  They have no scruple at demanding from men everything that
a man can be called on to relinquish in a woman's behalf, but they
do so without any of that grace which turns the demand made into a
favor conferred.

I have seen much of this in various cities of America, but much
more of it in New York than elsewhere.  I have heard young
Americans complain of it, swearing that they must change the whole
tenor of their habits toward women.  I have heard American ladies
speak of it with loathing and disgust.  For myself, I have
entertained on sundry occasions that sort of feeling for an
American woman which the close vicinity of an unclean animal
produces.  I have spoken of this with reference to street cars,
because in no position of life does an unfortunate man become more
liable to these anti-feminine atrocities than in the center of one
of these vehicles.  The woman, as she enters, drags after her a
misshapen, dirty mass of battered wirework, which she calls her
crinoline, and which adds as much to her grace and comfort as a log
of wood does to a donkey when tied to the animal's leg in a
paddock.  Of this she takes much heed, not managing it so that it
may be conveyed up the carriage with some decency, but striking it
about against men's legs, and heaving it with violence over
people's knees.  The touch of a real woman's dress is in itself
delicate; but these blows from a harpy's fins are as loathsome as a
snake's slime.  If there be two of them they talk loudly together,
having a theory that modesty has been put out of court by women's
rights.  But, though not modest, the woman I describe is ferocious
in her propriety.  She ignores the whole world around her as she
sits; with a raised chin and face flattened by affectation, she
pretends to declare aloud that she is positively not aware that any
man is even near her.  She speaks as though to her, in her
womanhood, the neighborhood of men was the same as that of dogs or
cats.  They are there, but she does not hear them, see them, or
even acknowledge them by any courtesy of motion.  But her own face
always gives her the lie.  In her assumption of indifference she
displays her nasty consciousness, and in each attempt at a would-be
propriety is guilty of an immodesty.  Who does not know the timid
retiring face of the young girl who when alone among men unknown to
her feels that it becomes her to keep herself secluded?  As many
men as there are around her, so many knights has such a one, ready
bucklered for her service, should occasion require such services.
Should it not, she passes on unmolested---but not, as she herself
will wrongly think, unheeded.  But as to her of whom I am speaking,
we may say that every twist of her body and every tone of her voice
is an unsuccessful falsehood.  She looks square at you in the face,
and you rise to give her your seat.  You rise from a deference to
your own old convictions, and from that courtesy which you have
ever paid to a woman's dress, let it be worn with ever such hideous
deformities.  She takes the place from which you have moved without
a word or a bow.  She twists herself round, banging your shins with
her wires, while her chin is still raised, and her face is still
flattened, and she directs her friend's attention to another seated
man, as though that place were also vacant, and necessarily at her
disposure.  Perhaps the man opposite has his own ideas about
chivalry.  I have seen such a thing, and have rejoiced to see it.

You will meet these women daily, hourly, everywhere in the streets.
Now and again you will find them in society, making themselves even
more odious there than elsewhere.  Who they are, whence they come,
and why they are so unlike that other race of women of which I have
spoken, you will settle for yourself.  Do we not all say of our
chance acquaintances, after half an hour's conversation, nay, after
half an hour spent in the same room without conversation, that this
woman is a lady, and that that other woman is not?  They jostle
each other even among us, but never seem to mix.  They are closely
allied; but neither imbues the other with her attributes.  Both
shall be equally well born, or both shall be equally ill born; but
still it is so.  The contrast exists in England; but in America it
is much stronger.  In England women become ladylike or vulgar.  In
the States they are either charming or odious.

See that female walking down Broadway.  She is not exactly such a
one as her I have attempted to describe on her entrance into the
street car; for this lady is well dressed, if fine clothes will
make well dressing.  The machinery of her hoops is not battered,
and altogether she is a personage much more distinguished in all
her expenditures.  But yet she is a copy of the other woman.  Look
at the train which she drags behind her over the dirty pavement,
where dogs have been, and chewers of tobacco, and everything
concerned with filth except a scavenger.  At every hundred yards
some unhappy man treads upon the silken swab which she trails
behind her---loosening it dreadfully at the girth one would say; and
then see the style of face and the expression of features with
which she accepts the sinner's half muttered apology.  The world,
she supposes, owes her everything because of her silken train, even
room enough in a crowded thoroughfare to drag it along unmolested.
But, according to her theory, she owes the world nothing in return.
She is a woman with perhaps a hundred dollars on her back, and
having done the world the honor of wearing them in the world's
presence, expects to be repaid by the world's homage and chivalry.
But chivalry owes her nothing---nothing, though she walk about
beneath a hundred times a hundred dollars---nothing, even though she
be a woman.  Let every woman learn this, that chivalry owes her
nothing unless she also acknowledges her debt to chivalry.  She
must acknowledge it and pay it; and then chivalry will not be
backward in making good her claims upon it.

All this has come of the street cars.  But as it was necessary that
I should say it somewhere, it is as well said on that subject as on
any other.  And now to continue with the street cars.  They run, as
I have said, the length of the town, taking parallel lines.  They
will take you from the Astor House, near the bottom of the town,
for miles and miles northward---half way up the Hudson River---for, I
believe, five pence.  They are very slow, averaging about five
miles an hour; but they are very sure.  For regular inhabitants,
who have to travel five or six miles perhaps to their daily work,
they are excellent.  I have nothing really to say against the
street cars.  But they do not fill the place of cabs.

There are, however, public carriages---roomy vehicles, dragged by
two horses, clean and nice, and very well suited to ladies visiting
the city.  But they have none of the attributes of the cab.  As a
rule, they are not to be found standing about.  They are very slow.
They are very dear.  A dollar an hour is the regular charge; but
one cannot regulate one's motion by the hour.  Going out to dinner
and back costs two dollars, over a distance which in London would
cost two shillings.  As a rule, the cost is four times that of a
cab, and the rapidity half that of a cab.  Under these
circumstances, I think I am justified in saying that there is no
mode of getting about in New York to see anything.

And now as to the other charge against New York, of there being
nothing to see.  How should there be anything there to see of
general interest?  In other large cities---cities as large in name
as New York---there are works of art, fine buildings, ruins, ancient
churches, picturesque costumes, and the tombs of celebrated men.
But in New York there are none of these things.  Art has not yet
grown up there.  One or two fine figures by Crawford are in the
town, especially that of the Sorrowing Indian, at the rooms of the
Historical Society; but art is a luxury in a city which follows but
slowly on the heels of wealth and civilization.  Of fine buildings---%
which, indeed, are comprised in art---there are none deserving
special praise or remark.  It might well have been that New York
should ere this have graced herself with something grand in
architecture; but she has not done so.  Some good architectural
effect there is, and much architectural comfort.  Of ruins, of
course, there can be none---none, at least, of such ruins as
travelers admire, though perhaps some of that sort which disgraces
rather than decorates.  Churches there are plenty, but none that
are ancient.  The costume is the same as our own; and I need hardly
say that it is not picturesque.  And the time for the tombs of
celebrated men has not yet come.  A great man's ashes are hardly of
value till they have all but ceased to exist.

The visitor to New York must seek his gratification and obtain his
instruction from the habits and manners of men.  The American,
though he dresses like an Englishman, and eats roast beef with a
silver fork---or sometimes with a steel knife---as does an
Englishman, is not like an Englishman in his mind, in his
aspirations, in his tastes, or in his politics.  In his mind he is
quicker, more universally intelligent, more ambitious of general
knowledge, less indulgent of stupidity and ignorance in others,
harder, sharper, brighter with the surface brightness of steel,
than is an Englishman; but he is more brittle, less enduring, less
malleable, and, I think, less capable of impressions.  The mind of
the Englishman has more imagination, but that of the American more
incision.  The American is a great observer; but he observes things
material rather than things social or picturesque.  He is a
constant and ready speculator; but all speculations, even those
which come of philosophy, are with him more or less material.  In
his aspirations the American is more constant than an Englishman---%
or I should rather say he is more constant in aspiring.  Every
citizen of the United States intends to do something.  Every one
thinks himself capable of some effort.  But in his aspirations he
is more limited than an Englishman.  The ambitious American never
soars so high as the ambitious Englishman.  He does not even see up
to so great a height, and, when he has raised himself somewhat
above the crowd, becomes sooner dizzy with his own altitude.  An
American of mark, though always anxious to show his mark, is always
fearful of a fall.  In his tastes the American imitates the
Frenchman.  Who shall dare to say that he is wrong, seeing that in
general matters of design and luxury the French have won for
themselves the foremost name?  I will not say that the American is
wrong, but I cannot avoid thinking that he is so.  I detest what is
called French taste; but the world is against me.  When I
complained to a landlord of a hotel out in the West that his
furniture was useless; that I could not write at a marble table
whose outside rim was curved into fantastic shapes; that a gold
clock in my bed-room which did not go would give me no aid in
washing myself; that a heavy, immovable curtain shut out the light;
and that papier-mache chairs with small, fluffy velvet seats were
bad to sit on, he answered me completely by telling me that his
house had been furnished not in accordance with the taste of
England, but with that of France.  I acknowledged the rebuke, gave
up my pursuits of literature and cleanliness, and hurried out of
the house as quickly as I could.  All America is now furnishing
itself by the rules which guided that hotel-keeper.  I do not
merely allude to actual household furniture---to chairs, tables, and
detestable gilt clocks.  The taste of America is becoming French in
its conversation, French in its comforts and French in its
discomforts, French in its eating and French in its dress, French
in its manners, and will become French in its art.  There are those
who will say that English taste is taking the same direction.  I do
not think so.  I strongly hope that it is not so.  And therefore I
say that an Englishman and an American differ in their tastes.

But of all differences between an Englishman and an American, that
in politics is the strongest and the most essential.  I cannot
here, in one paragraph, define that difference with sufficient
clearness to make my definition satisfactory; but I trust that some
idea of that difference may be conveyed by the general tenor of my
book.  The American and the Englishman are both republicans.  The
governments of the States and of England are probably the two
purest republican governments in the world.  I do not, of course,
here mean to say that the governments are more pure than others,
but that the systems are more absolutely republican.  And yet no
men can be much farther asunder in politics than the Englishman and
the American.  The American of the present day puts a ballot-box
into the hands of every citizen, and takes his stand upon that and
that only.  It is the duty of an American citizen to vote; and when
he has voted, he need trouble himself no further till the time for
voting shall come round again.  The candidate for whom he has voted
represents his will, if he have voted with the majority; and in
that case he has no right to look for further influence.  If he
have voted with the minority, he has no right to look for any
influence at all.  In either case he has done his political work,
and may go about his business till the next year, or the next two
or four years, shall have come round.  The Englishman, on the other
hand, will have no ballot-box, and is by no means inclined to
depend exclusively upon voters or upon voting.  As far as voting
can show it, he desires to get the sense of the country; but he
does not think that that sense will be shown by universal suffrage.
He thinks that property amounting to a thousand pounds will show
more of that sense than property amounting to a hundred; but he
will not, on that account, go to work and apportion votes to
wealth.  He thinks that the educated can show more of that sense
than the uneducated; but he does not therefore lay down any rule
about reading, writing, and arithmetic, or apportion votes to
learning.  He prefers that all these opinions of his shall bring
themselves out and operate by their own intrinsic weight.  Nor does
he at all confine himself to voting, in his anxiety to get the
sense of the country.  He takes it in any way that it will show
itself, uses it for what it is worth, or perhaps far more than it
is worth, and welds it into that gigantic lever by which the
political action of the country is moved.  Every man in Great
Britain, whether he possesses any actual vote or no, can do that
which is tantamount to voting every day of his life by the mere
expression of his opinion.  Public opinion in America has hitherto
been nothing, unless it has managed to express itself by a majority
of ballot-boxes.  Public opinion in England is everything, let
votes go as they may.  Let the people want a measure, and there is
no doubt of their obtaining it.  Only the people must want it---as
they did want Catholic emancipation, reform, and corn-law repeal,
and as they would want war if it were brought home to them that
their country was insulted.

In attempting to describe this difference in the political action
of the two countries, I am very far from taking all praise for
England or throwing any reproach on the States.  The political
action of the States is undoubtedly the more logical and the
clearer.  That, indeed, of England is so illogical and so little
clear that it would be quite impossible for any other nation to
assume it, merely by resolving to do so.  Whereas the political
action of the States might be assumed by any nation to-morrow, and
all its strength might be carried across the water in a few written
rules as are the prescriptions of a physician or the regulations of
an infirmary.  With us the thing has grown of habit, has been
fostered by tradition, has crept up uncared for, and in some parts
unnoticed.  It can be written in no book, can be described in no
words, can be copied by no statesmen, and I almost believe can be
understood by no people but that to whose peculiar uses it has been
adapted.

In speaking as I have here done of American taste and American
politics, I must allude to a special class of Americans who are to
be met more generally in New York than elsewhere---men who are
educated, who have generally traveled, who are almost always
agreeable, but who, as regards their politics, are to me the most
objectionable of all men.  As regards taste they are objectionable
to me also.  But that is a small thing; and as they are quite as
likely to be right as I am, I will say nothing against their taste.
But in politics it seems to me that these men have fallen into the
bitterest and perhaps into the basest of errors.  Of the man who
begins his life with mean political ideas, having sucked them in
with his mother's milk, there may be some hope.  The evil is at any
rate the fault of his forefathers rather than of himself.  But who
can have hope of him who, having been thrown by birth and fortune
into the running river of free political activity, has allowed
himself to be drifted into the stagnant level of general political
servility?  There are very many such Americans.  They call
themselves republicans, and sneer at the idea of a limited
monarchy, but they declare that there is no republic so safe, so
equal for all men, so purely democratic as that now existing in
France.  Under the French Empire all men are equal.  There is no
aristocracy; no oligarchy; no overshadowing of the little by the
great.  One superior is admitted---admitted on earth, as a superior
is also admitted in heaven.  Under him everything is level, and,
provided he be not impeded, everything is free.  He knows how to
rule, and the nation, allowing him the privilege of doing so, can
go along its course safely; can eat, drink, and be merry.  If few
men can rise high, so also can few men fall low.  Political
equality is the one thing desirable in a commonwealth, and by this
arrangement political equality is obtained.  Such is the modern
creed of many an educated republican of the States.

To me it seems that such a political state is about the vilest to
which a man can descend.  It amounts to a tacit abandonment of the
struggle which men are making for political truth and political
beneficence, in order that bread and meat may be eaten in peace
during the score of years or so that are at the moment passing over
us.  The politicians of this class have decided for themselves that
the summum bonum is to be found in bread and the circus games.  If
they be free to eat, free to rest, free to sleep, free to drink
little cups of coffee, while the world passes before them, on a
boulevard, they have that freedom which they covet.  But equality
is necessary as well as freedom.  There must be no towering trees
in this parterre to overshadow the clipped shrubs, and destroy the
uniformity of a growth which should never mount more than two feet
above the earth.  The equality of this politician would forbid any
to rise above him instead of inviting all to rise up to him.  It is
the equality of fear and of selfishness, and not the equality of
courage and philanthropy.  And brotherhood, too, must be invoked---%
fraternity as we may better call it in the jargon of the school.
Such politicians tell one much of fraternity, and define it too.
It consists in a general raising of the hat to all mankind; in a
daily walk that never hurries itself into a jostling trot,
inconvenient to passengers on the pavement; in a placid voice, a
soft smile, and a small cup of coffee on a boulevard.  It means all
this, but I could never find that it meant any more.  There is a
nation for which one is almost driven to think that such political
aspirations as these are suitable; but that nation is certainly not
the States of America.

And yet one finds many American gentlemen who have allowed
themselves to be drifted into such a theory.  They have begun the
world as republican citizens, and as such they must go on.  But in
their travels and their studies, and in the luxury of their life,
they have learned to dislike the rowdiness of their country's
politics.  They want things to be soft and easy; as republican as
you please, but with as little noise as possible.  The President is
there for four years.  Why not elect him for eight, for twelve, or
for life?---for eternity if it were possible to find one who could
continue to live?  It is to this way of thinking that Americans are
driven, when the polish of Europe has made the roughness of their
own elections odious to them.

``Have you seen any of our great institootions, sir?''  That of
course is a question which is put to every Englishman who has
visited New York, and the Englishman who intends to say that he has
seen New York, should visit many of them.  I went to schools,
hospitals, lunatic asylums, institutes for deaf and dumb, water-
works, historical societies, telegraph offices, and large
commercial establishments.  I rather think that I did my work in a
thorough and conscientious manner, and I owe much gratitude to
those who guided me on such occasions.  Perhaps I ought to describe
all these institutions; but were I to do so, I fear that I should
inflict fifty or sixty very dull pages on my readers.  If I could
make all that I saw as clear and intelligible to others as it was
made to me who saw it, I might do some good.  But I know that I
should fail.  I marveled much at the developed intelligence of a
room full of deaf and dumb pupils, and was greatly astonished at
the performance of one special girl, who seemed to be brighter and
quicker, and more rapidly easy with her pen than girls generally
are who can hear and talk; but I cannot convey my enthusiasm to
others.  On such a subject a writer may be correct, may be
exhaustive, may be statistically great; but he can hardly be
entertaining, and the chances are that he will not be instructive.

In all such matters, however, New York is pre-eminently great.  All
through the States suffering humanity receives so much attention
that humanity can hardly be said to suffer.  The daily recurring
boast of ``our glorious institootions, sir,'' always provokes the
ridicule of an Englishman.  The words have become ridiculous, and
it would, I think, be well for the nation if the term ``Institution''
could be excluded from its vocabulary.  But, in truth, they are
glorious.  The country in this respect boasts, but it has done that
which justifies a boast.  The arrangements for supplying New York
with water are magnificent.  The drainage of the new part of the
city is excellent.  The hospitals are almost alluring.  The lunatic
asylum which I saw was perfect---though I did not feel obliged to
the resident physician for introducing me to all the worst patients
as countrymen of my own.  ``An English lady, Mr.\ Trollope.  I'll
introduce you.  Quite a hopeless case.  Two old women.  They've
been here fifty years.  They're English.  Another gentleman from
England, Mr.\ Trollope.  A very interesting case!  Confirmed
inebriety.''

And as to the schools, it is almost impossible to mention them with
too high a praise.  I am speaking here specially of New York,
though I might say the same of Boston, or of all New England.  I do
not know any contrast that would be more surprising to an
Englishman, up to that moment ignorant of the matter, than that
which he would find by visiting first of all a free school in
London, and then a free school in New York.  If he would also learn
the number of children that are educated gratuitously in each of
the two cities, and also the number in each which altogether lack
education, he would, if susceptible of statistics, be surprised
also at that.  But seeing and hearing are always more effective
than mere figures.  The female pupil at a free school in London is,
as a rule, either a ragged pauper or a charity girl, if not
degraded, at least stigmatized by the badges and dress of the
charity.  We Englishmen know well the type of each, and have a
fairly correct idea of the amount of education which is imparted to
them.  We see the result afterward when the same girls become our
servants, and the wives of our grooms and porters.  The female
pupil at a free school in New York is neither a pauper nor a
charity girl.  She is dressed with the utmost decency.  She is
perfectly cleanly.  In speaking to her, you cannot in any degree
guess whether her father has a dollar a day, or three thousand
dollars a year.  Nor will you be enabled to guess by the manner in
which her associates treat her.  As regards her own manner to you,
it is always the same as though her father were in all respects
your equal.  As to the amount of her knowledge, I fairly confess
that it is terrific.  When in the first room which I visited, a
slight, slim creature was had up before me to explain to me the
properties of the hypothenuse, I fairly confess that, as regards
education, I backed down, and that I resolved to confine my
criticisms to manner, dress, and general behavior.  In the next
room I was more at my ease, finding that ancient Roman history was
on the tapis.  ``Why did the Romans run away with the Sabine women''
asked the mistress, herself a young woman of about three and
twenty.  ``Because they were pretty,'' simpered out a little girl
with a cherry mouth.  The answer did not give complete
satisfaction, and then followed a somewhat abstruse explanation on
the subject of population.  It was all done with good faith and a
serious intent, and showed what it was intended to show---that the
girls there educated had in truth reached the consideration of
important subjects, and that they were leagues beyond that terrible
repetition of A B C, to which, I fear, that most of our free
metropolitan schools are still necessarily confined.  You and I,
reader, were we called on to superintend the education of girls of
sixteen, might not select, as favorite points either the
hypothenuse or the ancient methods of populating young colonies.
There may be, and to us on the European side of the Atlantic there
will be, a certain amount of absurdity in the Transatlantic idea
that all knowledge is knowledge, and that it should be imparted if
it be not knowledge of evil.  But as to the general result, no
fair-minded man or woman can have a doubt.  That the lads and girls
in these schools are excellently educated, comes home as a fact to
the mind of any one who will look into the subject.  That girl
could not have got as fair at the hypothenuse without a competent
and abiding knowledge of much that is very far beyond the outside
limits of what such girls know with us.  It was at least manifest
in the other examination that the girls knew as well as I did who
were the Romans, and who were the Sabine women.  That all this is
of use, was shown in the very gestures and bearings of the girl.
Emollit mores, as Colonel Newcombe used to say.  That young woman
whom I had watched while she cooked her husband's dinner upon the
banks of the Mississippi had doubtless learned all about the Sabine
women, and I feel assured that she cooked her husband's dinner all
the better for that knowledge---and faced the hardships of the world
with a better front than she would have done had she been ignorant
on the subject.

In order to make a comparison between the schools of London and
those of New York, I have called them both free schools.  They are,
in fact, more free in New York than they are in London; because in
New York every boy and girl, let his parentage be what it may, can
attend these schools without any payment.  Thus an education as
good as the American mind can compass, prepared with every care,
carried on by highly-paid tutors, under ample surveillance,
provided with all that is most excellent in the way of rooms,
desks, books, charts, maps, and implements, is brought actually
within the reach of everybody.  I need not point out to Englishmen
how different is the nature of schools in London.  It must not,
however, be supposed that these are charity schools.  Such is not
their nature.  Let us say what we may as to the beauty of charity
as a virtue, the recipient of charity in its customary sense among
us is ever more or less degraded by the position.  In the States
that has been fully understood, and the schools to which I allude
are carefully preserved from any such taint.  Throughout the States
a separate tax is levied for the maintenance of these schools, and
as the taxpayer supports them, he is, of course, entitled to the
advantage which they confer.  The child of the non-taxpayer is also
entitled, and to him the boon, if strictly analyzed, will come in
the shape of a charity.  But under the system as it is arranged,
this is not analyzed.  It is understood that the school is open to
all in the ward to which it belongs, and no inquiry is made whether
the pupil's parent has or has not paid anything toward the school's
support.  I found this theory carried out so far that at the deaf
and dumb school, where some of the poorer children are wholly
provided by the institution, care is taken to clothe them in
dresses of different colors and different make, in order that
nothing may attach to them which has the appearance of a badge.
Political economists will see something of evil in this.  But
philanthropists will see very much that is good.

It is not without a purpose that I have given this somewhat glowing
account of a girls' school in New York so soon after my little
picture of New York women, as they behave themselves in the streets
and street cars.  It will, of course, be said that those women of
whom I have spoken, by no means in terms of admiration, are the
very girls whose education has been so excellent.  This of course
is so; but I beg to remark that I have by no means said that an
excellent school education will produce all female excellencies.
The fact, I take it, is this: that seeing how high in the scale
these girls have been raised, one is anxious that they should be
raised higher.  One is surprised at their pert vulgarity and
hideous airs, not because they are so low in our general
estimation, but because they are so high.  Women of the same class
in London are humble enough, and therefore rarely offend us who are
squeamish.  They show by their gestures that they hardly think
themselves good enough to sit by us; they apologize for their
presence; they conceive it to be their duty to be lowly in their
gesture.  The question is which is best, the crouching and
crawling, or the impudent, unattractive self-composure.  Not, my
reader, which action on her part may the better conduce to my
comfort or to yours.  That is by no means the question.  Which is
the better for the woman herself?  That, I take it, is the point to
be decided.  That there is something better than either, we shall
all agree---but to my thinking the crouching and crawling is the
lowest type of all.

At that school I saw some five or six hundred girls collected in
one room, and heard them sing.  The singing was very pretty, and it
was all very nice; but I own that I was rather startled, and to
tell the truth somewhat abashed, when I was invited to ``say a few
words to them.''  No idea of such a suggestion had dawned upon me,
and I felt myself quite at a loss.  To be called up before five
hundred men is bad enough, but how much worse before that number of
girls!  What could I say but that they were all very pretty?  As
far as I can remember, I did say that and nothing else.  Very
pretty they were, and neatly dressed, and attractive; but among
them all there was not a pair of rosy cheeks.  How should there be,
when every room in the building was heated up to the condition of
an oven by those damnable hot-air pipes.

In England a taste for very large shops has come up during the last
twenty years.  A firm is not doing a good business, or at any rate
a distinguished business, unless he can assert in his trade card
that he occupies at least half a dozen houses---Nos. 105, 106, 107,
108, 109 and 110.  The old way of paying for what you want over the
counter is gone; and when you buy a yard of tape or a new carriage---%
for either of which articles you will probably visit the same
establishment---you go through about the same amount of ceremony as
when you sell a thousand pounds out of the stocks in propria
persona.  But all this is still further exaggerated in New York.
Mr.\ Stewart's store there is perhaps the handsomest institution in
the city, and his hall of audience for new carpets is a magnificent
saloon.  ``You have nothing like that in England,'' my friend said to
me as he walked me through it in triumph.  ``I wish we had nothing
approaching to it,'' I answered.  For I confess to a liking for the
old-fashioned private shops.  Harper's establishment for the
manufacture and sale of books is also very wonderful.  Everything
is done on the premises, down to the very coloring of the paper
which lines the covers, and places the gilding on their backs.  The
firm prints, engraves, electroplates, sews, binds, publishes, and
sells wholesale and retail.  I have no doubt that the authors have
rooms in the attics where the other slight initiatory step is taken
toward the production of literature.

New York is built upon an island, which is I believe about ten
miles long, counting from the southern point at the Battery up to
Carmansville, to which place the city is presumed to extend
northward.  This island is called Manhattan, a name which I have
always thought would have been more graceful for the city than that
of New York.  It is formed by the Sound or East River, which
divides the continent from Long Island by the Hudson River, which
runs into the Sound, or rather joins it at the city foot, and by a
small stream called the Harlem River, which runs out of the Hudson
and meanders away into the Sound at the north of the city, thus
cutting the city off from the main-land.  The breadth of the island
does not much exceed two miles, and therefore the city is long, and
not capable of extension in point of breadth.  In its old days it
clustered itself round about the Point, and stretched itself up
from there along the quays of the two waters.  The streets down in
this part of the town are devious enough, twisting themselves about
with delightful irregularity; but as the city grew there came the
taste for parallelograms, and the upper streets are rectangular and
numbered.  Broadway, the street of New York with which the world is
generally best acquainted, begins at the southern point of the town
and goes northward through it.  For some two miles and a half it
walks away in a straight line, and then it turns to the left toward
the Hudson.  From that time Broadway never again takes a straight
course, but crosses the various avenues in an oblique direction
till it becomes the Bloomingdale Road, and under that name takes
itself out of town.  There are eleven so-called avenues, which
descend in absolutely straight lines from the northern, and at
present unsettled, extremity of the new town, making their way
southward till they lose themselves among the old streets.  These
are called First Avenue, Second Avenue, and so on.  The town had
already progressed two miles up northward from the Battery before
it had caught the parallelogramic fever from Philadelphia, for at
about that distance we find ``First Street''.  First Street runs
across the avenues from water to water, and then Second Street.  I
will not name them all, seeing that they go up to 154th Street!
They do so at least on the map and I believe on the lamp-posts.
But the houses are not yet built in order beyond 50th or 60th
Street.  The other hundred streets, each of two miles long, with
the avenues, which are mostly unoccupied for four or five miles, is
the ground over which the young New Yorkers are to spread
themselves.  I do not in the least doubt that they will occupy it
all, and that 154th Street will find itself too narrow a boundary
for the population.

I have said that there was some good architectural effect in New
York, and I alluded chiefly to that of the Fifth Avenue.  The Fifth
Avenue is the Belgrave Square, the Park Lane, and the Pall Mall of
New York.  It is certainly a very fine street.  The houses in it
are magnificent---not having that aristocratic look which some of
our detached London residences enjoy, or the palatial appearance of
an old-fashioned hotel in Paris, but an air of comfortable luxury
and commercial wealth which is not excelled by the best houses of
any other town that I know.  They are houses, not hotels or
palaces; but they are very roomy houses, with every luxury that
complete finish can give them.  Many of them cover large spaces of
the ground, and their rent will sometimes go up as high as 800
pounds and 1000 pounds a year.  Generally the best of these houses
are owned by those who live in them, and rent is not, therefore,
paid.  But this is not always the case, and the sums named above
may be taken as expressing their value.  In England a man should
have a very large income indeed who could afford to pay 1000 pounds
a year for his house in London.  Such a one would as a matter of
course have an establishment in the country, and be an earl, or a
duke, or a millionaire.  But it is different in New York.  The
resident there shows his wealth chiefly by his house; and though he
may probably have a villa at Newport or a box somewhere up the
Hudson, he has no second establishment.  Such a house, therefore,
will not represent a total expenditure of above 4000 pounds a year.

There are churches on each side of Fifth Avenue---perhaps five or
six within sight at one time---which add much to the beauty of the
street.  They are well built, and in fairly good taste.  These,
added to the general well-being and splendid comfort of the place,
give it an effect better than the architecture of the individual
houses would seem to warrant.  I own that I have enjoyed the vista
as I have walked up and down Fifth Avenue, and have felt that the
city had a right to be proud of its wealth.  But the greatness and
beauty and glory of wealth have on such occasions been all in all
with me.  I know no great man, no celebrated statesman, no
philanthropist of peculiar note who has lived in Fifth Avenue.
That gentleman on the right made a million of dollars by inventing
a shirt collar; this one on the left electrified the world by a
lotion; as to the gentleman at the corner there, there are rumors
about him and the Cuban slave trade but my informant by no means
knows that they are true.  Such are the aristocracy of Fifth
Avenue, I can only say that, if I could make a million dollars by a
lotion, I should certainly be right to live in such a house as one
of those.

The suburbs of New York are, by the nature of the localities,
divided from the city by water.  Jersey City and Hoboken are on the
other side of the Hudson, and in another State.  Williamsburg and
Brooklyn are on Long Island, which is a part of the State of New
York.  But these places are as easily reached as Lambeth is reached
from Westminster.  Steam ferries ply every three or four minutes;
and into these boats coaches, carts, and wagons of any size or
weight are driven.  In fact, they make no other stoppage to the
commerce than that occasioned by the payment of a few cents.  Such
payment, no doubt, is a stoppage; and therefore it is that Jersey
City, Brooklyn, and Williamsburg are, at any rate in appearance,
very dull and uninviting.  They are, however, very populous.  Many
of the quieter citizens prefer to live there; and I am told that
the Brooklyn tea parties consider themselves to be, in esthetic
feeling, very much ahead of anything of the kind in the more
opulent centers of the city.  In beauty of scenery Staten Island is
very much the prettiest of the suburbs of New York.  The view from
the hillside in Staten Island down upon New York harbor is very
lovely.  It is the only really good view of that magnificent harbor
which I have been able to find.  As for appreciating such beauty
when one is entering a port from sea or leaving it for sea, I do
not believe in any such power.  The ship creeps up or creeps out
while the mind is engaged on other matters.  The passenger is
uneasy either with hopes or fears, and then the grease of the
engines offends one's nostrils.  But it is worth the tourist's
while to look down upon New York harbor from the hillside in Staten
Island.  When I was there Fort Lafayette looked black in the center
of the channel, and we knew that it was crowded with the victims of
secession.  Fort Tompkins was being built to guard the pass---worthy
of a name of richer sound; and Fort something else was bristling
with new cannon.  Fort Hamilton, on Long Island, opposite, was
frowning at us; and immediately around us a regiment of volunteers
was receiving regimental stocks and boots from the hands of its
officers.  Everything was bristling with war; and one could not but
think that not in this way had New York raised herself so quickly
to her present greatness.

But the glory of New York is the Central Park---its glory in the
minds of all new Yorkers of the present day.  The first question
asked of you is whether you have seen the Central Park, and the
second is as to what you think of it.  It does not do to say simply
that it is fine, grand, beautiful, and miraculous.  You must swear
by cock and pie that it is more fine, more grand, more beautiful,
more miraculous than anything else of the kind anywhere.  Here you
encounter in its most annoying form that necessity for eulogium
which presses you everywhere.  For in truth, taken as it is at
present, the Central Park is not fine, nor grand, nor beautiful.
As to the miracle, let that pass.  It is perhaps as miraculous as
some other great latter-day miracles.

But the Central Park is a very great fact, and affords a strong
additional proof of the sense and energy of the people.  It is very
large, being over three miles long and about three-quarters of a
mile in breadth.  When it was found that New York was extending
itself, and becoming one of the largest cities of the world, a
space was selected between Fifth and Seventh Avenues, immediately
outside the limits of the city as then built, but nearly in the
center of the city as it is intended to be built.  The ground
around it became at once of great value; and I do not doubt that
the present fashion of Fifth Avenue about Twentieth Street will in
course of time move itself up to Fifth Avenue as it looks, or will
look, over the Park at Seventieth, Eightieth, and Ninetieth
Streets.  The great water-works of the city bring the Croton River,
whence New York is supplied, by an aqueduct over the Harlem River
into an enormous reservoir just above the Park; and hence it has
come to pass that there will be water not only for sanitary and
useful purposes, but also for ornament.  At present the Park, to
English eyes, seems to be all road.  The trees are not grown up;
and the new embankments, and new lakes, and new ditches, and new
paths give to the place anything but a picturesque appearance.  The
Central Park is good for what it will be rather than for what it
is.  The summer heat is so very great that I doubt much whether the
people of New York will ever enjoy such verdure as our parks show.
But there will be a pleasant assemblage of walks and water-works,
with fresh air and fine shrubs and flowers, immediately within the
reach of the citizens.  All that art and energy can do will be
done, and the Central Park doubtless will become one of the great
glories of New York.  When I was expected to declare that St.
James's Park, Green Park, Hyde Park, and Kensington Gardens
altogether were nothing to it, I confess that I could only remain
mute.

Those who desire to learn what are the secrets of society in New
York, I would refer to the Potiphar Papers.  The Potiphar Papers
are perhaps not as well known in England as they deserve to be.
They were published, I think, as much as seven or eight years ago;
but are probably as true now as they were then.  What I saw of
society in New York was quiet and pleasant enough; but doubtless I
did not climb into that circle in which Mrs.\ Potiphar held so
distinguished a position.  It may be true that gentlemen habitually
throw fragments of their supper and remnants of their wine on to
their host's carpets; but if so I did not see it.

As I progress in my work I feel that duty will call upon me to
write a separate chapter on hotels in general, and I will not,
therefore, here say much about those in New York.  I am inclined to
think that few towns in the world, if any, afford on the whole
better accommodation, but there are many in which the accommodation
is cheaper.  Of the railways also I ought to say something.  The
fact respecting them, which is most remarkable, is that of their
being continued into the center of the town through the streets.
The cars are not dragged through the city by locomotive engines,
but by horses; the pace therefore is slow, but the convenience to
travelers in being brought nearer to the center of trade must be
much felt.  It is as though passengers from Liverpool and
passengers from Bristol were carried on from Euston Square and
Paddington along the New Road, Portland Place, and Regent Street to
Pall Mall, or up the City Road to the Bank.  As a general rule,
however, the railways, railway cars, and all about them are ill
managed.  They are monopolies, and the public, through the press,
has no restraining power upon them as it has in England.  A parcel
sent by express over a distance of forty miles will not be
delivered within twenty-four hours.  I once made my plaint on this
subject at the bar or office of a hotel, and was told that no
remonstrance was of avail.  ``It is a monopoly,'' the man told me,
``and if we say anything, we are told that if we do not like it we
need not use it.''  In railway matters and postal matters time and
punctuality are not valued in the States as they are with us, and
the public seem to acknowledge that they must put up with defects---%
that they must grin and bear them in America, as the public no
doubt do in Austria, where such affairs are managed by a government
bureau.

In the beginning of this chapter I spoke of the population of New
York, and I cannot end it without remarking that out of that
population more than one-eighth is composed of Germans.  It is, I
believe, computed that there are about 120,000 Germans in the city,
and that only two other German cities in the world, Vienna and
Berlin have a larger German population than New York.  The Germans
are good citizens and thriving men, and are to be found prospering
all over the Northern and Western parts of the Union.  It seems
that they are excellently well adapted to colonization, though they
have in no instance become the dominant people in a colony, or
carried with them their own language or their own laws.  The French
have done so in Algeria, in some of the West India islands, and
quite as essentially into Lower Canada, where their language and
laws still prevail.  And yet it is, I think, beyond doubt that the
French are not good colonists, as are the Germans.

Of the ultimate destiny of New York as one of the ruling commercial
cities of the world, it is, I think, impossible to doubt.  Whether
or no it will ever equal London in population I will not pretend to
say; even should it do so, should its numbers so increase as to
enable it to say that it had done so, the question could not very
well be settled.  When it comes to pass that an assemblage of men
in one so-called city have to be counted by millions, there arises
the impossibility of defining the limits of that city, and of
saying who belong to it and who do not.  An arbitrary line may be
drawn, but that arbitrary line, though perhaps false when drawn as
including too much, soon becomes more false as including too
little.  Ealing, Acton, Fulham, Putney, Norwood, Sydenham,
Blackheath, Woolwich, Greenwich, Stratford, Highgate, and Hampstead
are, in truth, component parts of London, and very shortly Brighton
will be as much so.



\chapter{The Constitution of the State of New York}


As New York is the most populous State of the Union, having the
largest representation in Congress---on which account it has been
called the Empire State---I propose to state, as shortly as may be,
the nature of its separate constitution as a State.  Of course it
will be understood that the constitutions of the different States
are by no means the same.  They have been arranged according to the
judgment of the different people concerned, and have been altered
from time to time to suit such altered judgment.  But as the States
together form one nation, and on such matters as foreign affairs,
war, customs, and post-office regulations, are bound together as
much as are the English counties, it is, of course, necessary that
the constitution of each should in most matters assimilate itself
to those of the others.  These constitutions are very much alike.
A Governor, with two houses of legislature, generally called the
Senate and the House of Representatives, exists in each State.  In
the State of New York the Lower House is called the Assembly.  In
most States the Governor is elected annually; but in some States
for two years, as in New York.  In Pennsylvania he is elected for
three years.  The House of Representatives or the Assembly is, I
think, always elected for one session only; but as in many of the
States the legislature only sits once in two years, the election
recurs of course at the same interval.  The franchise in all the
States is nearly universal, but in no State is it perfectly so.
The Governor, Lieutenant-Governor, and other officers are elected
by vote of the people, as well as the members of the legislature.
Of course it will be understood that each State makes laws for
itself---that they are in nowise dependent on the Congress assembled
at Washington for their laws---unless for laws which refer to
matters between the United States as a nation and other nations, or
between one State and another.  Each State declares with what
punishment crimes shall be visited; what taxes shall be levied for
the use of the State; what laws shall be passed as to education;
what shall be the State judiciary.  With reference to the
judiciary, however, it must be understood that the United States as
a nation have separate national law courts, before which come all
cases litigated between State and State, and all cases which do not
belong in every respect to any one individual State.  In a
subsequent chapter I will endeavor to explain this more fully.  In
endeavoring to understand the Constitution of the United States, it
is essentially necessary that we should remember that we have
always to deal with two different political arrangements---that
which refers to the nation as a whole, and that which belongs to
each State as a separate governing power in itself.  What is law in
one State is not law in another, nevertheless there is a very great
likeness throughout these various constitutions, and any political
student who shall have thoroughly mastered one, will not have much
to learn in mastering the others.

This State, now called New York, was first settled by the Dutch in
1614, on Manhattan Island.  They established a government in 1629,
under the name of the New Netherlands.  In 1664 Charles II. granted
the province to his brother, James II., then Duke of York, and
possession was taken of the country on his behalf by one Colonel
Nichols.  In 1673 it was recaptured by the Dutch, but they could
not hold it, and the Duke of York again took possession by patent.
A legislative body was first assembled during the reign of Charles
II., in 1683; from which it will be seen that parliamentary
representation was introduced into the American colonies at a very
early date.  The Declaration of Independence was made by the
revolted colonies in 1776, and in 1777 the first constitution was
adopted by the State of New York.  In 1822 this was changed for
another; and the one of which I now purport to state some of the
details was brought into action in 1847.  In this constitution
there is a provision that it shall be overhauled and remodeled, if
needs be, once in twenty years.  Article XIII. Sec. 2.  ``At the
general election to be held in 1806, and in each twentieth year
thereafter, the question, `Shall there be a convention to revise
the constitution and amend the same?' shall be decided by the
electors qualified to vote for members of the legislature?''  So
that the New Yorkers, cannot be twitted with the presumption of
finality in reference to their legislative arrangements.

The present constitution begins with declaring the inviolability of
trial by jury, and of habeas corpus---``unless when, in cases of
rebellion or invasion, the public safety may require its
suspension.''  It does not say by whom it may be suspended, or who
is to judge of the public safety, but, at any rate, it may be
presumed that such suspension was supposed to come from the powers
of the State which enacted the law.  At the present moment, the
habeas corpus is suspended in New York, and this suspension has
proceeded not from the powers of the State, but from the Federal
government, without the sanction even of the Federal Congress.

``Every citizen may freely speak, write, and publish his sentiments
on all subjects, being responsible for the abuse of that right; and
no law shall be passed to restrain or abridge the liberty of speech
or of the press.''  Art. I. Sec. 8.  But at the present moment
liberty of speech and of the press is utterly abrogated in the
State of New York, as it is in other States.  I mention this not as
a reproach against either the State or the Federal government, but
to show how vain all laws are for the protection of such rights.
If they be not protected by the feelings of the people---if the
people are at any time, or from any cause, willing to abandon such
privileges, no written laws will preserve them.

In Article I. Sec. 14, there is a proviso that no land---land, that
is, used for agricultural purposes---shall be let on lease for a
longer period than twelve years.  ``No lease or grant of
agricultural land for a longer period than twelve years hereafter
made, in which shall be reserved any rent or service of any kind,
shall be valid.''  I do not understand the intended virtue of this
proviso, but it shows very clearly how different are the practices
with reference to land in England and America.  Farmers in the
States almost always are the owners of the land which they farm,
and such tenures as those by which the occupiers of land generally
hold their farms with us are almost unknown.  There is no such
relation as that of landlord and tenant as regards agricultural
holdings.

Every male citizen of New York may vote who is twenty-one, who has
been a citizen for ten days, who has lived in the State for a year,
and for four months in the county in which he votes.  He can vote
for all ``officers that now are, or hereafter may be, elective by
the people.''  Art, II. Sec. 1.  ``But,'' the section goes on to say,
``no man of color, unless he shall have been for three years a
citizen of the State, and for one year next preceding any election
shall have been possessed of a freehold estate of the value of 250
dollars, (50l.,) and shall have been actually rated, and paid a tax
thereon, shall be entitled to vote at such election.''  This is the
only embargo with which universal suffrage is laden in the State of
New York.

The third article provides for the election of the Senate and the
Assembly.  The Senate consists of thirty-two members.  And it may
here be remarked that large as is the State of New York, and great
as is its population, its Senate is less numerous than that of many
other States.  In Massachusetts, for instance, there are forty
Senators, though the population of Massachusetts is barely one-
third that of New York.  In Virginia, there are fifty Senators,
whereas the free population is not one-third of that of New York.
As a consequence, the Senate of New York is said to be filled with
men of a higher class than are generally found in the Senates of
other States.  Then follows in the article a list of the districts
which are to return the Senators.  These districts consist of one,
two, three, or in one case four counties, according to the
population.

The article does not give the number of members of the Lower House,
nor does it even state what amount of population shall be held as
entitled to a member.  It merely provides for the division of the
State into districts which shall contain an equal number, not of
population, but of voters.  The House of Assembly does consist of
128 members.

It is then stipulated that every member of both houses shall
receive three dollars a day, or twelve shillings, for their
services during the sitting of the legislature; but this sum is
never to exceed 300 dollars, or sixty pounds, in one year, unless
an extra session be called.  There is also an allowance for the
traveling expenses of members.  It is, I presume, generally known
that the members of the Congress at Washington are all paid, and
that the same is the case with reference to the legislatures of all
the States.

No member of the New York legislature can also be a member of the
Washington Congress, or hold any civil or military office under the
General States government.

A majority of each House must be present, or, as the article says,
``shall constitute a quorum to do business.''  Each House is to keep
a journal of its proceedings.  The doors are to be open---except
when the public welfare shall require secrecy.  A singular proviso
this in a country boasting so much of freedom!  For no speech or
debate in either House, shall the legislator be called in question
in any other place.  The legislature assembles on the first Tuesday
in January, and sits for about three months.  Its seat is at
Albany.

The executive power, Article IV., is to be vested in a Governor and
a Lieutenant-Governor, both of whom shall be chosen for two years.
The Governor must be a citizen of the United States, must be thirty
years of age, and have lived for the last four years in the State.
He is to be commander-in-chief of the military and naval forces of
the State, as is the President of those of the Union.  I see that
this is also the case in inland States, which one would say can
have no navies.  And with reference to some States it is enacted
that the Governor is commander-in-chief of the army, navy, and
militia, showing that some army over and beyond the militia may be
kept by the State.  In Tennessee, which is an inland State, it is
enacted that the Governor shall be ``commander-in-chief of the army
and navy of this State, and of the militia, except when they shall
be called into the service of the United States.''  In Ohio the same
is the case, except that there is no mention of militia.  In New
York there is no proviso with reference to the service of the
United States.  I mention this as it bears with some strength on
the question of the right of secession, and indicates the jealousy
of the individual States with reference to the Federal government.
The Governor can convene extra sessions of one House or of both.
He makes a message to the legislature when it meets---a sort of
Queen's speech; and he receives for his services a compensation to
be established by law.  In New York this amounts to 800l. a year.
In some States this is as low as 200l. and 300l.  In Virginia it is
1000l.  In California, 1200l.

The Governor can pardon, except in cases of treason.  He has also a
veto upon all bills sent up by the legislature.  If he exercise
this veto he returns the bill to the legislature with his reasons
for so doing.  If the bill on reconsideration by the Houses be
again passed by a majority of two-thirds in each house, it becomes
law in spite of the Governor's veto.  The veto of the President at
Washington is of the same nature.  Such are the powers of the
Governor.  But though they are very full, the Governor of each
State does not practically exercise any great political power, nor
is he, even politically, a great man.  You might live in a State
during the whole term of his government and hardly hear of him.
There is vested in him by the language of the constitution a much
wider power than that intrusted to the governor of our colonies.
But in our colonies everybody talks, and thinks, and knows about
the governor.  As far as the limits of the colony the governor is a
great man.  But this is not the case with reference to the
governors in the different States.

The next article provides that the Governor's ministers, viz, the
Secretary of State, the Controller, Treasurer, and Attorney-
General, shall be chosen every two years at a general election.  In
this respect the State constitution differs from that of the
national constitution.  The President at Washington names his own
ministers---subject to the approbation of the Senate.  He makes many
other appointments with the same limitation, and the Senate, I
believe, is not slow to interfere; but with reference to the
ministers it is understood that the names sent in by the President
shall stand.  Of the Secretary of State, Controller, etc.,
belonging to the different States, and who are elected by the
people, in a general way, one never hears.  No doubt they attend
their offices and take their pay, but they are not political
personages.

The next article, No.\ VI., refers to the judiciary, and is very
complicated.  As I cannot understand it, I will not attempt to
explain it.  Moreover, it is not within the scope of my ambition to
convey here all the details of the State constitution.  In Sec. 20
of this article it is provided that no judicial officer, except
justices of the peace, shall receive to his own use any fees or
perquisites of office.``  How pleasantly this enactment must sound
in the ears of the justices of the peace!

Article VII. refers to fiscal matters, and is more especially
interesting as showing how greatly the State of New York has
depended on its canals for its wealth.  These canals are the
property of the State; and by this article it seems to be provided
that they shall not only maintain themselves, but maintain to a
considerable extent the State expenditure also, and stand in lieu
of taxation.  It is provided, Section 6 that the ``legislature shall
not sell, lease, or otherwise dispose of any of the canals of the
State; but that they shall remain the property of the State, and
under its management forever.''  But in spite of its canals the
State does not seem to be doing very well, for I see that, in 1860,
its income was 4,780,000 dollars, and its expenditure 5,100,000,
whereas its debt was 32,500,000 dollars.  Of all the States,
Pennsylvania is the most indebted, Virginia the second, and New
York the third.  New Hampshire, Connecticut, Vermont, Delaware, and
Texas owe no State debts.  All the other State ships have taken in
ballast.

The militia is supposed to consist of all men capable of bearing
arms, under forty-five years of age.  But no one need be enrolled,
who from scruples of conscience is averse to bearing arms.  At the
present moment such scruples do not seem to be very general.  Then
follows, in Article XI., a detailed enactment as to the choosing of
militia officers.  It may be perhaps sufficient to say that the
privates are to choose the captains and the subalterns; the
captains and subalterns are to choose the field officers; and the
field officers the brigadier-generals and inspectors of brigade.
The Governor, however, with the consent of the Senate, shall
nominate all major-generals.  Now that real soldiers have
unfortunately become necessary, the above plan has not been found
to work well.

Such is the constitution of the State of New York, which has been
intended to work and does work quite separately from that of the
United States.  It will be seen that the purport has been to make
it as widely democratic as possible---to provide that all power of
all description shall come directly from the people, and that such
power shall return to the people at short intervals.  The Senate
and the Governor each remain for two years, but not for the same
two years.  If a new Senate commence its work in 1861, a new
Governor will come in in 1862.  But, nevertheless, there is in the
form of government as thus established an absence of that close and
immediate responsibility which attends our ministers.  When a man
has been voted in, it seems that responsibility is over for the
period of the required service.  He has been chosen, and the
country which has chosen him is to trust that he will do his best.
I do not know that this matters much with reference to the
legislature or governments of the different States, for their State
legislatures and governments are but puny powers; but in the
legislature and government at Washington it does matter very much.
But I shall have another opportunity of speaking on that subject.

Nothing has struck me so much in America as the fact that these
State legislatures are puny powers.  The absence of any tidings
whatever of their doings across the water is a proof of this.  Who
has heard of the legislature of New York or of Massachusetts?  It
is boasted here that their insignificance is a sign of the well-
being of the people; that the smallness of the power necessary for
carrying on the machine shows how beautifully the machine is
organized, and how well it works.  ``It is better to have little
governors than great governors,'' an American said to me once.  ``It
is our glory that we know how to live without having great men over
us to rule us.''  That glory, if ever it were a glory, has come to
an end.  It seems to me that all these troubles have come upon the
States because they have not placed high men in high places.  The
less of laws and the less of control the better, providing a people
can go right with few laws and little control.  One may say that no
laws and no control would be best of all---provided that none were
needed.  But this is not exactly the position of the American
people.

The two professions of law-making and of governing have become
unfashionable, low in estimation, and of no repute in the States.
The municipal powers of the cities have not fallen into the hands
of the leading men.  The word politician has come to bear the
meaning of political adventurer and almost of political blackleg.
If A calls B a politician, A intends to vilify B by so calling him.
Whether or no the best citizens of a State will ever be induced to
serve in the State legislature by a nobler consideration than that
of pay, or by a higher tone of political morals than that now
existing, I cannot say.  It seems to me that some great decrease in
the numbers of the State legislators should be a first step toward
such a consummation.  There are not many men in each State who can
afford to give up two or three months of the year to the State
service for nothing; but it may be presumed that in each State
there are a few.  Those who are induced to devote their time by the
payment of 60l. can hardly be the men most fitted for the purpose
of legislation.  It certainly has seemed to me that the members of
the State legislatures and of the State governments are not held in
that respect and treated with that confidence to which, in the eyes
of an Englishman, such functionaries should be held as entitled.



\chapter{Boston}


From New York we returned to Boston by Hartford, the capital or one
of the capitals of Connecticut.  This proud little State is
composed of two old provinces, of which Hartford and New Haven were
the two metropolitan towns.  Indeed, there was a third colony,
called Saybrook, which was joined to Hartford.  As neither of the
two could, of course, give way, when Hartford and New Haven were
made into one, the houses of legislature and the seat of government
are changed about year by year.  Connecticut is a very proud little
State, and has a pleasant legend of its own stanchness in the old
colonial days.  In 1662 the colonies were united, and a charter was
given to them by Charles II.  But some years later, in 1686, when
the bad days of James II. had come, this charter was considered to
be too liberal, and order was given that it should be suspended.
One Sir Edmund Andross had been appointed governor of all New
England, and sent word from Boston to Connecticut that the charter
itself should be given up to him.  This the men of Connecticut
refused to do.  Whereupon Sir Edmund with a military following
presented himself at their Assembly, declared their governing
powers to be dissolved, and, after much palaver, caused the charter
itself to be laid upon the table before him.  The discussion had
been long, having lasted through the day into the night, and the
room had been lighted with candles.  On a sudden each light
disappeared, and Sir Edmund with his followers were in the dark.
As a matter of course, when the light was restored the charter was
gone; and Sir Edmund, the governor-general, was baffled, as all
governors-general and all Sir Edmunds always are in such cases.
The charter was gone, a gallant Captain Wadsworth having carried it
off and hidden it in an oak-tree.  The charter was renewed when
William III. came to the throne, and now hangs triumphantly in the
State House at Hartford.  The charter oak has, alas! succumbed to
the weather, but was standing a few years since.  The men of
Hartford are very proud of their charter, and regard it as the
parent of their existing liberties quite as much as though no
national revolution of their own had intervened.

And, indeed, the Northern States of the Union---especially those of
New England---refer all their liberties to the old charters which
they held from the mother country.  They rebelled, as they
themselves would seem to say, and set themselves up as a separate
people, not because the mother country had refused to them by law
sufficient liberty and sufficient self-control, but because the
mother country infringed the liberties and powers of self-control
which she herself had given.  The mother country, so these States
declare, had acted the part of Sir Edmund Andross---had endeavored
to take away their charters.  So they also put out the lights, and
took themselves to an oak-tree of their own---which is still
standing, though winds from the infernal regions are now battering
its branches.  Long may it stand!

Whether the mother country did or did not infringe the charters she
had given, I will not here inquire.  As to the nature of those
alleged infringements, are they not written down to the number of
twenty-seven in the Declaration of Independence?  They mostly begin
with He.  ``He'' has done this, and ``He'' has done that.  The ``He'' is
poor George III., whose twenty-seven mortal sins against his
Transatlantic colonies are thus recapitulated.  It would avail
nothing to argue now whether those deeds were sins or virtues, nor
would it have availed then.  The child had grown up and was strong,
and chose to go alone into the world.  The young bird was fledged,
and flew away.  Poor George III. with his cackling was certainly
not efficacious in restraining such a flight.  But it is gratifying
to see how this new people, when they had it in their power to
change all their laws, to throw themselves upon any Utopian theory
that the folly of a wild philanthropy could devise, to discard as
abominable every vestige of English rule and English power,---it is
gratifying to see that, when they could have done all this, they
did not do so, but preferred to cling to things English.  Their old
colonial limits were still to be the borders of their States.
Their old charters were still to be regarded as the sources from
whence their State powers had come.  The old laws were to remain in
force.  The precedents of the English courts were to be held as
legal precedents in the courts of the new nation, and are now so
held.  It was still to be England, but England without a king
making his last struggle for political power.  This was the idea of
the people and this was their feeling; and that idea has been
carried out and that feeling has remained.

In the constitution of the State of New York nothing is said about
the religion of the people.  It was regarded as a subject with
which the constitution had no concern whatever.  But as soon as we
come among the stricter people of New England, we find that the
constitution-makers have not been able absolutely to ignore the
subject.  In Connecticut it is enjoined that, as it is the duty of
all men to worship the Supreme Being, and their right to render
that worship in the mode most consistent with their consciences, no
person shall be by law compelled to join or be classed with any
religious association.  The line of argument is hardly logical, the
conclusion not being in accordance with or hanging on the first of
the two premises.  But nevertheless the meaning is clear.  In a
free country no man shall be made to worship after any special
fashion; but it is decreed by the constitution that every man is
bound by duty to worship after some fashion.  The article then goes
on to say how they who do worship are to be taxed for the support
of their peculiar church.  I am not quite clear whether the New
Yorkers have not managed this difficulty with greater success.
When we come to the Old Bay State---to Massachusetts---we find the
Christian religion spoken of in the constitution as that which in
some one of its forms should receive the adherence of every good
citizen.

Hartford is a pleasant little town, with English-looking houses,
and an English-looking country around it.  Here, as everywhere
through the States, one is struck by the size and comfort of the
residences.  I sojourned there at the house of a friend, and could
find no limit to the number of spacious sitting-rooms which it
contained.  The modest dining-room and drawing-room which suffice
with us for men of seven or eight hundred a year would be regarded
as very mean accommodation by persons of similar incomes in the
States.

I found that Hartford was all alive with trade, and that wages were
high, because there are there two factories for the manufacture of
arms.  Colt's pistols come from Hartford, as also do Sharpe's
rifles.  Wherever arms can be prepared, or gunpowder; where clothes
or blankets fit for soldiers can be made, or tents or standards, or
things appertaining in any way to warfare, there trade was still
brisk.  No being is more costly in his requirements than a soldier,
and no soldier so costly as the American.  He must eat and drink of
the best, and have good boots and warm bedding, and good shelter.
There were during the Christmas of 1861 above half a million of
soldiers so to be provided---the President, in his message made in
December to Congress, declared the number to be above six hundred
thousand---and therefore in such places as Hartford trade was very
brisk.  I went over the rifle factory, and was shown everything,
but I do not know that I brought away much with me that was worth
any reader's attention.  The best of rifles, I have no doubt, were
being made with the greatest rapidity, and all were sent to the
army as soon as finished.  I saw some murderous-looking weapons,
with swords attached to them instead of bayonets, but have since
been told by soldiers that the old-fashioned bayonet is thought to
be more serviceable.

Immediately on my arrival in Boston I heard that Mr.\ Emerson was
going to lecture at the Tremont Hall on the subject of the war, and
I resolved to go and hear him.  I was acquainted with Mr.\ Emerson,
and by reputation knew him well.  Among us in England he is
regarded as transcendental and perhaps even as mystic in his
philosophy.  His ``Representative Men'' is the work by which he is
best known on our side of the water, and I have heard some readers
declare that they could not quite understand Mr.\ Emerson's
``Representative Men.''  For myself, I confess that I had broken down
over some portions of that book.  Since I had become acquainted
with him I had read others of his writings, especially his book on
England, and had found that he improved greatly on acquaintance.  I
think that he has confined his mysticism to the book above named.
In conversation he is very clear, and by no means above the small
practical things of the world.  He would, I fancy, know as well
what interest he ought to receive for his money as though he were
no philosopher, and I am inclined to think that if he held land he
would make his hay while the sun shone, as might any common farmer.
Before I had met Mr.\ Emerson, when my idea of him was formed simply
on the ``Representative Men,'' I should have thought that a lecture
from him on the war would have taken his hearers all among the
clouds.  As it was, I still had my doubts, and was inclined to fear
that a subject which could only be handled usefully at such a time
before a large audience by a combination of common sense, high
principles, and eloquence, would hardly be safe in Mr.\ Emerson's
hands.  I did not doubt the high principles, but feared much that
there would be a lack of common sense.  So many have talked on that
subject, and have shown so great a lack of common sense!  As to the
eloquence, that might be there or might not.

Mr.\ Emerson is a Massachusetts man, very well known in Boston, and
a great crowd was collected to hear him.  I suppose there were some
three thousand persons in the room.  I confess that when he took
his place before us my prejudices were against him.  The matter in
hand required no philosophy.  It required common sense, and the
very best of common sense.  It demanded that he should be
impassioned, for of what interest can any address be on a matter of
public politics without passion?  But it demanded that the passion
should be winnowed, and free from all rodomontade.  I fancied what
might be said on such a subject as to that overlauded star-spangled
banner, and how the star-spangled flag would look when wrapped in a
mist of mystic Platonism.

But from the beginning to the end there was nothing mystic---no
Platonism; and, if I remember rightly, the star-spangled banner was
altogether omitted.  To the national eagle he did allude.  ``Your
American eagle,'' he said, ``is very well.  Protect it here and
abroad.  But beware of the American peacock.''  He gave an account
of the war from the beginning, showing how it had arisen, and how
it had been conducted; and he did so with admirable simplicity and
truth.  He thought the North were right about the war; and as I
thought so also, I was not called upon to disagree with him.  He
was terse and perspicuous in his sentences, practical in his
advice, and, above all things, true in what he said to his audience
of themselves.  They who know America will understand how hard it
is for a public man in the States to practice such truth in his
addresses.  Fluid compliments and high-flown national eulogium are
expected.  In this instance none were forthcoming.  The North had
risen with patriotism to make this effort, and it was now warned
that in doing so it was simply doing its national duty.  And then
came the subject of slavery.  I had been told that Mr.\ Emerson was
an abolitionist, and knew that I must disagree with him on that
head, if on no other.  To me it has always seemed that to mix up
the question of general abolition with this war must be the work of
a man too ignorant to understand the real subject of the war, or
too false to his country to regard it.  Throughout the whole
lecture I was waiting for Mr.\ Emerson's abolition doctrine, but no
abolition doctrine came.  The words abolition and compensation were
mentioned, and then there was an end of the subject.  If Mr.\ %
Emerson be an abolitionist, he expressed his views very mildly on
that occasion.  On the whole, the lecture was excellent, and that
little advice about the peacock was in itself worth an hour's
attention.

That practice of lecturing is ``quite an institution'' in the States.
So it is in England, my readers will say.  But in England it is
done in a different way, with a different object, and with much
less of result.  With us, if I am not mistaken, lectures are mostly
given gratuitously by the lecturer.  They are got up here and there
with some philanthropical object, and in the hope that an hour at
the disposal of young men and women may be rescued from idleness.
The subjects chosen are social, literary, philanthropic, romantic,
geographical, scientific, religious---anything rather than
political.  The lecture-rooms are not usually filled to
overflowing, and there is often a question whether the real good
achieved is worth the trouble taken.  The most popular lectures are
given by big people, whose presence is likely to be attractive; and
the whole thing, I fear we must confess, is not pre-eminently
successful.  In the Northern States of America the matter stands on
a very different footing.  Lectures there are more popular than
either theaters or concerts.  Enormous halls are built for them.
Tickets for long courses are taken with avidity.  Very large sums
are paid to popular lecturers, so that the profession is lucrative---%
more so, I am given to understand, than is the cognate profession
of literature.  The whole thing is done in great style.  Music is
introduced.  The lecturer stands on a large raised platform, on
which sit around him the bald and hoary-headed and superlatively
wise.  Ladies come in large numbers, especially those who aspire to
soar above the frivolities of the world.  Politics is the subject
most popular, and most general.  The men and women of Boston could
no more do without their lectures than those of Paris could without
their theaters.  It is the decorous diversion of the best ordered
of her citizens.  The fast young men go to clubs, and the fast
young women to dances, as fast young men and women do in other
places that are wicked; but lecturing is the favorite diversion of
the steady-minded Bostonian.  After all, I do not know that the
result is very good.  It does not seem that much will be gained by
such lectures on either side of the Atlantic---except that
respectable killing of an evening which might otherwise be killed
less respectably.  It is but an industrious idleness, an attempt at
a royal road to information, that habit of attending lectures.  Let
any man or woman say what he has brought away from any such
attendance.  It is attractive, that idea of being studious without
any of the labor of study; but I fear it is illusive.  If an
evening can be so passed without ennui, I believe that that may be
regarded as the best result to be gained.  But then it so often
happens that the evening is not passed without ennui!  Of course in
saying this, I am not alluding to lectures given in special places
as a course of special study.  Medical lectures are, or may be, a
necessary part of medical education.  As many as two or three
thousand often attend these popular lectures in Boston, but I do
not know whether on that account the popular subjects are much
better understood.  Nevertheless I resolved to hear more, hoping
that I might in that way teach myself to understand what were the
popular politics in New England.  Whether or no I may have learned
this in any other way, I do not perhaps know; but at any rate I did
not learn it in this way.

The next lecture which I attended was also given in the Tremont
Hall, and on this occasion also the subject of the war was to be
treated.  The special treachery of the rebels was, I think, the
matter to be taken in hand.  On this occasion also the room was
full, and my hopes of a pleasant hour ran high.  For some fifteen
minutes I listened, and I am bound to say that the gentleman
discoursed in excellent English.  He was master of that wonderful
fluency which is peculiarly the gift of an American.  He went on
from one sentence to another with rhythmic tones and unerring
pronunciation.  He never faltered, never repeated his words, never
fell into those vile half-muttered hems and haws by which an
Englishman in such a position so generally betrays his timidity.
But during the whole time of my remaining in the room he did not
give expression to a single thought.  He went on from one soft
platitude to another, and uttered words from which I would defy any
one of his audience to carry away with them anything.  And yet it
seemed to me that his audience was satisfied.  I was not satisfied,
and managed to escape out of the room.

The next lecturer to whom I listened was Mr.\ Everett.  Mr.\ %
Everett's reputation as an orator is very great, and I was
especially anxious to hear him.  I had long since known that his
power of delivery was very marvelous; that his tones, elocution,
and action were all great; and that he was able to command the
minds and sympathies of his audience in a remarkable manner.  His
subject also was the war---or rather the causes of the war and its
qualification.  Had the North given to the South cause of
provocation?  Had the South been fair and honest in its dealings to
the North?  Had any compromise been possible by which the war might
have been avoided, and the rights and dignity of the North
preserved?  Seeing that Mr.\ Everett is a Northern man and was
lecturing to a Boston audience, one knew well how these questions
would be answered, but the manner of the answering would be
everything.  This lecture was given at Roxbury, one of the suburbs
of Boston.  So I went out to Roxbury with a party, and found myself
honored by being placed on the platform among the bald-headed ones
and the superlatively wise.  This privilege is naturally
gratifying, but it entails on him who is so gratified the
inconvenience of sitting at the lecturer's back, whereas it is,
perhaps, better for the listener to be before his face.

I could not but be amused by one little scenic incident.  When we
all went upon the platform, some one proposed that the clergymen
should lead the way out of the little waiting-room in which we
bald-headed ones and superlatively wise were assembled.  But to
this the manager of the affair demurred.  He wanted the clergymen
for a purpose, he said.  And so the profane ones led the way, and
the clergymen, of whom there might be some six or seven, clustered
in around the lecturer at last.  Early in his discourse, Mr.\ %
Everett told us what it was that the country needed at this period
of her trial.  Patriotism, courage, the bravery of the men, the
good wishes of the women, the self-denial of all---``and,'' continued
the lecturer, turning to his immediate neighbors, ``the prayers of
these holy men whom I see around me.''  It had not been for nothing
that the clergymen were detained.

Mr.\ Everett lectures without any book or paper before him, and
continues from first to last as though the words came from him on
the spur of the moment.  It is known, however, that it is his
practice to prepare his orations with great care and commit them
entirely to memory, as does an actor.  Indeed, he repeats the same
lecture over and over again, I am told, without the change of a
word or of an action.  I did not like Mr.\ Everett's lecture.  I did
not like what he said, or the seeming spirit in which it was
framed.  But I am bound to admit that his power of oratory is very
wonderful.  Those among his countrymen who have criticised his
manner in my hearing, have said that he is too florid, that there
is an affectation in the motion of his hands, and that the intended
pathos of his voice sometimes approaches too near the precipice
over which the fall is so deep and rapid, and at the bottom of
which lies absolute ridicule.  Judging for myself, I did not find
it so.  My position for seeing was not good, but my ear was not
offended.  Critics also should bear in mind that an orator does not
speak chiefly to them or for their approval.  He who writes, or
speaks, or sings for thousands, must write, speak, or sing as those
thousands would have him.  That to a dainty connoisseur will be
false music, which to the general ear shall be accounted as the
perfection of harmony.  An eloquence altogether suited to the
fastidious and hypercritical, would probably fail to carry off the
hearts and interest the sympathies of the young and eager.  As
regards manners, tone, and choice of words I think that the oratory
of Mr.\ Everett places him very high.  His skill in his work is
perfect.  He never falls back upon a word.  He never repeats
himself.  His voice is always perfectly under command.  As for
hesitation or timidity, the days for those failings have long
passed by with him.  When he makes a point, he makes it well, and
drives it home to the intelligence of every one before him.  Even
that appeal to the holy men around him sounded well---or would have
done so had I not been present at that little arrangement in the
anteroom.  On the audience at large it was manifestly effective.

But nevertheless the lecture gave me but a poor idea of Mr.\ Everett
as a politician, though it made me regard him highly as an orator.
It was impossible not to perceive that he was anxious to utter the
sentiments of the audience rather than his own; that he was making
himself an echo, a powerful and harmonious echo of what he
conceived to be public opinion in Boston at that moment; that he
was neither leading nor teaching the people before him, but
allowing himself to be led by them, so that he might best play his
present part for their delectation.  He was neither bold nor
honest, as Emerson had been, and I could not but feel that every
tyro of a politician before him would thus recognize his want of
boldness and of honesty.  As a statesman, or as a critic of
statecraft, and of other statesmen, he is wanting in backbone.  For
many years Mr.\ Everett has been not even inimical to Southern
politics and Southern courses, nor was he among those who, during
the last eight years previous to Mr.\  Lincoln's election, fought
the battle for Northern principles.  I do not say that on this
account he is now false to advocate the war.  But he cannot carry
men with him when, at his age, he advocates it by arguments opposed
to the tenor of his long political life.  His abuse of the South
and of Southern ideas was as virulent as might be that of a young
lad now beginning his political career, or of one who had through
life advocated abolition principles.  He heaped reproaches on poor
Virginia, whose position as the chief of the border States has
given to her hardly the possibility of avoiding a Scylla of ruin on
the one side, or a Charybdis of rebellion on the other.  When he
spoke as he did of Virginia, ridiculing the idea of her sacred
soil, even I, Englishman as I am, could not but think of
Washington, of Jefferson, of Randolph, and of Madison.  He should
not have spoken of Virginia as he did speak; for no man could have
known better Virginia's difficulties.  But Virginia was at a
discount in Boston, and Mr.\ Everett was speaking to a Boston
audience.  And then he referred to England and to Europe.  Mr.\ %
Everett has been minister to England, and knows the people.  He is
a student of history, and must, I think, know that England's career
has not been unhappy or unprosperous.  But England also was at a
discount in Boston, and Mr.\ Everett was speaking to a Boston
audience.  They are sending us their advice across the water, said
Mr.\ Everett.  And what is their advice to us?  That we should come
down from the high place we have built for ourselves, and be even
as they are.  They screech at us from the low depths in which they
are wallowng in their misery, and call on us to join them in their
wretchedness.  I am not quoting Mr.\ Everett's very words, for I
have not them by me; but I am not making them stronger, nor so
strong as he made them.  As I thought of Mr.\ Everett's reputation,
and of his years of study, of his long political life and
unsurpassed sources of information, I could not but grieve heartily
when I heard such words fall from him.  I could not but ask myself
whether it were impossible that under the present circumstances of
her constitution this great nation of America should produce an
honest, high-minded statesman.  When Lincoln and Hamlin, the
existing President and Vice-President of the States, were in 1860
as yet but the candidates of the Republican party, Bell and Everett
also were the candidates of the old Whig, conservative party.
Their express theory was this---that the question of slavery should
not be touched.  Their purpose was to crush agitation and restore
harmony by an impartial balance between the North and South: a fine
purpose---the finest of all purposes, had it been practicable.  But
such a course of compromise was now at a discount in Boston, and
Mr.\ Everett was speaking to a Boston audience.  As an orator, Mr.\ %
Everett's excellence is, I think, not to be questioned; but as a
politician I cannot give him a high rank.

After that I heard Mr.\ Wendell Phillips.  Of him, too, as an
orator, all the world of Massachusetts speaks with great
admiration, and I have no doubt so speaks with justice.  He is,
however, known as the hottest and most impassioned advocate of
abolition.  Not many months since the cause of abolition, as
advocated by him, was so unpopular in Boston, that Mr.\ Phillips was
compelled to address his audience surrounded by a guard of
policemen.  Of this gentleman I may at any rate say that he is
consistent, devoted, and disinterested.  He is an abolitionist by
profession, and seeks to find in every turn of the tide of politics
some stream on which he may bring himself nearer to his object.  In
the old days, previous to the selection of Mr.\ Lincoln, in days so
old that they are now nearly eighteen months past, Mr.\ Phillips was
an anti-Union man.  He advocated strongly the disseverance of the
Union, so that the country to which he belonged might have hands
clean from the taint of slavery.  He had probably acknowledged to
himself that while the North and South were bound together no hope
existed of emancipation, but that if the North stood alone the
South would become too weak to foster and keep alive the ``social
institution.''  In which, if such were his opinions, I am inclined
to agree with him.  But now he is all for the Union, thinking that
a victorious North can compel the immediate emancipation of
Southern slaves.  As to which I beg to say that I am bold to differ
from Mr.\ Phillips altogether.

It soon became evident to me that Mr.\ Phillips was unwell, and
lecturing at a disadvantage.  His manner was clearly that of an
accustomed orator, but his voice was weak, and he was not up to the
effect which he attempted to make.  His hearers were impatient,
repeatedly calling upon him to speak out, and on that account I
tried hard to feel kindly toward him and his lecture.  But I must
confess that I failed.  To me it seemed that the doctrine he
preached was one of rapine, bloodshed, and social destruction.  He
would call upon the government and upon Congress to enfranchise the
slaves at once---now during the war---so that the Southern power
might be destroyed by a concurrence of misfortunes.  And he would
do so at once, on the spur of the moment, fearing lest the South
should be before him, and themselves emancipate their own bondsmen.
I have sometimes thought that there is no being so venomous, so
blood-thirsty as a professed philanthropist; and that when the
philanthropist's ardor lies negroward, it then assumes the deepest
die of venom and blood-thirstiness.  There are four millions of
slaves in the Southern States, none of whom have any capacity for
self-maintenance or self-control.  Four millions of slaves, with
the necessities of children, with the passions of men, and the
ignorance of savages!  And Mr.\ Phillips would emancipate these at a
blow; would, were it possible for him to do so, set them loose upon
the soil to tear their masters, destroy each other, and make such a
hell upon the earth as has never even yet come from the
uncontrolled passions and unsatisfied wants of men.  But Congress
cannot do this.  All the members of Congress put together cannot,
according to the Constitution of the United States, emancipate a
single slave in South Carolina; not if they were all unanimous.  No
emancipation in a slave State can come otherwise than by the
legislative enactment of that State.  But it was then thought that
in this coming winter of 1860-61 the action of Congress might be
set aside.  The North possessed an enormous army under the control
of the President.  The South was in rebellion, and the President
could pronounce, and the army perhaps enforce, the confiscation of
all property held in slaves.  If any who held them were not
disloyal, the question of compensation might be settled afterward.
How those four million slaves should live, and how white men should
live among them, in some States or parts of States not equal to the
blacks in number---as to that Mr.\ Phillips did not give us his
opinion.

And Mr.\ Phillips also could not keep his tongue away from the
abominations of Englishmen and the miraculous powers of his own
countrymen.  It was on this occasion that he told us more than once
how Yankees carried brains in their fingers, whereas ``common
people''---alluding by that name to Europeans---had them only, if at
all, inside their brain-pans.  And then he informed us that Lord
Palmerston had always hated America.  Among the Radicals there
might be one or two who understood and valued the institutions of
America, but it was a well-known fact that Lord Palmerston was
hostile to the country.  Nothing but hidden enmity---enmity hidden
or not hidden---could be expected from England.  That the people of
Boston, or of Massachusetts, or of the North generally, should feel
sore against England, is to me intelligible.  I know how the minds
of men are moved in masses to certain feelings and that it ever
must be so.  Men in common talk are not bound to weigh their words,
to think, and speculate on their results, and be sure of the
premises on which their thoughts are founded.  But it is different
with a man who rises before two or three thousand of his countrymen
to teach and instruct them.  After that I heard no more political
lectures in Boston.

Of course I visited Bunker Hill, and went to Lexington and Concord.
From the top of the monument on Bunker Hill there is a fine view of
Boston harbor, and seen from thence the harbor is picturesque.  The
mouth is crowded with islands and jutting necks and promontories;
and though the shores are in no place rich enough to make the
scenery grand, the general effect is good.  The monument, however,
is so constructed that one can hardly get a view through the
windows at the top of it, and there is no outside gallery round it.
Immediately below the monument is a marble figure of Major Warren,
who fell there,---not from the top of the monument, as some one was
led to believe when informed that on that spot the major had
fallen.  Bunker Hill, which is little more than a mound, is at
Charlestown---a dull, populous, respectable, and very unattractive
suburb of Boston.

Bunker Hill has obtained a considerable name, and is accounted
great in the annals of American history.  In England we have all
heard of Bunker Hill, and some of us dislike the sound as much as
Frenchmen do that of Waterloo.  In the States men talk of Bunker
Hill as we may, perhaps, talk of Agincourt and such favorite
fields.  But, after all, little was done at Bunker Hill, and, as
far as I can learn, no victory was gained there by either party.
The road from Boston to the town of Concord, on which stands the
village of Lexington, is the true scene of the earliest and
greatest deeds of the men of Boston.  The monument at Bunker Hill
stands high and commands attention, while those at Lexington and
Concord are very lowly and command no attention.  But it is of that
road and what was done on it that Massachusetts should be proud.
When the colonists first began to feel that they were oppressed,
and a half resolve was made to resist that oppression by force,
they began to collect a few arms and some gunpowder at Concord, a
small town about eighteen miles from Boston.  Of this preparation
the English governor received tidings, and determined to send a
party of soldiers to seize the arms.  This he endeavored to do
secretly; but he was too closely watched, and word was sent down
over the waters by which Boston was then surrounded that the
colonists might be prepared for the soldiers.  At that time Boston
Neck, as it was, and is still called, was the only connection
between the town and the main-land, and the road over Boston Neck
did not lead to Concord.  Boats therefore were necessarily used,
and there was some difficulty in getting the soldiers to the
nearest point.  They made their way, however, to the road, and
continued their route as far as Lexington without interruption.
Here, however, they were attacked, and the first blood of that war
was shed.  They shot three or four of the---rebels, I suppose I
should in strict language call them, and then proceeded on to
Concord.  But at Concord they were stopped and repulsed, and along
the road back from Concord to Lexington they were driven with
slaughter and dismay.  And thus the rebellion was commenced which
led to the establishment of a people which, let us Englishmen say
and think what we may of them at this present moment, has made
itself one of the five great nations of the earth, and has enabled
us to boast that the two out of the five who enjoy the greatest
liberty and the widest prosperity speak the English language and
are known by English names.  For all that has come and is like to
come, I say again, long may that honor remain.  I could not but
feel that that road from Boston to Concord deserves a name in the
world's history greater, perhaps, than has yet been given to it.

Concord is at present to be noted as the residence of Mr.\ Emerson
and of Mr.\ Hawthorne, two of those many men of letters of whose
presence Boston and its neighborhood have reason to be proud.  Of
Mr.\ Emerson I have already spoken.  The author of the ``Scarlet
Letter'' I regard as certainly the first of American novelists.  I
know what men will say of Mr.\ Cooper,---and I also am an admirer of
Cooper's novels.  But I cannot think that Mr.\ Cooper's powers were
equal to those of Mr.\ Hawthorne, though his mode of thought may
have been more genial, and his choice of subjects more attractive
in their day.  In point of imagination, which, after all, is the
novelist's greatest gift, I hardly know any living author who can
he accounted superior to Mr.\ Hawthorne.

Very much has, undoubtedly, been done in Boston to carry out that
theory of Colonel Newcome's---Emollit mores, by which the Colonel
meant to signify his opinion that a competent knowledge of reading,
writing, and arithmetic, with a taste for enjoying those
accomplishments, goes very far toward the making of a man, and will
by no means mar a gentleman.  In Boston nearly every man, woman,
and child has had his or her manners so far softened; and though
they may still occasionally be somewhat rough to the outer touch,
the inward effect is plainly visible.  With us, especially among
our agricultural population, the absence of that inner softening is
as visible.

I went to see a public library in the city, which, if not founded
by Mr.\ Bates, whose name is so well known in London as connected
with the house of Messrs. Baring, has been greatly enriched by him.
It is by his money that it has been enabled to do its work.  In
this library there is a certain number of thousands of volumes---a
great many volumes, as there are in most public libraries.  There
are books of all classes, from ponderous unreadable folios, of
which learned men know the title-pages, down to the lightest
literature.  Novels are by no means eschewed,---are rather, if I
understood aright, considered as one of the staples of the library.
From this library any book, excepting such rare volumes as in all
libraries are considered holy, is given out to any inhabitant of
Boston, without any payment, on presentation of a simple request on
a prepared form.  In point of fact, it is a gratuitous circulating
library open to all Boston, rich or poor, young or old.  The books
seemed in general to be confided to young children, who came as
messengers from their fathers and mothers, or brothers and sisters.
No question whatever is asked, if the applicant is known or the
place of his residence undoubted.  If there be no such knowledge,
or there be any doubt as to the residence, the applicant is
questioned, the object being to confine the use of the library to
the bona fide inhabitants of the city.  Practically the books are
given to those who ask for them, whoever they may be.  Boston
contains over 200,000 inhabitants, and all those 200,000 are
entitled to them.  Some twenty men and women are kept employed from
morning till night in carrying on this circulating library; and
there is, moreover, attached to the establishment a large reading-
room supplied with papers and magazines, open to the public of
Boston on the same terms.

Of course I asked whether a great many of the books were not lost,
stolen, and destroyed; and of course I was told that there were no
losses, no thefts, and no destruction.  As to thefts, the librarian
did not seem to think that any instance of such an occurrence could
be found.  Among the poorer classes, a book might sometimes be lost
when they were changing their lodgings; but anything so lost was
more than replaced by the fines.  A book is taken out for a week,
and if not brought back at the end of that week---when the loan can
be renewed if the reader wishes---a fine, I think of two cents, is
incurred.  The children, when too late with the books, bring in the
two cents as a matter of course, and the sum so collected fully
replaces all losses.  It was all couleur de rose; the
librarianesses looked very pretty and learned, and, if I remember
aright, mostly wore spectacles; the head librarian was
enthusiastic; the nice, instructive books were properly dogs-eared;
my own productions were in enormous demand; the call for books over
the counter was brisk; and the reading-room was full of readers.

It has, I dare say, occurred to other travelers to remark that the
proceedings at such institutions, when visited by them on their
travels, are always rose colored.  It is natural that the bright
side should be shown to the visitor.  It may be that many books are
called for and returned unread; that many of those taken out are so
taken by persons who ought to pay for their novels at circulating
libraries; that the librarian and librarianesses get very tired of
their long hours of attendance, for I found that they were very
long; and that many idlers warm themselves in that reading-room.
Nevertheless the fact remains---the library is public to all the men
and women in Boston, and books are given out without payment to all
who may choose to ask for them.  Why should not the great Mr.\ Mudie
emulate Mr.\ Bates, and open a library in London on the same system?

The librarian took me into one special room, of which he himself
kept the key, to show me a present which the library had received
from the English government.  The room was filled with volumes of
two sizes, all bound alike, containing descriptions and drawings of
all the patents taken out in England.  According to this librarian,
such a work would be invaluable as to American patents; but he
conceived that the subject had become too confused to render any
such an undertaking possible.  ``I never allow a single volume to be
used for a moment without the presence of myself or one of my
assistants,'' said the librarian; and then he explained to me, when
I asked him why he was so particular, that the drawings would, as a
matter of course, be cut out and stolen if he omitted his care.
``But they may be copied,'' I said.  ``Yes; but if Jones merely copies
one, Smith may come after him and copy it also.  Jones will
probably desire to hinder Smith from having any evidence of such a
patent.''  As to the ordinary borrowing and returning of books, the
poorest laborer's child in Boston might be trusted as honest; but
when a question of trade came up---of commercial competition---then
the librarian was bound to bethink himself that his countrymen are
very smart.  ``I hope,'' said the librarian, ``you will let them know
in England how grateful we are for their present.''  And I hereby
execute that librarian's commission.

I shall always look back to social life in Boston with great
pleasure.  I met there many men and women whom to know is a
distinction, and with whom to be intimate is a great delight.  It
was a Puritan city, in which strict old Roundhead sentiments and
laws used to prevail; but now-a-days ginger is hot in the mouth
there, and, in spite of the war, there were cakes and ale.  There
was a law passed in Massachusetts in the old days that any girl
should be fined and imprisoned who allowed a young man to kiss her.
That law has now, I think, fallen into abeyance, and such matters
are regulated in Boston much as they are in other large towns
farther eastward.  It still, I conceive, calls itself a Puritan
city; but it has divested its Puritanism of austerity, and clings
rather to the politics and public bearing of its old fathers than
to their social manners and pristine severity of intercourse.  The
young girls are, no doubt, much more comfortable under the new
dispensation---and the elderly men also, as I fancy.  Sunday, as
regards the outer streets, is sabbatical.  But Sunday evenings
within doors I always found to be what my friends in that country
call ``quite a good time.''  It is not the thing in Boston to smoke
in the streets during the day; but the wisest, the sagest, and the
most holy---even those holy men whom the lecturer saw around him---%
seldom refuse a cigar in the dining-room as soon as the ladies have
gone.  Perhaps even the wicked weed would make its appearance
before that sad eclipse, thereby postponing or perhaps absolutely
annihilating the melancholy period of widowhood to both parties,
and would light itself under the very eyes of those who in sterner
cities will lend no countenance to such lightings.  Ah me, it was
very pleasant!  I confess I like this abandonment of the stricter
rules of the more decorous world.  I fear that there is within me
an aptitude to the milder debaucheries which makes such deviations
pleasant.  I like to drink and I like to smoke, but I do not like
to turn women out of the room.  Then comes the question whether one
can have all that one likes together.  In some small circles in New
England I found people simple enough to fancy that they could.  In
Massachusetts the Maine liquor law is still the law of the land,
but, like that other law to which I have alluded, it has fallen
very much out of use.  At any rate, it had not reached the houses
of the gentlemen with whom I had the pleasure of making
acquaintance.  But here I must guard myself from being
misunderstood.  I saw but one drunken man through all New England,
and he was very respectable.  He was, however, so uncommonly drunk
that he might be allowed to count for two or three.  The Puritans
of Boston are, of course, simple in their habits and simple in
their expenses.  Champagne and canvas-back ducks I found to be the
provisions most in vogue among those who desired to adhere closely
to the manner of their forefathers.  Upon the whole, I found the
ways of life which had been brought over in the ``Mayflower'' from
the stern sects of England, and preserved through the revolutionary
war for liberty, to be very pleasant ways; and I made up my mind
that a Yankee Puritan can be an uncommonly pleasant fellow.  I wish
that some of them did not dine so early; for when a man sits down
at half-past two, that keeping up of the after-dinner recreations
till bedtime becomes hard work.

In Boston the houses are very spacious and excellent, and they are
always furnished with those luxuries which it is so difficult to
introduce into an old house.  They have hot and cold water pipes
into every room, and baths attached to the bedchambers.  It is not
only that comfort is increased by such arrangements, but that much
labor is saved.  In an old English house it will occupy a servant
the best part of the day to carry water up and down for a large
family.  Everything also is spacious, commodious, and well lighted.
I certainly think that in house-building the Americans have gone
beyond us, for even our new houses are not commodious as are
theirs.  One practice which they have in their cities would hardly
suit our limited London spaces.  When the body of the house is
built, they throw out the dining-room behind.  It stands alone, as
it were, with no other chamber above it, and removed from the rest
of the house.  It is consequently behind the double drawing-rooms
which form the ground floor, and is approached from them and also
from the back of the hall.  The second entrance to the dining-room
is thus near the top of the kitchen stairs, which no doubt is its
proper position.  The whole of the upper part of the house is thus
kept for the private uses of the family.  To me this plan of
building recommended itself as being very commodious.

I found the spirit for the war quite as hot at Boston now (in
November) if not hotter than it was when I was there ten weeks
earlier; and I found also, to my grief, that the feeling against
England was as strong.  I can easily understand how difficult it
must have been, and still must be, to Englishmen at home to
understand this, and see how it has come to pass.  It has not
arisen, as I think, from the old jealousy of England.  It has not
sprung from that source which for years has induced certain
newspapers, especially the New York Herald, to vilify England.  I
do not think that the men of New England have ever been, as regards
this matter, in the same boat with the New York Herald.  But when
this war between the North and South first broke out, even before
there was as yet a war, the Northern men had taught themselves to
expect what they called British sympathy, meaning British
encouragement.  They regarded, and properly regarded, the action of
the South as a rebellion, and said among themselves that so staid
and conservative a nation as Great Britain would surely countenance
them in quelling rebels.  If not, should it come to pass that Great
Britain should show no such countenance and sympathy for Northern
law, if Great Britain did not respond to her friend as she was
expected to respond, then it would appear that cotton was king, at
least in British eyes.  The war did come, and Great Britain
regarded the two parties as belligerents, standing, as far as she
was concerned, on equal grounds.  This it was that first gave rise
to that fretful anger against England which has gone so far toward
ruining the Northern cause.  We know how such passions are swelled
by being ventilated, and how they are communicated from mind to
mind till they become national.  Politicians---American politicians
I here mean---have their own future careers ever before their eyes,
and are driven to make capital where they can.  Hence it is that
such men as Mr.\ Seward in the cabinet, and Mr.\ Everett out of it,
can reconcile it to themselves to speak as they have done of
England.  It was but the other day that Mr.\ Everett spoke, in one
of his orations, of the hope that still existed that the flag of
the United States might still float over the whole continent of
North America.  What would he say of an English statesman who
should speak of putting up the Union Jack on the State House in
Boston?  Such words tell for the moment on the hearers, and help to
gain some slight popularity; but they tell for more than a moment
on those who read them and remember them.

And then came the capture of Messrs. Slidell and Mason.  I was at
Boston when those men were taken out of the ``Trent'' by the ``San
Jacinto,'' and brought to Fort Warren in Boston Harbor.  Captain
Wilkes was the officer who had made the capture, and he immediately
was recognized as a hero.  He was invited to banquets and feted.
Speeches were made to him as speeches are commonly made to high
officers who come home, after many perils, victorious from the
wars.  His health was drunk with great applause, and thanks were
voted to him by one of the Houses of Congress.  It was said that a
sword was to be given to him, but I do not think that the gift was
consummated.  Should it not have been a policeman's truncheon?  Had
he at the best done any thing beyond a policeman's work?  Of
Captain Wilkes no one would complain for doing policeman's duty.
If his country were satisfied with the manner in which he did it,
England, if she quarreled at all, would not quarrel with him.  It
may now and again become the duty of a brave officer to do work of
so low a caliber.  It is a pity that an ambitious sailor should
find himself told off for so mean a task, but the world would know
that it is not his fault.  No one could blame Captain Wilkes for
acting policeman on the seas.  But who ever before heard of giving
a man glory for achievements so little glorious?  How Captain
Wilkes must have blushed when those speeches were made to him, when
that talk about the sword came up, when the thanks arrived to him
from Congress!  An officer receives his country's thanks when he
has been in great peril, and has borne himself gallantly through
his danger; when he has endured the brunt of war, and come through
it with victory; when he has exposed himself on behalf of his
country and singed his epaulets with an enemy's fire.  Captain
Wilkes tapped a merchantman on the shoulder in the high seas, and
told him that his passengers were wanted.  In doing this he showed
no lack of spirit, for it might be his duty; but where was his
spirit when he submitted to be thanked for such work?

And then there arose a clamor of justification among the lawyers;
judges and ex-judges flew to Wheaton, Phillimore, and Lord Stowell.
Before twenty-four hours were over, every man and every woman in
Boston were armed with precedents.  Then there was the burning of
the ``Caroline.''  England had improperly burned the ``Caroline'' on
Lake Erie, or rather in one of the American ports on Lake Erie, and
had then begged pardon.  If the States had been wrong, they would
beg pardon; but whether wrong or right, they would not give up
Slidell and Mason.  But the lawyers soon waxed stronger.  The men
were manifestly ambassadors, and as such contraband of war.  Wilkes
was quite right, only he should have seized the vessel also.  He
was quite right, for though Slidell and Mason might not be
ambassadors, they were undoubtedly carrying dispatches.  In a few
hours there began to be a doubt whether the men could be
ambassadors, because if called ambassadors, then the power that
sent the embassy must be presumed to be recognized.  That Captain
Wilkes had taken no dispatches, was true; but the captain suggested
a way out of this difficulty by declaring that he had regarded the
two men themselves as an incarnated embodiment of dispatches.  At
any rate, they were clearly contraband of war.  They were going to
do an injury to the North.  It was pretty to hear the charming
women of Boston, as they became learned in the law of nations:
``Wheaton is quite clear about it,'' one young girl said to me.  It
was the first I had ever heard of Wheaton, and so far was obliged
to knock under.  All the world, ladies and lawyers, expressed the
utmost confidence in the justice of the seizure; but it was clear
that all the world was in a state of the profoundest nervous
anxiety on the subject.  To me it seemed to be the most suicidal
act that any party in a life-and-death struggle ever committed.
All Americans on both sides had felt, from the beginning of the
war, that any assistance given by England to one or the other would
turn the scale.  The government of Mr.\ Lincoln must have learned by
this time that England was at least true in her neutrality; that no
desire for cotton would compel her to give aid to the South as long
as she herself was not ill treated by the North.  But it seemed as
though Mr.\ Seward, the President's Prime Minister, had no better
work on hand than that of showing in every way his indifference as
to courtesy with England.  Insults offered to England would, he
seemed to think, strengthen his hands.  He would let England know
that he did not care for her.  When our minister, Lord Lyons,
appealed to him regarding the suspension of the habeas corpus, Mr.\ %
Seward not only answered him with insolence, but instantly
published his answer in the papers.  He instituted a system of
passports, especially constructed so as to incommode Englishmen
proceeding from the States across the Atlantic.  He resolved to
make every Englishman in America feel himself in some way punished,
because England had not assisted the North.  And now came the
arrest of Slidell and Mason out of an English mail steamer, and Mr.\ %
Seward took care to let it be understood that, happen what might,
those two men should not be given up.

Nothing during all this time astonished me so much as the
estimation in which Mr.\ Seward was then held by his own party.  It
is, perhaps, the worst defect in the constitution of the States,
that no incapacity on the part of a minister, no amount of
condemnation expressed against him by the people or by Congress,
can put him out of office during the term of the existing
Presidency.  The President can dismiss him; but it generally
happens that the President is brought in on a ``platform'' which has
already nominated for him his cabinet as thoroughly as they have
nominated him.  Mr.\ Seward ran Mr.\ Lincoln very hard for the
position of candidate for the Presidency on the Republican
interest.  On the second voting of the Republican delegates at the
Convention at Chicago, Mr.\ Seward polled 184 to Mr.\ Lincoln's 181.
But as a clear half of the total number of votes was necessary---%
that is, 233 out of 465---there was necessarily a third polling, and
Mr.\ Lincoln won the day.  On that occasion Mr.\ Chase and Mr.\ %
Cameron, both of whom became members of Mr.\ Lincoln's cabinet, were
also candidates for the White House on the Republican side.  I
mention this here to show that though the President can in fact
dismiss his ministers, he is in a great manner bound to them, and
that a minister in Mr.\ Seward's position is hardly to be dismissed.
But from the 1st of November, 1861, till the day on which I left
the States, I do not think that I heard a good word spoken of Mr.\ %
Seward as a minister, even by one of his own party.  The Radical or
Abolitionist Republicans all abused him.  The Conservative or Anti-
abolition Republicans, to whose party he would consider himself as
belonging, spoke of him as a mistake.  He had been prominent as
Senator from New York, and had been Governor of the State of New
York, but had none of the aptitudes of a statesman.  He was there,
and it was a pity.  He was not so bad as Mr.\ Cameron, the Minister
for War; that was the best his own party could say for him, even in
his own State of New York.  As to the Democrats, their language
respecting him was as harsh as any that I have heard used toward
the Southern leaders.  He seemed to have no friends, no one who
trusted him; and yet he was the President's chief minister, and
seemed to have in his own hands the power of mismanaging all
foreign relations as he pleased.  But, in truth, the States of
America, great as they are, and much as they have done, have not
produced statesmen.  That theory of governing by the little men
rather than by the great has not been found to answer, and such
follies as those of Mr.\ Seward have been the consequence.

At Boston, and indeed elsewhere, I found that there was even then---%
at the time of the capture of these two men---no true conception of
the neutrality of England with reference to the two parties.  When
any argument was made, showing that England, who had carried these
messengers from the South, would undoubtedly have also carried
messengers from the North, the answer always was---``But the
Southerners are all rebels.  Will England regard us who are by
treaty her friend, as she does a people that is in rebellion
against its own government?''  That was the old story over again,
and as it was a very long story, it was hardly of use to go back
through all its details.  But the fact was that unless there had
been such absolute neutrality---such equality between the parties in
the eyes of England---even Captain Wilkes would not have thought of
stopping the ``Trent,'' or the government at Washington of justifying
such a proceeding.  And it must be remembered that the government
at Washington had justified that proceeding.  The Secretary of the
Navy had distinctly done so in his official report; and that report
had been submitted to the President and published by his order.  It
was because England was neutral between the North and South that
Captain Wilkes claimed to have the right of seizing those two men.
It had been the President's intention, some month or so before this
affair, to send Mr.\ Everett and other gentlemen over to England
with objects as regards the North similar to those which had caused
the sending of Slidell and Mason with reference to the South.  What
would Mr.\ Everett have thought had he been refused a passage from
Dover to Calais, because the carrying of him would have been toward
the South a breach of neutrality?  It would never have occurred to
him that he could become subject to such stoppage.  How should we
have been abused for Southern sympathies had we so acted!  We,
forsooth, who carry passengers about the world, from China and
Australia, round to Chili and Peru, who have the charge of the
world's passengers and letters, and as a nation incur out of our
pocket annually loss of some half million of pounds sterling for
the privilege of doing so, are to inquire the business of every
American traveler before we let him on board, and be stopped in our
work if we take anybody on one side whose journeyings may be
conceived by the other side to be to them prejudicial!  Not on such
terms will Englishmen be willing to spread civilization across the
ocean!  I do not pretend to understand Wheaton and Phillimore, or
even to have read a single word of any international law.  I have
refused to read any such, knowing that it would only confuse and
mislead me.  But I have my common sense to guide me.  Two men
living in one street, quarrel and shy brickbats at each other, and
make the whole street very uncomfortable.  Not only is no one to
interfere with them, but they are to have the privilege of deciding
that their brickbats have the right of way, rather than the
ordinary intercourse of the neighborhood!  If that be national law,
national law must be changed.  It might do for some centuries back,
but it cannot do now.  Up to this period my sympathies had been
with the North.  I thought, and still think, that the North had no
alternative, that the war had been forced upon them, and that they
had gone about their work with patriotic energy.  But this stopping
of an English mail steamer was too much for me.

What will they do in England? was now the question.  But for any
knowledge as to that I had to wait till I reached Washington.



\chapter{Cambridge and Lowell}


The two places of most general interest in the vicinity of Boston
are Cambridge and Lowell.  Cambridge is to Massachusetts, and, I
may almost say, is to all the Northern States, what Cambridge and
Oxford are to England.  It is the seat of the university which
gives the highest education to be attained by the highest classes
in that country.  Lowell also is in little to Massachusetts and to
New England what Manchester is to us in so great a degree.  It is
the largest and most prosperous cotton-manufacturing town in the
States.

Cambridge is not above three or four miles from Boston.  Indeed,
the town of Cambridge properly so called begins where Boston
ceases.  The Harvard College---that is its name, taken from one of
its original founders---is reached by horse-cars in twenty minutes
from the city.  An Englishman feels inclined to regard the place as
a suburb of Boston; but if he so expresses himself, he will not
find favor in the eyes of the men of Cambridge.

The university is not so large as I had expected to find it.  It
consists of Harvard College, as the undergraduates' department, and
of professional schools of law, medicine, divinity, and science.
In the few words that I will say about it I will confine myself to
Harvard College proper, conceiving that the professional schools
connected with it have not in themselves any special interest.  The
average number of undergraduates does not exceed 450, and these are
divided into four classes.  The average number of degrees taken
annually by bachelors of art is something under 100.  Four years'
residence is required for a degree, and at the end of that period a
degree is given as a matter of course if the candidate's conduct
has been satisfactory.  When a young man has pursued his studies
for that period, going through the required examinations and
lectures, he is not subjected to any final examination as is the
case with a candidate for a degree at Oxford and Cambridge.  It is,
perhaps, in this respect that the greatest difference exists
between the English universities and Harvard College.  With us a
young man may, I take it, still go through his three or four years
with a small amount of study.  But his doing so does not insure him
his degree.  If he have utterly wasted his time he is plucked, and
late but heavy punishment comes upon him.  At Cambridge, in
Massachusetts, the daily work of the men is made more obligatory;
but if this be gone through with such diligence as to enable the
student to hold his own during the four years, he has his degree as
a matter of course.  There are no degrees conferring special honor.
A man cannot go out ``in honors'' as he does with us.  There are no
``firsts'' or ``double firsts;'' no ``wranglers;'' no ``senior opts'' or
``junior opts.''  Nor are there prizes of fellowships and livings to
be obtained.  It is, I think, evident from this that the greatest
incentives to high excellence are wanting at Harvard College.
There is neither the reward of honor nor of money.  There is none
of that great competition which exists at our Cambridge for the
high place of Senior Wrangler; and, consequently, the degree of
excellence attained is no doubt lower than with us.  But I conceive
that the general level of the university education is higher there
than with us; that a young man is more sure of getting his
education, and that a smaller percentage of men leaves Harvard
College utterly uneducated than goes in that condition out of
Oxford or Cambridge.  The education at Harvard College is more
diversified in its nature, and study is more absolutely the
business of the place than it is at our universities.

The expense of education at Harvard College is not much lower than
at our colleges; with us there are, no doubt, more men who are
absolutely extravagant than at Cambridge, Massachusetts.  The
actual authorized expenditure in accordance with the rules is only
50l. per annum, i.e. 249 dollars; but this does not, by any means,
include everything.  Some of the richer young men may spend as much
as 300l. per annum, but the largest number vary their expenditure
from 100l. to 180l. per annum; and I take it the same thing may be
said of our universities.  There are many young men at Harvard
College of very small means.  They will live on 70l. per annum, and
will earn a great portion of that by teaching in the vacations.
There are thirty-six scholarships attached to the university,
varying in value from 20l. to 60l. per annum; and there is also a
beneficiary fund for supplying poor scholars with assistance during
their collegiate education.  Many are thus brought up at Cambridge
who have no means of their own; and I think I may say that the
consideration in which they are held among their brother students
is in no degree affected by their position.  I doubt whether we can
say so much of the Sizars and Bible clerks at our universities.

At Harvard College there is, of course, none of that old-fashioned,
time-honored, delicious, medieval life which lends so much grace
and beauty to our colleges.  There are no gates, no porter's
lodges, no butteries, no halls, no battels, and no common rooms.
There are no proctors, no bulldogs, no bursers, no deans, no
morning and evening chapel, no quads, no surplices, no caps and
gowns.  I have already said that there are no examinations for
degrees and no honors; and I can easily conceive that in the
absence of all these essentials many an Englishman will ask what
right Harvard College has to call itself a university.

I have said that there are no honors, and in our sense there are
none.  But I should give offense to my American friends if I did
not explain that there are prizes given---I think all in money, and
that they vary from fifty to ten dollars.  These are called deturs.
The degrees are given on Commencement Day, at which occasion
certain of the expectant graduates are selected to take parts in a
public literary exhibition.  To be so selected seems to be
tantamount to taking a degree in honors.  There is also a dinner on
Commencement Day, at which, however, ``no wine or other intoxicating
drink shall be served.''

It is required that every student shall attend some place of
Christian worship on Sundays; but he, or his parents for him, may
elect what denomination of church he shall attend.  There is a
university chapel on the university grounds which belongs, if I
remember aright, to the Episcopalian church.  The young men, for
the most part, live in college, having rooms in the college
buildings; but they do not board in those rooms.  There are
establishments in the town, under the patronage of the university,
at which dinner, breakfast, and supper are provided; and the young
men frequent one of these houses or another as they, or their
friends for them, may arrange.  Every young man not belonging to a
family resident within a hundred miles of Cambridge, and whose
parents are desirous to obtain the protection thus provided, is
placed, as regards his pecuniary management, under the care of a
patron; and this patron acts by him as a father does in England by
a boy at school.  He pays out his money for him and keeps him out
of debt.  The arrangement will not recommend itself to young men at
Oxford quite so powerfully as it may do to the fathers of some
young men who have been there.  The rules with regard to the
lodging and boarding houses are very stringent.  Any festive
entertainment is to be reported to the president.  No wine or
spirituous liquors may be used, etc.  It is not a picturesque
system, this; but it has its advantages.

There is a handsome library attached to the college which the young
men can use, but it is not as extensive as I had expected.  The
university is not well off for funds by which to increase it.  The
new museum in the college is also a handsome building.  The
edifices used for the undergraduates' chambers and for the lecture-
rooms are by no means handsome.  They are very ugly, red brick
houses, standing here and there without order.  There are seven
such; and they are called Brattle House, College House, Divinity
Hall, Hollis Hall, Holsworthy Hall, Massachusetts Hall, and
Stoughton Hall.  It is almost astonishing that buildings so ugly
should have been erected for such a purpose.  These, together with
the library, the museum, and the chapel, stand on a large green,
which might be made pretty enough if it were kept well mown, like
the gardens of our Cambridge colleges; but it is much neglected.
Here, again, the want of funds---the auqusta res domi---must be
pleaded as an excuse.  On the same green, but at some little
distance from any other building, stands the president's pleasant
house.

The immediate direction of the college is of course mainly in the
hands of the president, who is supreme.  But for the general
management of the institution there is a corporation, of which he
is one.  It is stated in the laws of the university that the
Corporation of the University and its Overseers constitute the
Government of the University.  The Corporation consists of the
President, five Fellows so called, and a Treasurer.  These Fellows
are chosen, as vacancies occur, by themselves, subject to the
concurrence of the Overseers.  But these Fellows are in nowise like
to the Fellows of our colleges, having no salaries attached to
their offices.  The Board of Overseers consists of the State
Governor, other State officers, the President and Treasurer of
Harvard College, and thirty other persons, men of note, chosen by
vote.  The Faculty of the College, in which is vested the immediate
care and government of the undergraduates, is composed of the
President and the Professors.  The Professors answer to the tutors
of our colleges, and upon them the education of the place depends.
I cannot complete this short notice of Harvard College without
saying that it is happy in the possession of that distinguished
natural philosopher Professor Agassiz.  M. Agassiz has collected at
Cambridge a museum of such things as natural philosophers delight
to show, which I am told is all but invaluable.  As my ignorance on
all such matters is of a depth which the professor can hardly
imagine, and which it would have shocked him to behold, I did not
visit the museum.  Taking the University of Harvard College as a
whole, I should say that it is most remarkable in this---that it
does really give to its pupils that education which it professes to
give.  Of our own universities other good things may be said, but
that one special good thing cannot always be said.

Cambridge boasts itself as the residence of four or five men well
known to fame on the American and also on the European side of the
ocean.  President Felton's* name is very familiar to us; and
wherever Greek scholarship is held in repute, that is known.  So
also is the name of Professor Agassiz, of whom I have spoken.
Russell Lowell is one of the professors of the college---that
Russell Lowell who sang of Birdofredum Sawin, and whose Biglow
Papers were edited with such an ardor of love by our Tom Brown,
Birdofredum is worthy of all the ardor.  Mr.\ Dana is also a
Cambridge man---he who was ``two years before the mast,'' and who
since that has written to us of Cuba.  But Mr.\ Dana, though
residing at Cambridge, is not of Cambridge; and, though a literary
man, he does not belong to literature.  He is---could he help it?---a
``special attorney.''  I must not, however, degrade him; for in the
States barristers and attorneys are all one.  I cannot but think
that he could help it, and that he should not give up to law what
was meant for mankind.  I fear, however, that successful Law has
caught him in her intolerant clutches, and that Literature, who
surely would be the nobler mistress, must wear the willow.  Last
and greatest is the poet-laureate of the West, for Mr.\ Longfellow
also lives at Cambridge.


* Since these words were written President Felton has died---I, as I
returned on my way homeward, had the melancholy privilege of being
present at his funeral.  I feel bound to record here the great
kindness with which Mr.\ Felton assisted me in obtaining such
information as I needed respecting the institution over which he
presided.


I am not at all aware whether the nature of the manufacturing
corporation of Lowell is generally understood by Englishmen.  I
confess that until I made personal acquaintance with the plan, I
was absolutely ignorant on the subject.  I knew that Lowell was a
manufacturing town at which cotton is made into calico, and at
which calico is printed---as is the case at Manchester; but I
conceived this was done at Lowell, as it is done at Manchester, by
individual enterprise---that I or any one else could open a mill at
Lowell, and that the manufacturers there were ordinary traders, as
they are at other manufacturing towns.  But this is by no means the
case.

That which most surprises an English visitor on going through the
mills at Lowell is the personal appearance of the men and women who
work at them.  As there are twice as many women as there are men,
it is to them that the attention is chiefly called.  They are not
only better dressed, cleaner, and better mounted in every respect
than the girls employed at manufactories in England, but they are
so infinitely superior as to make a stranger immediately perceive
that some very strong cause must have created the difference.  We
all know the class of young women whom we generally see serving
behind counters in the shops of our larger cities.  They are neat,
well dressed, careful, especially about their hair, composed in
their manner, and sometimes a little supercilious in the propriety
of their demeanor.  It is exactly the same class of young women
that one sees in the factories at Lowell.  They are not sallow, nor
dirty, nor ragged, nor rough.  They have about them no signs of
want, or of low culture.  Many of us also know the appearance of
those girls who work in the factories in England; and I think it
will be allowed that a second glance at them is not wanting to show
that they are in every respect inferior to the young women who
attend our shops.  The matter, indeed, requires no argument.  Any
young woman at a shop would be insulted by being asked whether she
had worked at a factory.  The difference with regard to the men at
Lowell is quite as strong, though not so striking.  Working men do
not show their status in the world by their outward appearance as
readily as women; and, as I have said before, the number of the
women greatly exceeded that of the men.

One would of course be disposed to say that the superior condition
of the workers must have been occasioned by superior wages; and
this, to a certain extent, has been the cause.  But the higher
payment is not the chief cause.  Women's wages, including all that
they receive at the Lowell factories, average about 14s. a week,
which is, I take it, fully a third more than women can earn in
Manchester, or did earn before the loss of the American cotton
began to tell upon them.  But if wages at Manchester were raised to
the Lowell standard, the Manchester women would not be clothed,
fed, cared for, and educated like the Lowell women.  The fact is,
that the workmen and the workwomen at Lowell are not exposed to the
chances of an open labor market.  They are taken in, as it were, to
a philanthropical manufacturing college, and then looked after and
regulated more as girls and lads at a great seminary, than as hands
by whose industry profit is to be made out of capital.  This is all
very nice and pretty at Lowell, but I am afraid it could not be
done at Manchester.

There are at present twelve different manufactories at Lowell, each
of which has what is called a separate corporation.  The Merrimack
Manufacturing Company was incorporated in 1822, and thus Lowell was
commenced.  The Lowell Machine-shop was incorporated in 1845, and
since that no new establishment has been added.  In 1821, a certain
Boston manufacturing company, which had mills at Waltham, near
Boston, was attracted by the water-power of the River Merrimack, on
which the present town of Lowell is situated.  A canal called the
Pawtucket Canal had been made for purposes of navigation from one
reach of the river to another, with the object of avoiding the
Pawtucket Falls; and this canal, with the adjacent water-power of
the river, was purchased for the Boston company.  The place was
then called Lowell, after one of the partners in that company.

It must be understood that water-power alone is used for preparing
the cotton and working the spindles and looms of the cotton mills.
Steam is applied in the two establishments in which the cottons are
printed, for the purposes of printing, but I think nowhere else.
When the mills are at full work, about two and a half million yards
of cotton goods are made every week, and nearly a million pounds of
cotton are consumed per week, (i e. 842,000 lbs.,) but the
consumption of coal is only 30,000 tons in the year.  This will
give some idea of the value of the water-power.  The Pawtucket
Canal was, as I say, bought, and Lowell was commenced.  The town
was incorporated in 1826, and the railway between it and Boston was
opened in 1835, under the superintendence of Mr.\ Jackson, the
gentleman by whom the purchase of the canal had in the first
instance been made.  Lowell now contains about 40,000 inhabitants.

The following extract is taken from the hand-book to Lowell: ``Mr.\ %
F. C. Lowell had, in his travels abroad, observed the effect of
large manufacturing establishments on the character of the people,
and in the establishment at Waltham the founders looked for a
remedy for these defects.  They thought that education and good
morals would even enhance the profit, and that they could compete
with Great Britain by introducing a more cultivated class of
operatives.  For this purpose they built boarding-houses, which,
under the direct supervision of the agent, were kept by discreet
matrons''---I can answer for the discreet matrons at Lowell---``mostly
widows, no boarders being allowed except operatives.  Agents and
overseers of high moral character were selected; regulations were
adopted at the mills and boarding-houses, by which only respectable
girls were employed.  The mills were nicely painted and swept''---I
can also answer for the painting and sweeping at Lowell---``trees set
out in the yards and along the streets, habits of neatness and
cleanliness encouraged; and the result justified the expenditure.
At Lowell the same policy has been adopted and extended; more
spacious mills and elegant boarding-houses have been erected;'' as
to the elegance, it may be a matter of taste, but as to the
comfort, there is no question---``the same care as to the classes
employed; more capital has been expended for cleanliness and
decoration; a hospital has been established for the sick, where,
for a small price, they have an experienced physician and skillful
nurses.  An institute, with an extensive library, for the use of
the mechanics, has been endowed.  The agents have stood forward in
the support of schools, churches, lectures, and lyceums, and their
influence contributed highly to the elevation of the moral and
intellectual character of the operatives.  Talent has been
encouraged, brought forward, and recommended.''  For some
considerable time the young women wrote, edited, and published a
newspaper among themselves, called the Lowell Offering.  ``And
Lowell has supplied agents and mechanics for the later
manufacturing places who have given tone to society, and extended
the beneficial influence of Lowell through the United States.
Girls from the country, with a true Yankee spirit of independence,
and confident in their own powers, pass a few years here, and then
return to get married with a dower secured by their exertions, with
more enlarged ideas and extended means of information, and their
places are supplied by younger relatives.  A large proportion of
the female population of New England has been employed at some time
in manufacturing establishments, and they are not on this account
less good wives, mothers, or educators of families.''  Then the
account goes on to tell how the health of the girls has been
improved by their attendance at the mills; how they put money into
the savings banks, and buy railway shares and farms; how there are
thirty churches in Lowell, a library, banks, and insurance office,;
how there is a cemetery, and a park; and how everything is
beautiful, philanthropic, profitable, and magnificent.

Thus Lowell is the realization of a commercial Utopia.  Of all the
statements made in the little book which I have quoted, I cannot
point out one which is exaggerated, much less false.  I should not
call the place elegant; in other respects I am disposed to stand by
the book.  Before I had made any inquiry into the cause of the
apparent comfort, it struck me at once that some great effort at
excellence was being made.  I went into one of the discreet
matrons' residences; and, perhaps, may give but an indifferent idea
of her discretion, when I say that she allowed me to go into the
bed-rooms.  If you want to ascertain the inner ways or habits of
life of any man, woman, or child, see, if it be practicable to do
so, his or her bed-room.  You will learn more by a minute's glance
round that holy of holies, than by any conversation.  Looking-
glasses and such like, suspended dresses, and toilet-belongings, if
taken without notice, cannot lie or even exaggerate.  The discreet
matron at first showed me rooms only prepared for use, for at the
period of my visit Lowell was by no means full; but she soon became
more intimate with me, and I went through the upper part of the
house.  My report must be altogether in her favor and in that of
Lowell.  Everything was cleanly, well ordered, and feminine.  There
was not a bed on which any woman need have hesitated to lay herself
if occasion required it.  I fear that this cannot be said of the
lodgings of the manufacturing classes at Manchester.  The boarders
all take their meals together.  As a rule, they have meat twice a
day.  Hot meat for dinner is with them as much a matter of course,
or probably more so, than with any Englishman or woman who may read
this book.  For in the States of America regulations on this matter
are much more rigid than with us.  Cold meat is rarely seen, and to
live a day without meat would be as great a privation as to pass a
night without bed.

The rules for the guidance of these boarding-houses are very rigid.
The houses themselves belong to the corporations, or different
manufacturing establishments, and the tenants are altogether in the
power of the managers.  None but operatives are to be taken in.
The tenants are answerable for improper conduct.  The doors are to
be closed at ten o'clock.  Any boarders who do not attend divine
worship are to be reported to the managers.  The yards and walks
are to be kept clean, and snow removed at once; and the inmates
must be vaccinated, etc. etc. etc.  It is expressly stated by the
Hamilton Company---and I believe by all the companies---that no one
shall be employed who is habitually absent from public worship on
Sunday, or who is known to be guilty of immorality, it is stated
that the average wages of the women are two dollars, or eight
shillings, a week, besides their board.  I found when I was there
that from three dollars to three and a half a week were paid to the
women, of which they paid one dollar and twenty-five cents for
their board.  As this would not fully cover the expense of their
keep, twenty-five cents a week for each was also paid to the
boarding-house keepers by the mill agents.  This substantially came
to the same thing, as it left the two dollars a week, or eight
shillings, with the girls over and above their cost of living.  The
board included washing, lights, food, bed, and attendance---leaving
a surplus of eight shillings a week for clothes and saving.  Now
let me ask any one acquainted with Manchester and its operatives,
whether that is not Utopia realized.  Factory girls, for whom every
comfort of life is secured, with 21l. a year over for saving and
dress!  One sees the failing, however, at a moment.  It is Utopia.
Any Lady Bountiful can tutor three or four peasants and make them
luxuriously comfortable.  But no Lady Bountiful can give luxurious
comfort to half a dozen parishes.  Lowell is now nearly forty years
old, and contains but 40,000 inhabitants.  From the very nature of
its corporations it cannot spread itself.  Chicago, which has grown
out of nothing in a much shorter period, and which has no
factories, has now 120,000 inhabitants.  Lowell is a very wonderful
place and shows what philanthropy can do; but I fear it also shows
what philanthropy cannot do.

There are, however, other establishments, conducted on the same
principle as those at Lowell, which have had the same amount, or
rather the same sort of success.  Lawrence is now a town of about
15,000 inhabitants, and Manchester of about 24,000, if I remember
rightly; and at those places the mills are also owned by
corporations and conducted as are those at Lowell.  But it seems to
me that as New England takes her place in the world as a great
mannfacturing country---which place she undoubtedly will take sooner
or later---she must abandon the hot-house method of providing for
her operatives with which she has commenced her work.  In the first
place, Lowell is not open as a manufacturing town to the
capitalists even of New England at large.  Stock may, I presume, be
bought in the corporations, but no interloper can establish a mill
there.  It is a close manufacturing community, bolstered up on all
sides, and has none of that capacity for providing employment for a
thickly growing population which belongs to such places as
Manchester and Leeds.  That it should under its present system have
been made in any degree profitable reflects great credit on the
managers; but the profit does reach an amount which in America can
be considered as remunerative.  The total capital invested by the
twelve corporations is thirteen million and a half of dollars, or
about two million seven hundred thousand pounds.  In only one of
the corporations, that of the Merrimack Company, does the profit
amount to twelve per cent.  In one, that of the Booth Company, it
falls below seven per cent.  The average profit of the various
establishments is something below nine per cent.  I am of course
speaking of Lowell as it was previous to the war.  American
capitalists are not, as a rule, contented with so low a rate of
interest as this.

The States in these matters have had a great advantage over
England.  They have been able to begin at the beginning.
Manufactories have grown up among us as our cities grew---from the
necessities and chances of the times.  When labor was wanted it was
obtained in the ordinary way; and so when houses were built they
were built in the ordinary way.  We had not the experience, and the
results either for good or bad, of other nations to guide us.  The
Americans, in seeing and resolving to adopt our commercial
successes, have resolved also, if possible, to avoid the evils
which have attended those successes.  It would be very desirable
that all our factory girls should read and write, wear clean
clothes, have decent beds, and eat hot meat every day.  But that is
now impossible.  Gradually, with very up-hill work, but still I
trust with sure work, much will be done to improve their position
and render their life respectable; but in England we can have no
Lowells.  In our thickly populated island any commercial Utopia is
out of the question.  Nor can, as I think, Lowell be taken as a
type of the future manufacturing towns of New England.  When New
England employs millions in her factories instead of thousands---the
hands employed at Lowell, when the mills are at full work, are
about 11,000---she must cease to provide for them their beds and
meals, their church-going proprieties and orderly modes of life.
In such an attempt she has all the experience of the world against
her.  But nevertheless I think she will have done much good.  The
tone which she will have given will not altogether lose its
influence.  Employment in a factory is now considered reputable by
a farmer and his children, and this idea will remain.  Factory work
is regarded as more respectable than domestic service, and this
prestige will not wear itself altogether out.  Those now employed
have a strong conception of the dignity of their own social
position, and their successors will inherit much of this, even
though they may find themselves excluded from the advantages of the
present Utopia.  The thing has begun well, but it can only be
regarded as a beginning.  Steam, it may be presumed, will become
the motive power of cotton mills in New England as it is with us;
and when it is so, the amount of work to be done at any one place
will not be checked by any such limit as that which now prevails at
Lowell.  Water-power is very cheap, but it cannot be extended; and
it would seem that no place can become large as a manufacturing
town which has to depend chiefly upon water.  It is not improbable
that steam may be brought into general use at Lowell, and that
Lowell may spread itself.  If it should spread itself widely, it
will lose its Utopian characteristics.

One cannot but be greatly struck by the spirit of philanthropy in
which the system of Lowell was at first instituted.  It may be
presumed that men who put their money into such an undertaking did
so with the object of commercial profit to themselves; but in this
case that was not their first object.  I think it may be taken for
granted that when Messrs. Jackson and Lowell went about their task,
their grand idea was to place factory work upon a respectable
footing---to give employment in mills which should not be unhealthy,
degrading, demoralizing, or hard in its circumstances.  Throughout
the Northern States of America the same feeling is to be seen.
Good and thoughtful men have been active to spread education, to
maintain health, to make work compatible with comfort and personal
dignity, and to divest the ordinary lot of man of the sting of that
curse which was supposed to be uttered when our first father was
ordered to eat his bread in the sweat of his brow.  One is driven
to contrast this feeling, of which on all sides one sees such ample
testimony, with that sharp desire for profit, that anxiety to do a
stroke of trade at every turn, that acknowledged necessity of being
smart, which we must own is quite as general as the nobler
propensity.  I believe that both phases of commercial activity may
be attributed to the same characteristic.  Men in trade in America
are not more covetous than tradesmen in England, nor probably are
they more generous or philanthropical.  But that which they do,
they are more anxious to do thoroughly and quickly.  They desire
that every turn taken shall be a great turn---or at any rate that it
shall be as great as possible.  They go ahead either for bad or
good with all the energy they have.  In the institutions at Lowell
I think we may allow that the good has very much prevailed.

I went over two of the mills, those of the Merrimack corporation
and of the Massachusetts.  At the former the printing establishment
only was at work; the cotton mills were closed.  I hardly know
whether it will interest any one to learn that something under half
a million yards of calico are here printed annually.  At the Lowell
Bleachery fifteen million yards are dyed annually.  The Merrimack
Cotton Mills were stopped, and so had the other mills at Lowell
been stopped, till some short time before my visit.  Trade had been
bad, and there had of course been a lack of cotton.  I was assured
that no severe suffering had been created by this stoppage.  The
greater number of hands had returned into the country---to the farms
from whence they had come; and though a discontinuance of work and
wages had of course produced hardship, there had been no actual
privation---no hunger and want.  Those of the work-people who had no
homes out of Lowell to which to betake themselves, and no means at
Lowell of living, had received relief before real suffering had
begun.  I was assured, with something of a smile of contempt at the
question, that there had been nothing like hunger.  But, as I said
before, visitors always see a great deal of rose color, and should
endeavor to allay the brilliancy of the tint with the proper amount
of human shading.  But do not let any visitor mix in the browns
with too heavy a hand!

At the Massachusetts Cotton Mills they were working with about two-
thirds of their full number of hands, and this, I was told, was
about the average of the number now employed throughout Lowell.
Working at this rate they had now on hand a supply of cotton to
last them for six months.  Their stocks had been increased lately,
and on asking from whence, I was informed that that last received
had come to them from Liverpool.  There is, I believe, no doubt but
that a considerable quantity of cotton has been shipped back from
England to the States since the civil war began.  I asked the
gentleman, to whose care at Lowell I was consigned, whether he
expected to get cotton from the South---for at that time Beaufort,
in South Carolina, had just been taken by the naval expedition.  He
had, he said, a political expectation of a supply of cotton, but
not a commercial expectation.  That at least was the gist of his
reply, and I found it to be both intelligent and intelligible.  The
Massachusetts Mills, when at full work, employ 1300 females and 400
males, and turn out 540,000 yards of calico per week.

On my return from Lowell in the smoking car, an old man came and
squeezed in next to me.  The place was terribly crowded, and as the
old man was thin and clean and quiet, I willingly made room for
him, so as to avoid the contiguity of a neighbor who might be
neither thin, nor clean, nor quiet.  He began talking to me in
whispers about the war, and I was suspicious that he was a
Southerner and a secessionist.  Under such circumstances his
company might not be agreeable, unless he could be induced to hold
his tongue.  At last he said, ``I come from Canada, you know, and
you---you're an Englishman, and therefore I can speak to you
openly;'' and he gave me an affectionate grip on the knee with his
old skinny hand.  I suppose I do look more like an Englishman than
an American, but I was surprised at his knowing me with such
certainty.  ``There is no mistaking you,'' he said, ``with your round
face and your red cheeks.  They don't look like that here,'' and he
gave me another grip.  I felt quite fond of the old man, and
offered him a cigar.



\chapter{The Rights of Women}


We all know that the subject which appears above as the title of
this chapter is a very favorite subject in America.  It is, I hope,
a very favorite subject here also, and I am inclined to think has
been so for many years past.  The rights of women, as
contradistinguished from the wrongs of women, has perhaps been the
most precious of the legacies left to us by the feudal ages.  How,
amid the rough darkness of old Teuton rule, women began to receive
that respect which is now their dearest right, is one of the most
interesting studies of history.  It came, I take it, chiefly from
their own conduct.  The women of the old classic races seem to have
enjoyed but a small amount of respect or of rights, and to have
deserved as little.  It may have been very well for one Caesar to
have said that his wife should be above suspicion; but his wife was
put away, and therefore either did not have her rights, or else had
justly forfeited them.  The daughter of the next Caesar lived in
Rome the life of a Messalina, and did not on that account seem to
have lost her ``position in society,'' till she absolutely declined
to throw any vail whatever over her propensities.  But as the Roman
empire fell, chivalry began.  For a time even chivalry afforded but
a dull time to the women.  During the musical period of the
Troubadours, ladies, I fancy, had but little to amuse them save the
music.  But that was the beginning, and from that time downward the
rights of women have progressed very favorably.  It may be that
they have not yet all that should belong to them.  If that be the
case, let the men lose no time in making up the difference.  But it
seems to me that the women who are now making their claims may
perhaps hardly know when they are well off.  It will be an ill
movement if they insist on throwing away any of the advantages they
have won.  As for the women in America especially, I must confess
that I think they have a ``good time.''  I make them my compliments
on their sagacity, intelligence, and attractions, but I utterly
refuse to them any sympathy for supposed wrongs.  O fortunatas, sua
si bona norint!  Whether or no, were I an American married man and
father of a family, I should not go in for the rights of man---that
is altogether another question.

This question of the rights of women divides itself into two heads---%
one of which is very important, worthy of much consideration,
capable perhaps of much philanthropic action, and at any rate
affording matter for grave discussion.  This is the question of
women's work: How far the work of the world, which is now borne
chiefly by men, should be thrown open to women further than is now
done?  The other seems to me to be worthy of no consideration, to
be capable of no action, to admit of no grave discussion.  This
refers to the political rights of women: How far the political
working of the world, which is now entirely in the hands of men,
should be divided between them and women?  The first question is
being debated on our side of the Atlantic as keenly perhaps as on
the American side.  As to that other question, I do not know that
much has ever been said about it in Europe.

``You are doing nothing in England toward the employment of
females,'' a lady said to me in one of the States soon after my
arrival in America.  ``Pardon me,'' I answered, ``I think we are doing
much, perhaps too much.  At any rate we are doing something.''  I
then explained to her how Miss Faithful had instituted a printing
establishment in London; how all the work in that concern was done
by females, except such heavy tasks as those for which women could
not be fitted, and I handed to her one of Miss Faithful's cards.
``Ah,'' said my American friend, ``poor creatures!  I have no doubt
their very flesh will be worked off their bones.''  I thought this a
little unjust on her part; but nevertheless it occurred to me as an
answer not unfit to be made by some other lady---by some woman who
had not already advocated the increased employment of women.  Let
Miss Faithful look to that.  Not that she will work the flesh off
her young women's bones, or allow such terrible consequences to
take place in Coram Street; not that she or that those connected
with her in that enterprise will do aught but good to those
employed therein.  It will not even be said of her individually, or
of her partners, that they have worked the flesh off women's bones;
but may it not come to this, that when the tasks now done by men
have been shifted to the shoulders of women, women themselves will
so complain?  May it not go further, and come even to this, that
women will have cause for such complaint?  I do not think that such
a result will come, because I do not think that the object desired
by those who are active in the matter will be attained.  Men, as a
general rule among civilized nations, have elected to earn their
own bread and the bread of the women also, and from this resolve on
their part I do not think that they will be beaten off.

We know that Mrs.\ Dall, an American lady, has taken up this
subject, and has written a book on it, in which great good sense
and honesty of purpose is shown.  Mrs.\ Dall is a strong advocate
for the increased employment of women, and I, with great deference,
disagree with her.  I allude to her book now because she has
pointed out, I think very strongly, the great reason why women do
not engage themselves advantageously in trade pursuits.  She by no
means overpraises her own sex, and openly declares that young women
will not consent to place themselves in fair competition with men.
They will not undergo the labor and servitude of long study at
their trades.  They will not give themselves up to an
apprenticeship.  They will not enter upon their tasks as though
they were to be the tasks of their lives.  They may have the same
physical and mental aptitudes for learning a trade as men, but they
have not the same devotion to the pursuit, and will not bind
themselves to it thoroughly as men do.  In all which I quite agree
with Mrs.\ Dall; and the English of it is---that the young women want
to get married.

God forbid that they should not so want.  Indeed, God has forbidden
in a very express way that there should be any lack of such a
desire on the part of women.  There has of late years arisen a
feeling among masses of the best of our English ladies that this
feminine propensity should be checked.  We are told that unmarried
women may be respectable, which we always knew; that they may be
useful, which we also acknowledge---thinking still that, if married,
they would be more useful; and that they may be happy, which we
trust---feeling confident, however, that they might in another
position be more happy.  But the question is not only as to the
respectability, usefulness, and happiness of womankind, but as to
that of men also.  If women can do without marriage, can men do so?
And if not, how are the men to get wives, if the women elect to
remain single?

It will be thought that I am treating the subject as though it were
simply jocose, but I beg to assure my reader that such is not my
intention.  It certainly is the fact that that disinclination to an
apprenticeship and unwillingness to bear the long training for a
trade, of which Mrs.\ Dall complains on the part of young women,
arise from the fact that they have other hopes with which such
apprenticeships would jar; and it is also certain that if such
disinclination be overcome on the part of any great number, it must
be overcome by the destruction or banishment of such hopes.  The
question is whether good or evil would result from such a change.
It is often said that whatever difficulty a woman may have in
getting a husband, no man need encounter difficulty in finding a
wife.  But, in spite of this seeming fact, I think it must be
allowed that if women are withdrawn from the marriage market, men
must be withdrawn from it also to the same extent.

In any broad view of this matter, we are bound to look not on any
individual case, and the possible remedies for such cases, but on
the position in the world occupied by women in general---on the
general happiness and welfare of the aggregate feminine world, and
perhaps also a little on the general happiness and welfare of the
aggregate male world.  When ladies and gentlemen advocate the right
of women to employment, they are taking very different ground from
that on which stand those less extensive philanthropists who exert
themselves for the benefit of distressed needlewomen, for instance,
or for the alleviation of the more bitter misery of governesses.
The two questions are in fact absolutely antagonistic to each
other.  The rights-of-women advocate is doing his best to create
that position for women from the possible misfortunes of which the
friend of the needlewomen is struggling to relieve them.  The one
is endeavoring to throw work from off the shoulders of men on to
the shoulders of women, and the other is striving to lessen the
burden which women are already bearing.  Of course it is good to
relieve distress in individual cases.  That Song of the Shirt,
which I regard as poetry of the immortal kind, has done an amount
of good infinitely wider than poor Hood ever ventured to hope.  Of
all such efforts I would speak not only with respect, but with
loving admiration.  But of those whose efforts are made to spread
work more widely among women---to call upon them to make for us our
watches, to print our books, to sit at our desks as clerks and to
add up our accounts---much as I may respect the individual operators
in such a movement, I can express no admiration for their judgment.

I have seen women with ropes round their necks drawing a harrow
over plowed ground.  No one will, I suppose, say that they approve
of that.  But it would not have shocked me to see men drawing a
harrow.  I should have thought it slow, unprofitable work; but my
feelings would not have been hurt.  There must, therefore, be some
limit; but if we men teach ourselves to believe that work is good
for women, where is the limit to be drawn, and who shall draw it?
It is true that there is now no actually defined limit.  There is
much work that is commonly open to both sexes.  Personal domestic
attendance is so, and the attendance in shops.  The use of the
needle is shared between men and women; and few, I take it, know
where the seamstress ends and where the tailor begins.  In many
trades a woman can be, and very often is, the owner and manager of
the business.  Painting is as much open to women as to men, as also
is literature.  There can be no defined limit; but nevertheless
there is at present a quasi limit, which the rights-of-women
advocates wish to move, and so to move that women shall do more
work and not less.  A woman now could not well be a cab-driver in
London; but are these advocates sure that no woman will be a cab-
driver when success has attended their efforts?  And would they
like to see a woman driving a cab?  For my part, I confess I do not
like to see a woman acting as road-keeper on a French railway.  I
have seen a woman acting as hostler at a public stage in Ireland.
I knew the circumstances---how her husband had become ill and
incapable, and how she had been allowed to earn the wages; but
nevertheless the sight was to me disagreeable, and seemed, as far
as it went, to degrade the sex.  Chivalry has been very active in
raising women from the hard and hardening tasks of the world; and
through this action they have become soft, tender, and virtuous.
It seems to me that they of whom I am now speaking are desirous of
undoing what chivalry has done.

The argument used is of course plain enough.  It is said that women
are left destitute in the world---destitute unless they can be self-
dependent, and that to women should be given the same open access
to wages that men possess, in order that they may be as self-
dependent as men.  Why should a young woman, for whom no father is
able to provide, not enjoy those means of provision which are open
to a young man so circumstanced?  But I think the answer is very
simple.  The young man, under the happiest circumstances which may
befall him, is bound to earn his bread.  The young woman is only so
bound when happy circumstances do not befall her.  Should we
endeavor to make the recurrence of unhappy circumstances more
general or less so?  What does any tradesman, any professional man,
any mechanic wish for his children?  Is it not this, that his sons
shall go forth and earn their bread, and that his daughters shall
remain with him till they are married?  Is not that the mother's
wish?  Is it not notorious that such is the wish of us all as to
our daughters?  In advocating the rights of women it is of other
men's girls that we think, never of our own.

But, nevertheless, what shall we do for those women who must earn
their bread by their own work?  Whatever we do, do not let us
willfully increase their number.  By opening trades to women, by
making them printers, watchmakers, accountants, or what not, we
shall not simply relieve those who must now earn their bread by
some such work or else starve.  It will not be within our power to
stop ourselves exactly at a certain point; to arrange that those
women who under existing circumstances may now be in want shall be
thus placed beyond want, but that no others shall be affected.
Men, I fear, will be too willing to relieve themselves of some
portion of their present burden, should the world's altered ways
enable them to do so.  At present a lawyer's clerk may earn perhaps
his two guineas a week, and he with his wife live on that in fair
comfort.  But if his wife, as well as he, has been brought up as a
lawyer's clerk, he will look to her also for some amount of wages.
I doubt whether the two guineas would be much increased, but I do
not doubt at all that the woman's position would be injured.

It seems to me that in discussing this subject philanthropists fail
to take hold of the right end of the argument.  Money returns from
work are very good, and work itself is good, as bringing such
returns and occupying both body and mind; but the world's work is
very hard, and workmen are too often overdriven.  The question
seems to me to be this---of all this work have the men got on their
own backs too heavy a share for them to bear, and should they seek
relief by throwing more of it upon women?  It is the rights of man
that we are in fact debating.  These watches are weary to make, and
this type is troublesome to set, We have battles to fight and
speeches to make, and our hands altogether are too full.  The women
are idle---many of them.  They shall make the watches for us and set
the type; and when they have done that, why should they not make
nails as they do sometimes in Worcestershire, or clean horses, or
drive the cabs?  They have had an easy time of it for these years
past, but we'll change that.  And then it would come to pass that
with ropes round their necks the women would be drawing harrows
across the fields.

I don't think this will come to pass.  The women generally do know
when they are well off, and are not particularly anxious to accept
the philanthropy proffered to them---as Mrs.\ Dall says, they do not
wish to bind themselves as apprentices to independent money-making.
This cry has been louder in America than with us, but even in
America it has not been efficacious for much.  There is in the
States, no doubt, a sort of hankering after increased influence, a
desire for that prominence of position which men attain by loud
voices and brazen foreheads, a desire in the female heart to be up
and doing something, if the female heart only knew what; but even
in the States it has hardly advanced beyond a few feminine
lectures.  In many branches of work women are less employed than in
England.  They are not so frequent behind counters in the shops,
and are rarely seen as servants in hotels.  The fires in such
houses are lighted and the rooms swept by men.  But the American
girls may say they do not desire to light fires and sweep rooms.
They are ambitious of the higher classes of work.  But those higher
branches of work require study, apprenticeship, a devotion of
youth; and that they will not give.  It is very well for a young
man to bind himself for four years, and to think of marrying four
years after that apprenticeship be over.  But such a prospectus
will not do for a girl.  While the sun shines the hay must be made,
and her sun shines earlier in the day than that of him who is to be
her husband.  Let him go through the apprenticeship and the work,
and she will have sufficient on her hands if she looks well after
his household.  Under nature's teaching she is aware of this, and
will not bind herself to any other apprenticeship, let Mrs.\ Dall
preach as she may.

I remember seeing, either at New York or Boston, a wooden figure of
a neat young woman, as large as life, standing at a desk with a
ledger before her, and looking as though the beau ideal of human
bliss were realized in her employment.  Under the figure there was
some notice respecting female accountants.  Nothing could be nicer
than the lady's figure, more flowing than the broad lines of her
drapery, or more attractive than her auburn ringlets.  There she
stood at work, earning her bread without any impediment to the
natural operation of her female charms, and adjusting the accounts
of some great firm with as much facility as grace.  I wonder
whether he who designed that figure had ever sat or stood at a desk
for six hours; whether he knew the dull hum of the brain which
comes from long attention to another man's figures; whether he had
ever soiled his own fingers with the everlasting work of office
hours, or worn his sleeves threadbare as he leaned, weary in body
and mind, upon his desk?  Work is a grand thing---the grandest thing
we have; but work is not picturesque, graceful, and in itself
alluring.  It sucks the sap out of men's bones, and bends their
backs, and sometimes breaks their hearts; but though it be so, I
for one would not wish to throw any heavier share of it on to a
woman's shoulders.  It was pretty to see those young women with
spectacles at the Boston library; but when I heard that they were
there from eight in the morning till nine at night, I pitied them
their loss of all the softness of home, and felt that they would
not willingly be there, if necessity were less stern.

Say that by advocating the rights of women, philanthropists succeed
in apportioning more work to their share, will they eat more, wear
better clothes, lie softer, and have altogether more of the fruits
of work than they do now?  That some would do so there can be no
doubt; but as little that some would have less.  If on the whole
they would not have more, for what good result is the movement
made?  The first question is, whether at the present time they have
less than their proper share.  There are, unquestionably, terrible
cases of female want; and so there are also of want among men.
Alas! do we not all feel that it must be so, let the
philanthropists be ever so energetic?  And if a woman be left
destitute, without the assistance of father, brother, or husband,
it would be hard if no means of earning subsistence were open to
her.  But the object now sought is not that of relieving such
distress.  It has a much wider tendency, or at any rate a wider
desire.  The idea is that women will ennoble themselves by making
themselves independent, by working for their own bread instead of
eating bread earned by men.  It is in that that these new
philosophers seem to me to err so greatly.  Humanity and chivalry
have succeeded, after a long struggle, in teaching the man to work
for the woman; and now the woman rebels against such teaching---not
because she likes the work, but because she desires the influence
which attends it.  But in this I wrong the woman---even the American
woman.  It is not she who desires it, but her philanthropical
philosophical friends who desire it for her.

If work were more equally divided between the sexes, some women
would, of course, receive more of the good things of the world.
But women generally would not do so.  The tendency, then, would be
to force young women out upon their own exertions.  Fathers would
soon learn to think that their daughters should be no more
dependent on them than their sons; men would expect their wives to
work at their own trades; brothers would be taught to think it hard
that their sisters should lean on them, and thus women, driven upon
their own resources, would hardly fare better than they do at
present.

After all it is a question of money, and a contest for that power
and influence which money gives.  At present, men have the position
of the Lower House of Parliament---they have to do the harder work,
but they hold the purse.  Even in England there has grown up a
feeling that the old law of the land gives a married man too much
power over the joint pecuniary resources of him and his wife, and
in America this feeling is much stronger, and the old law has been
modified.  Why should a married woman be able to possess nothing?
And if such be the law of the land, is it worth a woman's while to
marry and put herself in such a position?  Those are the questions
asked by the friends of the rights of women.  But the young women
do marry, and the men pour their earnings into their wives' laps.

If little has as yet been done in extending the rights of women by
giving them a greater share of the work of the world, still less
has been done toward giving them their portion of political
influence.  In the States there are many men of mark, and women of
mark also, who think that women should have votes for public
elections.  Mr.\ Wendell Phillips, the Boston lecturer who advocates
abolition, is an apostle in this cause also; and while I was at
Boston I read the provisions of a will lately left by a
millionaire, in which he bequeathed some very large sums of money
to be expended in agitation on this subject.  A woman is subject to
the law; why then should she not help to make the law?  A child is
subject to the law, and does not help to make it; but the child
lacks that discretion which the woman enjoys equally with the man.
That I take it is the amount of the argument in favor of the
political rights of women.  The logic of this is so conclusive that
I am prepared to acknowledge that it admits of no answer.  I will
only say that the mutual good relations between men and women,
which are so indispensable to our happiness, require that men and
women should not take to voting at the same time and on the same
result.  If it be decided that women shall have political power,
let them have it all to themselves for a season.  If that be so
resolved, I think we may safely leave it to them to name the time
at which they will begin.

I confess that in the States I have sometimes been driven to think
that chivalry has been carried too far---that there is an attempt to
make women think more of the rights of their womanhood than is
needful.  There are ladies' doors at hotels, and ladies' drawing-
rooms, ladies' sides on the ferry-boats, ladies' windows at the
post-office for the delivery of letters---which, by-the-by, is an
atrocious institution, as anybody may learn who will look at the
advertisements called personal in some of the New York papers.  Why
should not young ladies have their letters sent to their houses,
instead of getting them at a private window?  The post-office
clerks can tell stories about those ladies' windows.  But at every
turn it is necessary to make separate provision for ladies.  From
all this it comes to pass that the baker's daughter looks down from
a great height on her papa, and by no means thinks her brother good
enough for her associate.  Nature, the great restorer, comes in and
teaches her to fall in love with the butcher's son.  Thus the evil
is mitigated; but I cannot but wish that the young woman should not
see herself denominated a lady so often, and should receive fewer
lessons as to the extent of her privileges.  I would save her, if I
could, from working at the oven; I would give to her bread and meat
earned by her father's care and her brother's sweat; but when she
has received these good things, I would have her proud of the one
and by no means ashamed of the other.

Let women say what they will of their rights, or men who think
themselves generous say what they will for them, the question has
all been settled both for them and for us men by a higher power.
They are the nursing mothers of mankind, and in that law their fate
is written with all its joys and all its privileges.  It is for men
to make those joys as lasting and those privileges as perfect as
may be.  That women should have their rights no man will deny.  To
my thinking, neither increase of work nor increase of political
influence are among them.  The best right a woman has is the right
to a husband, and that is the right to which I would recommend
every young woman here and in the States to turn her best
attention.  On the whole, I think that my doctrine will be more
acceptable than that of Mrs.\ Dall or Mr.\ Wendell Phillips.



\chapter{Education}


The one matter in which, as far as my judgment goes, the people of
the United States have excelled us Englishmen, so as to justify
them in taking to themselves praise which we cannot take to
ourselves or refuse to them, is the matter of Education.  In saying
this, I do not think that I am proclaiming anything disgraceful to
England, though I am proclaiming much that is creditable to
America.  To the Americans of the States was given the good fortune
of beginning at the beginning.  The French at the time of their
revolution endeavored to reorganize everything, and to begin the
world again with new habits and grand theories; but the French as a
people were too old for such a change, and the theories fell to the
ground.  But in the States, after their revolution, an Anglo-Saxon
people had an opportunity of making a new State, with all the
experience of the world before them; and to this matter of
education they were from the first aware that they must look for
their success.  They did so; and unrivaled population, wealth, and
intelligence has been the result; and with these, looking at the
whole masses of the people---I think I am justified in saying---%
unrivaled comfort and happiness.  It is not that you, my reader, to
whom in this matter of education fortune and your parents have
probably been bountiful, would have been more happy in New York
than in London.  It is not that I, who, at any rate, can read and
write, have cause to wish that I had been an American.  But it is
this: if you and I can count up in a day all those on whom our eyes
may rest and learn the circumstances of their lives, we shall be
driven to conclude that nine-tenths of that number would have had a
better life as Americans than they can have in their spheres as
Englishmen.  The States are at a discount with us now, in the
beginning of this year of grace 1862; and Englishmen were not very
willing to admit the above statement, even when the States were not
at a discount.  But I do not think that a man can travel through
the States with his eyes open and not admit the fact.  Many things
will conspire to induce him to shut his eyes and admit no
conclusion favorable to the Americans.  Men and women will
sometimes be impudent to him; the better his coat, the greater the
impudence.  He will be pelted with the braggadocio of equality.
The corns of his Old World conservatism will be trampled on hourly
by the purposely vicious herd of uncouth democracy.  The fact that
he is paymaster will go for nothing, and will fail to insure
civility.  I shall never forget my agony as I saw and heard my desk
fall from a porter's hand on a railway station, as he tossed it
from him seven yards off on to the hard pavement.  I heard its
poor, weak intestines rattle in their death struggle, and knowing
that it was smashed, I forgot my position on American soil and
remonstrated.  ``It's my desk, and you have utterly destroyed it,'' I
said.  ``Ha! ha! ha!'' laughed the porter.  ``You've destroyed my
property,'' I rejoined, ``and it's no laughing matter.''  And then all
the crowd laughed.  ``Guess you'd better get it glued,'' said one.
So I gathered up the broken article and retired mournfully and
crestfallen into a coach.  This was very sad, and for the moment I
deplored the ill luck which had brought me to so savage a country.
Such and such like are the incidents which make an Englishman in
the States unhappy, and rouse his gall against the institutions of
the country; these things and the continued appliance of the
irritating ointment of American braggadocio with which his sores
are kept open.  But though I was badly off on that railway
platform, worse off than I should have been in England, all that
crowd of porters round me were better off than our English porters.
They had a ``good time'' of it.  And this, O my English brother who
has traveled through the States and returned disgusted, is the fact
throughout.  Those men whose familiarity was so disgusting to you
are having a good time of it.  ``They might be a little more civil,''
you say, ``and yet read and write just as well.''  True; but they are
arguing in their minds that civility to you will be taken by you
for subservience, or for an acknowledgment of superiority; and
looking at your habits of life---yours and mine together---I am not
quite sure that they are altogether wrong.  Have you ever realized
to yourself as a fact that the porter who carries your box has not
made himself inferior to you by the very act of carrying that box?
If not, that is the very lesson which the man wishes to teach you.

If a man can forget his own miseries in his journeyings, and think
of the people he comes to see rather than of himself, I think he
will find himself driven to admit that education has made life for
the million in the Northern States better than life for the million
is with us.  They have begun at the beginning, and have so managed
that every one may learn to read and write---have so managed that
almost every one does learn to read and write.  With us this cannot
now be done.  Population had come upon us in masses too thick for
management, before we had as yet acknowledged that it would be a
good thing that these masses should be educated.  Prejudices, too,
had sprung up, and habits, and strong sectional feelings, all
antagonistic to a great national system of education.  We are, I
suppose, now doing all that we can do; but comparatively it is
little.  I think I saw some time since that the cost for gratuitous
education, or education in part gratuitous, which had fallen upon
the nation had already amounted to the sum of 800,000l.; and I
think also that I read in the document which revealed to me this
fact a very strong opinion that government could not at present go
much further.  But if this matter were regarded in England as it is
regarded in Massachusetts, or rather, had it from some prosperous
beginning been put upon a similar footing, 800,000l. would not have
been esteemed a great expenditure for free education simply in the
City of London.  In 1857 the public schools of Boston cost
70,000l., and these schools were devoted to a population of about
180,000 souls.  Taking the population of London at two and a half
millions, the whole sum now devoted to England would, if expended
in the metropolis, make education there even cheaper than it is in
Boston.  In Boston, during 1857, there were above 24,000 pupils at
these public schools, giving more than one-eighth of the whole
population.  But I fear it would not be practicable for us to spend
800,000l. on the gratuitous education of London.  Rich as we are,
we should not know where to raise the money.  In Boston it is
raised by a separate tax.  It is a thing understood, acknowledged,
and made easy by being habitual---as is our national debt.  I do not
know that Boston is peculiarly blessed, but I quote the instance,
as I have a record of its schools before me.  At the three high
schools in Boston, at which the average of pupils is 526, about
13l. per head is paid for free education.  The average price per
annum of a child's schooling throughout these schools in Boston is
about 3l. for each.  To the higher schools any boy or girl may
attain without any expense, and the education is probably as good
as can be given, and as far advanced.  The only question is,
whether it is not advanced further than may be necessary.  Here, as
at New York, I was almost startled by the amount of knowledge
around me, and listened, as I might have done to an examination in
theology among young Brahmins.  When a young lad explained in my
hearing all the properties of the different levers as exemplified
by the bones of the human body, I bowed my head before him in
unaffected humility.  We, at our English schools, never got beyond
the use of those bones which he described with such accurate
scientific knowledge.  In one of the girls' schools they were
reading Milton, and when we entered were discussing the nature of
the pool in which the devil is described as wallowing.  The
question had been raised by one of the girls.  A pool, so called,
was supposed to contain but a small amount of water, and how could
the devil, being so large, get into it?  Then came the origin of
the word pool---from ``palus,'' a marsh, as we were told, some
dictionary attesting to the fact, and such a marsh might cover a
large expanse.  The ``Palus Maeotis'' was then quoted.  And so we
went on till Satan's theory of political liberty,

\begin{verse}
     ``Better to reign in hell than serve in heaven,''
\end{verse}

was thoroughly discussed and understood.  These girls of sixteen
and seventeen got up one after another and gave their opinions on
the subject---how far the devil was right, and how far he was
manifestly wrong.  I was attended by one of the directors or
guardians of the schools; and the teacher, I thought, was a little
embarrassed by her position.  But the girls themselves were as easy
in their demeanor as though they were stitching handkerchiefs at
home.

It is impossible to refrain from telling all this, and from making
a little innocent fun out of the superexcellencies of these
schools; but the total result on my mind was very greatly in their
favor.  And indeed the testimony came in both ways.  Not only was I
called on to form an opinion of what the men and women would become
from the education which was given to the boys and girls, but also
to say what must have been the education of the boys and girls from
what I saw of the men and women.  Of course it will be understood
that I am not here speaking of those I met in society or of their
children, but of the working people---of that class who find that a
gratuitous education for their children is needful, if any
considerable amount of education is to be given.  The result is to
be seen daily in the whole intercourse of life.  The coachman who
drives you, the man who mends your window, the boy who brings home
your purchases, the girl who stitches your wife's dress,---they all
carry with them sure signs of education, and show it in every word
they utter.

It will of course be understood that this is, in the separate
States, a matter of State law; indeed, I may go further, and say
that it is, in most of the States, a matter of State constitution.
It is by no means a matter of Federal constitution.  The United
States as a nation takes no heed of the education of its people.
All that is left to the judgment of the separate States.  In most
of the thirteen original States provision is made in the written
constitution for the general education of the people; but this is
not done in all.  I find that it was more frequently done in the
Northern or free-soil States than in those which admitted slavery,
as might have been expected.  In the constitutions of South
Carolina and Virginia I find no allusion to the public provision
for education; but in those of North Carolina and Georgia it is
enjoined.  The forty-first section of the constitution for North
Carolina enjoins that ``schools shall be established by the
legislature for the convenient instruction of youth, with such
salaries to the masters, paid by the public, as may enable them to
instruct at \emph{low prices}''---showing that the intention here was to
assist education, and not provide it altogether gratuitously.  I
think that provision for public education is enjoined in the
constitutions of all the States admitted into the Union since the
first Federal knot was tied except in that of Illinois.  Vermont
was the first so admitted, in 1791; and Vermont declares that ``a
competent number of schools ought to be maintained in each town for
the convenient instruction of youth.''  Ohio was the second, in
1802; and Ohio enjoins that ``the General Assembly shall make such
provisions by taxation or otherwise as, with the income arising
from the school trust fund, will secure a thorough and efficient
system of common schools throughout the State; but no religions or
other sect or sects shall ever have any exclusive right or control
of any part of the school funds of this State.''  In Indiana,
admitted in 1816, it is required that ``the General Assembly shall
provide by law for a general and uniform system of common schools.''
Illinois was admitted next, in 1818; but the constitution of
Illinois is silent on the subject of education.  It enjoins,
however, in lieu of this, that no person shall fight a duel or send
a challenge!  If he do, he is not only to be punished, but to be
deprived forever of the power of holding any office of honor or
profit in the State.  I have no reason, however, for supposing that
education is neglected in Illinois, or that dueling has been
abolished.  In Maine it is demanded that the towns---the whole
country is divided into what are called towns---shall make suitable
provision at their own expense for the support and maintenance of
public schools.

Some of these constitutional enactments are most magniloquently
worded, but not always with precise grammatical correctness.  That
for the famous Bay State of Massachusetts runs as follows: ``Wisdom
and knowledge, as well as virtue, diffused generally among the body
of the people, being necessary for the preservation of their rights
and liberties, and as these depend on spreading the opportunities
and advantages of education in the various parts of the country and
among the different orders of the people, it shall be the duty of
the legislatures and magistrates, in all future periods of this
commonwealth, to cherish the interest of literature and the
sciences, and of all seminaries of them, especially the University
at Cambridge, public schools and grammar schools, in the towns; to
encourage private societies and public institutions by rewards and
immunities for the promotion of agriculture, arts, sciences,
commerce, trades, manufactures, and a natural history of the
country; to countenance and inculcate the principles of humanity
and general benevolence, public and private charity, industry and
frugality, honesty and punctuality in all their dealings;
sincerity, good humor, and all social affections and generous
sentiments among the people.''  I must confess that, had the words
of that little constitutional enactment been made known to me
before I had seen its practical results, I should not have put much
faith in it.  Of all the public schools I have ever seen---by public
schools I mean schools for the people at large maintained at public
cost---those of Massachusetts are, I think, the best.  But of all
the educational enactments which I ever read, that of the same
State is, I should say, the worst.  In Texas now, of which as a
State the people of Massachusetts do not think much, they have done
it better: ``A general diffusion of knowledge being essential to the
preservation of the rights and liberties of the people, it shall be
the duty of the legislature of this State to make suitable
provision for the support and maintenance of public schools.''  So
say the Texans; but then the Texans had the advantage of a later
experience than any which fell in the way of the constitution-
makers of Massachusetts.

There is something of the magniloquence of the French style---of the
liberty, equality, and fraternity mode of eloquence---in the
preambles of most of these constitutions, which, but for their
success, would have seemed to have prophesied loudly of failure.
Those of New York and Pennsylvania are the least so, and that of
Massachusetts by far the most violently magniloquent.  They
generally commence by thanking God for the present civil and
religious liberty of the people, and by declaring that all men are
born free and equal.  New York and Pennsylvania, however, refrain
from any such very general remarks.

I am well aware that all these constitutional enactments are not
likely to obtain much credit in England.  It is not only that grand
phrases fail to convince us, but that they carry to our senses
almost an assurance of their own inefficiency.  When we hear that a
people have declared their intention of being henceforward better
than their neighbors, and going upon a new theory that shall lead
them direct to a terrestrial paradise, we button up our pockets and
lock up our spoons.  And that is what we have done very much as
regards the Americans.  We have walked with them and talked with
them, and bought with them and sold with them; but we have
mistrusted them as to their internal habits and modes of life,
thinking that their philanthropy was pretentious and that their
theories were vague.  Many cities in the States are but skeletons
of towns, the streets being there, and the houses numbered---but not
one house built out of ten that have been so counted up.  We have
regarded their institutions as we regard those cities, and have
been specially willing so to consider them because of the fine
language in which they have been paraded before us.  They have been
regarded as the skeletons of philanthropical systems, to which
blood and flesh and muscle, and even skin, are wanting.  But it is
at least but fair to inquire how far the promise made has been
carried out.  The elaborate wordings of the constitutions made by
the French politicians in the days of their great revolution have
always been to us no more than so many written grimaces; but we
should not have continued so to regard them had the political
liberty which they promised followed upon the promises so
magniloquently made.  As regards education in the States---at any
rate in the Northern and Western States---I think that the
assurances put forth in the various written constitutions have been
kept.  If this be so, an American citizen, let him be ever so
arrogant, ever so impudent if you will, is at any rate a civilized
being, and on the road to that cultivation which will sooner or
later divest him of his arrogance.  Emollit mores.  We quote here
our old friend the colonel again.  If a gentleman be compelled to
confine his classical allusions to one quotation, he cannot do
better than hang by that.

But has education been so general, and has it had the desired
result?  In the City of Boston, as I have said, I found that in
1857 about one-eighth of the whole population were then on the
books of the free public schools as pupils, and that about one-
ninth of the population formed the average daily attendance.  To
these numbers of course must be added all pupils of the richer
classes---those for whose education their parents chose to pay.  As
nearly as I can learn, the average duration of each pupil's
schooling is six years, and if this be figured out statistically, I
think it will show that education in Boston reaches a very large
majority---I might almost say the whole---of the population.  That
the education given in other towns of Massachusetts is not so good
as that given in Boston I do not doubt, but I have reason to
believe that it is quite as general.

I have spoken of one of the schools of New York.  In that city the
public schools are apportioned to the wards, and are so arranged
that in each ward of the city there are public schools of different
standing for the gratuitous use of the children.  The population of
the City of New York in 1857 was about 650,000, and in that year it
is stated that there were 135,000 pupils in the schools.  By this
it would appear that one person in five throughout the city was
then under process of education---which statement, however, I cannot
receive with implicit credence.  It is, however, also stated that
the daily attendances averaged something less than 50,000 a day,
and this latter statement probably implies some mistake in the
former one.  Taking the two together for what they are worth, they
show, I think, that school teaching is not only brought within the
reach of the population generally, but is used by almost all
classes.  At New York there are separate free schools for colored
children.  At Philadelphia I did not see the schools, but I was
assured that the arrangements there were equal to those at New York
and Boston.  Indeed I was told that they were infinitely better;
but then I was so told by a Philadelphian.  In the State of
Connecticut the public schools are certainly equal to those in any
part of the union.  As far as I could learn education---what we
should call advanced education---is brought within the reach of all
classes in the Northern and Western States of America---and, I would
wish to add here, to those of the Canadas also.

So much for the schools, and now for the results.  I do not know
that anything impresses a visitor more strongly with the amount of
books sold in the States, than the practice of selling them as it
has been adopted in the railway cars.  Personally the traveler will
find the system very disagreeable---as is everything connected with
these cars.  A young man enters during the journey---for the trade
is carried out while the cars are traveling, as is also a very
brisk trade in lollipops, sugar-candy, apples, and ham sandwiches---%
the young tradesman enters the car firstly with a pile of
magazines, or of novels bound like magazines.  These are chiefly
the ``Atlantic,'' published at Boston, ``Harper's Magazine,'' published
at New York, and a cheap series of novels published at
Philadelphia.  As he walks along he flings one at every passenger.
An Englishman, when he is first introduced to this manner of trade,
becomes much astonished.  He is probably reading, and on a sudden
he finds a fat, fluffy magazine, very unattractive in its exterior,
dropped on to the page he is perusing.  I thought at first that it
was a present from some crazed philanthropist, who was thus
endeavoring to disseminate literature.  But I was soon undeceived.
The bookseller, having gone down the whole car and the next,
returned, and beginning again where he had begun before, picked up
either his magazine or else the price of it.  Then, in some half
hour, he came again, with an armful or basket of books, and
distributed them in the same way.  They were generally novels, but
not always.  I do not think that any endeavor is made to assimilate
the book to the expected customer.  The object is to bring the book
and the man together, and in this way a very large sale is
effected.  The same thing is done with illustrated newspapers.  The
sale of political newspapers goes on so quickly in these cars that
no such enforced distribution is necessary.  I should say that the
average consumption of newspapers by an American must amount to
about three a day.  At Washington I begged the keeper of my
lodgings to let me have a paper regularly---one American newspaper
being much the same to me as another---and my host supplied me daily
with four.

But the numbers of the popular books of the day, printed and sold,
afford the most conclusive proof of the extent to which education
is carried in the States.  The readers of Tennyson, Mackay,
Dickens, Bulwer, Collins, Hughes, and Martin Tupper are to be
counted by tens of thousands in the States, to the thousands by
which they may be counted in our own islands.  I do not doubt that
I had fully fifteen copies of the ``Silver Cord'' thrown at my head
in different railway cars on the continent of America.  Nor is the
taste by any means confined to the literature of England.
Longfellow, Curtis, Holmes, Hawthorne, Lowell, Emerson, and Mrs.\ %
Stowe are almost as popular as their English rivals.  I do not say
whether or no the literature is well chosen, but there it is.  It
is printed, sold, and read.  The disposal of ten thousand copies of
a work is no large sale in America of a book published at a dollar;
but in England it is a very large sale of a book brought out at
five shillings.

I do not remember that I ever examined the rooms of an American
without finding books or magazines in them.  I do not speak here of
the houses of my friends, as of course the same remark would apply
as strongly in England; but of the houses of persons presumed to
earn their bread by the labor of their hands.  The opportunity for
such examination does not come daily; but when it has been in my
power I have made it, and have always found signs of education.
Men and women of the classes to which I allude talk of reading and
writing as of arts belonging to them as a matter of course, quite
as much as are the arts of eating and drinking.  A porter or a
farmer's servant in the States is not proud of reading and writing.
It is to him quite a matter of course.  The coachmen on their boxes
and the boots as they set in the halls of the hotels have
newspapers constantly in their hands.  The young women have them
also, and the children.  The fact comes home to one at every turn,
and at every hour, that the people are an educated people.  The
whole of this question between North and South is as well
understood by the servants as by their masters, is discussed as
vehemently by the private soldiers as by the officers.  The
politics of the country and the nature of its Constitution are
familiar to every laborer.  The very wording of the Declaration of
Independence is in the memory of every lad of sixteen.  Boys and
girls of a younger age than that know why Slidell and Mason were
arrested, and will tell you why they should have been given up, or
why they should have been held in durance.  The question of the war
with England is debated by every native pavior and hodman of New
York.

I know what Englishmen will say in answer to this.  They will
declare that they do not want their paviors and hodmen to talk
politics; that they are as well pleased that their coachmen and
cooks should not always have a newspaper in their hands; that
private soldiers will fight as well, and obey better, if they are
not trained to discuss the causes which have brought them into the
field.  An English gentleman will think that his gardener will be a
better gardener without than with any excessive political ardor,
and the English lady will prefer that her housemaid shall not have
a very pronounced opinion of her own as to the capabilities of the
cabinet ministers.  But I would submit to all Englishmen and
English women who may look at these pages whether such an opinion
or feeling on their part bears much, or even at all, upon the
subject.  I am not saying that the man who is driven in the coach
is better off because his coachman reads the paper, but that the
coachman himself who reads the paper is better off than the
coachman who does not and cannot.  I think that we are too apt, in
considering the ways and habits of any people, to judge of them by
the effect of those ways and habits on us, rather than by their
effects on the owners of them.  When we go among garlic eaters, we
condemn them because they are offensive to us; but to judge of them
properly we should ascertain whether or no the garlic be offensive
to them.  If we could imagine a nation of vegetarians hearing for
the first time of our habits as flesh eaters, we should feel sure
that they would be struck with horror at our blood-stained
banquets; but when they came to argue with us, we should bid them
inquire whether we flesh eaters did not live longer and do more
than the vegetarians.  When we express a dislike to the shoeboy
reading his newspaper, I apprehend we do so because we fear that
the shoeboy is coming near our own heels.  I know there is among us
a strong feeling that the lower classes are better without
politics, as there is also that they are better without crinoline
and artificial flowers; but if politics, and crinoline, and
artificial flowers are good at all, they are good for all who can
honestly come by them and honestly use them.  The political
coachman is perhaps less valuable to his master as a coachman than
he would be without his politics, but he with his politics is more
valuable to himself.  For myself, I do not like the Americans of
the lower orders.  I am not comfortable among them.  They tread on
my corns and offend me.  They make my daily life unpleasant.  But I
do respect them.  I acknowledge their intelligence and personal
dignity.  I know that they are men and women worthy to be so
called; I see that they are living as human beings in possession of
reasoning faculties; and I perceive that they owe this to the
progress that education has made among them.

After all, what is wanted in this world?  Is it not that men should
eat and drink, and read and write, and say their prayers?  Does not
that include everything, providing that they eat and drink enough,
read and write without restraint, and say their prayers without
hypocrisy?  When we talk of the advances of civilization, do we
mean anything but this, that men who now eat and drink badly shall
eat and drink well, and that those who cannot read and write now
shall learn to do so---the prayers following, as prayers will follow
upon such learning?  Civilization does not consist in the eschewing
of garlic or the keeping clean of a man's finger-nails.  It may
lead to such delicacies, and probably will do so.  But the man who
thinks that civilization cannot exist without them imagines that
the church cannot stand without the spire.  In the States of
America men do eat and drink, and do read and write.

But as to saying their prayers?  That, as far as I can see, has
come also, though perhaps not in a manner altogether satisfactory,
or to a degree which should be held to be sufficient.  Englishmen
of strong religious feeling will often be startled in America by
the freedom with which religious subjects are discussed, and the
ease with which the matter is treated; but he will very rarely be
shocked by that utter absence of all knowledge on the subject---that
total darkness which is still so common among the lower orders in
our own country.  It is not a common thing to meet an American who
belongs to no denomination of Christian worship, and who cannot
tell you why he belongs to that which he has chosen.

``But,'' it will be said, ``all the intelligence and education of this
people have not saved them from falling out among themselves and
their friends, and running into troubles by which they will be
ruined.  Their political arrangements have been so bad that, in
spite of all their reading and writing, they must go to the wall.''
I venture to express an opinion that they will by no means go to
the wall, and that they will be saved from such a destiny, if in no
other way, then by their education.  Of their political
arrangements, as I mean before long to rush into that perilous
subject, I will say nothing here.  But no political convulsions,
should such arise---no revolution in the Constitution, should such
be necessary---will have any wide effect on the social position of
the people to their serious detriment.  They have the great
qualities of the Anglo-Saxon race---industry, intelligence, and
self-confidence; and if these qualities will no longer suffice to
keep such a people on their legs, the world must be coming to an
end.

I have said that it is not a common thing to meet an American who
belongs to no denomination of Christian worship.  This I think is
so but I would not wish to be taken as saying that religion, on
that account, stands on a satisfactory footing in the States.  Of
all subjects of discussion, this is the most difficult.  It is one
as to which most of us feel that to some extent we must trust to
our prejudices rather than our judgments.  It is a matter on which
we do not dare to rely implicitly on our own reasoning faculties,
and therefore throw ourselves on the opinions of those whom we
believe to have been better men and deeper thinkers than ourselves.
For myself, I love the name of State and Church, and believe that
much of our English well-being has depended on it.  I have made up
my mind to think that union good, and am not to be turned away from
that conviction.  Nevertheless I am not prepared to argue the
matter.  One does not always carry one's proof at one's finger
ends.

But I feel very strongly that much of that which is evil in the
structure of American politics is owing to the absence of any
national religion, and that something also of social evil has
sprung from the same cause.  It is not that men do not say their
prayers.  For aught I know, they may do so as frequently and as
fervently, or more frequently and more fervently, than we do; but
there is a rowdiness, if I may be allowed to use such a word, in
their manner of doing so which robs religion of that reverence
which is, if not its essence, at any rate its chief protection.  It
is a part of their system that religion shall be perfectly free,
and that no man shall be in any way constrained in that matter.
Consequently, the question of a man's religion is regarded in a
free-and-easy way.  It is well, for instance, that a young lad
should go somewhere on a Sunday; but a sermon is a sermon, and it
does not much concern the lad's father whether his son hear the
discourse of a freethinker in the music-hall, or the eloquent but
lengthy outpouring of a preacher in a Methodist chapel.  Everybody
is bound to have a religion, but it does not much matter what it
is.

The difficulty in which the first fathers of the Revolution found
themselves on this question is shown by the constitutions of the
different States.  There can be no doubt that the inhabitants of
the New England States were, as things went, a strictly religious
community.  They had no idea of throwing over the worship of God,
as the French had attempted to do at their revolution.  They
intended that the new nation should be pre-eminently composed of a
God-fearing people; but they intended also that they should be a
people free in everything---free to choose their own forms of
worship.  They intended that the nation should be a Protestant
people; but they intended also that no man's conscience should be
coerced in the matter of his own religion.  It was hard to
reconcile these two things, and to explain to the citizens that it
behooved them to worship God---even under penalties for omission;
but that it was at the same time open to them to select any form of
worship that they pleased, however that form might differ from the
practices of the majority.  In Connecticut it is declared that it
is the duty of all men to worship the Supreme Being, the Creator
and Preserver of the universe, but that it is their right to render
that worship in the mode most consistent with the dictates of their
consciences.  And then, a few lines further down, the article skips
the great difficulty in a manner somewhat disingenuous, and
declares that each and every society of Christians in the State
shall have and enjoy the same and equal privileges.  But it does
not say whether a Jew shall be divested of those privileges, or, if
he be divested, how that treatment of him is to be reconciled with
the assurance that it is every man's right to worship the Supreme
Being in the mode most consistent with the dictates of his own
conscience.

In Rhode Island they were more honest.  It is there declared that
every man shall be free to worship God according to the dictates of
his own conscience, and to profess and by argument to maintain his
opinion in matters of religion; and that the same shall in no wise
diminish, enlarge, or affect his civil capacity.  Here it is simply
presumed that every man will worship a God, and no allusion is made
even to Christianity.

In Massachusetts they are again hardly honest.  ``It is the right,''
says the constitution, ``as well as the duty of all men in society
publicly and at stated seasons to worship the supreme Being, the
Great Creator and Preserver of the universe.''  And then it goes on
to say that every man may do so in what form he pleases; but
further down it declares that ``every denomination of Christians,
demeaning themselves peaceably and as good subjects of the
commonwealth, shall be equally under the protection of the law.''
But what about those who are not Christians?  In New Hampshire it
is exactly the same.  It is enacted that ``every individual has a
natural and unalienable right to worship God according to the
dictates of his own conscience and reason.''  And that ``every
denomination of Christians, demeaning themselves quietly and as
good citizens of the State, shall be equally under the protection
of the law.''  From all which it is, I think, manifest that the men
who framed these documents, desirous above all things of cutting
themselves and their people loose from every kind of trammel, still
felt the necessity of enforcing religion---of making it, to a
certain extent, a matter of State duty.  In the first constitution
of North Carolina it is enjoined ``that no person who shall deny the
being of God, or the truth of the Protestant religion, shall be
capable of holding any office or place of trust or profit.''  But
this was altered in the year 1836, and the words ``Christian
religion'' were substituted for ``Protestant religion.''

In New England the Congregationalists are, I think, the dominant
sect.  In Massachusetts, and I believe in the other New England
States, a man is presumed to be a Congregationalist if he do not
declare himself to be anything else; as with us the Church of
England counts all who do not specially have themselves counted
elsewhere.  The Congregationalist, as far as I can learn, is very
near to a Presbyterian.  In New England I think the Unitarians
would rank next in number; but a Unitarian in America is not the
same as a Unitarian with us.  Here, if I understand the nature of
his creed, a Unitarian does not recognize the divinity of our
Saviour.  In America he does do so, but throws over the doctrine of
the Trinity.  The Protestant Episcopalians muster strong in all the
great cities, and I fancy that they would be regarded as taking the
lead of the other religious denominations in New York.  Their
tendency is to high-church doctrines.  I wish they had not found it
necessary to alter the forms of our prayer-book in so many little
matters, as to which there was no national expediency for such
changes.  But it was probably thought necessary that a new people
should show their independence in all things.  The Roman Catholics
have a very strong party---as a matter of course---seeing how great
has been the emigration from Ireland; but here, as in Ireland---and
as indeed is the case all the world over---the Roman Catholics are
the hewers of wood and drawers of water.  The Germans, who have
latterly flocked into the States in such swarms that they have
almost Germanized certain States, have, of course, their own
churches.  In every town there are places of worship for Baptists,
Presbyterians, Methodists, Anabaptists, and every denomination of
Christianity; and the meeting-houses prepared for these sects are
not, as with us, hideous buildings, contrived to inspire disgust by
the enormity of their ugliness, nor are they called Salem,
Ebenezer, and Sion, nor do the ministers within them look in any
way like the Deputy-Shepherd.  The churches belonging to those
sects are often handsome.  This is especially the case in New York,
and the pastors are not unfrequently among the best educated and
most agreeable men whom the traveler will meet.  They are for the
most part well paid, and are enabled by their outward position to
hold that place in the world's ranks which should always belong to
a clergyman.  I have not been able to obtain information from which
I can state with anything like correctness what may be the average
income of ministers of the Gospel in the Northern States; but that
it is much higher than the average income of our parish clergymen,
admits, I think, of no doubt.  The stipends of clergymen in the
American towns are higher than those paid in the country.  The
opposite to this, I think, as a rule, is the case with us.

I have said that religion in the States is rowdy.  By that I mean
to imply that it seems to me to be divested of that reverential
order and strictness of rule which, according to our ideas, should
be attached to matters of religion.  One hardly knows where the
affairs of this world end, or where those of the next begin.  When
the holy men were had in at the lecture, were they doing stage-work
or church-work?  On hearing sermons, one is often driven to ask
one's self whether the discourse from the pulpit be in its nature
political or religious.  I heard an Episcopalian Protestant
clergyman talk of the scoffing nations of Europe, because at that
moment he was angry with England and France about Slidell and
Mason.  I have heard a chapter of the Bible read in Congress at the
desire of a member, and very badly read.  After which the chapter
itself and the reading of it became a subject of debate, partly
jocose and partly acrimonious.  It is a common thing for a
clergyman to change his profession and follow any other pursuit.  I
know two or three gentlemen who were once in that line of life, but
have since gone into other trades.  There is, I think, an
unexpressed determination on the part of the people to abandon all
reverence, and to regard religion from an altogether worldly point
of view.  They are willing to have religion, as they are willing to
have laws; but they choose to make it for themselves.  They do not
object to pay for it, but they like to have the handling of the
article for which they pay.  As the descendants of Puritans and
other godly Protestants, they will submit to religious teaching,
but as republicans they will have no priestcraft.  The French at
their revolution had the latter feeling without the former, and
were therefore consistent with themselves in abolishing all
worship.  The Americans desire to do the same thing politically,
but infidelity has had no charms for them.  They say their prayers,
and then seem to apologize for doing so, as though it were hardly
the act of a free and enlightened citizen, justified in ruling
himself as he pleases.  All this to me is rowdy.  I know no other
word by which I can so well describe it.

Nevertheless the nation is religious in its tendencies, and prone
to acknowledge the goodness of God in all things.  A man there is
expected to belong to some church, and is not, I think, well looked
on if he profess that he belongs to none.  He may be a
Swedenborgian, a Quaker, a Muggletonian,---anything will do, But it
is expected of him that he shall place himself under some flag, and
do his share in supporting the flag to which he belongs.  This duty
is, I think, generally fulfilled.



\chapter{From Boston to Washington}


From Boston, on the 27th of November, my wife returned to England,
leaving me to prosecute my journey southward to Washington by
myself.  I shall never forget the political feeling which prevailed
in Boston at that time, or the discussions on the subject of
Slidell and Mason, in which I felt myself bound to take a part.  Up
to that period I confess that my sympathies had been strongly with
the Northern side in the general question; and so they were still,
as far as I could divest the matter of its English bearings.  I
have always thought, and do think, that a war for the suppression
of the Southern rebellion could not have been avoided by the North
without an absolute loss of its political prestige.  Mr.\ Lincoln
was elected President of the United States in the autumn of 1860,
and any steps taken by him or his party toward a peaceable solution
of the difficulties which broke out immediately on his election
must have been taken before he entered upon his office.  South
Carolina threatened secession as soon as Mr.\ Lincoln's election was
known, while yet there were four months left of Mr.\ Buchanan's
government.  That Mr.\ Buchanan might, during those four months,
have prevented secession, few men, I think, will doubt when the
history of the time shall be written.  But instead of doing so he
consummated secession.  Mr.\ Buchanan is a Northern man, a
Pennsylvanian; but he was opposed to the party which had brought in
Mr.\ Lincoln, having thriven as a politician by his adherence to
Southern principles.  Now, when the struggle came, he could not
forget his party in his duty as President.  General Jackson's
position was much the same when Mr.\ Calhoun, on the question of the
tariff, endeavored to produce secession in South Carolina thirty
years ago, in 1832---excepting in this, that Jackson was himself a
Southern man.  But Jackson had a strong conception of the position
which he held as President of the United States.  He put his foot
on secession and crushed it, forcing Mr.\ Calhoun, as Senator from
South Carolina, to vote for that compromise as to the tariff which
the government of the day proposed.  South Carolina was as eager in
1832 for secession as she was in 1859-60; but the government was in
the hands of a strong man and an honest one.  Mr.\ Calhoun would
have been hung had he carried out his threats.  But Mr.\ Buchanan
had neither the power nor the honesty of General Jackson, and thus
secession was in fact consummated during his Presidency.

But Mr.\ Lincoln's party, it is said---and I believe truly said---%
might have prevented secession by making overtures to the South, or
accepting overtures from the South, before Mr.\ Lincoln himself had
been inaugurated.  That is to say, if Mr.\ Lincoln and the band of
politicians who with him had pushed their way to the top of their
party, and were about to fill the offices of State, chose to throw
overboard the political convictions which had bound them together
and insured their success---if they could bring themselves to adopt
on the subject of slavery the ideas of their opponents---then the
war might have been avoided, and secession also avoided.  I do
believe that had Mr.\ Lincoln at that time submitted himself to a
compromise in favor of the Democrats, promising the support of the
government to certain acts which would in fact have been in favor
of slavery, South Carolina would again have been foiled for the
time.  For it must be understood, that though South Carolina and
the Gulf States might have accepted certain compromises, they would
not have been satisfied in so accepting them.  The desired
secession, and nothing short of secession, would in truth have been
acceptable to them.  But in doing so Mr.\ Lincoln would have been
the most dishonest politician even in America.  The North would
have been in arms against him; and any true spirit of agreement
between the cotton-growing slave States and the manufacturing
States of the North, or the agricultural States of the West, would
have been as far off and as improbable as it is now.  Mr.\ %
Crittenden, who proffered his compromise to the Senate in December,
1860, was at that time one of the two Senators from Kentucky, a
slave State.  He now sits in the Lower House of Congress as a
member from the same State.  Kentucky is one of those border States
which has found it impossible to secede, and almost equally
impossible to remain in the Union.  It is one of the States into
which it was most probable that the war would be carried---Virginia,
Kentucky, and Missouri being the three States which have suffered
the most in this way.  Of Mr.\ Crittenden's own family, some have
gone with secession and some with the Union.  His name had been
honorably connected with American politics for nearly forty years,
and it is not surprising that he should have desired a compromise.
His terms were in fact these---a return to the Missouri compromise,
under which the Union pledged itself that no slavery should exist
north of 36.30 degrees N. lat., unless where it had so existed
prior to the date of that compromise; a pledge that Congress would
not interfere with slavery in the individual States---which under
the Constitution it cannot do; and a pledge that the Fugitive Slave
Law should be carried out by the Northern States.  Such a
compromise might seem to make very small demand on the forbearance
of the Republican party, which was now dominant.  The repeal of the
Missouri compromise had been to them a loss, and it might be said
that its re-enactment would be a gain.  But since that compromise
had been repealed, vast territories south of the line in question
had been added to the union, and the re-enactment of that
compromise would hand those vast regions over to absolute slavery,
as had been done with Texas.  This might be all very well for Mr.\ %
Crittenden in the slave State of Kentucky---for Mr.\ Crittenden,
although a slave owner, desired to perpetuate the Union; but it
would not have been well for New England or for the West.  As for
the second proposition, it is well understood that under the
Constitution Congress cannot interfere in any way in the question
of slavery in the individual States.  Congress has no more
constitutional power to abolish slavery in Maryland than she has to
introduce it into Massachusetts.  No such pledge, therefore, was
necessary on either side.  But such a pledge given by the North and
West would have acted as an additional tie upon them, binding them
to the finality of a constitutional enactment to which, as was of
course well known, they strongly object.  There was no question of
Congress interfering with slavery, with the purport of extending
its area by special enactment, and therefore by such a pledge the
North and West could gain nothing; but the South would in prestige
have gained much.

But that third proposition as to the Fugitive Slave Law and the
faithful execution of that law by the Northern and Western States
would, if acceded to by Mr.\ Lincoln's party, have amounted to an
unconditional surrender of everything.  What!  Massachusetts and
Connecticut carry out the Fugitive Slave Law?  Ohio carry out the
Fugitive Slave Law after the ``Dred Scott'' decision and all its
consequences?  Mr.\ Crittenden might as well have asked Connecticut,
Massachusetts, and Ohio to introduce slavery within their own
lands.  The Fugitive Slave Law was then, as it is now, the law of
the land; it was the law of the United States as voted by Congress,
and passed by the President, and acted on by the supreme judge of
the United States Court.  But it was a law to which no free State
had submitted itself, or would submit itself.  ``What!'' the English
reader will say, ``sundry States in the Union refuse to obey the
laws of the Union---refuse to submit to the constitutional action of
their own Congress?''  Yes.  Such has been the position of this
country!  To such a dead lock has it been brought by the attempted
but impossible amalgamation of North and South.  Mr.\ Crittenden's
compromise was moonshine.  It was utterly out of the question that
the free States should bind themselves to the rendition of escaped
slaves, or that Mr.\ Lincoln, who had just been brought in by their
voices, should agree to any compromise which should attempt so to
bind them.  Lord Palmerston might as well attempt to reenact the
Corn Laws.

Then comes the question whether Mr.\ Lincoln or his government could
have prevented the war after he had entered upon his office in
March, 1861?  I do not suppose that any one thinks that he could
have avoided secession and avoided the war also; that by any
ordinary effort of government he could have secured the adhesion of
the Gulf States to the Union after the first shot had been fired at
Fort Sumter.  The general opinion in England is, I take it, this---%
that secession then was manifestly necessary, and that all the
blood-shed and money-shed, and all this destruction of commerce and
of agriculture might have been prevented by a graceful adhesion to
an indisputable fact.  But there are some facts, even some
indisputable facts, to which a graceful adherence is not possible.
Could King Bomba have welcomed Garibaldi to Naples?  Can the Pope
shake hands with Victor Emmanuel?  Could the English have
surrendered to their rebel colonists peaceable possession of the
colonies?  The indisputability of a fact is not very easily settled
while the circumstances are in course of action by which the fact
is to be decided.  The men of the Northern States have not believed
in the necessity of secession, but have believed it to be their
duty to enforce the adherence of these States to the Union.  The
American governments have been much given to compromises, but had
Mr.\ Lincoln attempted any compromise by which any one Southern
State could have been let out of the Union, he would have been
impeached.  In all probability the whole Constitution would have
gone to ruin, and the Presidency would have been at an end.  At any
rate, his Presidency would have been at an end.  When secession, or
in other words rebellion, was once commenced, he had no alternative
but the use of coercive measures for putting it down---that is, he
had no alternative but war.  It is not to be supposed that he or
his ministry contemplated such a war as has existed---with 600,000
men in arms on one side, each man with his whole belongings
maintained at a cost of 150l. per annum, or ninety millions
sterling per annum for the army.  Nor did we when we resolved to
put down the French revolution think of such a national debt as we
now owe.  These things grow by degrees, and the mind also grows in
becoming used to them; but I cannot see that there was any moment
at which Mr.\ Lincoln could have stayed his hand and cried peace.
It is easy to say now that acquiescence in secession would have
been better than war, but there has been no moment when he could
have said so with any avail.  It was incumbent on him to put down
rebellion, or to be put down by it.  So it was with us in America
in 1776.

I do not think that we in England have quite sufficiently taken all
this into consideration.  We have been in the habit of exclaiming
very loudly against the war, execrating its cruelty and
anathematizing its results, as though the cruelty were all
superfluous and the results unnecessary.  But I do not remember to
have seen any statement as to what the Northern States should have
done---what they should have done, that is, as regards the South, or
when they should have done it.  It seems to me that we have decided
as regards them that civil war is a very bad thing, and that
therefore civil war should be avoided.  But bad things cannot
always be avoided.  It is this feeling on our part that has
produced so much irritation in them against us---reproducing, of
course, irritation on our part against them.  They cannot
understand that we should not wish them to be successful in putting
down a rebellion; nor can we understand why they should be
outrageous against us for standing aloof, and keeping our hands, if
it be only possible, out of the fire.

When Slidell and Mason were arrested, my opinions were not changed,
but my feelings were altered.  I seemed to acknowledge to myself
that the treatment to which England had been subjected, and the
manner in which that treatment was discussed, made it necessary
that I should regard the question as it existed between England and
the States, rather than in its reference to the North and South.  I
had always felt that as regarded the action of our government we
had been sans reproche; that in arranging our conduct we had
thought neither of money nor political influence, but simply of the
justice of the case---promising to abstain from all interference and
keeping that promise faithfully.  It had been quite clear to me
that the men of the North, and the women also, had failed to
appreciate this, looking, as men in a quarrel always do look, for
special favor on their side.  Everything that England did was
wrong.  If a private merchant, at his own risk, took a cargo of
rifles to some Southern port, that act to Northern eyes was an act
of English interference---of favor shown to the South by England as
a nation; but twenty shiploads of rifles sent from England to the
North merely signified a brisk trade and a desire for profit.  The
``James Adger,'' a Northern man-of-war, was refitted at Southampton
as a matter of course.  There was no blame to England for that.
But the Nashville, belonging to the Confederates, should not have
been allowed into English waters.  It was useless to speak of
neutrality.  No Northerner would understand that a rebel could have
any mutual right.  The South had no claim in his eyes as a
belligerent, though the North claimed all those rights which he
could only enjoy by the fact of there being a recognized war
between him and his enemy the South.  The North was learning to
hate England, and day by day the feeling grew upon me that, much as
I wished to espouse the cause of the North, I should have to
espouse the cause of my own country.  Then Slidell and Mason were
arrested, and I began to calculate how long I might remain in the
country.  ``There is no danger.  We are quite right,'' the lawyers
said.  ``There are Vattel, and Puffendorff, and Stowell, and
Phillimore, and Wheaton,'' said the ladies.  ``Ambassadors are
contraband all the world over---more so than gunpowder; and if taken
in a neutral bottom,'' etc.  I wonder why ships are always called
bottoms when spoken of with legal technicality?  But neither the
lawyers nor the ladies convinced me.  I know that there are matters
which will be read not in accordance with any written law, but in
accordance with the bias of the reader's mind.  Such laws are made
to be strained any way.  I knew how it would be.  All the legal
acumen of New England declared the seizure of Slidell and Mason to
be right.  The legal acumen of Old England has declared it to be
wrong; and I have no doubt that the ladies of Old England can prove
it to be wrong out of Yattel, Puffendorff, Stowell, Phillimore, and
Wheaton.

``But there's Grotius,'' I said, to an elderly female at New York,
who had quoted to me some half dozen writers on international law,
thinking thereby that I should trump her last card.  ``I've looked
into Grotius too,'' said she, ``and as far as I can see,'' etc. etc.
etc.  So I had to fall back again on the convictions to which
instinct and common sense had brought me.  I never doubted for a
moment that those convictions would be supported by English
lawyers.

I left Boston with a sad feeling at my heart that a quarrel was
imminent between England and the States, and that any such quarrel
must be destructive to the cause of the North.  I had never
believed that the States of New England and the Gulf States would
again become parts of one nation, but I had thought that the terms
of separation would be dictated by the North, and not by the South.
I had felt assured that South Carolina and the Gulf States, across
from the Atlantic to Texas, would succeed in forming themselves
into a separate confederation; but I had still hoped that Maryland,
Virginia, Kentucky, and Missouri might be saved to the grander
empire of the North, and that thus a great blow to slavery might be
the consequence of this civil war.  But such ascendency could only
fall to the North by reason of their command of the sea.  The
Northern ports were all open, and the Southern ports were all
closed.  But if this should be reversed.  If by England's action
the Southern ports should be opened, and the Northern ports closed,
the North could have no fair expectation of success.  The
ascendency in that case would all be with the South.  Up to that
moment---the Christmas of 1861---Maryland was kept in subjection by
the guns which General Dix had planted over the City of Baltimore.
Two-thirds of Virginia were in active rebellion, coerced originally
into that position by her dependence for the sale of her slaves on
the cotton States.  Kentucky was doubtful, and divided.  When the
Federal troops prevailed, Kentucky was loyal; when the Confederate
troops prevailed, Kentucky was rebellious.  The condition in
Missouri was much the same.  These four States, by two of which the
capital, with its District of Columbia, is surrounded, might be
gained or might be lost.  And these four States are susceptible of
white labor---as much so as Ohio and Illinois---are rich in
fertility, and rich also in all associations which must be dear to
Americans.  Without Virginia, Maryland, and Kentucky, without the
Potomac, the Chesapeake, and Mount Vernon, the North would indeed
be shorn of its glory!  But it seemed to be in the power of the
North to say under what terms secession should take place, and
where should be the line.  A Senator from South Carolina could
never again sit in the same chamber with one from Massachusetts;
but there need be no such bar against the border States.  So much
might at any rate be gained, and might stand hereafter as the
product of all that money spent on 600,000 soldiers.  But if the
Northerners should now elect to throw themselves into a quarrel
with England, if in the gratification of a shameless braggadocio
they should insist on doing what they liked, not only with their
own, but with the property of all others also, it certainly did
seem as though utter ruin must await their cause.  With England, or
one might say with Europe, against them, secession must be
accomplished, not on Northern terms, but on terms dictated by the
South.  The choice was then for them to make; and just at that time
it seemed as though they were resolved to throw away every good
card out of their hand.  Such had been the ministerial wisdom of
Mr.\ Seward.  I remember hearing the matter discussed in easy terms
by one of the United States Senators.  ``Remember, Mr.\ Trollope,'' he
said to me, ``we don't want a war with England.  If the choice is
given to us, we had rather not fight England.  Fighting is a bad
thing.  But remember this also, Mr.\ Trollope, that if the matter is
pressed on us, we have no great objection.  We had rather not, but
we don't care much one way or the other.''  What one individual may
say to another is not of much moment, but this Senator was
expressing the feelings of his constituents, who were the
legislature of the State from whence he came.  He was expressing
the general idea on the subject of a large body of Americans.  It
was not that he and his State had really no objection to the war.
Such a war loomed terribly large before the minds of them all.
They know it to be fraught with the saddest consequences.  It was
so regarded in the mind of that Senator.  But the braggadocio could
not be omitted.  Had be omitted it, he would have been untrue to
his constituency.

When I left Boston for Washington, nothing was as yet known of what
the English government or the English lawyers might say.  This was
in the first week in December, and the expected voice from England
could not be heard till the end of the second week.  It was a
period of great suspense, and of great sorrow also to the more
sober-minded Americans.  To me the idea of such a war was terrible.
It seemed that in these days all the hopes of our youth were being
shattered.  That poetic turning of the sword into a sickle, which
gladdened our hearts ten or twelve years since, had been clean
banished from men's minds.  To belong to a peace party was to be
either a fanatic, an idiot, or a driveler.  The arts of war had
become everything.  Armstrong guns, themselves indestructible but
capable of destroying everything within sight, and most things out
of sight, were the only recognized results of man's inventive
faculties.  To build bigger, stronger, and more ships than the
French was England's glory.  To hit a speck with a rifle bullet at
800 yards distance was an Englishman's first duty.  The proper use
for a young man's leisure hours was the practice of drilling.  All
this had come upon us with very quick steps since the beginning of
the Russian war.  But if fighting must needs be done, one did not
feel special grief at fighting a Russian.  That the Indian mutiny
should be put down was a matter of course.  That those Chinese
rascals should be forced into the harness of civilization was a
good thing.  That England should be as strong as France---or,
perhaps, if possible a little stronger---recommended itself to an
Englishman's mind as a State necessity.  But a war with the States
of America!  In thinking of it I began to believe that the world
was going backward.  Over sixty millions sterling of stock---railway
stock and such like---are held in America by Englishmen, and the
chances would be that before such a war could be finished the whole
of that would be confiscated.  Family connections between the
States and the British isles are almost as close as between one of
those islands and another.  The commercial intercourse between the
two countries has given bread to millions of Englishmen, and a
break in it would rob millions of their bread.  These people speak
our language, use our prayers, read our books, are ruled by our
laws, dress themselves in our image, are warm with our blood.  They
have all our virtues; and their vices are our own too, loudly as we
call out against them.  They are our sons and our daughters, the
source of our greatest pride, and as we grow old they should be the
staff of our age.  Such a war as we should now wage with the States
would be an unloosing of hell upon all that is best upon the
world's surface.  If in such a war we beat the Americans, they with
their proud stomachs would never forgive us.  If they should be
victors, we should never forgive ourselves.  I certainly could not
bring myself to speak of it with the equanimity of my friend the
Senator.

I went through New York to Philadelphia, and made a short visit to
the latter town.  Philadelphia seems to me to have thrown off its
Quaker garb, and to present itself to the world in the garments
ordinarily assumed by large cities---by which I intend to express my
opinion that the Philadelphians are not, in these latter days, any
better than their neighbors.  I am not sure whether in some
respects they may not perhaps be worse.  Quakers---Quakers
absolutely in the very flesh of close bonnets and brown knee-
breeches---are still to be seen there; but they are not numerous,
and would not strike the eye if one did not specially look for a
Quaker at Philadelphia.  It is a large town, with a very large
hotel---there are no doubt half a dozen large hotels, but one of
them is specially great---with long, straight streets, good shops
and markets, and decent, comfortable-looking houses.  The houses of
Philadelphia generally are not so large as those of other great
cities in the States.  They are more modest than those of New York,
and less commodious than those of Boston.  Their most striking
appendage is the marble steps at the front doors.  Two doors, as a
rule, enjoy one set of steps, on the outer edges of which there is
generally no parapet or raised curb-stone.  This, to my eye, gave
the houses an unfinished appearance---as though the marble ran
short, and no further expenditure could be made.  The frost came
when I was there, and then all these steps were covered up in
wooden cases.

The City of Philadelphia lies between the two rivers, the Delaware
and the Schuylkill.  Eight chief streets run from river to river,
and twenty-four principal cross-streets bisect the eight at right
angles.  The cross-streets are all called by their numbers.  In the
long streets the numbers of the houses are not consecutive, but
follow the numbers of the cross-streets; so that a person living on
Chestnut Street between Tenth Street and Eleventh Street, and ten
doors from Tenth Street, would live at No.\ 1010.  The opposite
house would be No.\ 1011.  It thus follows that the number of the
house indicates the exact block of houses in which it is situated.
I do not like the right-angled building of these towns, nor do I
like the sound of Twentieth Street and Thirtieth Street; but I must
acknowledge that the arrangement in Philadelphia has its
convenience.  In New York I found it by no means an easy thing to
arrive at the desired locality.

They boast in Philadelphia that they have half a million
inhabitants.  If this be taken as a true calculation, Philadelphia
is in size the fourth city in the world---putting out of the
question the cities of China, as to which we have heard so much and
believe so little.  But in making this calculation the citizens
include the population of a district on some sides ten miles
distant from Philadelphia.  It takes in other towns, connected with
it by railway but separated by large spaces of open country.
American cities are very proud of their population; but if they all
counted in this way, there would soon be no rural population left
at all.  There is a very fine bank at Philadelphia, and
Philadelphia is a town somewhat celebrated in its banking history.
My remarks here, however, apply simply to the external building,
and not to its internal honesty and wisdom, or to its commercial
credit.

In Philadelphia also stands the old house of Congress---the house in
which the Congress of the United States was held previous to 1800,
when the government and the Congress with it were moved to the new
City of Washington.  I believe, however, that the first Congress,
properly so called, was assembled at New York in 1789, the date of
the inauguration of the first President.  It was, however, here in
this building at Philadelphia that the independence of the Union
was declared in 1776, and that the Constitution of the United
States was framed.

Pennsylvania, with Philadelphia for its capital, was once the
leading State of the Union, leading by a long distance.  At the end
of the last century it beat all the other States in population, but
has since been surpassed by New York in all respects---in
population, commerce, wealth, and general activity.  Of course it
is known that Pennsylvania was granted to William Penn, the Quaker,
by Charles II.  I cannot completely understand what was the meaning
of such grants---how far they implied absolute possession in the
territory, or how far they confirmed simply the power of settling
and governing a colony.  In this case a very considerable property
was confirmed; as the claim made by Penn's children, after Penn's
death, was bought up by the commonwealth of Pennsylvania for
130,000l., which, in those days, was a large price for almost any
landed estate on the other side of the Atlantic.

Pennsylvania lies directly on the borders of slave land, being
immediately north of Maryland.  Mason and Dixon's line, of which we
hear so often, and which was first established as the division
between slave soil and free soil, runs between Pennsylvania and
Maryland.  The little State of Delaware, which lies between
Maryland and the Atlantic, is also tainted with slavery, but the
stain is not heavy nor indelible.  In a population of a hundred and
twelve thousand, there are not two thousand slaves, and of these
the owners generally would willingly rid themselves if they could.
It is, however, a point of honor with these owners, as it is also
in Maryland, not to sell their slaves; and a man who cannot sell
his slaves must keep them.  Were he to enfranchise them and send
them about their business, they would come back upon his hands.
Were he to enfranchise them and pay them wages for work, they would
get the wages, but he would not get the work.  They would get the
wages; but at the end of three months they would still fall back
upon his hands in debt and distress, looking to him for aid and
comfort as a child looks for it.  It is not easy to get rid of a
slave in a slave State.  That question of enfranchising slaves is
not one to be very readily solved.

In Pennsylvania the right of voting is confined to free white men.
In New York the colored free men have the right to vote, providing
they have a certain small property qualification, and have been
citizens for three years in the State, whereas a white man need
have been a citizen but for ten days, and need have no property
qualification---from which it is seen that the position of the negro
becomes worse, or less like that of a white man, as the border of
slave land is more nearly reached.  But, in the teeth of this
embargo on colored men, the constitution of Pennsylvania asserts
broadly that all men are born equally free and independent.  One
cannot conceive how two clauses can have found their way into the
same document so absolutely contradictory to each other.  The first
clause says that white men shall vote, and that black men shall
not---which means that all political action shall be confined to
white men.  The second clause says that all men are born equally
free and independent.

In Philadelphia I for the first time came across live
secessionists---secessionists who pronounced themselves to be such.
I will not say that I had met in other cities men who falsely
declared themselves true to the Union; but I had fancied, in regard
to some, that their words were a little stronger than their
feelings.  When a man's bread---and, much more, when the bread of
his wife and children---depends on his professing a certain line of
political conviction, it is very hard for him to deny his assent to
the truth of the argument.  One feels that a man, under such
circumstances, is bound to be convinced, unless he be in a position
which may make a stanch adherence to opposite politics a matter of
grave public importance.  In the North I had fancied that I could
sometimes read a secessionist tendency under a cloud of Unionist
protestations.  But in Philadelphia men did not seem to think it
necessary to have recourse to such a cloud.  I generally found, in
mixed society, that even there the discussion of secession was not
permitted; but in society that was not mixed I heard very strong
opinions expressed on each side.  With the Unionists nothing was so
strong as the necessity of keeping of Slidell and Mason; when I
suggested that the English government would probably require their
surrender, I was talked down and ridiculed.  ``Never that---come what
may.''  Then, within half an hour, I would be told by a secessionist
that England must demand reparation if she meant to retain any
place among the great nations of the world; but he also would
declare that the men would not be surrendered.  ``She must make the
demand,'' the secessionists would say, ``and then there will be war;
and after that we shall see whose ports will be blockaded!''  The
Southerner has ever looked to England for some breach of the
blockade quite as strongly as the North has looked to England for
sympathy and aid in keeping it.

The railway from Philadelphia to Baltimore passes along the top of
Chesapeake Bay and across the Susquehanna River; at least the
railway cars do so.  On one side of that river they are run on to a
huge ferry-boat, and are again run off at the other side.  Such an
operation would seem to be one of difficulty to us under any
circumstances; but as the Susquehanna is a tidal river, rising and
falling a considerable number of feet, the natural impediment in
the way of such an enterprise would, I think, have staggered us.
We should have built a bridge costing two or three millions
sterling, on which no conceivable amount of traffic would pay a
fair dividend.  Here, in crossing the Susquehanna, the boat is so
constructed that its deck shall be level with the line of the
railway at half tide, so that the inclined plane from the shore
down to the boat, or from the shore up to the boat, shall never
exceed half the amount of the rise or fall.  One would suppose that
the most intricate machinery would have been necessary for such an
arrangement; but it was all rough and simple, and apparently
managed by two negroes.  We would employ a small corps of engineers
to conduct such an operation, and men and women would be detained
in their carriages under all manner of threats as to the peril of
life and limb; but here everybody was expected to look out for
himself.  The cars were dragged up the inclined plane by a hawser
attached to an engine, which hawser, had the stress broken it, as I
could not but fancy probable, would have flown back and cut to
pieces a lot of us who were standing in front of the car.  But I do
not think that any such accident would have caused very much
attention.  Life and limbs are not held to be so precious here as
they are in England.  It may be a question whether with us they are
not almost too precious.  Regarding railways in America generally,
as to the relative safety of which, when compared with our own, we
have not in England a high opinion, I must say that I never saw any
accident or in any way became conversant with one.  It is said that
large numbers of men and women are slaughtered from time to time on
different lines; but if it be so, the newspapers make very light of
such cases.  I myself have seen no such slaughter, nor have I even
found myself in the vicinity of a broken bone.  Beyond the
Susquehanna we passed over a creek of Chesapeake Bay on a long
bridge.  The whole scenery here is very pretty, and the view up the
Susquehanna is fine.  This is the bay which divides the State of
Maryland into two parts, and which is blessed beyond all other bays
by the possession of canvas-back ducks.  Nature has done a great
deal for the State of Maryland, but in nothing more than in sending
thither these webfooted birds of Paradise.

Nature has done a great deal for Maryland; and Fortune also has
done much for it in these latter days in directing the war from its
territory.  But for the peculiar position of Washington as the
capital, all that is now being done in Virginia would have been
done in Maryland, and I must say that the Marylanders did their
best to bring about such a result.  Had the presence of the war
been regarded by the men of Baltimore as an unalloyed benefit, they
could not have made a greater struggle to bring it close to them.
Nevertheless fate has so far spared them.

As the position of Maryland and the course of events as they took
place in Baltimore on the commencement of secession had
considerable influence both in the North and in the South, I will
endeavor to explain how that State was affected, and how the
question was affected by that State.  Maryland, as I have said
before, is a slave State lying immediately south of Mason and
Dixon's line.  Small portions both of Virginia and of Delaware do
run north of Maryland, but practically Maryland is the frontier
State of the slave States.  It was therefore of much importance to
know which way Maryland would go in the event of secession among
the slave States becoming general; and of much also to ascertain
whether it could secede if desirous of doing so.  I am inclined to
think that as a State it was desirous of following Virginia, though
there are many in Maryland who deny this very stoutly.  But it was
at once evident that if loyalty to the North could not be had in
Maryland of its own free will, adherence to the North must be
enforced upon Maryland.  Otherwise the City of Washington could not
be maintained as the existing capital of the nation.

The question of the fidelity of the State to the Union was first
tried by the arrival at Baltimore of a certain Commissioner from
the State of Mississippi, who visited that city with the object of
inducing secession.  It must be understood that Baltimore is the
commercial capital of Maryland, whereas Annapolis is the seat of
government and the legislature---or is, in other terms, the
political capital.  Baltimore is a city containing 230,000
inhabitants, and is considered to have as strong and perhaps as
violent a mob as any city in the Union.  Of the above number 30,000
are negroes and 2000 are slaves.  The Commissioner made his appeal,
telling his tale of Southern grievances, declaring, among other
things, that secession was not intended to break up the government
but to perpetuate it, and asked for the assistance and sympathy of
Maryland.  This was in December, 1860.  The Commissioner was
answered by Governor Hicks, who was placed in a somewhat difficult
position.  The existing legislature of the State was presumed to be
secessionist, but the legislature was not sitting, nor in the
ordinary course of things would that legislature have been called
on to sit again.  The legislature of Maryland is elected every
other year, and in the ordinary course sits only once in the two
years.  That session had been held, and the existing legislature
was therefore exempt from further work---unless specially summoned
for an extraordinary session.  To do this is within the power of
the Governor.  But Governor Hicks, who seems to have been mainly
anxious to keep things quiet, and whose individual politics did not
come out strongly, was not inclined to issue the summons.  ``Let us
show moderation as well as firmness,'' he said; and that was about
all he did say to the Commissioner from Mississippi.  The Governor
after that was directly called on to convene the legislature; but
this he refused to do, alleging that it would not be safe to trust
the discussion of such a subject as secession to ``excited
politicians, many of whom, having nothing to lose from the
destruction of the government, may hope to derive some gain from
the ruin of the State!''  I quote these words, coming from the head
of the executive of the State and spoken with reference to the
legislature of the State, with the object of showing in what light
the political leaders of a State may be held in that very State to
which they belong.  If we are to judge of these legislators from
the opinion expressed by Governor Hicks, they could hardly have
been fit for their places.  That plan of governing by the little
men has certainly not answered.  It need hardly be said that
Governor Hicks, having expressed such an opinion of his State's
legislature, refused to call them to an extraordinary session.

On the 18th of April, 1860, Governor Hicks issued a proclamation to
the people of Maryland, begging them to be quiet, the chief object
of which, however, was that of promising that no troops should be
sent from their State, unless with the object of guarding the
neighboring City of Washington---a promise which he had no means of
fulfilling, seeing that the President of the United States is the
commander-in-chief of the army of the nation, and can summon the
militia of the several States.  This proclamation by the Governor
to the State was immediately backed up by one from the Mayor of
Baltimore to the city, in which he congratulates the citizens on
the Governor's promise that none of their troops are to be sent to
another State; and then he tells them that they shall be preserved
from the horrors of civil war.

But on the very next day the horrors of civil war began in
Baltimore.  By this time President Lincoln was collecting troops at
Washington for the protection of the capital; and that army of the
Potomac, which has ever since occupied the Virginian side of the
river, was in course of construction.  To join this, certain troops
from Massachusetts were sent down by the usual route, via New York,
Philadelphia, and Baltimore; but on their reaching Baltimore by
railway, the mob of that town refused to allow them to pass
through,---and a fight began.  Nine citizens were killed and two
soldiers, and as many more were wounded.  This, I think, was the
first blood spilt in the civil war; and the attack was first made
by the mob of the first slave city reached by the Northern
soldiers.  This goes far to show, not that the border States
desired secession, but that, when compelled to choose between
secession and Union, when not allowed by circumstances to remain
neutral, their sympathies were with their sister slave States
rather than with the North.

Then there was a great running about of official men between
Baltimore and Washington, and the President was besieged with
entreaties that no troops should be sent through Baltimore.  Now
this was hard enough upon President Lincoln, seeing that he was
bound to defend his capital, that he could get no troops from the
South, and that Baltimore is on the high-road from Washington both
to the West and to the North; but, nevertheless, he gave way.  Had
he not done so, all Baltimore would have been in a blaze of
rebellion, and the scene of the coming contest must have been
removed from Virginia to Maryland, and Congress and the government
must have traveled from Washington north to Philadelphia.  ``They
shall not come through Baltimore,'' said Mr.\ Lincoln.  ``But they
shall come through the State of Maryland.  They shall be passed
over Chesapeake Bay by water to Annapolis, and shall come up by
rail from thence.''  This arrangement was as distasteful to the
State of Maryland as the other; but Annapolis is a small town
without a mob, and the Marylanders had no means of preventing the
passage of the troops.  Attempts were made to refuse the use of the
Annapolis branch railway, but General Butler had the arranging of
that.  General Butler was a lawyer from Boston, and by no means
inclined to indulge the scruples of the Marylanders who had so
roughly treated his fellow-citizens from Massachusetts.  The troops
did therefore pass by Annapolis, much to the disgust of the State.
On the 27th of April, Governor Hicks, having now had a sufficiency
of individual responsibility, summoned the legislature of which he
had expressed so bad an opinion; but on this occasion he omitted to
repeat that opinion, and submitted his views in very proper terms
to the wisdom of the senators and representatives.  He entertains,
as he says, an honest conviction that the safety of Maryland lies
in preserving a neutral position between the North and the South.
Certainly, Governor Hicks, if it were only possible!  The
legislature again went to work to prevent, if it might be
prevented, the passage of troops through their State; but luckily
for them, they failed.  The President was bound to defend
Washington, and the Marylanders were denied their wish of having
their own fields made the fighting ground of the civil war.

That which appears to me to be the most remarkable feature in all
this is the antagonism between United States law and individual
State feeling.  Through the whole proceeding the Governor and the
State of Maryland seemed to have considered it quite reasonable to
oppose the constitutional power of the President and his
government.  It is argued in all the speeches and written documents
that were produced in Maryland at the time, that Maryland was true
to the Union; and yet she put herself in opposition to the
constitutional military power of the President.  Certain
Commissioners went from the State legislature to Washington in May,
and from their report it appears that the President had expressed
himself of opinion that Maryland might do this or that ``as long as
she had not taken and was not about to take a hostile attitude to
the Federal government!''  From which we are to gather that a denial
of that military power given to the President by the Constitution
was not considered as an attitude hostile to the Federal
government.  At any rate, it was direct disobedience to Federal
law.  I cannot but revert from this to the condition of the
Fugitive Slave Law.  Federal law, and indeed the original
constitution, plainly declare that fugitive slaves shall be given
up by the free-soil States.  Massachusetts proclaims herself to be
specially a Federal law-loving State.  But every man in
Massachusetts knows that no judge, no sheriff, no magistrate, no
policeman in that State would at this time, or then, when that
civil war was beginning, have lent a hand in any way to the
rendition of a fugitive slave.  The Federal law requires the State
to give up the fugitive, but the State law does not require judge,
sheriff, magistrate, or policeman to engage in such work, and no
judge, sheriff or magistrate will do so; consequently that Federal
law is dead in Massachusetts, as it is also in every free-soil
State,---dead, except in as much as there was life in it to create
ill blood as long as the North and South remained together, and
would be life in it for the same effect if they should again be
brought under the same flag.

On the 10th of May, the Maryland legislature, having received the
report of their Commissioners above mentioned, passed the following
resolution:---%

``Whereas, the war against the Confederate States is
unconstitutional and repugnant to civilization, and will result in
a bloody and shameful overthrow of our constitution, and while
recognizing the obligations of Maryland to the Union, we sympathize
with the South in the struggle for their rights; for the sake of
humanity we are for peace and reconciliation, and solemnly protest
against this war, and will take no part in it.

``\emph{Resolved}, That Maryland implores the President, in the name of
God, to cease this unholy war, at least until Congress assembles''---%
a period of above six months.  ``That Maryland desires and consents
to the recognition of the independence of the Confederate States.
The military occupation of Maryland is unconstitutional, and she
protests against it, though the violent interference with the
transit of the Federal troops is discountenanced.  That the
vindication of her rights be left to time and reason, and that a
convention under existing circumstances is inexpedient.''  From
which it is plain that Maryland would have seceded as effectually
as Georgia seceded, had she not been prevented by the interposition
of Washington between her and the Confederate States---the happy
intervention, seeing that she has thus been saved from becoming the
battle-ground of the contest.  But the legislature had to pay for
its rashness.  On the 13th of September thirteen of its members
were arrested, as were also two editors of newspapers presumed to
be secessionists.  A member of Congress was also arrested at the
same time, and a candidate for Governor Hicks's place, who belonged
to the secessionist party.  Previously, in the last days of June
and beginning of July, the chief of the police at Baltimore and the
members of the Board of Police had been arrested by General Banks,
who then held Baltimore in his power.

I should be sorry to be construed as saying that republican
institutions, or what may more properly be called democratic
institutions, have been broken down in the States of America.  I am
far from thinking that they have broken down.  Taking them and
their work as a whole, I think that they have shown and still show
vitality of the best order.  But the written Constitution of the
United States and of the several States, as bearing upon each
other, are not equal to the requirements made upon them.  That, I
think, is the conclusion to which a spectator should come.  It is
in that doctrine of finality that our friends have broken down---a
doctrine not expressed in their constitutions, and indeed expressly
denied in the Constitution of the United States, which provides the
mode in which amendments shall be made---but appearing plainly
enough in every word of self-gratulation which comes from them.
Political finality has ever proved a delusion---as has the idea of
finality in all human institutions.  I do not doubt but that the
republican form of government will remain and make progress in
North America, but such prolonged existence and progress must be
based on an acknowledgment of the necessity for change, and must
much depend on the facilities for change which shall be afforded.

I have described the condition of Baltimore as it was early in May,
1861.  I reached that city just seven months later, and its
condition was considerably altered.  There was no question then
whether troops should pass through Baltimore, or by an awkward
round through Annapolis, or not pass at all through Maryland.
General Dix, who had succeeded General Banks, was holding the city
in his grip, and martial law prevailed.  In such times as those, it
was bootless to inquire as to that promise that no troops should
pass southward through Baltimore.  What have such assurances ever
been worth in such days?  Baltimore was now a military depot in the
hands of the Northern army, and General Dix was not a man to stand
any trifling.  He did me the honor to take me to the top of Federal
Hill, a suburb of the city, on which he had raised great earthworks
and planted mighty cannons, and built tents and barracks for his
soldiery, and to show me how instantaneously he could destroy the
town from his exalted position.  ``This hill was made for the very
purpose,'' said General Dix; and no doubt he thought so.  Generals,
when they have fine positions and big guns and prostrate people
lying under their thumbs, are inclined to think that God's
providence has specially ordained them and their points of vantage.
It is a good thing in the mind of a general so circumstanced that
200,000 men should be made subject to a dozen big guns.  I confess
that to me, having had no military education, the matter appeared
in a different light, and I could not work up my enthusiasm to a
pitch which would have been suitable to the general's courtesy.
That hill, on which many of the poor of Baltimore had lived, was
desecrated in my eyes by those columbiads.  The neat earth-works
were ugly, as looked upon by me; and though I regarded General Dix
as energetic, and no doubt skillful in the work assigned to him, I
could not sympathize with his exultation.

Previously to the days of secession Baltimore had been guarded by
Fort McHenry, which lies on a spit of land running out into the bay
just below the town.  Hither I went with General Dix, and he
explained to me how the cannon had heretofore been pointed solely
toward the sea; that, however, now was all changed, and the mouths
of his bombs and great artillery were turned all the other way.
The commandant of the fort was with us, and other officers, and
they all spoke of this martial tenure as a great blessing.  Hearing
them, one could hardly fail to suppose that they had lived their
forty, fifty, or sixty years of life in full reliance on the powers
of a military despotism.  But not the less were they American
republicans, who, twelve months since, would have dilated on the
all-sufficiency of their republican institutions, and on the
absence of any military restraint in their country, with that
peculiar pride which characterizes the citizens of the States.
There are, however, some lessons which may be learned with singular
rapidity!

Such was the state of Baltimore when I visited that city.  I found,
nevertheless, that cakes and ale still prevailed there.  I am
inclined to think that cakes and ale prevail most freely in times
that are perilous, and when sources of sorrow abound.  I have seen
more reckless joviality in a town stricken by pestilence than I
ever encountered elsewhere.  There was General Dix seated on
Federal Hill with his cannon; and there, beneath his artillery,
were gentlemen hotly professing themselves to be secessionists, men
whose sons and brothers were in the Southern army, and women, alas!
whose brothers would be in one army, and their sons in another.
That was the part of it which was most heartrending in this border
land.  In New England and New York men's minds at any rate were
bent all in the same direction---as doubtless they were also in
Georgia and Alabama.  But here fathers were divided from sons, and
mothers from daughters.  Terrible tales were told of threats
uttered by one member of a family against another.  Old ties of
friendship were broken up.  Society had so divided itself that one
side could hold no terms of courtesy with the other.  ``When this is
over,'' one gentleman said to me, ``every man in Baltimore will have
a quarrel to the death on his hands with some friend whom he used
to love.''  The complaints made on both sides were eager and open-
mouthed against the other.

Late in the autumn an election for a new legislature of the State
had taken place, and the members returned were all supposed to be
Unionists.  That they were prepared to support the government is
certain.  But no known or presumed secessionist was allowed to vote
without first taking the oath of allegiance.  The election,
therefore, even if the numbers were true, cannot be looked upon as
a free election.  Voters were stopped at the poll and not allowed
to vote unless they would take an oath which would, on their parts,
undoubtedly have been false.  It was also declared in Baltimore
that men engaged to promote the Northern party were permitted to
vote five or six times over, and the enormous number of votes
polled on the government side gave some coloring to the statement.
At any rate, an election carried under General Dix's guns cannot be
regarded as an open election.  It was out of the question that any
election taken under such circumstances should be worth anything as
expressing the minds of the people.  Red and white had been
declared to be the colors of the Confederates, and red and white
had of course become the favorite colors of the Baltimore ladies.
Then it was given out that red and white would not be allowed in
the streets.  Ladies wearing red and white were requested to return
home.  Children decorated with red and white ribbons were stripped
of their bits of finery---much to their infantile disgust and
dismay.  Ladies would put red and white ornaments in their windows,
and the police would insist on the withdrawal of the colors.  Such
was the condition of Baltimore during the past winter.
Nevertheless cakes and ale abounded; and though there was deep
grief in the city, and wailing in the recesses of many houses, and
a feeling that the good times were gone, never to return within the
days of many of them, still there existed an excitement and a
consciousness of the importance of the crisis which was not
altogether unsatisfactory.  Men and women can endure to be ruined,
to be torn from their friends, to be overwhelmed with avalanches of
misfortune, better than they can endure to be dull.

Baltimore is, or at any rate was, an aspiring city, proud of its
commerce and proud of its society.  It has regarded itself as the
New York of the South, and to some extent has forced others so to
regard it also.  In many respects it is more like an English town
than most of its Transatlantic brethren, and the ways of its
inhabitants are English.  In old days a pack of fox hounds was kept
here---or indeed in days that are not yet very old, for I was told
of their doings by a gentleman who had long been a member of the
hunt.  The country looks as a hunting country should look, whereas
no man that ever crossed a field after a pack of hounds would feel
the slightest wish to attempt that process in New England or New
York.  There is in Baltimore an old inn with an old sign, standing
at the corner of Eutaw and Franklin Streets, just such as may still
be seen in the towns of Somersetshire, and before it there are to
be seen old wagons, covered and soiled and battered, about to
return from the city to the country, just as the wagons do in our
own agricultural counties.  I have seen nothing so thoroughly
English in any other part of the Union.

But canvas-back ducks and terrapins are the great glories of
Baltimore.  Of the nature of the former bird I believe all the
world knows something.  It is a wild duck which obtains the
peculiarity of its flavor from the wild celery on which it feeds.
This celery grows on the Chesapeake Bay, and I believe on the
Chesapeake Bay only.  At any rate, Baltimore is the headquarters of
the canvas-backs, and it is on the Chesapeake Bay that they are
shot.  I was kindly invited to go down on a shooting-party; but
when I learned that I should have to ensconce myself alone for
hours in a wet wooden box on the water's edge, waiting there for
the chance of a duck to come to me, I declined.  The fact of my
never having as yet been successful in shooting a bird of any kind
conduced somewhat, perhaps, to my decision. I must acknowledge that
the canvas-back duck fully deserves all the reputation it has
acquired.  As to the terrapin, I have not so much to say.  The
terrapin is a small turtle, found on the shores of Maryland and
Virginia, out of which a very rich soup is made.  It is cooked with
wines and spices, and is served in the shape of a hash, with heaps
of little bones mixed through it.  It is held in great repute, and
the guest is expected as a matter of course to be helped twice.
The man who did not eat twice of terrapin would be held in small
repute, as the Londoner is held who at a city banquet does not
partake of both thick and thin turtle.  I must, however, confess
that the terrapin for me had no surpassing charms.

Maryland was so called from Henrietta Maria, the wife of Charles
I., by which king, in 1632, the territory was conceded to the Roman
Catholic Lord Baltimore.  It was chiefly peopled by Roman
Catholics, but I do not think that there is now any such specialty
attaching to the State.  There are in it two or three old Roman
Catholic families, but the people have come down from the North,
and have no peculiar religious tendencies.  Some of Lord
Baltimore's descendants remained in the State up to the time of the
Revolution.  From Baltimore I went on to Washington.


\bigskip
\begin{center}{\textsc{END OF VOL. I.}}\end{center}

\end{document}



% End of Project Gutenberg Etext North America, V. 1, by Anthony Trollope
